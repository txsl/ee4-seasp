\documentclass[./main.tex]{subfiles} 
\begin{document}

\section{Widely Linear Filtering and Adaptive Spectral Estimation}

\subsection{Complex LMS and Widely Linear Modelling}

\subsubsection{The CLMS and ACLMS}
We are introduced to the Complex LMS, which is a modification of the LMS filter, but designed to operate with complex signals. The $ \mathbf{w}$ weight term is replaced with $ \mathbf{h}$, and the estimation is now $ \hat{y}(n) = \mathbf{h}^H(n) \mathbf{x}(n) $, with the weight update being performed as $ \mathbf{h}(n+1) = \mathbf{h}(n) + \mu e^{\ast}(n)\mathbf{x}(n) $.

We are also introduced to the Augmented CLMS, which is designed to identify the second order (if there is any) statistical relationship of the input and output. It is effectively an extension of the CLMS, and adds the weights $ \mathbf{g}$, in addition to $ \mathbf{h}$ as defined for the CLMS. The esimtation equation becomes  $ \hat{y}(n) = \mathbf{h}^H(n) \mathbf{x}(n) + \mathbf{g}^H(n) \mathbf{x}^\ast(n) $. The next estimated of $ \mathbf{g}$ is defined as $ \mathbf{g}(n+1) = \mathbf{g}(n) + \mu e^{\ast}(n)\mathbf{x}\ast(n) $.

We generated a WLMA(1) (Widely Linear Moving Average) process which is defined as $ y(n) = x(n) + b_1 x(n-1) + b_2 x^{\ast}(n-2) $, where $ x \sim \mathcal{N} (0,1) $. The coefficients are defined as $ b_1 = 1.5 + 1j $ and $ b_2 = 2.5 - 0.5j$. We run this process through both the CLMS and ACLMS filters. 100 indepdent sets of noise were generated and run through the filters (identical signals), with the mean of at each iteration taken. Figure \ref{fig:4_1_a_clms_err} shows the learning error. We can see that even in steady state, it appears to be fairly noisy. This is also apparent in figure \ref{fig:4_1_a_clms_weights} where we can see the estimated weights of $\mathbf{h}$.

\begin{figure}[h]
	\centering
	\begin{subfigure}[b]{0.49\textwidth}
		\resizebox{\textwidth}{!}{% This file was created by matlab2tikz.
% Minimal pgfplots version: 1.3
%
%The latest updates can be retrieved from
%  http://www.mathworks.com/matlabcentral/fileexchange/22022-matlab2tikz
%where you can also make suggestions and rate matlab2tikz.
%
\definecolor{mycolor1}{rgb}{0.00000,0.44700,0.74100}%
%
\begin{tikzpicture}

\begin{axis}[%
width=4in,
height=1.5in,
at={(1.011111in,0.641667in)},
scale only axis,
unbounded coords=jump,
xmin=0,
xmax=1000,
tick align=outside,
xlabel={Iteration},
xmajorgrids,
ymin=0,
ymax=12,
ylabel={Learning Curve (dB)},
ymajorgrids,
title style={font=\bfseries},
title={CLMS Error for WLMA(1) process},
axis x line*=bottom,
axis y line*=left
]
\addplot [color=mycolor1,solid,forget plot]
  table[row sep=crcr]{%
1	0.978418476502461\\
2	11.1746480652383\\
3	10.9501756148091\\
4	9.96204960371666\\
5	10.595245577336\\
6	9.41881741084485\\
7	9.69569116010431\\
8	9.78943994843067\\
9	8.81764036364594\\
10	10.0381398055888\\
11	9.93881525868508\\
12	10.2335169960308\\
13	10.2042352686411\\
14	9.73045985895639\\
15	10.4739661005185\\
16	10.2388790982439\\
17	9.19505508353108\\
18	9.83846180558536\\
19	10.1074140747702\\
20	8.74283483265922\\
21	9.76558558531127\\
22	9.92882422097187\\
23	9.26865737042914\\
24	9.70635800311539\\
25	9.42404837854917\\
26	9.45895181494702\\
27	9.90146439631465\\
28	10.4338447242868\\
29	9.46539164306112\\
30	9.32196032930987\\
31	8.97799532849667\\
32	8.96740280424866\\
33	9.59102022996159\\
34	10.0067768882894\\
35	9.37642019823879\\
36	9.39537072249494\\
37	9.8545711444564\\
38	10.0180885014752\\
39	9.13476117340177\\
40	9.13515412770262\\
41	8.68594062551714\\
42	8.98463427632205\\
43	9.08132401311856\\
44	9.54126310788422\\
45	10.1342852658108\\
46	8.8554081297494\\
47	8.32841361361119\\
48	8.72954823673159\\
49	9.08814636464844\\
50	8.78248615681039\\
51	8.30701911586552\\
52	9.03475228677092\\
53	8.65011019129\\
54	8.85663651746798\\
55	9.73932707060251\\
56	9.797252199705\\
57	8.38670116708862\\
58	8.58657334065346\\
59	8.50897401877225\\
60	9.63879052510689\\
61	8.8960502081045\\
62	8.24351165051243\\
63	9.90889624150234\\
64	9.14565215556308\\
65	9.20701320151282\\
66	9.49501676729981\\
67	8.97492854390148\\
68	9.08221270523488\\
69	9.31832839882081\\
70	8.62503451158929\\
71	8.65188441958773\\
72	9.09484133450917\\
73	8.85343904936455\\
74	8.63649182473724\\
75	7.97365756955759\\
76	8.19446571996081\\
77	9.0867297189594\\
78	9.40360940586272\\
79	8.10900797690339\\
80	9.63417795199049\\
81	8.79895532013\\
82	8.98399752905514\\
83	8.02632096488489\\
84	8.13370570046961\\
85	8.47113880737176\\
86	9.01635764868504\\
87	9.16493292671928\\
88	8.63213102171707\\
89	8.20124665588014\\
90	8.76732593722886\\
91	8.98562208578226\\
92	8.70692805963766\\
93	8.66542627139009\\
94	7.87172638139154\\
95	8.62920448734913\\
96	8.37402734006286\\
97	8.09632758862937\\
98	9.43377263836858\\
99	8.03155633016215\\
100	9.11928611667897\\
101	8.46813005841678\\
102	8.27476597331828\\
103	8.78255764738666\\
104	7.97537069362523\\
105	8.27302382117069\\
106	9.03089894593862\\
107	8.87784236246784\\
108	9.06541426250217\\
109	8.96838658794822\\
110	8.65094482858803\\
111	8.29026724848248\\
112	9.09265408129497\\
113	8.09195254548145\\
114	9.14384919456929\\
115	8.36913754979675\\
116	8.48542262169197\\
117	8.0079231496002\\
118	8.71635817955194\\
119	8.90464302015056\\
120	8.23477332859744\\
121	8.24158590286183\\
122	8.20767734548306\\
123	8.10047201799017\\
124	8.15027075153069\\
125	8.51950208327261\\
126	8.14024700732421\\
127	8.34180607689455\\
128	8.45527134382311\\
129	8.63291178795582\\
130	7.75066942670063\\
131	9.30519391123094\\
132	8.7983932252127\\
133	8.25164111761217\\
134	8.08016265287453\\
135	9.079306688028\\
136	8.42898016223174\\
137	8.47474629626729\\
138	8.65499519529204\\
139	8.83722715772593\\
140	8.62476859163182\\
141	8.36845675458015\\
142	8.4139490221132\\
143	8.60300844887218\\
144	8.52024818886579\\
145	9.09946546517336\\
146	7.57095104558088\\
147	7.59073066468146\\
148	8.49767940709955\\
149	8.20456164508365\\
150	8.48075040857737\\
151	8.20352175168926\\
152	8.50386744763493\\
153	8.28298563659886\\
154	8.40754289932445\\
155	8.80748300596333\\
156	8.6612363905237\\
157	9.39183799369943\\
158	8.10029238493684\\
159	7.58793009309208\\
160	8.98974309869471\\
161	7.93808406288375\\
162	8.01530738743223\\
163	8.53364986363636\\
164	8.00145815722872\\
165	8.18397012536489\\
166	7.59471692919865\\
167	7.7541219371321\\
168	7.74621359298238\\
169	8.4525347497993\\
170	7.16863798756714\\
171	8.72125026390063\\
172	7.78201114140701\\
173	8.18403318419521\\
174	8.13672552591379\\
175	8.65836228474046\\
176	8.75927110100558\\
177	7.34128815853414\\
178	8.60625696840949\\
179	8.36400377664279\\
180	7.79349374383023\\
181	7.87776470959556\\
182	8.17927471960594\\
183	8.34019770673727\\
184	8.4069356786414\\
185	7.78449258341391\\
186	9.08824301847006\\
187	8.37837722321817\\
188	8.04678788005768\\
189	7.48595409577921\\
190	8.70189903395465\\
191	7.9060441168663\\
192	7.51082210052085\\
193	8.53210636411075\\
194	8.40268271980305\\
195	8.37001761968269\\
196	7.60940163287192\\
197	7.73069738957137\\
198	7.82715386650584\\
199	8.44365261120787\\
200	8.48248635923199\\
201	8.21664048643482\\
202	9.07678428410058\\
203	8.26181316973742\\
204	8.72575822146842\\
205	8.15448704362174\\
206	8.47482811748342\\
207	8.68512494800431\\
208	7.80555438918927\\
209	7.69371940287396\\
210	7.13167978038543\\
211	8.55387705904867\\
212	8.20452007619009\\
213	8.72422795423658\\
214	8.20625094773974\\
215	8.89802807946156\\
216	8.3396369184816\\
217	8.16094515717358\\
218	8.24146818111144\\
219	8.01980774864135\\
220	8.972029581359\\
221	8.11944212324922\\
222	7.94955547913594\\
223	8.76134966714478\\
224	8.20667656294597\\
225	8.13803977802882\\
226	8.42418366683905\\
227	8.22913129126509\\
228	7.25708793492767\\
229	8.76660840442386\\
230	7.51759561594067\\
231	7.63842024569672\\
232	8.5138251241532\\
233	8.23333698161821\\
234	7.95223221672796\\
235	7.86267282075284\\
236	8.11537631091913\\
237	9.02281606629002\\
238	8.5928287665261\\
239	7.77771739221655\\
240	9.11122723232822\\
241	8.26299707351569\\
242	8.20339825897211\\
243	6.91434408730653\\
244	8.67084380506581\\
245	8.48768685129032\\
246	7.82588439244945\\
247	8.88344327983763\\
248	7.65423975562076\\
249	8.86167653713749\\
250	7.28571555865259\\
251	9.05103918708624\\
252	8.24566269258343\\
253	7.97331186405779\\
254	8.28904147064403\\
255	8.11270552381331\\
256	8.80919478411548\\
257	8.38063724643697\\
258	8.35699877751931\\
259	9.1098864082961\\
260	8.02706162689017\\
261	8.4432283212438\\
262	7.99804160440211\\
263	7.41097970997948\\
264	8.21730462007766\\
265	9.15825259192136\\
266	8.29942185670613\\
267	8.32635911724685\\
268	8.15548392161513\\
269	8.88994160828168\\
270	7.80444492618758\\
271	8.54672765795127\\
272	7.9558607179951\\
273	8.15350557674823\\
274	8.33171167308596\\
275	8.10048129565267\\
276	7.42627459534894\\
277	7.64968700437021\\
278	7.69987397010181\\
279	8.2500096507967\\
280	7.55101005538165\\
281	7.84002435992109\\
282	7.75690692609567\\
283	8.28251712406725\\
284	7.54465446476963\\
285	8.94502473103667\\
286	8.43768290160815\\
287	8.0660492798963\\
288	8.62605924170479\\
289	8.74858507472358\\
290	8.45238811990087\\
291	8.52272700080543\\
292	7.88065698161386\\
293	8.35535182880419\\
294	7.94065217740034\\
295	7.72677394593417\\
296	8.00701988098103\\
297	8.65915977980916\\
298	8.26284177185568\\
299	8.584834907389\\
300	7.44407826964995\\
301	8.3095238200269\\
302	8.72548558915787\\
303	7.75857720429523\\
304	8.6628059836064\\
305	8.09332905196062\\
306	8.25475688051263\\
307	7.94617815757297\\
308	8.38793444974587\\
309	7.70480992417405\\
310	8.41394305809479\\
311	8.86344865371768\\
312	8.73535515960326\\
313	7.66365178737473\\
314	8.35200441432658\\
315	7.49160028786295\\
316	8.37714523697317\\
317	8.28151296701145\\
318	7.83259383772895\\
319	8.72063730085576\\
320	8.29191917529877\\
321	7.6402100815069\\
322	8.12802045419279\\
323	8.69290166949892\\
324	8.3655050039032\\
325	7.90348446462256\\
326	8.11938591439539\\
327	7.26126635210141\\
328	7.79066197226708\\
329	7.51955769418855\\
330	7.06275616452166\\
331	8.02951619012525\\
332	8.8041890137012\\
333	7.94613418913437\\
334	8.2924196818091\\
335	8.31928287420826\\
336	8.34277518326859\\
337	8.03546680894844\\
338	8.79841004505239\\
339	8.60359172933607\\
340	7.34873754331963\\
341	8.67483782633583\\
342	8.22173329558639\\
343	8.15841905554374\\
344	8.32967335766761\\
345	8.24657393943408\\
346	7.66343716884193\\
347	8.10609177345477\\
348	8.04890826632689\\
349	7.17961710012004\\
350	7.88942647306088\\
351	7.47821006822119\\
352	7.82045791043446\\
353	8.76740828859431\\
354	8.50351112866128\\
355	8.38790225730624\\
356	7.20849815648415\\
357	8.42473205723921\\
358	8.1101077062567\\
359	8.91383560475946\\
360	8.54721101154102\\
361	7.63649772697985\\
362	8.38888904267639\\
363	8.66613043703403\\
364	8.58880570544158\\
365	8.05505537643296\\
366	8.37626069547404\\
367	8.47497075186391\\
368	8.4504653736717\\
369	8.31365504466834\\
370	7.95389806472099\\
371	8.32249780100456\\
372	7.75183790482614\\
373	8.2636678000698\\
374	7.91661390502477\\
375	8.28226406205333\\
376	8.58399127151718\\
377	8.60082442978069\\
378	7.13140463776923\\
379	8.42360642972515\\
380	8.7930050651568\\
381	7.81060220056679\\
382	8.43986863583077\\
383	8.29514778916268\\
384	8.08586617795511\\
385	8.98424575621438\\
386	8.41492567613096\\
387	8.01793774541766\\
388	8.9540951147763\\
389	7.73916809123043\\
390	8.30842712357569\\
391	7.55125008843945\\
392	7.80037228976095\\
393	8.32347481354555\\
394	8.52987573268195\\
395	7.94700185559795\\
396	7.05061311575665\\
397	8.60987529381201\\
398	8.16862045821038\\
399	7.57896977208027\\
400	8.21411863781719\\
401	8.1112820842251\\
402	8.66702449718854\\
403	8.23643435043723\\
404	8.2507565270195\\
405	8.49729268723286\\
406	8.4996179331433\\
407	8.17393183167159\\
408	7.02298974710715\\
409	8.03950446396625\\
410	8.65572405875572\\
411	7.80153611815677\\
412	8.22405908734264\\
413	7.66108907020459\\
414	7.61275864051671\\
415	7.98882595708692\\
416	7.62587791589837\\
417	7.18862623451336\\
418	8.29938332005559\\
419	7.69288334651714\\
420	8.45211172511111\\
421	8.33080266299293\\
422	8.31222835831744\\
423	8.75498276322718\\
424	8.1889459812516\\
425	7.70439077917389\\
426	7.95307113238275\\
427	7.26113025356155\\
428	7.99709301529604\\
429	8.06577685973314\\
430	8.05529043041925\\
431	7.8935195266568\\
432	8.1809870221571\\
433	8.35869323569923\\
434	8.32025859622214\\
435	8.09529269804594\\
436	7.93602386520936\\
437	8.1760948571093\\
438	8.24978886136273\\
439	8.11691003002663\\
440	8.46729739989982\\
441	8.1568771864871\\
442	8.49761211441289\\
443	7.76503628773368\\
444	7.97723633331643\\
445	7.70882656281467\\
446	7.99793116079879\\
447	7.95112441995846\\
448	7.48266643702583\\
449	7.87977725686728\\
450	8.43510712983086\\
451	8.02755509884637\\
452	7.42142488836557\\
453	8.04067597963296\\
454	8.16195642903865\\
455	9.05472422694393\\
456	7.79733341697717\\
457	7.59819205691478\\
458	7.57244539425658\\
459	8.76176625035635\\
460	8.46150015853411\\
461	8.49769609410851\\
462	8.10200044984507\\
463	8.06243086338964\\
464	8.15649033950373\\
465	8.35569999715784\\
466	8.83001468035551\\
467	8.17459056873746\\
468	7.98678172274626\\
469	7.92052209440712\\
470	8.0378923826136\\
471	7.49268541148038\\
472	8.18761106626653\\
473	7.27652354803027\\
474	7.70979460611202\\
475	8.01912494765647\\
476	8.25593387444789\\
477	8.60003633426948\\
478	7.12397686559971\\
479	7.60843846887434\\
480	9.10927705245081\\
481	8.67197702628762\\
482	7.62152627773656\\
483	8.28652921262032\\
484	9.14599767070585\\
485	8.51126466996834\\
486	7.27222107249861\\
487	8.46694776272654\\
488	8.33352857245726\\
489	7.9270327853984\\
490	7.60742901356672\\
491	8.03470570457118\\
492	8.54132457678933\\
493	7.95978072175746\\
494	8.36620440363553\\
495	7.49636096335816\\
496	7.86523887341447\\
497	8.43475462064344\\
498	8.59803390103383\\
499	7.71335547341923\\
500	8.15583676617329\\
501	7.45830439808586\\
502	8.58702917023615\\
503	7.4248734844308\\
504	8.17954980342772\\
505	8.10098086334792\\
506	7.64363911488436\\
507	7.45952580308258\\
508	7.99747085188436\\
509	8.01796622720863\\
510	8.63679174924456\\
511	8.2182267193561\\
512	7.91119500996887\\
513	8.41722611114209\\
514	8.7762844053931\\
515	7.77877924536834\\
516	7.95134224151972\\
517	8.83459024750026\\
518	7.70936026480683\\
519	7.77567551223616\\
520	8.52469444749337\\
521	7.87948655506517\\
522	7.62526494518746\\
523	7.82472007475517\\
524	7.6392593989705\\
525	7.6693624164557\\
526	8.97637370339283\\
527	7.98323426612323\\
528	8.15661664208453\\
529	8.33542053720575\\
530	8.29066214239149\\
531	7.73510513594508\\
532	8.28816790566641\\
533	7.76529250358989\\
534	8.17564378674182\\
535	8.31764234413775\\
536	8.6490268073342\\
537	8.45447132488911\\
538	8.81405878920084\\
539	8.46334915305011\\
540	8.18593214686424\\
541	7.90956072810582\\
542	7.02012796069053\\
543	8.38691286655425\\
544	8.16859823774578\\
545	8.01580317761971\\
546	8.73208353906915\\
547	7.79054490941626\\
548	8.63699944354667\\
549	8.09327528777127\\
550	8.19161440777681\\
551	8.43752463088643\\
552	8.18340604860275\\
553	7.37891895555541\\
554	7.6327489315559\\
555	7.88358471843644\\
556	7.60100069918181\\
557	8.22414421806298\\
558	7.67073196144196\\
559	8.99862839658418\\
560	8.29646140781599\\
561	8.51541225630818\\
562	7.89001022789398\\
563	8.36048963204898\\
564	8.14601270070989\\
565	8.25819626123059\\
566	8.03902378470785\\
567	8.74715893638013\\
568	8.68710537552938\\
569	8.0150779827494\\
570	8.3433144693709\\
571	8.30391884316167\\
572	8.51547363201534\\
573	8.82143951151637\\
574	7.73498452249757\\
575	7.97083085859178\\
576	8.59513519460801\\
577	7.54539573818619\\
578	8.92569871996753\\
579	8.36676858792574\\
580	8.40626696423235\\
581	8.07869538808033\\
582	8.10956310807928\\
583	7.77920322227516\\
584	8.7071474302053\\
585	7.98891133297629\\
586	9.14984777693605\\
587	7.61089221127548\\
588	7.63882740383883\\
589	8.01280232566744\\
590	8.55580432559238\\
591	7.58742950931615\\
592	8.41642326205544\\
593	8.23865841119263\\
594	8.16603445684208\\
595	7.5499953856636\\
596	8.38974937656453\\
597	7.78665686787973\\
598	8.45780602120351\\
599	8.46713109969374\\
600	8.22055828798331\\
601	7.67446767754316\\
602	6.92571466703612\\
603	7.91209861772504\\
604	9.43575000070508\\
605	7.4396364848611\\
606	7.90826023728806\\
607	8.98015388814482\\
608	8.16990668751841\\
609	7.67637690613637\\
610	8.3502474557309\\
611	8.38070313936044\\
612	7.95417252236723\\
613	8.54177025781795\\
614	8.4381960168568\\
615	8.0438049915111\\
616	8.42588667127313\\
617	8.86109244144015\\
618	8.22872802844272\\
619	8.29143590845652\\
620	8.35354654300022\\
621	7.93305664455906\\
622	8.34128561517485\\
623	8.7464440708249\\
624	8.59366428600588\\
625	8.70587510393704\\
626	7.76668901657292\\
627	8.1881989937409\\
628	8.35120693745982\\
629	8.18650112244647\\
630	7.78048850410649\\
631	8.29627626474584\\
632	8.39530483126257\\
633	8.30529615434665\\
634	8.16356520100376\\
635	8.23276760240862\\
636	8.89724567868456\\
637	8.13939721327079\\
638	7.89599515833799\\
639	7.81355615173923\\
640	8.12415871895267\\
641	8.93772524666482\\
642	8.22647528282061\\
643	8.54908077430592\\
644	7.79377477741413\\
645	8.71166812164784\\
646	8.53251744209193\\
647	8.14438694363729\\
648	7.87440644225462\\
649	8.18482567151671\\
650	7.68304824361113\\
651	8.04830947344924\\
652	8.32288171578447\\
653	8.3463007800562\\
654	7.42195118988348\\
655	8.42001466373031\\
656	7.99698128644752\\
657	7.8128008136698\\
658	7.93485465439234\\
659	8.40293343652288\\
660	8.61726642053559\\
661	7.72636047092455\\
662	7.92185799675563\\
663	7.49161391093365\\
664	7.72539724370542\\
665	7.90103914874322\\
666	8.17269981340903\\
667	8.43326914601242\\
668	7.80620186324698\\
669	8.86155203346803\\
670	8.24631927488015\\
671	8.03761084813072\\
672	8.53525621664679\\
673	8.04392256222684\\
674	8.47398350504709\\
675	8.22636621178754\\
676	8.80060239997681\\
677	8.28860803955596\\
678	7.93101554149725\\
679	8.13162733545226\\
680	8.29880118638796\\
681	7.63056110856125\\
682	8.5211011162529\\
683	7.35469849198391\\
684	8.05668745268221\\
685	8.40178627759698\\
686	8.44442541170749\\
687	7.99248889899342\\
688	8.33205842418116\\
689	8.06561764449632\\
690	9.11724001141224\\
691	8.04366393645961\\
692	7.57577848968943\\
693	8.28640319046022\\
694	7.46955521554889\\
695	8.4719390868385\\
696	8.44562055570369\\
697	7.93314325346529\\
698	7.64538844551247\\
699	8.35347654129582\\
700	7.5679212033718\\
701	8.06038442080163\\
702	8.28182045109591\\
703	8.53515938829221\\
704	8.51671408915461\\
705	8.73223763620674\\
706	8.00064606082396\\
707	8.68569249011998\\
708	8.39104427348682\\
709	7.96235451307877\\
710	8.71388517131225\\
711	7.73514402057156\\
712	9.41576254416994\\
713	8.14818656138831\\
714	8.08355096606658\\
715	8.56077976531149\\
716	9.11120719058963\\
717	7.92357588386945\\
718	8.44089232218535\\
719	7.91596429981221\\
720	8.32039410777576\\
721	9.14462503476274\\
722	8.93953365497657\\
723	8.31061230554165\\
724	7.96513332234691\\
725	8.76582090579888\\
726	8.60570906254482\\
727	7.49490074899741\\
728	7.72905804901552\\
729	7.93869402720679\\
730	9.08051469324024\\
731	8.601016515676\\
732	9.07500053701426\\
733	7.64067088860666\\
734	7.62171651402324\\
735	7.88083630481759\\
736	8.1073639137243\\
737	8.25712390382763\\
738	7.77513586934345\\
739	8.1621119513463\\
740	8.07524599066736\\
741	8.26148049512818\\
742	8.33185046052785\\
743	8.65355377553493\\
744	7.65635396462411\\
745	8.6284121275564\\
746	8.82835340037888\\
747	8.06858660946387\\
748	7.70110033597533\\
749	8.75578987761122\\
750	7.73175354182562\\
751	7.95683618381624\\
752	8.08440786712789\\
753	8.01418127871991\\
754	8.63836651682227\\
755	9.34932696119026\\
756	8.75257316295511\\
757	7.91604331010821\\
758	8.19523630191824\\
759	7.7613568746111\\
760	9.10562310362115\\
761	8.73398760624987\\
762	8.64132882876578\\
763	7.72619654489572\\
764	8.20291747269032\\
765	8.05581789763757\\
766	8.74758347435043\\
767	8.19888610227482\\
768	8.60331812611921\\
769	8.3384807225285\\
770	8.73774409527884\\
771	8.22049398082784\\
772	7.49717023332905\\
773	8.31618284831226\\
774	7.95381798416758\\
775	8.35104918160223\\
776	8.04787305229122\\
777	7.80816291781772\\
778	7.33778560268078\\
779	8.0773440544238\\
780	7.92948957439501\\
781	7.37417651659732\\
782	8.00686357328896\\
783	8.11312264109977\\
784	7.48104090980258\\
785	7.54634006140151\\
786	7.39382677503699\\
787	7.93921484530583\\
788	7.88159946232725\\
789	7.60237667139775\\
790	8.13995751988189\\
791	8.42161259491809\\
792	8.90386860380361\\
793	8.62267972201951\\
794	8.33047569195917\\
795	7.27810375304927\\
796	8.19518610958381\\
797	7.89562477154854\\
798	9.02561604283501\\
799	8.35339581108203\\
800	7.34247815217023\\
801	9.08819183001529\\
802	7.62692132850806\\
803	8.0359712591003\\
804	8.51891600291225\\
805	8.43584463322878\\
806	8.58284790623866\\
807	7.78873895828764\\
808	8.85864589228386\\
809	8.737865217048\\
810	7.70728545615217\\
811	8.91357777236881\\
812	8.33514735090824\\
813	7.91086346662276\\
814	8.58620491029477\\
815	8.72917367548743\\
816	7.72368135111397\\
817	7.9894082069517\\
818	8.27670847253891\\
819	8.10333974325536\\
820	8.19399924966751\\
821	8.01987045040841\\
822	7.82296452143053\\
823	7.86350984788217\\
824	7.93167179719385\\
825	7.97403660842394\\
826	8.44946675999672\\
827	7.74251557557003\\
828	7.90414843783815\\
829	7.55939669572846\\
830	8.37272660717461\\
831	6.9790000617048\\
832	7.91063371684002\\
833	8.13952357415724\\
834	8.76526155048213\\
835	8.61325824224407\\
836	8.76302656057859\\
837	8.20251754163633\\
838	7.96028089963077\\
839	6.84896714247783\\
840	7.38177365301208\\
841	8.37762328985989\\
842	8.54279611355751\\
843	8.11585333310436\\
844	8.2260271899404\\
845	7.15265991662995\\
846	8.05207676173925\\
847	8.22260392343473\\
848	7.06269486094399\\
849	7.7844446259715\\
850	7.84216661409754\\
851	9.3830883018158\\
852	8.13352024070928\\
853	7.94217803065302\\
854	7.15065756467067\\
855	8.06811028664517\\
856	7.56330890839364\\
857	7.8633271138294\\
858	8.95755008949317\\
859	8.46527126920884\\
860	8.01133446057353\\
861	7.84633521540883\\
862	8.13570461510041\\
863	8.02115526655744\\
864	7.96610186998023\\
865	8.33779609807446\\
866	7.49506126384316\\
867	8.30442971165632\\
868	7.888447325083\\
869	8.84026636863947\\
870	7.93001057207045\\
871	7.78413319199955\\
872	8.66064769626207\\
873	8.16397431731252\\
874	7.66967122981342\\
875	7.32200967585536\\
876	8.67470088573904\\
877	8.2775824856794\\
878	7.77053868450521\\
879	8.57702563637493\\
880	7.29848716528384\\
881	8.7973145443021\\
882	7.98025828245512\\
883	8.58940581290587\\
884	8.42224375613439\\
885	7.59317111628686\\
886	7.67978534382011\\
887	9.14057619583686\\
888	8.55485182057722\\
889	7.74045740062762\\
890	8.41851211511857\\
891	8.77826858071814\\
892	7.86222502967568\\
893	8.20104251410631\\
894	8.41750380609338\\
895	8.08335024400943\\
896	7.44599590971906\\
897	8.26716690160352\\
898	8.15969628943683\\
899	8.29029157539648\\
900	8.51109111781327\\
901	8.00545365745412\\
902	7.60873269049743\\
903	7.69212224675457\\
904	8.68911227199705\\
905	8.85650252361374\\
906	8.19110183667472\\
907	8.0492488425745\\
908	7.58190400947086\\
909	7.77857784283425\\
910	7.84672485823827\\
911	7.72467695668269\\
912	8.08864844211553\\
913	7.3486235978556\\
914	9.04152319968794\\
915	7.9539317081691\\
916	8.75896836315322\\
917	6.88897646200971\\
918	7.42997189067486\\
919	8.83050964206155\\
920	8.12995383466209\\
921	8.56543661701726\\
922	8.01960486516978\\
923	8.54885855678144\\
924	8.17456460251959\\
925	8.67827876261953\\
926	7.94922202042826\\
927	8.85070856234417\\
928	8.54208242813469\\
929	7.54890192170499\\
930	8.34734967377076\\
931	8.83343072455524\\
932	7.68390293446829\\
933	8.13915997368721\\
934	8.34219438364186\\
935	8.11600037696907\\
936	7.76027292722219\\
937	7.78924621308295\\
938	8.8850735737249\\
939	9.08121076614432\\
940	8.09266477734522\\
941	7.906468765097\\
942	7.98746218602212\\
943	7.5730576595715\\
944	7.76869301864019\\
945	8.71528794551803\\
946	8.81102673912418\\
947	8.04680071597351\\
948	8.05274650614175\\
949	8.06062719460419\\
950	8.21708357906122\\
951	7.93737325187393\\
952	7.81021992120517\\
953	7.74174193733813\\
954	7.86819977250176\\
955	7.80351987685307\\
956	8.72377766037878\\
957	8.01529343312298\\
958	8.99034034849021\\
959	7.74874094687119\\
960	8.68911680494838\\
961	8.05194932568683\\
962	8.2346808146754\\
963	8.51933601030427\\
964	7.89137057922454\\
965	8.1708716212451\\
966	7.541835357703\\
967	8.41998358711736\\
968	8.29158173786477\\
969	8.30735244033274\\
970	8.4836963685447\\
971	8.63467946907221\\
972	8.389998490155\\
973	7.95193800670774\\
974	8.36975145903639\\
975	8.56031723797357\\
976	8.36423438044599\\
977	7.40593247499479\\
978	8.83365681465594\\
979	8.09749589443681\\
980	8.68924827931927\\
981	8.27709240102406\\
982	8.19056651640182\\
983	7.63817428189496\\
984	7.15670949096761\\
985	7.98207658176838\\
986	7.72811605057158\\
987	8.71843876140392\\
988	8.09579340620336\\
989	8.06296320000828\\
990	8.14351085556954\\
991	8.07960850158319\\
992	8.00016727893671\\
993	7.54110896135345\\
994	8.17251328870472\\
995	8.01410456634727\\
996	7.95187325390919\\
997	8.14276683093359\\
998	9.05896809450491\\
999	8.050383501665\\
1000	-inf\\
};
\end{axis}
\end{tikzpicture}%}
		\caption{\textit{Learning Curve of the filter, defined as $ 10 \log|e(n)|^2 $}}
		\label{fig:4_1_a_clms_err}
	\end{subfigure}
	~ %add desired spacing between images, e. g. ~, \quad, \qquad, \hfill etc.
	%(or a blank line to force the subfigure onto a new line)
	\begin{subfigure}[b]{0.49\textwidth}
	 \resizebox{\textwidth}{!}{% This file was created by matlab2tikz.
% Minimal pgfplots version: 1.3
%
%The latest updates can be retrieved from
%  http://www.mathworks.com/matlabcentral/fileexchange/22022-matlab2tikz
%where you can also make suggestions and rate matlab2tikz.
%
\definecolor{mycolor1}{rgb}{0.00000,0.44700,0.74100}%
\definecolor{mycolor2}{rgb}{0.85000,0.32500,0.09800}%
\definecolor{mycolor3}{rgb}{0.92900,0.69400,0.12500}%
\definecolor{mycolor4}{rgb}{0.49400,0.18400,0.55600}%
%
\begin{tikzpicture}

\begin{axis}[%
width=4in,
height=1.5in,
at={(1.011111in,0.641667in)},
scale only axis,
xmin=0,
xmax=1000,
tick align=outside,
xlabel={Iteration},
xmajorgrids,
ymin=-0.2,
ymax=1.6,
ylabel={Estimated Weight},
ymajorgrids,
title style={font=\bfseries},
title={Estmated Weights using CLMS},
axis x line*=bottom,
axis y line*=left,
legend style={legend cell align=left,align=left,draw=white!15!black}
]
\addplot [color=mycolor1,solid]
  table[row sep=crcr]{%
1	0\\
2	0.0125268491508049\\
3	0.0214465185662781\\
4	0.0306126001316651\\
5	0.0438318018455387\\
6	0.0509278901549515\\
7	0.054430740554203\\
8	0.0616935098376428\\
9	0.0717716664478995\\
10	0.0766967517061852\\
11	0.0869777308885347\\
12	0.0972146376552673\\
13	0.103075179868393\\
14	0.110327622667147\\
15	0.119659034778203\\
16	0.128759270352713\\
17	0.137646233382429\\
18	0.145318039706115\\
19	0.15354957436101\\
20	0.162206159066383\\
21	0.16546158013171\\
22	0.175709996363633\\
23	0.180457916527353\\
24	0.18844015953111\\
25	0.196935783975192\\
26	0.200795141154888\\
27	0.208528932659543\\
28	0.217601176483078\\
29	0.22851255010456\\
30	0.238478915176465\\
31	0.245570116343956\\
32	0.254379622584251\\
33	0.261396229847333\\
34	0.263243839938133\\
35	0.27186768571951\\
36	0.279333464383487\\
37	0.284252379622012\\
38	0.294726624824341\\
39	0.300354092892214\\
40	0.305770835996309\\
41	0.31080322906848\\
42	0.316529509393594\\
43	0.321907826414644\\
44	0.326865232041409\\
45	0.335977773220122\\
46	0.346470296242763\\
47	0.349222034159742\\
48	0.358913106562388\\
49	0.363569641723021\\
50	0.3693034514854\\
51	0.376526280424963\\
52	0.382955718846233\\
53	0.391376525434208\\
54	0.397567523777971\\
55	0.40299253818481\\
56	0.411800185899668\\
57	0.419825409174106\\
58	0.422615908454835\\
59	0.430810451893431\\
60	0.435029414433344\\
61	0.440634800812889\\
62	0.448339812348756\\
63	0.455877093182013\\
64	0.462905221505544\\
65	0.467909550279072\\
66	0.47462516896849\\
67	0.478697299362971\\
68	0.482828247689585\\
69	0.488986430366623\\
70	0.493713685430479\\
71	0.496881265907292\\
72	0.503964374568078\\
73	0.506636627653194\\
74	0.512868991265275\\
75	0.517136892188141\\
76	0.522704043126487\\
77	0.527810199064835\\
78	0.534284875337259\\
79	0.541301980760668\\
80	0.54628813005534\\
81	0.548929730782947\\
82	0.553188764103988\\
83	0.556450911625045\\
84	0.559377355124939\\
85	0.56327111362809\\
86	0.568144959728356\\
87	0.572927937214826\\
88	0.576154048988121\\
89	0.581808795662657\\
90	0.58515022619077\\
91	0.593751621050047\\
92	0.596129911570282\\
93	0.598145082266916\\
94	0.601318588004752\\
95	0.605738902514092\\
96	0.611870178734588\\
97	0.615048970039849\\
98	0.619442162829483\\
99	0.623834463931911\\
100	0.628340389326107\\
101	0.631880978839214\\
102	0.633565832027262\\
103	0.639049202608359\\
104	0.642327016849404\\
105	0.644651219287863\\
106	0.646028721642191\\
107	0.648968602156993\\
108	0.650726425647093\\
109	0.653097682311294\\
110	0.653881407985982\\
111	0.659030342019358\\
112	0.659488170002542\\
113	0.662936136331538\\
114	0.669005740946493\\
115	0.673039170477985\\
116	0.676703315586537\\
117	0.681314658975752\\
118	0.683952939603338\\
119	0.6881612009323\\
120	0.692707621712425\\
121	0.696572263481836\\
122	0.703799287387974\\
123	0.708221159594841\\
124	0.712031808777998\\
125	0.716295308540043\\
126	0.720624278192752\\
127	0.721736752105523\\
128	0.727968755022423\\
129	0.732290306613537\\
130	0.734279756937431\\
131	0.738300884034634\\
132	0.743118279966604\\
133	0.746604644506774\\
134	0.747473364772344\\
135	0.752907815515465\\
136	0.757052030408588\\
137	0.755883644044197\\
138	0.759571545213367\\
139	0.763690469859735\\
140	0.767504269639022\\
141	0.767328067145235\\
142	0.768462300447699\\
143	0.773173175375978\\
144	0.772407196372076\\
145	0.77811424037635\\
146	0.779780716223356\\
147	0.781927976322125\\
148	0.787243914409983\\
149	0.787425535325453\\
150	0.789776677590482\\
151	0.790734687520601\\
152	0.789828416776132\\
153	0.791711422585354\\
154	0.795669740567876\\
155	0.798793669290469\\
156	0.799897286932003\\
157	0.802060859432485\\
158	0.800495087970234\\
159	0.802569933252314\\
160	0.802397410097756\\
161	0.808447816367471\\
162	0.807938196362707\\
163	0.806064267834991\\
164	0.80863260490521\\
165	0.808304189312578\\
166	0.806307939230489\\
167	0.805149298102231\\
168	0.806237617546395\\
169	0.807292271897716\\
170	0.810458439164723\\
171	0.811874359595002\\
172	0.814211017770726\\
173	0.817103859330831\\
174	0.818140882397632\\
175	0.823263591442787\\
176	0.825791866162054\\
177	0.829623023927285\\
178	0.831917311166847\\
179	0.834465348582862\\
180	0.834576891396395\\
181	0.836898675421165\\
182	0.839228600066504\\
183	0.843452624658632\\
184	0.847391729787005\\
185	0.849767765040039\\
186	0.852311711566538\\
187	0.856632884020626\\
188	0.85654845157132\\
189	0.85902860905945\\
190	0.86059217621166\\
191	0.862007278938606\\
192	0.862446134155755\\
193	0.863774155767713\\
194	0.867893651198309\\
195	0.870769988192385\\
196	0.873475039496298\\
197	0.872131107218675\\
198	0.87102148073363\\
199	0.874007706660497\\
200	0.87573897809601\\
201	0.876726406822194\\
202	0.879582033807006\\
203	0.882276519430924\\
204	0.881689796891815\\
205	0.881766670961325\\
206	0.882309642232175\\
207	0.882935962219708\\
208	0.88416247158936\\
209	0.887286956702307\\
210	0.888203225787893\\
211	0.886030758174135\\
212	0.887085269212878\\
213	0.888608924311593\\
214	0.888625137624109\\
215	0.890349322502939\\
216	0.897259385659439\\
217	0.899098403747832\\
218	0.900347294202623\\
219	0.900335076289532\\
220	0.90224360055288\\
221	0.899073257979447\\
222	0.902817136345528\\
223	0.90482542750616\\
224	0.905432206594195\\
225	0.905906345396256\\
226	0.904583309668756\\
227	0.903054915275662\\
228	0.904914096632836\\
229	0.903906999068808\\
230	0.903971410545292\\
231	0.905363379528904\\
232	0.906643465542221\\
233	0.907256141809098\\
234	0.907449359943754\\
235	0.906518973747867\\
236	0.907473736978742\\
237	0.91151022239771\\
238	0.908859695980318\\
239	0.911186703790259\\
240	0.914107806620378\\
241	0.914821500005474\\
242	0.916468338480804\\
243	0.917129125371523\\
244	0.918110298556949\\
245	0.918716371165242\\
246	0.921659612694909\\
247	0.924108823846835\\
248	0.924988708162743\\
249	0.927901297376425\\
250	0.930786233607788\\
251	0.931696068154242\\
252	0.931938657789631\\
253	0.932042483100509\\
254	0.928517046638257\\
255	0.929053739689006\\
256	0.928245539916413\\
257	0.926652808280126\\
258	0.928018841797425\\
259	0.932065894206903\\
260	0.934150353110868\\
261	0.935782742404316\\
262	0.937122299071683\\
263	0.938844569244412\\
264	0.937660433305765\\
265	0.940272563896825\\
266	0.942062640554717\\
267	0.941424551758864\\
268	0.939779939270277\\
269	0.942124303914678\\
270	0.94131766777651\\
271	0.940232137966606\\
272	0.940019218904529\\
273	0.943992137344366\\
274	0.946050688143391\\
275	0.948234820712725\\
276	0.950376323135406\\
277	0.953168853917426\\
278	0.952932290656261\\
279	0.954380540781842\\
280	0.955980783753683\\
281	0.954885618526889\\
282	0.95364952732195\\
283	0.954684858035572\\
284	0.955329787309892\\
285	0.955462872773946\\
286	0.957317368842169\\
287	0.957633966986483\\
288	0.958045847154537\\
289	0.960148772741638\\
290	0.95905421586431\\
291	0.959170271996639\\
292	0.961157670453151\\
293	0.962942164216335\\
294	0.96563102079224\\
295	0.969252517059527\\
296	0.970575345523627\\
297	0.972964419442065\\
298	0.973495500312377\\
299	0.971942904168769\\
300	0.971251035240868\\
301	0.968920325678043\\
302	0.967375966053112\\
303	0.965424718420607\\
304	0.96364726764625\\
305	0.967661220823834\\
306	0.969906020377236\\
307	0.972569568293682\\
308	0.975966773988419\\
309	0.979452482956355\\
310	0.978332295813269\\
311	0.979600300515382\\
312	0.977928625331763\\
313	0.982271474763938\\
314	0.98281742822942\\
315	0.984692836136987\\
316	0.984272304799566\\
317	0.983799627433185\\
318	0.987118611158943\\
319	0.985437643292502\\
320	0.984747208257501\\
321	0.981586618102343\\
322	0.977942844333491\\
323	0.977396935048484\\
324	0.977778845602414\\
325	0.980270187834516\\
326	0.982188163995639\\
327	0.979488100257397\\
328	0.979678949967758\\
329	0.981221687562771\\
330	0.982233773812041\\
331	0.980041389958709\\
332	0.979297479136491\\
333	0.980770586088061\\
334	0.982088358184394\\
335	0.977284491158223\\
336	0.979393674152188\\
337	0.98048865484012\\
338	0.977447442967194\\
339	0.978749271872948\\
340	0.974304385714231\\
341	0.97393618602655\\
342	0.975170134439816\\
343	0.973522491234075\\
344	0.97248412811252\\
345	0.972125493943257\\
346	0.972946647082148\\
347	0.974495439358967\\
348	0.972547662918956\\
349	0.972586523073239\\
350	0.970528875037817\\
351	0.970986432590283\\
352	0.970390160739049\\
353	0.968382706085745\\
354	0.969041580922205\\
355	0.973784902908856\\
356	0.976443136234046\\
357	0.976117941879799\\
358	0.978874611685138\\
359	0.979925710603692\\
360	0.98003069485244\\
361	0.980466490784931\\
362	0.982240424749015\\
363	0.984378607653385\\
364	0.983044783322159\\
365	0.980555641556653\\
366	0.980548534950974\\
367	0.980051629381148\\
368	0.98104165955408\\
369	0.979914788982438\\
370	0.97858826215787\\
371	0.977808361758698\\
372	0.978567654234281\\
373	0.977559383225012\\
374	0.979833296113299\\
375	0.981215067712443\\
376	0.982963017235752\\
377	0.982986931370195\\
378	0.984477000174084\\
379	0.986417190261894\\
380	0.982612221711114\\
381	0.981769129718137\\
382	0.981307489342072\\
383	0.982808489513231\\
384	0.981760452958688\\
385	0.984549207160401\\
386	0.979776415866492\\
387	0.980902274515385\\
388	0.981283491563939\\
389	0.983975399908319\\
390	0.984674470291841\\
391	0.982712701700604\\
392	0.982238941014139\\
393	0.983396637570515\\
394	0.988527409824086\\
395	0.989672440486576\\
396	0.990202485566338\\
397	0.990090982807894\\
398	0.990884914139\\
399	0.98911536096453\\
400	0.988173751820294\\
401	0.988657219992268\\
402	0.987297598809629\\
403	0.986035110071471\\
404	0.98636414708336\\
405	0.986668071829898\\
406	0.985860625951785\\
407	0.984573597539325\\
408	0.984357256172689\\
409	0.985735559308994\\
410	0.983948751912664\\
411	0.986014494265225\\
412	0.98645465379541\\
413	0.986092277018709\\
414	0.985178129218864\\
415	0.985487073712212\\
416	0.985591248043374\\
417	0.986786707255383\\
418	0.987418279257062\\
419	0.987690221393186\\
420	0.985109322880091\\
421	0.987336464375162\\
422	0.984889348715173\\
423	0.984695044969109\\
424	0.98817639845097\\
425	0.988331625619952\\
426	0.986877146377457\\
427	0.986557562271192\\
428	0.987154500974123\\
429	0.988351538547558\\
430	0.98832628025346\\
431	0.987474259713944\\
432	0.988407810281215\\
433	0.988004604909356\\
434	0.989338738613429\\
435	0.988430308154613\\
436	0.98777334377525\\
437	0.987822851908264\\
438	0.9869398569665\\
439	0.985892890002455\\
440	0.985582914415495\\
441	0.984085386901343\\
442	0.984148862535218\\
443	0.985217363235621\\
444	0.985543089352643\\
445	0.988291285907932\\
446	0.988850871032163\\
447	0.989455654355137\\
448	0.990317257451129\\
449	0.988739720029788\\
450	0.990215174907582\\
451	0.991644343381152\\
452	0.990008998187462\\
453	0.987992316865681\\
454	0.986489625467555\\
455	0.987071761443765\\
456	0.985113799952759\\
457	0.98635000474248\\
458	0.985988365178786\\
459	0.985771948955349\\
460	0.984190928588687\\
461	0.984283043227896\\
462	0.984586451685082\\
463	0.984950236463132\\
464	0.982510760040845\\
465	0.982785197339485\\
466	0.982443785787732\\
467	0.983271845825299\\
468	0.984040774650917\\
469	0.983097201109923\\
470	0.985294961322868\\
471	0.98603001202071\\
472	0.986198860972344\\
473	0.986931270673589\\
474	0.986955671335388\\
475	0.987495465477863\\
476	0.989436283895844\\
477	0.988995124218943\\
478	0.989712031275643\\
479	0.992697618703954\\
480	0.991928868755741\\
481	0.994502526729555\\
482	0.996587258617264\\
483	0.996215117310833\\
484	0.995877838684566\\
485	0.998871229888198\\
486	0.998284127791051\\
487	0.998006504519143\\
488	0.997885633439842\\
489	0.997319504998316\\
490	0.998319290746392\\
491	0.999318414469479\\
492	0.998675115143723\\
493	0.999651708925973\\
494	1.00174069701698\\
495	1.00293106562081\\
496	1.00617612934761\\
497	1.00883099231049\\
498	1.00676818751348\\
499	1.00664274806874\\
500	1.0081134200052\\
501	1.00785971320003\\
502	1.00681689382571\\
503	1.00522794920636\\
504	1.00221548631359\\
505	0.999540700051504\\
506	1.00071726162882\\
507	0.999750391356671\\
508	1.00073035693973\\
509	1.00303726778965\\
510	1.00382236778841\\
511	1.00326847430062\\
512	1.00267103991498\\
513	1.00440593153355\\
514	1.00119249313564\\
515	1.00055498579072\\
516	1.00325427676856\\
517	1.00159113060222\\
518	1.00108704326308\\
519	1.00265261086799\\
520	1.0021705817575\\
521	1.00395777982238\\
522	1.00167869511458\\
523	0.999536764934836\\
524	0.99907752191546\\
525	1.00044255319838\\
526	0.999174280011269\\
527	1.00172274963356\\
528	1.00192343319761\\
529	1.00362108765368\\
530	1.00368722252928\\
531	1.00399781531215\\
532	1.00239063766924\\
533	1.00136086067709\\
534	1.00056946140634\\
535	1.00320672441816\\
536	1.0049573520933\\
537	1.00568717786887\\
538	1.00667685239238\\
539	1.00677103254509\\
540	1.00861648164877\\
541	1.00999853901008\\
542	1.0095249455215\\
543	1.00751880980285\\
544	1.00742386147879\\
545	1.00693808007327\\
546	1.00726603716909\\
547	1.00998909605939\\
548	1.01104450814363\\
549	1.01005593133253\\
550	1.00751728968203\\
551	1.00811755092374\\
552	1.002730872026\\
553	1.00207625870896\\
554	1.00109139267624\\
555	0.999357666074643\\
556	0.998156246045751\\
557	0.998078383519041\\
558	0.994461393738523\\
559	0.99430559968148\\
560	0.991999133754393\\
561	0.991436737809326\\
562	0.9919337641194\\
563	0.99421433559256\\
564	0.992998504565895\\
565	0.99612407504243\\
566	0.994406566685284\\
567	0.998531509475797\\
568	1.00201786417354\\
569	0.999393122742665\\
570	1.0000474664187\\
571	0.999532335404429\\
572	1.00000609770481\\
573	0.999925974536006\\
574	0.99859508283387\\
575	0.996836166434833\\
576	0.998429850677411\\
577	0.999494044678718\\
578	0.998950340754344\\
579	0.99554387188226\\
580	0.993742384439683\\
581	0.995040417640634\\
582	0.993411051277784\\
583	0.993592595484396\\
584	0.996152108382518\\
585	0.996722777294117\\
586	0.998875475449924\\
587	0.996305427778896\\
588	0.997441923451924\\
589	0.999890801648557\\
590	1.00243374560755\\
591	1.00481931566581\\
592	1.00441588162771\\
593	1.00494089244711\\
594	1.0038127375739\\
595	1.00162520608367\\
596	1.00100403602335\\
597	1.00037955175942\\
598	1.00341566516028\\
599	1.003384006709\\
600	1.00400861362461\\
601	1.00333383369596\\
602	1.00276594056158\\
603	1.00168304841857\\
604	1.00280903984191\\
605	1.00344944614361\\
606	1.00320324031889\\
607	1.00544642075781\\
608	1.00841087729106\\
609	1.00549241077072\\
610	1.00311658539069\\
611	1.00312715677798\\
612	1.0043073058022\\
613	1.00429951732471\\
614	1.0056670842378\\
615	1.00457987444325\\
616	1.00523080886398\\
617	1.00381240648998\\
618	1.00194883950271\\
619	1.00158368641757\\
620	1.00073739513076\\
621	1.00287207026606\\
622	1.00234292830539\\
623	1.00522051711405\\
624	1.00593675106508\\
625	1.00362345423657\\
626	1.00286061190149\\
627	1.00272115742649\\
628	1.00481987297858\\
629	1.0066043819833\\
630	1.00442009703708\\
631	1.00432407775461\\
632	1.00673395289925\\
633	1.009390003392\\
634	1.0092974187958\\
635	1.0102974732349\\
636	1.01117493424676\\
637	1.01244311704981\\
638	1.0115630124198\\
639	1.0099014070054\\
640	1.00790323965958\\
641	1.00805916361579\\
642	1.00911547656677\\
643	1.00952718886328\\
644	1.00963139525357\\
645	1.01066389101351\\
646	1.0088286850786\\
647	1.00765834142212\\
648	1.00856614866638\\
649	1.00785950611399\\
650	1.00724390480774\\
651	1.00318778887438\\
652	1.0044992272757\\
653	1.00186032868389\\
654	0.999651221558446\\
655	1.00251303740346\\
656	1.00236722033886\\
657	0.999302722539186\\
658	0.99904250844541\\
659	0.995856816660617\\
660	0.994926630402193\\
661	0.996122845452824\\
662	0.996420554564978\\
663	0.997230829332583\\
664	0.995599748927568\\
665	0.995210129111451\\
666	0.993390034552349\\
667	0.993217176931081\\
668	0.991914938717328\\
669	0.992587425626186\\
670	0.99128381028183\\
671	0.992573330661093\\
672	0.993234888227588\\
673	0.992042952599602\\
674	0.99242726321958\\
675	0.993606298850532\\
676	0.991468423943035\\
677	0.990305278068146\\
678	0.990813109595078\\
679	0.990235999491093\\
680	0.991016550156793\\
681	0.989840548979163\\
682	0.98761579242751\\
683	0.985578648733062\\
684	0.986782958230564\\
685	0.986015854394016\\
686	0.986122531459073\\
687	0.987426169771035\\
688	0.985970163894378\\
689	0.985256961398272\\
690	0.987649893745065\\
691	0.98966731785647\\
692	0.991954208255432\\
693	0.988793636007511\\
694	0.989234465814229\\
695	0.988714141941381\\
696	0.989842026843962\\
697	0.989314106738078\\
698	0.989841605989488\\
699	0.990923714015612\\
700	0.9900244255558\\
701	0.990660447068053\\
702	0.986590283696101\\
703	0.989214064109123\\
704	0.985336358702708\\
705	0.985667019798153\\
706	0.985460117226622\\
707	0.985426767479723\\
708	0.98325151023078\\
709	0.984955953377635\\
710	0.985715093450777\\
711	0.985090908422406\\
712	0.984843170684789\\
713	0.986083550020931\\
714	0.984181724765149\\
715	0.981741657874374\\
716	0.979374767364936\\
717	0.979029250033003\\
718	0.978505985229804\\
719	0.978794100015278\\
720	0.979010975796469\\
721	0.976866016614976\\
722	0.974052758075043\\
723	0.976998181159938\\
724	0.976300986754274\\
725	0.976699722576071\\
726	0.9799312964516\\
727	0.979120051095507\\
728	0.978492834536391\\
729	0.979838463665669\\
730	0.980308247668519\\
731	0.979179407053133\\
732	0.980455085180619\\
733	0.978482475103768\\
734	0.977946919675511\\
735	0.978297639448419\\
736	0.97974256855749\\
737	0.981528184927984\\
738	0.984826835860132\\
739	0.982478545995502\\
740	0.982290539241303\\
741	0.982336139276418\\
742	0.977573025116605\\
743	0.977202410613538\\
744	0.976649637231298\\
745	0.973952548144562\\
746	0.975950398197839\\
747	0.977542193815609\\
748	0.981017025992393\\
749	0.978313342703029\\
750	0.975853216227124\\
751	0.976367580945566\\
752	0.978201030307241\\
753	0.978281072911787\\
754	0.977218922259295\\
755	0.974507883285721\\
756	0.971767261000378\\
757	0.970737796780659\\
758	0.973799104684578\\
759	0.972860717731971\\
760	0.975298715392517\\
761	0.97636784650753\\
762	0.97313009033283\\
763	0.974650034487734\\
764	0.975227916500047\\
765	0.97195912817613\\
766	0.969875874318336\\
767	0.971381765371319\\
768	0.968203988489196\\
769	0.971012694288862\\
770	0.969930583378294\\
771	0.970267131104794\\
772	0.970640199916948\\
773	0.971788816019218\\
774	0.971071923325572\\
775	0.973536504440593\\
776	0.975136615851274\\
777	0.97915944862098\\
778	0.977579651644151\\
779	0.974360828051346\\
780	0.974976776373812\\
781	0.974291433251479\\
782	0.97221757642258\\
783	0.970528140176599\\
784	0.970013949294772\\
785	0.967012042480859\\
786	0.968117017882642\\
787	0.967067059971355\\
788	0.968570516878527\\
789	0.968720020384789\\
790	0.96838527410598\\
791	0.969591209600815\\
792	0.96847155866557\\
793	0.97272057411856\\
794	0.973461169151174\\
795	0.975176105388832\\
796	0.975986234587445\\
797	0.974761539978462\\
798	0.971420112682317\\
799	0.969631466417535\\
800	0.970143312820976\\
801	0.968849643066751\\
802	0.965646001636939\\
803	0.968740379151351\\
804	0.96664342363013\\
805	0.967931012785179\\
806	0.96763293174216\\
807	0.969636816423773\\
808	0.973081998294683\\
809	0.971319563833092\\
810	0.971752496771197\\
811	0.973172361134022\\
812	0.973555019365907\\
813	0.974011972871292\\
814	0.97364325115199\\
815	0.97758044252795\\
816	0.978673860964248\\
817	0.979092566697853\\
818	0.981077146078702\\
819	0.982913874512634\\
820	0.98410277227874\\
821	0.982215869880635\\
822	0.983185683203195\\
823	0.983476033850245\\
824	0.986700724650412\\
825	0.986395957675259\\
826	0.98745618854419\\
827	0.990731765524544\\
828	0.991359062189936\\
829	0.991321552460225\\
830	0.990055851132429\\
831	0.988906975729379\\
832	0.989020983266054\\
833	0.990629078526984\\
834	0.990902598147532\\
835	0.993637441708995\\
836	0.999050422912368\\
837	0.996686327835483\\
838	0.99588681613827\\
839	0.99767507413386\\
840	0.995035197506401\\
841	0.993774551004301\\
842	0.99325959291292\\
843	0.995804361188273\\
844	0.994844895537087\\
845	0.994387334983211\\
846	0.995087279202774\\
847	0.994694734107234\\
848	0.995294690884731\\
849	0.994405896595521\\
850	0.994372609495346\\
851	0.992536162043256\\
852	0.991672398959541\\
853	0.991431486681328\\
854	0.98839435841837\\
855	0.989781197078478\\
856	0.986210417786132\\
857	0.986044790733023\\
858	0.986485533678378\\
859	0.987625838720531\\
860	0.988375567320531\\
861	0.989992295143442\\
862	0.990614395270217\\
863	0.990549773980127\\
864	0.992504376438015\\
865	0.992566101842612\\
866	0.991060665078254\\
867	0.98988478228692\\
868	0.987400771818135\\
869	0.986233424030136\\
870	0.98515139166128\\
871	0.985864428874132\\
872	0.987398707644086\\
873	0.987673346695966\\
874	0.988819890877668\\
875	0.98610678397949\\
876	0.982450050193083\\
877	0.979152189707472\\
878	0.976944029472327\\
879	0.977258231828988\\
880	0.975403315734014\\
881	0.973008997992168\\
882	0.975952523621384\\
883	0.973899411904924\\
884	0.976126552597026\\
885	0.975933905089002\\
886	0.974476065575199\\
887	0.975973663286789\\
888	0.975682711689328\\
889	0.974115027976657\\
890	0.975653939323302\\
891	0.976496563035124\\
892	0.976736359345028\\
893	0.976520131444653\\
894	0.977136910430046\\
895	0.980346345285357\\
896	0.980556020668544\\
897	0.983371049293121\\
898	0.984258979850928\\
899	0.985692542401638\\
900	0.986428313114863\\
901	0.987804720554439\\
902	0.984093912295469\\
903	0.983023983746179\\
904	0.983073721616914\\
905	0.985006303870455\\
906	0.982812781206977\\
907	0.983922502189736\\
908	0.981426987911577\\
909	0.98204774923734\\
910	0.981895313229314\\
911	0.981797377686006\\
912	0.98133427836594\\
913	0.986137257468272\\
914	0.986046711731051\\
915	0.986804233811188\\
916	0.98804343271004\\
917	0.986888985471459\\
918	0.986452554093318\\
919	0.985911341844679\\
920	0.988060771165685\\
921	0.989048037961729\\
922	0.988946258481775\\
923	0.988153970751658\\
924	0.984379507553311\\
925	0.983669810958552\\
926	0.985050992575191\\
927	0.985310489094384\\
928	0.986530836559773\\
929	0.986542865724224\\
930	0.985408078517887\\
931	0.98377716631137\\
932	0.986234465112633\\
933	0.986424240375664\\
934	0.98367186832452\\
935	0.983904472986656\\
936	0.984052037633297\\
937	0.983027879903963\\
938	0.983727627651981\\
939	0.985424463595522\\
940	0.981827544009428\\
941	0.984300242911514\\
942	0.9823556128519\\
943	0.982504099809106\\
944	0.982858718516433\\
945	0.983435063146199\\
946	0.984008321603555\\
947	0.986070622668052\\
948	0.986605487208302\\
949	0.989043344778251\\
950	0.99035290142529\\
951	0.988820523521291\\
952	0.990884076806313\\
953	0.991882533065716\\
954	0.990997483863705\\
955	0.990271445772986\\
956	0.99203487264591\\
957	0.991961299326523\\
958	0.992924261451618\\
959	0.991610998433747\\
960	0.990085613403338\\
961	0.992362840234089\\
962	0.99219608008028\\
963	0.991810284230637\\
964	0.992897085278387\\
965	0.993385174094213\\
966	0.991229212698903\\
967	0.991761447920058\\
968	0.996118362246128\\
969	0.998425688207256\\
970	0.997070865883421\\
971	0.993740371917591\\
972	0.993816976779922\\
973	0.995063041195511\\
974	0.995383789902828\\
975	0.995489220045212\\
976	0.992406245362031\\
977	0.991862161979792\\
978	0.988219869305166\\
979	0.98609057431477\\
980	0.986047256160306\\
981	0.986156368559226\\
982	0.984128987089249\\
983	0.983833543732287\\
984	0.985587535127592\\
985	0.984745817222673\\
986	0.985659741756418\\
987	0.98505792336154\\
988	0.987019437291237\\
989	0.986341551954047\\
990	0.987999384206286\\
991	0.990105873694327\\
992	0.991272194847686\\
993	0.99226484890537\\
994	0.993398303871292\\
995	0.992831833831998\\
996	0.991406938386419\\
997	0.99482315714933\\
998	0.993776219732379\\
999	0.992663884201089\\
1000	0.993338983178949\\
};
\addlegendentry{$\Re \{ \mathbf{h}_1^H\} $};

\addplot [color=mycolor2,solid]
  table[row sep=crcr]{%
1	0\\
2	4.5720720910419e-20\\
3	0.00234768919877495\\
4	0.00140549665353364\\
5	-0.000860657511390143\\
6	-0.000172705346366565\\
7	-0.00214855577517097\\
8	-0.00892565123049696\\
9	-0.0066525962318819\\
10	-0.00771012361644595\\
11	-0.00806816435675521\\
12	-0.00509732958555119\\
13	-0.00556627720641943\\
14	-0.00583187515050655\\
15	-0.00424193932825979\\
16	-0.00311627403854779\\
17	-0.00373859697848007\\
18	-0.00448452779560819\\
19	-0.0028712597980663\\
20	-0.00389025529673588\\
21	-0.00289330483993113\\
22	-0.00726640327981308\\
23	-0.00558068324845074\\
24	-0.00924901363745483\\
25	-0.0108818866229797\\
26	-0.00877353628486499\\
27	-0.00817182562987243\\
28	-0.00974828865965106\\
29	-0.0111087740174249\\
30	-0.00766691475497771\\
31	-0.00309063887345565\\
32	-0.00682210632575721\\
33	-0.00481449746092303\\
34	-0.00793453026154023\\
35	-0.0110579668553145\\
36	-0.00823957669615097\\
37	-0.0077242585157979\\
38	-0.00881482278215532\\
39	-0.00797322592692126\\
40	-0.0106351690458397\\
41	-0.00957353250917\\
42	-0.0112662819268154\\
43	-0.00939624178088952\\
44	-0.0138105069476743\\
45	-0.0163725711604565\\
46	-0.0185698648888489\\
47	-0.0187790076108323\\
48	-0.0197820891022692\\
49	-0.019037384341201\\
50	-0.0194230539129963\\
51	-0.0164956790391966\\
52	-0.016486386854078\\
53	-0.0144177099113857\\
54	-0.0130981555293674\\
55	-0.0111474515878152\\
56	-0.0110181848510115\\
57	-0.0129951678387393\\
58	-0.0104826586878032\\
59	-0.0123585352153961\\
60	-0.0106485616056248\\
61	-0.0107252666466631\\
62	-0.0134647959048677\\
63	-0.0115886929780389\\
64	-0.0146153976218434\\
65	-0.0144308445085552\\
66	-0.014243553166407\\
67	-0.0160879148929442\\
68	-0.0188758993990693\\
69	-0.0207689830892961\\
70	-0.0201347220658222\\
71	-0.0197763392083428\\
72	-0.0213347858087133\\
73	-0.020768663379361\\
74	-0.0212804048778209\\
75	-0.02156002927396\\
76	-0.0223264852960294\\
77	-0.0223372959286574\\
78	-0.0216031750299187\\
79	-0.0218828231568984\\
80	-0.0208170472652917\\
81	-0.0198818111115544\\
82	-0.019500528804526\\
83	-0.0210885127971568\\
84	-0.0209862385219677\\
85	-0.0210836971433851\\
86	-0.0186382557299881\\
87	-0.0163460260723136\\
88	-0.0154120866136501\\
89	-0.0131789740688731\\
90	-0.0143910848881984\\
91	-0.0136058624541153\\
92	-0.0131679596804076\\
93	-0.0127562462221484\\
94	-0.0125335923376148\\
95	-0.0128444939358302\\
96	-0.0155742239588138\\
97	-0.0119013979753157\\
98	-0.0120328598032696\\
99	-0.0106995892791732\\
100	-0.00980510443132761\\
101	-0.0092070661019335\\
102	-0.00826743339756113\\
103	-0.00573094648211847\\
104	-0.0038662731714637\\
105	-0.00473722863026655\\
106	-0.00484736884403787\\
107	-0.00569675237626842\\
108	-0.00806830166500803\\
109	-0.00883354647039524\\
110	-0.00844812303798889\\
111	-0.0103860148030108\\
112	-0.00770231947368176\\
113	-0.00451057447345828\\
114	-0.00384651905125684\\
115	-0.00544283937755881\\
116	-0.00737984750554208\\
117	-0.00885457084737299\\
118	-0.00661017486053398\\
119	-0.00576260021803775\\
120	-0.00574383741986813\\
121	-0.0084326985590774\\
122	-0.0107702351119902\\
123	-0.010836670140588\\
124	-0.00981300042350513\\
125	-0.0129324947611902\\
126	-0.0148605636138384\\
127	-0.0171665720388141\\
128	-0.0144307964097909\\
129	-0.0153957286203156\\
130	-0.0186505351217379\\
131	-0.020422734275573\\
132	-0.017730332752735\\
133	-0.0186712037924273\\
134	-0.0170705437432193\\
135	-0.0163052926751219\\
136	-0.0190117974908795\\
137	-0.0220593026784505\\
138	-0.0220210251159674\\
139	-0.0216137228579382\\
140	-0.0229374211185978\\
141	-0.0193185140377368\\
142	-0.0198833235238782\\
143	-0.0200116447897607\\
144	-0.0208899203473005\\
145	-0.016806336873719\\
146	-0.0159273953117188\\
147	-0.0153451040403636\\
148	-0.0146443871957115\\
149	-0.0143136857970759\\
150	-0.0157658016451766\\
151	-0.0160651251878893\\
152	-0.0165887193500637\\
153	-0.0201064044210513\\
154	-0.021402321928321\\
155	-0.0231672261145296\\
156	-0.0218499127601276\\
157	-0.0246425343440035\\
158	-0.0219540826034511\\
159	-0.0199695275164762\\
160	-0.0189628958435188\\
161	-0.0188042246997484\\
162	-0.0187123246798378\\
163	-0.0218754508393586\\
164	-0.0270050953857191\\
165	-0.0281682534230022\\
166	-0.0281672273012119\\
167	-0.0266157750239058\\
168	-0.0263787080795113\\
169	-0.0279173041478342\\
170	-0.0290375852003668\\
171	-0.0277729850374479\\
172	-0.0275805530255224\\
173	-0.0268758997482312\\
174	-0.0275064266130513\\
175	-0.0294083500232175\\
176	-0.0248004332037427\\
177	-0.0229882978508633\\
178	-0.0226375511210622\\
179	-0.0214484684602218\\
180	-0.0194884008349414\\
181	-0.0184495143451906\\
182	-0.0172286434539271\\
183	-0.017528408414814\\
184	-0.0152339435749657\\
185	-0.0144164392593042\\
186	-0.0142235866414814\\
187	-0.0158144858737739\\
188	-0.0172327475104284\\
189	-0.0148553534010585\\
190	-0.0151503448892106\\
191	-0.0137103366691039\\
192	-0.0139001973635714\\
193	-0.0159071747789181\\
194	-0.0139690840770287\\
195	-0.016048559971595\\
196	-0.0202935137587809\\
197	-0.0197799484591473\\
198	-0.0198808903440486\\
199	-0.020950810993002\\
200	-0.0217832949437361\\
201	-0.0203659427639355\\
202	-0.0200426697558871\\
203	-0.0217577345004566\\
204	-0.0219326695904449\\
205	-0.0243460372123799\\
206	-0.0232389250527018\\
207	-0.0210097741309708\\
208	-0.0183718208191495\\
209	-0.0187245797057071\\
210	-0.0179783748549852\\
211	-0.0175540557065881\\
212	-0.0203729128655892\\
213	-0.0203858556160342\\
214	-0.0196866101813333\\
215	-0.0211115822573313\\
216	-0.0216652037341541\\
217	-0.0205124176756317\\
218	-0.0186661486129533\\
219	-0.019478232479479\\
220	-0.0206949388172315\\
221	-0.0202414968907633\\
222	-0.0199124037433254\\
223	-0.0189298512954435\\
224	-0.0171145105968414\\
225	-0.0167509346485607\\
226	-0.0136986691074103\\
227	-0.0148059444374544\\
228	-0.0139405494003334\\
229	-0.0162884883825478\\
230	-0.0161503746087381\\
231	-0.0185198822707033\\
232	-0.0213348609152214\\
233	-0.020612076114291\\
234	-0.0202324118218324\\
235	-0.0189701623972297\\
236	-0.0177505091995525\\
237	-0.0141689888270692\\
238	-0.0136569791860191\\
239	-0.0132149426937115\\
240	-0.0102526478283962\\
241	-0.01061194230571\\
242	-0.00931688333707295\\
243	-0.00798903379334319\\
244	-0.00728010685105677\\
245	-0.00777470098885888\\
246	-0.00852858618115789\\
247	-0.00789319582667687\\
248	-0.00808592963540983\\
249	-0.00762047143848037\\
250	-0.00870115206897224\\
251	-0.00805209571915417\\
252	-0.00469577046324118\\
253	-0.00124028504450114\\
254	-0.00302414971644188\\
255	-0.00437623550459862\\
256	-0.00296076399977817\\
257	-0.00577337863241548\\
258	-0.00462982289732987\\
259	-0.00931107519642017\\
260	-0.00840812202310971\\
261	-0.00907740403394645\\
262	-0.00679200024290002\\
263	-0.0083447539263084\\
264	-0.00739997267025289\\
265	-0.00793797658775411\\
266	-0.0121305051592194\\
267	-0.0120710217771123\\
268	-0.0133099027562062\\
269	-0.0121875341256362\\
270	-0.0114257854515885\\
271	-0.00999136104311243\\
272	-0.00855216047401336\\
273	-0.00881521828704664\\
274	-0.00945993946694507\\
275	-0.0079388813195437\\
276	-0.00890488118336988\\
277	-0.00867103424662882\\
278	-0.00465329347400982\\
279	-0.00416644311958476\\
280	-0.00443178757597499\\
281	-0.00399314043546362\\
282	-0.00341893913357509\\
283	-0.00475215449177443\\
284	-0.00462248247052357\\
285	-0.00485550439186878\\
286	-0.00798297783060383\\
287	-0.00534564843794289\\
288	-0.0104310251055498\\
289	-0.0123041070771894\\
290	-0.0146079222786232\\
291	-0.0106138392303882\\
292	-0.00885341380092552\\
293	-0.00971406804383277\\
294	-0.00986094590772823\\
295	-0.0129210085855126\\
296	-0.0126641432183021\\
297	-0.016700996207717\\
298	-0.0184476236244324\\
299	-0.015097276631635\\
300	-0.0175781119017127\\
301	-0.017213229325495\\
302	-0.0139951700615481\\
303	-0.0125975021644316\\
304	-0.0121796170002191\\
305	-0.00951395946761391\\
306	-0.0114464169644535\\
307	-0.0113199822273607\\
308	-0.0126099969993627\\
309	-0.015362249560127\\
310	-0.0161628087811997\\
311	-0.0138176808254134\\
312	-0.0126716772320904\\
313	-0.0147902676628681\\
314	-0.0170885654122904\\
315	-0.016850167119925\\
316	-0.0167379854006959\\
317	-0.0157783463724322\\
318	-0.0152120191818131\\
319	-0.0132949877170643\\
320	-0.0141268061245719\\
321	-0.0114141496005008\\
322	-0.0108831340707747\\
323	-0.00995269336255446\\
324	-0.00789622823818687\\
325	-0.00858866391675342\\
326	-0.0112440369969612\\
327	-0.0130500592815683\\
328	-0.0111072165455312\\
329	-0.010185267659014\\
330	-0.00892996704427258\\
331	-0.00888895701590436\\
332	-0.00771302643711044\\
333	-0.0106057924418061\\
334	-0.00833514914602349\\
335	-0.0083907334632316\\
336	-0.00769015608945497\\
337	-0.00773027219286612\\
338	-0.00604060555873029\\
339	-0.00218005871952598\\
340	-0.000215232742744067\\
341	-0.00113892208366907\\
342	-0.00281375109143955\\
343	-0.00456600882490605\\
344	-0.00299353811398007\\
345	0.000424959912882966\\
346	0.00243041646803329\\
347	0.00182491249807814\\
348	0.00174253847348406\\
349	0.00231038526817432\\
350	0.00355655841642668\\
351	0.00583344436272298\\
352	0.00488461850197287\\
353	0.00450877137101378\\
354	0.00401039731994221\\
355	0.00193912432970016\\
356	0.000241456549391624\\
357	0.00436073075486993\\
358	0.00240171459171958\\
359	0.00639027378710438\\
360	0.00616231959432261\\
361	0.00427813384429228\\
362	0.00335800945743044\\
363	0.00243212812041793\\
364	0.00166577789359775\\
365	0.00203246557204976\\
366	0.00207294468036498\\
367	-0.00124527844598604\\
368	-0.00165919888131212\\
369	-0.000669373865688415\\
370	0.000636953307213776\\
371	0.000117223665266916\\
372	0.00106613299167079\\
373	0.00179706587989394\\
374	0.00401348759586482\\
375	0.00391091714174683\\
376	0.00362629202328454\\
377	0.000716159566999726\\
378	0.00189380018764823\\
379	0.000418733051168269\\
380	0.00308209435388243\\
381	0.00320170902113681\\
382	0.000973628818044348\\
383	0.00201019250360502\\
384	0.0032156043481385\\
385	0.00403327095269627\\
386	0.00605390204890421\\
387	0.00592525475259641\\
388	0.00843065656021276\\
389	0.00784998384251428\\
390	0.00817299190744087\\
391	0.00792225102990167\\
392	0.0112285970759738\\
393	0.00856350270824732\\
394	0.0118757943448481\\
395	0.0136268872879336\\
396	0.0112077647298305\\
397	0.00993532820223599\\
398	0.00864339264586532\\
399	0.0111574967581625\\
400	0.0104233291421124\\
401	0.0106117255094637\\
402	0.0114304384659951\\
403	0.0110696941283814\\
404	0.0124143404691773\\
405	0.012396017886703\\
406	0.0112129910811158\\
407	0.00789943339332663\\
408	0.00750605211224453\\
409	0.00900382533538898\\
410	0.0113800366916382\\
411	0.00647893795954936\\
412	0.00647394818516569\\
413	0.0082933773539188\\
414	0.00697968989527455\\
415	0.00783380682239441\\
416	0.00851520286249679\\
417	0.00690008518647353\\
418	0.00525656347025619\\
419	0.00789103024900583\\
420	0.00453645839329591\\
421	0.00455596323610712\\
422	0.00287994307619109\\
423	-0.000318482168379873\\
424	0.000546956585709375\\
425	0.00271852075423827\\
426	0.00203308427443428\\
427	0.00312568126131093\\
428	0.00342588516498702\\
429	0.0056931879328313\\
430	0.00658104292894006\\
431	0.00620506631429447\\
432	0.00224659830848938\\
433	-0.00129171499247285\\
434	-0.000989614999386327\\
435	-0.00183086418259916\\
436	-0.000293033919238408\\
437	0.00093932538656836\\
438	0.000318373200694996\\
439	-0.00154984607319008\\
440	-0.00320179660536647\\
441	-0.00119628680323233\\
442	0.000725752504895735\\
443	0.00022444451673819\\
444	-0.00203252641156981\\
445	-0.000190915493038083\\
446	0.00154608515984158\\
447	0.000845930801797639\\
448	0.000773852457862492\\
449	5.79077623430155e-05\\
450	0.00373779032632301\\
451	0.00248309710272855\\
452	0.00389431048701519\\
453	0.00626734412831445\\
454	0.0057636409952772\\
455	0.00298296602648114\\
456	0.00380385787723001\\
457	0.000374349389083439\\
458	-0.000624541864479312\\
459	-0.000212286378284092\\
460	0.00477249619917401\\
461	0.00716034886026035\\
462	0.00891618852798426\\
463	0.00974058304978644\\
464	0.0108914766048229\\
465	0.0081928275678403\\
466	0.00597110135898125\\
467	0.00918877793054511\\
468	0.00793012429918556\\
469	0.00740651121869371\\
470	0.00833187817169007\\
471	0.0106965374897326\\
472	0.0145457377018124\\
473	0.0164172441081886\\
474	0.0151558683306466\\
475	0.015756885429416\\
476	0.0142637926580313\\
477	0.0184143154818225\\
478	0.0171980809165725\\
479	0.0204777121649465\\
480	0.0197503166250637\\
481	0.0206378938647284\\
482	0.0211846957228382\\
483	0.021728697302002\\
484	0.0223806236983649\\
485	0.0248940538194852\\
486	0.023154746082437\\
487	0.0221728712090108\\
488	0.0207230353613866\\
489	0.0190637943105049\\
490	0.0184421171240485\\
491	0.0171929842959645\\
492	0.0153453875649211\\
493	0.0143653753898844\\
494	0.0147908174980029\\
495	0.0144752851310251\\
496	0.0131170329660069\\
497	0.0127521477531432\\
498	0.0114912442793815\\
499	0.0126188344650516\\
500	0.0125685220893044\\
501	0.0116470796406369\\
502	0.0112535303633244\\
503	0.00820348824279181\\
504	0.00647809262038866\\
505	0.00505292412668477\\
506	0.00397629713752376\\
507	0.00716785531607876\\
508	0.00747567871335833\\
509	0.00281059410749157\\
510	0.00263434028610953\\
511	0.0031519306959867\\
512	0.0028977031360493\\
513	0.00379266477962407\\
514	0.00411713334841724\\
515	0.00248409982741675\\
516	0.00148557562142345\\
517	0.000499354716715485\\
518	0.00264386341651858\\
519	3.43151610528326e-05\\
520	0.000136811053280643\\
521	-0.00253082727163242\\
522	-0.00126228768757048\\
523	-0.000895304720356616\\
524	0.00124372569374727\\
525	0.00174113619678973\\
526	0.0016315395032104\\
527	0.00496691677310845\\
528	0.00538918843807761\\
529	0.00653082570974282\\
530	0.00692947119858898\\
531	0.00556828535538309\\
532	0.0042179659070432\\
533	0.00421728473497468\\
534	0.00311521529548106\\
535	0.00456900053457329\\
536	0.00657491599090154\\
537	0.00587453308194124\\
538	0.00489325487222127\\
539	0.00592096670264691\\
540	0.00624194845451731\\
541	0.0080423450816623\\
542	0.00778773552944697\\
543	0.00575037012605977\\
544	0.00614218942147506\\
545	0.00483506487395067\\
546	0.00587112574123805\\
547	0.00688850427635587\\
548	0.00809117966539907\\
549	0.00819803967007467\\
550	0.00733846701891975\\
551	0.00688548606729875\\
552	0.00841060423965045\\
553	0.0105911906303132\\
554	0.0116762209498648\\
555	0.0115863068080432\\
556	0.0103060542587274\\
557	0.00912831852934555\\
558	0.00957107822375077\\
559	0.00860842432741789\\
560	0.00504120624648586\\
561	0.00869553884384812\\
562	0.00737652944196135\\
563	0.00472850423430196\\
564	0.00490708727777018\\
565	0.00631841780354913\\
566	0.00607894921562108\\
567	0.00584047527824318\\
568	0.00537021895765959\\
569	0.0074035352881914\\
570	0.0105554193196934\\
571	0.0088524902200415\\
572	0.00861285298315137\\
573	0.0129930313059493\\
574	0.0146806509212435\\
575	0.0157458041007326\\
576	0.0140548765392606\\
577	0.0134835032741476\\
578	0.013716486054535\\
579	0.0150504979612723\\
580	0.0144163884494358\\
581	0.0124854407157432\\
582	0.00938810704819725\\
583	0.0113492735128741\\
584	0.0140455466391044\\
585	0.012382958890828\\
586	0.013311441262918\\
587	0.0154937964792233\\
588	0.01713466001558\\
589	0.0200321461204177\\
590	0.0202199528943649\\
591	0.0199063985140726\\
592	0.0201001922577065\\
593	0.0215314812308205\\
594	0.0195000564032555\\
595	0.0187544433942346\\
596	0.0179612407222243\\
597	0.0160207007050806\\
598	0.0142790651394113\\
599	0.0177283642670005\\
600	0.0186008445456169\\
601	0.01753855398955\\
602	0.0157815342691323\\
603	0.0129486779170674\\
604	0.0152018879610729\\
605	0.0172104840916756\\
606	0.0156992362485199\\
607	0.0171573860083357\\
608	0.0160070284388362\\
609	0.0163212937005849\\
610	0.0147163167571948\\
611	0.0160189043442828\\
612	0.0156938638525601\\
613	0.0126539006896483\\
614	0.0115506855420743\\
615	0.0138115695696831\\
616	0.0160683273209553\\
617	0.0152959873773214\\
618	0.0193357765084439\\
619	0.0195284671306135\\
620	0.0155136421277896\\
621	0.0142322285347654\\
622	0.0130814111068618\\
623	0.0172762615430311\\
624	0.0203496909913591\\
625	0.0191772002963956\\
626	0.0192714582362742\\
627	0.0181325829191513\\
628	0.0200751846826266\\
629	0.0170078397494237\\
630	0.0152741179949222\\
631	0.0134219405730056\\
632	0.01345710913794\\
633	0.0139417234075877\\
634	0.0162013312972711\\
635	0.016808840865165\\
636	0.0165723732436009\\
637	0.0157949311382532\\
638	0.0161320391953867\\
639	0.0157573892973492\\
640	0.0158910861211462\\
641	0.0131907517762954\\
642	0.0102717705069292\\
643	0.00982911700120459\\
644	0.00746412891819965\\
645	0.00691334027543681\\
646	0.00542402203418367\\
647	0.00435642857480594\\
648	0.00602692997655242\\
649	0.00419490760432168\\
650	0.00655569603204987\\
651	0.00559054569092579\\
652	0.00521892343772652\\
653	0.00727636995757762\\
654	0.00701478597210693\\
655	0.0100208640739963\\
656	0.0134737602784612\\
657	0.0123592077892318\\
658	0.00949357766653181\\
659	0.0105771614984724\\
660	0.0119004753206334\\
661	0.0144084733814706\\
662	0.0150488623443239\\
663	0.0139106724424591\\
664	0.0142717969011796\\
665	0.0174088492274968\\
666	0.0181771903866405\\
667	0.0177756141519959\\
668	0.0171349068659275\\
669	0.0151985816137471\\
670	0.00982637999479504\\
671	0.00987258848826426\\
672	0.0125382766571511\\
673	0.0109169540084054\\
674	0.00947976565475023\\
675	0.00624833899267995\\
676	0.00587860466108405\\
677	0.00782130496034726\\
678	0.0083343702399646\\
679	0.00579236125426018\\
680	0.00701783441514883\\
681	0.00658758451777268\\
682	0.00259121311806382\\
683	0.00334718562975807\\
684	0.002738697477798\\
685	0.00613669717441955\\
686	0.00681412047745377\\
687	0.00889130305150235\\
688	0.0117924947972042\\
689	0.0133763484290967\\
690	0.0111192115313794\\
691	0.012268199830183\\
692	0.0104482938438465\\
693	0.0103643127782799\\
694	0.0103512948115605\\
695	0.0100664087403397\\
696	0.012122847984574\\
697	0.0139466262831743\\
698	0.0129401799938555\\
699	0.0118016648796793\\
700	0.0131784126963949\\
701	0.013151884696431\\
702	0.0140505511751081\\
703	0.0144625064570605\\
704	0.0155154211863212\\
705	0.0153863996011475\\
706	0.0145710384629179\\
707	0.0172563985880791\\
708	0.0140051838981566\\
709	0.0120214425935041\\
710	0.0123882243145653\\
711	0.0101302115397617\\
712	0.00779649293144087\\
713	0.0104920136066246\\
714	0.0115847534665601\\
715	0.0123074864826826\\
716	0.0156891768928543\\
717	0.0149281704289906\\
718	0.012422704715457\\
719	0.0124666156184747\\
720	0.0130950674218109\\
721	0.0123433003577273\\
722	0.0153958031568857\\
723	0.0214869835663824\\
724	0.0211180190408793\\
725	0.0231328767838147\\
726	0.0199966043013972\\
727	0.0213872999704998\\
728	0.0175781717887164\\
729	0.0198285496348149\\
730	0.0159978383426702\\
731	0.0164925094132677\\
732	0.0159240594730243\\
733	0.0157823347324986\\
734	0.0139165047545556\\
735	0.0109306886849073\\
736	0.00915885785239747\\
737	0.00901428889515305\\
738	0.0105132942433171\\
739	0.0116172763269438\\
740	0.0130974726723635\\
741	0.010874335555056\\
742	0.0100767566061789\\
743	0.00676159687375222\\
744	0.00530283437260112\\
745	0.00612642391337439\\
746	0.00297854142719876\\
747	0.000314625409514066\\
748	-0.000153354325753149\\
749	0.00426784678478924\\
750	0.00631772937685821\\
751	0.00729813127563942\\
752	0.00638906753865496\\
753	0.00422205327074308\\
754	0.00495457202601756\\
755	0.0104155672965514\\
756	0.0107285918695092\\
757	0.0144010136266823\\
758	0.0130523073919773\\
759	0.0131333175092797\\
760	0.0132232704177474\\
761	0.0136027060351188\\
762	0.0129414101751252\\
763	0.0103896434571377\\
764	0.00922340380117752\\
765	0.00899111411460297\\
766	0.0139488538741907\\
767	0.0144605457892984\\
768	0.0118933617235822\\
769	0.0136643863221405\\
770	0.0133340830817898\\
771	0.0113374502761577\\
772	0.00984201926098376\\
773	0.00576525716820205\\
774	0.0053088879531392\\
775	0.00697668562201404\\
776	0.00578313300627366\\
777	0.00529558927817922\\
778	0.0045877914852988\\
779	0.00532596446643193\\
780	0.00468585569601194\\
781	0.00223939423883828\\
782	0.00119837884414011\\
783	0.00122501343481417\\
784	0.0027936622497823\\
785	0.00372944288801607\\
786	0.00545487157641162\\
787	0.00527173272853269\\
788	0.0064051828277367\\
789	0.00724638339065795\\
790	0.0101586670330675\\
791	0.0107104185631908\\
792	0.0122783980949735\\
793	0.0112990215868999\\
794	0.0114819054658274\\
795	0.0107562303606668\\
796	0.0063714524896655\\
797	0.00765138396941327\\
798	0.00586502671264672\\
799	0.00563270929717371\\
800	0.00576513957490725\\
801	0.000944474854008809\\
802	0.00283638943627117\\
803	0.00263197459284636\\
804	0.0035671563155031\\
805	0.000495636620342733\\
806	-0.00154304193177953\\
807	-0.00340503374465391\\
808	-0.00396389744785039\\
809	-0.00230152640731909\\
810	-0.00432720235281241\\
811	-0.00780150203349507\\
812	-0.0091867574091811\\
813	-0.008082143800127\\
814	-0.00862680921629714\\
815	-0.00944787511711036\\
816	-0.00662114808469695\\
817	-0.00551992383556449\\
818	-0.00462190923162718\\
819	-0.00621900550252019\\
820	-0.00477703324003663\\
821	-0.00376444951822321\\
822	-0.00308419849962352\\
823	-0.00476149689551545\\
824	-0.0054305730252715\\
825	-0.0064567020185069\\
826	-0.00907349623814866\\
827	-0.00468108545761793\\
828	-0.000577378757764921\\
829	0.000892450768492557\\
830	0.00158537252778433\\
831	0.00246093448779584\\
832	0.00207941339254367\\
833	0.00124812669031236\\
834	0.000555102998633613\\
835	0.00305920832379431\\
836	0.0019405522844535\\
837	0.000505842703494722\\
838	0.0015673493121843\\
839	0.00106854444009894\\
840	0.000263682688638797\\
841	-4.5035809107294e-05\\
842	-0.00268373413247643\\
843	-5.47891497173248e-05\\
844	0.000245713211220852\\
845	0.000432927518898726\\
846	0.00145085489127935\\
847	0.00445247159810592\\
848	0.00322368286567824\\
849	0.0029921771335997\\
850	0.0025714134742303\\
851	0.00411106433751155\\
852	0.00393746697279545\\
853	0.00170020700710093\\
854	0.000836350569536899\\
855	0.00087671767335144\\
856	0.000185576833745517\\
857	-0.00249299935404592\\
858	-0.00122884892614012\\
859	0.00167471525252248\\
860	0.00219963509048018\\
861	0.00138971334562487\\
862	0.00352914446902503\\
863	0.00404723325580383\\
864	0.00359316966166693\\
865	0.00226071660683813\\
866	-0.00068184933709494\\
867	-0.000441217175853103\\
868	-0.00377928863897145\\
869	-0.00164531650361279\\
870	0.000724092494322668\\
871	0.005374137039112\\
872	0.00383507050815849\\
873	0.00335441571522447\\
874	0.00273942679750553\\
875	0.00547560332353199\\
876	0.00589318188811312\\
877	0.00531068364371223\\
878	0.0046014534816093\\
879	0.00653545808035317\\
880	0.00557928029417471\\
881	0.00165640332017224\\
882	0.00394929580189567\\
883	0.00322424185217313\\
884	0.00430861997980217\\
885	0.00236227931831145\\
886	0.00217056514739851\\
887	0.00231061650037493\\
888	0.00174561293166816\\
889	-0.00360201406722474\\
890	-0.00507937299737625\\
891	-0.00396126221077263\\
892	-0.00387829883501721\\
893	-0.00140490346405343\\
894	-0.00294576862403328\\
895	-0.00385034884894201\\
896	-0.00233761270018133\\
897	2.02456943016466e-05\\
898	-0.0034426787770985\\
899	-0.00304717369449503\\
900	-0.00857126556576704\\
901	-0.0115913567177585\\
902	-0.0136287907103418\\
903	-0.0117192520025436\\
904	-0.00955735969632669\\
905	-0.00898521985022955\\
906	-0.00940764963219225\\
907	-0.00898699280858181\\
908	-0.00733733223678227\\
909	-0.00784550250773035\\
910	-0.00704313742396697\\
911	-0.00865003296619781\\
912	-0.00793619039225799\\
913	-0.00726639640407729\\
914	-0.00777845959663096\\
915	-0.00707254866267473\\
916	-0.00860565226585625\\
917	-0.0101831190404427\\
918	-0.00999408775284947\\
919	-0.0125628140346392\\
920	-0.0125622474543485\\
921	-0.00921527452934847\\
922	-0.0101330533909003\\
923	-0.0116192370491414\\
924	-0.0175109414745744\\
925	-0.0175145299521371\\
926	-0.0164193566791969\\
927	-0.0123709326559735\\
928	-0.0138272915893128\\
929	-0.0156515040603244\\
930	-0.0130580063651982\\
931	-0.0102882609299413\\
932	-0.00727012136271036\\
933	-0.00659062336392284\\
934	-0.000399614229940706\\
935	-0.00212526058487055\\
936	-0.00133456447264011\\
937	0.000482809793261\\
938	-0.000468920221005206\\
939	-0.00119791429308367\\
940	-0.000660187535391459\\
941	-0.00394319707280538\\
942	-0.00399369709376224\\
943	-0.00340092208812392\\
944	-0.00387455702140838\\
945	-0.00617643586587163\\
946	-0.00498220127651816\\
947	-0.00494392283555132\\
948	-0.00473982403617179\\
949	-0.00418403753060573\\
950	-0.00446369150040115\\
951	-0.0069402509854827\\
952	-0.00535789131249856\\
953	-0.00680058947039213\\
954	-0.0053426123616824\\
955	-0.00571250733705675\\
956	-0.0046673892740697\\
957	-0.00686568936157184\\
958	-0.00465836605351479\\
959	-0.00317808892482368\\
960	-0.00216342855803828\\
961	-0.00181458061435071\\
962	-0.00174323700839998\\
963	-0.00324379809228061\\
964	-0.000423622922965328\\
965	-0.00237350589406084\\
966	-0.00263733671223525\\
967	-0.00289845512971367\\
968	-0.0034091102958563\\
969	0.000379651252967614\\
970	-0.00134348981607343\\
971	-0.00172032100551931\\
972	-0.00263892735097194\\
973	0.00238649036388343\\
974	0.00148918705311761\\
975	0.00295140232808763\\
976	0.00204469922733636\\
977	0.00231341289646423\\
978	0.00313845557929273\\
979	0.00333995321007145\\
980	0.00215515989334627\\
981	0.00409589402399321\\
982	0.00216016709443725\\
983	9.87095338773657e-05\\
984	0.000115374758082979\\
985	0.000153060076444111\\
986	-0.00178393876784831\\
987	-0.00320907661599691\\
988	-5.77415243787882e-05\\
989	0.00126694261154532\\
990	0.00238849240880233\\
991	0.00077323914503\\
992	-0.000145282977272974\\
993	0.00341199030476108\\
994	0.00309509031792099\\
995	0.00303226416802239\\
996	0.000708151201313015\\
997	-0.00122348944300773\\
998	-0.00137074716127412\\
999	-0.000658523480092075\\
1000	0.00123823314518573\\
};
\addlegendentry{$\Im \{ \mathbf{h}_1^H\}$};

\addplot [color=mycolor3,solid]
  table[row sep=crcr]{%
1	0\\
2	0\\
3	0.0213281662838053\\
4	0.0401261998317171\\
5	0.0529478903201689\\
6	0.0686648221691092\\
7	0.0819175072201788\\
8	0.094253162214864\\
9	0.105108181969535\\
10	0.10883639479208\\
11	0.122461072099215\\
12	0.13680055673024\\
13	0.154428450830606\\
14	0.166918603180204\\
15	0.180804310472859\\
16	0.195858496093235\\
17	0.212219255210694\\
18	0.221779589970765\\
19	0.23564391128932\\
20	0.249469275685336\\
21	0.257379562132406\\
22	0.269614994831141\\
23	0.282029056864293\\
24	0.29417127659862\\
25	0.303090823628487\\
26	0.314260494957574\\
27	0.327063607441954\\
28	0.340086278878869\\
29	0.351471219421114\\
30	0.362873150823619\\
31	0.376079266774263\\
32	0.383942306300266\\
33	0.391101468460156\\
34	0.403325856519835\\
35	0.413886960235054\\
36	0.419914182175806\\
37	0.429330743500451\\
38	0.440766532877785\\
39	0.456982077830472\\
40	0.467316493651772\\
41	0.475432397397882\\
42	0.485208918274671\\
43	0.49604773268192\\
44	0.505863962007698\\
45	0.51873879374181\\
46	0.531986619020984\\
47	0.539700051363828\\
48	0.547274375707371\\
49	0.559113134522248\\
50	0.569718445732459\\
51	0.578829090868223\\
52	0.584988787442348\\
53	0.59283730821955\\
54	0.600622668805883\\
55	0.610944972289478\\
56	0.626770681855952\\
57	0.638807955924624\\
58	0.644943133728274\\
59	0.652784017712011\\
60	0.660184845713747\\
61	0.674662373597182\\
62	0.680993774145319\\
63	0.686943593915973\\
64	0.695517880237583\\
65	0.702601524034078\\
66	0.70928512472916\\
67	0.719440521571973\\
68	0.727549535000125\\
69	0.736989666234687\\
70	0.747660069841141\\
71	0.754293784837324\\
72	0.763771026998924\\
73	0.769342504179764\\
74	0.776073976824394\\
75	0.781005371526166\\
76	0.787019462955756\\
77	0.79305180083891\\
78	0.802331774655453\\
79	0.815214023534205\\
80	0.817433601733533\\
81	0.83076324350035\\
82	0.840085338264545\\
83	0.846904772138477\\
84	0.854445248240286\\
85	0.86079289707528\\
86	0.868949057688254\\
87	0.880784907116353\\
88	0.886829824980008\\
89	0.889796409906543\\
90	0.896845544038376\\
91	0.900932026316489\\
92	0.908628420222333\\
93	0.916162218910357\\
94	0.920840573404143\\
95	0.924334602873741\\
96	0.931427432349162\\
97	0.938798743686235\\
98	0.945453335887642\\
99	0.957588710853988\\
100	0.960277875328086\\
101	0.967148638611281\\
102	0.971324549798755\\
103	0.97897301927246\\
104	0.986981894615559\\
105	0.988910530629917\\
106	0.991309243769802\\
107	1.00120707006107\\
108	1.00984035065399\\
109	1.01646419744127\\
110	1.02034951706728\\
111	1.0287333776644\\
112	1.03054398860433\\
113	1.03588820897818\\
114	1.03729021627156\\
115	1.04627674569509\\
116	1.05192729731907\\
117	1.05298681237105\\
118	1.05806017080742\\
119	1.06080613557942\\
120	1.0669398386183\\
121	1.07055575549752\\
122	1.07136180442956\\
123	1.07171623241171\\
124	1.07416773008851\\
125	1.07818355190595\\
126	1.08099240211568\\
127	1.08386552497379\\
128	1.0870928558048\\
129	1.09201541324087\\
130	1.09852405423479\\
131	1.10539632944654\\
132	1.11390525046062\\
133	1.11793141595778\\
134	1.12331224155271\\
135	1.12419015342978\\
136	1.13325027645348\\
137	1.13680358711702\\
138	1.14064860007268\\
139	1.14387480812966\\
140	1.15035120764414\\
141	1.15183213859238\\
142	1.15227650835221\\
143	1.15509805795254\\
144	1.16326249826368\\
145	1.16164882118083\\
146	1.16349780885951\\
147	1.16854564569768\\
148	1.16880494543234\\
149	1.17204643258302\\
150	1.1785544162328\\
151	1.18268743289211\\
152	1.18459179662588\\
153	1.1885930945805\\
154	1.190924711071\\
155	1.19332182582532\\
156	1.1960305753205\\
157	1.198536964455\\
158	1.21008526040699\\
159	1.2112790854527\\
160	1.21127818636537\\
161	1.21139835832818\\
162	1.21431156724146\\
163	1.22024068790041\\
164	1.22579110697499\\
165	1.22927608834466\\
166	1.23740006241786\\
167	1.23847321479168\\
168	1.23833053481394\\
169	1.24104759423571\\
170	1.24059672824948\\
171	1.24234214792598\\
172	1.24050712967277\\
173	1.24068666173904\\
174	1.24225409765453\\
175	1.24617704918402\\
176	1.24570944538061\\
177	1.24369778682702\\
178	1.24548042362714\\
179	1.2478329292744\\
180	1.24519323132855\\
181	1.24895120325853\\
182	1.25400278658747\\
183	1.25676092918275\\
184	1.25862636741638\\
185	1.2595463179809\\
186	1.26237080016269\\
187	1.25834443877649\\
188	1.26367529462543\\
189	1.26865715142141\\
190	1.2676569792575\\
191	1.27110544628566\\
192	1.27089435014572\\
193	1.27229835332191\\
194	1.2739681622948\\
195	1.27328717335829\\
196	1.27358001955265\\
197	1.27258349012823\\
198	1.27534864498285\\
199	1.27675144931095\\
200	1.27967036765979\\
201	1.28039985023755\\
202	1.2803477500657\\
203	1.28069480923296\\
204	1.28346036437374\\
205	1.28604912146775\\
206	1.29102963985441\\
207	1.29047482906437\\
208	1.2939365136144\\
209	1.29377887979769\\
210	1.29114677801124\\
211	1.29196592176066\\
212	1.29690950269769\\
213	1.30245287121001\\
214	1.30686979688327\\
215	1.31125495457675\\
216	1.31424456357688\\
217	1.3120191498793\\
218	1.31478619939318\\
219	1.3191783855567\\
220	1.32420030953372\\
221	1.32675199000349\\
222	1.33071815825694\\
223	1.33523006384696\\
224	1.33027737189651\\
225	1.33410614927122\\
226	1.33840721795601\\
227	1.3429621897727\\
228	1.34591207581317\\
229	1.34507342663348\\
230	1.34539614226459\\
231	1.34574819095335\\
232	1.34583896778757\\
233	1.34508388523913\\
234	1.34392946018424\\
235	1.34294300511533\\
236	1.34085630517339\\
237	1.33951771126033\\
238	1.34763242641849\\
239	1.35344447327305\\
240	1.35341239424884\\
241	1.35677890111832\\
242	1.35620343851832\\
243	1.35862074239358\\
244	1.36323201120198\\
245	1.3687827710606\\
246	1.37123338316112\\
247	1.3767663278313\\
248	1.38515763514344\\
249	1.39079117008293\\
250	1.39162564774236\\
251	1.39352457257907\\
252	1.39527390152338\\
253	1.39779627191258\\
254	1.40158221380022\\
255	1.40163492974683\\
256	1.40038011386893\\
257	1.40370392620646\\
258	1.40538822657744\\
259	1.40652901602611\\
260	1.40242486314624\\
261	1.40726394226927\\
262	1.40697892367391\\
263	1.40695617440535\\
264	1.40642426565498\\
265	1.41132149094755\\
266	1.40622402164217\\
267	1.40957440302913\\
268	1.40995890676653\\
269	1.4072915255507\\
270	1.40832907264368\\
271	1.40633184084856\\
272	1.41037817017405\\
273	1.41029787698487\\
274	1.40867813288778\\
275	1.41133565270818\\
276	1.41303924571096\\
277	1.41394250779752\\
278	1.41452489087027\\
279	1.41474773122582\\
280	1.42111669123138\\
281	1.42432773645717\\
282	1.42798563357126\\
283	1.42033225960841\\
284	1.41886760285483\\
285	1.42055132771899\\
286	1.42050335268637\\
287	1.42392355443074\\
288	1.42194096480922\\
289	1.42410734054965\\
290	1.42451064237411\\
291	1.42533618518246\\
292	1.42650186965351\\
293	1.42651646192421\\
294	1.42616276510629\\
295	1.42516128430614\\
296	1.43337645495935\\
297	1.43515209440388\\
298	1.42926942009168\\
299	1.42857384217974\\
300	1.42591897183988\\
301	1.42385417717501\\
302	1.42335984282578\\
303	1.42765734974233\\
304	1.42893297496623\\
305	1.42975707542436\\
306	1.4303623555836\\
307	1.43200393486885\\
308	1.43578409257593\\
309	1.437117324636\\
310	1.43523529329343\\
311	1.43639526105806\\
312	1.43584404895153\\
313	1.43653056605752\\
314	1.4407145522087\\
315	1.43628910707359\\
316	1.44084230826777\\
317	1.44230017539165\\
318	1.44598552583596\\
319	1.44859729691382\\
320	1.4511614349023\\
321	1.45357394738499\\
322	1.45409715066532\\
323	1.45777599745362\\
324	1.45838553257372\\
325	1.45619302828322\\
326	1.45574049381975\\
327	1.45936862075859\\
328	1.46288590405511\\
329	1.46631775879416\\
330	1.46551220764906\\
331	1.46237413258304\\
332	1.45521306634181\\
333	1.45851827979926\\
334	1.45863686610709\\
335	1.45833104341492\\
336	1.45719958286965\\
337	1.45882569410841\\
338	1.45927174667055\\
339	1.45989965089132\\
340	1.46065856509999\\
341	1.46541510038388\\
342	1.47046266392874\\
343	1.46696796171364\\
344	1.46302943066988\\
345	1.46089713161412\\
346	1.4611709305672\\
347	1.46502163801568\\
348	1.46745994939599\\
349	1.46317181079473\\
350	1.46589357003282\\
351	1.47062274089104\\
352	1.4684762143431\\
353	1.46477269869641\\
354	1.4644513754197\\
355	1.46792969996828\\
356	1.47021091349304\\
357	1.47168111104699\\
358	1.47577758790433\\
359	1.4779324755185\\
360	1.47767655152816\\
361	1.47820995973613\\
362	1.48046634954692\\
363	1.47902722081317\\
364	1.483509325868\\
365	1.48268469650981\\
366	1.4834432390797\\
367	1.48271313605761\\
368	1.48596550601624\\
369	1.48554594555347\\
370	1.48262167288131\\
371	1.48308248047193\\
372	1.47795467257361\\
373	1.47987178547942\\
374	1.47904181472648\\
375	1.47643287479226\\
376	1.47691861498932\\
377	1.47500540576284\\
378	1.47446593283522\\
379	1.47453747521528\\
380	1.4769242451301\\
381	1.47967718489922\\
382	1.47802050631667\\
383	1.47986240950596\\
384	1.47955631250513\\
385	1.47912522448979\\
386	1.48117638740935\\
387	1.47916057619678\\
388	1.48212367771031\\
389	1.48388794686876\\
390	1.48113242477241\\
391	1.47930081733608\\
392	1.48256173218922\\
393	1.48013169461098\\
394	1.48144722720388\\
395	1.48548639598761\\
396	1.48530412483389\\
397	1.48879844565512\\
398	1.49207518236464\\
399	1.49273041766392\\
400	1.49027949964351\\
401	1.49011549821153\\
402	1.4921154223467\\
403	1.49510199169617\\
404	1.49513588842763\\
405	1.50051514627682\\
406	1.50277511493912\\
407	1.49759046856688\\
408	1.49908328029368\\
409	1.49957395839384\\
410	1.5010619670045\\
411	1.49997979542871\\
412	1.49918121254216\\
413	1.5010509978079\\
414	1.50081838939092\\
415	1.5024499522417\\
416	1.49714435283242\\
417	1.49725720335222\\
418	1.49777149110436\\
419	1.4950743764278\\
420	1.49713154708511\\
421	1.49916667339228\\
422	1.49557032871252\\
423	1.49677189235097\\
424	1.49738971220835\\
425	1.50010821214579\\
426	1.50013989968851\\
427	1.50358484714949\\
428	1.50506381541458\\
429	1.50341632742469\\
430	1.50546761563466\\
431	1.50823228196603\\
432	1.50926072540985\\
433	1.50447094041179\\
434	1.50362612463707\\
435	1.50359849947669\\
436	1.50397195678802\\
437	1.50203075331945\\
438	1.50693432734094\\
439	1.50377519447093\\
440	1.50419174024036\\
441	1.50562014129051\\
442	1.50274467329686\\
443	1.49988571480448\\
444	1.49839731741415\\
445	1.49554507180169\\
446	1.4946303506475\\
447	1.49257268686581\\
448	1.49561446916143\\
449	1.49259326223224\\
450	1.49553091808302\\
451	1.49592951452975\\
452	1.49529172796436\\
453	1.49778431407296\\
454	1.49334505681054\\
455	1.49177689679238\\
456	1.49236256230272\\
457	1.48802949784269\\
458	1.48823318962034\\
459	1.48459759905867\\
460	1.48559239912804\\
461	1.48840933196747\\
462	1.48843148355213\\
463	1.48712380476248\\
464	1.48305821376142\\
465	1.48435295242448\\
466	1.48222070047161\\
467	1.48703839565118\\
468	1.48809915999854\\
469	1.48833582028502\\
470	1.48817578945642\\
471	1.48957467387775\\
472	1.48987306064742\\
473	1.49064448721322\\
474	1.49052171599727\\
475	1.49192336346978\\
476	1.4911706152076\\
477	1.4895765386995\\
478	1.49331926593196\\
479	1.49398306140494\\
480	1.49323900520544\\
481	1.49139744219278\\
482	1.49053934014165\\
483	1.49141062026712\\
484	1.49163633334332\\
485	1.49761392957068\\
486	1.49643648759468\\
487	1.49327647212698\\
488	1.49433298072152\\
489	1.49225806449051\\
490	1.49115180814067\\
491	1.48759093958066\\
492	1.4867293595505\\
493	1.48991369039698\\
494	1.48803433939343\\
495	1.48681774732971\\
496	1.48610795882626\\
497	1.48486987260364\\
498	1.47944404609319\\
499	1.47966576959961\\
500	1.48201888797288\\
501	1.48288040902494\\
502	1.48319686300853\\
503	1.47917436485\\
504	1.48265905127497\\
505	1.47931449258131\\
506	1.4743129222515\\
507	1.47400532041587\\
508	1.47401955014693\\
509	1.47134353087241\\
510	1.47755414294381\\
511	1.47816933388727\\
512	1.48016625703186\\
513	1.48158512890378\\
514	1.48116778343282\\
515	1.48104436309259\\
516	1.48275437658432\\
517	1.48388146003593\\
518	1.48815686769485\\
519	1.48761172817388\\
520	1.49087477335563\\
521	1.49255861344519\\
522	1.48850971204236\\
523	1.4928046539503\\
524	1.48979029944729\\
525	1.49068464822417\\
526	1.49255132300015\\
527	1.49951993387565\\
528	1.49860613214425\\
529	1.4987080000194\\
530	1.49714592745925\\
531	1.50078700185824\\
532	1.49668025513258\\
533	1.49723476744274\\
534	1.4972378922791\\
535	1.49436401033016\\
536	1.49539758224503\\
537	1.49776921247108\\
538	1.49602164043081\\
539	1.49993807417363\\
540	1.49770039577479\\
541	1.50147425381883\\
542	1.50310608678783\\
543	1.50153387788946\\
544	1.50242278859189\\
545	1.50274092307854\\
546	1.50205987143531\\
547	1.49803409907633\\
548	1.49958539841835\\
549	1.50239413601483\\
550	1.50492319420684\\
551	1.51134795018308\\
552	1.51228469126399\\
553	1.51296757463334\\
554	1.51494745788444\\
555	1.51671676752874\\
556	1.51446191445218\\
557	1.51430046138439\\
558	1.5157751136191\\
559	1.51967707147719\\
560	1.51870944432459\\
561	1.52176668828957\\
562	1.52158026960372\\
563	1.52444566905964\\
564	1.52223437157544\\
565	1.52317058606244\\
566	1.52402178573475\\
567	1.52332215430762\\
568	1.52299644264378\\
569	1.52182346059065\\
570	1.51735373811922\\
571	1.51920066819068\\
572	1.51713589931518\\
573	1.51769713257506\\
574	1.51417146882671\\
575	1.51406595950915\\
576	1.51646905289123\\
577	1.51707861967678\\
578	1.51466949538528\\
579	1.51557953934967\\
580	1.51452300773793\\
581	1.51738509219348\\
582	1.51929084900419\\
583	1.51673182569143\\
584	1.51430086734823\\
585	1.51336619826414\\
586	1.51233808700545\\
587	1.50716982683879\\
588	1.50722235086425\\
589	1.50775359057783\\
590	1.50502547427854\\
591	1.50621210720575\\
592	1.50602769064523\\
593	1.50898375704715\\
594	1.50918772696044\\
595	1.51176138080318\\
596	1.51004857557961\\
597	1.51251065226289\\
598	1.51265843193014\\
599	1.50921172247572\\
600	1.5118386293151\\
601	1.51143751371861\\
602	1.5079334667393\\
603	1.50805303569228\\
604	1.50787535868258\\
605	1.50716893503756\\
606	1.51195354000946\\
607	1.50790546952635\\
608	1.50737884057951\\
609	1.50859923463753\\
610	1.50853828763112\\
611	1.51507963473304\\
612	1.5137010521647\\
613	1.51084275383611\\
614	1.5122445574072\\
615	1.5107121656835\\
616	1.51355418589013\\
617	1.51602209508071\\
618	1.51328897364839\\
619	1.51074831589619\\
620	1.51320926280027\\
621	1.51368999948487\\
622	1.51097130798966\\
623	1.51130520859088\\
624	1.50500249357706\\
625	1.50207625558672\\
626	1.49633177894048\\
627	1.49656981378234\\
628	1.50017786002335\\
629	1.5038968498145\\
630	1.50330058777877\\
631	1.50548979748911\\
632	1.51117794842282\\
633	1.51344050586196\\
634	1.50920717148462\\
635	1.50910325291083\\
636	1.50292182507541\\
637	1.50596666110957\\
638	1.50496555184145\\
639	1.50802241377564\\
640	1.50541027102865\\
641	1.50281619768855\\
642	1.50890106250065\\
643	1.51163853738805\\
644	1.51319218966139\\
645	1.51217781941186\\
646	1.51459460650215\\
647	1.51822383879835\\
648	1.51688185115965\\
649	1.51642284099495\\
650	1.51293676330109\\
651	1.51090742400458\\
652	1.50731144186278\\
653	1.50933945826074\\
654	1.50935760147897\\
655	1.51041501373328\\
656	1.50965705583384\\
657	1.51117118658667\\
658	1.51545443747605\\
659	1.51436117412197\\
660	1.51155998698313\\
661	1.51041981245594\\
662	1.50772355136661\\
663	1.50825870328208\\
664	1.50624621617539\\
665	1.5064484334326\\
666	1.50148857436864\\
667	1.50462612817685\\
668	1.50372086099685\\
669	1.50321419901627\\
670	1.50526850173889\\
671	1.50499472093912\\
672	1.50739657488139\\
673	1.50626414558769\\
674	1.50895248218923\\
675	1.50656499258707\\
676	1.50764079408546\\
677	1.50601748007029\\
678	1.5025409737942\\
679	1.50009861557215\\
680	1.50035614303951\\
681	1.50089305806573\\
682	1.50185339784473\\
683	1.50143601201112\\
684	1.50085212927298\\
685	1.49839477844214\\
686	1.49424398019966\\
687	1.49530861853189\\
688	1.49628367864727\\
689	1.49712289900379\\
690	1.49826126377411\\
691	1.5008661549578\\
692	1.49859630932529\\
693	1.50085445490005\\
694	1.50421349122878\\
695	1.50462487355965\\
696	1.5044743506381\\
697	1.50255825093669\\
698	1.50378436018796\\
699	1.50333360375968\\
700	1.50407182058893\\
701	1.50508761412145\\
702	1.50659449506428\\
703	1.50971810697345\\
704	1.51077893318101\\
705	1.51267513194198\\
706	1.50704248173999\\
707	1.50793321830907\\
708	1.5012485377181\\
709	1.49531106124794\\
710	1.49215835653885\\
711	1.4938793160802\\
712	1.49420909069267\\
713	1.48785493019302\\
714	1.48243269627296\\
715	1.48677013005484\\
716	1.49111226945159\\
717	1.49335985527563\\
718	1.49347373301655\\
719	1.49363524673375\\
720	1.49338368506754\\
721	1.49160949789368\\
722	1.48797194211922\\
723	1.47861905585769\\
724	1.48118796851383\\
725	1.48149528344415\\
726	1.48120626853663\\
727	1.48473844004829\\
728	1.48253269578215\\
729	1.48342488840657\\
730	1.48305072841746\\
731	1.48572779256816\\
732	1.4882888314001\\
733	1.48790936563982\\
734	1.48643271221201\\
735	1.48335585312855\\
736	1.48032514912479\\
737	1.47864230704346\\
738	1.4788527156478\\
739	1.48373548333773\\
740	1.48008249586695\\
741	1.47731233367208\\
742	1.47322077411339\\
743	1.46943876394182\\
744	1.46916251550493\\
745	1.46635199068689\\
746	1.46930657041159\\
747	1.46966228790986\\
748	1.46754460599089\\
749	1.46869049212064\\
750	1.47208980328155\\
751	1.47158267217845\\
752	1.47438068816753\\
753	1.47735239860048\\
754	1.47626249926112\\
755	1.47666751788984\\
756	1.48162002559034\\
757	1.48036549282816\\
758	1.48360083453491\\
759	1.48415681174471\\
760	1.48785662584969\\
761	1.48583905300398\\
762	1.48531203979845\\
763	1.48669959626804\\
764	1.48578049853239\\
765	1.48616686846367\\
766	1.48568865684076\\
767	1.48588154191022\\
768	1.48647036849949\\
769	1.48476068062213\\
770	1.48590556246396\\
771	1.48446379945938\\
772	1.48715053039435\\
773	1.49056572513587\\
774	1.49136220319869\\
775	1.49290162968928\\
776	1.49328986075582\\
777	1.49370834184294\\
778	1.49355733355177\\
779	1.49653721774956\\
780	1.50130567430804\\
781	1.4974368972641\\
782	1.49963651398738\\
783	1.50127345239715\\
784	1.50048063430812\\
785	1.50119543179657\\
786	1.50018394969797\\
787	1.50132145373379\\
788	1.50317815231883\\
789	1.50022913156845\\
790	1.50360069403513\\
791	1.50407378042863\\
792	1.50684577894645\\
793	1.50327912448856\\
794	1.50104793886816\\
795	1.50042785682855\\
796	1.50045803519954\\
797	1.501625494103\\
798	1.50006658741302\\
799	1.49544765503487\\
800	1.49549863613251\\
801	1.4968562141608\\
802	1.49322022552784\\
803	1.49373790917364\\
804	1.49305931622539\\
805	1.49314916265831\\
806	1.4913980321877\\
807	1.48909528590617\\
808	1.4911243733554\\
809	1.48486014451194\\
810	1.48346327852811\\
811	1.4833836395876\\
812	1.48251910591937\\
813	1.48306583983749\\
814	1.48331522811318\\
815	1.48244890795534\\
816	1.48022266421784\\
817	1.4759907904293\\
818	1.47351264407918\\
819	1.47443374962613\\
820	1.4738179946296\\
821	1.47295965570438\\
822	1.4752081394011\\
823	1.4773742184435\\
824	1.47990357220908\\
825	1.48067611787291\\
826	1.48114570517628\\
827	1.47758926678818\\
828	1.47601522035194\\
829	1.4701327870985\\
830	1.4722127742838\\
831	1.47713686288242\\
832	1.47660202381099\\
833	1.47820751903734\\
834	1.47959867444661\\
835	1.48214788553996\\
836	1.48466708045871\\
837	1.48125260848791\\
838	1.48254277543596\\
839	1.47861539602556\\
840	1.48133017959587\\
841	1.48243732905639\\
842	1.48427372166623\\
843	1.4845724367243\\
844	1.48943737365912\\
845	1.48912426372742\\
846	1.48619515997796\\
847	1.4843163604638\\
848	1.48062838106037\\
849	1.47904356017658\\
850	1.48037425630608\\
851	1.48289097751585\\
852	1.48190404830324\\
853	1.47597085845447\\
854	1.474987592075\\
855	1.4751034425622\\
856	1.47292403609775\\
857	1.47035938626946\\
858	1.46599682511595\\
859	1.46815202832912\\
860	1.47332176526659\\
861	1.47734845294929\\
862	1.47495408617156\\
863	1.47721339140017\\
864	1.47428754306897\\
865	1.47272583584782\\
866	1.47952847659214\\
867	1.47777909130782\\
868	1.47278596978953\\
869	1.4711404246079\\
870	1.47290773814086\\
871	1.47214617734113\\
872	1.47332268787601\\
873	1.47363474160472\\
874	1.47602735093196\\
875	1.47518101961522\\
876	1.48086384232959\\
877	1.47611153206791\\
878	1.47511844838301\\
879	1.47586465090452\\
880	1.47230348152204\\
881	1.4727961638452\\
882	1.47189318422204\\
883	1.47437419661248\\
884	1.47300785516778\\
885	1.47437738586881\\
886	1.47348671389908\\
887	1.47603817372998\\
888	1.47063662161153\\
889	1.46944006433803\\
890	1.46697269521508\\
891	1.4677327660402\\
892	1.466668552653\\
893	1.46453361992764\\
894	1.46228271633345\\
895	1.46417909309357\\
896	1.4636705934244\\
897	1.46256479225804\\
898	1.46283057079172\\
899	1.46359712556178\\
900	1.46539121369298\\
901	1.46592950590265\\
902	1.46821355735916\\
903	1.4671899897365\\
904	1.46445296518196\\
905	1.46510634495729\\
906	1.46780430655767\\
907	1.46555953946903\\
908	1.46186026968828\\
909	1.45832006617225\\
910	1.45657600905409\\
911	1.45299273534589\\
912	1.45138983523058\\
913	1.44936329394573\\
914	1.45039979909711\\
915	1.44792997451383\\
916	1.44532169650131\\
917	1.44281297208277\\
918	1.44577418303825\\
919	1.44331552846658\\
920	1.4449142144248\\
921	1.44687020097023\\
922	1.4493906750396\\
923	1.45043476780563\\
924	1.45069770736623\\
925	1.45166673597514\\
926	1.45733915527108\\
927	1.46206280631996\\
928	1.46269963458647\\
929	1.46584016740457\\
930	1.46743003839844\\
931	1.47144323259083\\
932	1.47200196296984\\
933	1.47378681326383\\
934	1.4703657346414\\
935	1.47247416281503\\
936	1.47208960041997\\
937	1.47244688929431\\
938	1.47311009639428\\
939	1.47972436342924\\
940	1.48142008543367\\
941	1.47858418142664\\
942	1.48035503228056\\
943	1.48021258980743\\
944	1.48174392493924\\
945	1.48440528609831\\
946	1.48372416959442\\
947	1.48365497035704\\
948	1.48392208145087\\
949	1.48378190751372\\
950	1.4860728886847\\
951	1.48698982072013\\
952	1.48868984149279\\
953	1.48956788470079\\
954	1.48771217133313\\
955	1.48913545121056\\
956	1.49139347281392\\
957	1.49028655045455\\
958	1.49349396115339\\
959	1.49167551299773\\
960	1.48941392431616\\
961	1.49190382000313\\
962	1.4889813442216\\
963	1.49204841621745\\
964	1.49600357583271\\
965	1.49363280152095\\
966	1.49593628964865\\
967	1.4968840471052\\
968	1.4951843020323\\
969	1.49365606132819\\
970	1.4923721058021\\
971	1.49848864963206\\
972	1.49568049072298\\
973	1.49320612579924\\
974	1.49179935397114\\
975	1.4952528031859\\
976	1.49622555626372\\
977	1.50051821833288\\
978	1.50265889378324\\
979	1.50225403476557\\
980	1.49890712114095\\
981	1.49377717407737\\
982	1.49277314759306\\
983	1.49658292470963\\
984	1.49524380007413\\
985	1.4951778457961\\
986	1.49127943907227\\
987	1.48835126431823\\
988	1.48817144947036\\
989	1.48731975776764\\
990	1.48503798807852\\
991	1.48873657716106\\
992	1.48781629699314\\
993	1.48881186276363\\
994	1.4893372287263\\
995	1.49222123330751\\
996	1.49355677737203\\
997	1.4912222126111\\
998	1.49386494577566\\
999	1.49406416189569\\
1000	1.50065492815032\\
};
\addlegendentry{$\Re \{ \mathbf{h}_2^H\} $};

\addplot [color=mycolor4,solid]
  table[row sep=crcr]{%
1	0\\
2	0\\
3	0.0102744679001882\\
4	0.0209618704768064\\
5	0.0298784018711398\\
6	0.0404960968239174\\
7	0.0489693020390053\\
8	0.0571130190350501\\
9	0.0693166330815998\\
10	0.0794720509377423\\
11	0.0899491650363735\\
12	0.0969068537436958\\
13	0.106099956695141\\
14	0.117270236142598\\
15	0.122778365322976\\
16	0.134158542313759\\
17	0.143223722659138\\
18	0.149560231712967\\
19	0.157066325774536\\
20	0.165607199726398\\
21	0.174835174060615\\
22	0.182358321075516\\
23	0.193824164796593\\
24	0.198289813963819\\
25	0.20505026025545\\
26	0.21375528807271\\
27	0.224011864723855\\
28	0.232107725680359\\
29	0.245519424395837\\
30	0.248699875326455\\
31	0.256754238890476\\
32	0.262072015302391\\
33	0.268002108739473\\
34	0.273716639998243\\
35	0.288475499558949\\
36	0.297428688530371\\
37	0.305973257192571\\
38	0.31583902169202\\
39	0.322076487683755\\
40	0.327906340430087\\
41	0.337848296014015\\
42	0.343420566415172\\
43	0.34912998679706\\
44	0.354697396012565\\
45	0.357103084650346\\
46	0.36388133290261\\
47	0.372483640398044\\
48	0.374503072556286\\
49	0.375053992006352\\
50	0.380152658886414\\
51	0.382889172865689\\
52	0.387004954443919\\
53	0.395229586285764\\
54	0.398020456299121\\
55	0.404616036401944\\
56	0.407451862183359\\
57	0.412485379892808\\
58	0.420868349648938\\
59	0.426500035088256\\
60	0.431702565927664\\
61	0.432138256096466\\
62	0.435810665556959\\
63	0.438573814934135\\
64	0.449678900461224\\
65	0.455634149878337\\
66	0.462553465560787\\
67	0.470459939049513\\
68	0.473154198222496\\
69	0.479847587678458\\
70	0.487760451067718\\
71	0.488510117907151\\
72	0.49338537468792\\
73	0.501211564542104\\
74	0.503812438696167\\
75	0.511418887936598\\
76	0.512884848165436\\
77	0.513172995777895\\
78	0.517377400660825\\
79	0.523339567137686\\
80	0.527002160613108\\
81	0.535698636370333\\
82	0.539495793584345\\
83	0.545351037426412\\
84	0.548482217562917\\
85	0.553460005920949\\
86	0.559432121370974\\
87	0.562680837951463\\
88	0.568444070417122\\
89	0.573092468404491\\
90	0.576101712141122\\
91	0.579884088585674\\
92	0.583863387577019\\
93	0.588838417131528\\
94	0.594203001060617\\
95	0.596005857703127\\
96	0.595823607645397\\
97	0.598607258803539\\
98	0.607478953532033\\
99	0.608299773163912\\
100	0.610810333419998\\
101	0.611301193595127\\
102	0.613813454144965\\
103	0.619248146519785\\
104	0.621667738883249\\
105	0.623654824539701\\
106	0.628641041115579\\
107	0.634289220785643\\
108	0.638313340985788\\
109	0.646760550133751\\
110	0.652736169662443\\
111	0.658019328782097\\
112	0.665265093781945\\
113	0.664418672844892\\
114	0.669023976470792\\
115	0.670951347056348\\
116	0.674855912872964\\
117	0.679352783977585\\
118	0.678181741334575\\
119	0.683494860343079\\
120	0.688439407924402\\
121	0.692696752113751\\
122	0.694501618111603\\
123	0.698099574265914\\
124	0.70318153236059\\
125	0.706795538604306\\
126	0.709605944846303\\
127	0.712731590408955\\
128	0.715391198840636\\
129	0.717185973505651\\
130	0.722870548080711\\
131	0.727683962480821\\
132	0.731233167760256\\
133	0.729683533751432\\
134	0.735102323149829\\
135	0.738331156190473\\
136	0.741865738075858\\
137	0.742937700194056\\
138	0.747315978963292\\
139	0.751217535212207\\
140	0.753107251514617\\
141	0.757500110580969\\
142	0.762462062369154\\
143	0.767526337111256\\
144	0.76901188470514\\
145	0.778815062330201\\
146	0.782286274609203\\
147	0.783280924931669\\
148	0.785846217265591\\
149	0.791233256921106\\
150	0.79368443607314\\
151	0.795907580395422\\
152	0.798991877013056\\
153	0.80250104196165\\
154	0.801559294442584\\
155	0.802896787786086\\
156	0.802555063855604\\
157	0.803157194447689\\
158	0.805975824086853\\
159	0.809195962332661\\
160	0.807200239638657\\
161	0.815049715893007\\
162	0.817854241306353\\
163	0.82216573327135\\
164	0.823079911052601\\
165	0.825314022095176\\
166	0.827707808419971\\
167	0.830701078955179\\
168	0.830653562691346\\
169	0.830985640520537\\
170	0.837247247086111\\
171	0.838147998990997\\
172	0.842595793027305\\
173	0.843855999884061\\
174	0.846340928481585\\
175	0.847279625364521\\
176	0.852137883812404\\
177	0.857853883257676\\
178	0.861069462813217\\
179	0.862529124082524\\
180	0.86364108952359\\
181	0.86582754474547\\
182	0.86405532987819\\
183	0.866555220211405\\
184	0.867114702966895\\
185	0.866944164184253\\
186	0.866063977576916\\
187	0.86697364191378\\
188	0.869070213263038\\
189	0.871335806180759\\
190	0.873265896799006\\
191	0.876787335320793\\
192	0.877628408293373\\
193	0.879384159861729\\
194	0.883323886600162\\
195	0.882764048255692\\
196	0.880633492675001\\
197	0.882875183055626\\
198	0.886309437880393\\
199	0.889219800511751\\
200	0.88581297719764\\
201	0.884165178710544\\
202	0.892577287913853\\
203	0.895583097378573\\
204	0.891810325005247\\
205	0.895149180764509\\
206	0.895615518823748\\
207	0.895117301320936\\
208	0.89594149916538\\
209	0.89756181164723\\
210	0.900970997062826\\
211	0.900837119438845\\
212	0.903877950285511\\
213	0.903535277088815\\
214	0.905333393116288\\
215	0.90365642605527\\
216	0.90277737382133\\
217	0.903466106301292\\
218	0.905327771413078\\
219	0.908215509227514\\
220	0.910568799360289\\
221	0.910345475417525\\
222	0.911329880296385\\
223	0.913841168113059\\
224	0.912618406386606\\
225	0.913971398606189\\
226	0.913679292449877\\
227	0.91448532519375\\
228	0.918889073761149\\
229	0.919712405356822\\
230	0.916873157243356\\
231	0.917634351609761\\
232	0.918520185751282\\
233	0.919725296547798\\
234	0.923839269742848\\
235	0.924260162516029\\
236	0.926114726361361\\
237	0.922120135113268\\
238	0.925905897070384\\
239	0.923408348446835\\
240	0.923156151127097\\
241	0.927589559305522\\
242	0.926439701062387\\
243	0.927760838392547\\
244	0.92791526916502\\
245	0.931323426406761\\
246	0.930730897425172\\
247	0.930443890014835\\
248	0.93493886584065\\
249	0.935462168462278\\
250	0.932391754850457\\
251	0.935761350097297\\
252	0.937116159108168\\
253	0.936288951817237\\
254	0.93647713884978\\
255	0.938148213571127\\
256	0.936605639289606\\
257	0.93659411404794\\
258	0.93965001345923\\
259	0.940783967701081\\
260	0.941189781116824\\
261	0.94386239607128\\
262	0.946077822317467\\
263	0.947783698578016\\
264	0.945143663033556\\
265	0.937644152554644\\
266	0.939541046604707\\
267	0.941699501014693\\
268	0.945348788867183\\
269	0.94399855430578\\
270	0.946791393882411\\
271	0.949312078270326\\
272	0.95522879645122\\
273	0.959838545390813\\
274	0.960799148510305\\
275	0.958238145393298\\
276	0.959079829058423\\
277	0.953301399389331\\
278	0.954502636051112\\
279	0.95078907763955\\
280	0.953435100073985\\
281	0.955008242901102\\
282	0.961253210078263\\
283	0.96116087017429\\
284	0.96567512132067\\
285	0.967638185521507\\
286	0.971130753048061\\
287	0.965210977294424\\
288	0.965022217605696\\
289	0.963945493078237\\
290	0.965587235388907\\
291	0.965495536807767\\
292	0.968188997647252\\
293	0.96910979763592\\
294	0.968045855787362\\
295	0.971478512974293\\
296	0.971155418925059\\
297	0.969582411122263\\
298	0.969710806749578\\
299	0.971365835079807\\
300	0.974753531380518\\
301	0.974171126658937\\
302	0.971379293487214\\
303	0.9690431299785\\
304	0.9761102053373\\
305	0.97696052227003\\
306	0.97479853529678\\
307	0.978709180378384\\
308	0.974366875007287\\
309	0.977786322219247\\
310	0.976964978882505\\
311	0.977778198729157\\
312	0.978277803228685\\
313	0.982901005522364\\
314	0.983097449490126\\
315	0.987438008674919\\
316	0.990683155372587\\
317	0.995702043840023\\
318	0.993919910453195\\
319	0.990328950483759\\
320	0.992386580495781\\
321	0.991348734242075\\
322	0.989858346304969\\
323	0.982925998763807\\
324	0.985395172390484\\
325	0.989448577151592\\
326	0.990098952381615\\
327	0.989751089023063\\
328	0.990026243332406\\
329	0.989871439645108\\
330	0.990660062578259\\
331	0.992604934116339\\
332	0.989061063591521\\
333	0.993909725060832\\
334	0.993811217693844\\
335	0.995160042499133\\
336	0.994987905121612\\
337	0.992681562578116\\
338	0.993966238172671\\
339	0.993347746296258\\
340	0.995900319594961\\
341	0.995298749057453\\
342	0.996183386402775\\
343	0.99476708553375\\
344	0.989755341187636\\
345	0.990337789834786\\
346	0.987970623511535\\
347	0.990674268393694\\
348	0.989260137416944\\
349	0.991331047192952\\
350	0.992519536520184\\
351	0.994133354602002\\
352	0.990568924928295\\
353	0.987779024278303\\
354	0.990347480911263\\
355	0.993151111379072\\
356	0.989720511597432\\
357	0.993064019483934\\
358	0.990031868901774\\
359	0.989763256801052\\
360	0.993464892225224\\
361	0.990577806904125\\
362	0.990086072709525\\
363	0.988081507687431\\
364	0.988540678570902\\
365	0.987031719113633\\
366	0.989854278397324\\
367	0.987845220775289\\
368	0.986872452895746\\
369	0.989297451603159\\
370	0.989566526858617\\
371	0.987915565558973\\
372	0.983800915273818\\
373	0.984722738009802\\
374	0.987862965945598\\
375	0.989234540769299\\
376	0.988483724921886\\
377	0.989875973064309\\
378	0.988641732763539\\
379	0.987274520251176\\
380	0.98493481795672\\
381	0.979388191761287\\
382	0.982717391715392\\
383	0.980180383120094\\
384	0.984596876725497\\
385	0.985558598840609\\
386	0.988811057778237\\
387	0.990521199645002\\
388	0.99056283251098\\
389	0.993183197424258\\
390	0.997043226610294\\
391	0.995103640623464\\
392	0.994619007582446\\
393	0.997132055756035\\
394	0.996960399424123\\
395	0.997003487569512\\
396	1.00195059654661\\
397	1.00469029257258\\
398	1.00721184629228\\
399	1.00966098981305\\
400	1.00820315392033\\
401	1.00624617419724\\
402	1.0054326500414\\
403	1.00479100720526\\
404	1.00751641727932\\
405	1.00878216907144\\
406	1.01010219255384\\
407	1.01153913525639\\
408	1.01267271928374\\
409	1.01316491264235\\
410	1.01165135389971\\
411	1.01053250015051\\
412	1.01201700713408\\
413	1.01145619470757\\
414	1.0089178535981\\
415	1.00594101665898\\
416	1.00850057931456\\
417	1.01153230078319\\
418	1.01253048283444\\
419	1.0118378242994\\
420	1.01287588820469\\
421	1.0115553226477\\
422	1.01718995683106\\
423	1.01678331075747\\
424	1.0136490497431\\
425	1.01334302452511\\
426	1.01392847739116\\
427	1.01487319185633\\
428	1.01225766785937\\
429	1.01406057172912\\
430	1.01668216266754\\
431	1.01419930474188\\
432	1.01699996431317\\
433	1.01576566152989\\
434	1.01771640361028\\
435	1.02074406541113\\
436	1.01437169130399\\
437	1.01111890494787\\
438	1.01378032608305\\
439	1.01234849394887\\
440	1.01115712598809\\
441	1.01270239755919\\
442	1.01108242128681\\
443	1.0101921139252\\
444	1.00947535068934\\
445	1.00866840505981\\
446	1.01175496941792\\
447	1.01319659750241\\
448	1.01603597441548\\
449	1.01742993663835\\
450	1.02021756549358\\
451	1.02284394814724\\
452	1.02219054625998\\
453	1.02166296682464\\
454	1.02217796919907\\
455	1.01696220057532\\
456	1.01716861960377\\
457	1.01879244736967\\
458	1.01686934586821\\
459	1.01883386380855\\
460	1.0209987475298\\
461	1.01976424092828\\
462	1.0173864600611\\
463	1.01811983034245\\
464	1.01683242453698\\
465	1.01519058152317\\
466	1.01709511593268\\
467	1.01407589057684\\
468	1.01440169027247\\
469	1.0127821540545\\
470	1.01473547069044\\
471	1.01160948168619\\
472	1.01081216236828\\
473	1.0120287922868\\
474	1.01168739469924\\
475	1.00995488856948\\
476	1.00562652016938\\
477	1.00639885284159\\
478	1.00274348461851\\
479	1.00160806870326\\
480	1.00158294849534\\
481	0.999987935451312\\
482	1.00353854589677\\
483	1.00634185223353\\
484	1.00737549758523\\
485	1.01181523240588\\
486	1.01467171240193\\
487	1.01248851855661\\
488	1.01341400433612\\
489	1.01367418931366\\
490	1.01560470686797\\
491	1.01904458080901\\
492	1.02139691491922\\
493	1.02169401047061\\
494	1.02063364683207\\
495	1.01789167531069\\
496	1.01535880179834\\
497	1.01848617576796\\
498	1.01481177647839\\
499	1.01429377601322\\
500	1.01496953795639\\
501	1.01371595979515\\
502	1.01523062505088\\
503	1.02197540474618\\
504	1.02220304657802\\
505	1.02204805801678\\
506	1.0251221621344\\
507	1.02765507076906\\
508	1.02991720454232\\
509	1.02545461418324\\
510	1.02411825963606\\
511	1.02757118285187\\
512	1.03067452673437\\
513	1.03076964078759\\
514	1.03511894488815\\
515	1.03571644905962\\
516	1.03768896803652\\
517	1.03534319998022\\
518	1.0385453938553\\
519	1.03927551880789\\
520	1.04073326091727\\
521	1.0368088979264\\
522	1.0347366436304\\
523	1.03542535965929\\
524	1.03198563767845\\
525	1.03776525156903\\
526	1.03712161901583\\
527	1.02633528411952\\
528	1.02898754221738\\
529	1.02866653636466\\
530	1.02443650458356\\
531	1.02652462288613\\
532	1.02630115274091\\
533	1.02725973500557\\
534	1.02713538610786\\
535	1.02242801338583\\
536	1.01967579793811\\
537	1.01963726134956\\
538	1.02099454110155\\
539	1.01918866441593\\
540	1.02137708921411\\
541	1.01992055614289\\
542	1.0174695789067\\
543	1.02022637392452\\
544	1.0213228159724\\
545	1.0205843153181\\
546	1.01699032737359\\
547	1.01405545403016\\
548	1.01560819952056\\
549	1.01724423907147\\
550	1.0172510063739\\
551	1.01826491378174\\
552	1.01987620907487\\
553	1.02172156425225\\
554	1.02212190272716\\
555	1.0174778237433\\
556	1.01483396599553\\
557	1.01448231614109\\
558	1.012512048803\\
559	1.01237632866916\\
560	1.0103468688152\\
561	1.01052710554493\\
562	1.01170739702128\\
563	1.00961647831411\\
564	1.0098854548034\\
565	1.00828770266712\\
566	1.00953374497746\\
567	1.01066767396068\\
568	1.00921587150958\\
569	1.01176334860232\\
570	1.01613415292026\\
571	1.01427361330217\\
572	1.0112299463603\\
573	1.01307976669308\\
574	1.01282602022268\\
575	1.01201837554305\\
576	1.01356493839907\\
577	1.01189626748312\\
578	1.01386043154934\\
579	1.01195812257563\\
580	1.00864937903259\\
581	1.00791962759669\\
582	1.00221281757341\\
583	1.00692279848085\\
584	1.00372716466583\\
585	0.999118839659272\\
586	1.00276178136958\\
587	1.00979423914845\\
588	1.01027459326176\\
589	1.01058696157568\\
590	1.00812261740209\\
591	1.00309970357247\\
592	1.00222073196317\\
593	0.999633081525428\\
594	0.996382250630549\\
595	0.994496876719416\\
596	0.994817160649037\\
597	0.995138797680275\\
598	0.993481843406553\\
599	0.993375303384205\\
600	0.991555912357992\\
601	0.991276625622442\\
602	0.992961203121889\\
603	0.991144104137568\\
604	0.992339324000068\\
605	0.99473777876858\\
606	0.994717002315698\\
607	0.998274560340963\\
608	0.999054589055695\\
609	1.00010539117531\\
610	1.00082152644919\\
611	1.00088443463206\\
612	0.999979695268254\\
613	1.00334943862027\\
614	1.00764750057054\\
615	1.01036934965687\\
616	1.01171162687692\\
617	1.01325177227819\\
618	1.01114843023715\\
619	1.01462622735411\\
620	1.01084032103516\\
621	1.00806251014489\\
622	1.00805445886334\\
623	1.0074358466439\\
624	1.00701896572006\\
625	1.00948935162861\\
626	1.00801315360115\\
627	1.00645083155917\\
628	1.00353550667874\\
629	1.00316311244034\\
630	0.999896092409684\\
631	0.999437539670589\\
632	1.00005771797955\\
633	1.00131895034895\\
634	1.00031244074315\\
635	1.00096766206738\\
636	0.997078785429233\\
637	0.995570688703324\\
638	0.992174027171584\\
639	0.991819084533669\\
640	0.986956767258284\\
641	0.985174054137136\\
642	0.989522575138823\\
643	0.987346031528976\\
644	0.985812034190775\\
645	0.98722720470097\\
646	0.994537539603775\\
647	0.990299749138083\\
648	0.990448090749297\\
649	0.993109674095403\\
650	0.993819252041582\\
651	0.994480428229138\\
652	0.995052438826514\\
653	0.996345509796807\\
654	0.997254053764511\\
655	0.994883229391524\\
656	0.996844155233585\\
657	0.997216290231045\\
658	0.996384450581572\\
659	0.995044605121033\\
660	0.996503917959443\\
661	0.998138131692779\\
662	0.999099059307239\\
663	1.00173497761497\\
664	1.00189131574116\\
665	1.00004186348414\\
666	0.996234177444464\\
667	0.99656283513792\\
668	0.996601250189681\\
669	0.995163037702398\\
670	0.997774418223588\\
671	0.997987291749153\\
672	0.999085354326755\\
673	1.00535296818977\\
674	1.00466851819876\\
675	1.0060068519418\\
676	1.00794183636506\\
677	1.00393486898947\\
678	1.00736169204492\\
679	1.00919768139838\\
680	1.00490962930461\\
681	1.00107238260231\\
682	1.00253828415779\\
683	1.00339552467552\\
684	0.998142489316221\\
685	1.00023011085733\\
686	1.00381535484749\\
687	0.99888089096976\\
688	1.00141415754123\\
689	1.00385350525145\\
690	1.00505863967462\\
691	1.00629133495418\\
692	1.00579172517406\\
693	1.00998390676803\\
694	1.0061383484891\\
695	1.00211520995164\\
696	1.00452157436399\\
697	1.00581496364479\\
698	1.00011860852106\\
699	1.00042831745206\\
700	1.00111368242895\\
701	1.0012126798223\\
702	1.00138277860188\\
703	1.00116094556832\\
704	1.00399379153691\\
705	1.00369197779424\\
706	1.00108106511739\\
707	1.0036656173483\\
708	1.0059733908994\\
709	1.01034206405429\\
710	1.01070308054273\\
711	1.00850138012522\\
712	1.00610487501238\\
713	1.00851735586206\\
714	1.00737203499533\\
715	1.00080599410429\\
716	0.99902685837097\\
717	1.0009167065019\\
718	0.997229164140562\\
719	0.999511636251031\\
720	0.994513677037054\\
721	0.99221311950007\\
722	0.992198107874381\\
723	0.988493542491873\\
724	0.990162669386946\\
725	0.992113640485675\\
726	0.99280089230418\\
727	0.993339612235078\\
728	0.988953809982029\\
729	0.988063603269051\\
730	0.99121452373374\\
731	0.991805467177042\\
732	0.989446603532655\\
733	0.987579363317545\\
734	0.98699132728908\\
735	0.985160584964885\\
736	0.984692287779304\\
737	0.987500861077203\\
738	0.988327974293108\\
739	0.988109880216385\\
740	0.987814382691891\\
741	0.985391709157291\\
742	0.984385727250301\\
743	0.985671234603401\\
744	0.985297771315162\\
745	0.988072884860527\\
746	0.988493448900395\\
747	0.985861162533302\\
748	0.984232279798927\\
749	0.987305735396006\\
750	0.986734071993566\\
751	0.987235531268805\\
752	0.992739305869904\\
753	0.989884707834829\\
754	0.99227855913408\\
755	0.992592038947027\\
756	0.992335927042146\\
757	0.989543971455307\\
758	0.989199663554091\\
759	0.990816136372146\\
760	0.98523073278636\\
761	0.979303532644125\\
762	0.98208341535889\\
763	0.984080929886205\\
764	0.987527814130466\\
765	0.989672301224291\\
766	0.987765630565122\\
767	0.990682745675727\\
768	0.990239142442588\\
769	0.993698904405481\\
770	0.99416007358284\\
771	0.996354050617647\\
772	0.993941905590174\\
773	0.992701741167494\\
774	0.994547823250257\\
775	0.993478465183211\\
776	0.99858436822921\\
777	1.00048820388251\\
778	1.00231352878899\\
779	1.00405100229596\\
780	1.00462815273545\\
781	1.00611621860871\\
782	1.00842320341465\\
783	1.00973683312985\\
784	1.00851132005123\\
785	1.00914081194099\\
786	1.0089954058653\\
787	1.00748880140838\\
788	1.00411169681872\\
789	1.00677912813702\\
790	1.00142947253441\\
791	0.997471109487733\\
792	0.995269427742606\\
793	0.9950460062065\\
794	0.995998961928175\\
795	0.997432558802625\\
796	0.998091336922556\\
797	0.995676840593782\\
798	0.991018541928254\\
799	0.990960103423196\\
800	0.990473942763637\\
801	0.990509287211598\\
802	0.993349627392674\\
803	0.994117608976316\\
804	0.993244169253224\\
805	0.994173353331403\\
806	0.994263945746538\\
807	0.995150571678202\\
808	0.994856670638957\\
809	0.997816976319217\\
810	0.998676963096232\\
811	1.00037583299677\\
812	0.996390110679639\\
813	0.998895045262396\\
814	0.997299022619313\\
815	0.99302885089927\\
816	0.995753735781288\\
817	0.996332533737952\\
818	0.998519223044472\\
819	0.996165791779697\\
820	0.996809160853339\\
821	0.99732207414503\\
822	0.994646354633016\\
823	0.993019995905473\\
824	0.995381710548198\\
825	0.994682203756706\\
826	0.996843439269644\\
827	0.993732469359708\\
828	0.995050791546617\\
829	1.00044151786878\\
830	1.00238927133469\\
831	1.00461264525501\\
832	1.00412802600834\\
833	1.0019744931199\\
834	0.999250687028285\\
835	0.999182002203359\\
836	0.998218892430146\\
837	1.00340606088893\\
838	1.00299875419046\\
839	1.00585209102479\\
840	1.00341090468502\\
841	1.00566271121773\\
842	1.00761393923504\\
843	1.00551841451153\\
844	1.00301814884219\\
845	1.00764348945871\\
846	1.00732153133131\\
847	1.00425983983438\\
848	1.00229361089357\\
849	1.00253507027389\\
850	1.00325677315031\\
851	1.00387688471962\\
852	1.00988542887898\\
853	1.0116873846832\\
854	1.01297423761513\\
855	1.01148004504267\\
856	1.01263895526619\\
857	1.01427439761673\\
858	1.01624677107518\\
859	1.01660506097411\\
860	1.01895588903054\\
861	1.0222718562888\\
862	1.01781165165774\\
863	1.01607370917672\\
864	1.01883340342983\\
865	1.01938070178621\\
866	1.01526819942613\\
867	1.01579787228936\\
868	1.01548799875421\\
869	1.01557963039311\\
870	1.02293638478004\\
871	1.02173987610753\\
872	1.02124372718327\\
873	1.01972513682533\\
874	1.02005971858609\\
875	1.01997745252133\\
876	1.01905549525183\\
877	1.01914364288893\\
878	1.02073998339905\\
879	1.02055822774083\\
880	1.01990609167938\\
881	1.0193054831337\\
882	1.0189377572611\\
883	1.01566016545471\\
884	1.01370502535281\\
885	1.01177294896315\\
886	1.01286784606617\\
887	1.01062355560266\\
888	1.01391613488318\\
889	1.01706288700158\\
890	1.01679372204235\\
891	1.0127324154526\\
892	1.01224398664219\\
893	1.01483055102739\\
894	1.01221809513316\\
895	1.013762907844\\
896	1.01833427026887\\
897	1.01779110329166\\
898	1.01957089539407\\
899	1.02067365815131\\
900	1.02154281517432\\
901	1.01943285888818\\
902	1.02052469001216\\
903	1.02182829603432\\
904	1.02647313577754\\
905	1.02497486776171\\
906	1.02035972929593\\
907	1.02174295697376\\
908	1.01835094919557\\
909	1.02190985494787\\
910	1.02122200426814\\
911	1.02489989521364\\
912	1.02334863661968\\
913	1.02041500624263\\
914	1.020390742401\\
915	1.01604170798262\\
916	1.01630937714091\\
917	1.01728543190995\\
918	1.01646103970854\\
919	1.01497865568929\\
920	1.01335836893707\\
921	1.01688072113828\\
922	1.01738549900558\\
923	1.0199798461718\\
924	1.0163290680059\\
925	1.0207846359825\\
926	1.0171006198647\\
927	1.01170966799227\\
928	1.00997109635527\\
929	1.00818113090274\\
930	1.00708202277514\\
931	1.00929114437654\\
932	1.0098496621598\\
933	1.01126734870416\\
934	1.01280307107883\\
935	1.01571479865706\\
936	1.0176137136691\\
937	1.01695193796551\\
938	1.01189766412191\\
939	1.0107741355372\\
940	1.00864723549676\\
941	1.00724384921252\\
942	1.01007243765402\\
943	1.00914173652515\\
944	1.00821703403725\\
945	1.00546029290854\\
946	1.00634143688595\\
947	1.00596777523198\\
948	1.00722722546639\\
949	1.00319359819443\\
950	1.00286603156947\\
951	1.00152127125829\\
952	1.00349395073834\\
953	1.00321768437755\\
954	1.00387368962177\\
955	1.00369484387656\\
956	1.00343486622151\\
957	1.00196520651502\\
958	1.00611542237395\\
959	1.00363708655609\\
960	1.00305234272718\\
961	1.00947875343577\\
962	1.00904550755783\\
963	1.01038699491836\\
964	1.01124265824081\\
965	1.01127096029226\\
966	1.01569253996689\\
967	1.01669490389965\\
968	1.01635891678166\\
969	1.01690332967339\\
970	1.01806935418701\\
971	1.01954862940946\\
972	1.02022528122796\\
973	1.02411822170607\\
974	1.02429277353776\\
975	1.025726214998\\
976	1.02592118671055\\
977	1.02935337428998\\
978	1.02993461710672\\
979	1.03598258038066\\
980	1.03317656585208\\
981	1.03551856310546\\
982	1.0332531657906\\
983	1.03204652226327\\
984	1.02818536839434\\
985	1.02852370875734\\
986	1.02922806864768\\
987	1.03333519366159\\
988	1.03439892184062\\
989	1.03064071112136\\
990	1.0272155223203\\
991	1.02357792281135\\
992	1.02516295441226\\
993	1.02482028068744\\
994	1.02495861045959\\
995	1.02135720375994\\
996	1.02128023482889\\
997	1.01760181245516\\
998	1.01583106085115\\
999	1.01472083561291\\
1000	1.01550784519083\\
};
\addlegendentry{$\Im \{ \mathbf{h}_2^H \}$};

\end{axis}
\end{tikzpicture}%}
		\caption{\textit{Esimated Weights (both real and imaginary components)}}
		\label{fig:4_1_a_clms_weights}
	\end{subfigure}
	\caption{\textit{Learning Curve and Estimated Weights from the CLMS fitler}}
\end{figure}

Figure \ref{fig:4_1_a_aclms} shows the same plots but this time from the ACLMS filter. We can immediately see that the learning curve in figure \ref{fig:4_1_a_aclms_err} is significantly better than those seen in figure \ref{fig:4_1_a_clms_err} with the CLMS filter. This is also evident in the weights in figure \ref{fig:4_1_a_aclms_weights}. Whereas with the CLMS they appeared to not have a smooth steady state (which is backed up by the learning curve), here appear to be smooht. We also observe that we can see 4 weights (the real and imaginary components of both $ \mathbf{g}$ and $ \mathbf{h}$), as opposed to the two seen in the CLMS. We can observe that the values associated with $ b_2 $ are represented by $ \mathbf{g} $.

\begin{figure}[h]
	\centering
	\begin{subfigure}[b]{0.49\textwidth}
		\resizebox{\textwidth}{!}{% This file was created by matlab2tikz.
% Minimal pgfplots version: 1.3
%
%The latest updates can be retrieved from
%  http://www.mathworks.com/matlabcentral/fileexchange/22022-matlab2tikz
%where you can also make suggestions and rate matlab2tikz.
%
\definecolor{mycolor1}{rgb}{0.00000,0.44700,0.74100}%
%
\begin{tikzpicture}

\begin{axis}[%
width=4in,
height=1.8in,
at={(1.011111in,0.641667in)},
scale only axis,
unbounded coords=jump,
xmin=0,
xmax=1000,
tick align=outside,
xlabel={Iteration},
xmajorgrids,
ymin=-80,
ymax=20,
ylabel={Learning Curve (dB)},
ymajorgrids,
title style={font=\bfseries},
title={ACLMS Error for WLMA(1) process},
axis x line*=bottom,
axis y line*=left
]
\addplot [color=mycolor1,solid,forget plot]
  table[row sep=crcr]{%
1	-inf\\
2	-inf\\
3	10.9764630768605\\
4	9.97128410042059\\
5	10.5129010256136\\
6	9.29015987489539\\
7	9.48606514168232\\
8	9.52220841507323\\
9	8.49929180800309\\
10	9.71404329379409\\
11	9.56071689364072\\
12	9.78846941158669\\
13	9.69677742013191\\
14	9.25238930354185\\
15	9.91879463383668\\
16	9.59826774987238\\
17	8.40475249274762\\
18	9.01957790820317\\
19	9.22620786202349\\
20	7.7021372398023\\
21	8.76875931862642\\
22	8.90169777992309\\
23	8.20746474562454\\
24	8.53854714014746\\
25	8.21852881221329\\
26	8.30551625405255\\
27	8.64388104599975\\
28	9.15638708417752\\
29	8.06451187906804\\
30	7.89411973527138\\
31	7.42198891238047\\
32	7.10899777793976\\
33	7.84092226359127\\
34	8.28460366844863\\
35	7.42755970743691\\
36	7.57339432888759\\
37	7.95414444071769\\
38	7.91937553967966\\
39	7.04933916357508\\
40	6.94733270409969\\
41	6.50768543780927\\
42	6.72096939924377\\
43	6.81253170875274\\
44	7.21635055263805\\
45	7.76600671525884\\
46	6.29635976533866\\
47	5.75657862874917\\
48	6.07476961527507\\
49	6.53616744839529\\
50	5.95835430599017\\
51	5.57625579728137\\
52	6.19706467318145\\
53	5.53995967634771\\
54	5.99788474555143\\
55	6.70639465671242\\
56	6.65717155638325\\
57	5.05334411364801\\
58	5.27134438354368\\
59	5.0677194393672\\
60	6.42473117683413\\
61	5.2752155806322\\
62	4.59352182192691\\
63	6.30638222634486\\
64	5.42353707423124\\
65	5.39004573345339\\
66	5.60599257419985\\
67	4.75225466411548\\
68	5.38437858595591\\
69	5.43354713550278\\
70	4.12655825168611\\
71	4.50001114969064\\
72	4.50784977623109\\
73	4.24034443804195\\
74	4.31339028785324\\
75	3.34139795648375\\
76	3.54597180818302\\
77	4.38509881248586\\
78	4.91700819088966\\
79	2.85111118525423\\
80	4.79634957877514\\
81	3.55508906597475\\
82	3.82644757279099\\
83	2.76857067967772\\
84	3.09068826434567\\
85	3.31443629573351\\
86	3.58844957253773\\
87	3.71445219180415\\
88	3.06485040480615\\
89	2.57923012593849\\
90	3.06167671433582\\
91	3.05588304204481\\
92	2.96533988241115\\
93	2.78407821620527\\
94	1.92430654344587\\
95	2.45024101072767\\
96	2.69561704147348\\
97	2.45880563346952\\
98	3.52696382962867\\
99	1.66445443357313\\
100	2.33467780886966\\
101	1.49189614456445\\
102	2.08007792465209\\
103	2.08672009817051\\
104	0.72827024265899\\
105	1.04400431598137\\
106	2.67576964044233\\
107	1.66419152053567\\
108	2.08164599503734\\
109	1.85915562823552\\
110	1.61148496162346\\
111	0.650869631051087\\
112	1.42254156465115\\
113	0.751608916511079\\
114	1.65070582720715\\
115	0.917777817855487\\
116	0.674073407893245\\
117	0.0514327907639733\\
118	1.02695763687249\\
119	1.19187934865686\\
120	-0.269944642048342\\
121	-0.131904251474694\\
122	-0.364118566594601\\
123	-0.0708189843911174\\
124	-0.469179934805415\\
125	-0.41845055184143\\
126	-0.579257772670734\\
127	-0.0748623664361422\\
128	-0.355331050741244\\
129	-0.0153807054054522\\
130	-0.635946594865844\\
131	0.71963301146075\\
132	-0.459011366624103\\
133	-0.700936935325126\\
134	-1.63401991401108\\
135	-0.0859306970006149\\
136	-1.09080164025349\\
137	-0.635877468137562\\
138	-0.772532237767418\\
139	-0.474040656545545\\
140	-1.36321111535449\\
141	-1.48446624247833\\
142	-1.25327229243838\\
143	-0.813315108516969\\
144	-1.2295774705881\\
145	-1.47759873466522\\
146	-2.56891296639678\\
147	-2.3320706835115\\
148	-2.02825596898869\\
149	-1.91665241829569\\
150	-2.32683570375166\\
151	-2.70072334128692\\
152	-2.36903114526709\\
153	-2.7241593146441\\
154	-2.65581789804721\\
155	-2.83808523711356\\
156	-2.67835846022437\\
157	-1.72604286082898\\
158	-3.40934310404887\\
159	-4.40868453004594\\
160	-1.9684069259501\\
161	-3.4555176748801\\
162	-3.20357711977468\\
163	-2.90161526646798\\
164	-3.8313654845392\\
165	-3.20458934610706\\
166	-4.68253281747112\\
167	-4.98965981861225\\
168	-4.22901288987208\\
169	-3.87501107745953\\
170	-5.14206715533441\\
171	-4.05233196519724\\
172	-4.60322895211268\\
173	-4.49268153485676\\
174	-4.55388128064541\\
175	-3.91590385908994\\
176	-3.88435058770121\\
177	-5.22126828567383\\
178	-4.13061728274657\\
179	-5.26639961625374\\
180	-4.82446930517704\\
181	-5.29606035495589\\
182	-5.21395699421413\\
183	-4.734154083844\\
184	-5.25040844254879\\
185	-5.85341426913402\\
186	-4.90701043922816\\
187	-4.86694012284118\\
188	-5.44134579392696\\
189	-6.81981737117056\\
190	-5.0851596750058\\
191	-6.54675575720323\\
192	-6.58521129195407\\
193	-4.97171262637303\\
194	-6.22446120143538\\
195	-6.39277662932985\\
196	-7.60013005254121\\
197	-6.48338286672742\\
198	-6.81911922156493\\
199	-6.21598826178141\\
200	-6.57632744772191\\
201	-5.94334987382185\\
202	-6.4686758438295\\
203	-7.34264991936562\\
204	-6.37928953303415\\
205	-6.34120506339931\\
206	-6.71371053252457\\
207	-6.85614402900732\\
208	-7.80265242866499\\
209	-8.20076914921733\\
210	-8.33229567010684\\
211	-6.47314690899631\\
212	-7.50693415213092\\
213	-7.05397685635699\\
214	-7.396564707977\\
215	-6.2814433232937\\
216	-8.17235809176\\
217	-8.08055652108537\\
218	-7.69054703124526\\
219	-7.49899435343449\\
220	-7.34113206812797\\
221	-8.45162109326273\\
222	-7.83609230281307\\
223	-8.94081543436404\\
224	-8.27903436624434\\
225	-8.69098211149628\\
226	-8.54347925044105\\
227	-8.53238047462509\\
228	-10.4272613070268\\
229	-8.9986746920167\\
230	-10.2976783842486\\
231	-9.79941727619997\\
232	-8.9871950211875\\
233	-9.40322202270408\\
234	-9.99744761217464\\
235	-10.3849061532229\\
236	-10.1342157674712\\
237	-8.94202427577202\\
238	-9.29986841261937\\
239	-9.89071314895659\\
240	-8.84635947712717\\
241	-10.2289576582902\\
242	-10.1224571408136\\
243	-10.6304503481718\\
244	-9.82069819236454\\
245	-10.0785691024241\\
246	-9.98525522345002\\
247	-9.06650682524307\\
248	-10.4090339416927\\
249	-10.1083099037507\\
250	-11.6219750855764\\
251	-9.96693063693394\\
252	-11.2919205476879\\
253	-11.4293897390073\\
254	-11.5477471324372\\
255	-11.8106846034084\\
256	-11.263079380564\\
257	-11.3573610068099\\
258	-11.210537403974\\
259	-11.2013679764941\\
260	-11.6754841011143\\
261	-11.8143619167638\\
262	-12.1100909087506\\
263	-13.27929102195\\
264	-11.7078634721348\\
265	-11.681265937411\\
266	-11.9572476669388\\
267	-12.4230191334211\\
268	-13.3878981766134\\
269	-11.9442139652565\\
270	-13.2008742176209\\
271	-11.8083148708025\\
272	-12.165267580905\\
273	-13.1423469493102\\
274	-12.7556248860813\\
275	-13.1546028863212\\
276	-14.3041549159721\\
277	-13.8963784374726\\
278	-14.330938755137\\
279	-12.7976800555637\\
280	-13.8715740386101\\
281	-13.4705775076174\\
282	-15.6113384609727\\
283	-14.1778445262299\\
284	-13.7563938877737\\
285	-12.4158821709476\\
286	-14.2083520299019\\
287	-14.6146505766945\\
288	-13.46931416082\\
289	-13.5247955990763\\
290	-14.3982511686686\\
291	-13.8114981253949\\
292	-15.3564528029395\\
293	-14.5368533129691\\
294	-15.1260218595853\\
295	-14.3142407952645\\
296	-15.1441755994875\\
297	-14.6218541444398\\
298	-14.3076822277972\\
299	-15.0334652072343\\
300	-16.9374981668027\\
301	-15.4166109097518\\
302	-14.6922132327915\\
303	-15.5968773752425\\
304	-15.7361080830295\\
305	-15.7271275835386\\
306	-15.1091789000835\\
307	-15.2500433821537\\
308	-14.8896083309895\\
309	-16.5932846195804\\
310	-15.3358709131813\\
311	-15.3872168460558\\
312	-15.5018472368726\\
313	-16.5183351038312\\
314	-16.2274306326519\\
315	-16.3776964765691\\
316	-16.0522280008779\\
317	-16.2045214988564\\
318	-17.4662838029602\\
319	-16.4041884478682\\
320	-17.244320495604\\
321	-18.7074289072086\\
322	-17.3825060393076\\
323	-16.3855722091555\\
324	-17.0044160323569\\
325	-18.0630251248854\\
326	-18.1013062313056\\
327	-17.501725229496\\
328	-18.1991264778377\\
329	-18.7090625106385\\
330	-19.5088079630743\\
331	-19.7399562026387\\
332	-17.4504296499464\\
333	-18.2246520031339\\
334	-18.5290304439098\\
335	-18.2356907112586\\
336	-18.012183434057\\
337	-18.7060220611333\\
338	-18.076787331527\\
339	-18.8488303550318\\
340	-18.5971052713207\\
341	-18.4855835487002\\
342	-19.6103409064883\\
343	-19.4349351270903\\
344	-19.103791829317\\
345	-18.8010357448736\\
346	-19.0049679160487\\
347	-19.2673654373368\\
348	-20.2005606410142\\
349	-19.7774352607679\\
350	-19.1067455648791\\
351	-20.5431679067782\\
352	-20.9901876151951\\
353	-19.2188455018129\\
354	-18.5801464949481\\
355	-18.9142452247912\\
356	-20.24959556046\\
357	-19.5530676070208\\
358	-20.1053515533918\\
359	-19.9502927160421\\
360	-20.2363268486009\\
361	-20.3981939225941\\
362	-20.3913586157623\\
363	-20.0744000497998\\
364	-20.9813166997359\\
365	-21.1161970269757\\
366	-21.5302534835776\\
367	-20.7932585031394\\
368	-20.4316649943343\\
369	-21.3638374860971\\
370	-21.8204945742607\\
371	-22.352100602971\\
372	-22.1621644334163\\
373	-21.2485715823951\\
374	-21.2873599892751\\
375	-21.3009190112497\\
376	-21.8997112326143\\
377	-21.191649937864\\
378	-23.3600062616681\\
379	-22.1709107360637\\
380	-21.70409164136\\
381	-22.4024648649603\\
382	-21.7692859484566\\
383	-21.9737238062041\\
384	-22.3460073216003\\
385	-21.4977138420994\\
386	-22.6290502103529\\
387	-22.4311272158314\\
388	-22.351206068394\\
389	-22.7142649204284\\
390	-23.8932308764797\\
391	-23.5890105044333\\
392	-23.2909013563954\\
393	-22.6048611107998\\
394	-22.80936293104\\
395	-23.6870610478813\\
396	-24.3015664826051\\
397	-22.9859708776987\\
398	-23.6274541398413\\
399	-24.6063571560844\\
400	-23.5235916274974\\
401	-23.948549749926\\
402	-23.6427068102107\\
403	-23.4628817759401\\
404	-23.6475322812905\\
405	-24.1468115991635\\
406	-24.464084615369\\
407	-24.9402732301274\\
408	-25.2931875050841\\
409	-24.8229301149293\\
410	-24.2773480686632\\
411	-24.7767001313829\\
412	-24.8515962010266\\
413	-25.6740778623924\\
414	-25.0998056032022\\
415	-26.2680214332221\\
416	-25.6289267673772\\
417	-26.7930889438116\\
418	-25.1952224169139\\
419	-25.8614928143804\\
420	-25.2313536541905\\
421	-25.8560533284655\\
422	-25.3601774189647\\
423	-25.2393605917475\\
424	-25.6668311903975\\
425	-26.2754434038471\\
426	-25.939245838872\\
427	-26.3693201506773\\
428	-26.1461433325662\\
429	-26.3394742049611\\
430	-26.6549096449138\\
431	-26.1353989126268\\
432	-27.5165897456477\\
433	-26.0708625412972\\
434	-26.7354552773485\\
435	-27.1907435441365\\
436	-28.0657180549543\\
437	-26.5256208576022\\
438	-27.7997591944252\\
439	-26.5498068510423\\
440	-26.6234223686263\\
441	-27.4304030137006\\
442	-27.4826382166728\\
443	-28.6858159869714\\
444	-28.0855783138998\\
445	-28.4856623015665\\
446	-28.2608177154418\\
447	-27.941261438552\\
448	-29.5281400898807\\
449	-27.5274182434215\\
450	-27.4676021369257\\
451	-29.1886042570481\\
452	-29.0471222235957\\
453	-29.1736960734431\\
454	-28.3522705663808\\
455	-28.0676021513856\\
456	-30.0810398465565\\
457	-29.322395193933\\
458	-28.9936111862027\\
459	-28.1169781502028\\
460	-28.2243470628219\\
461	-29.2262750382496\\
462	-29.0463244870588\\
463	-29.9464929429221\\
464	-30.3060574329231\\
465	-29.4582291176914\\
466	-28.4945645087388\\
467	-29.1478713285302\\
468	-29.9714797270531\\
469	-29.6363597328985\\
470	-30.7433887401308\\
471	-30.2617616410061\\
472	-29.7533789214305\\
473	-31.2731701998431\\
474	-31.234456633307\\
475	-31.1967386377836\\
476	-30.5663986110048\\
477	-30.209576586183\\
478	-31.3600236857615\\
479	-30.9742675966417\\
480	-29.6624551421875\\
481	-30.1311355927753\\
482	-31.6048461678428\\
483	-30.6068060344747\\
484	-29.0155674278365\\
485	-31.0711708714833\\
486	-32.3964261743576\\
487	-31.1326701673816\\
488	-31.7871863686837\\
489	-32.168055332755\\
490	-31.660041219306\\
491	-31.8158016975395\\
492	-30.9648784544437\\
493	-32.1167097576017\\
494	-31.3702253691346\\
495	-32.8990740642298\\
496	-32.7003250990824\\
497	-33.6520249828856\\
498	-32.167860298136\\
499	-33.088650867489\\
500	-32.555244365884\\
501	-32.9963301026149\\
502	-32.7645267811499\\
503	-33.0765160122246\\
504	-33.5886994073166\\
505	-33.1089227246625\\
506	-33.1755609573342\\
507	-33.2614848901717\\
508	-33.3652152926579\\
509	-33.1447643634909\\
510	-32.6506017859353\\
511	-33.106028683983\\
512	-34.1370508540133\\
513	-33.150361885974\\
514	-33.3078297083231\\
515	-32.7528190991046\\
516	-34.2059927416088\\
517	-32.2401355999032\\
518	-34.0206678118128\\
519	-34.2281874753832\\
520	-33.6411713289646\\
521	-35.110324583042\\
522	-33.9950235803571\\
523	-35.7506524394283\\
524	-33.8324740505235\\
525	-34.4233191275975\\
526	-34.2107541906759\\
527	-34.789652038283\\
528	-34.1198835942507\\
529	-35.2952005457844\\
530	-33.7948476676432\\
531	-36.4818241963455\\
532	-34.8506011209049\\
533	-36.1364770862797\\
534	-35.6259866875892\\
535	-35.2833613279428\\
536	-34.8593373575338\\
537	-35.9498233937278\\
538	-35.2208515537142\\
539	-35.9459159317918\\
540	-35.7466175242343\\
541	-36.1885105498768\\
542	-37.3176872965912\\
543	-35.0635099627709\\
544	-35.7789611441754\\
545	-36.1486635771765\\
546	-36.5480121981479\\
547	-37.1657163062812\\
548	-36.3109751317935\\
549	-36.4134151149909\\
550	-35.0519187944861\\
551	-35.8740473159988\\
552	-37.2808192264556\\
553	-37.6209149319384\\
554	-38.307368860654\\
555	-38.1610658559005\\
556	-36.7164370540003\\
557	-36.6985243161219\\
558	-37.0654227179106\\
559	-37.5118367750535\\
560	-36.8144067903966\\
561	-36.145061254799\\
562	-38.185939946314\\
563	-37.978022566645\\
564	-37.6928749527211\\
565	-38.0224664634067\\
566	-37.0245733533237\\
567	-37.7066848079656\\
568	-37.1691343181835\\
569	-38.8242919542771\\
570	-38.3000540649907\\
571	-39.0304560734975\\
572	-38.9310968520937\\
573	-38.602497976496\\
574	-39.2177015114649\\
575	-38.4136734738346\\
576	-38.5032214237355\\
577	-39.3215429285838\\
578	-37.6865183155146\\
579	-39.7666701114546\\
580	-38.667931225365\\
581	-39.4895214336618\\
582	-38.9203940644247\\
583	-40.5777154632099\\
584	-39.6436524620996\\
585	-39.2298021861128\\
586	-40.1282865845856\\
587	-40.2407966192307\\
588	-40.936963871\\
589	-40.0283521433942\\
590	-39.6606364464466\\
591	-40.2065771282456\\
592	-39.1652350310916\\
593	-40.2315867040086\\
594	-40.6181310554297\\
595	-41.2709280870297\\
596	-40.1349670465967\\
597	-41.8048544890595\\
598	-40.1244257822226\\
599	-39.5549337785347\\
600	-40.1390565503716\\
601	-42.3957271023405\\
602	-42.4181347691959\\
603	-40.6531822533885\\
604	-39.732967940902\\
605	-41.3224568643846\\
606	-41.5714474606685\\
607	-40.0933424195193\\
608	-41.7234982394319\\
609	-42.6220884493252\\
610	-41.7723834450594\\
611	-41.606663616761\\
612	-42.6502594727357\\
613	-39.2272852184747\\
614	-41.8193518081315\\
615	-42.9761703831466\\
616	-41.2029739624608\\
617	-41.8875099560931\\
618	-41.7453387481357\\
619	-42.255073667365\\
620	-42.5415472372277\\
621	-43.9156101656475\\
622	-42.263037379269\\
623	-42.5395752298567\\
624	-42.2826173509301\\
625	-43.5851255487177\\
626	-43.6401359154451\\
627	-43.2025063078648\\
628	-42.8893880868161\\
629	-44.6460582459845\\
630	-42.838108867768\\
631	-41.8325978784296\\
632	-43.0279312886046\\
633	-43.1030879062614\\
634	-44.702313674965\\
635	-43.8280610623141\\
636	-44.0596595595707\\
637	-45.4000064129344\\
638	-44.4675675087406\\
639	-45.4781784281383\\
640	-45.2356635944018\\
641	-43.3151936302041\\
642	-44.6911379692207\\
643	-44.5185244339074\\
644	-44.6547668056976\\
645	-43.6644798995787\\
646	-44.7843577337678\\
647	-45.1489198707274\\
648	-45.8697068228694\\
649	-45.6689276953822\\
650	-45.4902489166988\\
651	-45.6319448010216\\
652	-45.7188343025563\\
653	-45.208342879076\\
654	-46.6832537565466\\
655	-44.7533820325811\\
656	-46.446809641883\\
657	-45.4618086660464\\
658	-47.1118527745558\\
659	-46.0705787350502\\
660	-45.6171830030813\\
661	-46.9680430534816\\
662	-46.3825117090656\\
663	-47.6799085208183\\
664	-46.666943539443\\
665	-48.2623698355407\\
666	-46.3643012506101\\
667	-46.6249044656063\\
668	-46.7134065058185\\
669	-45.8015630827864\\
670	-47.2149091616197\\
671	-47.1319216963264\\
672	-47.533193995044\\
673	-46.1576062319532\\
674	-47.1745143055406\\
675	-47.0721413214982\\
676	-47.4630567065984\\
677	-46.9325808859367\\
678	-48.5380774091236\\
679	-48.1114139746152\\
680	-48.2008062323674\\
681	-47.5728812425371\\
682	-48.4701685630438\\
683	-48.8524043615095\\
684	-48.2167519679591\\
685	-47.6262689982447\\
686	-48.3918327683272\\
687	-48.8925429135166\\
688	-48.5606550970209\\
689	-47.4249863851428\\
690	-46.1096837770394\\
691	-48.8274610169012\\
692	-49.2826775300723\\
693	-48.5942703448348\\
694	-50.6165370265054\\
695	-48.1339667303257\\
696	-48.9653376936968\\
697	-49.913344741013\\
698	-49.6603100779188\\
699	-49.4208817572557\\
700	-50.1084198292883\\
701	-48.9700147665692\\
702	-49.1598570930649\\
703	-48.8302588067713\\
704	-49.5942919863228\\
705	-50.1135098113045\\
706	-49.7964987332339\\
707	-49.5587203771478\\
708	-49.1135677787464\\
709	-50.8320076666852\\
710	-49.9623343963569\\
711	-50.0626991544445\\
712	-49.7203231230206\\
713	-51.5946181034043\\
714	-51.5243206891086\\
715	-49.9763840856072\\
716	-50.2744388609338\\
717	-51.2398703599879\\
718	-51.5452479340886\\
719	-52.6921120355456\\
720	-51.1749221115005\\
721	-50.7279829791021\\
722	-52.1019257697597\\
723	-51.7584870054454\\
724	-51.4642901834906\\
725	-50.269739306045\\
726	-52.0788590592165\\
727	-52.8432444744629\\
728	-51.9101509652207\\
729	-51.7561531524453\\
730	-51.3985789767904\\
731	-51.0963742937406\\
732	-52.0162310291547\\
733	-53.8499494086913\\
734	-54.1850666955725\\
735	-53.6017598524813\\
736	-52.5132026973548\\
737	-52.8901518297189\\
738	-53.6666641574828\\
739	-53.6619129760881\\
740	-53.125873611647\\
741	-53.0463229471505\\
742	-52.7218753879534\\
743	-52.7806290206851\\
744	-53.8898553000826\\
745	-53.6433563631655\\
746	-52.3464905196235\\
747	-52.9230636776561\\
748	-53.6693287608536\\
749	-53.6605726867911\\
750	-54.1810487905639\\
751	-53.572368293063\\
752	-54.2415859255211\\
753	-54.6374855700892\\
754	-53.5672397500317\\
755	-53.5401412281598\\
756	-54.3507059121573\\
757	-54.199653779012\\
758	-54.9929106815966\\
759	-54.1799145103938\\
760	-54.0472401381017\\
761	-54.1022399108688\\
762	-54.5307128947114\\
763	-55.0979817158688\\
764	-55.8332199083862\\
765	-54.8931290401711\\
766	-54.7707196816081\\
767	-55.1651926965397\\
768	-54.7958867405412\\
769	-55.2538428898607\\
770	-54.8763941322987\\
771	-55.3263250752801\\
772	-55.2653448866378\\
773	-56.2996814054629\\
774	-56.0124328427653\\
775	-55.8709274893822\\
776	-55.3482963451942\\
777	-56.6082435953192\\
778	-56.9898845511283\\
779	-56.4104224661709\\
780	-56.1815082832404\\
781	-57.5118186151284\\
782	-55.8771915439742\\
783	-56.8301531591704\\
784	-58.3251905901928\\
785	-57.5701504140732\\
786	-57.6764887253348\\
787	-56.4159424596974\\
788	-56.9889036792141\\
789	-57.5761055322448\\
790	-57.1400584523475\\
791	-56.313431331694\\
792	-56.2683405594215\\
793	-57.0345572706697\\
794	-57.5765395449705\\
795	-58.6895470176755\\
796	-58.1058686820457\\
797	-56.6014928493584\\
798	-58.6483559621962\\
799	-57.7077322126788\\
800	-57.7211855778174\\
801	-58.1336071142623\\
802	-59.3098031265322\\
803	-58.9434638333493\\
804	-57.8203901151264\\
805	-58.166106894682\\
806	-58.6239321812459\\
807	-59.2946009524081\\
808	-58.634014579044\\
809	-57.8840694134314\\
810	-59.1160282788714\\
811	-58.3894218922931\\
812	-59.1710950154419\\
813	-60.0088253863317\\
814	-58.453976764855\\
815	-60.3173297969276\\
816	-60.0507465764931\\
817	-59.0195489133071\\
818	-59.9353251009378\\
819	-58.9448057548646\\
820	-60.2369284117042\\
821	-60.0157908405227\\
822	-60.6831677139027\\
823	-59.2962758789494\\
824	-61.4150381402812\\
825	-59.9541993528914\\
826	-60.8704005202393\\
827	-61.0631535897962\\
828	-61.8493212363796\\
829	-60.2859770793507\\
830	-60.3230957024361\\
831	-62.5956865146388\\
832	-60.7886424868555\\
833	-60.4733140977579\\
834	-60.1496738649364\\
835	-60.6208402084171\\
836	-61.4589622899565\\
837	-61.8039323933182\\
838	-61.936703709299\\
839	-62.3187274873411\\
840	-62.2414171442793\\
841	-60.7107101773125\\
842	-62.0872975298958\\
843	-61.3020255375359\\
844	-62.4898978288385\\
845	-62.76278386396\\
846	-62.5408443175201\\
847	-62.8247451585318\\
848	-62.7946583719882\\
849	-62.0415678024624\\
850	-63.0659620995524\\
851	-62.259515307728\\
852	-63.8175117816225\\
853	-62.5597781141872\\
854	-63.467047023972\\
855	-63.8805140513516\\
856	-63.9749799599475\\
857	-62.742203259476\\
858	-62.1448571115442\\
859	-61.305153588631\\
860	-62.9011018469736\\
861	-64.168968342952\\
862	-63.4284459180499\\
863	-64.0805229509672\\
864	-62.9181616843244\\
865	-63.445787821883\\
866	-64.6483447711015\\
867	-65.2539838561534\\
868	-63.9798074672701\\
869	-62.6770973731766\\
870	-64.0572721908351\\
871	-63.6224908520302\\
872	-64.8619084188473\\
873	-65.0554572968395\\
874	-64.6643880718596\\
875	-64.3971873481834\\
876	-65.425537002074\\
877	-65.2547331800853\\
878	-65.1505537427969\\
879	-64.9651777693261\\
880	-65.4145916982198\\
881	-63.3912656960544\\
882	-66.2642577534528\\
883	-64.4115691993786\\
884	-64.6936814607394\\
885	-66.1955295160525\\
886	-65.5782356575376\\
887	-64.8726413283569\\
888	-65.9593674037483\\
889	-65.7077511161714\\
890	-66.0765518330781\\
891	-65.2817904407481\\
892	-66.8852505348307\\
893	-66.6118989138528\\
894	-65.4054939319467\\
895	-66.0875034909297\\
896	-67.1073963717742\\
897	-65.1993892007953\\
898	-66.5714735760151\\
899	-65.5132076052578\\
900	-65.5619038470493\\
901	-67.4068439133279\\
902	-66.6311294598117\\
903	-67.6466508535739\\
904	-65.904726078924\\
905	-66.1678438710736\\
906	-67.3811750151403\\
907	-68.0539180930957\\
908	-68.9525164073097\\
909	-67.1825781400156\\
910	-67.3070917724578\\
911	-69.1381427242343\\
912	-67.4047054018761\\
913	-67.1028154951535\\
914	-66.6795471641032\\
915	-68.3739209774939\\
916	-67.3251514139185\\
917	-68.5378987767326\\
918	-68.910797928373\\
919	-67.0704084654355\\
920	-67.4389894910793\\
921	-68.4119839475289\\
922	-68.5513419558362\\
923	-68.0956125641317\\
924	-67.2019798695839\\
925	-67.2147933930512\\
926	-69.3166656754513\\
927	-68.8366431857867\\
928	-68.8383495830576\\
929	-69.4650779168366\\
930	-68.8470466527778\\
931	-69.484652885618\\
932	-69.9783923555375\\
933	-70.3320963122031\\
934	-69.0088029211749\\
935	-69.26479022602\\
936	-69.4187589255147\\
937	-69.6096249963542\\
938	-69.0402628479045\\
939	-71.1893326603361\\
940	-71.0903238940101\\
941	-70.5944814240258\\
942	-69.6956304696529\\
943	-69.809260134212\\
944	-70.8670987252345\\
945	-69.9178166788081\\
946	-69.7636983734981\\
947	-69.7559214992168\\
948	-71.2827649862024\\
949	-71.2942868317267\\
950	-70.2825639860705\\
951	-70.9646251020251\\
952	-70.6756029871969\\
953	-71.755185778001\\
954	-71.6837274812086\\
955	-71.5151921180131\\
956	-70.7897116834428\\
957	-70.5136255968429\\
958	-70.1942364253655\\
959	-70.6438296038715\\
960	-71.2425451404962\\
961	-73.0229250250432\\
962	-70.6242857347571\\
963	-71.6519643499747\\
964	-72.6150832722489\\
965	-71.875085558713\\
966	-72.2915504188079\\
967	-72.2112743008496\\
968	-72.7738057267141\\
969	-72.7737505741609\\
970	-71.52600031159\\
971	-72.5487130309728\\
972	-72.9371578083016\\
973	-72.211604472378\\
974	-70.8718668174834\\
975	-72.248627536045\\
976	-72.0032411693207\\
977	-74.4685155800597\\
978	-71.8212450208149\\
979	-72.9374393852696\\
980	-72.0873180174053\\
981	-73.388997073626\\
982	-73.2916509570263\\
983	-73.5222031352792\\
984	-75.5975131985663\\
985	-74.1852926305516\\
986	-74.1901715411717\\
987	-73.2095563426103\\
988	-73.7085152736622\\
989	-75.2389650168534\\
990	-75.1459084617502\\
991	-73.1296300323065\\
992	-73.7075804766279\\
993	-75.224942013118\\
994	-74.4418298570548\\
995	-74.7808908540737\\
996	-75.631404942994\\
997	-73.3564866451428\\
998	-74.9512315629355\\
999	-74.7740718312944\\
1000	-inf\\
};
\end{axis}
\end{tikzpicture}%}
		\caption{\textit{Learning Curve of the filter, defined as $ 10 \log|e(n)|^2 $}}
		\label{fig:4_1_a_aclms_err}
	\end{subfigure}
	~ %add desired spacing between images, e. g. ~, \quad, \qquad, \hfill etc.
	%(or a blank line to force the subfigure onto a new line)
	\begin{subfigure}[b]{0.49\textwidth}
		\resizebox{\textwidth}{!}{% This file was created by matlab2tikz.
% Minimal pgfplots version: 1.3
%
%The latest updates can be retrieved from
%  http://www.mathworks.com/matlabcentral/fileexchange/22022-matlab2tikz
%where you can also make suggestions and rate matlab2tikz.
%
\definecolor{mycolor1}{rgb}{0.00000,0.44700,0.74100}%
\definecolor{mycolor2}{rgb}{0.85000,0.32500,0.09800}%
\definecolor{mycolor3}{rgb}{0.92900,0.69400,0.12500}%
\definecolor{mycolor4}{rgb}{0.49400,0.18400,0.55600}%
\definecolor{mycolor5}{rgb}{0.46600,0.67400,0.18800}%
\definecolor{mycolor6}{rgb}{0.30100,0.74500,0.93300}%
\definecolor{mycolor7}{rgb}{0.63500,0.07800,0.18400}%
%
\begin{tikzpicture}

\begin{axis}[%
width=4in,
height=1.8in,
at={(1.011111in,0.641667in)},
scale only axis,
xmin=0,
xmax=1000,
tick align=outside,
xlabel={Iteration},
xmajorgrids,
ymin=-0.5,
ymax=2.5,
ylabel={Estimated Weight},
ymajorgrids,
title style={font=\bfseries},
title={Estmated Weights using ACLMS},
axis x line*=bottom,
axis y line*=left,
legend style={legend cell align=left,align=left,draw=white!15!black}
]
\addplot [color=mycolor1,solid]
  table[row sep=crcr]{%
1	0\\
2	0\\
3	0\\
4	0.00937882978582563\\
5	0.0229104274077277\\
6	0.0302503857741873\\
7	0.0339894544117851\\
8	0.0413949642271044\\
9	0.0518073543344772\\
10	0.056863031919812\\
11	0.0673397547223057\\
12	0.0778754070107918\\
13	0.0840540078616087\\
14	0.0912876472000026\\
15	0.100775083474871\\
16	0.109889818306571\\
17	0.118827505105822\\
18	0.126704745187245\\
19	0.135121525423776\\
20	0.143530867924618\\
21	0.147450518780114\\
22	0.157226502694016\\
23	0.162743012436439\\
24	0.171113083199287\\
25	0.179837811540274\\
26	0.184769862272889\\
27	0.19248081854395\\
28	0.201770966339817\\
29	0.212595528483722\\
30	0.222295216855847\\
31	0.229646734957628\\
32	0.238643391522428\\
33	0.245638241367997\\
34	0.24875123520538\\
35	0.257578790835771\\
36	0.265731477438932\\
37	0.27157363767727\\
38	0.281207786085848\\
39	0.287146891040736\\
40	0.292765574538408\\
41	0.298258064453838\\
42	0.30435187851613\\
43	0.310414151337808\\
44	0.316252139566297\\
45	0.325661189703348\\
46	0.334943090812802\\
47	0.33894521557746\\
48	0.34807942924428\\
49	0.352962187533145\\
50	0.359462002147088\\
51	0.36622338037172\\
52	0.372870883007174\\
53	0.381060866538733\\
54	0.386732549329545\\
55	0.393082392595858\\
56	0.401296665647143\\
57	0.407930915838909\\
58	0.41204044775298\\
59	0.419168686986489\\
60	0.423863684977652\\
61	0.429640079961031\\
62	0.436099165230916\\
63	0.443274081681119\\
64	0.450204297960264\\
65	0.45556055713103\\
66	0.461866539304895\\
67	0.466613436907644\\
68	0.470850683582096\\
69	0.476787669949233\\
70	0.481982313119412\\
71	0.485834184808433\\
72	0.492091184611693\\
73	0.495441821411242\\
74	0.500731329883857\\
75	0.505393930881598\\
76	0.510620987307191\\
77	0.515575055732272\\
78	0.521059534938738\\
79	0.527303986596313\\
80	0.531644855801341\\
81	0.534899856380015\\
82	0.539328661804794\\
83	0.542412119280916\\
84	0.545954885635465\\
85	0.549375478958531\\
86	0.554237300842026\\
87	0.558820422198818\\
88	0.562535832269131\\
89	0.567729147067518\\
90	0.572106438035974\\
91	0.578580972344361\\
92	0.582368670224013\\
93	0.585820600707682\\
94	0.589722076878833\\
95	0.594693104484784\\
96	0.599352173412219\\
97	0.603361948846594\\
98	0.607894352455824\\
99	0.611821245337642\\
100	0.61626825864458\\
101	0.619774604370438\\
102	0.622797091349465\\
103	0.628095688250415\\
104	0.631668182631673\\
105	0.634160906122383\\
106	0.637001102247074\\
107	0.641041335416443\\
108	0.643910185862009\\
109	0.647396068253798\\
110	0.649667797734087\\
111	0.653958044947125\\
112	0.656101063816879\\
113	0.659756086433431\\
114	0.664879205153712\\
115	0.668478082477123\\
116	0.672296349709089\\
117	0.676232733799499\\
118	0.680253094371058\\
119	0.683992105231324\\
120	0.687141128188522\\
121	0.690479321491852\\
122	0.694659729632401\\
123	0.697601973913074\\
124	0.701056236494421\\
125	0.704403557455097\\
126	0.707932344002842\\
127	0.710143016015873\\
128	0.714072603186743\\
129	0.717386835896705\\
130	0.719451048135125\\
131	0.722949166536649\\
132	0.726724777135804\\
133	0.729556444338025\\
134	0.731972261346818\\
135	0.735439256307007\\
136	0.738955406833753\\
137	0.740959328716763\\
138	0.744239606872849\\
139	0.747017071874182\\
140	0.750451066986066\\
141	0.752593258175901\\
142	0.754761179993754\\
143	0.758007285283717\\
144	0.760273845202236\\
145	0.764039669131645\\
146	0.765901334928359\\
147	0.767939801753622\\
148	0.771007309791792\\
149	0.772780542941793\\
150	0.775231673280126\\
151	0.77722823858816\\
152	0.77857002655484\\
153	0.780935258337604\\
154	0.783762135494852\\
155	0.786508164322466\\
156	0.787938635269667\\
157	0.790099646759052\\
158	0.791067956416347\\
159	0.792998478870306\\
160	0.79482645683116\\
161	0.797582894472611\\
162	0.798912875696142\\
163	0.799861795104666\\
164	0.801738219791712\\
165	0.803126879974566\\
166	0.804320442477098\\
167	0.805548563762519\\
168	0.806902856530166\\
169	0.808554677685298\\
170	0.810273602994489\\
171	0.812556070239254\\
172	0.81417424071049\\
173	0.81611190629958\\
174	0.817609423307338\\
175	0.819618926005723\\
176	0.821995951079677\\
177	0.823786939382514\\
178	0.825754819598398\\
179	0.827727069827736\\
180	0.829035354850041\\
181	0.83056608897784\\
182	0.832206413080125\\
183	0.834182553973062\\
184	0.836203588659866\\
185	0.83787871629399\\
186	0.839984268727036\\
187	0.842007769936547\\
188	0.843045229828089\\
189	0.84460987201817\\
190	0.846143748126754\\
191	0.847834941961539\\
192	0.848897255708956\\
193	0.850490976892568\\
194	0.852689686186724\\
195	0.854377628264877\\
196	0.855796005338912\\
197	0.856520896287454\\
198	0.857607710298211\\
199	0.859282690789028\\
200	0.860914242558091\\
201	0.86242556445829\\
202	0.864214617180604\\
203	0.86537689063157\\
204	0.866342387404333\\
205	0.867519829641199\\
206	0.869241360101951\\
207	0.870783129955556\\
208	0.871954830906865\\
209	0.873292074885376\\
210	0.874370604370895\\
211	0.875417779957382\\
212	0.876945415510885\\
213	0.87842210454688\\
214	0.879390174862869\\
215	0.880726942525822\\
216	0.88277846646513\\
217	0.883895675815228\\
218	0.885287049266682\\
219	0.885992877780764\\
220	0.887597114732001\\
221	0.888502984755599\\
222	0.889758001682968\\
223	0.891236128712016\\
224	0.891945880501798\\
225	0.892859718366399\\
226	0.893625998894468\\
227	0.894300292571277\\
228	0.895073767767493\\
229	0.896043696312313\\
230	0.896734991436364\\
231	0.897513143235692\\
232	0.898560587484899\\
233	0.899647768806937\\
234	0.900465529250245\\
235	0.901310261262306\\
236	0.902333909732395\\
237	0.903838493577579\\
238	0.904474053298722\\
239	0.905544345795978\\
240	0.906977889257524\\
241	0.908033825955022\\
242	0.909127749883304\\
243	0.909891739301972\\
244	0.910796514804595\\
245	0.911669412669484\\
246	0.912549470252225\\
247	0.913789595680897\\
248	0.91456655429048\\
249	0.915704894485602\\
250	0.916777015757939\\
251	0.917692627512922\\
252	0.918690007245681\\
253	0.919232898640679\\
254	0.919564493103683\\
255	0.920287625107815\\
256	0.920964794967674\\
257	0.921587012612909\\
258	0.922594693268334\\
259	0.923649498809759\\
260	0.924571055713688\\
261	0.925635558365097\\
262	0.92648406652262\\
263	0.927100689724633\\
264	0.927718293360206\\
265	0.928878946170313\\
266	0.929508048065284\\
267	0.930043020562792\\
268	0.930580840784119\\
269	0.931320665502669\\
270	0.931878791783826\\
271	0.932501613326542\\
272	0.933123707446681\\
273	0.934165124296509\\
274	0.935113262122394\\
275	0.935866280815038\\
276	0.93648975961555\\
277	0.937126699568393\\
278	0.937616640185386\\
279	0.938237580190383\\
280	0.938940360467913\\
281	0.939386400207328\\
282	0.939768654360106\\
283	0.940339796034089\\
284	0.940931629524628\\
285	0.94153873670709\\
286	0.942252060041889\\
287	0.942736196069027\\
288	0.943361847027704\\
289	0.944188065595998\\
290	0.944506280892505\\
291	0.945060870987528\\
292	0.945681033600937\\
293	0.946165290324989\\
294	0.946825975888725\\
295	0.947439765497519\\
296	0.947969963232902\\
297	0.948623312542641\\
298	0.949155987377349\\
299	0.949808956076157\\
300	0.950110395185681\\
301	0.950421189586095\\
302	0.950877224676313\\
303	0.951360527259958\\
304	0.951728051991929\\
305	0.952264602292655\\
306	0.952815713252737\\
307	0.953434600184168\\
308	0.954225374762259\\
309	0.954850987808685\\
310	0.955351986885066\\
311	0.955959785646972\\
312	0.956428353145519\\
313	0.956947838465993\\
314	0.957440951157627\\
315	0.958012337866218\\
316	0.958373581379816\\
317	0.958679887043489\\
318	0.959227867056164\\
319	0.959590214283173\\
320	0.960016581325285\\
321	0.960190230057855\\
322	0.960435406705825\\
323	0.96085166441319\\
324	0.961352966850835\\
325	0.961753273580286\\
326	0.962167108937479\\
327	0.962349106056633\\
328	0.962697104705152\\
329	0.963019807132313\\
330	0.963269488874055\\
331	0.963553913273593\\
332	0.963955338013088\\
333	0.964314430066754\\
334	0.964755437684189\\
335	0.965019191798488\\
336	0.96542800399618\\
337	0.965861635817861\\
338	0.966082650911036\\
339	0.966505645002972\\
340	0.966574773976401\\
341	0.966883465290854\\
342	0.967256356041373\\
343	0.967482756887262\\
344	0.967781376238917\\
345	0.968135734006397\\
346	0.968448655407301\\
347	0.968858176744182\\
348	0.969189604596359\\
349	0.969460548942879\\
350	0.969726244270852\\
351	0.970020671542675\\
352	0.970279367798591\\
353	0.970523798469182\\
354	0.970804165715089\\
355	0.971313957293384\\
356	0.971666466369007\\
357	0.972054620036181\\
358	0.972497506223739\\
359	0.972828888892263\\
360	0.973105042994089\\
361	0.973361953489845\\
362	0.973670909668516\\
363	0.973984787172845\\
364	0.974250529961384\\
365	0.974450817731919\\
366	0.974683089899873\\
367	0.974880989048235\\
368	0.975208186106521\\
369	0.975403531686131\\
370	0.975562466291176\\
371	0.975832708961088\\
372	0.976039184336958\\
373	0.97620102758922\\
374	0.976532591880281\\
375	0.976898110678117\\
376	0.977203968304492\\
377	0.977494800598646\\
378	0.977743265720852\\
379	0.978013575936716\\
380	0.978171703296716\\
381	0.978402866910276\\
382	0.978609992764899\\
383	0.978824826460691\\
384	0.979040148106347\\
385	0.97932803485984\\
386	0.979403207785627\\
387	0.979632861372774\\
388	0.979889415416838\\
389	0.980038949726753\\
390	0.980256654622538\\
391	0.980367320374784\\
392	0.980551244575291\\
393	0.980816643916694\\
394	0.981140887595874\\
395	0.981319535156906\\
396	0.981465576458855\\
397	0.981665324952657\\
398	0.981881045547789\\
399	0.982047192773106\\
400	0.982199962196208\\
401	0.982407220417057\\
402	0.982576778220337\\
403	0.982720601013615\\
404	0.982938872864003\\
405	0.983076950587012\\
406	0.983297889967374\\
407	0.983432459332796\\
408	0.983559136503445\\
409	0.983739120158734\\
410	0.983895529382722\\
411	0.984066412309621\\
412	0.98424135525411\\
413	0.984371322003224\\
414	0.984532380783372\\
415	0.984714802072897\\
416	0.984860299167708\\
417	0.98500899749877\\
418	0.985164250906115\\
419	0.985294655697994\\
420	0.985384117638116\\
421	0.985547072383808\\
422	0.985617356785183\\
423	0.985795222452477\\
424	0.98597800700596\\
425	0.986080748089856\\
426	0.986183898544522\\
427	0.986310934908227\\
428	0.986470294023324\\
429	0.986653861282383\\
430	0.986765874299684\\
431	0.986891169129879\\
432	0.987054847594244\\
433	0.987184155074002\\
434	0.987318657965286\\
435	0.987433296196961\\
436	0.987518842750718\\
437	0.987609620872565\\
438	0.987689517627261\\
439	0.987798828807491\\
440	0.98793347450221\\
441	0.988009189336333\\
442	0.98815262840165\\
443	0.98829310952514\\
444	0.988397543369328\\
445	0.988554636646227\\
446	0.988661839648134\\
447	0.988794090993372\\
448	0.988898311213226\\
449	0.988995294646317\\
450	0.989153210854421\\
451	0.989272681785887\\
452	0.989342582251372\\
453	0.989424366931712\\
454	0.989528592601207\\
455	0.989644989401843\\
456	0.989722962045369\\
457	0.989807441966365\\
458	0.989889160209361\\
459	0.989995793380296\\
460	0.99010720363944\\
461	0.990214812537544\\
462	0.990285064336589\\
463	0.990379512457664\\
464	0.990463826885571\\
465	0.990557847564766\\
466	0.990650531450951\\
467	0.990761265685894\\
468	0.9908555905917\\
469	0.990916693326304\\
470	0.991036022709058\\
471	0.991112736942149\\
472	0.991203402792458\\
473	0.991258064221549\\
474	0.991312150822566\\
475	0.991405926209379\\
476	0.991517807526712\\
477	0.991609801786761\\
478	0.991676124311063\\
479	0.991761542119138\\
480	0.991853666148727\\
481	0.991970304361765\\
482	0.992041613902235\\
483	0.992117592257022\\
484	0.992233927451339\\
485	0.992361124183723\\
486	0.992418851033294\\
487	0.992485174796201\\
488	0.992538405817525\\
489	0.992621712781575\\
490	0.992703960784499\\
491	0.99276058159818\\
492	0.992837537787658\\
493	0.992910569496283\\
494	0.993001655666319\\
495	0.993079740657944\\
496	0.993147500750167\\
497	0.993247367082708\\
498	0.993266995165461\\
499	0.993328158471607\\
500	0.993406476679148\\
501	0.993475990956104\\
502	0.993533014962416\\
503	0.993557199434978\\
504	0.993600201435922\\
505	0.993643569628963\\
506	0.99372349465236\\
507	0.993772438014051\\
508	0.993832657763925\\
509	0.993909336069498\\
510	0.993972770062488\\
511	0.994036918326649\\
512	0.994086678887919\\
513	0.994157877205333\\
514	0.994194386563476\\
515	0.994249165034852\\
516	0.994313470872533\\
517	0.994367111385387\\
518	0.994442409744882\\
519	0.994495369576384\\
520	0.994556485535107\\
521	0.994613841956725\\
522	0.994643070281708\\
523	0.994674172643546\\
524	0.994725864794956\\
525	0.994795918367137\\
526	0.994848972601323\\
527	0.994902739577842\\
528	0.99496573841344\\
529	0.995020802310646\\
530	0.995071848249442\\
531	0.995112320397226\\
532	0.995153066748487\\
533	0.995191603356142\\
534	0.995241841822271\\
535	0.995300352828231\\
536	0.995367069850815\\
537	0.995412559061173\\
538	0.995460900110357\\
539	0.995505476097869\\
540	0.995564824683156\\
541	0.995607961933788\\
542	0.995633482795487\\
543	0.995665311493567\\
544	0.995722184085734\\
545	0.995767464755416\\
546	0.995812197941904\\
547	0.995854984636759\\
548	0.995906845466685\\
549	0.995943894257551\\
550	0.995973159678954\\
551	0.996046261897939\\
552	0.996061473075638\\
553	0.996081523272893\\
554	0.996115187445379\\
555	0.996141806091644\\
556	0.996176630220171\\
557	0.996242997580193\\
558	0.996254682579946\\
559	0.99629399868441\\
560	0.99632697581416\\
561	0.996363712870622\\
562	0.996396160268909\\
563	0.996437744425949\\
564	0.996470889768763\\
565	0.996517876636763\\
566	0.996546134330715\\
567	0.996602965177875\\
568	0.996650712447406\\
569	0.996676269316527\\
570	0.996710432002217\\
571	0.996746009874548\\
572	0.996779925473597\\
573	0.996797041391573\\
574	0.996817056228222\\
575	0.996843156004798\\
576	0.996880012836404\\
577	0.996908087608099\\
578	0.996942007579987\\
579	0.996968255457657\\
580	0.996988299307067\\
581	0.997027179763141\\
582	0.997055177194584\\
583	0.997079936252471\\
584	0.997120948795942\\
585	0.99715383068104\\
586	0.99719660304406\\
587	0.99721262449374\\
588	0.997243533232327\\
589	0.997276592459052\\
590	0.997304127560348\\
591	0.997333965487521\\
592	0.997364295603549\\
593	0.997398404954449\\
594	0.997418432874928\\
595	0.99742571127408\\
596	0.997455231517224\\
597	0.997474768984485\\
598	0.997508438324153\\
599	0.997542705103865\\
600	0.997564965901185\\
601	0.997592561192885\\
602	0.997612191047504\\
603	0.997633757274686\\
604	0.997664820357274\\
605	0.997689381206489\\
606	0.997710141707787\\
607	0.99774047981507\\
608	0.997776995856334\\
609	0.997787793788623\\
610	0.997803308432333\\
611	0.997824275911554\\
612	0.997844974294905\\
613	0.997870536030841\\
614	0.997912714702734\\
615	0.997936859280908\\
616	0.997961223840984\\
617	0.99798616390165\\
618	0.998003437935683\\
619	0.998019416576855\\
620	0.998036302124946\\
621	0.998054687751156\\
622	0.998074974485282\\
623	0.99810246162231\\
624	0.998121570442057\\
625	0.998144500705383\\
626	0.998155380968748\\
627	0.998170153094156\\
628	0.998188280334748\\
629	0.998211972103777\\
630	0.998228072747476\\
631	0.998250017832558\\
632	0.998268698860077\\
633	0.99829217318511\\
634	0.998305192610235\\
635	0.998322333520862\\
636	0.998349183642704\\
637	0.998364436135076\\
638	0.998371071562461\\
639	0.9983848943778\\
640	0.998390137223634\\
641	0.998409903224616\\
642	0.998430234870319\\
643	0.998443877574663\\
644	0.998457837393338\\
645	0.998477922868325\\
646	0.998492120457814\\
647	0.998507455823709\\
648	0.998522209214755\\
649	0.998536007627718\\
650	0.998548417377307\\
651	0.998563829978872\\
652	0.998579019174419\\
653	0.99859169294339\\
654	0.998597126741994\\
655	0.998615873759356\\
656	0.998630442307952\\
657	0.998638785488518\\
658	0.998656113914701\\
659	0.998666643394346\\
660	0.998682012846415\\
661	0.998694069232949\\
662	0.998705896899412\\
663	0.998721095459851\\
664	0.998729323636963\\
665	0.99873953746816\\
666	0.998746379058089\\
667	0.998760743401161\\
668	0.998769419761438\\
669	0.998787030593552\\
670	0.998796646635693\\
671	0.998806989569757\\
672	0.998818853034004\\
673	0.998824058274935\\
674	0.998838124738024\\
675	0.998851713576373\\
676	0.99886420321391\\
677	0.998874417161023\\
678	0.998883878037844\\
679	0.998890628039804\\
680	0.998902012422912\\
681	0.998911462325443\\
682	0.998921676475141\\
683	0.998924457110672\\
684	0.998932705771157\\
685	0.998943929189679\\
686	0.998952084305278\\
687	0.998959673078489\\
688	0.998968377122133\\
689	0.998978520411087\\
690	0.998992840550213\\
691	0.999008522975721\\
692	0.999017774670599\\
693	0.999026301947104\\
694	0.999035701509796\\
695	0.999046286520032\\
696	0.999059061431999\\
697	0.999064419682188\\
698	0.999075198665462\\
699	0.999082897119769\\
700	0.999090777707957\\
701	0.999096705985155\\
702	0.999102025042354\\
703	0.999113560250874\\
704	0.999121361745005\\
705	0.999129072528267\\
706	0.999134912859874\\
707	0.99914454284551\\
708	0.999152279513065\\
709	0.999162604826763\\
710	0.999172767448006\\
711	0.999181827997269\\
712	0.999191941019246\\
713	0.999201933693184\\
714	0.999208540377508\\
715	0.999214911950957\\
716	0.999223470267824\\
717	0.999228911355319\\
718	0.999234387704518\\
719	0.999241597045365\\
720	0.999248537361371\\
721	0.999257453075993\\
722	0.999263578390023\\
723	0.999273622530164\\
724	0.999278341708724\\
725	0.999285981721349\\
726	0.999294440730488\\
727	0.999300256607401\\
728	0.999307345382748\\
729	0.999315848707996\\
730	0.999324588408905\\
731	0.999329995744186\\
732	0.999338860220413\\
733	0.999342540118686\\
734	0.999349634975172\\
735	0.999354864489817\\
736	0.999361278267475\\
737	0.999367676135781\\
738	0.999375282480643\\
739	0.999381163298111\\
740	0.999387296150945\\
741	0.999393183860035\\
742	0.999398373282006\\
743	0.999403362646163\\
744	0.999408268261919\\
745	0.999412101796057\\
746	0.999420277105376\\
747	0.999427282973954\\
748	0.999434578939884\\
749	0.999440213669387\\
750	0.999446006753184\\
751	0.999452182057607\\
752	0.999458441294042\\
753	0.999463448136303\\
754	0.999467819328069\\
755	0.999473327246012\\
756	0.999477616257671\\
757	0.999482836259308\\
758	0.999489037456425\\
759	0.999492781967682\\
760	0.999501548076788\\
761	0.999506611353794\\
762	0.999509355814861\\
763	0.99951330145661\\
764	0.999517982172683\\
765	0.999519139069018\\
766	0.999523508449761\\
767	0.999529051504356\\
768	0.999533402595945\\
769	0.999541063595798\\
770	0.999546223706045\\
771	0.999551204194893\\
772	0.999557674893249\\
773	0.999563561141414\\
774	0.999566503735159\\
775	0.999571693923066\\
776	0.999576349437384\\
777	0.999581654284518\\
778	0.999583817064295\\
779	0.999586313162417\\
780	0.999589719115872\\
781	0.999591705551769\\
782	0.999594285246377\\
783	0.999597347448101\\
784	0.999601476935319\\
785	0.999603164390996\\
786	0.999606071375435\\
787	0.999608617944773\\
788	0.999613259090683\\
789	0.99961589336211\\
790	0.999618476711754\\
791	0.999624389165348\\
792	0.999627267917895\\
793	0.999632234338911\\
794	0.999636916163439\\
795	0.999640312845771\\
796	0.999644706548125\\
797	0.999645952914954\\
798	0.999650719469363\\
799	0.999653627392999\\
800	0.999656952293094\\
801	0.999661197316188\\
802	0.999662683379526\\
803	0.999665861811474\\
804	0.999668204382461\\
805	0.999672521295512\\
806	0.99967574256051\\
807	0.999678797894365\\
808	0.999683099673668\\
809	0.999685845830634\\
810	0.999689247555651\\
811	0.999694329199562\\
812	0.999697019152517\\
813	0.999698889048376\\
814	0.999702343630323\\
815	0.999707199470764\\
816	0.999709303496841\\
817	0.999712019631768\\
818	0.999715711713656\\
819	0.999719051918615\\
820	0.999722430749664\\
821	0.999723975299989\\
822	0.999727243155236\\
823	0.999729034676386\\
824	0.999733824545515\\
825	0.999736258987408\\
826	0.999739654086587\\
827	0.999742913019177\\
828	0.999745104234389\\
829	0.999746922901237\\
830	0.999749704714377\\
831	0.999750864123002\\
832	0.99975214495094\\
833	0.999755210267185\\
834	0.999758522195851\\
835	0.999761341420938\\
836	0.999765226999141\\
837	0.999767260189721\\
838	0.999769535582092\\
839	0.999772168825868\\
840	0.999773746242864\\
841	0.999775281238708\\
842	0.999778306671546\\
843	0.999780529931949\\
844	0.999782633227457\\
845	0.999783775420368\\
846	0.99978580949922\\
847	0.999787993018532\\
848	0.999789685625233\\
849	0.999791237684823\\
850	0.999792929835565\\
851	0.999795581651333\\
852	0.999797307057676\\
853	0.999799010344901\\
854	0.999800581376852\\
855	0.999801916076076\\
856	0.999802783004176\\
857	0.999804869522358\\
858	0.999807415881051\\
859	0.999809565751309\\
860	0.999811391746419\\
861	0.999813281337291\\
862	0.999814727433019\\
863	0.999815587526321\\
864	0.999818112251138\\
865	0.999820611906301\\
866	0.999822101840842\\
867	0.999823650126457\\
868	0.999824837489387\\
869	0.999827155772074\\
870	0.999829073806634\\
871	0.999830668288757\\
872	0.999833150882369\\
873	0.999834491846142\\
874	0.999835849347104\\
875	0.999836713511943\\
876	0.999837950008279\\
877	0.999838975098646\\
878	0.999839776358426\\
879	0.999841211946337\\
880	0.999842703555346\\
881	0.999844041754442\\
882	0.999846412463746\\
883	0.999847183357824\\
884	0.999849120679471\\
885	0.999850386574145\\
886	0.999851695495195\\
887	0.999853927493687\\
888	0.999855889969681\\
889	0.999856798759253\\
890	0.999858235790506\\
891	0.999859946481623\\
892	0.999861011815355\\
893	0.999862036385395\\
894	0.999863522907889\\
895	0.999865082626034\\
896	0.999866286844571\\
897	0.999867815836421\\
898	0.999869236494192\\
899	0.999870325622316\\
900	0.999871561252594\\
901	0.999873215309583\\
902	0.999873935899528\\
903	0.999875021413397\\
904	0.999876079652665\\
905	0.999878016841256\\
906	0.999878931766736\\
907	0.999880049534399\\
908	0.999880930537385\\
909	0.999881854500622\\
910	0.999882894874542\\
911	0.999883993554983\\
912	0.999884918985348\\
913	0.999886360892025\\
914	0.999887773753876\\
915	0.999889099893233\\
916	0.999890108437973\\
917	0.999890877085292\\
918	0.999891863774205\\
919	0.999893382411494\\
920	0.999894612177795\\
921	0.999896166960556\\
922	0.999897060825906\\
923	0.999897620673395\\
924	0.999898320007772\\
925	0.999899571828937\\
926	0.999900433107926\\
927	0.999901539799886\\
928	0.99990264442078\\
929	0.999903585699929\\
930	0.999904533605519\\
931	0.999905398550892\\
932	0.999906145009538\\
933	0.99990678658874\\
934	0.999907540371506\\
935	0.999908456284811\\
936	0.999909312040686\\
937	0.999910154166411\\
938	0.999911099991507\\
939	0.99991211920088\\
940	0.999912614791192\\
941	0.999913432816313\\
942	0.999914173327304\\
943	0.999915101743754\\
944	0.999915952910516\\
945	0.999916653693362\\
946	0.999917669576747\\
947	0.999918592591596\\
948	0.999919470507178\\
949	0.999920335384513\\
950	0.999920969440742\\
951	0.999921584634539\\
952	0.999922473474668\\
953	0.99992316346706\\
954	0.999923931952483\\
955	0.9999243953693\\
956	0.99992537392891\\
957	0.99992609868353\\
958	0.999927248960899\\
959	0.999927909788365\\
960	0.999928511510639\\
961	0.999929290974375\\
962	0.999929997459092\\
963	0.999931156985417\\
964	0.999931788205467\\
965	0.999932233698787\\
966	0.999932446740588\\
967	0.999933177047661\\
968	0.999934055123163\\
969	0.999934970856243\\
970	0.999935574393445\\
971	0.99993589136847\\
972	0.999936621277645\\
973	0.999937511948448\\
974	0.999938314654435\\
975	0.999939029412711\\
976	0.999939385692584\\
977	0.999939861815386\\
978	0.999940300740012\\
979	0.999940681067024\\
980	0.999941268973043\\
981	0.999941856044402\\
982	0.999942310616306\\
983	0.99994281579276\\
984	0.999943433466084\\
985	0.999943788258359\\
986	0.999944413919626\\
987	0.999945076193048\\
988	0.999945750919453\\
989	0.999946310569212\\
990	0.999946976476929\\
991	0.999947405879897\\
992	0.999947955355257\\
993	0.999948604763281\\
994	0.999949081848322\\
995	0.999949659300652\\
996	0.999950044668968\\
997	0.999950636038574\\
998	0.999951270732125\\
999	0.999951570581959\\
1000	0.99995197753144\\
};
\addlegendentry{$\Re \{ \mathbf{h}_1^H \}$};

\addplot [color=mycolor2,solid]
  table[row sep=crcr]{%
1	0\\
2	0\\
3	0\\
4	-0.000920740763196139\\
5	-0.00319547002783943\\
6	-0.00240577900493247\\
7	-0.0044155216477574\\
8	-0.0108302220203519\\
9	-0.0085307065134633\\
10	-0.00930192558248409\\
11	-0.00953341787311071\\
12	-0.00669368902472225\\
13	-0.00675188024491923\\
14	-0.00714459673847804\\
15	-0.0055047323998616\\
16	-0.00452466387373259\\
17	-0.00544671926331453\\
18	-0.00635038696070041\\
19	-0.00517995749217255\\
20	-0.00586593509001396\\
21	-0.00484714643040368\\
22	-0.00849214174460414\\
23	-0.00730417913050259\\
24	-0.0105793914718613\\
25	-0.0119445765224016\\
26	-0.0101192135271002\\
27	-0.00977206501025421\\
28	-0.0108540259369038\\
29	-0.0127067403439872\\
30	-0.0094638524431731\\
31	-0.00539637176848866\\
32	-0.00863554929808679\\
33	-0.00741573023491972\\
34	-0.0102843267773926\\
35	-0.0124544687768235\\
36	-0.010187908046682\\
37	-0.00949937675359384\\
38	-0.00968315196102582\\
39	-0.00878805905123908\\
40	-0.0101622797814708\\
41	-0.00943011647030966\\
42	-0.0105815501964088\\
43	-0.00924839997226945\\
44	-0.0126349593408432\\
45	-0.0145387736685121\\
46	-0.0162911195879469\\
47	-0.0158671375329857\\
48	-0.0163408161443202\\
49	-0.0158231312129781\\
50	-0.0163834027640212\\
51	-0.0145596654557966\\
52	-0.013952926409102\\
53	-0.0127309685126585\\
54	-0.0116158000590798\\
55	-0.0105447363458507\\
56	-0.0113912410000238\\
57	-0.0118507737985645\\
58	-0.0104640852708002\\
59	-0.0115698678065147\\
60	-0.0102437772754935\\
61	-0.0105208243289207\\
62	-0.0124150969623392\\
63	-0.0114221912315846\\
64	-0.012950827962606\\
65	-0.0126209406286529\\
66	-0.012655831687258\\
67	-0.0135223803108206\\
68	-0.0151222689249005\\
69	-0.0166168160586573\\
70	-0.0166136534977004\\
71	-0.0161873607166522\\
72	-0.0170783763987056\\
73	-0.016415899755664\\
74	-0.0166126531909298\\
75	-0.0171343232589627\\
76	-0.0174077688750338\\
77	-0.0171003929081494\\
78	-0.0164428278401388\\
79	-0.016096561694237\\
80	-0.015730326521686\\
81	-0.0157048011808638\\
82	-0.0154180632874871\\
83	-0.0161618126735737\\
84	-0.0161914530591213\\
85	-0.0161913265351336\\
86	-0.0145719033981334\\
87	-0.0137993492363755\\
88	-0.0134901468002803\\
89	-0.0121473089698196\\
90	-0.0126190769209853\\
91	-0.0122472642332065\\
92	-0.0122521872576056\\
93	-0.0115239550159016\\
94	-0.011024271897225\\
95	-0.0115581328016613\\
96	-0.0135682343564978\\
97	-0.0117177295089909\\
98	-0.0111467782881821\\
99	-0.0094619609433881\\
100	-0.00857121320448123\\
101	-0.00861101901387951\\
102	-0.00844041743118162\\
103	-0.00690406527986998\\
104	-0.00567746800497235\\
105	-0.00539271439880608\\
106	-0.00545875222001007\\
107	-0.00566178097163526\\
108	-0.00678237038338508\\
109	-0.00717677123932114\\
110	-0.00707623400693786\\
111	-0.00769698186650949\\
112	-0.00728747555025025\\
113	-0.00580924045325431\\
114	-0.00530532909549488\\
115	-0.00595221641226258\\
116	-0.00662448090821299\\
117	-0.00718621323865255\\
118	-0.00627502647062189\\
119	-0.00620894694628258\\
120	-0.00535858753755984\\
121	-0.00629630342151646\\
122	-0.00715156938961811\\
123	-0.00746964156294747\\
124	-0.00677371669881954\\
125	-0.0078227218075851\\
126	-0.00839355871880654\\
127	-0.0085631414933483\\
128	-0.00803251994310901\\
129	-0.00879076245725902\\
130	-0.00998798955573851\\
131	-0.0106558310414303\\
132	-0.00923909959159767\\
133	-0.00979633033241309\\
134	-0.00976688784120895\\
135	-0.00951670573563124\\
136	-0.0109582926623108\\
137	-0.0121578302283952\\
138	-0.0120622054675553\\
139	-0.0113234542997721\\
140	-0.0111645367410733\\
141	-0.00932003615830302\\
142	-0.00943275343441349\\
143	-0.00992148076029458\\
144	-0.01013351594138\\
145	-0.00834449744858081\\
146	-0.00779222448604345\\
147	-0.00771852041275663\\
148	-0.00719858032463106\\
149	-0.00702664219361796\\
150	-0.00784970758401263\\
151	-0.007404490560069\\
152	-0.00791382573277408\\
153	-0.00865010457726242\\
154	-0.00868314747257319\\
155	-0.00852220431124145\\
156	-0.0078770166614952\\
157	-0.00825822816668971\\
158	-0.00770189198381985\\
159	-0.00751649481876704\\
160	-0.00713119528708422\\
161	-0.00746015570959849\\
162	-0.00754380764555099\\
163	-0.00850424388541419\\
164	-0.0100389048360167\\
165	-0.0100207382254152\\
166	-0.00986045042227854\\
167	-0.00949226726226027\\
168	-0.00945523604935184\\
169	-0.0101009413894546\\
170	-0.0105355045764777\\
171	-0.0101130361624282\\
172	-0.0102373736617008\\
173	-0.00996409637028935\\
174	-0.0102638492920075\\
175	-0.0106007819051271\\
176	-0.00975177534153767\\
177	-0.00961958954821958\\
178	-0.00972649762512787\\
179	-0.00919185069685974\\
180	-0.00901041575576885\\
181	-0.00880276519263598\\
182	-0.00860507137799619\\
183	-0.0088288608206806\\
184	-0.00838368858682593\\
185	-0.00830882768253285\\
186	-0.00792600375020442\\
187	-0.00785455541337881\\
188	-0.00827376381722125\\
189	-0.00771721216384426\\
190	-0.00776200099409985\\
191	-0.00756302377165377\\
192	-0.00762529370593036\\
193	-0.00798104638821626\\
194	-0.00735444967400286\\
195	-0.00741653954525025\\
196	-0.00804612213389644\\
197	-0.00796114943152074\\
198	-0.00796185268322196\\
199	-0.00785460253951293\\
200	-0.00809885822106412\\
201	-0.00819978407934646\\
202	-0.007899045474195\\
203	-0.0080261143299322\\
204	-0.00802293063530897\\
205	-0.00821387736405633\\
206	-0.00800481333528631\\
207	-0.00772277403685232\\
208	-0.00734959107083133\\
209	-0.00758481540437084\\
210	-0.00722394589793882\\
211	-0.00720678403938501\\
212	-0.00764055791735359\\
213	-0.00758146990574229\\
214	-0.00754113559830671\\
215	-0.00783856496616171\\
216	-0.00793050020803213\\
217	-0.0075057731303251\\
218	-0.00724355638648211\\
219	-0.00754833845555902\\
220	-0.00770183064201271\\
221	-0.00796780634757544\\
222	-0.00793757086140123\\
223	-0.00764151005273407\\
224	-0.00717580959300404\\
225	-0.00714995826585954\\
226	-0.00683873125769482\\
227	-0.00682022786853055\\
228	-0.00687476844153018\\
229	-0.00712178019129921\\
230	-0.00710648220502076\\
231	-0.00742336212392725\\
232	-0.00764947515595813\\
233	-0.00769643435946464\\
234	-0.00758912227192723\\
235	-0.0075397919875916\\
236	-0.0073527702324487\\
237	-0.00695873754026029\\
238	-0.00693402176164402\\
239	-0.00701923914800897\\
240	-0.00640089463486466\\
241	-0.00616406343873898\\
242	-0.00586984424627459\\
243	-0.00576508137326088\\
244	-0.00550066142367854\\
245	-0.00558352764943279\\
246	-0.00585039467166453\\
247	-0.00554896692870342\\
248	-0.00549465652517486\\
249	-0.00552049958577378\\
250	-0.00553354447092271\\
251	-0.00533366839519772\\
252	-0.00487100517332648\\
253	-0.00466477822772527\\
254	-0.00463296809087696\\
255	-0.00467579638077025\\
256	-0.00451898832242474\\
257	-0.00485117392201042\\
258	-0.00464951848714616\\
259	-0.00482120996351521\\
260	-0.0047360314678994\\
261	-0.00481692043797195\\
262	-0.00458178490615567\\
263	-0.00453301818257793\\
264	-0.0044779741913085\\
265	-0.00453605082975972\\
266	-0.0047371436970792\\
267	-0.00482619957520008\\
268	-0.00502510712676776\\
269	-0.00491781790119706\\
270	-0.00485752931547848\\
271	-0.00467725175370806\\
272	-0.00454108302800611\\
273	-0.00450365899912407\\
274	-0.00466338919290653\\
275	-0.00450633643028268\\
276	-0.00453446578103818\\
277	-0.00435555850281689\\
278	-0.00406687433161675\\
279	-0.00390935092032186\\
280	-0.00389144926144939\\
281	-0.00392841568911585\\
282	-0.00383853533290053\\
283	-0.00387836409802384\\
284	-0.0039019703276885\\
285	-0.00390706780785984\\
286	-0.00410419918876257\\
287	-0.00405665162658647\\
288	-0.00427627334502044\\
289	-0.004308635703282\\
290	-0.00448370831825261\\
291	-0.00422572663332688\\
292	-0.00409837779958881\\
293	-0.00408603656539803\\
294	-0.00402919695191717\\
295	-0.00418082476203073\\
296	-0.00416538508243334\\
297	-0.00446063382986456\\
298	-0.00447568534603187\\
299	-0.00423557518707563\\
300	-0.00427431094971647\\
301	-0.00413145052587009\\
302	-0.00396170976429691\\
303	-0.00389068452193919\\
304	-0.00367914162909446\\
305	-0.00353656862597927\\
306	-0.00351359284538606\\
307	-0.00358832019351656\\
308	-0.00348350872214095\\
309	-0.00362133680716933\\
310	-0.00380193977367514\\
311	-0.00376663646496554\\
312	-0.00379202667112276\\
313	-0.00381164690998343\\
314	-0.00388141478802133\\
315	-0.00380432333947157\\
316	-0.00384034019654877\\
317	-0.0038185344119459\\
318	-0.00382261060899081\\
319	-0.00375834997412963\\
320	-0.0037841908644371\\
321	-0.00359559181155257\\
322	-0.00355248417655707\\
323	-0.00353952232817691\\
324	-0.00335202362762329\\
325	-0.00336961437574592\\
326	-0.00345562618452309\\
327	-0.00346144283675863\\
328	-0.00338329549696958\\
329	-0.0033494497888569\\
330	-0.00331680589716546\\
331	-0.00323972131719281\\
332	-0.00322928157757766\\
333	-0.00334490247048848\\
334	-0.00324446048051503\\
335	-0.00319245907051905\\
336	-0.00312210964454182\\
337	-0.00304492217258351\\
338	-0.00293028182304007\\
339	-0.00278049313254709\\
340	-0.00275798855012599\\
341	-0.00269528281081402\\
342	-0.00266932879838929\\
343	-0.00281788959027424\\
344	-0.00282245490076562\\
345	-0.0027217929331876\\
346	-0.00257619306341514\\
347	-0.00257329687534663\\
348	-0.00252904818866525\\
349	-0.00250225826528816\\
350	-0.00237924428323937\\
351	-0.00225249726062414\\
352	-0.00223595719692849\\
353	-0.00223094087562396\\
354	-0.00224055685314109\\
355	-0.00227814525468523\\
356	-0.00227957297495048\\
357	-0.00216419314397712\\
358	-0.00223119379216126\\
359	-0.00203311023953854\\
360	-0.001982959456927\\
361	-0.00196590956094437\\
362	-0.00194433108622492\\
363	-0.00195797351092188\\
364	-0.00199106419462269\\
365	-0.00201649780714848\\
366	-0.0019614636413649\\
367	-0.00203222023818977\\
368	-0.00205074762349903\\
369	-0.00202433780533571\\
370	-0.00200544185743671\\
371	-0.00198238536661594\\
372	-0.00191584404634635\\
373	-0.00190542832066806\\
374	-0.00189845407774061\\
375	-0.00187303927692364\\
376	-0.0018474848392225\\
377	-0.00182836188352722\\
378	-0.00178715790201682\\
379	-0.00186084767428253\\
380	-0.0018247960241858\\
381	-0.00180048958492816\\
382	-0.00179725808148503\\
383	-0.00175491491720112\\
384	-0.00167494402314376\\
385	-0.00159488993385771\\
386	-0.00158355573299134\\
387	-0.00158043101126484\\
388	-0.00145525842370464\\
389	-0.00146457814656282\\
390	-0.00139946131937141\\
391	-0.00139635143786248\\
392	-0.00131621963394453\\
393	-0.00133257221671527\\
394	-0.00125295374468468\\
395	-0.00121954897027075\\
396	-0.0012456093447772\\
397	-0.00127286620958312\\
398	-0.00132134860767571\\
399	-0.00124071664085907\\
400	-0.00120720303639896\\
401	-0.00119134629868454\\
402	-0.0011515521973159\\
403	-0.0011449595688092\\
404	-0.00111457433534147\\
405	-0.0010878666026018\\
406	-0.00112628918712903\\
407	-0.00120884540631137\\
408	-0.00118798598464774\\
409	-0.00122904989959303\\
410	-0.00116456202078929\\
411	-0.00124249627139724\\
412	-0.00121059743845204\\
413	-0.00116650292058238\\
414	-0.00118922584760948\\
415	-0.00115969625418752\\
416	-0.00117080369468062\\
417	-0.00119094010884278\\
418	-0.00121344968431993\\
419	-0.00112965784156034\\
420	-0.00115768563045275\\
421	-0.00113193293372635\\
422	-0.00114487120591939\\
423	-0.00120662254409491\\
424	-0.00117314272075785\\
425	-0.00113381682473352\\
426	-0.00109816756638205\\
427	-0.00107517785763731\\
428	-0.00101740819931183\\
429	-0.000985417494065118\\
430	-0.000955813865531279\\
431	-0.000940651598850408\\
432	-0.000985236102228883\\
433	-0.00105496288861742\\
434	-0.00103660711852526\\
435	-0.00102651677702821\\
436	-0.00102005103377131\\
437	-0.00100078289460576\\
438	-0.0010248018545951\\
439	-0.00104332530474378\\
440	-0.00106135421908987\\
441	-0.0010032826525025\\
442	-0.000945534918009458\\
443	-0.000936458870996285\\
444	-0.000971255090811645\\
445	-0.000941559202946419\\
446	-0.00093415902587528\\
447	-0.000946934402749471\\
448	-0.00094120458389668\\
449	-0.000964958495919891\\
450	-0.000902529827413628\\
451	-0.000935356244725709\\
452	-0.000944671176995431\\
453	-0.000901063702576273\\
454	-0.000882628564341676\\
455	-0.000904248201497928\\
456	-0.00091067669330158\\
457	-0.000931059203265639\\
458	-0.000927062016159372\\
459	-0.000916157032848509\\
460	-0.000892154391409029\\
461	-0.000850943620174995\\
462	-0.000806528235645325\\
463	-0.000789580724554505\\
464	-0.000764813067912276\\
465	-0.000780318855436802\\
466	-0.000786828378744673\\
467	-0.000748258734752507\\
468	-0.00073887919678757\\
469	-0.000738809398495592\\
470	-0.000717640425656308\\
471	-0.000697274142541514\\
472	-0.000627157625520886\\
473	-0.000594993580963973\\
474	-0.000618853826316008\\
475	-0.00061083662259854\\
476	-0.000619909739466929\\
477	-0.000609124830080353\\
478	-0.000629823566621177\\
479	-0.000584619409789389\\
480	-0.000591943904572733\\
481	-0.000577545360728728\\
482	-0.00055518958432003\\
483	-0.000554859477331749\\
484	-0.000560076073985153\\
485	-0.000546134870516576\\
486	-0.000542811683623858\\
487	-0.000530266380022177\\
488	-0.000556586748131027\\
489	-0.000568335534665846\\
490	-0.000563327729775743\\
491	-0.000560441771159412\\
492	-0.000568301557472884\\
493	-0.000566677953415283\\
494	-0.000554922588729296\\
495	-0.000559255199316486\\
496	-0.000545724553246362\\
497	-0.000557289450998629\\
498	-0.000559554709613787\\
499	-0.00054040965359444\\
500	-0.000534891074601973\\
501	-0.000553740567865577\\
502	-0.00054331404370106\\
503	-0.000539174471257099\\
504	-0.00053949493623281\\
505	-0.000542283027456252\\
506	-0.000547878070726508\\
507	-0.000527404701653592\\
508	-0.000520175208884849\\
509	-0.00053787686748936\\
510	-0.000544931569560258\\
511	-0.000542635830164207\\
512	-0.000534458183907116\\
513	-0.000509540704645053\\
514	-0.000489445875170595\\
515	-0.000508078950994268\\
516	-0.000513645245603313\\
517	-0.000510224054761574\\
518	-0.000508039588668471\\
519	-0.000510781754320413\\
520	-0.00050654628339713\\
521	-0.000512217695121393\\
522	-0.000510814489874724\\
523	-0.000503235257070034\\
524	-0.000491893681285418\\
525	-0.000496541170517301\\
526	-0.000491733650878732\\
527	-0.000470365675949199\\
528	-0.000457300813999154\\
529	-0.00043075512334046\\
530	-0.000429053307350943\\
531	-0.000428331808820363\\
532	-0.000402606421981797\\
533	-0.000405848387321868\\
534	-0.00041280468716814\\
535	-0.000405595920887453\\
536	-0.000396473560935356\\
537	-0.000408429090321235\\
538	-0.000395358011232224\\
539	-0.000389137663236526\\
540	-0.000384604482942142\\
541	-0.000378760026919598\\
542	-0.000374283987684506\\
543	-0.000386337068245779\\
544	-0.000388158378912208\\
545	-0.000388939233120004\\
546	-0.000375314393784892\\
547	-0.000353803123346242\\
548	-0.000343693400682875\\
549	-0.000343622237500644\\
550	-0.00034469524922422\\
551	-0.000345501785362795\\
552	-0.000326943499132093\\
553	-0.000316079261532478\\
554	-0.000308596983105854\\
555	-0.000304057470365673\\
556	-0.000304522319569477\\
557	-0.00030413057476312\\
558	-0.000303101558354013\\
559	-0.000302950640137659\\
560	-0.00029517254977539\\
561	-0.000276519501612659\\
562	-0.000273072666335065\\
563	-0.000271418979650205\\
564	-0.000277397698592543\\
565	-0.000274608158762678\\
566	-0.000262388124354393\\
567	-0.00026289668723241\\
568	-0.000267052991614634\\
569	-0.000249790618599606\\
570	-0.00023377666319423\\
571	-0.000241433017485512\\
572	-0.000241811903226839\\
573	-0.000219647241590395\\
574	-0.000218187197353616\\
575	-0.000222552763314143\\
576	-0.000226562543981072\\
577	-0.000232786590014997\\
578	-0.000223063272962311\\
579	-0.000207273603219231\\
580	-0.000208777065551178\\
581	-0.000216465859969796\\
582	-0.00022361429579938\\
583	-0.000213380303912004\\
584	-0.000206611685094751\\
585	-0.000207343910648893\\
586	-0.000195351271937816\\
587	-0.000194655776409358\\
588	-0.000184846300604593\\
589	-0.000181579950363747\\
590	-0.000183901375304725\\
591	-0.000177165179010706\\
592	-0.000175986123142326\\
593	-0.000175313720452277\\
594	-0.000178554155445797\\
595	-0.000180804333707393\\
596	-0.000177026743159248\\
597	-0.00018041675117122\\
598	-0.000184444899494178\\
599	-0.000178426382208125\\
600	-0.000171788038430054\\
601	-0.000183946702969921\\
602	-0.000185097606963949\\
603	-0.000193301853850538\\
604	-0.000181734866385522\\
605	-0.000176289914436236\\
606	-0.000176846321244359\\
607	-0.00017149752519624\\
608	-0.000176468684240108\\
609	-0.000175735599971419\\
610	-0.000178119892441272\\
611	-0.000169586113262601\\
612	-0.000166030786728796\\
613	-0.000169323554988705\\
614	-0.000177348205724863\\
615	-0.000171877522847746\\
616	-0.000166789581738709\\
617	-0.00016073182206414\\
618	-0.000144944416735894\\
619	-0.000139043301505445\\
620	-0.000141840735821237\\
621	-0.000143820248347827\\
622	-0.000146576320108415\\
623	-0.000132312661992166\\
624	-0.000120747623591477\\
625	-0.000123236111908465\\
626	-0.000124588110258045\\
627	-0.000126481427404664\\
628	-0.000122010999352538\\
629	-0.000128280864901957\\
630	-0.000127887167530095\\
631	-0.000129371324255317\\
632	-0.000128155815729764\\
633	-0.000120210782184853\\
634	-0.000117807883432565\\
635	-0.000116262425324539\\
636	-0.00011342799986056\\
637	-0.000113682701595768\\
638	-0.000109381717945348\\
639	-0.000103574765671877\\
640	-9.84966271025919e-05\\
641	-9.41384646328144e-05\\
642	-9.99723460655376e-05\\
643	-0.000101755061885383\\
644	-0.000102382846032499\\
645	-9.95674568448256e-05\\
646	-9.98194995271794e-05\\
647	-9.72941786609743e-05\\
648	-9.41434419197159e-05\\
649	-9.91769477941342e-05\\
650	-9.64975672771304e-05\\
651	-9.50363444238483e-05\\
652	-9.95963474552509e-05\\
653	-9.56836719896812e-05\\
654	-9.60689829823485e-05\\
655	-9.39249883543857e-05\\
656	-8.61896386521115e-05\\
657	-8.87319725793084e-05\\
658	-8.84763394486578e-05\\
659	-8.35961074353412e-05\\
660	-8.01533729664749e-05\\
661	-7.47306917764569e-05\\
662	-7.02986660791775e-05\\
663	-7.0148457368042e-05\\
664	-6.87054267886314e-05\\
665	-6.38422547844887e-05\\
666	-6.34089415069432e-05\\
667	-5.83716887061933e-05\\
668	-5.5244783574688e-05\\
669	-5.35529322749266e-05\\
670	-6.20846839750677e-05\\
671	-6.19465344182586e-05\\
672	-5.81517356620769e-05\\
673	-5.82751029934305e-05\\
674	-5.5204764370644e-05\\
675	-5.68728396283679e-05\\
676	-5.57299990894539e-05\\
677	-5.44383921542692e-05\\
678	-5.50168330647812e-05\\
679	-5.81809815952572e-05\\
680	-5.37882500900117e-05\\
681	-5.85959702614378e-05\\
682	-6.43161516319172e-05\\
683	-5.9873580818527e-05\\
684	-6.24924837913806e-05\\
685	-5.59366632025191e-05\\
686	-5.12748907160314e-05\\
687	-4.53979788519837e-05\\
688	-3.97981765873441e-05\\
689	-3.85961734535901e-05\\
690	-3.50474456833324e-05\\
691	-3.45356930510185e-05\\
692	-3.60683976878188e-05\\
693	-3.52139859615052e-05\\
694	-3.71379459286182e-05\\
695	-3.55819699600792e-05\\
696	-3.57202175788448e-05\\
697	-3.1874368433681e-05\\
698	-3.26115887548758e-05\\
699	-3.37234883753707e-05\\
700	-3.45079293434951e-05\\
701	-3.40272617595515e-05\\
702	-3.30776287225464e-05\\
703	-3.13193890687485e-05\\
704	-3.15189735762079e-05\\
705	-2.98119271595486e-05\\
706	-2.96353895311998e-05\\
707	-2.67510205946884e-05\\
708	-3.12863168776484e-05\\
709	-3.39328685548904e-05\\
710	-3.36903066411133e-05\\
711	-3.4828744232045e-05\\
712	-3.71955723205436e-05\\
713	-3.62319281156899e-05\\
714	-3.61013285928282e-05\\
715	-3.2698195287574e-05\\
716	-3.08720021484135e-05\\
717	-3.29683931690894e-05\\
718	-3.22383351791836e-05\\
719	-3.11881603062359e-05\\
720	-2.92638582880345e-05\\
721	-3.14395397575935e-05\\
722	-2.88103691441671e-05\\
723	-2.59387315144508e-05\\
724	-2.86231354900014e-05\\
725	-2.8082945828957e-05\\
726	-2.77726564220538e-05\\
727	-2.62262304361325e-05\\
728	-2.67160199504699e-05\\
729	-2.59018047210053e-05\\
730	-3.03097297911295e-05\\
731	-2.92343077291249e-05\\
732	-3.03072100518304e-05\\
733	-3.19096749783793e-05\\
734	-3.37904356205587e-05\\
735	-3.655145303759e-05\\
736	-3.81065170553452e-05\\
737	-3.68974178659943e-05\\
738	-3.67439774972988e-05\\
739	-3.62519930315415e-05\\
740	-3.46108364074532e-05\\
741	-3.44633107731449e-05\\
742	-3.44619795225686e-05\\
743	-3.58759187055231e-05\\
744	-3.56745814090577e-05\\
745	-3.57875865764767e-05\\
746	-3.6780941612772e-05\\
747	-3.9291957190657e-05\\
748	-4.00862916056039e-05\\
749	-3.61820411587446e-05\\
750	-3.49187279989017e-05\\
751	-3.47324241627458e-05\\
752	-3.34658998504793e-05\\
753	-3.43988818558868e-05\\
754	-3.47833967212866e-05\\
755	-3.31110914146639e-05\\
756	-3.12872492066838e-05\\
757	-2.89896241525878e-05\\
758	-2.82819565384178e-05\\
759	-2.74297777879345e-05\\
760	-2.75491312789134e-05\\
761	-2.62103696016568e-05\\
762	-2.66486940068593e-05\\
763	-2.70731431512752e-05\\
764	-2.88046465110316e-05\\
765	-2.7090152204892e-05\\
766	-2.36961807839799e-05\\
767	-2.324154842911e-05\\
768	-2.46950048282011e-05\\
769	-2.38718691258069e-05\\
770	-2.25956368524664e-05\\
771	-2.43469892463326e-05\\
772	-2.49712938708428e-05\\
773	-2.60952508774318e-05\\
774	-2.4139353411268e-05\\
775	-2.38113773863748e-05\\
776	-2.40541506774661e-05\\
777	-2.46625437772442e-05\\
778	-2.54645316469634e-05\\
779	-2.53023148256944e-05\\
780	-2.46664197407544e-05\\
781	-2.43596962080597e-05\\
782	-2.46408596349542e-05\\
783	-2.25482423354973e-05\\
784	-2.02357825080604e-05\\
785	-1.88866652035708e-05\\
786	-1.87362094679876e-05\\
787	-1.78917430739776e-05\\
788	-1.73545491682849e-05\\
789	-1.59001063237695e-05\\
790	-1.57498276023326e-05\\
791	-1.50886722697902e-05\\
792	-1.48815627311633e-05\\
793	-1.49117763087902e-05\\
794	-1.5257493290132e-05\\
795	-1.48189595166494e-05\\
796	-1.62724252800423e-05\\
797	-1.59245347100713e-05\\
798	-1.75210755397577e-05\\
799	-1.69423690395784e-05\\
800	-1.55683244987088e-05\\
801	-1.73049543016846e-05\\
802	-1.60384335375546e-05\\
803	-1.69460538661734e-05\\
804	-1.58225486369324e-05\\
805	-1.74627459217578e-05\\
806	-1.8341659083363e-05\\
807	-1.89177133407322e-05\\
808	-1.94721009107308e-05\\
809	-1.91819854473957e-05\\
810	-1.88422931708538e-05\\
811	-1.97300916957144e-05\\
812	-1.91752513694942e-05\\
813	-1.95569323208852e-05\\
814	-2.00539710602418e-05\\
815	-1.99283795327786e-05\\
816	-1.87074951002573e-05\\
817	-1.77153398927807e-05\\
818	-1.656495402623e-05\\
819	-1.57076168465895e-05\\
820	-1.58900945637508e-05\\
821	-1.53517473376602e-05\\
822	-1.52675989897061e-05\\
823	-1.6289181976852e-05\\
824	-1.64088184989612e-05\\
825	-1.70100040515693e-05\\
826	-1.72629998480123e-05\\
827	-1.66794182806361e-05\\
828	-1.68189277147762e-05\\
829	-1.62166885509413e-05\\
830	-1.63447531400425e-05\\
831	-1.64071676427018e-05\\
832	-1.65309794289528e-05\\
833	-1.6369441340748e-05\\
834	-1.66086697954052e-05\\
835	-1.6183924349358e-05\\
836	-1.60502292431633e-05\\
837	-1.55001088763028e-05\\
838	-1.52460727743133e-05\\
839	-1.49898035949032e-05\\
840	-1.54961319204684e-05\\
841	-1.55570692545462e-05\\
842	-1.62260503676706e-05\\
843	-1.59619501061057e-05\\
844	-1.72073905661196e-05\\
845	-1.70445006285394e-05\\
846	-1.73853306558625e-05\\
847	-1.63987261537131e-05\\
848	-1.61557143773357e-05\\
849	-1.62712320445658e-05\\
850	-1.65519994333002e-05\\
851	-1.6231038054212e-05\\
852	-1.61474766248814e-05\\
853	-1.64310190466562e-05\\
854	-1.56551099088614e-05\\
855	-1.46889257766273e-05\\
856	-1.46009078596733e-05\\
857	-1.48127632549396e-05\\
858	-1.41191776493295e-05\\
859	-1.36181990171575e-05\\
860	-1.35721847552022e-05\\
861	-1.33931089863436e-05\\
862	-1.33706973405076e-05\\
863	-1.28976210263379e-05\\
864	-1.22063610765062e-05\\
865	-1.28153071197637e-05\\
866	-1.2847797288891e-05\\
867	-1.27845080944036e-05\\
868	-1.34479728003739e-05\\
869	-1.26949216667521e-05\\
870	-1.18207836348954e-05\\
871	-1.1241788975065e-05\\
872	-1.16990281848017e-05\\
873	-1.16111094826529e-05\\
874	-1.19301093835816e-05\\
875	-1.15038435367301e-05\\
876	-1.10278513875211e-05\\
877	-1.08814311344949e-05\\
878	-1.12931884357554e-05\\
879	-1.08931874461509e-05\\
880	-1.07420067078345e-05\\
881	-1.13593204364594e-05\\
882	-1.10067685232523e-05\\
883	-1.11105234264944e-05\\
884	-1.06323662344193e-05\\
885	-1.07986796608444e-05\\
886	-1.07968402295701e-05\\
887	-1.0735743109723e-05\\
888	-1.08480699535704e-05\\
889	-1.14941513225009e-05\\
890	-1.14164345220105e-05\\
891	-1.16932261299102e-05\\
892	-1.17161061861746e-05\\
893	-1.14436584377315e-05\\
894	-1.1176132293456e-05\\
895	-1.17129285392673e-05\\
896	-1.15209909537366e-05\\
897	-1.08843844070976e-05\\
898	-1.16301545564325e-05\\
899	-1.16819326304391e-05\\
900	-1.2404494544525e-05\\
901	-1.24103311677425e-05\\
902	-1.2947912430599e-05\\
903	-1.23892040920468e-05\\
904	-1.22124860991121e-05\\
905	-1.18397514500511e-05\\
906	-1.16058332859205e-05\\
907	-1.14245220168343e-05\\
908	-1.08377053603523e-05\\
909	-1.12400236818681e-05\\
910	-1.08525547304659e-05\\
911	-1.05680577694713e-05\\
912	-1.00349591783941e-05\\
913	-9.86378956829101e-06\\
914	-9.39325811055653e-06\\
915	-9.0391951754951e-06\\
916	-9.05688820843725e-06\\
917	-9.22277563819153e-06\\
918	-8.85103592946141e-06\\
919	-8.96057695941301e-06\\
920	-8.85937488304833e-06\\
921	-8.45742263167842e-06\\
922	-8.0952937722226e-06\\
923	-8.01779887271893e-06\\
924	-8.4865581528846e-06\\
925	-8.30047627960804e-06\\
926	-7.86687793105717e-06\\
927	-7.58067842984596e-06\\
928	-7.83786534414323e-06\\
929	-7.81919028899036e-06\\
930	-7.57442724864526e-06\\
931	-7.27693631320393e-06\\
932	-7.04472768494137e-06\\
933	-6.7703247522233e-06\\
934	-6.22923146941611e-06\\
935	-6.52967799210705e-06\\
936	-6.44519523074276e-06\\
937	-6.38374211337729e-06\\
938	-6.22229952889073e-06\\
939	-5.92725099573598e-06\\
940	-5.97145445483339e-06\\
941	-6.01485633871448e-06\\
942	-5.99065192024491e-06\\
943	-5.7681860448281e-06\\
944	-5.83608997501426e-06\\
945	-6.10225420101763e-06\\
946	-5.84546807670061e-06\\
947	-5.91026309955658e-06\\
948	-5.75499714635352e-06\\
949	-5.71419035565375e-06\\
950	-5.52164050678313e-06\\
951	-5.76988929664523e-06\\
952	-5.3925219057342e-06\\
953	-5.5737420269564e-06\\
954	-5.26812436098542e-06\\
955	-5.28959592021262e-06\\
956	-5.4341770333226e-06\\
957	-5.81868647402198e-06\\
958	-5.79672463354514e-06\\
959	-5.77422605340704e-06\\
960	-5.32991943454607e-06\\
961	-5.24956569187929e-06\\
962	-5.28978145529536e-06\\
963	-5.18918201439896e-06\\
964	-5.12759752879942e-06\\
965	-5.09554104397827e-06\\
966	-5.03527997343053e-06\\
967	-5.13566969364845e-06\\
968	-4.9197614822402e-06\\
969	-4.87200393150992e-06\\
970	-4.91664858684283e-06\\
971	-4.92026716272818e-06\\
972	-4.95146994520305e-06\\
973	-4.70601526329638e-06\\
974	-4.65858335313569e-06\\
975	-4.62002171745818e-06\\
976	-4.92553070412083e-06\\
977	-4.88133631014679e-06\\
978	-4.71680059424042e-06\\
979	-4.55473471498608e-06\\
980	-4.78451503477797e-06\\
981	-4.82687599032629e-06\\
982	-4.891582764738e-06\\
983	-4.92855333314528e-06\\
984	-4.9098071394759e-06\\
985	-4.88785804900911e-06\\
986	-4.84240691372558e-06\\
987	-4.72752396116738e-06\\
988	-4.40454311670153e-06\\
989	-4.26001212961364e-06\\
990	-4.24826759280642e-06\\
991	-4.3380541494862e-06\\
992	-4.20191207986382e-06\\
993	-4.00014360534953e-06\\
994	-3.97673540360037e-06\\
995	-3.89446398087438e-06\\
996	-3.90463222963933e-06\\
997	-3.93677035531307e-06\\
998	-3.99025162776992e-06\\
999	-4.03373181137961e-06\\
1000	-3.89371711603832e-06\\
};
\addlegendentry{$\Im \{\mathbf{h}_1^H \}$};

\addplot [color=mycolor3,solid]
  table[row sep=crcr]{%
1	0\\
2	0\\
3	0\\
4	-3.63656045362114e-06\\
5	-0.00053472529022547\\
6	-0.00109050048203345\\
7	-0.0060249467016215\\
8	-0.0113026052465025\\
9	-0.0132800502559847\\
10	-0.0184957853796606\\
11	-0.0170260684944852\\
12	-0.0123706640384226\\
13	-0.0155428219857504\\
14	-0.0177279010868541\\
15	-0.0165460300517043\\
16	-0.0149468884520541\\
17	-0.0157254983978997\\
18	-0.0158992759681355\\
19	-0.0157136639555116\\
20	-0.0174415854190358\\
21	-0.0201995088869602\\
22	-0.0215019010212129\\
23	-0.0224794876164566\\
24	-0.0263864400476732\\
25	-0.0280838388347377\\
26	-0.0261057922387889\\
27	-0.0264897242963343\\
28	-0.0266747012470158\\
29	-0.0255086637400682\\
30	-0.0205257710194818\\
31	-0.019980125249873\\
32	-0.0209826253587736\\
33	-0.0221200631830068\\
34	-0.0276422113975377\\
35	-0.0305618398735611\\
36	-0.0285496599983624\\
37	-0.0291305433496944\\
38	-0.0266871604629706\\
39	-0.0260563964685031\\
40	-0.0287703752495175\\
41	-0.0274982466903479\\
42	-0.0279991856550728\\
43	-0.0275303445253713\\
44	-0.0288692843629897\\
45	-0.0267743199142479\\
46	-0.0263267543263156\\
47	-0.0268532846252729\\
48	-0.0223024231668033\\
49	-0.0233958005578373\\
50	-0.0226304908035969\\
51	-0.0220019942346091\\
52	-0.0214049541076323\\
53	-0.0183722862699727\\
54	-0.0172934134435934\\
55	-0.0149234948927663\\
56	-0.0133684121798521\\
57	-0.01337617116777\\
58	-0.0132184408412127\\
59	-0.012262478895853\\
60	-0.0127831461825246\\
61	-0.0131839867044571\\
62	-0.0132947051533338\\
63	-0.0131806213456104\\
64	-0.0130544375187527\\
65	-0.0137728392031428\\
66	-0.0130108468564009\\
67	-0.0141309690164693\\
68	-0.0150087685146784\\
69	-0.0144797617019339\\
70	-0.0142922871739422\\
71	-0.0142535431267772\\
72	-0.0145563487398782\\
73	-0.0158008006547377\\
74	-0.0165662586291654\\
75	-0.0163687473940004\\
76	-0.0159513306776761\\
77	-0.0152341409244249\\
78	-0.0136143425734603\\
79	-0.0121679346419752\\
80	-0.0122033179444699\\
81	-0.0132880822077575\\
82	-0.0135447430803539\\
83	-0.0142086019331019\\
84	-0.0140846394347488\\
85	-0.0145396460086603\\
86	-0.0123866183375343\\
87	-0.0126378889614398\\
88	-0.0142228475503811\\
89	-0.0122390487420186\\
90	-0.0123320429596798\\
91	-0.0107616175431094\\
92	-0.0106627507329809\\
93	-0.0111559666260943\\
94	-0.0107128589737556\\
95	-0.00921305507477829\\
96	-0.00908714215244063\\
97	-0.00679359195404469\\
98	-0.00585776278671947\\
99	-0.00521760495833847\\
100	-0.00502133052613388\\
101	-0.00578303688441465\\
102	-0.00528018816055269\\
103	-0.00278589199514705\\
104	-0.00227886909856542\\
105	-0.00395419165205751\\
106	-0.00415997501854267\\
107	-0.00336792064512492\\
108	-0.00467129742720674\\
109	-0.00562822599404101\\
110	-0.00625882498463956\\
111	-0.00583843653559724\\
112	-0.00750498320718778\\
113	-0.00686212730954477\\
114	-0.00492128423488753\\
115	-0.00468872655423816\\
116	-0.00467377029286452\\
117	-0.00384518160886037\\
118	-0.00263968238705271\\
119	-0.00234125913537739\\
120	-0.00228802794726492\\
121	-0.00238764128169803\\
122	-0.00198197942514971\\
123	-0.00264076708706303\\
124	-0.00174111270858539\\
125	-0.002081730262982\\
126	-0.00179132223918039\\
127	-0.00232571201079361\\
128	-0.00141729030954571\\
129	-0.00119206442864988\\
130	-0.00162696385344746\\
131	-0.00113146350955562\\
132	0.000211807884972825\\
133	0.000429873209508144\\
134	0.000219200479479412\\
135	0.000856272905193551\\
136	0.00104584455394959\\
137	0.000555376872086347\\
138	0.00117736855132409\\
139	0.00163937108703909\\
140	0.00209534669199029\\
141	0.00224421604783633\\
142	0.00178654898852829\\
143	0.00264176910405115\\
144	0.00193902842325218\\
145	0.00301133943983628\\
146	0.00344499053189572\\
147	0.00313872761231559\\
148	0.00356905734523874\\
149	0.00359653558332043\\
150	0.00333087797376578\\
151	0.00323464728055454\\
152	0.00216229826443857\\
153	0.00209098064696788\\
154	0.00245426260748778\\
155	0.00263794603076935\\
156	0.00180738047272604\\
157	0.00160535530425685\\
158	0.000721447315164822\\
159	0.00062043576088815\\
160	3.51125526770003e-05\\
161	0.000548909926191677\\
162	0.000299030207071416\\
163	-0.000870907604600387\\
164	-0.00163627289027757\\
165	-0.00138851401231888\\
166	-0.00177838686837852\\
167	-0.00228319813690094\\
168	-0.00268188339014873\\
169	-0.00361192866338864\\
170	-0.00391075631796245\\
171	-0.00399896655011049\\
172	-0.00454668650324812\\
173	-0.00463231249239813\\
174	-0.00495744501961064\\
175	-0.00554863628844308\\
176	-0.00540735388387911\\
177	-0.0051784049294068\\
178	-0.00518990878889366\\
179	-0.0052904582435678\\
180	-0.00520625843948076\\
181	-0.00491680625010658\\
182	-0.00486541068613819\\
183	-0.00497014487254696\\
184	-0.00480099664784425\\
185	-0.00449790413820804\\
186	-0.00466412505964573\\
187	-0.00412910221679724\\
188	-0.00469745517856443\\
189	-0.00456523206800618\\
190	-0.00481250444877548\\
191	-0.00456439349520786\\
192	-0.00499510865281757\\
193	-0.00524206054239796\\
194	-0.00464242281679939\\
195	-0.00465401712946409\\
196	-0.00504519709417018\\
197	-0.00553192116592374\\
198	-0.00558763313820676\\
199	-0.0053162719911503\\
200	-0.00533485611875919\\
201	-0.00549083526305203\\
202	-0.00538408895279916\\
203	-0.00567277848761709\\
204	-0.00622922774416261\\
205	-0.00619436580652059\\
206	-0.00587661824471925\\
207	-0.00550098856554281\\
208	-0.00554562482228548\\
209	-0.00579541820596652\\
210	-0.00552734562895753\\
211	-0.00562575009075597\\
212	-0.00515465485476151\\
213	-0.00488839100312574\\
214	-0.00486235688138091\\
215	-0.00493199020116716\\
216	-0.00465894928082369\\
217	-0.00453046770695991\\
218	-0.00406712014152619\\
219	-0.00446035359553557\\
220	-0.00423164792768238\\
221	-0.00424691090530161\\
222	-0.0040133209205563\\
223	-0.00395511154031243\\
224	-0.00412120739718299\\
225	-0.00417010214939822\\
226	-0.00436807011795174\\
227	-0.00468639211697963\\
228	-0.00493470284803588\\
229	-0.00522557256351432\\
230	-0.00545924218655785\\
231	-0.00569441540380224\\
232	-0.00588966623766976\\
233	-0.00590309552052637\\
234	-0.00606442326104896\\
235	-0.00626711825749761\\
236	-0.00619274837861866\\
237	-0.00558389961739035\\
238	-0.00567463087766096\\
239	-0.00558748505172588\\
240	-0.00505573098247627\\
241	-0.00475227002805898\\
242	-0.00434720372197892\\
243	-0.0040761038874041\\
244	-0.0038223322723994\\
245	-0.00388496954286966\\
246	-0.00386410522400585\\
247	-0.00327606696930831\\
248	-0.00304616726731578\\
249	-0.00300268805961638\\
250	-0.00259610952619219\\
251	-0.0026046282378966\\
252	-0.00216476238658616\\
253	-0.00239662288213195\\
254	-0.00273720921189328\\
255	-0.00279612858349284\\
256	-0.00291201111086737\\
257	-0.00314983314134484\\
258	-0.0027770770392154\\
259	-0.00299465646026802\\
260	-0.00278512543931008\\
261	-0.00248651609009586\\
262	-0.00237148344948122\\
263	-0.00239304029435992\\
264	-0.00230151543669203\\
265	-0.00230812444702507\\
266	-0.00246961697543067\\
267	-0.00267664608689852\\
268	-0.00294168883262853\\
269	-0.00307806085273504\\
270	-0.00320990376992182\\
271	-0.00316061294356748\\
272	-0.00316902635274227\\
273	-0.00289832385959273\\
274	-0.00262454538512809\\
275	-0.00247345301385163\\
276	-0.00230184077727172\\
277	-0.0022002549316594\\
278	-0.00216097326718251\\
279	-0.00202374199598453\\
280	-0.00178312936818284\\
281	-0.00186160747867702\\
282	-0.00209210068837718\\
283	-0.00228079828722884\\
284	-0.00221351404606806\\
285	-0.00235088727955367\\
286	-0.00220917970373541\\
287	-0.00234777583116216\\
288	-0.00234861086965127\\
289	-0.0021692492496971\\
290	-0.00258760935035649\\
291	-0.00243700331566782\\
292	-0.00230241514824987\\
293	-0.00241725230960658\\
294	-0.00230078427886622\\
295	-0.00210217091377441\\
296	-0.00205184797241076\\
297	-0.00215510875183733\\
298	-0.00213621207821087\\
299	-0.00195943113010434\\
300	-0.00213047464759842\\
301	-0.00221808716715132\\
302	-0.00214364829662119\\
303	-0.00202749999418459\\
304	-0.00205283129898866\\
305	-0.00200639168081684\\
306	-0.0019833374310645\\
307	-0.00176904355459219\\
308	-0.00147624112169719\\
309	-0.00139575757974922\\
310	-0.00141849478029794\\
311	-0.0013872533576192\\
312	-0.00141380796026547\\
313	-0.00120595705681375\\
314	-0.00125255652080295\\
315	-0.000975553854822561\\
316	-0.00109354009250354\\
317	-0.00119155954013461\\
318	-0.00102292499381101\\
319	-0.00107179729497948\\
320	-0.000977172483298581\\
321	-0.00104828470592314\\
322	-0.00109232358637747\\
323	-0.00107298422829377\\
324	-0.000955706670368506\\
325	-0.000992581258371239\\
326	-0.000992491642915601\\
327	-0.00103321011730429\\
328	-0.000962290140525359\\
329	-0.000961856209771546\\
330	-0.00106919198743542\\
331	-0.00120876000339686\\
332	-0.00117078738283814\\
333	-0.00118222229825065\\
334	-0.00111031219708011\\
335	-0.00113864388415933\\
336	-0.00104997594642393\\
337	-0.00098160415611189\\
338	-0.00108853061887687\\
339	-0.00101016158554337\\
340	-0.0010953845715588\\
341	-0.00104891227377059\\
342	-0.00103620296273946\\
343	-0.00126184354401609\\
344	-0.00134585512906362\\
345	-0.00126205744635215\\
346	-0.00117366812942284\\
347	-0.00106423949914305\\
348	-0.0010236819633223\\
349	-0.000932295463178604\\
350	-0.000835479683094912\\
351	-0.000823402490351958\\
352	-0.0008676713088423\\
353	-0.000911736398239925\\
354	-0.000917365842910812\\
355	-0.000689722033473792\\
356	-0.000570198357538802\\
357	-0.000376049348113665\\
358	-0.000205845529915808\\
359	-0.00016342838454999\\
360	-0.000171193576716295\\
361	-0.000152879877982882\\
362	-0.000214658360599929\\
363	-0.000167727798394012\\
364	-0.000208961820231509\\
365	-0.000252545477271428\\
366	-0.000297413729226789\\
367	-0.00036462234504389\\
368	-0.000345068234863279\\
369	-0.000404504881068741\\
370	-0.000444558915911715\\
371	-0.000461264140630725\\
372	-0.000452452688812128\\
373	-0.000541390236881228\\
374	-0.000478196430360292\\
375	-0.000330958675546327\\
376	-0.000289903892384836\\
377	-0.000219806155520914\\
378	-0.00016112465427776\\
379	-0.000143126148077441\\
380	-0.000201563144920474\\
381	-0.000154671824426508\\
382	-0.000148301488601363\\
383	-0.0001881329159054\\
384	-0.000164094450893335\\
385	-0.000128518659470253\\
386	-0.000245761349604009\\
387	-0.000209236142761233\\
388	-0.000135991316339597\\
389	-0.000216243147548565\\
390	-0.000220946544135808\\
391	-0.00024197063243697\\
392	-0.000260639239595598\\
393	-0.000180773362133107\\
394	-6.25315389858533e-05\\
395	-7.32662593056699e-05\\
396	-7.24825984559345e-05\\
397	-6.05169394242124e-05\\
398	-4.76024216097906e-05\\
399	-2.33684495735386e-05\\
400	-1.64406424988235e-05\\
401	1.54203728647106e-05\\
402	3.20024938419253e-05\\
403	2.1565386485968e-05\\
404	0.000108147998056347\\
405	8.31361305831798e-05\\
406	0.000109266833848618\\
407	4.6894132007637e-05\\
408	6.85433431614022e-05\\
409	6.47562749848407e-05\\
410	7.98382378636193e-05\\
411	4.92617926379684e-05\\
412	8.88853344884947e-05\\
413	8.37878852972001e-05\\
414	0.000113360184278389\\
415	0.000110830942105571\\
416	0.000108344941131748\\
417	0.000116870469012302\\
418	9.43628265594454e-05\\
419	0.000132047400385768\\
420	8.6507056998871e-05\\
421	6.07148834870141e-05\\
422	1.63943841904155e-06\\
423	-7.78820224339392e-06\\
424	4.31470085597859e-05\\
425	1.25852196190076e-05\\
426	1.69915504598389e-05\\
427	5.0767405980096e-05\\
428	8.96614677619282e-05\\
429	0.000156279155974597\\
430	0.00015627795848093\\
431	0.000171021403389327\\
432	0.000151912404840378\\
433	0.000102158120467678\\
434	9.28833389574395e-05\\
435	9.44758570053012e-05\\
436	6.18407668144754e-05\\
437	5.86582664860558e-05\\
438	-2.10097298738745e-05\\
439	-3.39392676759646e-05\\
440	-2.67890726277779e-05\\
441	-6.89498287447004e-05\\
442	-5.00673436932559e-05\\
443	-2.04652076031109e-05\\
444	-6.04800353923159e-05\\
445	-2.23778858276216e-05\\
446	-3.98416846520487e-05\\
447	-1.37622989537478e-05\\
448	-3.05720451698289e-05\\
449	-1.11963020499089e-05\\
450	5.05666501673976e-05\\
451	3.81383106479992e-05\\
452	3.28627429207679e-05\\
453	1.07927318901284e-05\\
454	2.81583524119881e-05\\
455	1.35017971594272e-05\\
456	-8.92879795903369e-06\\
457	-3.13205527318162e-05\\
458	-5.96784949852746e-05\\
459	-5.35763568299026e-05\\
460	-3.32463444582558e-05\\
461	-2.27032941395231e-05\\
462	-5.13451668724311e-05\\
463	-7.0375158506433e-05\\
464	-5.80887960434876e-05\\
465	-7.95413999749756e-05\\
466	-8.46391604025574e-05\\
467	-6.49067280819278e-05\\
468	-6.22529815159006e-05\\
469	-8.29897776654898e-05\\
470	-4.25813317286678e-05\\
471	-3.64090346571201e-05\\
472	-1.66394092772839e-05\\
473	-3.70637327684393e-05\\
474	-6.64897730985238e-05\\
475	-4.37208235566674e-05\\
476	-3.07662941719943e-05\\
477	-2.32549096624963e-05\\
478	-3.7400990642488e-05\\
479	-2.2441660058552e-05\\
480	-2.27827689478426e-05\\
481	-7.33189037064242e-06\\
482	-1.7794952973533e-05\\
483	-2.44897000138108e-05\\
484	1.13033680290938e-05\\
485	4.92510283744484e-05\\
486	3.71188216754824e-05\\
487	2.7015989413787e-05\\
488	-1.41965373499876e-05\\
489	-1.54682157666696e-05\\
490	-1.69508704108632e-06\\
491	-3.14348273434758e-05\\
492	-3.10987959050447e-05\\
493	-2.75475813220262e-05\\
494	-1.10786349382998e-05\\
495	7.92598825039418e-06\\
496	6.40423321309148e-06\\
497	1.86637339513109e-05\\
498	-3.78933289828524e-05\\
499	-2.71743167512036e-05\\
500	-1.04848680559755e-05\\
501	4.89717505050948e-07\\
502	-2.64122906312586e-05\\
503	-4.27848203200918e-05\\
504	-7.24259375220077e-05\\
505	-0.000100257598516533\\
506	-8.18672648187702e-05\\
507	-7.01549770945889e-05\\
508	-7.36217606926968e-05\\
509	-5.84491605113892e-05\\
510	-6.87149021304189e-05\\
511	-5.76824118332308e-05\\
512	-5.234431373522e-05\\
513	-4.18242875286809e-05\\
514	-6.43626759343678e-05\\
515	-7.35844740360113e-05\\
516	-6.72829536305721e-05\\
517	-6.07692007685395e-05\\
518	-3.05556753999254e-05\\
519	-2.07932711166156e-05\\
520	-1.46846338504262e-05\\
521	-2.00549439185597e-05\\
522	-3.03890008665741e-05\\
523	-4.52744195110673e-05\\
524	-3.61618180296381e-05\\
525	-1.8912169426041e-05\\
526	-1.53596327652902e-06\\
527	-6.92615036265853e-06\\
528	1.22583673183503e-05\\
529	1.99118915751883e-05\\
530	1.33141074724938e-05\\
531	6.17284894792415e-07\\
532	1.1593688471237e-05\\
533	6.36961733631362e-06\\
534	-2.08157123356011e-06\\
535	3.71877822783272e-06\\
536	2.02248614541818e-05\\
537	1.04607373273737e-06\\
538	9.56250327346983e-06\\
539	6.96943773534743e-07\\
540	2.26003881636845e-05\\
541	2.5932032395523e-05\\
542	1.61932456444736e-05\\
543	-5.45818202470775e-06\\
544	7.28350138739066e-07\\
545	-1.60870971870745e-06\\
546	-1.10255152601162e-05\\
547	-9.78013068272739e-07\\
548	1.19288441702253e-05\\
549	6.41491917266242e-06\\
550	4.08319360466191e-06\\
551	3.62048286151306e-05\\
552	2.1120790089692e-05\\
553	1.15008418301702e-05\\
554	2.35907797138288e-05\\
555	5.74004746706049e-06\\
556	1.40417381186507e-05\\
557	4.40995138499169e-05\\
558	2.89761313566352e-05\\
559	1.95990600326734e-05\\
560	2.97413137470272e-05\\
561	2.90035358601976e-05\\
562	3.24646084330841e-05\\
563	2.80701535784995e-05\\
564	2.96083928215974e-05\\
565	4.49945645422466e-05\\
566	4.86753654625343e-05\\
567	6.03991711553029e-05\\
568	5.67897131088085e-05\\
569	5.39041494398924e-05\\
570	6.20880222057802e-05\\
571	5.48641419040422e-05\\
572	5.77507840874211e-05\\
573	3.25875646860799e-05\\
574	2.1792075922756e-05\\
575	2.23925295130312e-05\\
576	2.35656083666825e-05\\
577	1.47653886500587e-05\\
578	1.6630688310716e-05\\
579	1.60899460715231e-05\\
580	7.65692688026578e-06\\
581	1.4201449105392e-05\\
582	1.16595314417306e-05\\
583	1.28387615612639e-05\\
584	1.99355384442198e-05\\
585	2.848185140311e-05\\
586	3.29184088300938e-05\\
587	2.47933171262187e-05\\
588	3.18772761853636e-05\\
589	3.1516884527328e-05\\
590	3.0128054818472e-05\\
591	3.87082571986429e-05\\
592	4.63548640273404e-05\\
593	4.70574635074274e-05\\
594	4.48534948344245e-05\\
595	2.60348370290983e-05\\
596	3.73359851707132e-05\\
597	3.43212428270036e-05\\
598	3.69942806840238e-05\\
599	5.04279906588669e-05\\
600	4.68726863982559e-05\\
601	4.33302715986621e-05\\
602	4.79527438844018e-05\\
603	4.71010437036677e-05\\
604	4.22421534322608e-05\\
605	4.98943474113203e-05\\
606	4.56973414216863e-05\\
607	4.5185153786694e-05\\
608	5.73450784492131e-05\\
609	4.95415462775783e-05\\
610	5.1415767721986e-05\\
611	5.04197318917588e-05\\
612	4.56448643652332e-05\\
613	4.69083375520644e-05\\
614	5.36590391617383e-05\\
615	6.04581867920034e-05\\
616	6.68462182079522e-05\\
617	6.72922575630398e-05\\
618	6.75834798338909e-05\\
619	6.54087484049596e-05\\
620	6.04730274702839e-05\\
621	5.52445864208707e-05\\
622	5.626422540488e-05\\
623	5.86489373561735e-05\\
624	6.0614247585516e-05\\
625	5.92333920534235e-05\\
626	5.13349518200137e-05\\
627	5.00014339564146e-05\\
628	4.65193114733959e-05\\
629	4.9230256566566e-05\\
630	5.06465911367766e-05\\
631	6.05332288463954e-05\\
632	5.81555314474274e-05\\
633	6.17789646753869e-05\\
634	5.72593844221542e-05\\
635	5.16978175077317e-05\\
636	6.12333697298638e-05\\
637	6.06923406457722e-05\\
638	5.49304993167232e-05\\
639	5.71365564118689e-05\\
640	4.71479302010856e-05\\
641	5.43197526157341e-05\\
642	5.97251430940224e-05\\
643	5.88809464837146e-05\\
644	5.73037439669591e-05\\
645	6.13179671701618e-05\\
646	6.0930775873187e-05\\
647	5.8500461215222e-05\\
648	6.05094593321903e-05\\
649	5.59264238974057e-05\\
650	5.20016518537315e-05\\
651	5.18034932964635e-05\\
652	4.97093257447048e-05\\
653	5.07943259442909e-05\\
654	4.48651487849877e-05\\
655	4.57453630220719e-05\\
656	4.74498771770566e-05\\
657	4.5622990211207e-05\\
658	4.89421042552894e-05\\
659	4.66816014149628e-05\\
660	4.56890116618985e-05\\
661	4.54848102097891e-05\\
662	4.63083784056228e-05\\
663	4.70939706149085e-05\\
664	4.55728502178585e-05\\
665	4.21531241037502e-05\\
666	3.81833058045455e-05\\
667	3.97179648183678e-05\\
668	3.83680476712277e-05\\
669	4.20676977305771e-05\\
670	3.51759266921956e-05\\
671	3.47358991703686e-05\\
672	3.38703175693204e-05\\
673	2.89133970886106e-05\\
674	2.91713690966205e-05\\
675	3.1300259010035e-05\\
676	3.1033253220212e-05\\
677	2.90173469597772e-05\\
678	2.48812069887366e-05\\
679	1.88671587034737e-05\\
680	1.84601236298713e-05\\
681	1.74596946770355e-05\\
682	1.41696934504542e-05\\
683	1.04886795645873e-05\\
684	3.97756165190987e-06\\
685	2.40306372087336e-06\\
686	1.57926340513548e-06\\
687	3.91714068894041e-07\\
688	7.40558329176922e-07\\
689	2.32955419738676e-06\\
690	3.80937188676409e-06\\
691	9.73768050050091e-06\\
692	8.45421253092052e-06\\
693	7.44067664084696e-06\\
694	8.37214357755937e-06\\
695	1.00893448465836e-05\\
696	7.3167040538609e-06\\
697	6.60262163912355e-06\\
698	8.35857878646368e-06\\
699	5.37920352234221e-06\\
700	4.97118370748211e-06\\
701	4.20538151456069e-06\\
702	3.29350229702263e-06\\
703	4.51609761348634e-06\\
704	4.5724950135527e-06\\
705	8.19752020682352e-08\\
706	-2.09241321367266e-06\\
707	-3.1353031835539e-06\\
708	-5.81902681027296e-06\\
709	-5.78471536282425e-06\\
710	-4.35392227197822e-06\\
711	-3.26093307681659e-06\\
712	-7.43366889234457e-06\\
713	-6.80643978667456e-06\\
714	-5.98400631531234e-06\\
715	-5.06660828299977e-06\\
716	-5.57639738378288e-06\\
717	-8.58393845093312e-06\\
718	-1.03808304277649e-05\\
719	-9.00738653712003e-06\\
720	-9.7535789098516e-06\\
721	-1.10448229175711e-05\\
722	-1.65146836266675e-05\\
723	-1.28482381710002e-05\\
724	-1.67396339959116e-05\\
725	-1.68529701753958e-05\\
726	-1.56864906503208e-05\\
727	-1.55884291160365e-05\\
728	-1.4166181967252e-05\\
729	-1.17554909163972e-05\\
730	-1.21713051666256e-05\\
731	-1.22995949381832e-05\\
732	-1.20319845255397e-05\\
733	-1.54660031986355e-05\\
734	-1.44947082388034e-05\\
735	-1.62189211496185e-05\\
736	-1.82246490040086e-05\\
737	-1.75677825479178e-05\\
738	-1.47730424700607e-05\\
739	-1.45031326429703e-05\\
740	-1.47940822979974e-05\\
741	-1.62173555615394e-05\\
742	-1.65988934880311e-05\\
743	-1.95097590970109e-05\\
744	-1.96094585101796e-05\\
745	-2.08836715577969e-05\\
746	-2.0006108638058e-05\\
747	-2.03729942944775e-05\\
748	-1.91746858003889e-05\\
749	-1.75397012713702e-05\\
750	-1.63124107215347e-05\\
751	-1.48136757952466e-05\\
752	-1.25437742943343e-05\\
753	-1.34884696211027e-05\\
754	-1.48277864931428e-05\\
755	-1.45982518508352e-05\\
756	-1.47730707211292e-05\\
757	-1.30543186456851e-05\\
758	-1.25295302724699e-05\\
759	-1.18136369678809e-05\\
760	-8.81522275487115e-06\\
761	-7.65883583441983e-06\\
762	-1.00826092717069e-05\\
763	-1.10091776026229e-05\\
764	-1.17677762879781e-05\\
765	-1.49547385596909e-05\\
766	-1.47339159368723e-05\\
767	-1.33699407847263e-05\\
768	-1.44920584120069e-05\\
769	-1.10454730898147e-05\\
770	-1.02935785379715e-05\\
771	-1.03542409482473e-05\\
772	-7.62938674998296e-06\\
773	-6.5581822314414e-06\\
774	-6.12867378427962e-06\\
775	-5.50152506798292e-06\\
776	-4.40112408970176e-06\\
777	-4.01667171925625e-06\\
778	-4.85345928417605e-06\\
779	-4.99552978434537e-06\\
780	-6.84530559408413e-06\\
781	-7.69822702387362e-06\\
782	-8.42671541514696e-06\\
783	-8.25645262807572e-06\\
784	-6.73335075323194e-06\\
785	-7.69439468578966e-06\\
786	-8.05678875011815e-06\\
787	-8.96150514756255e-06\\
788	-8.56092324660246e-06\\
789	-8.01527688844878e-06\\
790	-9.49857102127544e-06\\
791	-6.40720801543926e-06\\
792	-8.62837001172458e-06\\
793	-8.68996866974169e-06\\
794	-8.3137183166485e-06\\
795	-8.46760608496852e-06\\
796	-8.07882939937786e-06\\
797	-1.0826651070201e-05\\
798	-1.16442620299586e-05\\
799	-1.2377972406226e-05\\
800	-1.15455212029749e-05\\
801	-1.20137478293544e-05\\
802	-1.28493932857892e-05\\
803	-1.34520593815308e-05\\
804	-1.39422576145747e-05\\
805	-1.36001825791556e-05\\
806	-1.51554122124657e-05\\
807	-1.49802662146173e-05\\
808	-1.61283261937126e-05\\
809	-1.74810216733284e-05\\
810	-1.68687638983572e-05\\
811	-1.5252824412989e-05\\
812	-1.58073073096788e-05\\
813	-1.70840161693827e-05\\
814	-1.70495016188231e-05\\
815	-1.65646395321528e-05\\
816	-1.66164504282711e-05\\
817	-1.62702602990789e-05\\
818	-1.4838824722038e-05\\
819	-1.45497726885672e-05\\
820	-1.40470039301931e-05\\
821	-1.45998116696296e-05\\
822	-1.37099698077117e-05\\
823	-1.42784913502603e-05\\
824	-1.19951850282921e-05\\
825	-1.26698288026439e-05\\
826	-1.2110865703829e-05\\
827	-1.15675363375674e-05\\
828	-1.21047563682096e-05\\
829	-1.20645845183319e-05\\
830	-1.13556732384305e-05\\
831	-1.19185486038115e-05\\
832	-1.29007562596767e-05\\
833	-1.21319328725059e-05\\
834	-1.12437565344644e-05\\
835	-1.09610562960594e-05\\
836	-1.04796355432842e-05\\
837	-1.05112358408238e-05\\
838	-1.06341462405518e-05\\
839	-9.63054795669434e-06\\
840	-9.86218430385539e-06\\
841	-1.01278544106511e-05\\
842	-1.00103001726604e-05\\
843	-9.75451401984139e-06\\
844	-1.03378740496562e-05\\
845	-1.14980968634849e-05\\
846	-1.18549030419754e-05\\
847	-1.17277801749423e-05\\
848	-1.19146113173303e-05\\
849	-1.20917877157031e-05\\
850	-1.19535280313147e-05\\
851	-1.14347396004604e-05\\
852	-1.21000709832039e-05\\
853	-1.26439158961879e-05\\
854	-1.22742816324698e-05\\
855	-1.28660511143079e-05\\
856	-1.35426684307182e-05\\
857	-1.34496080821313e-05\\
858	-1.32430028710939e-05\\
859	-1.25405059763135e-05\\
860	-1.22604293060996e-05\\
861	-1.24118916197592e-05\\
862	-1.28358732813165e-05\\
863	-1.39945789450079e-05\\
864	-1.31679840524809e-05\\
865	-1.2180676465011e-05\\
866	-1.22284801235407e-05\\
867	-1.26144971241791e-05\\
868	-1.28943093736106e-05\\
869	-1.20089380420429e-05\\
870	-1.11603909310037e-05\\
871	-1.08631317865558e-05\\
872	-1.08024046910372e-05\\
873	-1.10891999846299e-05\\
874	-1.14952559407494e-05\\
875	-1.15535336393423e-05\\
876	-1.19744346338849e-05\\
877	-1.22068538250505e-05\\
878	-1.27420980008386e-05\\
879	-1.31320236406059e-05\\
880	-1.28785521862718e-05\\
881	-1.30464422494361e-05\\
882	-1.19838263111082e-05\\
883	-1.2675335559716e-05\\
884	-1.22122756724383e-05\\
885	-1.22041294828225e-05\\
886	-1.15011510838377e-05\\
887	-1.13003587940307e-05\\
888	-1.09360187532633e-05\\
889	-1.14202487517652e-05\\
890	-1.12902029147299e-05\\
891	-1.14878364269712e-05\\
892	-1.18428043367368e-05\\
893	-1.22397142023634e-05\\
894	-1.17388472688766e-05\\
895	-1.16700859876109e-05\\
896	-1.14758086861665e-05\\
897	-1.13829811595398e-05\\
898	-1.1566848221754e-05\\
899	-1.18032452498662e-05\\
900	-1.21405512077433e-05\\
901	-1.17672275204222e-05\\
902	-1.21148373131427e-05\\
903	-1.21178470384786e-05\\
904	-1.22674353792361e-05\\
905	-1.15070825178624e-05\\
906	-1.16846212618595e-05\\
907	-1.17205398128282e-05\\
908	-1.181102353724e-05\\
909	-1.20748139866892e-05\\
910	-1.20743090308142e-05\\
911	-1.17932775301087e-05\\
912	-1.18985778409564e-05\\
913	-1.12093260301435e-05\\
914	-1.11660353776777e-05\\
915	-1.08751327223918e-05\\
916	-1.09994420742022e-05\\
917	-1.0879673332064e-05\\
918	-1.07107438477977e-05\\
919	-1.03764037789296e-05\\
920	-1.04562572313904e-05\\
921	-9.9582994858948e-06\\
922	-9.87880569771495e-06\\
923	-1.02337081706334e-05\\
924	-1.06021700366756e-05\\
925	-1.02252800302333e-05\\
926	-9.85753064026175e-06\\
927	-9.8589690791433e-06\\
928	-9.5731826166314e-06\\
929	-9.40323445757598e-06\\
930	-9.09514699631084e-06\\
931	-9.03442713574265e-06\\
932	-9.15169757346571e-06\\
933	-9.33165895515041e-06\\
934	-9.24590429738979e-06\\
935	-9.53333345132536e-06\\
936	-9.34712169239602e-06\\
937	-9.07504160849894e-06\\
938	-8.89249237719329e-06\\
939	-8.66117593877965e-06\\
940	-8.98806674127502e-06\\
941	-8.93358717322379e-06\\
942	-8.87791009587338e-06\\
943	-8.53114327622785e-06\\
944	-8.22595772303868e-06\\
945	-8.44379808188466e-06\\
946	-8.23139385846034e-06\\
947	-8.14346534637077e-06\\
948	-7.88054349660442e-06\\
949	-7.7641929537952e-06\\
950	-7.69502662901371e-06\\
951	-7.67803688370696e-06\\
952	-7.47765433496717e-06\\
953	-7.57837049378082e-06\\
954	-7.38842277473502e-06\\
955	-7.42315990574556e-06\\
956	-7.31147632581158e-06\\
957	-7.34052382885885e-06\\
958	-7.09346831760505e-06\\
959	-7.08769815197111e-06\\
960	-7.11977765552425e-06\\
961	-7.06175947214746e-06\\
962	-6.9567010524745e-06\\
963	-6.30654508945225e-06\\
964	-6.29676157555018e-06\\
965	-6.47290970054074e-06\\
966	-6.80773438666374e-06\\
967	-6.79391684918902e-06\\
968	-6.66364191812596e-06\\
969	-6.59143227855146e-06\\
970	-6.4700746256803e-06\\
971	-6.88676894076995e-06\\
972	-6.80536393793315e-06\\
973	-6.46868531447457e-06\\
974	-6.19382518141896e-06\\
975	-6.10176863001132e-06\\
976	-6.32909897654606e-06\\
977	-6.20220590200837e-06\\
978	-6.24369496684735e-06\\
979	-6.35746335064616e-06\\
980	-6.69620872936981e-06\\
981	-6.64416419760448e-06\\
982	-6.70102495183635e-06\\
983	-6.72148908465153e-06\\
984	-6.54784652355261e-06\\
985	-6.74939025130152e-06\\
986	-6.72713184201977e-06\\
987	-6.53443249369408e-06\\
988	-6.31840715968816e-06\\
989	-6.21322111256793e-06\\
990	-5.95516324938299e-06\\
991	-6.10332615872952e-06\\
992	-5.98013403716522e-06\\
993	-5.5726867465963e-06\\
994	-5.4541356758755e-06\\
995	-5.27642096460834e-06\\
996	-5.34258126234131e-06\\
997	-5.19816629176114e-06\\
998	-5.22014834178633e-06\\
999	-5.26984288348229e-06\\
1000	-5.17292703731922e-06\\
};
\addlegendentry{$\Re \{ \mathbf{g}_1^H \}$};

\addplot [color=mycolor4,solid]
  table[row sep=crcr]{%
1	0\\
2	0\\
3	0\\
4	0.00157327365770406\\
5	0.00447368069753032\\
6	0.00235133421525968\\
7	0.00175978214474647\\
8	0.00482292152890212\\
9	0.0030022373154883\\
10	0.000920002485363626\\
11	0.00159811748301728\\
12	-0.00181704703617359\\
13	-0.00432669044321994\\
14	-0.00354798432549027\\
15	-0.00612101163921818\\
16	-0.00624223380623484\\
17	-0.00381348096665293\\
18	-0.00307702116877304\\
19	-0.00379088740351494\\
20	-0.00435692056080752\\
21	-0.00651065936932903\\
22	-0.00457141272821505\\
23	-0.00539333737358983\\
24	-0.00242953303272157\\
25	-0.00120789525807153\\
26	-0.00356059693834488\\
27	-0.00399885261425521\\
28	-0.00421474848478614\\
29	0.00102496515032098\\
30	-0.00236045759085479\\
31	-0.00558479147655795\\
32	-0.00197843668308262\\
33	-0.00247948430035052\\
34	-0.00244329954282673\\
35	0.000326583708556728\\
36	-0.00212549132674097\\
37	-0.00408910218725525\\
38	-0.00406793559016663\\
39	-0.00612540975458189\\
40	-0.00768096870064545\\
41	-0.00793098828540771\\
42	-0.00698886999930887\\
43	-0.00707998592365581\\
44	-0.00329216867200468\\
45	-0.000857084112030234\\
46	0.0007498906367418\\
47	0.000448145176185267\\
48	0.00334535278403076\\
49	0.00330264079877413\\
50	0.00566685639838253\\
51	0.00556013839508369\\
52	0.00465770271671224\\
53	0.00549833237814537\\
54	0.00376274653583269\\
55	0.00374244245115023\\
56	0.00588604020345971\\
57	0.00503004275575865\\
58	0.00384239619917904\\
59	0.0054746248264807\\
60	0.00488632717858889\\
61	0.00551649508672007\\
62	0.00786101039786343\\
63	0.00693467223748108\\
64	0.00854367293957347\\
65	0.00783182087084969\\
66	0.00769851025480587\\
67	0.00839409371775898\\
68	0.00908463228690636\\
69	0.0101251742816654\\
70	0.010989834414718\\
71	0.0101093064474844\\
72	0.010682987721681\\
73	0.00942074026415238\\
74	0.00910126711400841\\
75	0.0109534876319079\\
76	0.0121655927746142\\
77	0.0112629366475856\\
78	0.0106476266416748\\
79	0.0112353875816921\\
80	0.00981413081331729\\
81	0.00961180425765987\\
82	0.0087714578605819\\
83	0.00914146237625738\\
84	0.00885172137061046\\
85	0.00815552148035184\\
86	0.00698154490700098\\
87	0.00649835225664637\\
88	0.00607982842335972\\
89	0.0053016866181572\\
90	0.00625040175844778\\
91	0.00691808111297686\\
92	0.00698950153066265\\
93	0.00581684369115258\\
94	0.00563957357266273\\
95	0.00761160351638639\\
96	0.0100247383231617\\
97	0.00775783923827714\\
98	0.00727627337049158\\
99	0.00549199732340669\\
100	0.00486494332927845\\
101	0.00524532543909992\\
102	0.0050147310740688\\
103	0.00436440782330074\\
104	0.00335959656292249\\
105	0.00237034493518071\\
106	0.00194787903540482\\
107	0.00224369742206304\\
108	0.00210161649885411\\
109	0.00212462986233277\\
110	0.00151316029421584\\
111	0.00172828671336753\\
112	0.00160954337742277\\
113	0.000423933480707644\\
114	0.000824914778250403\\
115	0.00140411260738425\\
116	0.00203318172939409\\
117	0.00335912661707904\\
118	0.00289073745666322\\
119	0.00303605418276981\\
120	0.00171570743342375\\
121	0.00281809457498871\\
122	0.00394763886376021\\
123	0.00411215294040727\\
124	0.00386620819225101\\
125	0.0047288069682862\\
126	0.00548427898816641\\
127	0.00506021515557926\\
128	0.00516629234289898\\
129	0.00576290318897149\\
130	0.00654875687358224\\
131	0.00694333005692433\\
132	0.00627628770530144\\
133	0.00670451323716824\\
134	0.00703904195189757\\
135	0.00707124993727298\\
136	0.00940445008883224\\
137	0.0100874993148978\\
138	0.0099137797171665\\
139	0.00891187405677608\\
140	0.0085695724091816\\
141	0.00638691258490484\\
142	0.00633329449162286\\
143	0.00710482001831043\\
144	0.0067035232458948\\
145	0.00548894616877695\\
146	0.00498370839817825\\
147	0.00497157894838811\\
148	0.004174661671749\\
149	0.00381268256145129\\
150	0.00460650592547594\\
151	0.00395034441371037\\
152	0.00425830415707813\\
153	0.00517206561463156\\
154	0.00540438405850214\\
155	0.00542610672412913\\
156	0.00452235896757286\\
157	0.00433583802875121\\
158	0.00365278854631445\\
159	0.0041277712687998\\
160	0.00350558937446874\\
161	0.00413604298929663\\
162	0.00386365996593239\\
163	0.00411671995760567\\
164	0.00528559347488256\\
165	0.00485334546193628\\
166	0.00450211630178116\\
167	0.00428350918732955\\
168	0.00441282413031644\\
169	0.00486281934304571\\
170	0.00550438910125848\\
171	0.00524104017672165\\
172	0.00547386654802751\\
173	0.00508967218986356\\
174	0.00532482287040926\\
175	0.00535764724061595\\
176	0.0046562957987727\\
177	0.00472459747357668\\
178	0.00486640196731997\\
179	0.00437786953048446\\
180	0.0042935232581724\\
181	0.0043266392972878\\
182	0.00409353736329386\\
183	0.00438451300678876\\
184	0.00418861121794295\\
185	0.00431686984945496\\
186	0.00405563156889589\\
187	0.00397259830382201\\
188	0.00408685102794388\\
189	0.00376620892296216\\
190	0.00379337043179146\\
191	0.00381975032145631\\
192	0.0039477610288762\\
193	0.00413023372059302\\
194	0.00395640196497579\\
195	0.00414528728169071\\
196	0.00455177511160404\\
197	0.00424950106426342\\
198	0.0041560522482217\\
199	0.00420549558443921\\
200	0.00472066025868679\\
201	0.00457282257134202\\
202	0.0042309531130538\\
203	0.00449828910961222\\
204	0.00423761933543456\\
205	0.00435329858644486\\
206	0.00443913957306412\\
207	0.00433562404620445\\
208	0.00406320557639031\\
209	0.00429324341031333\\
210	0.00412673884541891\\
211	0.00412657911236047\\
212	0.00474704737133237\\
213	0.00467659297540247\\
214	0.00473867832962555\\
215	0.00503940900570943\\
216	0.00533946199366337\\
217	0.00483362593018631\\
218	0.00460302785469698\\
219	0.0047728196050044\\
220	0.0049842546796043\\
221	0.00537769202490658\\
222	0.00529745808557422\\
223	0.00525542053604553\\
224	0.00468921300425392\\
225	0.00471467904650367\\
226	0.00437304347855073\\
227	0.00409410413894853\\
228	0.00415832184988593\\
229	0.0043844816092053\\
230	0.00423570739162173\\
231	0.00446262036864396\\
232	0.00455915993658535\\
233	0.00469460239092363\\
234	0.00462497522006483\\
235	0.00468276825941312\\
236	0.00476681459197688\\
237	0.00443898007592075\\
238	0.00447729520346743\\
239	0.0046722648898329\\
240	0.00416404430962696\\
241	0.00401144956168326\\
242	0.00383500984098466\\
243	0.00380681321278693\\
244	0.00342649847812193\\
245	0.00347517259309816\\
246	0.00381236145278191\\
247	0.00341061711219607\\
248	0.00334469732463888\\
249	0.00346041374504987\\
250	0.0035326092893383\\
251	0.00327925203814179\\
252	0.00302614027453347\\
253	0.00293805068204213\\
254	0.0026331931160818\\
255	0.00271446683617453\\
256	0.00253743755627409\\
257	0.00274677100634144\\
258	0.00266192973844226\\
259	0.00275168898909662\\
260	0.00273220220944917\\
261	0.00295269457626438\\
262	0.00278923649757065\\
263	0.00281749687032295\\
264	0.00294984360875615\\
265	0.00311172247662734\\
266	0.00312854768160323\\
267	0.00313232092310024\\
268	0.00338854485110159\\
269	0.00329671691934487\\
270	0.00315942139328388\\
271	0.00289139675471637\\
272	0.00271589726241535\\
273	0.00285032616545607\\
274	0.00311266002618855\\
275	0.0030502812069129\\
276	0.00319999964823892\\
277	0.00296813468729019\\
278	0.00283453432311019\\
279	0.0026354760298584\\
280	0.00271345114074468\\
281	0.00266159396373096\\
282	0.00257221737314085\\
283	0.00248494647444471\\
284	0.00254842951955852\\
285	0.00245159556270539\\
286	0.00273164007580245\\
287	0.00268327652169942\\
288	0.00282786163627325\\
289	0.00282150117600395\\
290	0.00292122254366334\\
291	0.00268898428755057\\
292	0.00259416084247521\\
293	0.0025548782882123\\
294	0.00258002230596089\\
295	0.00269245253989586\\
296	0.0027784689296843\\
297	0.00309464580446782\\
298	0.00307306721908457\\
299	0.00285945170931077\\
300	0.00280931623223916\\
301	0.00267032108679439\\
302	0.00255323938747041\\
303	0.00242555142640476\\
304	0.00211017850548202\\
305	0.00202224473415171\\
306	0.00190874886537059\\
307	0.00215710088908794\\
308	0.00203045972648705\\
309	0.00218450039690307\\
310	0.0023929082910546\\
311	0.00239933884435263\\
312	0.00241528667313393\\
313	0.00243040171234266\\
314	0.00247782385079747\\
315	0.00242376900153652\\
316	0.00231371322298139\\
317	0.00224893571450438\\
318	0.00239691705155953\\
319	0.0023081461765334\\
320	0.00237686560374963\\
321	0.00221009406540076\\
322	0.00220305134710292\\
323	0.00214149294935922\\
324	0.00196325230982946\\
325	0.00203283988507751\\
326	0.0021326581905009\\
327	0.00205395513995399\\
328	0.00201750874270222\\
329	0.00198139585813887\\
330	0.00192383847061522\\
331	0.00194041287682611\\
332	0.00188226000090502\\
333	0.00199559386630393\\
334	0.00194996714340024\\
335	0.00188980286315085\\
336	0.00192013032362634\\
337	0.0018428276618304\\
338	0.00171462201862329\\
339	0.00162086536808132\\
340	0.00157906151452318\\
341	0.00148577588346649\\
342	0.00148020554720525\\
343	0.00167156331154269\\
344	0.00172560537235272\\
345	0.00167210388184049\\
346	0.00151568028483011\\
347	0.00158067535751145\\
348	0.00157558078307812\\
349	0.0015428802898374\\
350	0.00138209193746524\\
351	0.00134002658148643\\
352	0.00133176706296911\\
353	0.00129741675843679\\
354	0.00127621815523285\\
355	0.00139013473677268\\
356	0.00135971287684236\\
357	0.0013507033849787\\
358	0.00147712360910227\\
359	0.00126198947146555\\
360	0.00121576356081547\\
361	0.00117521928434791\\
362	0.00118662261207578\\
363	0.00122075759244688\\
364	0.00130572327080338\\
365	0.00134753100449133\\
366	0.00129507840506437\\
367	0.00131858885157837\\
368	0.00135788797622036\\
369	0.00134497349757808\\
370	0.00133224874958394\\
371	0.00134934003716439\\
372	0.001297216238527\\
373	0.00124066529160482\\
374	0.00127602339851885\\
375	0.00130455121291227\\
376	0.00130750994413715\\
377	0.00130822606971185\\
378	0.00130553463070451\\
379	0.00140397419478625\\
380	0.00138063814178528\\
381	0.00138685958201702\\
382	0.00138079511326152\\
383	0.0012795025810501\\
384	0.00119560767409\\
385	0.00109082181682339\\
386	0.00105426526782948\\
387	0.00106161285231802\\
388	0.000951044000547097\\
389	0.000924267395754392\\
390	0.000893859081240121\\
391	0.000875851860165507\\
392	0.000834951279257901\\
393	0.000882199030933522\\
394	0.000877873621976897\\
395	0.000812173379167875\\
396	0.000798098403279342\\
397	0.000800552357361\\
398	0.000854449697851498\\
399	0.000826002873420711\\
400	0.000794612909404978\\
401	0.000795937576348573\\
402	0.000779673398371707\\
403	0.000748700983352198\\
404	0.000715875931809919\\
405	0.000658030151758896\\
406	0.000716178502425339\\
407	0.00077584881442361\\
408	0.000737549681860697\\
409	0.000811458328343058\\
410	0.000767692536327829\\
411	0.000807177816567456\\
412	0.000785247784230442\\
413	0.000779660259639487\\
414	0.00083645113140816\\
415	0.000812900660530932\\
416	0.000825301110262898\\
417	0.000846183721756763\\
418	0.000867671338333425\\
419	0.000800856185300097\\
420	0.000791813203171139\\
421	0.000746954459810465\\
422	0.000732074406008598\\
423	0.000788525360703602\\
424	0.000784475489081817\\
425	0.000734466969728578\\
426	0.000669042604661269\\
427	0.000663154686730955\\
428	0.00062301162000513\\
429	0.000616700134657657\\
430	0.000605517501524103\\
431	0.000577750508740638\\
432	0.000638292548428994\\
433	0.000685159141502915\\
434	0.000668694343077817\\
435	0.000680318325809177\\
436	0.000699174089602845\\
437	0.000655836391703051\\
438	0.000681562572000594\\
439	0.000688708482720767\\
440	0.00070391436006245\\
441	0.000638406564837197\\
442	0.000590101343656356\\
443	0.000587010836666614\\
444	0.000620556737451384\\
445	0.000603660375280856\\
446	0.000595097121794745\\
447	0.000605313310240598\\
448	0.000608806841719906\\
449	0.000625110596022834\\
450	0.000572108984124834\\
451	0.000612294956196268\\
452	0.000633603199308107\\
453	0.000586891814615477\\
454	0.000591681323705885\\
455	0.000592894980957745\\
456	0.000604008674532853\\
457	0.000625976690156395\\
458	0.00061974234794816\\
459	0.00060501863269676\\
460	0.000601659749287181\\
461	0.000573550558472314\\
462	0.000522224329424075\\
463	0.000515375170380241\\
464	0.000512375903410046\\
465	0.000525266108677432\\
466	0.000510872803859313\\
467	0.000502625147829892\\
468	0.000496891266208771\\
469	0.000481133208680428\\
470	0.000484098524268451\\
471	0.000481901666712842\\
472	0.000417319513866025\\
473	0.000383455629141047\\
474	0.000404808745458879\\
475	0.000422207384940831\\
476	0.000438511493303811\\
477	0.000462828785726673\\
478	0.000491573625939083\\
479	0.000454886157111157\\
480	0.000468660057567368\\
481	0.000454510318071592\\
482	0.000417987190308825\\
483	0.000418002235871736\\
484	0.000414452605536375\\
485	0.00042375118493809\\
486	0.000422527920068321\\
487	0.000399399023750454\\
488	0.000427817810030436\\
489	0.000451026960358654\\
490	0.000434293680694126\\
491	0.000424417132919081\\
492	0.000431452487660919\\
493	0.00043346642031568\\
494	0.000433497080062692\\
495	0.000458117459488553\\
496	0.000437041602609544\\
497	0.00047417548331551\\
498	0.000453747107241576\\
499	0.000435390040752784\\
500	0.000438267326356972\\
501	0.000454969614854395\\
502	0.000425341363437096\\
503	0.000397850863745499\\
504	0.000392068865116836\\
505	0.000380345592007808\\
506	0.000383591352107038\\
507	0.00036011123551455\\
508	0.000361109325378236\\
509	0.000374985280228406\\
510	0.000379280885912181\\
511	0.000378618214786978\\
512	0.00037487752213314\\
513	0.000344859373584717\\
514	0.00031402632064325\\
515	0.000333491972097217\\
516	0.000347970778129769\\
517	0.000339503357455449\\
518	0.000354153806769795\\
519	0.000342919310807692\\
520	0.000348351249596188\\
521	0.000357492684294604\\
522	0.000345995917108713\\
523	0.000345604250205745\\
524	0.000325189638192079\\
525	0.000337868153882867\\
526	0.000347923352878086\\
527	0.000327274893821401\\
528	0.000323211875075345\\
529	0.000310378699739403\\
530	0.00030764093194081\\
531	0.000307253630310635\\
532	0.000270351569862913\\
533	0.000271954084501176\\
534	0.000288172707655614\\
535	0.000284718523015622\\
536	0.000281899246804875\\
537	0.000288665202834037\\
538	0.000272571876569534\\
539	0.000270722103856664\\
540	0.000281639765918897\\
541	0.000293681073072219\\
542	0.000285146639930123\\
543	0.000283301300479995\\
544	0.000295484339502691\\
545	0.000305675776473848\\
546	0.00030098200300199\\
547	0.000277473795909325\\
548	0.000280392550383738\\
549	0.000273499335951818\\
550	0.000263473045226926\\
551	0.000268845074930981\\
552	0.000240403591928609\\
553	0.000229474811438086\\
554	0.000228810669883899\\
555	0.000224153148936137\\
556	0.000223272026329182\\
557	0.000234687642084339\\
558	0.000225579910367268\\
559	0.000230289250268919\\
560	0.00021926370298153\\
561	0.000201429242312376\\
562	0.00020137662673486\\
563	0.000202399309656454\\
564	0.000213420965055083\\
565	0.000215158311893791\\
566	0.000203567634152439\\
567	0.000207020795437453\\
568	0.000212372975559216\\
569	0.000193026845612853\\
570	0.000188016642785378\\
571	0.000200793056724545\\
572	0.000197907229884344\\
573	0.00017500727523727\\
574	0.00017723057922842\\
575	0.000183525972324637\\
576	0.000184898347387556\\
577	0.000187961595833709\\
578	0.000176958503276987\\
579	0.000165110146484097\\
580	0.000163269641285357\\
581	0.000179085911818592\\
582	0.000182117353307313\\
583	0.000176562682995556\\
584	0.000180832101800146\\
585	0.000179904754079362\\
586	0.00016087012352769\\
587	0.000161653785933457\\
588	0.000158049584678209\\
589	0.000163560797933952\\
590	0.000170232075659218\\
591	0.000163457079452943\\
592	0.000167644422016235\\
593	0.000173850692244056\\
594	0.000172007753801632\\
595	0.000170015376075824\\
596	0.000165860077911744\\
597	0.000167418647193156\\
598	0.000176052272008811\\
599	0.000178427035456482\\
600	0.000169126162419297\\
601	0.000183600545872949\\
602	0.000186297919414728\\
603	0.000191080729321186\\
604	0.000179137778728292\\
605	0.000177221173244655\\
606	0.000176598908546198\\
607	0.000168379578417897\\
608	0.000173660524716825\\
609	0.000170743694419382\\
610	0.000169850169808304\\
611	0.00016380425181966\\
612	0.000158003127117854\\
613	0.000158390401730647\\
614	0.000170447960565957\\
615	0.00017011019350974\\
616	0.000164861976325348\\
617	0.000159515692055786\\
618	0.000146082870569679\\
619	0.000139048834496952\\
620	0.000140145730208448\\
621	0.000139524608460465\\
622	0.000140800902403204\\
623	0.000133472985253141\\
624	0.000121597364458438\\
625	0.000126100788427134\\
626	0.000128472559174241\\
627	0.000128103045016524\\
628	0.000125247060879434\\
629	0.000133820811797481\\
630	0.000134743843939442\\
631	0.000134366571215736\\
632	0.000130508111449158\\
633	0.000122290807970992\\
634	0.000121562160478061\\
635	0.000126572679387387\\
636	0.000124794738208126\\
637	0.00012915369970115\\
638	0.000122927850429719\\
639	0.000119735465154431\\
640	0.00011203051176451\\
641	0.000102361755032387\\
642	0.000109614033908503\\
643	0.000109864309500673\\
644	0.000106732475984223\\
645	0.000100625327033975\\
646	0.000102065301455355\\
647	0.000100291589211069\\
648	9.79150689252602e-05\\
649	0.000101411462068299\\
650	0.000101034835374443\\
651	0.000101288899067918\\
652	0.000107116422349835\\
653	0.000100587799015203\\
654	0.000101112877010053\\
655	0.00010336940529445\\
656	9.61693048104714e-05\\
657	9.59949825790369e-05\\
658	9.65294541357721e-05\\
659	9.03681356909863e-05\\
660	8.82367421403711e-05\\
661	8.38711092418317e-05\\
662	7.72831168046488e-05\\
663	7.86852077731058e-05\\
664	7.67062191659069e-05\\
665	7.47121278505446e-05\\
666	7.27012320787928e-05\\
667	7.20897707489406e-05\\
668	6.87851916637991e-05\\
669	6.44746641178822e-05\\
670	6.99916486233747e-05\\
671	6.75484778840058e-05\\
672	6.16886453526071e-05\\
673	5.8917158740168e-05\\
674	5.67484509443058e-05\\
675	5.73748322340087e-05\\
676	5.81062029982691e-05\\
677	5.64312126283615e-05\\
678	5.53559125738917e-05\\
679	5.90986315628772e-05\\
680	5.55559110569386e-05\\
681	5.82721279569628e-05\\
682	6.21095588944315e-05\\
683	5.64845955440097e-05\\
684	5.76399467925915e-05\\
685	4.89744928566898e-05\\
686	4.76528709387545e-05\\
687	3.95602520747004e-05\\
688	3.31550026127963e-05\\
689	3.31888607522197e-05\\
690	2.93427311107386e-05\\
691	3.07059902142563e-05\\
692	3.00217907193947e-05\\
693	2.8252255973199e-05\\
694	3.2820352198025e-05\\
695	3.00177752419589e-05\\
696	3.14967645608449e-05\\
697	2.60993376894516e-05\\
698	2.70001582713037e-05\\
699	2.7613704413347e-05\\
700	2.84393304440618e-05\\
701	2.69593003385343e-05\\
702	2.56489539627235e-05\\
703	2.26974747761495e-05\\
704	2.33989028561725e-05\\
705	2.34381168380988e-05\\
706	2.25338129045854e-05\\
707	2.00431209417808e-05\\
708	2.13448978349842e-05\\
709	2.39051182071005e-05\\
710	2.52150486748399e-05\\
711	2.73708970664695e-05\\
712	2.96631077467373e-05\\
713	3.09210230103488e-05\\
714	3.25912556177923e-05\\
715	2.97632864924101e-05\\
716	2.79129446959562e-05\\
717	3.00716264136327e-05\\
718	2.85320284301783e-05\\
719	2.90847424788295e-05\\
720	2.7892643100102e-05\\
721	3.07330554241469e-05\\
722	2.95682677180706e-05\\
723	3.10324693473614e-05\\
724	3.29885846469096e-05\\
725	3.08727371404847e-05\\
726	2.84585271259398e-05\\
727	2.9166616986773e-05\\
728	2.95685612432888e-05\\
729	2.946353796044e-05\\
730	3.30217152081558e-05\\
731	3.19678943275369e-05\\
732	3.38014729373286e-05\\
733	3.63402048734931e-05\\
734	4.01972574421537e-05\\
735	4.31536048826264e-05\\
736	4.38343396018579e-05\\
737	4.22904343082624e-05\\
738	4.230519667831e-05\\
739	4.33634804922537e-05\\
740	4.22373943052883e-05\\
741	4.13784209169358e-05\\
742	4.09458845901629e-05\\
743	4.02691354046451e-05\\
744	3.89966540850465e-05\\
745	3.95251857446178e-05\\
746	4.11909152221259e-05\\
747	4.46787465822735e-05\\
748	4.4285061860857e-05\\
749	4.06395394182326e-05\\
750	3.91149865349767e-05\\
751	3.82102099042194e-05\\
752	3.77647223235917e-05\\
753	3.8947907780441e-05\\
754	3.86814028103741e-05\\
755	3.79894850809504e-05\\
756	3.70839931282614e-05\\
757	3.50255912667879e-05\\
758	3.49228600884595e-05\\
759	3.35976807976409e-05\\
760	3.6199743752643e-05\\
761	3.4192100850856e-05\\
762	3.33803305561497e-05\\
763	3.28201174401764e-05\\
764	3.40783423089665e-05\\
765	3.16983081270274e-05\\
766	2.91070731317324e-05\\
767	2.96306960541061e-05\\
768	3.04451143911178e-05\\
769	3.0435331399727e-05\\
770	2.88386192940186e-05\\
771	3.10425552245766e-05\\
772	3.2889520934193e-05\\
773	3.42653149962158e-05\\
774	3.20610062480754e-05\\
775	3.14730629309934e-05\\
776	3.12665860368003e-05\\
777	3.14565489555508e-05\\
778	3.1729120882381e-05\\
779	3.10431814626992e-05\\
780	2.99542484663128e-05\\
781	2.88425315327285e-05\\
782	2.82643609820238e-05\\
783	2.57186852939125e-05\\
784	2.35744198340445e-05\\
785	2.19037806955986e-05\\
786	2.23048055111696e-05\\
787	2.10525740682374e-05\\
788	2.05840529993332e-05\\
789	1.95576529905134e-05\\
790	1.98159972944342e-05\\
791	2.03135883899707e-05\\
792	1.97623520823827e-05\\
793	1.95404787974173e-05\\
794	1.96878520512653e-05\\
795	1.89544490633801e-05\\
796	2.08369661772015e-05\\
797	2.03516498010722e-05\\
798	2.20394190529142e-05\\
799	2.10020503045138e-05\\
800	1.97337330287273e-05\\
801	2.06101381435211e-05\\
802	1.91193948094671e-05\\
803	2.01144036105238e-05\\
804	1.87938005291585e-05\\
805	2.10205333498127e-05\\
806	2.13354236216003e-05\\
807	2.20703679232933e-05\\
808	2.23612316569657e-05\\
809	2.14462359955253e-05\\
810	2.06445209092903e-05\\
811	2.24040549488557e-05\\
812	2.0834639817826e-05\\
813	2.13494398605724e-05\\
814	2.20935993645688e-05\\
815	2.1939579313755e-05\\
816	2.11246139637588e-05\\
817	1.98255774531769e-05\\
818	1.93104294260744e-05\\
819	1.80461459803257e-05\\
820	1.84288068153313e-05\\
821	1.76653449676637e-05\\
822	1.79431165347439e-05\\
823	1.83638432876389e-05\\
824	1.96081948895342e-05\\
825	2.02656278301613e-05\\
826	2.03086499198559e-05\\
827	2.03222019543217e-05\\
828	2.02241891587994e-05\\
829	1.94577674523524e-05\\
830	1.89273573312562e-05\\
831	1.9300327809841e-05\\
832	1.94440635741813e-05\\
833	1.91540555001621e-05\\
834	1.93884155791206e-05\\
835	1.90010581678971e-05\\
836	1.82616260734706e-05\\
837	1.77666032823489e-05\\
838	1.76142496630827e-05\\
839	1.79380752202672e-05\\
840	1.82051622326269e-05\\
841	1.75921007265314e-05\\
842	1.85556975437587e-05\\
843	1.85917894107156e-05\\
844	1.95793022478944e-05\\
845	1.90714136474774e-05\\
846	1.97939918626741e-05\\
847	1.9198310851429e-05\\
848	1.88603679861828e-05\\
849	1.87521394446405e-05\\
850	1.86866560925193e-05\\
851	1.80951907259818e-05\\
852	1.77034749793937e-05\\
853	1.75816974136049e-05\\
854	1.67041521645783e-05\\
855	1.55500088605556e-05\\
856	1.50002339679811e-05\\
857	1.52029761077398e-05\\
858	1.49233826918022e-05\\
859	1.41370504389476e-05\\
860	1.37057706149878e-05\\
861	1.37420159232751e-05\\
862	1.36709942336153e-05\\
863	1.24894242076866e-05\\
864	1.20668838500566e-05\\
865	1.30625344947718e-05\\
866	1.27610424640635e-05\\
867	1.28095993954154e-05\\
868	1.31941217168982e-05\\
869	1.21810676081502e-05\\
870	1.13583426368511e-05\\
871	1.06521637306523e-05\\
872	1.12814174593526e-05\\
873	1.08458836212632e-05\\
874	1.1005752414431e-05\\
875	1.06483023530343e-05\\
876	1.00730852720022e-05\\
877	9.71337987709557e-06\\
878	9.96347161578478e-06\\
879	9.56765720273837e-06\\
880	9.53730656822305e-06\\
881	9.93678779586934e-06\\
882	1.0012043951927e-05\\
883	9.90468504788548e-06\\
884	9.85983347851259e-06\\
885	1.00110292071551e-05\\
886	9.97981317699196e-06\\
887	9.96196978084963e-06\\
888	1.03031186723968e-05\\
889	1.0641819529406e-05\\
890	1.05010128879297e-05\\
891	1.0798547579172e-05\\
892	1.06919949533967e-05\\
893	1.07706414572877e-05\\
894	1.03646871972802e-05\\
895	1.04370851286634e-05\\
896	1.04536025101944e-05\\
897	9.7744522917691e-06\\
898	1.03503288877133e-05\\
899	1.0308417642567e-05\\
900	1.08891117442381e-05\\
901	1.06979021458889e-05\\
902	1.11459907653119e-05\\
903	1.05160389515853e-05\\
904	1.0310043630494e-05\\
905	1.03575040326496e-05\\
906	9.93712100992644e-06\\
907	9.96396579485141e-06\\
908	9.3042388244346e-06\\
909	9.65986638201264e-06\\
910	9.19543998488677e-06\\
911	9.03790390467121e-06\\
912	8.69103458289946e-06\\
913	8.59913755109627e-06\\
914	8.23274883321562e-06\\
915	7.96795837785473e-06\\
916	7.74289860721905e-06\\
917	7.89508754851876e-06\\
918	7.6319088567058e-06\\
919	7.73058057814883e-06\\
920	7.60274310029158e-06\\
921	7.51823156025794e-06\\
922	7.01289391050497e-06\\
923	6.85005279608816e-06\\
924	6.84362305181167e-06\\
925	6.60725206592466e-06\\
926	6.25755772823033e-06\\
927	6.17692273976672e-06\\
928	6.52552543613012e-06\\
929	6.6730670609766e-06\\
930	6.41923051148408e-06\\
931	6.23806562727385e-06\\
932	6.16393566811303e-06\\
933	5.82169955413128e-06\\
934	5.42035522448975e-06\\
935	5.58834501726319e-06\\
936	5.53799362883164e-06\\
937	5.52103345703559e-06\\
938	5.47135927425097e-06\\
939	5.22301697702103e-06\\
940	5.38094568414549e-06\\
941	5.05530067084327e-06\\
942	5.03505549334645e-06\\
943	4.91663167002092e-06\\
944	5.00681002535452e-06\\
945	5.18389591771563e-06\\
946	5.15725396075675e-06\\
947	5.16715329846726e-06\\
948	5.27763669664931e-06\\
949	5.23824182424482e-06\\
950	4.96848842925583e-06\\
951	5.02616850114062e-06\\
952	4.74199863802467e-06\\
953	5.00872184981011e-06\\
954	4.7599299788923e-06\\
955	4.78196769136423e-06\\
956	5.05877797974604e-06\\
957	5.26900506524636e-06\\
958	5.55621249815075e-06\\
959	5.49914688433428e-06\\
960	4.82974150818606e-06\\
961	4.84391905321582e-06\\
962	4.68286527998583e-06\\
963	4.73962389671714e-06\\
964	4.76649132140546e-06\\
965	4.5691475275351e-06\\
966	4.3795262065558e-06\\
967	4.41642233377664e-06\\
968	4.21498938630605e-06\\
969	4.39090316897441e-06\\
970	4.24198671644912e-06\\
971	3.99124464415057e-06\\
972	4.0192143956607e-06\\
973	4.12420208944678e-06\\
974	4.01834168878155e-06\\
975	3.84117119672537e-06\\
976	3.88734151104951e-06\\
977	3.85707079498027e-06\\
978	3.61429450436457e-06\\
979	3.48823415954877e-06\\
980	3.7120582108865e-06\\
981	3.78899084432534e-06\\
982	3.86533547741944e-06\\
983	3.9078812955366e-06\\
984	3.9222296409862e-06\\
985	3.82428023680694e-06\\
986	3.69043252483175e-06\\
987	3.63001024758559e-06\\
988	3.44642892272469e-06\\
989	3.43803831151136e-06\\
990	3.56834674278888e-06\\
991	3.65383886999272e-06\\
992	3.38766011425411e-06\\
993	3.26506474125748e-06\\
994	3.34128447161712e-06\\
995	3.33346654681726e-06\\
996	3.44322034330632e-06\\
997	3.49489628064283e-06\\
998	3.60058924869438e-06\\
999	3.55944638606163e-06\\
1000	3.53639259896597e-06\\
};
\addlegendentry{$\Im \{ \mathbf{g}_1^H \}$};

\addplot [color=mycolor5,solid]
  table[row sep=crcr]{%
1	0\\
2	0\\
3	0\\
4	0.0190294580473175\\
5	0.031991364937328\\
6	0.0478910991455361\\
7	0.0612554111750119\\
8	0.0737263642383762\\
9	0.0847334718863975\\
10	0.089256707637721\\
11	0.102994414579083\\
12	0.117349942393817\\
13	0.134707858090599\\
14	0.147681632216975\\
15	0.161651705826203\\
16	0.176888032837748\\
17	0.193022799755833\\
18	0.20314501048947\\
19	0.216910260067176\\
20	0.230593608963045\\
21	0.238837943214839\\
22	0.251149251025328\\
23	0.263770287923386\\
24	0.275649793231069\\
25	0.285857501779259\\
26	0.297859186585636\\
27	0.310181198102162\\
28	0.32328152751021\\
29	0.335420542488299\\
30	0.347544216413303\\
31	0.359877633349178\\
32	0.368447113552785\\
33	0.376553189019629\\
34	0.388739431132306\\
35	0.399482473832352\\
36	0.407710826495721\\
37	0.417897739719764\\
38	0.429307236335303\\
39	0.44431269975996\\
40	0.454632219991287\\
41	0.463717443148292\\
42	0.472970530774085\\
43	0.483389968677642\\
44	0.493313878572109\\
45	0.505660789248431\\
46	0.518259655820114\\
47	0.527137078793379\\
48	0.535258769199481\\
49	0.545893707490415\\
50	0.5562164550236\\
51	0.564938495202437\\
52	0.572005872321647\\
53	0.580562555255296\\
54	0.588704658327679\\
55	0.598005589153934\\
56	0.611453484518769\\
57	0.622772901742294\\
58	0.62963842542681\\
59	0.637144915725444\\
60	0.644128034073221\\
61	0.656889337784811\\
62	0.664263655240056\\
63	0.670812247400371\\
64	0.680340032851471\\
65	0.688338221977241\\
66	0.69612282061895\\
67	0.705519577308563\\
68	0.713505812240725\\
69	0.722381966184384\\
70	0.732256158759372\\
71	0.739929332951929\\
72	0.748030166610583\\
73	0.754839699145694\\
74	0.761966359691723\\
75	0.768120708219634\\
76	0.774228567961987\\
77	0.780980790488435\\
78	0.78936873868564\\
79	0.799887422247599\\
80	0.804235397306675\\
81	0.81407363816179\\
82	0.821796670531222\\
83	0.828917472342546\\
84	0.835438257403378\\
85	0.841416667113197\\
86	0.84833794745456\\
87	0.856904502485668\\
88	0.863751987191776\\
89	0.869297967856352\\
90	0.875615339331156\\
91	0.880931330297294\\
92	0.888024831349678\\
93	0.894610038748959\\
94	0.900613220521165\\
95	0.905507671654154\\
96	0.911765003829775\\
97	0.918709823273057\\
98	0.924331339313138\\
99	0.933929722463352\\
100	0.938235848605249\\
101	0.944875276917135\\
102	0.949934544506126\\
103	0.956119504472553\\
104	0.962808027582163\\
105	0.966599231522148\\
106	0.970688218560818\\
107	0.978251266842809\\
108	0.984738304588724\\
109	0.990436849471024\\
110	0.995659175605109\\
111	1.0017353852539\\
112	1.00518975593814\\
113	1.01128353694921\\
114	1.01520149796478\\
115	1.02179244827493\\
116	1.02711382230268\\
117	1.03080632223539\\
118	1.03580209826121\\
119	1.04002866531438\\
120	1.04560172116412\\
121	1.04941719399808\\
122	1.05305042817244\\
123	1.05594873995467\\
124	1.05979485698675\\
125	1.06365037990131\\
126	1.06763348544445\\
127	1.07140407347364\\
128	1.07535859810428\\
129	1.0795944796597\\
130	1.08436308380037\\
131	1.08875093404256\\
132	1.09507501676935\\
133	1.09965188607842\\
134	1.10388117649191\\
135	1.1068823783471\\
136	1.11244155117608\\
137	1.11658257288963\\
138	1.12090637720452\\
139	1.12486059919841\\
140	1.13002268569454\\
141	1.13391852694581\\
142	1.13671473489277\\
143	1.1400666270394\\
144	1.14505387412077\\
145	1.14761394029303\\
146	1.15126224134356\\
147	1.15468600744476\\
148	1.1571218344345\\
149	1.16044265814739\\
150	1.16442174735202\\
151	1.16818997780549\\
152	1.17120670613469\\
153	1.1746919143783\\
154	1.17760882683876\\
155	1.18080552920156\\
156	1.18399219114566\\
157	1.18714343169866\\
158	1.19248360889814\\
159	1.19503212507323\\
160	1.19730743520055\\
161	1.20010899496198\\
162	1.202924304073\\
163	1.20630701295505\\
164	1.20979280361881\\
165	1.21282442312444\\
166	1.21669303920277\\
167	1.21888681194753\\
168	1.22067249396444\\
169	1.2232158747802\\
170	1.22524949849949\\
171	1.22729073731818\\
172	1.22922555609931\\
173	1.2313313062633\\
174	1.23373129384207\\
175	1.23636556707416\\
176	1.23884875450678\\
177	1.24085280714902\\
178	1.24277420272744\\
179	1.24563098112198\\
180	1.24753221863909\\
181	1.25000378165408\\
182	1.25271839360882\\
183	1.25493944958444\\
184	1.25725451774258\\
185	1.25963200656252\\
186	1.26193317063158\\
187	1.26376354489724\\
188	1.26655158412139\\
189	1.2691405052338\\
190	1.27050866867489\\
191	1.27308707894656\\
192	1.27470289128068\\
193	1.2764322309794\\
194	1.27887110588501\\
195	1.28076052597325\\
196	1.28270115007912\\
197	1.28395227247303\\
198	1.28603808987346\\
199	1.28782263428642\\
200	1.29018635212583\\
201	1.29212637153956\\
202	1.29406131791055\\
203	1.2961264078881\\
204	1.29809884314861\\
205	1.30021993564345\\
206	1.3027105823678\\
207	1.30464349046265\\
208	1.30677840405727\\
209	1.30820454411991\\
210	1.30939481762167\\
211	1.31085349670371\\
212	1.31325256461859\\
213	1.31551189484641\\
214	1.31773001580101\\
215	1.31982051636367\\
216	1.3220624732593\\
217	1.32348094316997\\
218	1.32525509075266\\
219	1.32721894927512\\
220	1.32924738241661\\
221	1.33152013681839\\
222	1.33330516767628\\
223	1.33529290390349\\
224	1.33655337097064\\
225	1.33834161298342\\
226	1.34014706671965\\
227	1.34200828493217\\
228	1.34361485614158\\
229	1.34461449664119\\
230	1.3462783606285\\
231	1.34738814418309\\
232	1.34866239674327\\
233	1.35006258914161\\
234	1.35131540646772\\
235	1.3525002175777\\
236	1.35349177548169\\
237	1.35464955814794\\
238	1.35678905521554\\
239	1.35866332081974\\
240	1.3599682847083\\
241	1.36194161809845\\
242	1.36325735107896\\
243	1.36475854288037\\
244	1.36620444400419\\
245	1.36789693633997\\
246	1.36933940745549\\
247	1.37100450185477\\
248	1.37314538855509\\
249	1.37463059537116\\
250	1.37620024080712\\
251	1.37722105606521\\
252	1.37884840149132\\
253	1.38012896134575\\
254	1.38145319326376\\
255	1.3826124550039\\
256	1.38361532252049\\
257	1.38502202818838\\
258	1.38625340050057\\
259	1.38747386977979\\
260	1.38847852207428\\
261	1.38974600921143\\
262	1.39080550329632\\
263	1.39178567480904\\
264	1.39260603502386\\
265	1.39404276049535\\
266	1.39489034637883\\
267	1.39612019582984\\
268	1.39715650646767\\
269	1.39782571626226\\
270	1.39895980516614\\
271	1.39970473770407\\
272	1.40101399101236\\
273	1.40190349146791\\
274	1.4027218805042\\
275	1.40389158198902\\
276	1.4048358982472\\
277	1.40568942787703\\
278	1.40653262679422\\
279	1.40733164871643\\
280	1.40858198321744\\
281	1.40948466018948\\
282	1.41051067995185\\
283	1.41075132022167\\
284	1.4114789679583\\
285	1.41232762391851\\
286	1.41339453053722\\
287	1.4144149864452\\
288	1.41515113748415\\
289	1.41626092537153\\
290	1.41714906605503\\
291	1.41800299722188\\
292	1.4189419947774\\
293	1.4195818784192\\
294	1.42038122695513\\
295	1.42097983986435\\
296	1.42209483467897\\
297	1.42291163498206\\
298	1.42356808831045\\
299	1.42441140567919\\
300	1.42507315017791\\
301	1.4255695340289\\
302	1.42629692354553\\
303	1.42739475905905\\
304	1.42809463432937\\
305	1.4287887294118\\
306	1.42948124941632\\
307	1.43024689624337\\
308	1.43116130325786\\
309	1.43195662838743\\
310	1.43242680399444\\
311	1.43317483636223\\
312	1.43392935720985\\
313	1.43470896297557\\
314	1.43546447243386\\
315	1.4359183421407\\
316	1.43659322738356\\
317	1.43727461849705\\
318	1.43802538470567\\
319	1.43864795632128\\
320	1.43943828750151\\
321	1.44015087762993\\
322	1.44063295237302\\
323	1.44137420035346\\
324	1.44210426352622\\
325	1.44259957230813\\
326	1.44306113461907\\
327	1.44368332805713\\
328	1.44431291964943\\
329	1.44487762163859\\
330	1.44528474651972\\
331	1.44559874182084\\
332	1.4458276582647\\
333	1.44651990010964\\
334	1.447018426963\\
335	1.44757279148993\\
336	1.44808110675994\\
337	1.44864909840813\\
338	1.44917313446797\\
339	1.44968450229158\\
340	1.45023010042361\\
341	1.45082195290799\\
342	1.45145830920112\\
343	1.45177579283651\\
344	1.45207264888289\\
345	1.4524719656969\\
346	1.45301380037503\\
347	1.45354837468266\\
348	1.45409360967226\\
349	1.45438584155478\\
350	1.45486821149045\\
351	1.45543646203762\\
352	1.45578837742712\\
353	1.45607809501021\\
354	1.45654967293022\\
355	1.45713764294368\\
356	1.45772385009274\\
357	1.45812954994556\\
358	1.45872676873812\\
359	1.4591966985334\\
360	1.45962547146771\\
361	1.46006745154323\\
362	1.46046522345953\\
363	1.46082135862231\\
364	1.46132872634021\\
365	1.46171598443879\\
366	1.46207492911461\\
367	1.46240518946261\\
368	1.46284912038731\\
369	1.46323412056841\\
370	1.4635468220251\\
371	1.46392677697632\\
372	1.46417485062921\\
373	1.46450362871222\\
374	1.46484531874051\\
375	1.4651679589713\\
376	1.46554873902216\\
377	1.46586690979983\\
378	1.46625415513837\\
379	1.46648786691933\\
380	1.46685489726185\\
381	1.46732810854406\\
382	1.46759427983268\\
383	1.46797609360967\\
384	1.46832439407955\\
385	1.46862918676894\\
386	1.46905449402126\\
387	1.46933255815842\\
388	1.46971048353913\\
389	1.47007507760613\\
390	1.4703164566988\\
391	1.4705637935188\\
392	1.4708698482242\\
393	1.47110313089906\\
394	1.4714099340108\\
395	1.47176763442722\\
396	1.47204333227166\\
397	1.47229763637367\\
398	1.47266058138216\\
399	1.47293366085639\\
400	1.47313989835304\\
401	1.4734203698968\\
402	1.47375217367075\\
403	1.47408496345076\\
404	1.47438750433967\\
405	1.47471229502548\\
406	1.4750449714535\\
407	1.47524145775036\\
408	1.47549987161904\\
409	1.47568114865185\\
410	1.47597117693667\\
411	1.47619812747204\\
412	1.4764251572999\\
413	1.47669124355088\\
414	1.47689829148277\\
415	1.47715851563112\\
416	1.47728673744078\\
417	1.47746743929895\\
418	1.47763294299722\\
419	1.47784953139142\\
420	1.4780771951928\\
421	1.47831502903325\\
422	1.47847986067532\\
423	1.47872370213877\\
424	1.47898071679134\\
425	1.47924043211072\\
426	1.47943394461003\\
427	1.47968743366177\\
428	1.47989548696178\\
429	1.4800756973582\\
430	1.48028442067683\\
431	1.48051559036157\\
432	1.48073111339761\\
433	1.48083017378205\\
434	1.48102303432772\\
435	1.48122472472899\\
436	1.48141051519076\\
437	1.48157673634069\\
438	1.4818144387478\\
439	1.48195297031034\\
440	1.48217070794977\\
441	1.48239358914209\\
442	1.48255571561941\\
443	1.48272921199911\\
444	1.48286747823\\
445	1.48299472212556\\
446	1.4831323977937\\
447	1.48328187674032\\
448	1.48347326324628\\
449	1.48357804310422\\
450	1.48378088116581\\
451	1.48396103597483\\
452	1.48409394120551\\
453	1.48425284857231\\
454	1.48437780848397\\
455	1.48451984305339\\
456	1.48472321111994\\
457	1.48480551682644\\
458	1.48493098251405\\
459	1.48503443493921\\
460	1.48521697980432\\
461	1.48541109971814\\
462	1.48556154998732\\
463	1.48570181439081\\
464	1.48580101530749\\
465	1.48593666267731\\
466	1.48606015679764\\
467	1.48626643016565\\
468	1.48641371431802\\
469	1.48655357161595\\
470	1.48668381350548\\
471	1.48680641319224\\
472	1.48694320089078\\
473	1.48707086906305\\
474	1.48716177127893\\
475	1.48728744149726\\
476	1.48740141726304\\
477	1.48751018596002\\
478	1.48764584744084\\
479	1.48774356381817\\
480	1.48785250703802\\
481	1.48799077491376\\
482	1.48811676509749\\
483	1.48821272325905\\
484	1.48832908509976\\
485	1.48853792341966\\
486	1.48865309531782\\
487	1.48871759068115\\
488	1.48884006780245\\
489	1.48894021687002\\
490	1.48904059693458\\
491	1.48912268335515\\
492	1.4892046095496\\
493	1.48934889893301\\
494	1.48943844731475\\
495	1.48955105820953\\
496	1.4896422914402\\
497	1.48972513083539\\
498	1.4897887781792\\
499	1.48989283148409\\
500	1.48999480179607\\
501	1.49009345567652\\
502	1.49017993678629\\
503	1.49025447152516\\
504	1.49035653525441\\
505	1.49041821656164\\
506	1.49049154349054\\
507	1.49058171927449\\
508	1.49066656334863\\
509	1.49074083864403\\
510	1.49085167281782\\
511	1.49094822694602\\
512	1.49105070488445\\
513	1.49113498682388\\
514	1.49121720241447\\
515	1.49131442146294\\
516	1.49141134845473\\
517	1.49149520870774\\
518	1.49163072533793\\
519	1.49170687262215\\
520	1.49180389165917\\
521	1.49190475549658\\
522	1.49195832909071\\
523	1.49206165046122\\
524	1.4921159183569\\
525	1.49218994844043\\
526	1.49227647283417\\
527	1.49238980363304\\
528	1.49245107905952\\
529	1.49253094866621\\
530	1.49259548800802\\
531	1.49268999974779\\
532	1.49274385593025\\
533	1.49282316278274\\
534	1.4928826373895\\
535	1.49293195718862\\
536	1.49300260073186\\
537	1.49308534073148\\
538	1.4931426591591\\
539	1.49324009209734\\
540	1.49330329700301\\
541	1.49338249531182\\
542	1.49344654141279\\
543	1.49348068549119\\
544	1.49355066759835\\
545	1.49361519020269\\
546	1.49366906140492\\
547	1.49371814104422\\
548	1.49377439400698\\
549	1.49384033141157\\
550	1.49390761821393\\
551	1.49399953235623\\
552	1.49407749699442\\
553	1.49413304120236\\
554	1.49418866829575\\
555	1.49423976892947\\
556	1.49428762175879\\
557	1.49434569888115\\
558	1.49441484903982\\
559	1.49447704704118\\
560	1.49453054695571\\
561	1.49460006437015\\
562	1.49466664262618\\
563	1.49471708889124\\
564	1.49476404536085\\
565	1.49482540246598\\
566	1.49488666339871\\
567	1.49494128489974\\
568	1.49499224664304\\
569	1.49505155521107\\
570	1.49507917550457\\
571	1.49513441341853\\
572	1.49517279566187\\
573	1.49521647927984\\
574	1.49525768067833\\
575	1.49530445653968\\
576	1.4953607709598\\
577	1.49541257794482\\
578	1.49544360859468\\
579	1.4955053980258\\
580	1.4955370253725\\
581	1.49559124765943\\
582	1.49564390109359\\
583	1.49568707147755\\
584	1.495717675394\\
585	1.49576439682437\\
586	1.49580477499414\\
587	1.49583806784391\\
588	1.49587662931113\\
589	1.49590720577691\\
590	1.4959404220322\\
591	1.49598553840727\\
592	1.49602005891675\\
593	1.49606874922016\\
594	1.49611264061683\\
595	1.49615684351311\\
596	1.49618417996593\\
597	1.49623541611198\\
598	1.49626756205201\\
599	1.4962993982765\\
600	1.4963531160397\\
601	1.49639006213232\\
602	1.49641400687806\\
603	1.4964435095303\\
604	1.49647745800065\\
605	1.49652019329749\\
606	1.49655725194107\\
607	1.49658072612924\\
608	1.49662304130296\\
609	1.49665917458\\
610	1.49668771520564\\
611	1.49673551844036\\
612	1.49676668812479\\
613	1.49678851111534\\
614	1.49683137180927\\
615	1.49685833175931\\
616	1.4968896345882\\
617	1.49693483285107\\
618	1.49696973765113\\
619	1.49699645803221\\
620	1.49703506042284\\
621	1.49706398185088\\
622	1.49708415428539\\
623	1.49712154173828\\
624	1.49714371971265\\
625	1.49717466451461\\
626	1.49719532536511\\
627	1.49722217692843\\
628	1.49725562625804\\
629	1.49728866036612\\
630	1.49731346965531\\
631	1.49734304392533\\
632	1.4973862683621\\
633	1.49741863321131\\
634	1.49744388960327\\
635	1.49746649247738\\
636	1.49748717664779\\
637	1.49751892275968\\
638	1.49754019205333\\
639	1.49756841834836\\
640	1.49758663712237\\
641	1.49760459285533\\
642	1.49764425579287\\
643	1.4976706130228\\
644	1.49770042956428\\
645	1.49772421279726\\
646	1.49775654829808\\
647	1.497786505681\\
648	1.49780802157156\\
649	1.49782484388106\\
650	1.49784055083305\\
651	1.49786000474867\\
652	1.49787280721665\\
653	1.4978952127335\\
654	1.49791930464679\\
655	1.4979363991124\\
656	1.49795967645611\\
657	1.49798117814984\\
658	1.49800600104627\\
659	1.49802261474792\\
660	1.49803809363787\\
661	1.49805920663875\\
662	1.49807329856122\\
663	1.4980920052363\\
664	1.4981052683025\\
665	1.49812334772244\\
666	1.49813395020068\\
667	1.49815450016514\\
668	1.49817466626266\\
669	1.49818960040827\\
670	1.49821359862371\\
671	1.49822946995732\\
672	1.49825089822272\\
673	1.49826652771132\\
674	1.49828969562582\\
675	1.49830526640289\\
676	1.49832457219155\\
677	1.49834401676992\\
678	1.49835879422039\\
679	1.49836897576955\\
680	1.49838545839889\\
681	1.49840344274479\\
682	1.49841781843273\\
683	1.49843442330939\\
684	1.49844603868486\\
685	1.49845901491142\\
686	1.49847167100918\\
687	1.49849031958742\\
688	1.49850400568224\\
689	1.49852014673457\\
690	1.49853736419591\\
691	1.49856130584807\\
692	1.49857362999481\\
693	1.4985859241711\\
694	1.49860405412783\\
695	1.4986153603825\\
696	1.49862805112182\\
697	1.49864007093989\\
698	1.49865554212583\\
699	1.49866575603401\\
700	1.49867886598756\\
701	1.49869102851322\\
702	1.49870759877451\\
703	1.4987235386256\\
704	1.49874043195726\\
705	1.49875422879715\\
706	1.49876475771488\\
707	1.49877800280945\\
708	1.49878704228696\\
709	1.49879802149399\\
710	1.49880783056246\\
711	1.49882282789733\\
712	1.49883452445008\\
713	1.49884671612981\\
714	1.49885398409723\\
715	1.498866605357\\
716	1.49888226274727\\
717	1.49889523553782\\
718	1.49890569952293\\
719	1.49891835863167\\
720	1.49892733411446\\
721	1.49893652442929\\
722	1.49894778655971\\
723	1.49895234700381\\
724	1.49896377870677\\
725	1.49897375634327\\
726	1.49898657594986\\
727	1.49899905410933\\
728	1.49900615360165\\
729	1.49901566545176\\
730	1.49902521983415\\
731	1.4990388527519\\
732	1.49905269919136\\
733	1.49906279099607\\
734	1.49906880844829\\
735	1.49907561998825\\
736	1.49908087992021\\
737	1.49908941764725\\
738	1.49909729260971\\
739	1.49910768124222\\
740	1.49911346118904\\
741	1.49912002653931\\
742	1.49912835885026\\
743	1.49913587463833\\
744	1.49914551602448\\
745	1.49915215761378\\
746	1.49916154241369\\
747	1.4991714213277\\
748	1.4991785893977\\
749	1.49918695820402\\
750	1.49919737568639\\
751	1.49920438807197\\
752	1.49921345523089\\
753	1.4992227302647\\
754	1.49922932575303\\
755	1.49923783633516\\
756	1.49924854490278\\
757	1.49925704416958\\
758	1.49926560167955\\
759	1.49927328043218\\
760	1.49928209900424\\
761	1.49929148234108\\
762	1.49929946871742\\
763	1.4993066360795\\
764	1.49931284962814\\
765	1.49931882531809\\
766	1.49932600778427\\
767	1.49933396915374\\
768	1.49934081660461\\
769	1.49934740170144\\
770	1.49935522307787\\
771	1.49936165035935\\
772	1.4993687326193\\
773	1.49937676292294\\
774	1.49938279082489\\
775	1.4993893492984\\
776	1.49939552457284\\
777	1.49940243720506\\
778	1.49940781368093\\
779	1.49941408296705\\
780	1.49942103738262\\
781	1.49942586221078\\
782	1.49943088835371\\
783	1.49943783979457\\
784	1.49944291793737\\
785	1.49944763499376\\
786	1.49945231267125\\
787	1.49945698789937\\
788	1.49946289458344\\
789	1.4994676394177\\
790	1.49947354277088\\
791	1.49947936979003\\
792	1.49948639808574\\
793	1.49949165003125\\
794	1.49949617673967\\
795	1.49950051721699\\
796	1.49950447779624\\
797	1.49950897272447\\
798	1.4995141275772\\
799	1.4995177514803\\
800	1.49952313535995\\
801	1.49952797180664\\
802	1.49953222555923\\
803	1.499535325555\\
804	1.49954028431111\\
805	1.49954507618589\\
806	1.49954994898505\\
807	1.49955360993774\\
808	1.49955670384047\\
809	1.49955991790135\\
810	1.49956453554087\\
811	1.49956927966972\\
812	1.49957491936213\\
813	1.49957853787356\\
814	1.4995824774644\\
815	1.4995871017695\\
816	1.49959054925491\\
817	1.49959331441139\\
818	1.49959766888603\\
819	1.49960208219177\\
820	1.49960594044181\\
821	1.4996093638936\\
822	1.49961373151161\\
823	1.49961753317413\\
824	1.49962231675134\\
825	1.49962559939807\\
826	1.49962919761467\\
827	1.49963232948655\\
828	1.49963503615087\\
829	1.49963699473682\\
830	1.49964103339293\\
831	1.49964544752108\\
832	1.49964777569408\\
833	1.49965170747829\\
834	1.49965537350908\\
835	1.49966015765679\\
836	1.49966422279508\\
837	1.4996671997972\\
838	1.49967041014783\\
839	1.49967224444118\\
840	1.49967538987608\\
841	1.49967858123361\\
842	1.49968234974243\\
843	1.49968497470196\\
844	1.49968925200945\\
845	1.49969181350323\\
846	1.49969351352682\\
847	1.49969602445791\\
848	1.49969814702946\\
849	1.49970024037834\\
850	1.49970333264292\\
851	1.49970635777825\\
852	1.49970966848859\\
853	1.49971107679121\\
854	1.49971371234165\\
855	1.49971572199022\\
856	1.49971757847548\\
857	1.49971981885518\\
858	1.49972158685395\\
859	1.4997251968604\\
860	1.4997296578581\\
861	1.49973290371947\\
862	1.49973461138321\\
863	1.49973755077922\\
864	1.49973941414091\\
865	1.49974143887115\\
866	1.49974498321305\\
867	1.49974680765571\\
868	1.49974862362477\\
869	1.49975076694\\
870	1.49975420106153\\
871	1.49975645879829\\
872	1.49975908482764\\
873	1.49976134048893\\
874	1.49976363609683\\
875	1.49976533547878\\
876	1.49976824860771\\
877	1.49976990246966\\
878	1.49977196694033\\
879	1.49977420449558\\
880	1.49977620789278\\
881	1.49977824155954\\
882	1.49978086244084\\
883	1.49978278679128\\
884	1.49978475924289\\
885	1.49978746687453\\
886	1.49978920013351\\
887	1.49979163999086\\
888	1.49979366162431\\
889	1.49979524043144\\
890	1.49979694997615\\
891	1.49979861617677\\
892	1.49980067891159\\
893	1.499801819984\\
894	1.49980351980321\\
895	1.49980551458097\\
896	1.4998072189039\\
897	1.4998087768067\\
898	1.49981058018206\\
899	1.49981227907049\\
900	1.49981454040536\\
901	1.49981654486684\\
902	1.49981815435986\\
903	1.49981971896106\\
904	1.49982097721716\\
905	1.49982295556943\\
906	1.49982525991862\\
907	1.49982661021506\\
908	1.49982766244512\\
909	1.49982861974653\\
910	1.4998302040823\\
911	1.49983155790653\\
912	1.49983256488933\\
913	1.49983393337757\\
914	1.49983559814917\\
915	1.49983725284699\\
916	1.49983847300174\\
917	1.49984024408585\\
918	1.49984177923701\\
919	1.49984269682744\\
920	1.49984447256336\\
921	1.49984589318095\\
922	1.49984778716167\\
923	1.49984922626068\\
924	1.49985082543232\\
925	1.49985232011891\\
926	1.49985499141421\\
927	1.49985649079044\\
928	1.49985806338891\\
929	1.4998597444024\\
930	1.49986087046371\\
931	1.49986250086957\\
932	1.49986383247938\\
933	1.49986525676328\\
934	1.49986618407452\\
935	1.49986749738902\\
936	1.49986871914507\\
937	1.49987005493858\\
938	1.49987144783297\\
939	1.49987299942399\\
940	1.49987426107414\\
941	1.49987497857727\\
942	1.49987607820584\\
943	1.49987738060113\\
944	1.49987864680334\\
945	1.49987975431888\\
946	1.49988116608404\\
947	1.49988251148931\\
948	1.49988376808728\\
949	1.49988470404851\\
950	1.49988585634752\\
951	1.49988718135013\\
952	1.4998881823141\\
953	1.49988945653359\\
954	1.49989029205399\\
955	1.49989145592254\\
956	1.49989239861763\\
957	1.49989379079289\\
958	1.4998950983839\\
959	1.49989629319826\\
960	1.49989736144391\\
961	1.4998986076733\\
962	1.4998992784681\\
963	1.49990058565883\\
964	1.49990196511343\\
965	1.49990271084598\\
966	1.49990375588884\\
967	1.49990453970308\\
968	1.49990549473577\\
969	1.49990604980558\\
970	1.49990689698798\\
971	1.49990823622073\\
972	1.49990911438293\\
973	1.49990988968644\\
974	1.4999107085563\\
975	1.49991210249993\\
976	1.49991309813383\\
977	1.49991423325038\\
978	1.49991496988125\\
979	1.49991603466809\\
980	1.49991673454704\\
981	1.49991715405723\\
982	1.49991802558614\\
983	1.49991906729323\\
984	1.49991983262686\\
985	1.499920379598\\
986	1.49992102201684\\
987	1.49992160448704\\
988	1.49992238524688\\
989	1.49992320279744\\
990	1.49992371902403\\
991	1.49992439337743\\
992	1.49992515491936\\
993	1.49992609338477\\
994	1.49992673858238\\
995	1.49992758433516\\
996	1.49992834466222\\
997	1.49992890075145\\
998	1.49992980931049\\
999	1.49993056446326\\
1000	1.49993143787691\\
};
\addlegendentry{$\Re \{ \mathbf{h}_2^H \}$};

\addplot [color=mycolor6,solid]
  table[row sep=crcr]{%
1	0\\
2	0\\
3	0\\
4	0.0109109225144983\\
5	0.0200105700822136\\
6	0.0307723634789804\\
7	0.0392758036344719\\
8	0.0475917528735678\\
9	0.0596709744544197\\
10	0.0697143934747991\\
11	0.0801463168604572\\
12	0.0871969338010564\\
13	0.0963913463691757\\
14	0.107586538481248\\
15	0.113635885148561\\
16	0.124910794559752\\
17	0.133984661515467\\
18	0.140843368064024\\
19	0.148373338384615\\
20	0.156836194727104\\
21	0.165846612758902\\
22	0.173585156678593\\
23	0.184534625405932\\
24	0.189749870992924\\
25	0.197007330656206\\
26	0.205961581005901\\
27	0.215498556839975\\
28	0.223590679144696\\
29	0.236248067274236\\
30	0.240480603917496\\
31	0.247879432200523\\
32	0.254030346659843\\
33	0.259986493635054\\
34	0.266572878125877\\
35	0.279270250522933\\
36	0.288335461662781\\
37	0.296770817691156\\
38	0.30553434963041\\
39	0.31155331932795\\
40	0.317157106638016\\
41	0.326233725782187\\
42	0.332053777054049\\
43	0.338350510906115\\
44	0.344749006561495\\
45	0.348752912647061\\
46	0.355781022692779\\
47	0.363401401844496\\
48	0.367037331389659\\
49	0.369764278519986\\
50	0.375939520369211\\
51	0.379934053280401\\
52	0.384833107660141\\
53	0.392533581895021\\
54	0.39627906324063\\
55	0.402554792317783\\
56	0.406887148919403\\
57	0.412591262609264\\
58	0.419784434276505\\
59	0.425200025932875\\
60	0.430530780085322\\
61	0.433874773577149\\
62	0.438532519820043\\
63	0.442225317323643\\
64	0.45128672095486\\
65	0.457425740229188\\
66	0.464166693136808\\
67	0.471148808083134\\
68	0.475527219356118\\
69	0.481928620677577\\
70	0.488123616802486\\
71	0.490604426656363\\
72	0.495201020170175\\
73	0.501582594921084\\
74	0.504903716832873\\
75	0.511683830473673\\
76	0.514757184238879\\
77	0.516841627625532\\
78	0.521416404583379\\
79	0.526705073588107\\
80	0.530746700335084\\
81	0.537593625986567\\
82	0.541465850394275\\
83	0.546732439047956\\
84	0.550196225840014\\
85	0.554409689486756\\
86	0.559085758635315\\
87	0.56330010923\\
88	0.568870494710228\\
89	0.573473020507066\\
90	0.577156430725653\\
91	0.581419356748032\\
92	0.585657584400246\\
93	0.590341313973717\\
94	0.594891681390385\\
95	0.598118941932502\\
96	0.600674454213761\\
97	0.604030213867579\\
98	0.609006783174954\\
99	0.611708332499581\\
100	0.614334095831222\\
101	0.617343293261497\\
102	0.620930447407228\\
103	0.624734607399458\\
104	0.627974069004901\\
105	0.630634477414677\\
106	0.634748959424133\\
107	0.639659186826232\\
108	0.643306552500546\\
109	0.648891327746552\\
110	0.653816296778856\\
111	0.657995648874148\\
112	0.662747458008903\\
113	0.66504441422326\\
114	0.668608179015216\\
115	0.671852936624707\\
116	0.675335550607655\\
117	0.679140919942926\\
118	0.68077603176276\\
119	0.68529022718001\\
120	0.688713684804203\\
121	0.69222454319804\\
122	0.694726786775243\\
123	0.697865453413914\\
124	0.70149367786931\\
125	0.704361922405755\\
126	0.707355391500051\\
127	0.710419853260459\\
128	0.713167586626316\\
129	0.7158518091322\\
130	0.719607450559096\\
131	0.72250222237773\\
132	0.725693681177468\\
133	0.727611068796013\\
134	0.731264166599767\\
135	0.733719087505375\\
136	0.737526638752789\\
137	0.739983640461118\\
138	0.742959005683481\\
139	0.745622038080323\\
140	0.748021048986504\\
141	0.750821452687649\\
142	0.754178662522309\\
143	0.757592365494801\\
144	0.760265083867558\\
145	0.764652522877035\\
146	0.767760874584438\\
147	0.769451536031788\\
148	0.771587019526525\\
149	0.77462016380478\\
150	0.776720528959705\\
151	0.778902191149876\\
152	0.781424493289564\\
153	0.784088804431705\\
154	0.785806158976299\\
155	0.78756855050451\\
156	0.789245673838249\\
157	0.790941717315379\\
158	0.793547392968245\\
159	0.795810323966682\\
160	0.79686163501016\\
161	0.800643789423283\\
162	0.802870889103917\\
163	0.805100945271032\\
164	0.806948406874124\\
165	0.808839952605919\\
166	0.810862893617338\\
167	0.812851769820932\\
168	0.814196264419916\\
169	0.815965161282383\\
170	0.818906076667058\\
171	0.82033571841828\\
172	0.822907661262409\\
173	0.824472308102597\\
174	0.826243226726813\\
175	0.827725150489998\\
176	0.830308402747034\\
177	0.833250948064786\\
178	0.835020900368778\\
179	0.83645036975167\\
180	0.838220972067263\\
181	0.84004137180057\\
182	0.840971567614418\\
183	0.842754528921754\\
184	0.84436646778971\\
185	0.845529428209994\\
186	0.846501207349368\\
187	0.848236384309404\\
188	0.849963486834932\\
189	0.851479200118771\\
190	0.852774179089627\\
191	0.854990583025214\\
192	0.856432523587785\\
193	0.857812689056446\\
194	0.859836549264292\\
195	0.860995400850905\\
196	0.862124905649146\\
197	0.863518953324962\\
198	0.865074749446035\\
199	0.866436187621135\\
200	0.867234909668759\\
201	0.86844425570263\\
202	0.870631096728019\\
203	0.872177076530681\\
204	0.8726950867677\\
205	0.874420646038257\\
206	0.875651447421569\\
207	0.876848554013329\\
208	0.878178415326461\\
209	0.87944385723842\\
210	0.880769374801385\\
211	0.881751921793253\\
212	0.88342863458753\\
213	0.884345011421682\\
214	0.885965550300969\\
215	0.886855551870678\\
216	0.887997900197091\\
217	0.88905469001295\\
218	0.890254551359335\\
219	0.891782984977058\\
220	0.893085017696444\\
221	0.894342195476174\\
222	0.895328860005843\\
223	0.896537516505634\\
224	0.897338802087107\\
225	0.898524480422278\\
226	0.899371200452335\\
227	0.900457644381702\\
228	0.901924132932047\\
229	0.902716095067014\\
230	0.903223419576351\\
231	0.904001970644596\\
232	0.904798981394955\\
233	0.906123625209449\\
234	0.907544240587861\\
235	0.908365760994647\\
236	0.909396762739957\\
237	0.90983856811719\\
238	0.911246204796486\\
239	0.91198079219436\\
240	0.912716803030714\\
241	0.914113682126636\\
242	0.914863753523297\\
243	0.915696647833637\\
244	0.916261689899939\\
245	0.917359586411444\\
246	0.918158321181927\\
247	0.918773855268346\\
248	0.920124951132955\\
249	0.920847341707312\\
250	0.921503404789438\\
251	0.922310571892278\\
252	0.923337446011181\\
253	0.92395793573095\\
254	0.924616843921665\\
255	0.925432200852884\\
256	0.925937633159147\\
257	0.926775230869661\\
258	0.927801606514316\\
259	0.928452065319263\\
260	0.929370251487805\\
261	0.930308862380573\\
262	0.931098416678937\\
263	0.931828180291126\\
264	0.932106042783327\\
265	0.932262982159061\\
266	0.933129666975752\\
267	0.933939232038208\\
268	0.934785766317716\\
269	0.935355734993683\\
270	0.93623379731911\\
271	0.936920450319322\\
272	0.937972481306673\\
273	0.939000382154116\\
274	0.939619061721818\\
275	0.940011829331826\\
276	0.940629133164645\\
277	0.940644424012706\\
278	0.941188249711769\\
279	0.941477140949455\\
280	0.94225561871219\\
281	0.942861262183984\\
282	0.94363470957952\\
283	0.944093331278181\\
284	0.944918764573924\\
285	0.945471204745499\\
286	0.946386807298974\\
287	0.946630430199497\\
288	0.947069474149935\\
289	0.947569678871502\\
290	0.948178404747182\\
291	0.948772464109509\\
292	0.949448233401305\\
293	0.949837427135796\\
294	0.950291428701118\\
295	0.950891564070309\\
296	0.951326769377\\
297	0.951710875735904\\
298	0.952191887979254\\
299	0.952779944594821\\
300	0.953353145553776\\
301	0.953654805257099\\
302	0.95401284534435\\
303	0.954409745964321\\
304	0.955078725730869\\
305	0.955560174285681\\
306	0.955852229094199\\
307	0.956537650788902\\
308	0.956676261238354\\
309	0.957289478210116\\
310	0.957708571570564\\
311	0.958230405081588\\
312	0.958792243265348\\
313	0.959439893185175\\
314	0.95979284204479\\
315	0.960471377791367\\
316	0.960953356242076\\
317	0.961570669381478\\
318	0.961951516174177\\
319	0.962128507239646\\
320	0.962667077119857\\
321	0.962992485252314\\
322	0.963194786552958\\
323	0.96329180771312\\
324	0.963796520031451\\
325	0.964315273861927\\
326	0.964667873796743\\
327	0.964970256541033\\
328	0.96527829184996\\
329	0.96560832490743\\
330	0.965870620430648\\
331	0.966185012331522\\
332	0.966332257297832\\
333	0.966846648673347\\
334	0.967180744556848\\
335	0.967527607333915\\
336	0.967857274023738\\
337	0.968035104812025\\
338	0.968318522252224\\
339	0.968731560135806\\
340	0.969171685731031\\
341	0.969361870761625\\
342	0.969711802827216\\
343	0.970003888557917\\
344	0.970259850119327\\
345	0.970580726007678\\
346	0.970757547594312\\
347	0.971096192881464\\
348	0.971375200835019\\
349	0.971710489778075\\
350	0.971945684284441\\
351	0.972263640557986\\
352	0.972421523423877\\
353	0.972630524830856\\
354	0.97298307684231\\
355	0.973366969591254\\
356	0.973542035890671\\
357	0.973862365492301\\
358	0.974052691542447\\
359	0.974301476188723\\
360	0.974685222312542\\
361	0.974844338131144\\
362	0.97506889151438\\
363	0.975322875238598\\
364	0.975629315038424\\
365	0.975905774900901\\
366	0.976182703318536\\
367	0.976382444686792\\
368	0.976685540361213\\
369	0.977044128344568\\
370	0.977306363179393\\
371	0.977454446156597\\
372	0.977545454361634\\
373	0.977769973894405\\
374	0.978062618736392\\
375	0.978323204974491\\
376	0.978560042796385\\
377	0.978810560415082\\
378	0.979030425750281\\
379	0.979178179095021\\
380	0.979333617111298\\
381	0.979447911326368\\
382	0.979688360893763\\
383	0.979804997913782\\
384	0.980071017704088\\
385	0.980235707679196\\
386	0.980541849385101\\
387	0.980791059385183\\
388	0.980979947196619\\
389	0.981217967035374\\
390	0.98147169648021\\
391	0.981581436112806\\
392	0.981748519756085\\
393	0.982005713926213\\
394	0.982177993712585\\
395	0.98239081231775\\
396	0.982621925601623\\
397	0.982816460085203\\
398	0.983031984764689\\
399	0.983246038185012\\
400	0.98339511509145\\
401	0.983540997621407\\
402	0.983686193086263\\
403	0.983888952466443\\
404	0.984115208271954\\
405	0.984304422866021\\
406	0.984497042984704\\
407	0.98468534181602\\
408	0.984856651389061\\
409	0.985026033857446\\
410	0.985128894263477\\
411	0.985296783519476\\
412	0.985467546410435\\
413	0.985630404640353\\
414	0.985741942895145\\
415	0.985837071674072\\
416	0.986007571474729\\
417	0.986190489248664\\
418	0.986312238041431\\
419	0.986456135474629\\
420	0.986609402767811\\
421	0.986761999480879\\
422	0.986983048888723\\
423	0.987096601865369\\
424	0.987191614538781\\
425	0.987302225565122\\
426	0.987432107638068\\
427	0.987555112429363\\
428	0.987637958949875\\
429	0.987832654063176\\
430	0.987977432556947\\
431	0.988043267765902\\
432	0.988175160798226\\
433	0.988290144866358\\
434	0.988446653011321\\
435	0.98861472496983\\
436	0.988642776407633\\
437	0.98870367656647\\
438	0.988838528320267\\
439	0.988945964166181\\
440	0.989054931779298\\
441	0.989170353136676\\
442	0.989280057622664\\
443	0.989393508142124\\
444	0.989478933562838\\
445	0.989556007352693\\
446	0.989688760047634\\
447	0.989796039554584\\
448	0.98992546286353\\
449	0.990021807740508\\
450	0.990141894667383\\
451	0.990284536366883\\
452	0.990397101530601\\
453	0.990465551500089\\
454	0.990549436196644\\
455	0.990612407064839\\
456	0.990744602764329\\
457	0.99085459223221\\
458	0.990922326373206\\
459	0.99103726601739\\
460	0.991161191952883\\
461	0.991246986633332\\
462	0.991308381196398\\
463	0.991401014669173\\
464	0.991503521115145\\
465	0.99158501872693\\
466	0.991695503591278\\
467	0.991781489545108\\
468	0.991865223214088\\
469	0.991936033288548\\
470	0.992033751855603\\
471	0.992099469831856\\
472	0.992141617716939\\
473	0.992243250354319\\
474	0.992298478652565\\
475	0.992347093488421\\
476	0.992372273119052\\
477	0.99245856084364\\
478	0.992545491783514\\
479	0.992575174570846\\
480	0.992618970661504\\
481	0.992694676224835\\
482	0.992795820705172\\
483	0.992877516566383\\
484	0.992976067910045\\
485	0.993114620617454\\
486	0.993200396356636\\
487	0.993242449620669\\
488	0.993317345861235\\
489	0.993395257144465\\
490	0.993470127129786\\
491	0.993555764750908\\
492	0.993638827150559\\
493	0.993723407511206\\
494	0.99376951841696\\
495	0.99382925178952\\
496	0.993863543184586\\
497	0.993921144532944\\
498	0.993949952249576\\
499	0.994015843037294\\
500	0.994073093880824\\
501	0.994126670005508\\
502	0.994190375333938\\
503	0.994285874973219\\
504	0.994336654224188\\
505	0.994406887694054\\
506	0.994480703246379\\
507	0.994544169759546\\
508	0.994596225079222\\
509	0.994620654192185\\
510	0.994676274676112\\
511	0.994763563560357\\
512	0.994836594399319\\
513	0.994875953723309\\
514	0.99495485450306\\
515	0.995018000509187\\
516	0.995084186570701\\
517	0.995130150660671\\
518	0.995213044336238\\
519	0.995256087622027\\
520	0.995300993771005\\
521	0.995340660406928\\
522	0.995380132001307\\
523	0.99542792440823\\
524	0.995454010418604\\
525	0.995528423109187\\
526	0.995571637865237\\
527	0.995567080721822\\
528	0.995634747753354\\
529	0.995687201771885\\
530	0.995719970726981\\
531	0.995773721519471\\
532	0.99580125830878\\
533	0.995858898385352\\
534	0.995902480765542\\
535	0.995915878208922\\
536	0.995951438911245\\
537	0.995999185129361\\
538	0.996042168960835\\
539	0.996070443719598\\
540	0.996111484478129\\
541	0.996148915443395\\
542	0.996175705984265\\
543	0.99622205429972\\
544	0.99628246128677\\
545	0.996329633195311\\
546	0.996352908475768\\
547	0.996378942949503\\
548	0.996417649507116\\
549	0.996464858277724\\
550	0.996504865524372\\
551	0.996546032591128\\
552	0.996578628004017\\
553	0.996622053662748\\
554	0.996655716970306\\
555	0.996670744925258\\
556	0.996686828396456\\
557	0.996708512825616\\
558	0.996739421465702\\
559	0.996768994628718\\
560	0.996803080059881\\
561	0.996840378542812\\
562	0.996882854721589\\
563	0.99690310699741\\
564	0.996932068129909\\
565	0.996959121539907\\
566	0.996996351334604\\
567	0.997020288850024\\
568	0.997050911864358\\
569	0.99709603184722\\
570	0.99714478600251\\
571	0.997166751801335\\
572	0.997193146669671\\
573	0.997226522496876\\
574	0.997260263509024\\
575	0.997285134127633\\
576	0.997312866435973\\
577	0.997340797901357\\
578	0.997369426675373\\
579	0.997396847847189\\
580	0.997414676068901\\
581	0.99744378572258\\
582	0.997451375032632\\
583	0.997485433998078\\
584	0.997496631056645\\
585	0.997515716088095\\
586	0.997546697104988\\
587	0.997596426558929\\
588	0.997617496562763\\
589	0.997638051415155\\
590	0.997652628989547\\
591	0.997667488500082\\
592	0.997693892591086\\
593	0.997715025718018\\
594	0.997728659231906\\
595	0.99774203632856\\
596	0.997762036617844\\
597	0.997784959945542\\
598	0.997802032275459\\
599	0.997836322503355\\
600	0.997858713314354\\
601	0.997886170299287\\
602	0.997906260677302\\
603	0.997914652407226\\
604	0.997936246934533\\
605	0.99797510393985\\
606	0.997998093537415\\
607	0.998022791773794\\
608	0.998052137676488\\
609	0.998069042445917\\
610	0.998090119476014\\
611	0.998108141643293\\
612	0.998128102657284\\
613	0.998145299474889\\
614	0.998180827245493\\
615	0.998210859726558\\
616	0.998231229337222\\
617	0.998252187482206\\
618	0.998270040532996\\
619	0.998299347649784\\
620	0.998310010150898\\
621	0.998325327114799\\
622	0.998341050084194\\
623	0.998355409962539\\
624	0.99837669573088\\
625	0.99839674524753\\
626	0.998410666804298\\
627	0.998421334291449\\
628	0.998429996354744\\
629	0.998443343467663\\
630	0.998454104939861\\
631	0.998467317806444\\
632	0.998485795967003\\
633	0.998501078812253\\
634	0.998514642293116\\
635	0.998527654500548\\
636	0.99853610681978\\
637	0.998547806386621\\
638	0.998552924640921\\
639	0.998567241987323\\
640	0.998574018034229\\
641	0.998579347975293\\
642	0.998603562578546\\
643	0.998615914912864\\
644	0.998625123453411\\
645	0.998639404706146\\
646	0.998665333469313\\
647	0.998675151013315\\
648	0.998688069331048\\
649	0.99870588745986\\
650	0.99872240751363\\
651	0.998735308396598\\
652	0.998751937733608\\
653	0.998765660353704\\
654	0.99877979552594\\
655	0.998785161761351\\
656	0.998799741031613\\
657	0.998809680319542\\
658	0.998818487079382\\
659	0.998829671057405\\
660	0.998845759716983\\
661	0.998862298712784\\
662	0.998871234220623\\
663	0.998886734921991\\
664	0.998896279245623\\
665	0.998902474105836\\
666	0.998906458605132\\
667	0.998919298982786\\
668	0.998930682845323\\
669	0.998939543284952\\
670	0.998957606866905\\
671	0.998968448172636\\
672	0.998978358780235\\
673	0.998994211193906\\
674	0.999001455397395\\
675	0.999015547555167\\
676	0.999024015851932\\
677	0.999029311168079\\
678	0.999042452251618\\
679	0.999052480823713\\
680	0.999056230166518\\
681	0.999062103866142\\
682	0.999073833305541\\
683	0.999081258297163\\
684	0.999084379686099\\
685	0.999091982020022\\
686	0.99910570725487\\
687	0.999108615001641\\
688	0.999118799366124\\
689	0.999129427082872\\
690	0.999136564308956\\
691	0.999149893681084\\
692	0.999157422714396\\
693	0.999166261924314\\
694	0.999173305240991\\
695	0.999177144342136\\
696	0.999193104915205\\
697	0.999202831657865\\
698	0.999203017683627\\
699	0.999212580647341\\
700	0.999221807100191\\
701	0.999228498548745\\
702	0.99923620396633\\
703	0.999243975042954\\
704	0.999254524026977\\
705	0.99926229625547\\
706	0.99926948580241\\
707	0.999279166692118\\
708	0.999289022279106\\
709	0.999299770152946\\
710	0.999304225140829\\
711	0.999308366419928\\
712	0.999314765197844\\
713	0.999324864035692\\
714	0.999331320585835\\
715	0.999332408245918\\
716	0.999338292005863\\
717	0.999346962493596\\
718	0.999352013513174\\
719	0.999357677130709\\
720	0.999358615882397\\
721	0.999364610192622\\
722	0.999370201265568\\
723	0.999374700191181\\
724	0.999381407758934\\
725	0.999387230791347\\
726	0.999394897129027\\
727	0.999400936120659\\
728	0.999402777580682\\
729	0.99940934498375\\
730	0.999418644730721\\
731	0.9994252748795\\
732	0.999430290368546\\
733	0.999436581335157\\
734	0.999442398376245\\
735	0.999445259695286\\
736	0.999449462071286\\
737	0.999457495692602\\
738	0.99946385638116\\
739	0.999468238790242\\
740	0.999473210707471\\
741	0.999475344764927\\
742	0.999480700733788\\
743	0.999485534145846\\
744	0.999490249967771\\
745	0.999495891583578\\
746	0.999500377393163\\
747	0.999504857333923\\
748	0.999508551618313\\
749	0.999514928446731\\
750	0.999519889205749\\
751	0.99952486410551\\
752	0.999531559412594\\
753	0.999535228431796\\
754	0.999541689338632\\
755	0.999548062857116\\
756	0.99955444447602\\
757	0.999558244226567\\
758	0.999562956420369\\
759	0.999566435522153\\
760	0.999568442870098\\
761	0.999569812087619\\
762	0.999576913511405\\
763	0.999582987708946\\
764	0.999588935433778\\
765	0.999593489647411\\
766	0.999596603343728\\
767	0.99960027839817\\
768	0.999604071285035\\
769	0.999609630979079\\
770	0.999612462025461\\
771	0.999617957828394\\
772	0.999621136857222\\
773	0.999624657502678\\
774	0.999628810582194\\
775	0.999632251905427\\
776	0.999639269216633\\
777	0.999643243655019\\
778	0.999647097265516\\
779	0.999649941759643\\
780	0.999653538016067\\
781	0.99965804759384\\
782	0.999662266003554\\
783	0.999665614600039\\
784	0.999668548496727\\
785	0.999671025079153\\
786	0.999673540165875\\
787	0.999675794175623\\
788	0.999678372882157\\
789	0.999681971879098\\
790	0.999683064539959\\
791	0.999684904536965\\
792	0.999687034505206\\
793	0.99969106157562\\
794	0.99969595153086\\
795	0.999700102094594\\
796	0.99970218839396\\
797	0.99970446791551\\
798	0.999705935954748\\
799	0.999708297573178\\
800	0.999710573093907\\
801	0.99971364410896\\
802	0.99971856434032\\
803	0.999721568488368\\
804	0.999723548309892\\
805	0.999728013194129\\
806	0.99973109073235\\
807	0.999734609970832\\
808	0.999737378795878\\
809	0.999740866429529\\
810	0.999743480468551\\
811	0.999745669876075\\
812	0.999746737155949\\
813	0.999750362140716\\
814	0.999752289501467\\
815	0.999753551718669\\
816	0.999756626904976\\
817	0.999759216083827\\
818	0.999761743296988\\
819	0.999762770127599\\
820	0.999766527134837\\
821	0.999769169154684\\
822	0.999770659492571\\
823	0.999773019291253\\
824	0.999776435549696\\
825	0.999778273751316\\
826	0.999781209962804\\
827	0.999782882799906\\
828	0.999785462278055\\
829	0.999788578353516\\
830	0.999791748572492\\
831	0.99979474560902\\
832	0.999796114843071\\
833	0.999797170313565\\
834	0.999799136645434\\
835	0.999801484699441\\
836	0.999803255934724\\
837	0.999806692712092\\
838	0.999808545245853\\
839	0.999811616036385\\
840	0.999813069359032\\
841	0.99981498525082\\
842	0.999817150481023\\
843	0.999818789380312\\
844	0.999820344409932\\
845	0.999823247197642\\
846	0.999824840268249\\
847	0.999825912583536\\
848	0.999827207336286\\
849	0.9998289853254\\
850	0.999830906657258\\
851	0.99983218197952\\
852	0.999835033679302\\
853	0.999836663090802\\
854	0.99983812576143\\
855	0.999839354575928\\
856	0.999841022405965\\
857	0.999842606055017\\
858	0.999844563238101\\
859	0.999846889296729\\
860	0.999849129507488\\
861	0.999850812617206\\
862	0.999851130610364\\
863	0.999852066790613\\
864	0.999853941750868\\
865	0.99985568323851\\
866	0.999856554761095\\
867	0.999857828546491\\
868	0.999858617508142\\
869	0.999860051545491\\
870	0.999862747303571\\
871	0.999863977463223\\
872	0.999865039025728\\
873	0.999866149612904\\
874	0.999867708220513\\
875	0.999869431589504\\
876	0.999870431888196\\
877	0.999871627740171\\
878	0.999873359367986\\
879	0.999874295048285\\
880	0.999875514373909\\
881	0.999876462016679\\
882	0.999877923074611\\
883	0.999878714147075\\
884	0.999880358113782\\
885	0.999881549257583\\
886	0.99988291084742\\
887	0.999883648522265\\
888	0.999885465106139\\
889	0.99988706593669\\
890	0.999887817306225\\
891	0.99988849839999\\
892	0.999889731362466\\
893	0.99989060045741\\
894	0.999890951756935\\
895	0.999892549897875\\
896	0.999894351274376\\
897	0.999894927851911\\
898	0.999896674563239\\
899	0.99989792610182\\
900	0.999899262881863\\
901	0.999899743248168\\
902	0.999901020788767\\
903	0.99990218615817\\
904	0.999903357552441\\
905	0.999904571851065\\
906	0.999905103760758\\
907	0.999906355079078\\
908	0.999907362043412\\
909	0.999908416357538\\
910	0.999909155812931\\
911	0.999910786644933\\
912	0.999911614534708\\
913	0.999912485580399\\
914	0.999912847407935\\
915	0.999913651289751\\
916	0.999914656752884\\
917	0.999915376967959\\
918	0.999915907834654\\
919	0.99991647399529\\
920	0.999917394798994\\
921	0.999918972897949\\
922	0.999919679224538\\
923	0.999920717283854\\
924	0.999921550045222\\
925	0.999922849218219\\
926	0.999923469413046\\
927	0.999923865099171\\
928	0.9999245879299\\
929	0.999925262387312\\
930	0.999925962653343\\
931	0.999926983728509\\
932	0.999927712050589\\
933	0.999928234670515\\
934	0.999929177404646\\
935	0.999930240535565\\
936	0.999931221138775\\
937	0.9999318621341\\
938	0.999932190182017\\
939	0.999932907700777\\
940	0.99993334005097\\
941	0.999933587310468\\
942	0.999934428446127\\
943	0.999935053188533\\
944	0.999935673770491\\
945	0.999936088345363\\
946	0.999936771824451\\
947	0.999937442498133\\
948	0.999938284071868\\
949	0.999938571251974\\
950	0.999939173897652\\
951	0.999939707524985\\
952	0.999940396271508\\
953	0.999940755577129\\
954	0.999941336596267\\
955	0.999941958723973\\
956	0.99994248121783\\
957	0.999942807837983\\
958	0.999943456588094\\
959	0.999943830132517\\
960	0.999944112699616\\
961	0.999945239789943\\
962	0.999945674053144\\
963	0.999946442947055\\
964	0.999946875092772\\
965	0.999947317558451\\
966	0.999948060734878\\
967	0.999948635819054\\
968	0.999949094799626\\
969	0.999949771597513\\
970	0.999950167527015\\
971	0.999950740360602\\
972	0.999951461745077\\
973	0.999952100643986\\
974	0.999952687222455\\
975	0.999953218094599\\
976	0.999953848661171\\
977	0.999954781645317\\
978	0.999955098480398\\
979	0.999956069803035\\
980	0.999956419415531\\
981	0.999957475821081\\
982	0.99995778375074\\
983	0.999958107859343\\
984	0.999958350306012\\
985	0.999958668544677\\
986	0.999958999175192\\
987	0.999959538658975\\
988	0.99995994068825\\
989	0.999960171682823\\
990	0.99996051700824\\
991	0.999960651343036\\
992	0.999961209465533\\
993	0.999961569841287\\
994	0.999961924597815\\
995	0.999962312600442\\
996	0.999962740547053\\
997	0.999962903668681\\
998	0.999963362489418\\
999	0.99996389882152\\
1000	0.999964249398846\\
};
\addlegendentry{$\Im \{ \mathbf{h}_2^H \}$};

\addplot [color=mycolor7,solid]
  table[row sep=crcr]{%
1	0\\
2	0\\
3	0\\
4	0.0296518946255946\\
5	0.0520392978397778\\
6	0.0799881740827882\\
7	0.10142684692335\\
8	0.124292614794878\\
9	0.145073733564458\\
10	0.164735091172416\\
11	0.18701880370652\\
12	0.209677295531519\\
13	0.233955733213185\\
14	0.258241026040121\\
15	0.27984842645767\\
16	0.304968727204608\\
17	0.328620378553452\\
18	0.347535445558854\\
19	0.369204119471535\\
20	0.392606303748482\\
21	0.409504615429821\\
22	0.430509351923911\\
23	0.452081263812957\\
24	0.47109419596679\\
25	0.492340487180995\\
26	0.511052802394362\\
27	0.529157763334699\\
28	0.550119896640927\\
29	0.573133027418634\\
30	0.59237304301401\\
31	0.61006470444925\\
32	0.627043115618748\\
33	0.644076510334591\\
34	0.664251066978181\\
35	0.684350657134\\
36	0.701319289840682\\
37	0.719257127993235\\
38	0.738835043161436\\
39	0.759778227701751\\
40	0.776904295412096\\
41	0.792704571019034\\
42	0.80764947796855\\
43	0.823038524511818\\
44	0.839351805786401\\
45	0.857602354577744\\
46	0.878349778681726\\
47	0.893128730844188\\
48	0.905930636336591\\
49	0.920973774842446\\
50	0.936231229159495\\
51	0.950959540930647\\
52	0.963546703737758\\
53	0.977838828488109\\
54	0.991779717329408\\
55	1.00578568339911\\
56	1.02361542310998\\
57	1.04170820679701\\
58	1.053863238326\\
59	1.06695844287008\\
60	1.07960940523891\\
61	1.0970646293169\\
62	1.11128518944752\\
63	1.1230451515249\\
64	1.13983958080619\\
65	1.154261270919\\
66	1.16832133172755\\
67	1.18371669320173\\
68	1.197460465591\\
69	1.21120873245108\\
70	1.22562077409732\\
71	1.23782381173601\\
72	1.25017939024091\\
73	1.26326360695913\\
74	1.27598724837967\\
75	1.28736282533393\\
76	1.29743431226883\\
77	1.30808691226586\\
78	1.32121493814029\\
79	1.33535306177288\\
80	1.34508406430987\\
81	1.35980830785705\\
82	1.37160420410147\\
83	1.38399353371291\\
84	1.39357906762438\\
85	1.40366888299443\\
86	1.41402962276546\\
87	1.42588216547881\\
88	1.43809183657813\\
89	1.44831239844316\\
90	1.45787334410131\\
91	1.46832255267198\\
92	1.47942868851948\\
93	1.48977954676158\\
94	1.49977787793083\\
95	1.50789646879227\\
96	1.5183716404173\\
97	1.52796367627231\\
98	1.53695014496768\\
99	1.54965733700311\\
100	1.55798631515008\\
101	1.56855120361711\\
102	1.57727769944159\\
103	1.58609452721765\\
104	1.59581269434267\\
105	1.60340850711551\\
106	1.61146085052471\\
107	1.62208725045831\\
108	1.63194498377608\\
109	1.64207381276116\\
110	1.65149322461884\\
111	1.66060579991809\\
112	1.668187490773\\
113	1.67767668824805\\
114	1.68486781147606\\
115	1.69487383372791\\
116	1.70280681368007\\
117	1.71066722809839\\
118	1.71735534576215\\
119	1.72570490153115\\
120	1.73478609686396\\
121	1.74189341692453\\
122	1.74884180413683\\
123	1.7556024876444\\
124	1.76231624688875\\
125	1.76918513311572\\
126	1.77643236412656\\
127	1.78307190836094\\
128	1.79032641383924\\
129	1.79762497228644\\
130	1.80546383212501\\
131	1.81188520770147\\
132	1.82108482137071\\
133	1.82837084962379\\
134	1.83505438327873\\
135	1.84080412188258\\
136	1.84907451688743\\
137	1.85556230481343\\
138	1.86237092707409\\
139	1.8693340750588\\
140	1.87663393886202\\
141	1.88280827236832\\
142	1.88891477694581\\
143	1.89521333151275\\
144	1.90184511898824\\
145	1.90792268432632\\
146	1.9145626808647\\
147	1.91970242042007\\
148	1.92466600163608\\
149	1.93066265543179\\
150	1.93668958999083\\
151	1.94240824773384\\
152	1.94770717195445\\
153	1.95341199558035\\
154	1.95879293050683\\
155	1.96422158576745\\
156	1.9698945523954\\
157	1.97564146111833\\
158	1.98288896246958\\
159	1.98763809928726\\
160	1.99151400894522\\
161	1.99765401808458\\
162	2.00230134337885\\
163	2.00754062918083\\
164	2.01322226693787\\
165	2.01792010906592\\
166	2.023026664079\\
167	2.02683033304648\\
168	2.03071683829389\\
169	2.03487792023317\\
170	2.03973153222738\\
171	2.04312129311163\\
172	2.04796207409285\\
173	2.05205952741785\\
174	2.05642015041082\\
175	2.0610429861325\\
176	2.06559415295189\\
177	2.07036953037435\\
178	2.07374392123661\\
179	2.07848554870618\\
180	2.08234619445933\\
181	2.08629757281278\\
182	2.09030126276358\\
183	2.09432009753047\\
184	2.09856903821435\\
185	2.1026172017927\\
186	2.10606392470079\\
187	2.11049942944538\\
188	2.11480124061558\\
189	2.11871256975134\\
190	2.12166094387671\\
191	2.12584122813503\\
192	2.12917560849126\\
193	2.13230927248997\\
194	2.1364523220083\\
195	2.14003062534201\\
196	2.14360829772399\\
197	2.14639496611107\\
198	2.14954336524279\\
199	2.15270826198306\\
200	2.15634342853983\\
201	2.1597445492163\\
202	2.16333226274153\\
203	2.16728746245458\\
204	2.17047019146923\\
205	2.17410942901782\\
206	2.17751167462971\\
207	2.18075621314065\\
208	2.18428695747747\\
209	2.18702679545799\\
210	2.18955944085759\\
211	2.19182847274713\\
212	2.19537918887444\\
213	2.19848780125633\\
214	2.20191679274219\\
215	2.2051524442674\\
216	2.20897951970317\\
217	2.21178100688137\\
218	2.21456628282922\\
219	2.2177019915546\\
220	2.22068906906406\\
221	2.22400352015028\\
222	2.22690626032721\\
223	2.22981751675517\\
224	2.23243069240771\\
225	2.23523856455785\\
226	2.237938703629\\
227	2.24079364939761\\
228	2.24363732509709\\
229	2.24550275367225\\
230	2.24825375084338\\
231	2.25038383150041\\
232	2.25254584569915\\
233	2.25509464077241\\
234	2.25745613095491\\
235	2.2595515911795\\
236	2.26157360571687\\
237	2.2636895850577\\
238	2.26666015649818\\
239	2.26937986733161\\
240	2.27145631300173\\
241	2.27439445005467\\
242	2.27658234986703\\
243	2.27889679201497\\
244	2.28078883179899\\
245	2.28335759598946\\
246	2.28579267833846\\
247	2.28803469611675\\
248	2.29101122126384\\
249	2.2931395702979\\
250	2.29563180221081\\
251	2.29729460357803\\
252	2.29975523736479\\
253	2.30181281676136\\
254	2.30371503770686\\
255	2.30569585688998\\
256	2.30747885743881\\
257	2.30963154314168\\
258	2.31164505664387\\
259	2.31369358525551\\
260	2.31577982008654\\
261	2.31761124871223\\
262	2.31948083257654\\
263	2.32129302759179\\
264	2.32263229859359\\
265	2.32458498470199\\
266	2.32662560592117\\
267	2.328510552365\\
268	2.33026420223084\\
269	2.33175430082971\\
270	2.33371578466813\\
271	2.33511110912382\\
272	2.33709576115982\\
273	2.33877924272399\\
274	2.34025386462874\\
275	2.34200224872714\\
276	2.34359721150302\\
277	2.34487374462598\\
278	2.34619879645997\\
279	2.34748295183316\\
280	2.34924887926701\\
281	2.35062802865576\\
282	2.3521694042666\\
283	2.35320572389525\\
284	2.3545823550595\\
285	2.35598177419747\\
286	2.35786457878127\\
287	2.35931558373084\\
288	2.36063206972049\\
289	2.36220840437039\\
290	2.36384569483107\\
291	2.36523101711912\\
292	2.36674947857826\\
293	2.36797834711429\\
294	2.36931906197832\\
295	2.37055396238084\\
296	2.37201292758167\\
297	2.37329843585654\\
298	2.37458439813481\\
299	2.37581801480164\\
300	2.37711553789209\\
301	2.37799083700733\\
302	2.37916442368618\\
303	2.38058651686241\\
304	2.38172024916607\\
305	2.3829743540219\\
306	2.38414093288828\\
307	2.38545147926686\\
308	2.38667324384143\\
309	2.38807846405482\\
310	2.38898608418837\\
311	2.39025032541788\\
312	2.3915018624151\\
313	2.3928377983409\\
314	2.39387805315927\\
315	2.39491650812521\\
316	2.39595048388418\\
317	2.39712742221653\\
318	2.39829451310272\\
319	2.39920090321321\\
320	2.40036525807667\\
321	2.4013779455272\\
322	2.40212779708809\\
323	2.40310484657404\\
324	2.4042275749797\\
325	2.40525246291761\\
326	2.40613107905839\\
327	2.40707448526464\\
328	2.40797334006249\\
329	2.40886540595754\\
330	2.40963094859974\\
331	2.4102131263015\\
332	2.4108349740155\\
333	2.4119470708152\\
334	2.41278268856176\\
335	2.41358797059558\\
336	2.41444193102913\\
337	2.41534773084135\\
338	2.4161521186429\\
339	2.41710650834424\\
340	2.41796637033928\\
341	2.41873445577853\\
342	2.41971219184516\\
343	2.42041440660771\\
344	2.42106551800626\\
345	2.42181443705382\\
346	2.42263520640309\\
347	2.42343267572289\\
348	2.42418587876381\\
349	2.42481797910229\\
350	2.42545172874835\\
351	2.42625206256659\\
352	2.42682733907828\\
353	2.4273981916708\\
354	2.42825098948644\\
355	2.42915081862568\\
356	2.43000850838541\\
357	2.43061487103741\\
358	2.43141012227104\\
359	2.43211204526234\\
360	2.43288687820233\\
361	2.43361494302148\\
362	2.4342614855663\\
363	2.43495256465329\\
364	2.43571844783233\\
365	2.43636121694411\\
366	2.43699556022552\\
367	2.4375948733685\\
368	2.43827566916203\\
369	2.43894634870542\\
370	2.43954198461845\\
371	2.44009118565102\\
372	2.44061110526121\\
373	2.4411536965559\\
374	2.44175781747987\\
375	2.44230013582706\\
376	2.44290546636222\\
377	2.44346247609188\\
378	2.44408826314605\\
379	2.44451277836996\\
380	2.44507750040679\\
381	2.44572580916301\\
382	2.44621584897218\\
383	2.44681417319917\\
384	2.44739203302972\\
385	2.4479141768036\\
386	2.44856688448367\\
387	2.4490714107585\\
388	2.44960075737838\\
389	2.45024019308109\\
390	2.45071058417575\\
391	2.45116599601428\\
392	2.45159659287321\\
393	2.45201888028433\\
394	2.45256004244813\\
395	2.45311162510305\\
396	2.45357954336541\\
397	2.45397860677259\\
398	2.45453078662111\\
399	2.45498007775492\\
400	2.45532285279165\\
401	2.45578780318927\\
402	2.45623944847727\\
403	2.45673595649628\\
404	2.45720129668316\\
405	2.45769294060169\\
406	2.4581525880159\\
407	2.45856175906857\\
408	2.45897021756783\\
409	2.45931980208218\\
410	2.45970335278357\\
411	2.46014911135472\\
412	2.46052522700089\\
413	2.46092945158009\\
414	2.46126388559509\\
415	2.46162154379911\\
416	2.46193850630161\\
417	2.4622803197575\\
418	2.46257219725257\\
419	2.46294398857993\\
420	2.4632830606626\\
421	2.46368899517554\\
422	2.46403264862525\\
423	2.46442865716368\\
424	2.46484635750315\\
425	2.465224756295\\
426	2.46553486373412\\
427	2.46590479410188\\
428	2.46620753398875\\
429	2.46651343645959\\
430	2.46686163980918\\
431	2.46719581993398\\
432	2.46754783027533\\
433	2.46781242182189\\
434	2.4681802864034\\
435	2.4685093872775\\
436	2.4688183186199\\
437	2.46908685388427\\
438	2.46945566913893\\
439	2.46972668652095\\
440	2.47005047698468\\
441	2.47040849707274\\
442	2.47068321568649\\
443	2.47098821162604\\
444	2.47123778204724\\
445	2.47150867401444\\
446	2.47175409524691\\
447	2.47201794558454\\
448	2.47232369731528\\
449	2.472526087466\\
450	2.47281635824582\\
451	2.47313023697549\\
452	2.47336781302125\\
453	2.47358858110762\\
454	2.47382177970617\\
455	2.47406965397662\\
456	2.47437826342058\\
457	2.47458032532557\\
458	2.47480062898348\\
459	2.47500856097339\\
460	2.47529753637305\\
461	2.4755710731633\\
462	2.47583345890842\\
463	2.47608829426637\\
464	2.47627889312968\\
465	2.47649830504753\\
466	2.47673564546159\\
467	2.47703297753149\\
468	2.47728065696055\\
469	2.47749973006238\\
470	2.47772245632144\\
471	2.47792415127483\\
472	2.47812312855295\\
473	2.47835823246042\\
474	2.47853378179595\\
475	2.47871504758463\\
476	2.47890140857363\\
477	2.47910120780065\\
478	2.47932726212198\\
479	2.47949397003145\\
480	2.47966517576277\\
481	2.47991494335304\\
482	2.48016242253703\\
483	2.48033798264812\\
484	2.48052976914354\\
485	2.48082377256408\\
486	2.48102762565054\\
487	2.48116804294377\\
488	2.48137403756207\\
489	2.48154983162541\\
490	2.48171588294911\\
491	2.48188853813756\\
492	2.48205933108099\\
493	2.48227510383574\\
494	2.48244340543593\\
495	2.48263019943333\\
496	2.48277769411587\\
497	2.48293161390163\\
498	2.48307485488688\\
499	2.48325637869397\\
500	2.48340711930923\\
501	2.48356548468289\\
502	2.4837070354864\\
503	2.48387812197702\\
504	2.4840267754686\\
505	2.48416496323671\\
506	2.484307494822\\
507	2.48444361411625\\
508	2.48458064786079\\
509	2.48472114069203\\
510	2.48488527715197\\
511	2.48505892323916\\
512	2.48521512621707\\
513	2.48534931211955\\
514	2.48549708463822\\
515	2.48566452373609\\
516	2.48582276364449\\
517	2.48595343620252\\
518	2.48614087725975\\
519	2.48627102478371\\
520	2.48640331629042\\
521	2.48655963075506\\
522	2.48667022119687\\
523	2.48680603649909\\
524	2.48690454038429\\
525	2.48703704202747\\
526	2.48716197122061\\
527	2.48732058865631\\
528	2.48743499610891\\
529	2.48757213489603\\
530	2.48768545795217\\
531	2.48783526916026\\
532	2.48792466975525\\
533	2.48805269359606\\
534	2.48815504606571\\
535	2.48826455443423\\
536	2.48838032208337\\
537	2.48851494647165\\
538	2.48862860833561\\
539	2.48876481851663\\
540	2.48887998128499\\
541	2.48899938205408\\
542	2.48910141229618\\
543	2.48917935660319\\
544	2.48930680295167\\
545	2.48941388961423\\
546	2.48951713985479\\
547	2.48962271799778\\
548	2.48971085414129\\
549	2.48982479636892\\
550	2.48992773845562\\
551	2.49005120333288\\
552	2.49015899455461\\
553	2.49025644973002\\
554	2.49034179607038\\
555	2.49041822102847\\
556	2.49049842029008\\
557	2.49057996924159\\
558	2.49068090004934\\
559	2.49077107119003\\
560	2.49086378531219\\
561	2.49096650814789\\
562	2.49107649671868\\
563	2.49116068522726\\
564	2.49124460681292\\
565	2.49133763899699\\
566	2.49142403944948\\
567	2.49151434748452\\
568	2.49161016617592\\
569	2.49170478923893\\
570	2.49177744458037\\
571	2.49186225273786\\
572	2.49193459205004\\
573	2.49201762035588\\
574	2.49210108389\\
575	2.49217269255679\\
576	2.49225696132736\\
577	2.49234579284572\\
578	2.49240992267347\\
579	2.49249999350444\\
580	2.49256649446626\\
581	2.49265294073695\\
582	2.49272530586104\\
583	2.49280003858782\\
584	2.49285826546799\\
585	2.49293106480716\\
586	2.49300078183674\\
587	2.49307267062692\\
588	2.49313450438737\\
589	2.49319245318888\\
590	2.49325847764211\\
591	2.49333021944629\\
592	2.49338561855956\\
593	2.49345883035825\\
594	2.49352608536356\\
595	2.49359463440657\\
596	2.49364193992491\\
597	2.49371521414203\\
598	2.49376931061363\\
599	2.49382744883539\\
600	2.49390267506942\\
601	2.49396713967012\\
602	2.49401185313224\\
603	2.49405456174813\\
604	2.49411148094764\\
605	2.49419456378772\\
606	2.49424759068665\\
607	2.49429925480096\\
608	2.4943714031321\\
609	2.49442710251442\\
610	2.49447389715285\\
611	2.49453478341989\\
612	2.49459270726286\\
613	2.49463867195029\\
614	2.49471028726662\\
615	2.49476531942433\\
616	2.49481319732086\\
617	2.49487374940573\\
618	2.49492989200202\\
619	2.4949811113167\\
620	2.49503757984658\\
621	2.49509191626809\\
622	2.49513200846107\\
623	2.49518622318123\\
624	2.49523593038187\\
625	2.49528647888837\\
626	2.49533157625295\\
627	2.49537474919454\\
628	2.49542427770865\\
629	2.49547694447797\\
630	2.4955144144996\\
631	2.4955595530205\\
632	2.49561986842735\\
633	2.49567043057442\\
634	2.49571853875755\\
635	2.49575958804263\\
636	2.49579748758482\\
637	2.49584692571102\\
638	2.4958812658396\\
639	2.49592203389701\\
640	2.49595376726015\\
641	2.49598744208381\\
642	2.49604245655602\\
643	2.49608372193318\\
644	2.49612837679775\\
645	2.49616706582748\\
646	2.49621531620328\\
647	2.49625623232048\\
648	2.49629339902478\\
649	2.49632713868763\\
650	2.49636063096328\\
651	2.4963891924202\\
652	2.4964226827549\\
653	2.49645767935455\\
654	2.49649662500777\\
655	2.49652612517738\\
656	2.49656738100697\\
657	2.49659892949669\\
658	2.49663236600916\\
659	2.49665925106587\\
660	2.49669084839886\\
661	2.49672783928166\\
662	2.49675461303735\\
663	2.49678622310651\\
664	2.49681149025666\\
665	2.4968401912559\\
666	2.49686480918628\\
667	2.49689780417272\\
668	2.49692885100151\\
669	2.49695604227098\\
670	2.49699556063905\\
671	2.49702507259708\\
672	2.49705587509248\\
673	2.49708625779451\\
674	2.49711882649326\\
675	2.49714832943235\\
676	2.49717728187022\\
677	2.49720658199018\\
678	2.49723570998897\\
679	2.49725958868856\\
680	2.49728585471949\\
681	2.49731149647553\\
682	2.49733691940627\\
683	2.4973649395478\\
684	2.49738801022788\\
685	2.49741196298844\\
686	2.49743997945535\\
687	2.49746741840743\\
688	2.49748987157377\\
689	2.49751504029843\\
690	2.49754273081232\\
691	2.497581164809\\
692	2.4976048513497\\
693	2.49762584757194\\
694	2.49765182352792\\
695	2.49766905974737\\
696	2.49769517658756\\
697	2.49772135123681\\
698	2.49774126007595\\
699	2.49776219298469\\
700	2.49778449992921\\
701	2.49780444620407\\
702	2.4978268565627\\
703	2.49785062637382\\
704	2.49787602940267\\
705	2.49790046212064\\
706	2.49792176906592\\
707	2.49794275612522\\
708	2.49796315150256\\
709	2.49798538939863\\
710	2.49800367881114\\
711	2.49802632612145\\
712	2.49804583248284\\
713	2.4980698519573\\
714	2.49808587985587\\
715	2.49810317733804\\
716	2.49812395455806\\
717	2.49814777286763\\
718	2.49816483962637\\
719	2.49818327027029\\
720	2.49819851703055\\
721	2.49821559338182\\
722	2.49823543344504\\
723	2.49825211600596\\
724	2.49827012995594\\
725	2.49828782574482\\
726	2.49830929049176\\
727	2.49832783806212\\
728	2.49834022727986\\
729	2.4983563770562\\
730	2.49837232144518\\
731	2.49839299976653\\
732	2.49841266426327\\
733	2.49843191572278\\
734	2.49844386903893\\
735	2.49845584191675\\
736	2.49846816990712\\
737	2.49848349058829\\
738	2.49849824401496\\
739	2.49851152697795\\
740	2.49852395253928\\
741	2.49853739505707\\
742	2.49855063305865\\
743	2.49856541643722\\
744	2.4985816185768\\
745	2.49859338704977\\
746	2.49860884606028\\
747	2.49862538900206\\
748	2.49863964507808\\
749	2.49865202413462\\
750	2.49866682554093\\
751	2.49867863296947\\
752	2.49869272449907\\
753	2.49870568702727\\
754	2.49871742784723\\
755	2.498730988205\\
756	2.49874668961206\\
757	2.49875961390938\\
758	2.49877296125958\\
759	2.49878528775896\\
760	2.49879707549345\\
761	2.49881055075358\\
762	2.498823594201\\
763	2.49883709876901\\
764	2.49884795900357\\
765	2.49885911846678\\
766	2.49886979016773\\
767	2.49888286980364\\
768	2.49889357657506\\
769	2.49890576250404\\
770	2.49891699671116\\
771	2.49892851932923\\
772	2.49893949881804\\
773	2.49895045923289\\
774	2.49896006941141\\
775	2.49897030964344\\
776	2.49898139174224\\
777	2.49899289262933\\
778	2.4990025917401\\
779	2.49901142727453\\
780	2.49902164166266\\
781	2.49903102694928\\
782	2.49903893115128\\
783	2.49904959887606\\
784	2.49905761804405\\
785	2.49906460728416\\
786	2.49907274545237\\
787	2.49907996363854\\
788	2.49908924354339\\
789	2.4990979492279\\
790	2.49910644024012\\
791	2.49911452677286\\
792	2.49912481908764\\
793	2.49913539190165\\
794	2.49914472126816\\
795	2.49915397187406\\
796	2.49916088262577\\
797	2.49916944092034\\
798	2.49917761882482\\
799	2.49918607158013\\
800	2.49919490636822\\
801	2.49920199242462\\
802	2.49921038568741\\
803	2.49921732575309\\
804	2.49922416208713\\
805	2.49923308492076\\
806	2.49924111495599\\
807	2.49924953118069\\
808	2.49925590741819\\
809	2.49926373895633\\
810	2.49927275722014\\
811	2.49927973093198\\
812	2.49928831143406\\
813	2.49929575740238\\
814	2.49930179013706\\
815	2.49931027658626\\
816	2.49931728332465\\
817	2.49932278515374\\
818	2.49932995724485\\
819	2.49933662223863\\
820	2.49934410242982\\
821	2.49935034570219\\
822	2.49935666538181\\
823	2.49936291200019\\
824	2.49937016864109\\
825	2.49937587571946\\
826	2.49938219430738\\
827	2.49938809788527\\
828	2.49939336915967\\
829	2.49939804414934\\
830	2.49940398206738\\
831	2.49941125980796\\
832	2.49941558217606\\
833	2.49942162482495\\
834	2.49942748026803\\
835	2.49943469816842\\
836	2.49944155744625\\
837	2.49944733381004\\
838	2.49945261581094\\
839	2.49945732956287\\
840	2.4994614375583\\
841	2.49946623643547\\
842	2.49947264943184\\
843	2.4994779794107\\
844	2.49948382922534\\
845	2.49948902296022\\
846	2.49949294728327\\
847	2.49949726213712\\
848	2.49950187996532\\
849	2.49950568964765\\
850	2.49951038797345\\
851	2.4995147472505\\
852	2.4995204377114\\
853	2.49952437280334\\
854	2.49952871277805\\
855	2.49953238402045\\
856	2.49953611252803\\
857	2.49953968683419\\
858	2.49954385367049\\
859	2.49954972204937\\
860	2.49955587658854\\
861	2.49956088356601\\
862	2.4995646394435\\
863	2.49956937597362\\
864	2.4995730986397\\
865	2.49957703205717\\
866	2.49958169492535\\
867	2.49958502156949\\
868	2.49958830815551\\
869	2.49959173566769\\
870	2.49959665743514\\
871	2.49960042725406\\
872	2.49960442962599\\
873	2.49960848600138\\
874	2.49961248814956\\
875	2.49961583062558\\
876	2.49961936832361\\
877	2.49962271291872\\
878	2.49962627388085\\
879	2.49962965881966\\
880	2.49963313992178\\
881	2.49963627162437\\
882	2.49964071428309\\
883	2.49964379523919\\
884	2.49964763139702\\
885	2.49965173406087\\
886	2.49965461821843\\
887	2.4996578023969\\
888	2.49966171888018\\
889	2.49966512270651\\
890	2.49966814826867\\
891	2.49967114801005\\
892	2.49967489763668\\
893	2.49967749692045\\
894	2.4996802030178\\
895	2.49968371217597\\
896	2.49968686692678\\
897	2.49968942101584\\
898	2.49969291721157\\
899	2.49969598897757\\
900	2.49969971869003\\
901	2.49970303644114\\
902	2.49970561732932\\
903	2.49970823831873\\
904	2.49971058409984\\
905	2.49971377657178\\
906	2.4997170876483\\
907	2.49971974977847\\
908	2.49972174559531\\
909	2.49972384472333\\
910	2.49972639157208\\
911	2.49972883328716\\
912	2.49973072610792\\
913	2.4997332894171\\
914	2.49973588820785\\
915	2.49973873582102\\
916	2.4997408500837\\
917	2.49974392404968\\
918	2.49974608324852\\
919	2.49974781517271\\
920	2.49975105948132\\
921	2.49975373576002\\
922	2.49975648621541\\
923	2.49975885027497\\
924	2.49976140184086\\
925	2.4997640559407\\
926	2.49976733543715\\
927	2.49976942930509\\
928	2.49977217526662\\
929	2.49977461555473\\
930	2.49977638384932\\
931	2.49977877910425\\
932	2.49978123219879\\
933	2.49978329518908\\
934	2.49978501760071\\
935	2.49978737016214\\
936	2.49978945599799\\
937	2.4997913911721\\
938	2.4997933126032\\
939	2.49979569967424\\
940	2.49979771515824\\
941	2.49979945569607\\
942	2.49980124895673\\
943	2.49980314956781\\
944	2.49980505341974\\
945	2.49980678098581\\
946	2.49980887354324\\
947	2.49981107713465\\
948	2.49981308480187\\
949	2.49981474103783\\
950	2.4998165285828\\
951	2.4998184421355\\
952	2.49982006868942\\
953	2.49982197458099\\
954	2.49982340358356\\
955	2.49982506400139\\
956	2.49982660022784\\
957	2.49982859181275\\
958	2.49983048039569\\
959	2.49983241405514\\
960	2.49983405018242\\
961	2.49983607303875\\
962	2.49983737436745\\
963	2.49983921396762\\
964	2.49984111556805\\
965	2.49984250746794\\
966	2.49984425037139\\
967	2.49984563534416\\
968	2.49984726010475\\
969	2.499848636942\\
970	2.4998500602258\\
971	2.4998519195267\\
972	2.49985349528291\\
973	2.499854893239\\
974	2.49985627409973\\
975	2.49985810148982\\
976	2.49985968672948\\
977	2.499861395661\\
978	2.4998624715192\\
979	2.49986423193621\\
980	2.49986547828866\\
981	2.49986679892949\\
982	2.49986813579395\\
983	2.49986959506269\\
984	2.49987082551456\\
985	2.49987176294666\\
986	2.49987298392932\\
987	2.49987399807039\\
988	2.49987537656082\\
989	2.49987663196059\\
990	2.49987759055208\\
991	2.49987878029596\\
992	2.49988005893417\\
993	2.49988131130425\\
994	2.49988235021377\\
995	2.49988358786799\\
996	2.49988472430868\\
997	2.49988575688856\\
998	2.4998871112915\\
999	2.49988845203749\\
1000	2.49988960647066\\
};
\addlegendentry{$\Re \{  \mathbf{g}_2^H \}$};

\addplot [color=mycolor1,solid]
  table[row sep=crcr]{%
1	0\\
2	0\\
3	0\\
4	-0.00449685290961247\\
5	-0.00832394302603545\\
6	-0.0144410951337337\\
7	-0.0177374902681555\\
8	-0.019863424014048\\
9	-0.0278722916295163\\
10	-0.0356206896802645\\
11	-0.0408180372246277\\
12	-0.0448711087701628\\
13	-0.0482494836226544\\
14	-0.0544516233239618\\
15	-0.0574680203166815\\
16	-0.0635449001358211\\
17	-0.066915562717075\\
18	-0.0703902202723492\\
19	-0.0740423470549594\\
20	-0.0780035757637871\\
21	-0.0841610763301496\\
22	-0.0867006561065523\\
23	-0.0929352073731613\\
24	-0.0939536454989119\\
25	-0.0981100062286457\\
26	-0.103089454697149\\
27	-0.107589838207103\\
28	-0.110717234276813\\
29	-0.117528103130678\\
30	-0.120346434989929\\
31	-0.124968959120786\\
32	-0.127580728996724\\
33	-0.131731955327304\\
34	-0.133640846513065\\
35	-0.139793203756278\\
36	-0.146594739869209\\
37	-0.151158189726442\\
38	-0.155561715583049\\
39	-0.157606936236401\\
40	-0.159650885691933\\
41	-0.164838836073519\\
42	-0.167205847396198\\
43	-0.170659909152211\\
44	-0.172526546038983\\
45	-0.172826568982478\\
46	-0.175646147091599\\
47	-0.179758082540708\\
48	-0.180982205679904\\
49	-0.181564928536554\\
50	-0.184020563654371\\
51	-0.186473898430587\\
52	-0.189221247017886\\
53	-0.193951512015264\\
54	-0.195984356042855\\
55	-0.19949156979156\\
56	-0.200008505738186\\
57	-0.202450204002538\\
58	-0.207006120141702\\
59	-0.209202145866645\\
60	-0.212895388741484\\
61	-0.213265128973037\\
62	-0.215092027506917\\
63	-0.217245378352042\\
64	-0.221699732201115\\
65	-0.225151401783287\\
66	-0.228880091826288\\
67	-0.232379585034839\\
68	-0.233838507728646\\
69	-0.236680795420332\\
70	-0.239444433384933\\
71	-0.240328024634096\\
72	-0.241836619357328\\
73	-0.245632533993987\\
74	-0.246913157362802\\
75	-0.250597077906794\\
76	-0.251908468897588\\
77	-0.252639303948168\\
78	-0.25479998976991\\
79	-0.256725820838776\\
80	-0.259450057978868\\
81	-0.26261028184579\\
82	-0.264245109106401\\
83	-0.266557702860805\\
84	-0.268094909670577\\
85	-0.27028198557742\\
86	-0.273082209408181\\
87	-0.275130769809534\\
88	-0.27817741041533\\
89	-0.281368070924301\\
90	-0.282648604071896\\
91	-0.285536355066454\\
92	-0.287550667370126\\
93	-0.290042417005562\\
94	-0.292879889718668\\
95	-0.294385220136374\\
96	-0.294722783166324\\
97	-0.296571988732896\\
98	-0.299300540064248\\
99	-0.300070321608233\\
100	-0.301989141266941\\
101	-0.30325751663352\\
102	-0.305099066798024\\
103	-0.307263782402378\\
104	-0.308959713576899\\
105	-0.310488553599863\\
106	-0.312794660136062\\
107	-0.314721260024911\\
108	-0.315880919837674\\
109	-0.318626891673535\\
110	-0.321898796389136\\
111	-0.323805042890809\\
112	-0.3269508350191\\
113	-0.328310091589713\\
114	-0.330525845609373\\
115	-0.331885466691624\\
116	-0.333230116023726\\
117	-0.335412677869353\\
118	-0.336214186922371\\
119	-0.338791347343139\\
120	-0.340805981813864\\
121	-0.34245102889395\\
122	-0.343463649296409\\
123	-0.345566509166725\\
124	-0.347662233011167\\
125	-0.348911867258532\\
126	-0.350199315376038\\
127	-0.351672467167321\\
128	-0.353534283334211\\
129	-0.354695405281587\\
130	-0.356174879942347\\
131	-0.357262885627302\\
132	-0.35880640831419\\
133	-0.359373081486702\\
134	-0.361272163545695\\
135	-0.36279910832193\\
136	-0.363876579562301\\
137	-0.364586240519045\\
138	-0.366151198733557\\
139	-0.367769751853273\\
140	-0.368609702039472\\
141	-0.37065415741964\\
142	-0.372724734127111\\
143	-0.374542512215341\\
144	-0.375457248009893\\
145	-0.378759223860809\\
146	-0.380725224913567\\
147	-0.381445710799773\\
148	-0.382889326744678\\
149	-0.384716726731149\\
150	-0.385273695868536\\
151	-0.386313811014179\\
152	-0.387477195815985\\
153	-0.388399511539793\\
154	-0.389401786865642\\
155	-0.390173582900298\\
156	-0.391504634358792\\
157	-0.392272264312693\\
158	-0.393405649897864\\
159	-0.394665871472664\\
160	-0.395169184253367\\
161	-0.397551589783152\\
162	-0.398619873875631\\
163	-0.399375742691957\\
164	-0.399560961380623\\
165	-0.400304600469371\\
166	-0.40111140732504\\
167	-0.40238045533319\\
168	-0.403246191369573\\
169	-0.403873417118277\\
170	-0.40551923840694\\
171	-0.406252417174297\\
172	-0.407939998321258\\
173	-0.408973060429302\\
174	-0.40971147312652\\
175	-0.410393828923289\\
176	-0.412349586045219\\
177	-0.414413037317111\\
178	-0.415433058218641\\
179	-0.416218409938099\\
180	-0.417258900789077\\
181	-0.418265164353022\\
182	-0.418611650031407\\
183	-0.419463858711563\\
184	-0.420569794676149\\
185	-0.421122913014918\\
186	-0.421657197804784\\
187	-0.422845300406877\\
188	-0.423596291060577\\
189	-0.424353885962326\\
190	-0.42511552513358\\
191	-0.426453175378573\\
192	-0.427266585602165\\
193	-0.427977217552405\\
194	-0.429228123635557\\
195	-0.42982625669763\\
196	-0.430201782153243\\
197	-0.431140749378704\\
198	-0.431969614784137\\
199	-0.43278624538189\\
200	-0.432921590923104\\
201	-0.433571546116977\\
202	-0.434873186912826\\
203	-0.435753795601654\\
204	-0.435865739330856\\
205	-0.436650522067539\\
206	-0.437246035612964\\
207	-0.43798265686953\\
208	-0.438703977459385\\
209	-0.439309877698615\\
210	-0.440149042133339\\
211	-0.440581178722813\\
212	-0.441348623154408\\
213	-0.441687908347252\\
214	-0.442483462379597\\
215	-0.442685064052743\\
216	-0.443141265894099\\
217	-0.443891964548717\\
218	-0.444587116087483\\
219	-0.445270334650606\\
220	-0.445875885580748\\
221	-0.44629842689038\\
222	-0.446822665793915\\
223	-0.447449276851977\\
224	-0.448057371736212\\
225	-0.448689940337096\\
226	-0.449196347367074\\
227	-0.449651336005973\\
228	-0.450540651748008\\
229	-0.450891506471649\\
230	-0.451107955272124\\
231	-0.451445870126945\\
232	-0.451680004842907\\
233	-0.452480015230244\\
234	-0.453379220337885\\
235	-0.453761387392148\\
236	-0.454479402422502\\
237	-0.454918378249602\\
238	-0.455485994809448\\
239	-0.455683934756567\\
240	-0.456168339324009\\
241	-0.456926606903729\\
242	-0.457377906073866\\
243	-0.4577689948938\\
244	-0.458016906412462\\
245	-0.458505897322269\\
246	-0.45879649134081\\
247	-0.45903038592542\\
248	-0.459675186100665\\
249	-0.460003054142919\\
250	-0.46021813946631\\
251	-0.460744285756024\\
252	-0.461381146343522\\
253	-0.461719899869532\\
254	-0.462036850202895\\
255	-0.462464190448966\\
256	-0.462780586853195\\
257	-0.463065947119129\\
258	-0.463592120984962\\
259	-0.463792000376561\\
260	-0.464386582438371\\
261	-0.464833586077011\\
262	-0.465362821386874\\
263	-0.465710270112471\\
264	-0.465866211374478\\
265	-0.46580168225553\\
266	-0.466297997845198\\
267	-0.46663038509493\\
268	-0.466946834550025\\
269	-0.467378378802219\\
270	-0.467917299555171\\
271	-0.468343692646157\\
272	-0.468931384169474\\
273	-0.46954058672216\\
274	-0.46984002969211\\
275	-0.469991596114231\\
276	-0.470357976980469\\
277	-0.470319178707333\\
278	-0.470699713090372\\
279	-0.470777980076219\\
280	-0.471076955957285\\
281	-0.471352394689484\\
282	-0.471754651619204\\
283	-0.472146969630856\\
284	-0.472615144354551\\
285	-0.472888820177126\\
286	-0.473320877353971\\
287	-0.473324566771704\\
288	-0.473406551905131\\
289	-0.473613090487835\\
290	-0.473781982070867\\
291	-0.474209500468607\\
292	-0.474601753953306\\
293	-0.474761892965155\\
294	-0.474926879482678\\
295	-0.475242506587136\\
296	-0.475364252790876\\
297	-0.475418886312552\\
298	-0.475710969519586\\
299	-0.476122946686553\\
300	-0.47638190374535\\
301	-0.476564454874769\\
302	-0.476786341835246\\
303	-0.476889076559578\\
304	-0.477312994485493\\
305	-0.477657878555772\\
306	-0.47776941272253\\
307	-0.478098574804942\\
308	-0.478029070184096\\
309	-0.478326744041731\\
310	-0.478517638734005\\
311	-0.478778881762273\\
312	-0.479018688766803\\
313	-0.479359917264582\\
314	-0.479451750482635\\
315	-0.479914129319945\\
316	-0.480170724702047\\
317	-0.480481656685315\\
318	-0.480634029707537\\
319	-0.480719214579259\\
320	-0.480963274216173\\
321	-0.481130059823292\\
322	-0.481223313782924\\
323	-0.481198532958903\\
324	-0.481452868732179\\
325	-0.48177252390085\\
326	-0.481946354799612\\
327	-0.482082210366741\\
328	-0.482196599111397\\
329	-0.482397079855312\\
330	-0.482545747881219\\
331	-0.482754992209758\\
332	-0.482902772550593\\
333	-0.483103656562933\\
334	-0.483284274401876\\
335	-0.483443456020514\\
336	-0.483612383571424\\
337	-0.483646723207503\\
338	-0.483828084164307\\
339	-0.484122257615305\\
340	-0.484328435307634\\
341	-0.484363405348982\\
342	-0.484512250194654\\
343	-0.484650444038633\\
344	-0.484855670240935\\
345	-0.48508570821166\\
346	-0.485168736319554\\
347	-0.485326775200451\\
348	-0.485386419330791\\
349	-0.485612546673023\\
350	-0.485743223957801\\
351	-0.485913395027793\\
352	-0.485973967639953\\
353	-0.486099292448083\\
354	-0.486255637450703\\
355	-0.48643515717558\\
356	-0.486458800635879\\
357	-0.486650284008091\\
358	-0.486638773326681\\
359	-0.486811243510417\\
360	-0.487071815497474\\
361	-0.487137558276843\\
362	-0.487261215562493\\
363	-0.487400751610298\\
364	-0.487543901144756\\
365	-0.487668386364013\\
366	-0.487817904020042\\
367	-0.487882656031114\\
368	-0.488025321460219\\
369	-0.488236735594294\\
370	-0.488387175410161\\
371	-0.488429171354171\\
372	-0.488490259073316\\
373	-0.488620265751005\\
374	-0.488768597228249\\
375	-0.488883893843854\\
376	-0.488976644941888\\
377	-0.48910159500018\\
378	-0.489226250290772\\
379	-0.489293533616763\\
380	-0.489349341632169\\
381	-0.489327915098828\\
382	-0.489479257034785\\
383	-0.489519558510979\\
384	-0.489680828149104\\
385	-0.48979644147751\\
386	-0.489942830342381\\
387	-0.490070611317775\\
388	-0.490165591155663\\
389	-0.490272680695516\\
390	-0.490421404439334\\
391	-0.49050279117001\\
392	-0.490603374652356\\
393	-0.490757974381343\\
394	-0.490890002111627\\
395	-0.491018073838592\\
396	-0.491142812222782\\
397	-0.49122413695697\\
398	-0.49129499877345\\
399	-0.491419694776673\\
400	-0.491510177671855\\
401	-0.491578608933208\\
402	-0.491649719036638\\
403	-0.491749025596683\\
404	-0.491864056324392\\
405	-0.491957832590112\\
406	-0.492010760199863\\
407	-0.492108411510816\\
408	-0.492217290361069\\
409	-0.492294429609517\\
410	-0.492340307086003\\
411	-0.492405077843342\\
412	-0.492496642037866\\
413	-0.492598699088447\\
414	-0.49263165569993\\
415	-0.492679367026443\\
416	-0.49278344800026\\
417	-0.492873780943877\\
418	-0.492936701440716\\
419	-0.49303390163879\\
420	-0.493082391020492\\
421	-0.493155991911642\\
422	-0.493288669508486\\
423	-0.493329331858204\\
424	-0.493368013441255\\
425	-0.493426520782725\\
426	-0.493500835471497\\
427	-0.49355835610616\\
428	-0.493590739442871\\
429	-0.493717644037396\\
430	-0.493805036936674\\
431	-0.493824302382424\\
432	-0.493867762395274\\
433	-0.493934296682932\\
434	-0.494023629469398\\
435	-0.494118325974195\\
436	-0.494117456060931\\
437	-0.494143554698157\\
438	-0.494187574673867\\
439	-0.494244443598534\\
440	-0.494262208022038\\
441	-0.49432851569827\\
442	-0.494409761555489\\
443	-0.494471546393405\\
444	-0.494499331841271\\
445	-0.494553046331819\\
446	-0.494640111950356\\
447	-0.494677276971408\\
448	-0.494734170700555\\
449	-0.494782494313533\\
450	-0.494855291534147\\
451	-0.49491676872466\\
452	-0.494970779327172\\
453	-0.49500488850897\\
454	-0.495046953686803\\
455	-0.495063433796851\\
456	-0.495123599550416\\
457	-0.495199862131899\\
458	-0.495235947980704\\
459	-0.495314291025021\\
460	-0.49537974515381\\
461	-0.495429384198112\\
462	-0.49545854861611\\
463	-0.495506966570144\\
464	-0.495580006720456\\
465	-0.495615757466169\\
466	-0.495681200250724\\
467	-0.495722689297605\\
468	-0.495764539744377\\
469	-0.49579457549125\\
470	-0.495854662392357\\
471	-0.495893792743053\\
472	-0.495927885078054\\
473	-0.496005023471019\\
474	-0.496035304494364\\
475	-0.49606155686464\\
476	-0.4960666400818\\
477	-0.496117262586973\\
478	-0.496163229739214\\
479	-0.496191623536523\\
480	-0.496208431809209\\
481	-0.496259417628539\\
482	-0.496327830669411\\
483	-0.496375565062771\\
484	-0.496432417493526\\
485	-0.496497719598184\\
486	-0.496542941385211\\
487	-0.496574854741341\\
488	-0.496604390489008\\
489	-0.496638004149821\\
490	-0.496681299529379\\
491	-0.496727877526853\\
492	-0.496768450205283\\
493	-0.496809197004224\\
494	-0.496836992977399\\
495	-0.496862608397084\\
496	-0.49688240337076\\
497	-0.49690270453529\\
498	-0.496924486868261\\
499	-0.496953371000035\\
500	-0.49698421268385\\
501	-0.497011318314363\\
502	-0.497041705456698\\
503	-0.497105456469064\\
504	-0.497125064387714\\
505	-0.497169164442442\\
506	-0.497212908661987\\
507	-0.497250902722742\\
508	-0.497276636241923\\
509	-0.497283652707865\\
510	-0.497304997372921\\
511	-0.497346441647777\\
512	-0.497380718108108\\
513	-0.49740617380572\\
514	-0.497457351249935\\
515	-0.497481276077521\\
516	-0.497516707095777\\
517	-0.497540262754398\\
518	-0.497578580420318\\
519	-0.497598310514177\\
520	-0.497616733077452\\
521	-0.497627860754025\\
522	-0.497645626832248\\
523	-0.497664961741425\\
524	-0.497674058761631\\
525	-0.497711540391145\\
526	-0.497727439292373\\
527	-0.497715356132211\\
528	-0.497757851881556\\
529	-0.497791460904233\\
530	-0.497805579301483\\
531	-0.497830642633863\\
532	-0.497847165553547\\
533	-0.497877675995522\\
534	-0.497888892942366\\
535	-0.497894289228194\\
536	-0.497907598496675\\
537	-0.497925446090297\\
538	-0.497955207498288\\
539	-0.49795578572396\\
540	-0.49797427592432\\
541	-0.497998629310152\\
542	-0.498012605121722\\
543	-0.498041796448464\\
544	-0.498070677517821\\
545	-0.498098238176144\\
546	-0.498114753421874\\
547	-0.498137911725245\\
548	-0.498152467170448\\
549	-0.498183824015105\\
550	-0.498203365112199\\
551	-0.498216527106785\\
552	-0.498229131164427\\
553	-0.498254228999562\\
554	-0.498270447812681\\
555	-0.498272241162346\\
556	-0.498278701814086\\
557	-0.49828015873218\\
558	-0.498291033745642\\
559	-0.498301151768783\\
560	-0.498324475994206\\
561	-0.498348476754367\\
562	-0.498364803573486\\
563	-0.498371367647437\\
564	-0.498388237998717\\
565	-0.498397100296146\\
566	-0.498416216889608\\
567	-0.498430532188775\\
568	-0.498446412206862\\
569	-0.498475624191119\\
570	-0.498512067968347\\
571	-0.498522455993332\\
572	-0.498541107107655\\
573	-0.498564554331022\\
574	-0.498588351966439\\
575	-0.498595987349818\\
576	-0.49860349785797\\
577	-0.498613352713833\\
578	-0.498633396100612\\
579	-0.498652112375572\\
580	-0.498665121864004\\
581	-0.498675763636665\\
582	-0.498669383173863\\
583	-0.498691762198804\\
584	-0.49870201700773\\
585	-0.498708678531979\\
586	-0.498728540123132\\
587	-0.498762135996461\\
588	-0.498775220545112\\
589	-0.498792859163093\\
590	-0.498802869115339\\
591	-0.498812098026656\\
592	-0.498825177654122\\
593	-0.498839468677044\\
594	-0.498842364509107\\
595	-0.498841987949172\\
596	-0.4988521195753\\
597	-0.498859292364721\\
598	-0.498866483929215\\
599	-0.498891984501862\\
600	-0.498899855352529\\
601	-0.498912890472403\\
602	-0.498922783578485\\
603	-0.498923418345787\\
604	-0.498939131659976\\
605	-0.498967109248193\\
606	-0.498977523395463\\
607	-0.498996758889806\\
608	-0.499012177426462\\
609	-0.499019834818314\\
610	-0.499031642731054\\
611	-0.499036378342466\\
612	-0.499050402246782\\
613	-0.499057215746689\\
614	-0.499075917393339\\
615	-0.499094156333642\\
616	-0.499107749143093\\
617	-0.499117338635767\\
618	-0.499130296265061\\
619	-0.499153487154614\\
620	-0.499154207020687\\
621	-0.499162030138444\\
622	-0.499169882728969\\
623	-0.499177210381028\\
624	-0.499192544163085\\
625	-0.499201977226749\\
626	-0.49921342879988\\
627	-0.49921529507878\\
628	-0.499214478502941\\
629	-0.49921500516726\\
630	-0.499216374855574\\
631	-0.499221452948081\\
632	-0.4992243140565\\
633	-0.499232912181117\\
634	-0.499240190129598\\
635	-0.499248280099052\\
636	-0.499256107678475\\
637	-0.499260477747206\\
638	-0.499264223882212\\
639	-0.499271722675366\\
640	-0.499278426088973\\
641	-0.499280568263786\\
642	-0.499289230592558\\
643	-0.4992939215229\\
644	-0.499295669715058\\
645	-0.499301590401855\\
646	-0.499311016777077\\
647	-0.499314021819232\\
648	-0.499321426887468\\
649	-0.499327877761436\\
650	-0.499338706645477\\
651	-0.499346266269689\\
652	-0.499357426820425\\
653	-0.499365723351335\\
654	-0.499370931421476\\
655	-0.499376190679327\\
656	-0.499387435355276\\
657	-0.499389383810049\\
658	-0.499390467493946\\
659	-0.499396869619649\\
660	-0.499407849982506\\
661	-0.499419359079468\\
662	-0.499426343534615\\
663	-0.499435042121434\\
664	-0.499441965589939\\
665	-0.499446329590332\\
666	-0.499448009055788\\
667	-0.499453308796135\\
668	-0.499458870303454\\
669	-0.499463234489953\\
670	-0.499467188034682\\
671	-0.499474258532655\\
672	-0.499478805058336\\
673	-0.499486595623311\\
674	-0.499488989053341\\
675	-0.499497977826764\\
676	-0.499502129785175\\
677	-0.499505153585204\\
678	-0.499512106456327\\
679	-0.49951745531148\\
680	-0.499519636952657\\
681	-0.499519664409499\\
682	-0.499523506049221\\
683	-0.499528710218981\\
684	-0.499528820921945\\
685	-0.499534732045704\\
686	-0.499542457516076\\
687	-0.499544872655917\\
688	-0.499550132322274\\
689	-0.499555666592298\\
690	-0.499559504869933\\
691	-0.49956784674271\\
692	-0.499570092998403\\
693	-0.499575441542965\\
694	-0.499575198121839\\
695	-0.499577300591736\\
696	-0.499588644302375\\
697	-0.499593597718168\\
698	-0.499590784532221\\
699	-0.499595150470955\\
700	-0.499599686268904\\
701	-0.499602828328254\\
702	-0.499605783833324\\
703	-0.499608751303693\\
704	-0.499612922396298\\
705	-0.499617754521617\\
706	-0.499621982913843\\
707	-0.499627710743648\\
708	-0.499632384985376\\
709	-0.499638007467307\\
710	-0.499641066056122\\
711	-0.499641834041342\\
712	-0.499645163901093\\
713	-0.499651990619531\\
714	-0.499656531728641\\
715	-0.499656983555202\\
716	-0.499658932645473\\
717	-0.49966310153015\\
718	-0.499664989438546\\
719	-0.499666855820988\\
720	-0.49966675677728\\
721	-0.499669670456088\\
722	-0.499674919163934\\
723	-0.499679093487185\\
724	-0.499681952171118\\
725	-0.499685710664796\\
726	-0.499689810440515\\
727	-0.49969380513424\\
728	-0.499694218824191\\
729	-0.499699065739842\\
730	-0.49970368233328\\
731	-0.499705739863931\\
732	-0.499706459899457\\
733	-0.49971002414871\\
734	-0.499712273840684\\
735	-0.499713120571555\\
736	-0.499716324682247\\
737	-0.499721280747512\\
738	-0.499725324222757\\
739	-0.499727538230341\\
740	-0.499731465457275\\
741	-0.49973271618053\\
742	-0.499735112646075\\
743	-0.499736345935878\\
744	-0.499737799939015\\
745	-0.499740360342925\\
746	-0.499741682414405\\
747	-0.499742346161585\\
748	-0.499744949508971\\
749	-0.4997491146481\\
750	-0.499752009608566\\
751	-0.499754821513368\\
752	-0.499758300711799\\
753	-0.499758945509731\\
754	-0.499763286101585\\
755	-0.499767330319679\\
756	-0.499769667581207\\
757	-0.499772051386602\\
758	-0.499773561149275\\
759	-0.499776483380567\\
760	-0.499776130103925\\
761	-0.499776848610334\\
762	-0.499779682119475\\
763	-0.499782866164931\\
764	-0.499786830621312\\
765	-0.499788713572376\\
766	-0.499790937416937\\
767	-0.499792588893843\\
768	-0.499794429583492\\
769	-0.499798770901078\\
770	-0.499799661963423\\
771	-0.499803052120331\\
772	-0.499803881510838\\
773	-0.499804160199838\\
774	-0.499807216621061\\
775	-0.499808641548957\\
776	-0.499812260787322\\
777	-0.499813354357477\\
778	-0.499815214987561\\
779	-0.499816616388319\\
780	-0.499817685267176\\
781	-0.499820351477425\\
782	-0.499822823438129\\
783	-0.499824506402746\\
784	-0.499826351833181\\
785	-0.499827604042298\\
786	-0.499829183009414\\
787	-0.499830276960136\\
788	-0.499831781758882\\
789	-0.499833985836547\\
790	-0.499834514959254\\
791	-0.499834689465519\\
792	-0.499835618217277\\
793	-0.499837862024676\\
794	-0.499841270891759\\
795	-0.499844159747632\\
796	-0.499844688056136\\
797	-0.499846974816849\\
798	-0.499846448069758\\
799	-0.49984747700541\\
800	-0.499849067771316\\
801	-0.499849522461305\\
802	-0.499852813438488\\
803	-0.499855354102752\\
804	-0.499856171948694\\
805	-0.499857884967721\\
806	-0.499859329252559\\
807	-0.499861186612254\\
808	-0.499863464954445\\
809	-0.499865875888095\\
810	-0.499866638711498\\
811	-0.499867185891339\\
812	-0.499867528087231\\
813	-0.499869569839829\\
814	-0.499870129612321\\
815	-0.499870630607596\\
816	-0.49987282261598\\
817	-0.499874958397408\\
818	-0.499876166266471\\
819	-0.499876137050326\\
820	-0.49987866838275\\
821	-0.499880316063551\\
822	-0.499880559352439\\
823	-0.499881450114663\\
824	-0.499883124732412\\
825	-0.499883494922655\\
826	-0.499885699493198\\
827	-0.499886880135773\\
828	-0.499888723316713\\
829	-0.49989102110091\\
830	-0.499892599824975\\
831	-0.499893879342724\\
832	-0.499895016783605\\
833	-0.499895813277696\\
834	-0.499896829215108\\
835	-0.499898493270582\\
836	-0.499899581159483\\
837	-0.499901388538403\\
838	-0.499902354853641\\
839	-0.499904420215185\\
840	-0.499904603318418\\
841	-0.499905731991562\\
842	-0.499906512732717\\
843	-0.499907665261424\\
844	-0.499907808364007\\
845	-0.499909871135427\\
846	-0.499911088440273\\
847	-0.499912146153799\\
848	-0.499913150857325\\
849	-0.499913823694275\\
850	-0.499914912575621\\
851	-0.499915207036312\\
852	-0.499916784367834\\
853	-0.499917562847516\\
854	-0.49991819382199\\
855	-0.499919442686022\\
856	-0.49992019713178\\
857	-0.499921158746614\\
858	-0.499922891459904\\
859	-0.499924440826144\\
860	-0.499925668202416\\
861	-0.499926427833237\\
862	-0.499926783319991\\
863	-0.499927147238048\\
864	-0.499928476644372\\
865	-0.499929225716186\\
866	-0.49992908124848\\
867	-0.499930018008924\\
868	-0.499930023835682\\
869	-0.499930841384478\\
870	-0.499932324569444\\
871	-0.49993337477712\\
872	-0.499933484569094\\
873	-0.499934035433599\\
874	-0.499934539101166\\
875	-0.499935477965819\\
876	-0.499935980238324\\
877	-0.49993651690344\\
878	-0.499937263134829\\
879	-0.499937746471198\\
880	-0.499938370955063\\
881	-0.499938784771027\\
882	-0.499939589215989\\
883	-0.499939994593189\\
884	-0.49994103829611\\
885	-0.499941521709529\\
886	-0.499942222898588\\
887	-0.499942359544629\\
888	-0.499943252909858\\
889	-0.499944107511812\\
890	-0.499944554597667\\
891	-0.499944809879001\\
892	-0.499945588712222\\
893	-0.499946289834882\\
894	-0.4999462871738\\
895	-0.499947421499861\\
896	-0.499948430485795\\
897	-0.49994889686274\\
898	-0.499949783414837\\
899	-0.499950476647536\\
900	-0.499950612920582\\
901	-0.49995057596726\\
902	-0.499951123064802\\
903	-0.499951796209971\\
904	-0.499952587750196\\
905	-0.499953136653969\\
906	-0.49995349143745\\
907	-0.499954263022495\\
908	-0.499954950548268\\
909	-0.499955490350496\\
910	-0.49995594693913\\
911	-0.499956899137389\\
912	-0.499957362273265\\
913	-0.499958132162778\\
914	-0.499958184025179\\
915	-0.499958885431051\\
916	-0.499959362492488\\
917	-0.499959640106591\\
918	-0.499959759043512\\
919	-0.499960136471682\\
920	-0.49996067348634\\
921	-0.499961632878469\\
922	-0.499961871657126\\
923	-0.499962429403314\\
924	-0.499962415725936\\
925	-0.499963240433366\\
926	-0.49996333827433\\
927	-0.499963585444499\\
928	-0.499963939791914\\
929	-0.499964193637579\\
930	-0.499964755251663\\
931	-0.499965482399758\\
932	-0.499965862946871\\
933	-0.499966041321533\\
934	-0.499966908238958\\
935	-0.499967414751125\\
936	-0.499967848602357\\
937	-0.499968162089241\\
938	-0.499968232491017\\
939	-0.499968703186665\\
940	-0.499968880544355\\
941	-0.499968990367449\\
942	-0.499969361702715\\
943	-0.499969648895743\\
944	-0.499969939631822\\
945	-0.499970084801936\\
946	-0.49997026363827\\
947	-0.499970627974683\\
948	-0.499970998082746\\
949	-0.499971149779112\\
950	-0.499971495093615\\
951	-0.499971618121992\\
952	-0.49997191065731\\
953	-0.499971891989108\\
954	-0.499972292234232\\
955	-0.499972591570472\\
956	-0.499972788762511\\
957	-0.499972773325571\\
958	-0.499973086347674\\
959	-0.499973255883347\\
960	-0.499973333665756\\
961	-0.499973876887228\\
962	-0.499974082177624\\
963	-0.499974379775755\\
964	-0.499974457962554\\
965	-0.499974684365294\\
966	-0.499975072549345\\
967	-0.499975359251779\\
968	-0.499975496439649\\
969	-0.499975941504564\\
970	-0.499976089808551\\
971	-0.49997637390104\\
972	-0.499976631712004\\
973	-0.499977135366138\\
974	-0.499977561826711\\
975	-0.499977834823028\\
976	-0.499977968813061\\
977	-0.499978554724962\\
978	-0.499978711696123\\
979	-0.499979321397216\\
980	-0.499979346940345\\
981	-0.49998023359442\\
982	-0.499980324515045\\
983	-0.499980369541399\\
984	-0.499980459565476\\
985	-0.499980600085886\\
986	-0.499980706416983\\
987	-0.499981033315934\\
988	-0.499981450029091\\
989	-0.499981555783621\\
990	-0.499981837322939\\
991	-0.499981916483488\\
992	-0.499982268614692\\
993	-0.499982480436317\\
994	-0.499982601083651\\
995	-0.499982822717406\\
996	-0.499982930147056\\
997	-0.499983006559334\\
998	-0.4999831944907\\
999	-0.499983485308105\\
1000	-0.49998358637908\\
};
\addlegendentry{$\Im \{ \mathbf{g}_2^H \}$};

\end{axis}
\end{tikzpicture}%}
		\caption{\textit{Esimated Weights (both real and imaginary components)}}
		\label{fig:4_1_a_aclms_weights}
	\end{subfigure}
	\caption{\textit{Learning Curve and Estimated Weights from the ACLMS fitler}}
	\label{fig:4_1_a_aclms}
\end{figure}

We can conclude that the CLMS was unable to represent the WLMA properly - the coefficients of $b_2$ which were from the conjugate of $x$ were nowhere to be seen in the CLMS, but very clear in the ACLMS. The fact that the learning curve of the CLMS showed a large error is evidence that the model was a poor fit for the CLMS. From the work carried out here, there appears to be no mathematical disadvantage to using the ACLMS, even if the data does not need it to be represented, since in the worst case the $ \mathbf{g} $ terms would settle at 0 (and it would behave as if it were a CLMS filter).

%  \begin{figure}[h]
%  	\centering 
% 	\resizebox{0.6\textwidth}{!}{\input{q5/q5_cum.tikz}}
%   	\caption{\textit{Cumulative representation of the variance of each eigenvector}}
%   	\label{fig:q5_4}
%  \end{figure}


% \begin{equation}
% S = \frac{1}{N}X' X'^T
% \end{equation}


%  \begin{figure}[h]
%  	\centering 
% 	\resizebox{0.6\textwidth}{!}{\input{q5/q5_cum.tikz}}
%   	\caption{\textit{Cumulative representation of the variance of each eigenvector}}
%   	\label{fig:q5_4}
%  \end{figure}

% \begin{figure}[h]
% 	\centering 
%  	\setlength\figureheight{0.4\textwidth}
% 	\setlength\figurewidth{0.7\textwidth} 
%  	\input{p_1/1.tikz}
%  	\caption{\textit{The four randomly generated subsets}}
%  	\label{fig:q1}
% \end{figure}


%  \begin{figure}[h]
%          \centering
%          \begin{subfigure}[b]{0.45\textwidth}
%             \resizebox{\textwidth}{!}{\input{part_4/q8_num_1.tikz}}
%   			\caption{\textit{1 Tree}}
%          \end{subfigure}
%          ~ %add desired spacing between images, e. g. ~, \quad, \qquad, \hfill etc.
%           %(or a blank line to force the subfigure onto a new line)
%          \begin{subfigure}[b]{0.45\textwidth}
%             \resizebox{\textwidth}{!}{\input{part_4/q8_num_3.tikz}}
%   			\caption{\textit{2 Trees}}
%          \end{subfigure}
 		
%          \begin{subfigure}[b]{0.45\textwidth}
%             \resizebox{\textwidth}{!}{\input{part_4/q8_num_5.tikz}}
%   			\caption{\textit{5 Trees}}
%          \end{subfigure}
%          ~ %add desired spacing between images, e. g. ~, \quad, \qquad, \hfill etc.
%           %(or a blank line to force the subfigure onto a new line)
%          \begin{subfigure}[b]{0.45\textwidth}
%             \resizebox{\textwidth}{!}{\input{part_4/q8_num_10.tikz}}
%   			\caption{\textit{10 Trees}}
%          \end{subfigure}
         
%          \begin{subfigure}[b]{0.45\textwidth}
%             \resizebox{\textwidth}{!}{\input{part_4/q8_num_20.tikz}}
%   			\caption{\textit{20 Trees}}
%          \end{subfigure}
 		
%  		\label{q9i}
% 		\caption{\textit{Varying the Number of Trees in the Forest}}
%  \end{figure}

\end{document}

