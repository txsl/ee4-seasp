\documentclass[./main.tex]{subfiles} 
\begin{document}

\section{Widely Linear Filtering and Adaptive Spectral Estimation}

\subsection{Complex LMS and Widely Linear Modelling}

\subsubsection{The CLMS and ACLMS}
We are introduced to the Complex LMS, which is a modification of the LMS filter, but designed to operate with complex signals. The $ \mathbf{w}$ weight term is replaced with $ \mathbf{h}$, and the estimation is now $ \hat{y}(n) = \mathbf{h}^H(n) \mathbf{x}(n) $, with the weight update being performed as $ \mathbf{h}(n+1) = \mathbf{h}(n) + \mu e^{\ast}(n)\mathbf{x}(n) $.

We are also introduced to the Augmented CLMS, which is designed to capture the second order statistical relationship between 

We generated a WLMA(1) (Widely Linear Moving Average) process which is defined as $ y(n) = x(n) + b_1 x(n-1) + b_2 x^{\ast}(n-2) $, where $ x \sim \mathcal{N} (0,1) $. The coefficients are defined as $ b_1 = 1.5 + 1j $ and $ b_2 = 2.5 - 0.5j$.

%  \begin{figure}[h]
%  	\centering 
% 	\resizebox{0.6\textwidth}{!}{\input{q5/q5_cum.tikz}}
%   	\caption{\textit{Cumulative representation of the variance of each eigenvector}}
%   	\label{fig:q5_4}
%  \end{figure}


% \begin{equation}
% S = \frac{1}{N}X' X'^T
% \end{equation}


%  \begin{figure}[h]
%  	\centering 
% 	\resizebox{0.6\textwidth}{!}{\input{q5/q5_cum.tikz}}
%   	\caption{\textit{Cumulative representation of the variance of each eigenvector}}
%   	\label{fig:q5_4}
%  \end{figure}

% \begin{figure}[h]
% 	\centering 
%  	\setlength\figureheight{0.4\textwidth}
% 	\setlength\figurewidth{0.7\textwidth} 
%  	\input{p_1/1.tikz}
%  	\caption{\textit{The four randomly generated subsets}}
%  	\label{fig:q1}
% \end{figure}


%  \begin{figure}[h]
%          \centering
%          \begin{subfigure}[b]{0.45\textwidth}
%             \resizebox{\textwidth}{!}{\input{part_4/q8_num_1.tikz}}
%   			\caption{\textit{1 Tree}}
%          \end{subfigure}
%          ~ %add desired spacing between images, e. g. ~, \quad, \qquad, \hfill etc.
%           %(or a blank line to force the subfigure onto a new line)
%          \begin{subfigure}[b]{0.45\textwidth}
%             \resizebox{\textwidth}{!}{\input{part_4/q8_num_3.tikz}}
%   			\caption{\textit{2 Trees}}
%          \end{subfigure}
 		
%          \begin{subfigure}[b]{0.45\textwidth}
%             \resizebox{\textwidth}{!}{\input{part_4/q8_num_5.tikz}}
%   			\caption{\textit{5 Trees}}
%          \end{subfigure}
%          ~ %add desired spacing between images, e. g. ~, \quad, \qquad, \hfill etc.
%           %(or a blank line to force the subfigure onto a new line)
%          \begin{subfigure}[b]{0.45\textwidth}
%             \resizebox{\textwidth}{!}{\input{part_4/q8_num_10.tikz}}
%   			\caption{\textit{10 Trees}}
%          \end{subfigure}
         
%          \begin{subfigure}[b]{0.45\textwidth}
%             \resizebox{\textwidth}{!}{\input{part_4/q8_num_20.tikz}}
%   			\caption{\textit{20 Trees}}
%          \end{subfigure}
 		
%  		\label{q9i}
% 		\caption{\textit{Varying the Number of Trees in the Forest}}
%  \end{figure}

\end{document}

