\documentclass[./main.tex]{subfiles} 
\begin{document}

\subsection{Spectrum of Autoregressive Processes}

\subsubsection{Shortcomings of the Unbiased ACF}
As discussed in \cite{Mandic2014}, the unbiased ACF may not always be invertible. In order to use modern spectrum estimation techniques such as this section, the autocorrelation matrix must be invterible. If it is not, we are unable to continue using the Yule Walker equations to compute weights for the estimated model.

\subsubsection{Modelling an AR Process}

Given the equation $ x(n) = 2.76x(n-1) -3.81x(n-2) + 2.65x(n-3) - 0.92x(n-4) + w(n) $ where $ w \sim \mathcal{N} (0, 1)  $. We are also adviced to discard the first 500 samples as the filter initiailises itself. Given a range of potential model orders to estimate the signal from, we find that an AR(5) process is liekly the best fit, as determined by visual inspection. There appears to be little value increasing the model order significantly - as the peaks come more defined, but overall they add little more.

\begin{figure}[h]
	\centering 
	\resizebox{\textwidth}{!}{% This file was created by matlab2tikz v0.4.7 (commit a43cd4b78840fd166f3a8d462e163c30134293e1) running on MATLAB 8.3.
% Copyright (c) 2008--2014, Nico Schlömer <nico.schloemer@gmail.com>
% All rights reserved.
% Minimal pgfplots version: 1.3
% 
% The latest updates can be retrieved from
%   http://www.mathworks.com/matlabcentral/fileexchange/22022-matlab2tikz
% where you can also make suggestions and rate matlab2tikz.
% 
%
% defining custom colors
\definecolor{mycolor1}{rgb}{0.00000,0.00000,0.17241}%
\definecolor{mycolor2}{rgb}{1.00000,0.10345,0.72414}%
\definecolor{mycolor3}{rgb}{1.00000,0.82759,0.00000}%
%
\begin{tikzpicture}

\begin{axis}[%
width=8in,
height=2in,
scale only axis,
xmin=0,
xmax=0.5,
xlabel={Normalised Frequency},
xmajorgrids,
ymin=20,
ymax=180,
ylabel={Power (dB)},
ymajorgrids,
axis x line*=bottom,
axis y line*=left,
legend style={draw=black,fill=white,legend cell align=left}
]
\addplot [color=blue,solid,forget plot]
  table[row sep=crcr]{-1	58.0983474944829\\
-0.99951171875	53.3349444554309\\
-0.9990234375	35.6293272090586\\
-0.99853515625	48.6729390744488\\
-0.998046875	55.3684151402721\\
-0.99755859375	50.6950691583429\\
-0.9970703125	20.0468058851133\\
-0.99658203125	47.8615754958149\\
-0.99609375	55.4744048311516\\
-0.99560546875	51.4322599340343\\
-0.9951171875	31.4533057049926\\
-0.99462890625	49.5814448690386\\
-0.994140625	57.0900964830545\\
-0.99365234375	53.8502075854728\\
-0.9931640625	32.1602005392821\\
-0.99267578125	46.5826841726847\\
-0.9921875	56.563042542526\\
-0.99169921875	54.4740243136231\\
-0.9912109375	37.3435825682477\\
-0.99072265625	48.3005753350028\\
-0.990234375	58.6697652649785\\
-0.98974609375	58.1942794701978\\
-0.9892578125	45.6964430313352\\
-0.98876953125	36.7783851608077\\
-0.98828125	54.6718449948536\\
-0.98779296875	55.9400579070192\\
-0.9873046875	44.0779898872455\\
-0.98681640625	37.9107456536076\\
-0.986328125	55.4051519870759\\
-0.98583984375	57.529461195891\\
-0.9853515625	48.4923309549828\\
-0.98486328125	27.3727875491823\\
-0.984375	52.2417330559808\\
-0.98388671875	55.7166052687792\\
-0.9833984375	47.0597099147624\\
-0.98291015625	30.4933012359776\\
-0.982421875	53.2614676292987\\
-0.98193359375	57.2559639040886\\
-0.9814453125	50.8461106528318\\
-0.98095703125	35.8761184962632\\
-0.98046875	52.5286604874996\\
-0.97998046875	57.3301426523455\\
-0.9794921875	51.8055566530012\\
-0.97900390625	23.5738082282284\\
-0.978515625	48.3137825385711\\
-0.97802734375	55.257750798976\\
-0.9775390625	50.8318975161185\\
-0.97705078125	39.2120511712081\\
-0.9765625	52.3969160248261\\
-0.97607421875	57.5954545930468\\
-0.9755859375	52.8194891005743\\
-0.97509765625	27.3650366373428\\
-0.974609375	49.6340069704897\\
-0.97412109375	57.7677265574538\\
-0.9736328125	54.9566091353663\\
-0.97314453125	35.7426627244779\\
-0.97265625	45.9043301080298\\
-0.97216796875	56.6516607257251\\
-0.9716796875	55.3600112796424\\
-0.97119140625	40.3170173429863\\
-0.970703125	42.0975462180886\\
-0.97021484375	54.5274221208016\\
-0.9697265625	53.3999674293199\\
-0.96923828125	35.1625579003576\\
-0.96875	44.9330708563706\\
-0.96826171875	57.076419692148\\
-0.9677734375	57.0839501982978\\
-0.96728515625	45.3122326367133\\
-0.966796875	39.6962699780208\\
-0.96630859375	54.8811036378002\\
-0.9658203125	55.8540758471102\\
-0.96533203125	43.1040641071204\\
-0.96484375	36.9881384521847\\
-0.96435546875	55.8960558715602\\
-0.9638671875	58.3044829532781\\
-0.96337890625	49.8447072298583\\
-0.962890625	24.0610445408795\\
-0.96240234375	52.2327510944812\\
-0.9619140625	56.4215982847581\\
-0.96142578125	48.7733983137998\\
-0.9609375	26.350840439487\\
-0.96044921875	52.7733204273626\\
-0.9599609375	57.6877510236431\\
-0.95947265625	52.1438528800313\\
-0.958984375	30.3763115541596\\
-0.95849609375	49.8460304317489\\
-0.9580078125	56.1377905200799\\
-0.95751953125	51.0958882022726\\
-0.95703125	16.2200081486803\\
-0.95654296875	48.0364260753046\\
-0.9560546875	55.3959763884952\\
-0.95556640625	50.7926700560077\\
-0.955078125	30.2592375738934\\
-0.95458984375	51.0536802675657\\
-0.9541015625	58.2128232206991\\
-0.95361328125	55.4334584178517\\
-0.953125	40.0259900119771\\
-0.95263671875	45.8979248997389\\
-0.9521484375	55.2840733615383\\
-0.95166015625	52.7827594585524\\
-0.951171875	33.6206722291856\\
-0.95068359375	48.2071839461221\\
-0.9501953125	57.9119529315748\\
-0.94970703125	56.8564869867101\\
-0.94921875	42.9744280587303\\
-0.94873046875	37.9885097537022\\
-0.9482421875	53.9237482197664\\
-0.94775390625	54.1761701116264\\
-0.947265625	40.8337324041267\\
-0.94677734375	43.4568701157208\\
-0.9462890625	56.1683158415783\\
-0.94580078125	56.7369943283008\\
-0.9453125	45.3684059736904\\
-0.94482421875	36.0303062013504\\
-0.9443359375	54.245052493009\\
-0.94384765625	56.2704053762278\\
-0.943359375	46.3375297073635\\
-0.94287109375	33.4693071325028\\
-0.9423828125	53.7309363093136\\
-0.94189453125	56.634398092533\\
-0.94140625	47.9674983794002\\
-0.94091796875	30.6225137258101\\
-0.9404296875	53.7251596051564\\
-0.93994140625	58.0492116521947\\
-0.939453125	52.4498912045414\\
-0.93896484375	37.567885522\\
-0.9384765625	51.7368831706263\\
-0.93798828125	57.0650840553271\\
-0.9375	52.2090639278373\\
-0.93701171875	26.1320680134492\\
-0.9365234375	46.1866153091389\\
-0.93603515625	53.912628859697\\
-0.935546875	49.5661266870821\\
-0.93505859375	37.1988382402091\\
-0.9345703125	51.4273548650589\\
-0.93408203125	56.8148046131265\\
-0.93359375	52.3530086911459\\
-0.93310546875	34.4613621666478\\
-0.9326171875	49.5006765268705\\
-0.93212890625	56.4592405808957\\
-0.931640625	52.2659158300967\\
-0.93115234375	21.6137142094722\\
-0.9306640625	49.6521980361008\\
-0.93017578125	58.5442321184873\\
-0.9296875	56.5740285912342\\
-0.92919921875	39.7446436625875\\
-0.9287109375	43.2722785075131\\
-0.92822265625	56.5930486393485\\
-0.927734375	56.5130040422184\\
-0.92724609375	43.4153908200508\\
-0.9267578125	39.4759257857553\\
-0.92626953125	55.3236387787002\\
-0.92578125	56.6096055029422\\
-0.92529296875	46.5208447255563\\
-0.9248046875	36.8076419460423\\
-0.92431640625	53.3925561976794\\
-0.923828125	55.3500609634261\\
-0.92333984375	45.0355065642554\\
-0.9228515625	29.6649708450413\\
-0.92236328125	52.2351739501846\\
-0.921875	54.9007875975951\\
-0.92138671875	45.4901005444441\\
-0.9208984375	36.0149564145193\\
-0.92041015625	53.8173395227466\\
-0.919921875	56.7702734439299\\
-0.91943359375	48.8337292445075\\
-0.9189453125	29.994967435579\\
-0.91845703125	52.7070943076116\\
-0.91796875	57.3571317412287\\
-0.91748046875	51.6039245391751\\
-0.9169921875	24.8083680944277\\
-0.91650390625	48.7768866306368\\
-0.916015625	55.5370925370013\\
-0.91552734375	51.1007490992783\\
-0.9150390625	29.6396157198268\\
-0.91455078125	46.5901352671491\\
-0.9140625	53.6133410139346\\
-0.91357421875	48.5071113562976\\
-0.9130859375	19.3600451848867\\
-0.91259765625	48.2576753501859\\
-0.912109375	54.8312071302776\\
-0.91162109375	49.9517161600543\\
-0.9111328125	34.0254461621899\\
-0.91064453125	52.1104282171718\\
-0.91015625	58.5205430255728\\
-0.90966796875	55.3307158620561\\
-0.9091796875	37.8336115287087\\
-0.90869140625	46.5777562127638\\
-0.908203125	56.5539604980364\\
-0.90771484375	54.9049501701691\\
-0.9072265625	39.3967970627028\\
-0.90673828125	45.2664250240157\\
-0.90625	56.3765888327907\\
-0.90576171875	55.6284753021904\\
-0.9052734375	41.5848404899827\\
-0.90478515625	42.3679379522857\\
-0.904296875	55.5994935273121\\
-0.90380859375	55.5281874740219\\
-0.9033203125	40.4388773179097\\
-0.90283203125	41.4249775609506\\
-0.90234375	57.5146260393235\\
-0.90185546875	59.5921205047397\\
-0.9013671875	51.987590387489\\
-0.90087890625	30.5856056086611\\
-0.900390625	50.5824657545643\\
-0.89990234375	54.7952334195619\\
-0.8994140625	46.7868981548524\\
-0.89892578125	27.6654339108595\\
-0.8984375	51.4299853774601\\
-0.89794921875	55.4125835681151\\
-0.8974609375	48.1199326547603\\
-0.89697265625	29.9428045384473\\
-0.896484375	51.2521131004214\\
-0.89599609375	55.6751000014021\\
-0.8955078125	49.2106183413267\\
-0.89501953125	33.8349964567216\\
-0.89453125	51.878415728587\\
-0.89404296875	56.6209833472496\\
-0.8935546875	50.802548565241\\
-0.89306640625	30.5877951223842\\
-0.892578125	51.4231921428708\\
-0.89208984375	57.6396771614726\\
-0.8916015625	53.6244677946504\\
-0.89111328125	32.2564968907961\\
-0.890625	46.5402944138648\\
-0.89013671875	55.2448364567476\\
-0.8896484375	51.7949457206458\\
-0.88916015625	25.4050070302493\\
-0.888671875	46.747122261753\\
-0.88818359375	55.9453944074399\\
-0.8876953125	53.4161521689305\\
-0.88720703125	37.7525811824899\\
-0.88671875	48.8401856370156\\
-0.88623046875	57.6295842207152\\
-0.8857421875	55.9571626858881\\
-0.88525390625	40.0515807699049\\
-0.884765625	43.2995749156726\\
-0.88427734375	56.5192515441822\\
-0.8837890625	56.8390045314416\\
-0.88330078125	44.9714905487449\\
-0.8828125	30.4549034866642\\
-0.88232421875	52.1052540062562\\
-0.8818359375	53.5465957748819\\
-0.88134765625	41.4298664626898\\
-0.880859375	39.8044867997201\\
-0.88037109375	54.3197343327254\\
-0.8798828125	55.3881241156607\\
-0.87939453125	45.1419825845073\\
-0.87890625	42.0740593905369\\
-0.87841796875	55.2623655916051\\
-0.8779296875	56.471678094049\\
-0.87744140625	45.4975198215686\\
-0.876953125	39.9221033086327\\
-0.87646484375	57.1362543762058\\
-0.8759765625	60.2955495505245\\
-0.87548828125	53.9221607264814\\
-0.875	25.556339547249\\
-0.87451171875	50.1080851485299\\
-0.8740234375	56.8730590216271\\
-0.87353515625	52.7581995569812\\
-0.873046875	39.7805944452816\\
-0.87255859375	50.6950377632013\\
-0.8720703125	56.1398441112359\\
-0.87158203125	51.2237239617884\\
-0.87109375	29.2637187875442\\
-0.87060546875	48.586844512757\\
-0.8701171875	55.2175749308394\\
-0.86962890625	50.0234713540816\\
-0.869140625	23.9123104840939\\
-0.86865234375	50.8458852962468\\
-0.8681640625	57.6559368390684\\
-0.86767578125	53.6515904172449\\
-0.8671875	28.4896757318356\\
-0.86669921875	49.3969348599845\\
-0.8662109375	58.6143813576012\\
-0.86572265625	56.9450653120085\\
-0.865234375	41.6559040288686\\
-0.86474609375	40.4224473709763\\
-0.8642578125	54.4928296688578\\
-0.86376953125	53.7971165007171\\
-0.86328125	38.5405648809359\\
-0.86279296875	45.2481666764759\\
-0.8623046875	56.3309503882954\\
-0.86181640625	55.3199250099371\\
-0.861328125	38.8498284429498\\
-0.86083984375	45.7681334370733\\
-0.8603515625	58.8507865160712\\
-0.85986328125	59.9436208622953\\
-0.859375	50.8042960787726\\
-0.85888671875	30.2457283979124\\
-0.8583984375	53.0540677286146\\
-0.85791015625	56.5599597602185\\
-0.857421875	48.3487411523812\\
-0.85693359375	24.6726221712806\\
-0.8564453125	51.2191457885018\\
-0.85595703125	54.9763564880975\\
-0.85546875	46.5768344577154\\
-0.85498046875	34.2677019774362\\
-0.8544921875	53.1274448995444\\
-0.85400390625	56.2852135513751\\
-0.853515625	47.806430522795\\
-0.85302734375	34.9263739442567\\
-0.8525390625	55.1481563620925\\
-0.85205078125	59.4224211782826\\
-0.8515625	53.8274960535374\\
-0.85107421875	20.5554689615695\\
-0.8505859375	48.1165752510922\\
-0.85009765625	55.5925152205242\\
-0.849609375	50.342363944172\\
-0.84912109375	28.6129019365405\\
-0.8486328125	52.2663522712794\\
-0.84814453125	58.5458013965374\\
-0.84765625	54.2278999001133\\
-0.84716796875	24.4613971337424\\
-0.8466796875	49.0762380513162\\
-0.84619140625	58.2444493859075\\
-0.845703125	55.8323158991894\\
-0.84521484375	38.4807400703254\\
-0.8447265625	47.5697074441648\\
-0.84423828125	57.858898114234\\
-0.84375	56.3838766621473\\
-0.84326171875	39.3008616248875\\
-0.8427734375	44.5470312530222\\
-0.84228515625	57.799908292314\\
-0.841796875	58.0590506967182\\
-0.84130859375	46.5719086392043\\
-0.8408203125	40.1099443547011\\
-0.84033203125	55.6977014469591\\
-0.83984375	57.1290175142254\\
-0.83935546875	46.4926094745846\\
-0.8388671875	33.9594446596194\\
-0.83837890625	54.1341409927289\\
-0.837890625	56.6420056566601\\
-0.83740234375	47.8022298494352\\
-0.8369140625	36.8610666025875\\
-0.83642578125	53.8779110197028\\
-0.8359375	56.7441863765519\\
-0.83544921875	48.8470206700534\\
-0.8349609375	36.2456808478305\\
-0.83447265625	52.9592293470138\\
-0.833984375	56.0545286993953\\
-0.83349609375	47.669463568278\\
-0.8330078125	37.8765856986432\\
-0.83251953125	55.3348295895813\\
-0.83203125	59.0480508049702\\
-0.83154296875	52.8333748506597\\
-0.8310546875	29.4353299725857\\
-0.83056640625	51.078007619649\\
-0.830078125	57.0793846071514\\
-0.82958984375	51.9988271894758\\
-0.8291015625	37.2510356230004\\
-0.82861328125	53.2020794097192\\
-0.828125	58.911648117791\\
-0.82763671875	54.5629653847049\\
-0.8271484375	27.977555820669\\
-0.82666015625	48.1448070498159\\
-0.826171875	57.2547392504058\\
-0.82568359375	54.3568290318361\\
-0.8251953125	33.7315791191501\\
-0.82470703125	47.8284618903961\\
-0.82421875	57.7481436284259\\
-0.82373046875	56.3553741311991\\
-0.8232421875	42.5160447171166\\
-0.82275390625	41.9043419545123\\
-0.822265625	53.4694523642718\\
-0.82177734375	50.9666026738734\\
-0.8212890625	20.3521540567742\\
-0.82080078125	49.5003111071294\\
-0.8203125	59.0516157955461\\
-0.81982421875	57.6561974916389\\
-0.8193359375	42.2318056771906\\
-0.81884765625	44.5553761738423\\
-0.818359375	58.362937480568\\
-0.81787109375	59.0616476307613\\
-0.8173828125	48.6078563620422\\
-0.81689453125	43.0995657293421\\
-0.81640625	57.7402036207182\\
-0.81591796875	59.8915041088045\\
-0.8154296875	51.857763826913\\
-0.81494140625	35.3704450476135\\
-0.814453125	53.8203648650737\\
-0.81396484375	57.3205817992888\\
-0.8134765625	48.9877382411828\\
-0.81298828125	33.3702792023252\\
-0.8125	55.0068435394703\\
-0.81201171875	59.1688165794477\\
-0.8115234375	52.9762972284431\\
-0.81103515625	23.9483034752848\\
-0.810546875	50.1559859882716\\
-0.81005859375	56.105178580317\\
-0.8095703125	50.125735197407\\
-0.80908203125	41.4764204563539\\
-0.80859375	56.7280442851695\\
-0.80810546875	61.3705562689678\\
-0.8076171875	56.9976976446544\\
-0.80712890625	34.1378174238037\\
-0.806640625	46.7618573056765\\
-0.80615234375	56.6372130153308\\
-0.8056640625	53.6404671035249\\
-0.80517578125	38.8642738555632\\
-0.8046875	52.2950167956601\\
-0.80419921875	59.6528553290896\\
-0.8037109375	57.3095329535644\\
-0.80322265625	44.1962973099881\\
-0.802734375	50.2574120491083\\
-0.80224609375	59.0701587895892\\
-0.8017578125	57.6476174716065\\
-0.80126953125	42.7870676483802\\
-0.80078125	44.1091836409895\\
-0.80029296875	57.2215983515111\\
-0.7998046875	57.1567915834124\\
-0.79931640625	44.9475088813185\\
-0.798828125	46.1088572837873\\
-0.79833984375	58.2864612236987\\
-0.7978515625	58.9472673536995\\
-0.79736328125	48.0159983812299\\
-0.796875	36.2100820753291\\
-0.79638671875	55.782504386364\\
-0.7958984375	58.394162953389\\
-0.79541015625	50.080109392956\\
-0.794921875	32.7327874313951\\
-0.79443359375	52.714268175345\\
-0.7939453125	55.8565830216521\\
-0.79345703125	46.8430642630583\\
-0.79296875	39.0769357567977\\
-0.79248046875	55.5876590161722\\
-0.7919921875	58.383247854756\\
-0.79150390625	50.1805660089406\\
-0.791015625	27.9827644309688\\
-0.79052734375	54.0380337524034\\
-0.7900390625	58.6014755958152\\
-0.78955078125	52.2320701725928\\
-0.7890625	25.2200262419031\\
-0.78857421875	52.1224517587992\\
-0.7880859375	58.0026282491481\\
-0.78759765625	52.8798079426582\\
-0.787109375	38.5562893635953\\
-0.78662109375	54.3690985044483\\
-0.7861328125	60.1316607766106\\
-0.78564453125	56.2148120166207\\
-0.78515625	37.2056183224373\\
-0.78466796875	49.5613702141368\\
-0.7841796875	58.2320663783135\\
-0.78369140625	55.5603813494673\\
-0.783203125	36.5023690357796\\
-0.78271484375	47.012852039796\\
-0.7822265625	57.2087861677124\\
-0.78173828125	55.1326628383632\\
-0.78125	36.6037936269305\\
-0.78076171875	47.2856338424429\\
-0.7802734375	58.0331234829075\\
-0.77978515625	56.9018492407611\\
-0.779296875	40.9814146856853\\
-0.77880859375	40.8062066452456\\
-0.7783203125	55.8067776973477\\
-0.77783203125	55.5224847990584\\
-0.77734375	39.5636261844844\\
-0.77685546875	43.2772950376118\\
-0.7763671875	57.1728807698891\\
-0.77587890625	57.3522753490763\\
-0.775390625	43.5612334807971\\
-0.77490234375	39.9378350769848\\
-0.7744140625	57.3302805758982\\
-0.77392578125	59.1427137030912\\
-0.7734375	49.9962124676037\\
-0.77294921875	34.8392767392346\\
-0.7724609375	54.1536924735554\\
-0.77197265625	56.7664961041277\\
-0.771484375	46.4587549627357\\
-0.77099609375	33.260534914269\\
-0.7705078125	55.2070899521835\\
-0.77001953125	57.9663677854165\\
-0.76953125	48.7092390723386\\
-0.76904296875	38.3770619835405\\
-0.7685546875	57.1831916420622\\
-0.76806640625	60.5292768011542\\
-0.767578125	52.9981379602383\\
-0.76708984375	27.7333790059351\\
-0.7666015625	56.3479284107656\\
-0.76611328125	61.9792302935097\\
-0.765625	57.4389019900554\\
-0.76513671875	30.0239703017812\\
-0.7646484375	50.2989725317252\\
-0.76416015625	59.6487142180271\\
-0.763671875	57.1747214889305\\
-0.76318359375	40.206238129791\\
-0.7626953125	47.6592619018297\\
-0.76220703125	57.5352847406702\\
-0.76171875	55.2469654754021\\
-0.76123046875	39.194269933017\\
-0.7607421875	50.1849775766346\\
-0.76025390625	58.9455561996753\\
-0.759765625	56.8936987777236\\
-0.75927734375	41.935487805636\\
-0.7587890625	49.3776605731727\\
-0.75830078125	58.9694067598584\\
-0.7578125	57.1364838621161\\
-0.75732421875	37.2351505131876\\
-0.7568359375	47.7109359606287\\
-0.75634765625	60.3454853093869\\
-0.755859375	60.4378737289842\\
-0.75537109375	48.1767417550554\\
-0.7548828125	35.8942137506633\\
-0.75439453125	56.6176381934212\\
-0.75390625	58.1271821442059\\
-0.75341796875	46.2089461046711\\
-0.7529296875	43.8612776370392\\
-0.75244140625	59.4634505789028\\
-0.751953125	61.4984934095417\\
-0.75146484375	52.892579604711\\
-0.7509765625	33.6824532009534\\
-0.75048828125	56.3488993371626\\
-0.75	60.3004381250721\\
-0.74951171875	52.8948953568873\\
-0.7490234375	10.7367394535698\\
-0.74853515625	54.2786975931851\\
-0.748046875	59.3819018472302\\
-0.74755859375	52.4346453197918\\
-0.7470703125	21.4086267525882\\
-0.74658203125	55.7443330359176\\
-0.74609375	61.5881322067954\\
-0.74560546875	57.2199125877655\\
-0.7451171875	40.1607110291498\\
-0.74462890625	53.5614283303735\\
-0.744140625	60.5998603654208\\
-0.74365234375	56.8734769341939\\
-0.7431640625	30.6066144201004\\
-0.74267578125	50.0038152108121\\
-0.7421875	59.770571343287\\
-0.74169921875	57.3636883784714\\
-0.7412109375	37.6392620881167\\
-0.74072265625	48.912728705962\\
-0.740234375	59.8269557730086\\
-0.73974609375	58.7877999822202\\
-0.7392578125	46.1438745568327\\
-0.73876953125	48.8544671705806\\
-0.73828125	59.2601462873061\\
-0.73779296875	58.4985340217933\\
-0.7373046875	44.7448483805801\\
-0.73681640625	47.3679188452454\\
-0.736328125	59.7588101275876\\
-0.73583984375	60.0857172946022\\
-0.7353515625	48.0249752037821\\
-0.73486328125	37.750651929991\\
-0.734375	57.2714697883483\\
-0.73388671875	59.1119807498592\\
-0.7333984375	48.4802611524919\\
-0.73291015625	36.9705406827446\\
-0.732421875	56.8768625351875\\
-0.73193359375	59.2543839562631\\
-0.7314453125	49.5164688417723\\
-0.73095703125	38.8538318381961\\
-0.73046875	58.1383894123621\\
-0.72998046875	61.5121696956571\\
-0.7294921875	54.6443584906452\\
-0.72900390625	30.665245500945\\
-0.728515625	53.8004411632546\\
-0.72802734375	58.7952054615155\\
-0.7275390625	51.3714334257063\\
-0.72705078125	25.5869815296914\\
-0.7265625	56.7499623012081\\
-0.72607421875	62.4538840103362\\
-0.7255859375	58.6553222825536\\
-0.72509765625	43.5414225244646\\
-0.724609375	51.1079638428769\\
-0.72412109375	58.2877827916292\\
-0.7236328125	53.8510097091248\\
-0.72314453125	30.1271174627159\\
-0.72265625	53.4424214011822\\
-0.72216796875	60.7855363686942\\
-0.7216796875	57.4583642787067\\
-0.72119140625	38.9325610726218\\
-0.720703125	52.4383803354683\\
-0.72021484375	61.5552971502551\\
-0.7197265625	60.1034901697095\\
-0.71923828125	46.8018565832892\\
-0.71875	47.9664896600261\\
-0.71826171875	59.4020026371746\\
-0.7177734375	58.8836732823901\\
-0.71728515625	43.7908857064233\\
-0.716796875	42.9791666523841\\
-0.71630859375	59.0347979613922\\
-0.7158203125	60.3546436272188\\
-0.71533203125	50.5724191427189\\
-0.71484375	27.4277533571567\\
-0.71435546875	52.8388107581331\\
-0.7138671875	54.683685875148\\
-0.71337890625	41.257454985805\\
-0.712890625	46.4233744724327\\
-0.71240234375	59.6079718054186\\
-0.7119140625	60.9580188903601\\
-0.71142578125	51.9886343727816\\
-0.7109375	38.7388338986276\\
-0.71044921875	56.7640907650083\\
-0.7099609375	60.1258609819319\\
-0.70947265625	52.8181553687749\\
-0.708984375	26.9226333611761\\
-0.70849609375	52.8306311730227\\
-0.7080078125	57.3062041540666\\
-0.70751953125	49.0121902823783\\
-0.70703125	32.9163333216362\\
-0.70654296875	56.3997151040349\\
-0.7060546875	60.9151719249576\\
-0.70556640625	55.5903897128827\\
-0.705078125	38.0723315032035\\
-0.70458984375	52.9849776528344\\
-0.7041015625	58.9540560296208\\
-0.70361328125	53.9511478832296\\
-0.703125	24.7122359493008\\
-0.70263671875	50.4496535959716\\
-0.7021484375	57.4731139663445\\
-0.70166015625	51.9497742724675\\
-0.701171875	23.6974729209269\\
-0.70068359375	54.3768734573656\\
-0.7001953125	61.1079032165951\\
-0.69970703125	57.7579865594123\\
-0.69921875	41.5733439364071\\
-0.69873046875	50.7620245530001\\
-0.6982421875	59.0066814259203\\
-0.69775390625	55.792523288238\\
-0.697265625	34.7746603196431\\
-0.69677734375	51.4110568679043\\
-0.6962890625	60.0334412862863\\
-0.69580078125	56.823034281023\\
-0.6953125	33.8816155375995\\
-0.69482421875	55.0834163841451\\
-0.6943359375	64.7645285876854\\
-0.69384765625	64.6516634734891\\
-0.693359375	55.2933306076539\\
-0.69287109375	44.545874317914\\
-0.6923828125	57.8650204574791\\
-0.69189453125	59.1676314926302\\
-0.69140625	47.1880563768792\\
-0.69091796875	44.3687846086232\\
-0.6904296875	60.1675321266771\\
-0.68994140625	62.197685154857\\
-0.689453125	53.9313354011432\\
-0.68896484375	40.271883095396\\
-0.6884765625	57.2425505346065\\
-0.68798828125	60.3243364910978\\
-0.6875	51.5362889617035\\
-0.68701171875	32.2776194846682\\
-0.6865234375	57.2832668329585\\
-0.68603515625	61.17415246249\\
-0.685546875	53.5043445949778\\
-0.68505859375	23.3482970342388\\
-0.6845703125	57.2339113222582\\
-0.68408203125	62.7439053638468\\
-0.68359375	58.2533887971763\\
-0.68310546875	43.4779313205342\\
-0.6826171875	55.1482963305181\\
-0.68212890625	61.2360311058381\\
-0.681640625	56.7021738961657\\
-0.68115234375	31.4530776425919\\
-0.6806640625	53.4592757105876\\
-0.68017578125	61.4088545600714\\
-0.6796875	58.1996548038442\\
-0.67919921875	38.204263793814\\
-0.6787109375	51.4663245018901\\
-0.67822265625	60.8751419630401\\
-0.677734375	58.6484009621248\\
-0.67724609375	40.2861914930889\\
-0.6767578125	48.7060386033036\\
-0.67626953125	59.6629890581053\\
-0.67578125	57.5676286334583\\
-0.67529296875	34.8324933433132\\
-0.6748046875	51.6014734781842\\
-0.67431640625	62.796586604748\\
-0.673828125	62.4726779248382\\
-0.67333984375	49.8675979343441\\
-0.6728515625	38.5445822393303\\
-0.67236328125	58.6758845907988\\
-0.671875	60.2797456785072\\
-0.67138671875	49.5002451650854\\
-0.6708984375	44.8741708987693\\
-0.67041015625	60.323470098385\\
-0.669921875	62.6880367074731\\
-0.66943359375	55.1887221958616\\
-0.6689453125	30.7782745356677\\
-0.66845703125	52.378940214865\\
-0.66796875	55.8198761265222\\
-0.66748046875	44.2020631018587\\
-0.6669921875	45.3981614833407\\
-0.66650390625	60.3730409930516\\
-0.666015625	62.571360151482\\
-0.66552734375	54.391970098815\\
-0.6650390625	21.4033949074837\\
-0.66455078125	55.4818089564867\\
-0.6640625	59.8573946741524\\
-0.66357421875	51.3685370928975\\
-0.6630859375	28.8343252838171\\
-0.66259765625	58.0017857837487\\
-0.662109375	62.35183662612\\
-0.66162109375	55.2604404399723\\
-0.6611328125	16.3391745101625\\
-0.66064453125	58.5206341082035\\
-0.66015625	64.8192913587835\\
-0.65966796875	61.1295883507063\\
-0.6591796875	39.9172990981694\\
-0.65869140625	49.7907695560241\\
-0.658203125	60.3549074563346\\
-0.65771484375	57.5960823288954\\
-0.6572265625	41.1602731432741\\
-0.65673828125	55.1333534356503\\
-0.65625	62.9151422403024\\
-0.65576171875	60.0677278445188\\
-0.6552734375	39.7018268005208\\
-0.65478515625	52.6682105954057\\
-0.654296875	63.1951910573838\\
-0.65380859375	61.8591159275693\\
-0.6533203125	45.8285399615937\\
-0.65283203125	50.1337847446543\\
-0.65234375	63.1901083487547\\
-0.65185546875	63.4775642663961\\
-0.6513671875	51.898824360401\\
-0.65087890625	46.5055596494886\\
-0.650390625	61.8762993472308\\
-0.64990234375	63.3392961940215\\
-0.6494140625	52.7472661791762\\
-0.64892578125	42.7053767564614\\
-0.6484375	61.8087344849443\\
-0.64794921875	64.7433073969266\\
-0.6474609375	56.8376585259942\\
-0.64697265625	27.331269724844\\
-0.646484375	59.2306210659534\\
-0.64599609375	64.6408145977485\\
-0.6455078125	59.8110064306748\\
-0.64501953125	33.871890487866\\
-0.64453125	50.8845007575703\\
-0.64404296875	58.9365113471193\\
-0.6435546875	52.9677124784981\\
-0.64306640625	20.9294992267572\\
-0.642578125	56.683199313912\\
-0.64208984375	62.6025425420634\\
-0.6416015625	57.7413504642138\\
-0.64111328125	30.6465908834753\\
-0.640625	54.5511683126972\\
-0.64013671875	62.3446907632255\\
-0.6396484375	58.8592472242145\\
-0.63916015625	34.6870655200015\\
-0.638671875	51.4365343183221\\
-0.63818359375	61.2950205703881\\
-0.6376953125	58.9654408079857\\
-0.63720703125	39.7805817778748\\
-0.63671875	48.2828452173109\\
-0.63623046875	58.9459200567637\\
-0.6357421875	55.5485640258822\\
-0.63525390625	19.3396824724098\\
-0.634765625	55.5268051963729\\
-0.63427734375	64.3156930450863\\
-0.6337890625	62.4705463470025\\
-0.63330078125	45.6188489979401\\
-0.6328125	49.2694396775271\\
-0.63232421875	63.0767549481605\\
-0.6318359375	63.3924984551223\\
-0.63134765625	50.8182810820403\\
-0.630859375	41.6204656194036\\
-0.63037109375	61.4410282492798\\
-0.6298828125	63.7685838176021\\
-0.62939453125	55.0470915165227\\
-0.62890625	19.5794155387061\\
-0.62841796875	56.124644913528\\
-0.6279296875	59.5674079965032\\
-0.62744140625	49.5316517753369\\
-0.626953125	46.5533940633047\\
-0.62646484375	62.4252452661939\\
-0.6259765625	65.6983664700018\\
-0.62548828125	60.6436423235677\\
-0.625	50.1550493789801\\
-0.62451171875	58.1117671964251\\
-0.6240234375	61.9254965750785\\
-0.62353515625	55.2694357147336\\
-0.623046875	27.7462383045521\\
-0.62255859375	56.5407092239835\\
-0.6220703125	62.5172399133797\\
-0.62158203125	58.7502114932371\\
-0.62109375	51.0277341332708\\
-0.62060546875	59.5320495424071\\
-0.6201171875	63.7640084848388\\
-0.61962890625	59.3556322568577\\
-0.619140625	44.2688616192309\\
-0.61865234375	54.3959283987773\\
-0.6181640625	60.3647461490908\\
-0.61767578125	54.3921301009911\\
-0.6171875	39.6721312194072\\
-0.61669921875	60.5756818975759\\
-0.6162109375	66.4040205440189\\
-0.61572265625	62.8230214923436\\
-0.615234375	44.8586463949584\\
-0.61474609375	56.8736352322506\\
-0.6142578125	66.6615956939505\\
-0.61376953125	66.2624249969549\\
-0.61328125	55.3770323576716\\
-0.61279296875	36.5520485375973\\
-0.6123046875	57.8131100015384\\
-0.61181640625	58.8563636919537\\
-0.611328125	43.2749308645194\\
-0.61083984375	47.1495013910313\\
-0.6103515625	61.2268083913049\\
-0.60986328125	61.2437067665966\\
-0.609375	47.018727701237\\
-0.60888671875	51.0889544165529\\
-0.6083984375	64.934502377097\\
-0.60791015625	67.0018040661703\\
-0.607421875	60.6829434040908\\
-0.60693359375	49.6364791119938\\
-0.6064453125	59.2980845331357\\
-0.60595703125	62.4108670940338\\
-0.60546875	54.5114569998323\\
-0.60498046875	31.3698094634859\\
-0.6044921875	57.6427883400231\\
-0.60400390625	62.0025699470597\\
-0.603515625	54.8894455516197\\
-0.60302734375	32.9385473928017\\
-0.6025390625	57.4488687932636\\
-0.60205078125	62.3540228585833\\
-0.6015625	56.4554192327508\\
-0.60107421875	38.7281933579644\\
-0.6005859375	56.5490456596666\\
-0.60009765625	61.43185518892\\
-0.599609375	54.5250258882588\\
-0.59912109375	36.9824876599372\\
-0.5986328125	59.8191548232684\\
-0.59814453125	65.1274318906309\\
-0.59765625	60.3796566471801\\
-0.59716796875	30.4327242094931\\
-0.5966796875	54.4176585692576\\
-0.59619140625	63.2511238291725\\
-0.595703125	59.9858994033447\\
-0.59521484375	33.9014859052449\\
-0.5947265625	53.6692605076679\\
-0.59423828125	63.9002882883667\\
-0.59375	62.5479543401191\\
-0.59326171875	48.2232124292652\\
-0.5927734375	45.2443512707616\\
-0.59228515625	58.6917902444745\\
-0.591796875	56.7011058916792\\
-0.59130859375	32.3850914699835\\
-0.5908203125	54.0079705333919\\
-0.59033203125	63.3679966271387\\
-0.58984375	61.236394006559\\
-0.58935546875	42.687793387442\\
-0.5888671875	53.8006811331166\\
-0.58837890625	64.7468757405972\\
-0.587890625	63.709922916348\\
-0.58740234375	46.6783018231683\\
-0.5869140625	51.6185745565877\\
-0.58642578125	66.2926119989901\\
-0.5859375	67.6708089611406\\
-0.58544921875	58.3283165874712\\
-0.5849609375	28.562374511913\\
-0.58447265625	61.212590157339\\
-0.583984375	65.2090024734997\\
-0.58349609375	57.7008322273604\\
-0.5830078125	43.823821037422\\
-0.58251953125	62.6757778865606\\
-0.58203125	67.1209572336196\\
-0.58154296875	61.6734953879553\\
-0.5810546875	32.7064380718892\\
-0.58056640625	55.9561915865801\\
-0.580078125	63.4259276516424\\
-0.57958984375	58.3494466852723\\
-0.5791015625	26.7068356113826\\
-0.57861328125	58.6922708451621\\
-0.578125	66.0950757980715\\
-0.57763671875	63.4841752535765\\
-0.5771484375	50.660319433232\\
-0.57666015625	55.7163867971369\\
-0.576171875	63.7837716668002\\
-0.57568359375	61.6048718468009\\
-0.5751953125	44.332494004276\\
-0.57470703125	48.1367464869902\\
-0.57421875	60.0709231083413\\
-0.57373046875	57.8418172667553\\
-0.5732421875	40.0563269429882\\
-0.57275390625	55.0360676912708\\
-0.572265625	63.6370089583025\\
-0.57177734375	61.688281834492\\
-0.5712890625	46.4775280641353\\
-0.57080078125	49.375737449852\\
-0.5703125	59.8522388886407\\
-0.56982421875	55.9094000366057\\
-0.5693359375	24.3778249895187\\
-0.56884765625	61.3224198339702\\
-0.568359375	69.9014992565478\\
-0.56787109375	69.756547368521\\
-0.5673828125	61.0981383214291\\
-0.56689453125	44.074058295476\\
-0.56640625	58.1818862762086\\
-0.56591796875	59.7234933864128\\
-0.5654296875	44.0302520814831\\
-0.56494140625	49.5190345398594\\
-0.564453125	63.6917144486508\\
-0.56396484375	64.1738535802963\\
-0.5634765625	51.6009603167544\\
-0.56298828125	52.4989013619755\\
-0.5625	67.0111530757895\\
-0.56201171875	69.3197432087217\\
-0.5615234375	62.2074947533853\\
-0.56103515625	41.2121300565869\\
-0.560546875	61.5585077254802\\
-0.56005859375	67.1924141225368\\
-0.5595703125	62.9094932915617\\
-0.55908203125	48.3073183876957\\
-0.55859375	57.7113145750423\\
-0.55810546875	63.4518277570469\\
-0.5576171875	57.6505701059628\\
-0.55712890625	31.2253200668251\\
-0.556640625	59.0146905738988\\
-0.55615234375	65.0273008310414\\
-0.5556640625	59.8652221637674\\
-0.55517578125	19.8163903690393\\
-0.5546875	57.0716074336834\\
-0.55419921875	64.5075857067746\\
-0.5537109375	60.2011400277824\\
-0.55322265625	44.3095997690148\\
-0.552734375	60.4745997015324\\
-0.55224609375	66.7190234625856\\
-0.5517578125	62.2328663949523\\
-0.55126953125	27.5730441623433\\
-0.55078125	60.0229297056142\\
-0.55029296875	68.9565181643639\\
-0.5498046875	67.2434986826054\\
-0.54931640625	52.7233037611212\\
-0.548828125	54.952402655397\\
-0.54833984375	67.4072116581676\\
-0.5478515625	67.9280723997424\\
-0.54736328125	58.1309127999016\\
-0.546875	48.163887032946\\
-0.54638671875	61.9472343851763\\
-0.5458984375	61.9648578218822\\
-0.54541015625	42.8575243892536\\
-0.544921875	54.9101636467512\\
-0.54443359375	67.7754162527597\\
-0.5439453125	68.3620799625709\\
-0.54345703125	56.9355924586978\\
-0.54296875	43.2160315216121\\
-0.54248046875	65.6900244194799\\
-0.5419921875	68.6978761824497\\
-0.54150390625	60.9575242709025\\
-0.541015625	46.5252222652577\\
-0.54052734375	64.6879652804797\\
-0.5400390625	68.6409616286813\\
-0.53955078125	61.7193939765603\\
-0.5390625	34.7987173235896\\
-0.53857421875	63.6093878442633\\
-0.5380859375	69.6146515383152\\
-0.53759765625	65.4292201798425\\
-0.537109375	40.8836263561729\\
-0.53662109375	55.9415115696219\\
-0.5361328125	65.8348760547666\\
-0.53564453125	63.4017014377836\\
-0.53515625	50.8403088414384\\
-0.53466796875	58.3838143034027\\
-0.5341796875	64.7162472291581\\
-0.53369140625	59.6459950585669\\
-0.533203125	21.8270475396299\\
-0.53271484375	60.3009781309252\\
-0.5322265625	67.3451674544641\\
-0.53173828125	63.0015457022619\\
-0.53125	38.8966487705345\\
-0.53076171875	62.4753611213943\\
-0.5302734375	70.7385123951344\\
-0.52978515625	69.006246996675\\
-0.529296875	54.8387616360816\\
-0.52880859375	53.3727332224033\\
-0.5283203125	66.6291265205139\\
-0.52783203125	66.0679172057407\\
-0.52734375	49.1574103485303\\
-0.52685546875	53.4588766372143\\
-0.5263671875	67.5631900693025\\
-0.52587890625	67.5128936229567\\
-0.525390625	53.4819864871439\\
-0.52490234375	57.1831796836009\\
-0.5244140625	70.9149751114617\\
-0.52392578125	72.8051680218795\\
-0.5234375	65.3400267752893\\
-0.52294921875	36.867268224973\\
-0.5224609375	61.8821795478835\\
-0.52197265625	66.7136572467618\\
-0.521484375	58.1300566455392\\
-0.52099609375	41.719309441095\\
-0.5205078125	66.0730137476332\\
-0.52001953125	70.3998591220777\\
-0.51953125	64.5516086385168\\
-0.51904296875	42.7112220030353\\
-0.5185546875	62.6456305594253\\
-0.51806640625	68.7122054778251\\
-0.517578125	63.2651182706057\\
-0.51708984375	25.9490402858039\\
-0.5166015625	63.0599585895351\\
-0.51611328125	70.4062089652472\\
-0.515625	67.0694263189625\\
-0.51513671875	44.3614465221923\\
-0.5146484375	58.0725733012019\\
-0.51416015625	69.4266186964405\\
-0.513671875	69.3957289730717\\
-0.51318359375	62.1439858420195\\
-0.5126953125	60.7174045119645\\
-0.51220703125	66.1496062722545\\
-0.51171875	63.9739029874063\\
-0.51123046875	54.9937033211338\\
-0.5107421875	64.0021171217594\\
-0.51025390625	70.6571629775721\\
-0.509765625	68.8111748369304\\
-0.50927734375	55.0346922025466\\
-0.5087890625	54.5933254299915\\
-0.50830078125	67.5104968645288\\
-0.5078125	67.9453538706604\\
-0.50732421875	58.8919256653672\\
-0.5068359375	59.2989211983551\\
-0.50634765625	68.5491498281047\\
-0.505859375	68.430936666248\\
-0.50537109375	55.5678577736564\\
-0.5048828125	47.814601391375\\
-0.50439453125	66.7340402112138\\
-0.50390625	68.5850615939817\\
-0.50341796875	58.3877233615594\\
-0.5029296875	44.1333224495527\\
-0.50244140625	65.4696083933043\\
-0.501953125	68.0793206763596\\
-0.50146484375	58.1183427866855\\
-0.5009765625	45.4829807732141\\
-0.50048828125	67.1630380199249\\
-0.5	70.914867067158\\
-0.49951171875	65.0690707398374\\
-0.4990234375	52.6512276065712\\
-0.49853515625	65.0738472510248\\
-0.498046875	69.2274603544097\\
-0.49755859375	63.0891845195157\\
-0.4970703125	55.0931871107345\\
-0.49658203125	69.3034959781322\\
-0.49609375	73.9914469927938\\
-0.49560546875	70.1285527841811\\
-0.4951171875	51.7502347371823\\
-0.49462890625	58.3290406864264\\
-0.494140625	68.9469647479016\\
-0.49365234375	67.3693944637045\\
-0.4931640625	55.9533353037616\\
-0.49267578125	61.9076639048623\\
-0.4921875	69.5147515386655\\
-0.49169921875	67.1161261586687\\
-0.4912109375	46.7621700704472\\
-0.49072265625	57.291707389026\\
-0.490234375	69.9137349109583\\
-0.48974609375	70.5624734520814\\
-0.4892578125	62.3657173365545\\
-0.48876953125	51.456453076803\\
-0.48828125	60.3649321417209\\
-0.48779296875	59.5026883533075\\
-0.4873046875	54.0321825738577\\
-0.48681640625	66.6626599264256\\
-0.486328125	73.1365855850372\\
-0.48583984375	71.8690846951131\\
-0.4853515625	60.2446478536551\\
-0.48486328125	36.9342212097629\\
-0.484375	62.4856276459706\\
-0.48388671875	62.9202276546752\\
-0.4833984375	53.5887787448478\\
-0.48291015625	66.5636547339365\\
-0.482421875	74.7659281018014\\
-0.48193359375	74.7723415880662\\
-0.4814453125	66.4486237303438\\
-0.48095703125	61.1765586058212\\
-0.48046875	72.2990819033708\\
-0.47998046875	74.7265103026617\\
-0.4794921875	68.0563053830253\\
-0.47900390625	51.3190565218491\\
-0.478515625	67.465400156105\\
-0.47802734375	72.748773352467\\
-0.4775390625	68.4388319557861\\
-0.47705078125	57.9356319639416\\
-0.4765625	67.8451222800485\\
-0.47607421875	73.0369460392358\\
-0.4755859375	69.1933928486631\\
-0.47509765625	48.8481009466265\\
-0.474609375	57.0812982266367\\
-0.47412109375	67.2610231490959\\
-0.4736328125	63.6390503813052\\
-0.47314453125	38.0154955999452\\
-0.47265625	60.2007963585195\\
-0.47216796875	67.3181145152257\\
-0.4716796875	61.4030630826871\\
-0.47119140625	51.0580495719614\\
-0.470703125	69.9568308381324\\
-0.47021484375	75.6345107908807\\
-0.4697265625	72.6043674044599\\
-0.46923828125	56.6956941169131\\
-0.46875	61.2007613452708\\
-0.46826171875	72.209970545987\\
-0.4677734375	71.1580499352917\\
-0.46728515625	55.2575950840679\\
-0.466796875	58.4332034015981\\
-0.46630859375	72.4706935295315\\
-0.4658203125	73.5297893088972\\
-0.46533203125	64.3380733754222\\
-0.46484375	48.4572237500578\\
-0.46435546875	66.28272011884\\
-0.4638671875	68.7702530785816\\
-0.46337890625	62.1849329666682\\
-0.462890625	63.8759129620218\\
-0.46240234375	71.4633640204484\\
-0.4619140625	70.4237766501141\\
-0.46142578125	53.5508246859656\\
-0.4609375	59.0869908584688\\
-0.46044921875	73.9103213931934\\
-0.4599609375	75.3990587680606\\
-0.45947265625	66.0209805400055\\
-0.458984375	55.619377384405\\
-0.45849609375	74.310686007356\\
-0.4580078125	78.8337630851796\\
-0.45751953125	75.0017172146814\\
-0.45703125	58.9357091058442\\
-0.45654296875	57.3368368773933\\
-0.4560546875	67.6679823492681\\
-0.45556640625	62.082987570461\\
-0.455078125	30.8314154204655\\
-0.45458984375	68.093594195015\\
-0.4541015625	74.0643682580081\\
-0.45361328125	69.3973058934486\\
-0.453125	36.0418073965452\\
-0.45263671875	65.9664757095299\\
-0.4521484375	74.8289575079775\\
-0.45166015625	72.8104859059157\\
-0.451171875	58.0748062089695\\
-0.45068359375	61.7534136563604\\
-0.4501953125	72.6579639797875\\
-0.44970703125	71.997420359437\\
-0.44921875	63.539034681925\\
-0.44873046875	69.8226719177583\\
-0.4482421875	77.6987049936352\\
-0.44775390625	77.964042893774\\
-0.447265625	69.8469851731064\\
-0.44677734375	36.6504764803799\\
-0.4462890625	60.4620237137398\\
-0.44580078125	63.0879660603071\\
-0.4453125	37.8874822618782\\
-0.44482421875	63.9084403626808\\
-0.4443359375	75.6431499503387\\
-0.44384765625	77.334106790559\\
-0.443359375	71.2976882320073\\
-0.44287109375	57.431143148472\\
-0.4423828125	63.8417168277952\\
-0.44189453125	66.7182776029252\\
-0.44140625	58.0008425912404\\
-0.44091796875	63.7923152953289\\
-0.4404296875	74.6083149565758\\
-0.43994140625	76.3916532478757\\
-0.439453125	69.3361395101584\\
-0.43896484375	39.7240108169146\\
-0.4384765625	64.9852777656252\\
-0.43798828125	70.892724140662\\
-0.4375	64.9963333612786\\
-0.43701171875	54.2558971709484\\
-0.4365234375	68.7960244643648\\
-0.43603515625	72.6937441898049\\
-0.435546875	66.3636208419947\\
-0.43505859375	51.6898699308458\\
-0.4345703125	67.865071742327\\
-0.43408203125	71.9066950840622\\
-0.43359375	65.3868891832791\\
-0.43310546875	65.6513948333814\\
-0.4326171875	77.9322230040266\\
-0.43212890625	81.880200377222\\
-0.431640625	78.4137971150275\\
-0.43115234375	62.9817283536827\\
-0.4306640625	59.9528083512296\\
-0.43017578125	73.1527889417052\\
-0.4296875	71.5228472692965\\
-0.42919921875	50.9288778684288\\
-0.4287109375	66.3876805905788\\
-0.42822265625	77.176427881305\\
-0.427734375	76.8659493171693\\
-0.42724609375	64.7963270270515\\
-0.4267578125	57.9701885219571\\
-0.42626953125	75.1570413011029\\
-0.42578125	77.9904961282132\\
-0.42529296875	72.1951985067541\\
-0.4248046875	60.2278943326019\\
-0.42431640625	68.6039001759203\\
-0.423828125	71.8208082169051\\
-0.42333984375	63.1482995444658\\
-0.4228515625	39.5135546576944\\
-0.42236328125	69.5462367342383\\
-0.421875	74.8914530308109\\
-0.42138671875	71.7099929816144\\
-0.4208984375	58.9373936474139\\
-0.42041015625	32.0815804397294\\
-0.419921875	38.7069514219859\\
-0.41943359375	47.2993770063357\\
-0.4189453125	67.7794764844909\\
-0.41845703125	74.9069247719036\\
-0.41796875	73.9275414805711\\
-0.41748046875	62.0005933348619\\
-0.4169921875	59.0153640809679\\
-0.41650390625	74.7005548880512\\
-0.416015625	78.9221421070039\\
-0.41552734375	78.3198282180301\\
-0.4150390625	77.9111557951187\\
-0.41455078125	79.0683516635332\\
-0.4140625	76.739163345515\\
-0.41357421875	64.2527735325539\\
-0.4130859375	57.0748446960406\\
-0.41259765625	74.7757353126271\\
-0.412109375	77.2318365020611\\
-0.41162109375	69.2361337264978\\
-0.4111328125	61.2321764452526\\
-0.41064453125	75.9811046580192\\
-0.41015625	78.9718780490448\\
-0.40966796875	70.3304315143892\\
-0.4091796875	50.0070210462559\\
-0.40869140625	79.1467373973801\\
-0.408203125	85.4892045273799\\
-0.40771484375	83.6258262817271\\
-0.4072265625	71.8731315388421\\
-0.40673828125	59.7394611682598\\
-0.40625	76.3916759778565\\
-0.40576171875	78.2091181679303\\
-0.4052734375	69.5731980112212\\
-0.40478515625	49.9395693641628\\
-0.404296875	67.1281263918283\\
-0.40380859375	67.1290823002669\\
-0.4033203125	40.1436818772185\\
-0.40283203125	66.4740805357217\\
-0.40234375	76.5625766241949\\
-0.40185546875	75.7163587444123\\
-0.4013671875	63.3776118642351\\
-0.40087890625	60.7747557652295\\
-0.400390625	72.0372055841328\\
-0.39990234375	68.5650703883869\\
-0.3994140625	45.7087054011349\\
-0.39892578125	75.059993857255\\
-0.3984375	82.870087786635\\
-0.39794921875	80.9093944221607\\
-0.3974609375	62.5618984487526\\
-0.39697265625	71.3782628601334\\
-0.396484375	85.4650758044108\\
-0.39599609375	87.5975299214831\\
-0.3955078125	80.6756743435157\\
-0.39501953125	37.6079027272651\\
-0.39453125	77.3440684882041\\
-0.39404296875	85.0369555305084\\
-0.3935546875	82.4735212967767\\
-0.39306640625	66.2509983723866\\
-0.392578125	63.2979697795697\\
-0.39208984375	78.143610178095\\
-0.3916015625	78.0370732908438\\
-0.39111328125	68.2767053533896\\
-0.390625	66.590743543712\\
-0.39013671875	73.7592849244836\\
-0.3896484375	72.8766040438401\\
-0.38916015625	77.0804145648831\\
-0.388671875	86.3583725093187\\
-0.38818359375	90.6535271160912\\
-0.3876953125	89.597184237079\\
-0.38720703125	82.6264791368355\\
-0.38671875	69.5529023773128\\
-0.38623046875	73.423751829981\\
-0.3857421875	80.8159032340607\\
-0.38525390625	85.5203729587791\\
-0.384765625	88.9878638892085\\
-0.38427734375	89.3840406083098\\
-0.3837890625	84.6070545359288\\
-0.38330078125	73.9903544367084\\
-0.3828125	78.4690317469602\\
-0.38232421875	84.0865502844635\\
-0.3818359375	80.1419754057861\\
-0.38134765625	66.7005570739045\\
-0.380859375	83.6689763855385\\
-0.38037109375	91.8934775699516\\
-0.3798828125	92.3918852452678\\
-0.37939453125	86.7249635233442\\
-0.37890625	82.2483919437309\\
-0.37841796875	87.9941027334088\\
-0.3779296875	89.2439022423443\\
-0.37744140625	82.0153677432322\\
-0.376953125	76.4140360369576\\
-0.37646484375	89.7225227250348\\
-0.3759765625	94.7778184558175\\
-0.37548828125	93.1743961591435\\
-0.375	84.3269261643277\\
-0.37451171875	57.6978953271855\\
-0.3740234375	62.9363032255756\\
-0.37353515625	65.5507881565805\\
-0.373046875	69.6627429095972\\
-0.37255859375	82.1361246064511\\
-0.3720703125	85.774862889613\\
-0.37158203125	80.3248856005095\\
-0.37109375	53.7255405644001\\
-0.37060546875	77.8173721591176\\
-0.3701171875	85.2288839560229\\
-0.36962890625	80.9770188436795\\
-0.369140625	63.2943759868229\\
-0.36865234375	83.6694752562662\\
-0.3681640625	91.7027733564865\\
-0.36767578125	91.0494173277348\\
-0.3671875	80.9577189710606\\
-0.36669921875	65.4333122458508\\
-0.3662109375	82.6848028305417\\
-0.36572265625	84.1157206009966\\
-0.365234375	71.1424556669159\\
-0.36474609375	77.2261817001538\\
-0.3642578125	90.5996471683339\\
-0.36376953125	93.3560296887624\\
-0.36328125	88.4816280278384\\
-0.36279296875	77.1757047035764\\
-0.3623046875	85.3530080364105\\
-0.36181640625	92.0240427965977\\
-0.361328125	91.9098933378966\\
-0.36083984375	87.3488416401946\\
-0.3603515625	83.1212324495762\\
-0.35986328125	80.7894870366368\\
-0.359375	70.5213499877986\\
-0.35888671875	68.7651302275383\\
-0.3583984375	85.6747687014382\\
-0.35791015625	90.304218081375\\
-0.357421875	86.7686066589739\\
-0.35693359375	68.2224450925351\\
-0.3564453125	73.6822886363732\\
-0.35595703125	87.6157267751558\\
-0.35546875	90.1135864533012\\
-0.35498046875	87.0666805309384\\
-0.3544921875	82.1487819436112\\
-0.35400390625	81.3371332929414\\
-0.353515625	85.7239604726104\\
-0.35302734375	93.4768605502132\\
-0.3525390625	100.120060146289\\
-0.35205078125	103.649773501076\\
-0.3515625	103.857358730613\\
-0.35107421875	100.693454451162\\
-0.3505859375	93.8242390065795\\
-0.35009765625	82.3440304922892\\
-0.349609375	66.5674817362102\\
-0.34912109375	66.9585935133334\\
-0.3486328125	69.3634699160843\\
-0.34814453125	57.2280857203198\\
-0.34765625	65.9694383521991\\
-0.34716796875	84.305875883266\\
-0.3466796875	90.5583978324968\\
-0.34619140625	88.8137452708128\\
-0.345703125	71.1562179342151\\
-0.34521484375	81.0500077326724\\
-0.3447265625	96.980755662721\\
-0.34423828125	101.617194503319\\
-0.34375	99.5748253595847\\
-0.34326171875	89.0043523407048\\
-0.3427734375	78.8306206281154\\
-0.34228515625	93.913914977674\\
-0.341796875	97.6653401544845\\
-0.34130859375	94.4582605689948\\
-0.3408203125	90.0399511207787\\
-0.34033203125	95.0194477068301\\
-0.33984375	98.904644991693\\
-0.33935546875	97.6772018775227\\
-0.3388671875	91.2960937279039\\
-0.33837890625	81.3773736116656\\
-0.337890625	78.7034689197511\\
-0.33740234375	84.9458872388593\\
-0.3369140625	93.4525772445824\\
-0.33642578125	99.2881872542043\\
-0.3359375	101.204196116687\\
-0.33544921875	99.9823714124503\\
-0.3349609375	98.7165921484532\\
-0.33447265625	99.9612269848058\\
-0.333984375	100.028891603472\\
-0.33349609375	95.1816609556897\\
-0.3330078125	85.0956572198753\\
-0.33251953125	90.9177700224685\\
-0.33203125	97.102576022808\\
-0.33154296875	95.3093839776248\\
-0.3310546875	84.0074247099529\\
-0.33056640625	88.6702549248488\\
-0.330078125	100.102041353499\\
-0.32958984375	103.879434584948\\
-0.3291015625	103.049494181695\\
-0.32861328125	99.9659952843048\\
-0.328125	98.8675619067356\\
-0.32763671875	101.101653744061\\
-0.3271484375	103.372458348508\\
-0.32666015625	103.844994239085\\
-0.326171875	101.204328403577\\
-0.32568359375	92.9372648914094\\
-0.3251953125	81.4711046198778\\
-0.32470703125	94.3300722906535\\
-0.32421875	100.187206349446\\
-0.32373046875	98.6235194622553\\
-0.3232421875	88.7534404394998\\
-0.32275390625	75.5067226687312\\
-0.322265625	85.4799333048203\\
-0.32177734375	81.7507224234003\\
-0.3212890625	79.3495467328411\\
-0.32080078125	96.9641000751897\\
-0.3203125	103.099143391796\\
-0.31982421875	101.143947437534\\
-0.3193359375	89.4699402261996\\
-0.31884765625	96.7386454315553\\
-0.318359375	108.007212452923\\
-0.31787109375	111.337422827622\\
-0.3173828125	109.209472454176\\
-0.31689453125	102.787005204833\\
-0.31640625	96.5075241153549\\
-0.31591796875	91.7214162843355\\
-0.3154296875	72.7410375725865\\
-0.31494140625	97.4209027344783\\
-0.314453125	109.163787518493\\
-0.31396484375	112.243677977501\\
-0.3134765625	108.049149096364\\
-0.31298828125	91.2930056531243\\
-0.3125	99.6991957862558\\
-0.31201171875	110.516088905325\\
-0.3115234375	111.29816699283\\
-0.31103515625	103.97274826422\\
-0.310546875	97.5306540043609\\
-0.31005859375	107.527684678125\\
-0.3095703125	109.79594456437\\
-0.30908203125	101.756929292174\\
-0.30859375	90.7904433688858\\
-0.30810546875	109.912296283152\\
-0.3076171875	115.500518893878\\
-0.30712890625	112.733293155335\\
-0.306640625	97.2054865894442\\
-0.30615234375	102.153293521191\\
-0.3056640625	115.675490702458\\
-0.30517578125	118.893306832598\\
-0.3046875	116.30522590756\\
-0.30419921875	111.635103589614\\
-0.3037109375	113.771146131697\\
-0.30322265625	117.176737737154\\
-0.302734375	116.627606511945\\
-0.30224609375	114.625804760201\\
-0.3017578125	117.86344348301\\
-0.30126953125	122.51131107475\\
-0.30078125	123.912773668588\\
-0.30029296875	121.260455230547\\
-0.2998046875	113.670186160417\\
-0.29931640625	97.5686297792309\\
-0.298828125	74.5740514607808\\
-0.29833984375	95.3054285116443\\
-0.2978515625	105.0982093064\\
-0.29736328125	112.8747224361\\
-0.296875	117.800961729227\\
-0.29638671875	120.799857542564\\
-0.2958984375	123.395165760277\\
-0.29541015625	125.548242611474\\
-0.294921875	125.600800564161\\
-0.29443359375	121.368712964792\\
-0.2939453125	107.069279075224\\
-0.29345703125	97.4541650428343\\
-0.29296875	117.992333905884\\
-0.29248046875	122.272307880302\\
-0.2919921875	119.458502156223\\
-0.29150390625	113.304137091982\\
-0.291015625	115.867593760004\\
-0.29052734375	118.354131368198\\
-0.2900390625	116.165604631174\\
-0.28955078125	124.752979231882\\
-0.2890625	135.277745887419\\
-0.28857421875	140.531411864229\\
-0.2880859375	140.923787404449\\
-0.28759765625	135.685144625913\\
-0.287109375	119.26890624775\\
-0.28662109375	118.639695110746\\
-0.2861328125	131.779185918107\\
-0.28564453125	132.735739672642\\
-0.28515625	124.810564537983\\
-0.28466796875	116.595966096479\\
-0.2841796875	127.235086023498\\
-0.28369140625	130.272040915486\\
-0.283203125	134.439505771918\\
-0.28271484375	144.768361895555\\
-0.2822265625	152.290215670988\\
-0.28173828125	155.687494302067\\
-0.28125	155.239729422516\\
-0.28076171875	150.883583314564\\
-0.2802734375	143.621489881627\\
-0.27978515625	142.34418816084\\
-0.279296875	147.116603759107\\
-0.27880859375	151.274841615132\\
-0.2783203125	155.992235181148\\
-0.27783203125	160.516784009867\\
-0.27734375	162.893997353358\\
-0.27685546875	161.948706667979\\
-0.2763671875	156.358366113924\\
-0.27587890625	145.238025270522\\
-0.275390625	151.098186525242\\
-0.27490234375	160.409661104179\\
-0.2744140625	163.922242867391\\
-0.27392578125	163.212636444484\\
-0.2734375	158.64829142382\\
-0.27294921875	149.946198900339\\
-0.2724609375	139.616392321535\\
-0.27197265625	137.968568338478\\
-0.271484375	135.911567611763\\
-0.27099609375	124.784422458248\\
-0.2705078125	123.689067938829\\
-0.27001953125	135.203782845396\\
-0.26953125	137.982839426616\\
-0.26904296875	135.627663524288\\
-0.2685546875	132.801553584912\\
-0.26806640625	134.677775142302\\
-0.267578125	136.451134890872\\
-0.26708984375	134.737234421909\\
-0.2666015625	129.425422197328\\
-0.26611328125	122.402891125154\\
-0.265625	116.202390036276\\
-0.26513671875	109.107706397714\\
-0.2646484375	118.671094162311\\
-0.26416015625	129.644642605665\\
-0.263671875	134.996245479094\\
-0.26318359375	136.596862040404\\
-0.2626953125	136.151764858091\\
-0.26220703125	135.670144129938\\
-0.26171875	135.780735713195\\
-0.26123046875	134.737882230589\\
-0.2607421875	130.429486081856\\
-0.26025390625	120.084639240175\\
-0.259765625	102.952481166405\\
-0.25927734375	115.295764426539\\
-0.2587890625	118.681941872209\\
-0.25830078125	111.67192474406\\
-0.2578125	108.650246134247\\
-0.25732421875	124.575016207134\\
-0.2568359375	132.275915101827\\
-0.25634765625	135.142427663948\\
-0.255859375	134.674273957914\\
-0.25537109375	131.579967431602\\
-0.2548828125	126.694985674488\\
-0.25439453125	121.552352737421\\
-0.25390625	117.169443879161\\
-0.25341796875	112.403841275327\\
-0.2529296875	106.766966838949\\
-0.25244140625	102.939471482126\\
-0.251953125	100.801265437434\\
-0.25146484375	95.8338631241275\\
-0.2509765625	82.4736218915921\\
-0.25048828125	89.371087930145\\
-0.25	108.66320420064\\
-0.24951171875	119.324265446783\\
-0.2490234375	125.075434040072\\
-0.24853515625	126.57542580871\\
-0.248046875	123.226012427598\\
-0.24755859375	114.460547487037\\
-0.2470703125	119.579507173435\\
-0.24658203125	130.049814209065\\
-0.24609375	134.874298159793\\
-0.24560546875	135.323674183586\\
-0.2451171875	131.657090850677\\
-0.24462890625	123.997991798586\\
-0.244140625	121.934329223342\\
-0.24365234375	129.191260367515\\
-0.2431640625	133.424504245769\\
-0.24267578125	133.996029231018\\
-0.2421875	131.391458685736\\
-0.24169921875	125.691710874076\\
-0.2412109375	118.583967937324\\
-0.24072265625	120.311062439946\\
-0.240234375	127.931152333599\\
-0.23974609375	133.694149173011\\
-0.2392578125	137.027715842101\\
-0.23876953125	137.909516111366\\
-0.23828125	136.120814978354\\
-0.23779296875	131.358975322802\\
-0.2373046875	123.749862324925\\
-0.23681640625	114.429626673282\\
-0.236328125	94.9493862844397\\
-0.23583984375	114.697277988251\\
-0.2353515625	131.888288023028\\
-0.23486328125	140.319299633709\\
-0.234375	143.93412630864\\
-0.23388671875	143.568401102769\\
-0.2333984375	139.051146551463\\
-0.23291015625	131.060102761353\\
-0.232421875	131.950517220316\\
-0.23193359375	139.371248889953\\
-0.2314453125	143.952602521825\\
-0.23095703125	146.411123312439\\
-0.23046875	147.279834590544\\
-0.22998046875	145.924283966441\\
-0.2294921875	140.74943581195\\
-0.22900390625	128.545554512218\\
-0.228515625	126.085987774122\\
-0.22802734375	137.163034258518\\
-0.2275390625	139.663517652822\\
-0.22705078125	135.770381403749\\
-0.2265625	131.412525099808\\
-0.22607421875	137.456836158981\\
-0.2255859375	140.867576351771\\
-0.22509765625	136.555060362823\\
-0.224609375	115.445210299733\\
-0.22412109375	136.700615237075\\
-0.2236328125	148.248136647809\\
-0.22314453125	151.875270725525\\
-0.22265625	150.278707552142\\
-0.22216796875	143.368107055514\\
-0.2216796875	133.547746428172\\
-0.22119140625	136.999124209961\\
-0.220703125	139.525220763241\\
-0.22021484375	133.852421824027\\
-0.2197265625	125.871062878654\\
-0.21923828125	138.880157592359\\
-0.21875	145.241849129103\\
-0.21826171875	145.016323787033\\
-0.2177734375	137.905711835599\\
-0.21728515625	115.310310842527\\
-0.216796875	126.180618063057\\
-0.21630859375	135.18071296599\\
-0.2158203125	134.156304529695\\
-0.21533203125	128.974372538252\\
-0.21484375	131.121335871808\\
-0.21435546875	134.286821901049\\
-0.2138671875	130.284105246204\\
-0.21337890625	120.580599203933\\
-0.212890625	134.99721869845\\
-0.21240234375	143.917306756779\\
-0.2119140625	146.48932020821\\
-0.21142578125	144.045357805811\\
-0.2109375	136.21793265436\\
-0.21044921875	128.26333011335\\
-0.2099609375	134.759577728675\\
-0.20947265625	138.436517132376\\
-0.208984375	136.635868988689\\
-0.20849609375	129.68663250778\\
-0.2080078125	118.683083028619\\
-0.20751953125	113.312688109344\\
-0.20703125	110.714660967036\\
-0.20654296875	113.238436224572\\
-0.2060546875	123.619083566003\\
-0.20556640625	129.169043180954\\
-0.205078125	130.444936787012\\
-0.20458984375	129.748449699441\\
-0.2041015625	129.873011359486\\
-0.20361328125	130.66827803124\\
-0.203125	129.994846260067\\
-0.20263671875	128.094573620796\\
-0.2021484375	128.09270594616\\
-0.20166015625	129.660225029413\\
-0.201171875	128.73790154234\\
-0.20068359375	122.145699289168\\
-0.2001953125	99.5254155495257\\
-0.19970703125	108.477316083946\\
-0.19921875	119.348557625269\\
-0.19873046875	117.63621004922\\
-0.1982421875	104.68109709841\\
-0.19775390625	115.916618918063\\
-0.197265625	126.966839243274\\
-0.19677734375	130.537054992427\\
-0.1962890625	129.136121178423\\
-0.19580078125	123.270276016507\\
-0.1953125	115.318290128112\\
-0.19482421875	115.72319958095\\
-0.1943359375	118.617420451685\\
-0.19384765625	117.701999553851\\
-0.193359375	113.249885998011\\
-0.19287109375	106.366343802299\\
-0.1923828125	99.045587133845\\
-0.19189453125	96.4657638478649\\
-0.19140625	98.0247206237179\\
-0.19091796875	93.4185801512839\\
-0.1904296875	63.1448102050065\\
-0.18994140625	103.313408739957\\
-0.189453125	114.515434548491\\
-0.18896484375	118.109589596257\\
-0.1884765625	115.628178526074\\
-0.18798828125	103.324208726746\\
-0.1875	91.7773684079639\\
-0.18701171875	113.701293180779\\
-0.1865234375	120.009555049558\\
-0.18603515625	120.450900693064\\
-0.185546875	116.521490301565\\
-0.18505859375	108.510514096506\\
-0.1845703125	98.761051532094\\
-0.18408203125	95.3328606819847\\
-0.18359375	95.2164194874931\\
-0.18310546875	96.7521869155409\\
-0.1826171875	100.096669272568\\
-0.18212890625	101.873836541966\\
-0.181640625	101.382862514472\\
-0.18115234375	99.5488018585284\\
-0.1806640625	96.8878556735035\\
-0.18017578125	93.35718837213\\
-0.1796875	92.8383014361383\\
-0.17919921875	98.7393789694684\\
-0.1787109375	105.16086040165\\
-0.17822265625	109.764182664897\\
-0.177734375	112.154764909945\\
-0.17724609375	111.748012968481\\
-0.1767578125	107.695576705396\\
-0.17626953125	100.848501362414\\
-0.17578125	102.173696117756\\
-0.17529296875	107.085657076546\\
-0.1748046875	106.550559506422\\
-0.17431640625	100.061897912525\\
-0.173828125	102.632192396162\\
-0.17333984375	112.048871316461\\
-0.1728515625	115.698746589558\\
-0.17236328125	113.640256343248\\
-0.171875	103.839932973445\\
-0.17138671875	53.9289310965851\\
-0.1708984375	100.057083340093\\
-0.17041015625	106.99915723873\\
-0.169921875	107.716097941818\\
-0.16943359375	109.195326277577\\
-0.1689453125	112.679936620681\\
-0.16845703125	114.044492669364\\
-0.16796875	111.38757046098\\
-0.16748046875	103.755590118014\\
-0.1669921875	92.6808467730252\\
-0.16650390625	91.5730830478051\\
-0.166015625	90.4288554241866\\
-0.16552734375	87.7620777025694\\
-0.1650390625	96.117323513385\\
-0.16455078125	100.964295480857\\
-0.1640625	99.6605072248816\\
-0.16357421875	95.247731260265\\
-0.1630859375	98.1802731226977\\
-0.16259765625	102.501314852937\\
-0.162109375	101.512651215724\\
-0.16162109375	94.2055760080886\\
-0.1611328125	87.9235173932676\\
-0.16064453125	93.1828895446522\\
-0.16015625	91.3401911626727\\
-0.15966796875	67.6140340704034\\
-0.1591796875	89.0896492936637\\
-0.15869140625	101.044833327336\\
-0.158203125	102.968443064217\\
-0.15771484375	97.2116803303565\\
-0.1572265625	86.19442404925\\
-0.15673828125	97.9475459723765\\
-0.15625	105.417074351521\\
-0.15576171875	107.009123601688\\
-0.1552734375	104.523334979506\\
-0.15478515625	98.3735437494776\\
-0.154296875	87.6554159066004\\
-0.15380859375	82.366441559386\\
-0.1533203125	93.0577321183688\\
-0.15283203125	97.682798456623\\
-0.15234375	95.3301279391212\\
-0.15185546875	83.6888432031342\\
-0.1513671875	90.6823357865738\\
-0.15087890625	102.447714651567\\
-0.150390625	106.394403694636\\
-0.14990234375	104.915177777195\\
-0.1494140625	97.8849895602593\\
-0.14892578125	82.4423937574878\\
-0.1484375	62.8951766509757\\
-0.14794921875	63.926201676696\\
-0.1474609375	54.3083198865586\\
-0.14697265625	80.5713385359989\\
-0.146484375	85.4164093812254\\
-0.14599609375	81.5249004438052\\
-0.1455078125	88.0162130630555\\
-0.14501953125	98.8114766998071\\
-0.14453125	103.493553738036\\
-0.14404296875	102.978267873977\\
-0.1435546875	97.130282972185\\
-0.14306640625	83.2216590111739\\
-0.142578125	35.5906737243414\\
-0.14208984375	76.9505094674492\\
-0.1416015625	88.7217501621467\\
-0.14111328125	96.2379123071217\\
-0.140625	98.7117947819762\\
-0.14013671875	95.2044758025952\\
-0.1396484375	86.9072439562696\\
-0.13916015625	94.2346088628558\\
-0.138671875	101.735954370548\\
-0.13818359375	102.739358869063\\
-0.1376953125	98.0350997086958\\
-0.13720703125	92.7426409500127\\
-0.13671875	97.8796639559849\\
-0.13623046875	101.048656914974\\
-0.1357421875	97.6945296709185\\
-0.13525390625	83.018626686943\\
-0.134765625	72.8224573871042\\
-0.13427734375	92.2213110370639\\
-0.1337890625	96.3068286736427\\
-0.13330078125	96.6449895834272\\
-0.1328125	97.5585645054109\\
-0.13232421875	98.0682857803101\\
-0.1318359375	94.8469694601429\\
-0.13134765625	86.2665218207293\\
-0.130859375	86.329085121251\\
-0.13037109375	94.7919310768764\\
-0.1298828125	98.6828489788919\\
-0.12939453125	100.533263384271\\
-0.12890625	102.790440830309\\
-0.12841796875	104.802155532404\\
-0.1279296875	104.942269025051\\
-0.12744140625	102.541687305462\\
-0.126953125	97.4047740915575\\
-0.12646484375	89.2205866039686\\
-0.1259765625	83.4450533793057\\
-0.12548828125	91.9067149115024\\
-0.125	98.7337741946926\\
-0.12451171875	101.010231646039\\
-0.1240234375	99.6234637812853\\
-0.12353515625	97.3574799137484\\
-0.123046875	98.6309351346146\\
-0.12255859375	100.411632185203\\
-0.1220703125	98.6344953491069\\
-0.12158203125	93.2232020801304\\
-0.12109375	92.594890236895\\
-0.12060546875	97.1664723262218\\
-0.1201171875	97.6863173048835\\
-0.11962890625	93.2150046806395\\
-0.119140625	89.8371468778735\\
-0.11865234375	93.6826374975815\\
-0.1181640625	94.1429777553859\\
-0.11767578125	86.458451953031\\
-0.1171875	74.4267153762754\\
-0.11669921875	88.172269311377\\
-0.1162109375	91.002333303599\\
-0.11572265625	80.4459152629886\\
-0.115234375	71.9340520100619\\
-0.11474609375	94.0563619001943\\
-0.1142578125	98.9520683320459\\
-0.11376953125	95.1689676513302\\
-0.11328125	78.0368168680365\\
-0.11279296875	89.0019045200705\\
-0.1123046875	99.6383723818075\\
-0.11181640625	101.048831604733\\
-0.111328125	96.232765378963\\
-0.11083984375	92.8178996767274\\
-0.1103515625	100.041472035206\\
-0.10986328125	104.643636739127\\
-0.109375	104.935863795279\\
-0.10888671875	102.009840124511\\
-0.1083984375	97.263009911167\\
-0.10791015625	91.614319889549\\
-0.107421875	82.6486361632774\\
-0.10693359375	64.2028348602625\\
-0.1064453125	75.2028491734482\\
-0.10595703125	81.3366260748083\\
-0.10546875	79.339054109549\\
-0.10498046875	80.7537306432869\\
-0.1044921875	87.5642278248005\\
-0.10400390625	89.6394392863566\\
-0.103515625	84.7716520715617\\
-0.10302734375	65.3853813056729\\
-0.1025390625	67.745641263629\\
-0.10205078125	79.372754159822\\
-0.1015625	76.2308223906138\\
-0.10107421875	58.7527558045696\\
-0.1005859375	73.8182177985813\\
-0.10009765625	80.5363987802001\\
-0.099609375	75.6746584002045\\
-0.09912109375	44.1710311342991\\
-0.0986328125	76.8139909681197\\
-0.09814453125	86.083719376768\\
-0.09765625	88.5986490638088\\
-0.09716796875	90.1427041707452\\
-0.0966796875	93.2383440570549\\
-0.09619140625	95.3774143434749\\
-0.095703125	94.522032296022\\
-0.09521484375	89.9618314963461\\
-0.0947265625	80.7135021001742\\
-0.09423828125	61.8156726990346\\
-0.09375	71.4405454702341\\
-0.09326171875	86.0407013124646\\
-0.0927734375	92.9001967644251\\
-0.09228515625	95.0080306400882\\
-0.091796875	93.0025103002194\\
-0.09130859375	87.4727848244832\\
-0.0908203125	82.2156265119646\\
-0.09033203125	81.378641336291\\
-0.08984375	80.461307128326\\
-0.08935546875	81.961388129765\\
-0.0888671875	86.8197687322979\\
-0.08837890625	88.507501755138\\
-0.087890625	84.1976060458333\\
-0.08740234375	70.1596556819312\\
-0.0869140625	68.8110302470464\\
-0.08642578125	76.3989539280722\\
-0.0859375	69.5013307509336\\
-0.08544921875	70.0194114274186\\
-0.0849609375	85.3247414274367\\
-0.08447265625	90.2003189028513\\
-0.083984375	90.4294972939259\\
-0.08349609375	94.11969997802\\
-0.0830078125	100.743346534426\\
-0.08251953125	104.085622356009\\
-0.08203125	102.536533895603\\
-0.08154296875	93.9755460724788\\
-0.0810546875	78.9846256091559\\
-0.08056640625	94.5884637975774\\
-0.080078125	100.752572617264\\
-0.07958984375	99.5179840038311\\
-0.0791015625	90.605087566336\\
-0.07861328125	63.3397136741292\\
-0.078125	80.6077052949852\\
-0.07763671875	85.1053840967927\\
-0.0771484375	76.8947427951381\\
-0.07666015625	71.5010734879631\\
-0.076171875	85.3769123679657\\
-0.07568359375	89.5072698642376\\
-0.0751953125	88.4569100132446\\
-0.07470703125	86.2456502969386\\
-0.07421875	84.5202053133017\\
-0.07373046875	77.9610679765922\\
-0.0732421875	52.6364242742568\\
-0.07275390625	79.3709220263003\\
-0.072265625	87.7369761798029\\
-0.07177734375	86.1964231006543\\
-0.0712890625	71.0305712423776\\
-0.07080078125	79.0792754771083\\
-0.0703125	92.7894305472686\\
-0.06982421875	96.7164280618463\\
-0.0693359375	94.6482682554997\\
-0.06884765625	85.8043700144218\\
-0.068359375	58.8517725440028\\
-0.06787109375	71.9987586471001\\
-0.0673828125	80.6181974693754\\
-0.06689453125	80.3796542714777\\
-0.06640625	75.5109064856731\\
-0.06591796875	70.0774525228849\\
-0.0654296875	78.6484214032286\\
-0.06494140625	87.6978090490571\\
-0.064453125	91.4866176598043\\
-0.06396484375	89.7946883113632\\
-0.0634765625	80.0832250251725\\
-0.06298828125	64.5169766635001\\
-0.0625	82.9841242889914\\
-0.06201171875	87.9119178871606\\
-0.0615234375	85.3279553838404\\
-0.06103515625	81.1183471398356\\
-0.060546875	86.3676718267228\\
-0.06005859375	90.3181996324631\\
-0.0595703125	88.2415555675799\\
-0.05908203125	77.7312649086447\\
-0.05859375	65.4167346402624\\
-0.05810546875	80.1748331046881\\
-0.0576171875	84.1185927405294\\
-0.05712890625	83.3669144666221\\
-0.056640625	83.3843094112987\\
-0.05615234375	83.8795369843273\\
-0.0556640625	79.6348848526197\\
-0.05517578125	66.8864351659822\\
-0.0546875	76.6034881196223\\
-0.05419921875	85.7018757117113\\
-0.0537109375	86.7515249818443\\
-0.05322265625	80.5086298853689\\
-0.052734375	59.6883427130988\\
-0.05224609375	63.2122617618822\\
-0.0517578125	72.8863561780914\\
-0.05126953125	71.3850234565665\\
-0.05078125	70.9688464634426\\
-0.05029296875	76.1058962036319\\
-0.0498046875	79.0063320954792\\
-0.04931640625	81.6683399448381\\
-0.048828125	85.8658353379033\\
-0.04833984375	88.0023681335393\\
-0.0478515625	85.3461504088377\\
-0.04736328125	76.0020233122673\\
-0.046875	75.9743325200458\\
-0.04638671875	85.5634675293978\\
-0.0458984375	87.8921430625081\\
-0.04541015625	83.4535981439809\\
-0.044921875	71.4986297364877\\
-0.04443359375	71.0020585899561\\
-0.0439453125	75.3597635254039\\
-0.04345703125	64.9832182480367\\
-0.04296875	63.9401922043359\\
-0.04248046875	85.319331169052\\
-0.0419921875	93.0293362996208\\
-0.04150390625	95.6034576367354\\
-0.041015625	94.9119888290039\\
-0.04052734375	91.5861087290305\\
-0.0400390625	85.2784210282259\\
-0.03955078125	73.889831323871\\
-0.0390625	54.7387300941699\\
-0.03857421875	58.097537722004\\
-0.0380859375	54.6719502361204\\
-0.03759765625	72.8776265449139\\
-0.037109375	85.9975956112846\\
-0.03662109375	91.6474127381245\\
-0.0361328125	91.7078421748668\\
-0.03564453125	85.4169706769966\\
-0.03515625	66.9284343487812\\
-0.03466796875	73.3894952130462\\
-0.0341796875	81.9829434048636\\
-0.03369140625	79.8814673612874\\
-0.033203125	73.5449788734434\\
-0.03271484375	80.7202942527146\\
-0.0322265625	85.9355273135806\\
-0.03173828125	84.136333755039\\
-0.03125	75.1976424121713\\
-0.03076171875	74.1636921080332\\
-0.0302734375	81.7354506263253\\
-0.02978515625	83.4767390680268\\
-0.029296875	84.880870076297\\
-0.02880859375	89.7569697826228\\
-0.0283203125	92.5943010090168\\
-0.02783203125	90.232279454053\\
-0.02734375	81.2955749463573\\
-0.02685546875	83.7503533715595\\
-0.0263671875	93.7986226076816\\
-0.02587890625	97.3650498388441\\
-0.025390625	95.6744490258525\\
-0.02490234375	88.7107975261069\\
-0.0244140625	76.9173218572763\\
-0.02392578125	77.6318321383223\\
-0.0234375	84.1126017682818\\
-0.02294921875	87.7253597914395\\
-0.0224609375	89.2833539035948\\
-0.02197265625	87.2724477154547\\
-0.021484375	77.5865984038389\\
-0.02099609375	58.3617188516809\\
-0.0205078125	83.4443470434266\\
-0.02001953125	90.6772606558402\\
-0.01953125	91.0205359778646\\
-0.01904296875	85.8690841401953\\
-0.0185546875	74.8737552634476\\
-0.01806640625	66.9970144073564\\
-0.017578125	70.5592646871618\\
-0.01708984375	72.2458025941564\\
-0.0166015625	75.1979156910833\\
-0.01611328125	76.8137165415458\\
-0.015625	72.9983380582437\\
-0.01513671875	60.839891909615\\
-0.0146484375	59.3174449973537\\
-0.01416015625	64.058240370149\\
-0.013671875	58.1583366830206\\
-0.01318359375	69.7236829457419\\
-0.0126953125	79.5398553275265\\
-0.01220703125	80.8277909866809\\
-0.01171875	73.1867739750736\\
-0.01123046875	72.2321938581467\\
-0.0107421875	85.2418740917543\\
-0.01025390625	89.5953239753207\\
-0.009765625	86.494583378766\\
-0.00927734375	73.7629224003438\\
-0.0087890625	80.2430836828565\\
-0.00830078125	90.793924161082\\
-0.0078125	93.0444027127361\\
-0.00732421875	88.6962975006426\\
-0.0068359375	74.8561735415655\\
-0.00634765625	67.1745425508319\\
-0.005859375	78.7335640274907\\
-0.00537109375	77.4899152932268\\
-0.0048828125	66.8232374959824\\
-0.00439453125	72.8415046590633\\
-0.00390625	80.4780726844455\\
-0.00341796875	79.8372138262443\\
-0.0029296875	70.6917869267042\\
-0.00244140625	50.9739398355379\\
-0.001953125	62.3843259531691\\
-0.00146484375	61.0685411562139\\
-0.0009765625	34.6682231583663\\
-0.00048828125	61.0301391781166\\
0	67.4682326928463\\
0.00048828125	61.0301391781166\\
0.0009765625	34.6682231583663\\
0.00146484375	61.0685411562139\\
0.001953125	62.3843259531691\\
0.00244140625	50.9739398355379\\
0.0029296875	70.6917869267042\\
0.00341796875	79.8372138262443\\
0.00390625	80.4780726844455\\
0.00439453125	72.8415046590633\\
0.0048828125	66.8232374959824\\
0.00537109375	77.4899152932268\\
0.005859375	78.7335640274907\\
0.00634765625	67.1745425508319\\
0.0068359375	74.8561735415655\\
0.00732421875	88.6962975006426\\
0.0078125	93.0444027127361\\
0.00830078125	90.793924161082\\
0.0087890625	80.2430836828565\\
0.00927734375	73.7629224003438\\
0.009765625	86.494583378766\\
0.01025390625	89.5953239753207\\
0.0107421875	85.2418740917543\\
0.01123046875	72.2321938581467\\
0.01171875	73.1867739750736\\
0.01220703125	80.8277909866809\\
0.0126953125	79.5398553275265\\
0.01318359375	69.7236829457419\\
0.013671875	58.1583366830206\\
0.01416015625	64.058240370149\\
0.0146484375	59.3174449973537\\
0.01513671875	60.839891909615\\
0.015625	72.9983380582437\\
0.01611328125	76.8137165415458\\
0.0166015625	75.1979156910833\\
0.01708984375	72.2458025941564\\
0.017578125	70.5592646871618\\
0.01806640625	66.9970144073564\\
0.0185546875	74.8737552634476\\
0.01904296875	85.8690841401953\\
0.01953125	91.0205359778646\\
0.02001953125	90.6772606558402\\
0.0205078125	83.4443470434266\\
0.02099609375	58.3617188516809\\
0.021484375	77.5865984038389\\
0.02197265625	87.2724477154547\\
0.0224609375	89.2833539035948\\
0.02294921875	87.7253597914395\\
0.0234375	84.1126017682818\\
0.02392578125	77.6318321383223\\
0.0244140625	76.9173218572763\\
0.02490234375	88.7107975261069\\
0.025390625	95.6744490258525\\
0.02587890625	97.3650498388441\\
0.0263671875	93.7986226076816\\
0.02685546875	83.7503533715595\\
0.02734375	81.2955749463573\\
0.02783203125	90.232279454053\\
0.0283203125	92.5943010090168\\
0.02880859375	89.7569697826228\\
0.029296875	84.880870076297\\
0.02978515625	83.4767390680268\\
0.0302734375	81.7354506263253\\
0.03076171875	74.1636921080332\\
0.03125	75.1976424121713\\
0.03173828125	84.136333755039\\
0.0322265625	85.9355273135806\\
0.03271484375	80.7202942527146\\
0.033203125	73.5449788734434\\
0.03369140625	79.8814673612874\\
0.0341796875	81.9829434048636\\
0.03466796875	73.3894952130462\\
0.03515625	66.9284343487812\\
0.03564453125	85.4169706769966\\
0.0361328125	91.7078421748668\\
0.03662109375	91.6474127381245\\
0.037109375	85.9975956112846\\
0.03759765625	72.8776265449139\\
0.0380859375	54.6719502361204\\
0.03857421875	58.097537722004\\
0.0390625	54.7387300941699\\
0.03955078125	73.889831323871\\
0.0400390625	85.2784210282259\\
0.04052734375	91.5861087290305\\
0.041015625	94.9119888290039\\
0.04150390625	95.6034576367354\\
0.0419921875	93.0293362996208\\
0.04248046875	85.319331169052\\
0.04296875	63.9401922043359\\
0.04345703125	64.9832182480367\\
0.0439453125	75.3597635254039\\
0.04443359375	71.0020585899561\\
0.044921875	71.4986297364877\\
0.04541015625	83.4535981439809\\
0.0458984375	87.8921430625081\\
0.04638671875	85.5634675293978\\
0.046875	75.9743325200458\\
0.04736328125	76.0020233122673\\
0.0478515625	85.3461504088377\\
0.04833984375	88.0023681335393\\
0.048828125	85.8658353379033\\
0.04931640625	81.6683399448381\\
0.0498046875	79.0063320954792\\
0.05029296875	76.1058962036319\\
0.05078125	70.9688464634426\\
0.05126953125	71.3850234565665\\
0.0517578125	72.8863561780914\\
0.05224609375	63.2122617618822\\
0.052734375	59.6883427130988\\
0.05322265625	80.5086298853689\\
0.0537109375	86.7515249818443\\
0.05419921875	85.7018757117113\\
0.0546875	76.6034881196223\\
0.05517578125	66.8864351659822\\
0.0556640625	79.6348848526197\\
0.05615234375	83.8795369843273\\
0.056640625	83.3843094112987\\
0.05712890625	83.3669144666221\\
0.0576171875	84.1185927405294\\
0.05810546875	80.1748331046881\\
0.05859375	65.4167346402624\\
0.05908203125	77.7312649086447\\
0.0595703125	88.2415555675799\\
0.06005859375	90.3181996324631\\
0.060546875	86.3676718267228\\
0.06103515625	81.1183471398356\\
0.0615234375	85.3279553838404\\
0.06201171875	87.9119178871606\\
0.0625	82.9841242889914\\
0.06298828125	64.5169766635001\\
0.0634765625	80.0832250251725\\
0.06396484375	89.7946883113632\\
0.064453125	91.4866176598043\\
0.06494140625	87.6978090490571\\
0.0654296875	78.6484214032286\\
0.06591796875	70.0774525228849\\
0.06640625	75.5109064856731\\
0.06689453125	80.3796542714777\\
0.0673828125	80.6181974693754\\
0.06787109375	71.9987586471001\\
0.068359375	58.8517725440028\\
0.06884765625	85.8043700144218\\
0.0693359375	94.6482682554997\\
0.06982421875	96.7164280618463\\
0.0703125	92.7894305472686\\
0.07080078125	79.0792754771083\\
0.0712890625	71.0305712423776\\
0.07177734375	86.1964231006543\\
0.072265625	87.7369761798029\\
0.07275390625	79.3709220263003\\
0.0732421875	52.6364242742568\\
0.07373046875	77.9610679765922\\
0.07421875	84.5202053133017\\
0.07470703125	86.2456502969386\\
0.0751953125	88.4569100132446\\
0.07568359375	89.5072698642376\\
0.076171875	85.3769123679657\\
0.07666015625	71.5010734879631\\
0.0771484375	76.8947427951381\\
0.07763671875	85.1053840967927\\
0.078125	80.6077052949852\\
0.07861328125	63.3397136741292\\
0.0791015625	90.605087566336\\
0.07958984375	99.5179840038311\\
0.080078125	100.752572617264\\
0.08056640625	94.5884637975774\\
0.0810546875	78.9846256091559\\
0.08154296875	93.9755460724788\\
0.08203125	102.536533895603\\
0.08251953125	104.085622356009\\
0.0830078125	100.743346534426\\
0.08349609375	94.11969997802\\
0.083984375	90.4294972939259\\
0.08447265625	90.2003189028513\\
0.0849609375	85.3247414274367\\
0.08544921875	70.0194114274186\\
0.0859375	69.5013307509336\\
0.08642578125	76.3989539280722\\
0.0869140625	68.8110302470464\\
0.08740234375	70.1596556819312\\
0.087890625	84.1976060458333\\
0.08837890625	88.507501755138\\
0.0888671875	86.8197687322979\\
0.08935546875	81.961388129765\\
0.08984375	80.461307128326\\
0.09033203125	81.378641336291\\
0.0908203125	82.2156265119646\\
0.09130859375	87.4727848244832\\
0.091796875	93.0025103002194\\
0.09228515625	95.0080306400882\\
0.0927734375	92.9001967644251\\
0.09326171875	86.0407013124646\\
0.09375	71.4405454702341\\
0.09423828125	61.8156726990346\\
0.0947265625	80.7135021001742\\
0.09521484375	89.9618314963461\\
0.095703125	94.522032296022\\
0.09619140625	95.3774143434749\\
0.0966796875	93.2383440570549\\
0.09716796875	90.1427041707452\\
0.09765625	88.5986490638088\\
0.09814453125	86.083719376768\\
0.0986328125	76.8139909681197\\
0.09912109375	44.1710311342991\\
0.099609375	75.6746584002045\\
0.10009765625	80.5363987802001\\
0.1005859375	73.8182177985813\\
0.10107421875	58.7527558045696\\
0.1015625	76.2308223906138\\
0.10205078125	79.372754159822\\
0.1025390625	67.745641263629\\
0.10302734375	65.3853813056729\\
0.103515625	84.7716520715617\\
0.10400390625	89.6394392863566\\
0.1044921875	87.5642278248005\\
0.10498046875	80.7537306432869\\
0.10546875	79.339054109549\\
0.10595703125	81.3366260748083\\
0.1064453125	75.2028491734482\\
0.10693359375	64.2028348602625\\
0.107421875	82.6486361632774\\
0.10791015625	91.614319889549\\
0.1083984375	97.263009911167\\
0.10888671875	102.009840124511\\
0.109375	104.935863795279\\
0.10986328125	104.643636739127\\
0.1103515625	100.041472035206\\
0.11083984375	92.8178996767274\\
0.111328125	96.232765378963\\
0.11181640625	101.048831604733\\
0.1123046875	99.6383723818075\\
0.11279296875	89.0019045200705\\
0.11328125	78.0368168680365\\
0.11376953125	95.1689676513302\\
0.1142578125	98.9520683320459\\
0.11474609375	94.0563619001943\\
0.115234375	71.9340520100619\\
0.11572265625	80.4459152629886\\
0.1162109375	91.002333303599\\
0.11669921875	88.172269311377\\
0.1171875	74.4267153762754\\
0.11767578125	86.458451953031\\
0.1181640625	94.1429777553859\\
0.11865234375	93.6826374975815\\
0.119140625	89.8371468778735\\
0.11962890625	93.2150046806395\\
0.1201171875	97.6863173048835\\
0.12060546875	97.1664723262218\\
0.12109375	92.594890236895\\
0.12158203125	93.2232020801304\\
0.1220703125	98.6344953491069\\
0.12255859375	100.411632185203\\
0.123046875	98.6309351346146\\
0.12353515625	97.3574799137484\\
0.1240234375	99.6234637812853\\
0.12451171875	101.010231646039\\
0.125	98.7337741946926\\
0.12548828125	91.9067149115024\\
0.1259765625	83.4450533793057\\
0.12646484375	89.2205866039686\\
0.126953125	97.4047740915575\\
0.12744140625	102.541687305462\\
0.1279296875	104.942269025051\\
0.12841796875	104.802155532404\\
0.12890625	102.790440830309\\
0.12939453125	100.533263384271\\
0.1298828125	98.6828489788919\\
0.13037109375	94.7919310768764\\
0.130859375	86.329085121251\\
0.13134765625	86.2665218207293\\
0.1318359375	94.8469694601429\\
0.13232421875	98.0682857803101\\
0.1328125	97.5585645054109\\
0.13330078125	96.6449895834272\\
0.1337890625	96.3068286736427\\
0.13427734375	92.2213110370639\\
0.134765625	72.8224573871042\\
0.13525390625	83.018626686943\\
0.1357421875	97.6945296709185\\
0.13623046875	101.048656914974\\
0.13671875	97.8796639559849\\
0.13720703125	92.7426409500127\\
0.1376953125	98.0350997086958\\
0.13818359375	102.739358869063\\
0.138671875	101.735954370548\\
0.13916015625	94.2346088628558\\
0.1396484375	86.9072439562696\\
0.14013671875	95.2044758025952\\
0.140625	98.7117947819762\\
0.14111328125	96.2379123071217\\
0.1416015625	88.7217501621467\\
0.14208984375	76.9505094674492\\
0.142578125	35.5906737243414\\
0.14306640625	83.2216590111739\\
0.1435546875	97.130282972185\\
0.14404296875	102.978267873977\\
0.14453125	103.493553738036\\
0.14501953125	98.8114766998071\\
0.1455078125	88.0162130630555\\
0.14599609375	81.5249004438052\\
0.146484375	85.4164093812254\\
0.14697265625	80.5713385359989\\
0.1474609375	54.3083198865586\\
0.14794921875	63.926201676696\\
0.1484375	62.8951766509757\\
0.14892578125	82.4423937574878\\
0.1494140625	97.8849895602593\\
0.14990234375	104.915177777195\\
0.150390625	106.394403694636\\
0.15087890625	102.447714651567\\
0.1513671875	90.6823357865738\\
0.15185546875	83.6888432031342\\
0.15234375	95.3301279391212\\
0.15283203125	97.682798456623\\
0.1533203125	93.0577321183688\\
0.15380859375	82.366441559386\\
0.154296875	87.6554159066004\\
0.15478515625	98.3735437494776\\
0.1552734375	104.523334979506\\
0.15576171875	107.009123601688\\
0.15625	105.417074351521\\
0.15673828125	97.9475459723765\\
0.1572265625	86.19442404925\\
0.15771484375	97.2116803303565\\
0.158203125	102.968443064217\\
0.15869140625	101.044833327336\\
0.1591796875	89.0896492936637\\
0.15966796875	67.6140340704034\\
0.16015625	91.3401911626727\\
0.16064453125	93.1828895446522\\
0.1611328125	87.9235173932676\\
0.16162109375	94.2055760080886\\
0.162109375	101.512651215724\\
0.16259765625	102.501314852937\\
0.1630859375	98.1802731226977\\
0.16357421875	95.247731260265\\
0.1640625	99.6605072248816\\
0.16455078125	100.964295480857\\
0.1650390625	96.117323513385\\
0.16552734375	87.7620777025694\\
0.166015625	90.4288554241866\\
0.16650390625	91.5730830478051\\
0.1669921875	92.6808467730252\\
0.16748046875	103.755590118014\\
0.16796875	111.38757046098\\
0.16845703125	114.044492669364\\
0.1689453125	112.679936620681\\
0.16943359375	109.195326277577\\
0.169921875	107.716097941818\\
0.17041015625	106.99915723873\\
0.1708984375	100.057083340093\\
0.17138671875	53.9289310965851\\
0.171875	103.839932973445\\
0.17236328125	113.640256343248\\
0.1728515625	115.698746589558\\
0.17333984375	112.048871316461\\
0.173828125	102.632192396162\\
0.17431640625	100.061897912525\\
0.1748046875	106.550559506422\\
0.17529296875	107.085657076546\\
0.17578125	102.173696117756\\
0.17626953125	100.848501362414\\
0.1767578125	107.695576705396\\
0.17724609375	111.748012968481\\
0.177734375	112.154764909945\\
0.17822265625	109.764182664897\\
0.1787109375	105.16086040165\\
0.17919921875	98.7393789694684\\
0.1796875	92.8383014361383\\
0.18017578125	93.35718837213\\
0.1806640625	96.8878556735035\\
0.18115234375	99.5488018585284\\
0.181640625	101.382862514472\\
0.18212890625	101.873836541966\\
0.1826171875	100.096669272568\\
0.18310546875	96.7521869155409\\
0.18359375	95.2164194874931\\
0.18408203125	95.3328606819847\\
0.1845703125	98.761051532094\\
0.18505859375	108.510514096506\\
0.185546875	116.521490301565\\
0.18603515625	120.450900693064\\
0.1865234375	120.009555049558\\
0.18701171875	113.701293180779\\
0.1875	91.7773684079639\\
0.18798828125	103.324208726746\\
0.1884765625	115.628178526074\\
0.18896484375	118.109589596257\\
0.189453125	114.515434548491\\
0.18994140625	103.313408739957\\
0.1904296875	63.1448102050065\\
0.19091796875	93.4185801512839\\
0.19140625	98.0247206237179\\
0.19189453125	96.4657638478649\\
0.1923828125	99.045587133845\\
0.19287109375	106.366343802299\\
0.193359375	113.249885998011\\
0.19384765625	117.701999553851\\
0.1943359375	118.617420451685\\
0.19482421875	115.72319958095\\
0.1953125	115.318290128112\\
0.19580078125	123.270276016507\\
0.1962890625	129.136121178423\\
0.19677734375	130.537054992427\\
0.197265625	126.966839243274\\
0.19775390625	115.916618918063\\
0.1982421875	104.68109709841\\
0.19873046875	117.63621004922\\
0.19921875	119.348557625269\\
0.19970703125	108.477316083946\\
0.2001953125	99.5254155495257\\
0.20068359375	122.145699289168\\
0.201171875	128.73790154234\\
0.20166015625	129.660225029413\\
0.2021484375	128.09270594616\\
0.20263671875	128.094573620796\\
0.203125	129.994846260067\\
0.20361328125	130.66827803124\\
0.2041015625	129.873011359486\\
0.20458984375	129.748449699441\\
0.205078125	130.444936787012\\
0.20556640625	129.169043180954\\
0.2060546875	123.619083566003\\
0.20654296875	113.238436224572\\
0.20703125	110.714660967036\\
0.20751953125	113.312688109344\\
0.2080078125	118.683083028619\\
0.20849609375	129.68663250778\\
0.208984375	136.635868988689\\
0.20947265625	138.436517132376\\
0.2099609375	134.759577728675\\
0.21044921875	128.26333011335\\
0.2109375	136.21793265436\\
0.21142578125	144.045357805811\\
0.2119140625	146.48932020821\\
0.21240234375	143.917306756779\\
0.212890625	134.99721869845\\
0.21337890625	120.580599203933\\
0.2138671875	130.284105246204\\
0.21435546875	134.286821901049\\
0.21484375	131.121335871808\\
0.21533203125	128.974372538252\\
0.2158203125	134.156304529695\\
0.21630859375	135.18071296599\\
0.216796875	126.180618063057\\
0.21728515625	115.310310842527\\
0.2177734375	137.905711835599\\
0.21826171875	145.016323787033\\
0.21875	145.241849129103\\
0.21923828125	138.880157592359\\
0.2197265625	125.871062878654\\
0.22021484375	133.852421824027\\
0.220703125	139.525220763241\\
0.22119140625	136.999124209961\\
0.2216796875	133.547746428172\\
0.22216796875	143.368107055514\\
0.22265625	150.278707552142\\
0.22314453125	151.875270725525\\
0.2236328125	148.248136647809\\
0.22412109375	136.700615237075\\
0.224609375	115.445210299733\\
0.22509765625	136.555060362823\\
0.2255859375	140.867576351771\\
0.22607421875	137.456836158981\\
0.2265625	131.412525099808\\
0.22705078125	135.770381403749\\
0.2275390625	139.663517652822\\
0.22802734375	137.163034258518\\
0.228515625	126.085987774122\\
0.22900390625	128.545554512218\\
0.2294921875	140.74943581195\\
0.22998046875	145.924283966441\\
0.23046875	147.279834590544\\
0.23095703125	146.411123312439\\
0.2314453125	143.952602521825\\
0.23193359375	139.371248889953\\
0.232421875	131.950517220316\\
0.23291015625	131.060102761353\\
0.2333984375	139.051146551463\\
0.23388671875	143.568401102769\\
0.234375	143.93412630864\\
0.23486328125	140.319299633709\\
0.2353515625	131.888288023028\\
0.23583984375	114.697277988251\\
0.236328125	94.9493862844397\\
0.23681640625	114.429626673282\\
0.2373046875	123.749862324925\\
0.23779296875	131.358975322802\\
0.23828125	136.120814978354\\
0.23876953125	137.909516111366\\
0.2392578125	137.027715842101\\
0.23974609375	133.694149173011\\
0.240234375	127.931152333599\\
0.24072265625	120.311062439946\\
0.2412109375	118.583967937324\\
0.24169921875	125.691710874076\\
0.2421875	131.391458685736\\
0.24267578125	133.996029231018\\
0.2431640625	133.424504245769\\
0.24365234375	129.191260367515\\
0.244140625	121.934329223342\\
0.24462890625	123.997991798586\\
0.2451171875	131.657090850677\\
0.24560546875	135.323674183586\\
0.24609375	134.874298159793\\
0.24658203125	130.049814209065\\
0.2470703125	119.579507173435\\
0.24755859375	114.460547487037\\
0.248046875	123.226012427598\\
0.24853515625	126.57542580871\\
0.2490234375	125.075434040072\\
0.24951171875	119.324265446783\\
0.25	108.66320420064\\
0.25048828125	89.371087930145\\
0.2509765625	82.4736218915921\\
0.25146484375	95.8338631241275\\
0.251953125	100.801265437434\\
0.25244140625	102.939471482126\\
0.2529296875	106.766966838949\\
0.25341796875	112.403841275327\\
0.25390625	117.169443879161\\
0.25439453125	121.552352737421\\
0.2548828125	126.694985674488\\
0.25537109375	131.579967431602\\
0.255859375	134.674273957914\\
0.25634765625	135.142427663948\\
0.2568359375	132.275915101827\\
0.25732421875	124.575016207134\\
0.2578125	108.650246134247\\
0.25830078125	111.67192474406\\
0.2587890625	118.681941872209\\
0.25927734375	115.295764426539\\
0.259765625	102.952481166405\\
0.26025390625	120.084639240175\\
0.2607421875	130.429486081856\\
0.26123046875	134.737882230589\\
0.26171875	135.780735713195\\
0.26220703125	135.670144129938\\
0.2626953125	136.151764858091\\
0.26318359375	136.596862040404\\
0.263671875	134.996245479094\\
0.26416015625	129.644642605665\\
0.2646484375	118.671094162311\\
0.26513671875	109.107706397714\\
0.265625	116.202390036276\\
0.26611328125	122.402891125154\\
0.2666015625	129.425422197328\\
0.26708984375	134.737234421909\\
0.267578125	136.451134890872\\
0.26806640625	134.677775142302\\
0.2685546875	132.801553584912\\
0.26904296875	135.627663524288\\
0.26953125	137.982839426616\\
0.27001953125	135.203782845396\\
0.2705078125	123.689067938829\\
0.27099609375	124.784422458248\\
0.271484375	135.911567611763\\
0.27197265625	137.968568338478\\
0.2724609375	139.616392321535\\
0.27294921875	149.946198900339\\
0.2734375	158.64829142382\\
0.27392578125	163.212636444484\\
0.2744140625	163.922242867391\\
0.27490234375	160.409661104179\\
0.275390625	151.098186525242\\
0.27587890625	145.238025270522\\
0.2763671875	156.358366113924\\
0.27685546875	161.948706667979\\
0.27734375	162.893997353358\\
0.27783203125	160.516784009867\\
0.2783203125	155.992235181148\\
0.27880859375	151.274841615132\\
0.279296875	147.116603759107\\
0.27978515625	142.34418816084\\
0.2802734375	143.621489881627\\
0.28076171875	150.883583314564\\
0.28125	155.239729422516\\
0.28173828125	155.687494302067\\
0.2822265625	152.290215670988\\
0.28271484375	144.768361895555\\
0.283203125	134.439505771918\\
0.28369140625	130.272040915486\\
0.2841796875	127.235086023498\\
0.28466796875	116.595966096479\\
0.28515625	124.810564537983\\
0.28564453125	132.735739672642\\
0.2861328125	131.779185918107\\
0.28662109375	118.639695110746\\
0.287109375	119.26890624775\\
0.28759765625	135.685144625913\\
0.2880859375	140.923787404449\\
0.28857421875	140.531411864229\\
0.2890625	135.277745887419\\
0.28955078125	124.752979231882\\
0.2900390625	116.165604631174\\
0.29052734375	118.354131368198\\
0.291015625	115.867593760004\\
0.29150390625	113.304137091982\\
0.2919921875	119.458502156223\\
0.29248046875	122.272307880302\\
0.29296875	117.992333905884\\
0.29345703125	97.4541650428343\\
0.2939453125	107.069279075224\\
0.29443359375	121.368712964792\\
0.294921875	125.600800564161\\
0.29541015625	125.548242611474\\
0.2958984375	123.395165760277\\
0.29638671875	120.799857542564\\
0.296875	117.800961729227\\
0.29736328125	112.8747224361\\
0.2978515625	105.0982093064\\
0.29833984375	95.3054285116443\\
0.298828125	74.5740514607808\\
0.29931640625	97.5686297792309\\
0.2998046875	113.670186160417\\
0.30029296875	121.260455230547\\
0.30078125	123.912773668588\\
0.30126953125	122.51131107475\\
0.3017578125	117.86344348301\\
0.30224609375	114.625804760201\\
0.302734375	116.627606511945\\
0.30322265625	117.176737737154\\
0.3037109375	113.771146131697\\
0.30419921875	111.635103589614\\
0.3046875	116.30522590756\\
0.30517578125	118.893306832598\\
0.3056640625	115.675490702458\\
0.30615234375	102.153293521191\\
0.306640625	97.2054865894442\\
0.30712890625	112.733293155335\\
0.3076171875	115.500518893878\\
0.30810546875	109.912296283152\\
0.30859375	90.7904433688858\\
0.30908203125	101.756929292174\\
0.3095703125	109.79594456437\\
0.31005859375	107.527684678125\\
0.310546875	97.5306540043609\\
0.31103515625	103.97274826422\\
0.3115234375	111.29816699283\\
0.31201171875	110.516088905325\\
0.3125	99.6991957862558\\
0.31298828125	91.2930056531243\\
0.3134765625	108.049149096364\\
0.31396484375	112.243677977501\\
0.314453125	109.163787518493\\
0.31494140625	97.4209027344783\\
0.3154296875	72.7410375725865\\
0.31591796875	91.7214162843355\\
0.31640625	96.5075241153549\\
0.31689453125	102.787005204833\\
0.3173828125	109.209472454176\\
0.31787109375	111.337422827622\\
0.318359375	108.007212452923\\
0.31884765625	96.7386454315553\\
0.3193359375	89.4699402261996\\
0.31982421875	101.143947437534\\
0.3203125	103.099143391796\\
0.32080078125	96.9641000751897\\
0.3212890625	79.3495467328411\\
0.32177734375	81.7507224234003\\
0.322265625	85.4799333048203\\
0.32275390625	75.5067226687312\\
0.3232421875	88.7534404394998\\
0.32373046875	98.6235194622553\\
0.32421875	100.187206349446\\
0.32470703125	94.3300722906535\\
0.3251953125	81.4711046198778\\
0.32568359375	92.9372648914094\\
0.326171875	101.204328403577\\
0.32666015625	103.844994239085\\
0.3271484375	103.372458348508\\
0.32763671875	101.101653744061\\
0.328125	98.8675619067356\\
0.32861328125	99.9659952843048\\
0.3291015625	103.049494181695\\
0.32958984375	103.879434584948\\
0.330078125	100.102041353499\\
0.33056640625	88.6702549248488\\
0.3310546875	84.0074247099529\\
0.33154296875	95.3093839776248\\
0.33203125	97.102576022808\\
0.33251953125	90.9177700224685\\
0.3330078125	85.0956572198753\\
0.33349609375	95.1816609556897\\
0.333984375	100.028891603472\\
0.33447265625	99.9612269848058\\
0.3349609375	98.7165921484532\\
0.33544921875	99.9823714124503\\
0.3359375	101.204196116687\\
0.33642578125	99.2881872542043\\
0.3369140625	93.4525772445824\\
0.33740234375	84.9458872388593\\
0.337890625	78.7034689197511\\
0.33837890625	81.3773736116656\\
0.3388671875	91.2960937279039\\
0.33935546875	97.6772018775227\\
0.33984375	98.904644991693\\
0.34033203125	95.0194477068301\\
0.3408203125	90.0399511207787\\
0.34130859375	94.4582605689948\\
0.341796875	97.6653401544845\\
0.34228515625	93.913914977674\\
0.3427734375	78.8306206281154\\
0.34326171875	89.0043523407048\\
0.34375	99.5748253595847\\
0.34423828125	101.617194503319\\
0.3447265625	96.980755662721\\
0.34521484375	81.0500077326724\\
0.345703125	71.1562179342151\\
0.34619140625	88.8137452708128\\
0.3466796875	90.5583978324968\\
0.34716796875	84.305875883266\\
0.34765625	65.9694383521991\\
0.34814453125	57.2280857203198\\
0.3486328125	69.3634699160843\\
0.34912109375	66.9585935133334\\
0.349609375	66.5674817362102\\
0.35009765625	82.3440304922892\\
0.3505859375	93.8242390065795\\
0.35107421875	100.693454451162\\
0.3515625	103.857358730613\\
0.35205078125	103.649773501076\\
0.3525390625	100.120060146289\\
0.35302734375	93.4768605502132\\
0.353515625	85.7239604726104\\
0.35400390625	81.3371332929414\\
0.3544921875	82.1487819436112\\
0.35498046875	87.0666805309384\\
0.35546875	90.1135864533012\\
0.35595703125	87.6157267751558\\
0.3564453125	73.6822886363732\\
0.35693359375	68.2224450925351\\
0.357421875	86.7686066589739\\
0.35791015625	90.304218081375\\
0.3583984375	85.6747687014382\\
0.35888671875	68.7651302275383\\
0.359375	70.5213499877986\\
0.35986328125	80.7894870366368\\
0.3603515625	83.1212324495762\\
0.36083984375	87.3488416401946\\
0.361328125	91.9098933378966\\
0.36181640625	92.0240427965977\\
0.3623046875	85.3530080364105\\
0.36279296875	77.1757047035764\\
0.36328125	88.4816280278384\\
0.36376953125	93.3560296887624\\
0.3642578125	90.5996471683339\\
0.36474609375	77.2261817001538\\
0.365234375	71.1424556669159\\
0.36572265625	84.1157206009966\\
0.3662109375	82.6848028305417\\
0.36669921875	65.4333122458508\\
0.3671875	80.9577189710606\\
0.36767578125	91.0494173277348\\
0.3681640625	91.7027733564865\\
0.36865234375	83.6694752562662\\
0.369140625	63.2943759868229\\
0.36962890625	80.9770188436795\\
0.3701171875	85.2288839560229\\
0.37060546875	77.8173721591176\\
0.37109375	53.7255405644001\\
0.37158203125	80.3248856005095\\
0.3720703125	85.774862889613\\
0.37255859375	82.1361246064511\\
0.373046875	69.6627429095972\\
0.37353515625	65.5507881565805\\
0.3740234375	62.9363032255756\\
0.37451171875	57.6978953271855\\
0.375	84.3269261643277\\
0.37548828125	93.1743961591435\\
0.3759765625	94.7778184558175\\
0.37646484375	89.7225227250348\\
0.376953125	76.4140360369576\\
0.37744140625	82.0153677432322\\
0.3779296875	89.2439022423443\\
0.37841796875	87.9941027334088\\
0.37890625	82.2483919437309\\
0.37939453125	86.7249635233442\\
0.3798828125	92.3918852452678\\
0.38037109375	91.8934775699516\\
0.380859375	83.6689763855385\\
0.38134765625	66.7005570739045\\
0.3818359375	80.1419754057861\\
0.38232421875	84.0865502844635\\
0.3828125	78.4690317469602\\
0.38330078125	73.9903544367084\\
0.3837890625	84.6070545359288\\
0.38427734375	89.3840406083098\\
0.384765625	88.9878638892085\\
0.38525390625	85.5203729587791\\
0.3857421875	80.8159032340607\\
0.38623046875	73.423751829981\\
0.38671875	69.5529023773128\\
0.38720703125	82.6264791368355\\
0.3876953125	89.597184237079\\
0.38818359375	90.6535271160912\\
0.388671875	86.3583725093187\\
0.38916015625	77.0804145648831\\
0.3896484375	72.8766040438401\\
0.39013671875	73.7592849244836\\
0.390625	66.590743543712\\
0.39111328125	68.2767053533896\\
0.3916015625	78.0370732908438\\
0.39208984375	78.143610178095\\
0.392578125	63.2979697795697\\
0.39306640625	66.2509983723866\\
0.3935546875	82.4735212967767\\
0.39404296875	85.0369555305084\\
0.39453125	77.3440684882041\\
0.39501953125	37.6079027272651\\
0.3955078125	80.6756743435157\\
0.39599609375	87.5975299214831\\
0.396484375	85.4650758044108\\
0.39697265625	71.3782628601334\\
0.3974609375	62.5618984487526\\
0.39794921875	80.9093944221607\\
0.3984375	82.870087786635\\
0.39892578125	75.059993857255\\
0.3994140625	45.7087054011349\\
0.39990234375	68.5650703883869\\
0.400390625	72.0372055841328\\
0.40087890625	60.7747557652295\\
0.4013671875	63.3776118642351\\
0.40185546875	75.7163587444123\\
0.40234375	76.5625766241949\\
0.40283203125	66.4740805357217\\
0.4033203125	40.1436818772185\\
0.40380859375	67.1290823002669\\
0.404296875	67.1281263918283\\
0.40478515625	49.9395693641628\\
0.4052734375	69.5731980112212\\
0.40576171875	78.2091181679303\\
0.40625	76.3916759778565\\
0.40673828125	59.7394611682598\\
0.4072265625	71.8731315388421\\
0.40771484375	83.6258262817271\\
0.408203125	85.4892045273799\\
0.40869140625	79.1467373973801\\
0.4091796875	50.0070210462559\\
0.40966796875	70.3304315143892\\
0.41015625	78.9718780490448\\
0.41064453125	75.9811046580192\\
0.4111328125	61.2321764452526\\
0.41162109375	69.2361337264978\\
0.412109375	77.2318365020611\\
0.41259765625	74.7757353126271\\
0.4130859375	57.0748446960406\\
0.41357421875	64.2527735325539\\
0.4140625	76.739163345515\\
0.41455078125	79.0683516635332\\
0.4150390625	77.9111557951187\\
0.41552734375	78.3198282180301\\
0.416015625	78.9221421070039\\
0.41650390625	74.7005548880512\\
0.4169921875	59.0153640809679\\
0.41748046875	62.0005933348619\\
0.41796875	73.9275414805711\\
0.41845703125	74.9069247719036\\
0.4189453125	67.7794764844909\\
0.41943359375	47.2993770063357\\
0.419921875	38.7069514219859\\
0.42041015625	32.0815804397294\\
0.4208984375	58.9373936474139\\
0.42138671875	71.7099929816144\\
0.421875	74.8914530308109\\
0.42236328125	69.5462367342383\\
0.4228515625	39.5135546576944\\
0.42333984375	63.1482995444658\\
0.423828125	71.8208082169051\\
0.42431640625	68.6039001759203\\
0.4248046875	60.2278943326019\\
0.42529296875	72.1951985067541\\
0.42578125	77.9904961282132\\
0.42626953125	75.1570413011029\\
0.4267578125	57.9701885219571\\
0.42724609375	64.7963270270515\\
0.427734375	76.8659493171693\\
0.42822265625	77.176427881305\\
0.4287109375	66.3876805905788\\
0.42919921875	50.9288778684288\\
0.4296875	71.5228472692965\\
0.43017578125	73.1527889417052\\
0.4306640625	59.9528083512296\\
0.43115234375	62.9817283536827\\
0.431640625	78.4137971150275\\
0.43212890625	81.880200377222\\
0.4326171875	77.9322230040266\\
0.43310546875	65.6513948333814\\
0.43359375	65.3868891832791\\
0.43408203125	71.9066950840622\\
0.4345703125	67.865071742327\\
0.43505859375	51.6898699308458\\
0.435546875	66.3636208419947\\
0.43603515625	72.6937441898049\\
0.4365234375	68.7960244643648\\
0.43701171875	54.2558971709484\\
0.4375	64.9963333612786\\
0.43798828125	70.892724140662\\
0.4384765625	64.9852777656252\\
0.43896484375	39.7240108169146\\
0.439453125	69.3361395101584\\
0.43994140625	76.3916532478757\\
0.4404296875	74.6083149565758\\
0.44091796875	63.7923152953289\\
0.44140625	58.0008425912404\\
0.44189453125	66.7182776029252\\
0.4423828125	63.8417168277952\\
0.44287109375	57.431143148472\\
0.443359375	71.2976882320073\\
0.44384765625	77.334106790559\\
0.4443359375	75.6431499503387\\
0.44482421875	63.9084403626808\\
0.4453125	37.8874822618782\\
0.44580078125	63.0879660603071\\
0.4462890625	60.4620237137398\\
0.44677734375	36.6504764803799\\
0.447265625	69.8469851731064\\
0.44775390625	77.964042893774\\
0.4482421875	77.6987049936352\\
0.44873046875	69.8226719177583\\
0.44921875	63.539034681925\\
0.44970703125	71.997420359437\\
0.4501953125	72.6579639797875\\
0.45068359375	61.7534136563604\\
0.451171875	58.0748062089695\\
0.45166015625	72.8104859059157\\
0.4521484375	74.8289575079775\\
0.45263671875	65.9664757095299\\
0.453125	36.0418073965452\\
0.45361328125	69.3973058934486\\
0.4541015625	74.0643682580081\\
0.45458984375	68.093594195015\\
0.455078125	30.8314154204655\\
0.45556640625	62.082987570461\\
0.4560546875	67.6679823492681\\
0.45654296875	57.3368368773933\\
0.45703125	58.9357091058442\\
0.45751953125	75.0017172146814\\
0.4580078125	78.8337630851796\\
0.45849609375	74.310686007356\\
0.458984375	55.619377384405\\
0.45947265625	66.0209805400055\\
0.4599609375	75.3990587680606\\
0.46044921875	73.9103213931934\\
0.4609375	59.0869908584688\\
0.46142578125	53.5508246859656\\
0.4619140625	70.4237766501141\\
0.46240234375	71.4633640204484\\
0.462890625	63.8759129620218\\
0.46337890625	62.1849329666682\\
0.4638671875	68.7702530785816\\
0.46435546875	66.28272011884\\
0.46484375	48.4572237500578\\
0.46533203125	64.3380733754222\\
0.4658203125	73.5297893088972\\
0.46630859375	72.4706935295315\\
0.466796875	58.4332034015981\\
0.46728515625	55.2575950840679\\
0.4677734375	71.1580499352917\\
0.46826171875	72.209970545987\\
0.46875	61.2007613452708\\
0.46923828125	56.6956941169131\\
0.4697265625	72.6043674044599\\
0.47021484375	75.6345107908807\\
0.470703125	69.9568308381324\\
0.47119140625	51.0580495719614\\
0.4716796875	61.4030630826871\\
0.47216796875	67.3181145152257\\
0.47265625	60.2007963585195\\
0.47314453125	38.0154955999452\\
0.4736328125	63.6390503813052\\
0.47412109375	67.2610231490959\\
0.474609375	57.0812982266367\\
0.47509765625	48.8481009466265\\
0.4755859375	69.1933928486631\\
0.47607421875	73.0369460392358\\
0.4765625	67.8451222800485\\
0.47705078125	57.9356319639416\\
0.4775390625	68.4388319557861\\
0.47802734375	72.748773352467\\
0.478515625	67.465400156105\\
0.47900390625	51.3190565218491\\
0.4794921875	68.0563053830253\\
0.47998046875	74.7265103026617\\
0.48046875	72.2990819033708\\
0.48095703125	61.1765586058212\\
0.4814453125	66.4486237303438\\
0.48193359375	74.7723415880662\\
0.482421875	74.7659281018014\\
0.48291015625	66.5636547339365\\
0.4833984375	53.5887787448478\\
0.48388671875	62.9202276546752\\
0.484375	62.4856276459706\\
0.48486328125	36.9342212097629\\
0.4853515625	60.2446478536551\\
0.48583984375	71.8690846951131\\
0.486328125	73.1365855850372\\
0.48681640625	66.6626599264256\\
0.4873046875	54.0321825738577\\
0.48779296875	59.5026883533075\\
0.48828125	60.3649321417209\\
0.48876953125	51.456453076803\\
0.4892578125	62.3657173365545\\
0.48974609375	70.5624734520814\\
0.490234375	69.9137349109583\\
0.49072265625	57.291707389026\\
0.4912109375	46.7621700704472\\
0.49169921875	67.1161261586687\\
0.4921875	69.5147515386655\\
0.49267578125	61.9076639048623\\
0.4931640625	55.9533353037616\\
0.49365234375	67.3693944637045\\
0.494140625	68.9469647479016\\
0.49462890625	58.3290406864264\\
0.4951171875	51.7502347371823\\
0.49560546875	70.1285527841811\\
0.49609375	73.9914469927938\\
0.49658203125	69.3034959781322\\
0.4970703125	55.0931871107345\\
0.49755859375	63.0891845195157\\
0.498046875	69.2274603544097\\
0.49853515625	65.0738472510248\\
0.4990234375	52.6512276065712\\
0.49951171875	65.0690707398374\\
0.5	70.914867067158\\
0.50048828125	67.1630380199249\\
0.5009765625	45.4829807732141\\
0.50146484375	58.1183427866855\\
0.501953125	68.0793206763596\\
0.50244140625	65.4696083933043\\
0.5029296875	44.1333224495527\\
0.50341796875	58.3877233615594\\
0.50390625	68.5850615939817\\
0.50439453125	66.7340402112138\\
0.5048828125	47.814601391375\\
0.50537109375	55.5678577736564\\
0.505859375	68.430936666248\\
0.50634765625	68.5491498281047\\
0.5068359375	59.2989211983551\\
0.50732421875	58.8919256653672\\
0.5078125	67.9453538706604\\
0.50830078125	67.5104968645288\\
0.5087890625	54.5933254299915\\
0.50927734375	55.0346922025466\\
0.509765625	68.8111748369304\\
0.51025390625	70.6571629775721\\
0.5107421875	64.0021171217594\\
0.51123046875	54.9937033211338\\
0.51171875	63.9739029874063\\
0.51220703125	66.1496062722545\\
0.5126953125	60.7174045119645\\
0.51318359375	62.1439858420195\\
0.513671875	69.3957289730717\\
0.51416015625	69.4266186964405\\
0.5146484375	58.0725733012019\\
0.51513671875	44.3614465221923\\
0.515625	67.0694263189625\\
0.51611328125	70.4062089652472\\
0.5166015625	63.0599585895351\\
0.51708984375	25.9490402858039\\
0.517578125	63.2651182706057\\
0.51806640625	68.7122054778251\\
0.5185546875	62.6456305594253\\
0.51904296875	42.7112220030353\\
0.51953125	64.5516086385168\\
0.52001953125	70.3998591220777\\
0.5205078125	66.0730137476332\\
0.52099609375	41.719309441095\\
0.521484375	58.1300566455392\\
0.52197265625	66.7136572467618\\
0.5224609375	61.8821795478835\\
0.52294921875	36.867268224973\\
0.5234375	65.3400267752893\\
0.52392578125	72.8051680218795\\
0.5244140625	70.9149751114617\\
0.52490234375	57.1831796836009\\
0.525390625	53.4819864871439\\
0.52587890625	67.5128936229567\\
0.5263671875	67.5631900693025\\
0.52685546875	53.4588766372143\\
0.52734375	49.1574103485303\\
0.52783203125	66.0679172057407\\
0.5283203125	66.6291265205139\\
0.52880859375	53.3727332224033\\
0.529296875	54.8387616360816\\
0.52978515625	69.006246996675\\
0.5302734375	70.7385123951344\\
0.53076171875	62.4753611213943\\
0.53125	38.8966487705345\\
0.53173828125	63.0015457022619\\
0.5322265625	67.3451674544641\\
0.53271484375	60.3009781309252\\
0.533203125	21.8270475396299\\
0.53369140625	59.6459950585669\\
0.5341796875	64.7162472291581\\
0.53466796875	58.3838143034027\\
0.53515625	50.8403088414384\\
0.53564453125	63.4017014377836\\
0.5361328125	65.8348760547666\\
0.53662109375	55.9415115696219\\
0.537109375	40.8836263561729\\
0.53759765625	65.4292201798425\\
0.5380859375	69.6146515383152\\
0.53857421875	63.6093878442633\\
0.5390625	34.7987173235896\\
0.53955078125	61.7193939765603\\
0.5400390625	68.6409616286813\\
0.54052734375	64.6879652804797\\
0.541015625	46.5252222652577\\
0.54150390625	60.9575242709025\\
0.5419921875	68.6978761824497\\
0.54248046875	65.6900244194799\\
0.54296875	43.2160315216121\\
0.54345703125	56.9355924586978\\
0.5439453125	68.3620799625709\\
0.54443359375	67.7754162527597\\
0.544921875	54.9101636467512\\
0.54541015625	42.8575243892536\\
0.5458984375	61.9648578218822\\
0.54638671875	61.9472343851763\\
0.546875	48.163887032946\\
0.54736328125	58.1309127999016\\
0.5478515625	67.9280723997424\\
0.54833984375	67.4072116581676\\
0.548828125	54.952402655397\\
0.54931640625	52.7233037611212\\
0.5498046875	67.2434986826054\\
0.55029296875	68.9565181643639\\
0.55078125	60.0229297056142\\
0.55126953125	27.5730441623433\\
0.5517578125	62.2328663949523\\
0.55224609375	66.7190234625856\\
0.552734375	60.4745997015324\\
0.55322265625	44.3095997690148\\
0.5537109375	60.2011400277824\\
0.55419921875	64.5075857067746\\
0.5546875	57.0716074336834\\
0.55517578125	19.8163903690393\\
0.5556640625	59.8652221637674\\
0.55615234375	65.0273008310414\\
0.556640625	59.0146905738988\\
0.55712890625	31.2253200668251\\
0.5576171875	57.6505701059628\\
0.55810546875	63.4518277570469\\
0.55859375	57.7113145750423\\
0.55908203125	48.3073183876957\\
0.5595703125	62.9094932915617\\
0.56005859375	67.1924141225368\\
0.560546875	61.5585077254802\\
0.56103515625	41.2121300565869\\
0.5615234375	62.2074947533853\\
0.56201171875	69.3197432087217\\
0.5625	67.0111530757895\\
0.56298828125	52.4989013619755\\
0.5634765625	51.6009603167544\\
0.56396484375	64.1738535802963\\
0.564453125	63.6917144486508\\
0.56494140625	49.5190345398594\\
0.5654296875	44.0302520814831\\
0.56591796875	59.7234933864128\\
0.56640625	58.1818862762086\\
0.56689453125	44.074058295476\\
0.5673828125	61.0981383214291\\
0.56787109375	69.756547368521\\
0.568359375	69.9014992565478\\
0.56884765625	61.3224198339702\\
0.5693359375	24.3778249895187\\
0.56982421875	55.9094000366057\\
0.5703125	59.8522388886407\\
0.57080078125	49.375737449852\\
0.5712890625	46.4775280641353\\
0.57177734375	61.688281834492\\
0.572265625	63.6370089583025\\
0.57275390625	55.0360676912708\\
0.5732421875	40.0563269429882\\
0.57373046875	57.8418172667553\\
0.57421875	60.0709231083413\\
0.57470703125	48.1367464869902\\
0.5751953125	44.332494004276\\
0.57568359375	61.6048718468009\\
0.576171875	63.7837716668002\\
0.57666015625	55.7163867971369\\
0.5771484375	50.660319433232\\
0.57763671875	63.4841752535765\\
0.578125	66.0950757980715\\
0.57861328125	58.6922708451621\\
0.5791015625	26.7068356113826\\
0.57958984375	58.3494466852723\\
0.580078125	63.4259276516424\\
0.58056640625	55.9561915865801\\
0.5810546875	32.7064380718892\\
0.58154296875	61.6734953879553\\
0.58203125	67.1209572336196\\
0.58251953125	62.6757778865606\\
0.5830078125	43.823821037422\\
0.58349609375	57.7008322273604\\
0.583984375	65.2090024734997\\
0.58447265625	61.212590157339\\
0.5849609375	28.562374511913\\
0.58544921875	58.3283165874712\\
0.5859375	67.6708089611406\\
0.58642578125	66.2926119989901\\
0.5869140625	51.6185745565877\\
0.58740234375	46.6783018231683\\
0.587890625	63.709922916348\\
0.58837890625	64.7468757405972\\
0.5888671875	53.8006811331166\\
0.58935546875	42.687793387442\\
0.58984375	61.236394006559\\
0.59033203125	63.3679966271387\\
0.5908203125	54.0079705333919\\
0.59130859375	32.3850914699835\\
0.591796875	56.7011058916792\\
0.59228515625	58.6917902444745\\
0.5927734375	45.2443512707616\\
0.59326171875	48.2232124292652\\
0.59375	62.5479543401191\\
0.59423828125	63.9002882883667\\
0.5947265625	53.6692605076679\\
0.59521484375	33.9014859052449\\
0.595703125	59.9858994033447\\
0.59619140625	63.2511238291725\\
0.5966796875	54.4176585692576\\
0.59716796875	30.4327242094931\\
0.59765625	60.3796566471801\\
0.59814453125	65.1274318906309\\
0.5986328125	59.8191548232684\\
0.59912109375	36.9824876599372\\
0.599609375	54.5250258882588\\
0.60009765625	61.43185518892\\
0.6005859375	56.5490456596666\\
0.60107421875	38.7281933579644\\
0.6015625	56.4554192327508\\
0.60205078125	62.3540228585833\\
0.6025390625	57.4488687932636\\
0.60302734375	32.9385473928017\\
0.603515625	54.8894455516197\\
0.60400390625	62.0025699470597\\
0.6044921875	57.6427883400231\\
0.60498046875	31.3698094634859\\
0.60546875	54.5114569998323\\
0.60595703125	62.4108670940338\\
0.6064453125	59.2980845331357\\
0.60693359375	49.6364791119938\\
0.607421875	60.6829434040908\\
0.60791015625	67.0018040661703\\
0.6083984375	64.934502377097\\
0.60888671875	51.0889544165529\\
0.609375	47.018727701237\\
0.60986328125	61.2437067665966\\
0.6103515625	61.2268083913049\\
0.61083984375	47.1495013910313\\
0.611328125	43.2749308645194\\
0.61181640625	58.8563636919537\\
0.6123046875	57.8131100015384\\
0.61279296875	36.5520485375973\\
0.61328125	55.3770323576716\\
0.61376953125	66.2624249969549\\
0.6142578125	66.6615956939505\\
0.61474609375	56.8736352322506\\
0.615234375	44.8586463949584\\
0.61572265625	62.8230214923436\\
0.6162109375	66.4040205440189\\
0.61669921875	60.5756818975759\\
0.6171875	39.6721312194072\\
0.61767578125	54.3921301009911\\
0.6181640625	60.3647461490908\\
0.61865234375	54.3959283987773\\
0.619140625	44.2688616192309\\
0.61962890625	59.3556322568577\\
0.6201171875	63.7640084848388\\
0.62060546875	59.5320495424071\\
0.62109375	51.0277341332708\\
0.62158203125	58.7502114932371\\
0.6220703125	62.5172399133797\\
0.62255859375	56.5407092239835\\
0.623046875	27.7462383045521\\
0.62353515625	55.2694357147336\\
0.6240234375	61.9254965750785\\
0.62451171875	58.1117671964251\\
0.625	50.1550493789801\\
0.62548828125	60.6436423235677\\
0.6259765625	65.6983664700018\\
0.62646484375	62.4252452661939\\
0.626953125	46.5533940633047\\
0.62744140625	49.5316517753369\\
0.6279296875	59.5674079965032\\
0.62841796875	56.124644913528\\
0.62890625	19.5794155387061\\
0.62939453125	55.0470915165227\\
0.6298828125	63.7685838176021\\
0.63037109375	61.4410282492798\\
0.630859375	41.6204656194036\\
0.63134765625	50.8182810820403\\
0.6318359375	63.3924984551223\\
0.63232421875	63.0767549481605\\
0.6328125	49.2694396775271\\
0.63330078125	45.6188489979401\\
0.6337890625	62.4705463470025\\
0.63427734375	64.3156930450863\\
0.634765625	55.5268051963729\\
0.63525390625	19.3396824724098\\
0.6357421875	55.5485640258822\\
0.63623046875	58.9459200567637\\
0.63671875	48.2828452173109\\
0.63720703125	39.7805817778748\\
0.6376953125	58.9654408079857\\
0.63818359375	61.2950205703881\\
0.638671875	51.4365343183221\\
0.63916015625	34.6870655200015\\
0.6396484375	58.8592472242145\\
0.64013671875	62.3446907632255\\
0.640625	54.5511683126972\\
0.64111328125	30.6465908834753\\
0.6416015625	57.7413504642138\\
0.64208984375	62.6025425420634\\
0.642578125	56.683199313912\\
0.64306640625	20.9294992267572\\
0.6435546875	52.9677124784981\\
0.64404296875	58.9365113471193\\
0.64453125	50.8845007575703\\
0.64501953125	33.871890487866\\
0.6455078125	59.8110064306748\\
0.64599609375	64.6408145977485\\
0.646484375	59.2306210659534\\
0.64697265625	27.331269724844\\
0.6474609375	56.8376585259942\\
0.64794921875	64.7433073969266\\
0.6484375	61.8087344849443\\
0.64892578125	42.7053767564614\\
0.6494140625	52.7472661791762\\
0.64990234375	63.3392961940215\\
0.650390625	61.8762993472308\\
0.65087890625	46.5055596494886\\
0.6513671875	51.898824360401\\
0.65185546875	63.4775642663961\\
0.65234375	63.1901083487547\\
0.65283203125	50.1337847446543\\
0.6533203125	45.8285399615937\\
0.65380859375	61.8591159275693\\
0.654296875	63.1951910573838\\
0.65478515625	52.6682105954057\\
0.6552734375	39.7018268005208\\
0.65576171875	60.0677278445188\\
0.65625	62.9151422403024\\
0.65673828125	55.1333534356503\\
0.6572265625	41.1602731432741\\
0.65771484375	57.5960823288954\\
0.658203125	60.3549074563346\\
0.65869140625	49.7907695560241\\
0.6591796875	39.9172990981694\\
0.65966796875	61.1295883507063\\
0.66015625	64.8192913587835\\
0.66064453125	58.5206341082035\\
0.6611328125	16.3391745101625\\
0.66162109375	55.2604404399723\\
0.662109375	62.35183662612\\
0.66259765625	58.0017857837487\\
0.6630859375	28.8343252838171\\
0.66357421875	51.3685370928975\\
0.6640625	59.8573946741524\\
0.66455078125	55.4818089564867\\
0.6650390625	21.4033949074837\\
0.66552734375	54.391970098815\\
0.666015625	62.571360151482\\
0.66650390625	60.3730409930516\\
0.6669921875	45.3981614833407\\
0.66748046875	44.2020631018587\\
0.66796875	55.8198761265222\\
0.66845703125	52.378940214865\\
0.6689453125	30.7782745356677\\
0.66943359375	55.1887221958616\\
0.669921875	62.6880367074731\\
0.67041015625	60.323470098385\\
0.6708984375	44.8741708987693\\
0.67138671875	49.5002451650854\\
0.671875	60.2797456785072\\
0.67236328125	58.6758845907988\\
0.6728515625	38.5445822393303\\
0.67333984375	49.8675979343441\\
0.673828125	62.4726779248382\\
0.67431640625	62.796586604748\\
0.6748046875	51.6014734781842\\
0.67529296875	34.8324933433132\\
0.67578125	57.5676286334583\\
0.67626953125	59.6629890581053\\
0.6767578125	48.7060386033036\\
0.67724609375	40.2861914930889\\
0.677734375	58.6484009621248\\
0.67822265625	60.8751419630401\\
0.6787109375	51.4663245018901\\
0.67919921875	38.204263793814\\
0.6796875	58.1996548038442\\
0.68017578125	61.4088545600714\\
0.6806640625	53.4592757105876\\
0.68115234375	31.4530776425919\\
0.681640625	56.7021738961657\\
0.68212890625	61.2360311058381\\
0.6826171875	55.1482963305181\\
0.68310546875	43.4779313205342\\
0.68359375	58.2533887971763\\
0.68408203125	62.7439053638468\\
0.6845703125	57.2339113222582\\
0.68505859375	23.3482970342388\\
0.685546875	53.5043445949778\\
0.68603515625	61.17415246249\\
0.6865234375	57.2832668329585\\
0.68701171875	32.2776194846682\\
0.6875	51.5362889617035\\
0.68798828125	60.3243364910978\\
0.6884765625	57.2425505346065\\
0.68896484375	40.271883095396\\
0.689453125	53.9313354011432\\
0.68994140625	62.197685154857\\
0.6904296875	60.1675321266771\\
0.69091796875	44.3687846086232\\
0.69140625	47.1880563768792\\
0.69189453125	59.1676314926302\\
0.6923828125	57.8650204574791\\
0.69287109375	44.545874317914\\
0.693359375	55.2933306076539\\
0.69384765625	64.6516634734891\\
0.6943359375	64.7645285876854\\
0.69482421875	55.0834163841451\\
0.6953125	33.8816155375995\\
0.69580078125	56.823034281023\\
0.6962890625	60.0334412862863\\
0.69677734375	51.4110568679043\\
0.697265625	34.7746603196431\\
0.69775390625	55.792523288238\\
0.6982421875	59.0066814259203\\
0.69873046875	50.7620245530001\\
0.69921875	41.5733439364071\\
0.69970703125	57.7579865594123\\
0.7001953125	61.1079032165951\\
0.70068359375	54.3768734573656\\
0.701171875	23.6974729209269\\
0.70166015625	51.9497742724675\\
0.7021484375	57.4731139663445\\
0.70263671875	50.4496535959716\\
0.703125	24.7122359493008\\
0.70361328125	53.9511478832296\\
0.7041015625	58.9540560296208\\
0.70458984375	52.9849776528344\\
0.705078125	38.0723315032035\\
0.70556640625	55.5903897128827\\
0.7060546875	60.9151719249576\\
0.70654296875	56.3997151040349\\
0.70703125	32.9163333216362\\
0.70751953125	49.0121902823783\\
0.7080078125	57.3062041540666\\
0.70849609375	52.8306311730227\\
0.708984375	26.9226333611761\\
0.70947265625	52.8181553687749\\
0.7099609375	60.1258609819319\\
0.71044921875	56.7640907650083\\
0.7109375	38.7388338986276\\
0.71142578125	51.9886343727816\\
0.7119140625	60.9580188903601\\
0.71240234375	59.6079718054186\\
0.712890625	46.4233744724327\\
0.71337890625	41.257454985805\\
0.7138671875	54.683685875148\\
0.71435546875	52.8388107581331\\
0.71484375	27.4277533571567\\
0.71533203125	50.5724191427189\\
0.7158203125	60.3546436272188\\
0.71630859375	59.0347979613922\\
0.716796875	42.9791666523841\\
0.71728515625	43.7908857064233\\
0.7177734375	58.8836732823901\\
0.71826171875	59.4020026371746\\
0.71875	47.9664896600261\\
0.71923828125	46.8018565832892\\
0.7197265625	60.1034901697095\\
0.72021484375	61.5552971502551\\
0.720703125	52.4383803354683\\
0.72119140625	38.9325610726218\\
0.7216796875	57.4583642787067\\
0.72216796875	60.7855363686942\\
0.72265625	53.4424214011822\\
0.72314453125	30.1271174627159\\
0.7236328125	53.8510097091248\\
0.72412109375	58.2877827916292\\
0.724609375	51.1079638428769\\
0.72509765625	43.5414225244646\\
0.7255859375	58.6553222825536\\
0.72607421875	62.4538840103362\\
0.7265625	56.7499623012081\\
0.72705078125	25.5869815296914\\
0.7275390625	51.3714334257063\\
0.72802734375	58.7952054615155\\
0.728515625	53.8004411632546\\
0.72900390625	30.665245500945\\
0.7294921875	54.6443584906452\\
0.72998046875	61.5121696956571\\
0.73046875	58.1383894123621\\
0.73095703125	38.8538318381961\\
0.7314453125	49.5164688417723\\
0.73193359375	59.2543839562631\\
0.732421875	56.8768625351875\\
0.73291015625	36.9705406827446\\
0.7333984375	48.4802611524919\\
0.73388671875	59.1119807498592\\
0.734375	57.2714697883483\\
0.73486328125	37.750651929991\\
0.7353515625	48.0249752037821\\
0.73583984375	60.0857172946022\\
0.736328125	59.7588101275876\\
0.73681640625	47.3679188452454\\
0.7373046875	44.7448483805801\\
0.73779296875	58.4985340217933\\
0.73828125	59.2601462873061\\
0.73876953125	48.8544671705806\\
0.7392578125	46.1438745568327\\
0.73974609375	58.7877999822202\\
0.740234375	59.8269557730086\\
0.74072265625	48.912728705962\\
0.7412109375	37.6392620881167\\
0.74169921875	57.3636883784714\\
0.7421875	59.770571343287\\
0.74267578125	50.0038152108121\\
0.7431640625	30.6066144201004\\
0.74365234375	56.8734769341939\\
0.744140625	60.5998603654208\\
0.74462890625	53.5614283303735\\
0.7451171875	40.1607110291498\\
0.74560546875	57.2199125877655\\
0.74609375	61.5881322067954\\
0.74658203125	55.7443330359176\\
0.7470703125	21.4086267525882\\
0.74755859375	52.4346453197918\\
0.748046875	59.3819018472302\\
0.74853515625	54.2786975931851\\
0.7490234375	10.7367394535698\\
0.74951171875	52.8948953568873\\
0.75	60.3004381250721\\
0.75048828125	56.3488993371626\\
0.7509765625	33.6824532009534\\
0.75146484375	52.892579604711\\
0.751953125	61.4984934095417\\
0.75244140625	59.4634505789028\\
0.7529296875	43.8612776370392\\
0.75341796875	46.2089461046711\\
0.75390625	58.1271821442059\\
0.75439453125	56.6176381934212\\
0.7548828125	35.8942137506633\\
0.75537109375	48.1767417550554\\
0.755859375	60.4378737289842\\
0.75634765625	60.3454853093869\\
0.7568359375	47.7109359606287\\
0.75732421875	37.2351505131876\\
0.7578125	57.1364838621161\\
0.75830078125	58.9694067598584\\
0.7587890625	49.3776605731727\\
0.75927734375	41.935487805636\\
0.759765625	56.8936987777236\\
0.76025390625	58.9455561996753\\
0.7607421875	50.1849775766346\\
0.76123046875	39.194269933017\\
0.76171875	55.2469654754021\\
0.76220703125	57.5352847406702\\
0.7626953125	47.6592619018297\\
0.76318359375	40.206238129791\\
0.763671875	57.1747214889305\\
0.76416015625	59.6487142180271\\
0.7646484375	50.2989725317252\\
0.76513671875	30.0239703017812\\
0.765625	57.4389019900554\\
0.76611328125	61.9792302935097\\
0.7666015625	56.3479284107656\\
0.76708984375	27.7333790059351\\
0.767578125	52.9981379602383\\
0.76806640625	60.5292768011542\\
0.7685546875	57.1831916420622\\
0.76904296875	38.3770619835405\\
0.76953125	48.7092390723386\\
0.77001953125	57.9663677854165\\
0.7705078125	55.2070899521835\\
0.77099609375	33.260534914269\\
0.771484375	46.4587549627357\\
0.77197265625	56.7664961041277\\
0.7724609375	54.1536924735554\\
0.77294921875	34.8392767392346\\
0.7734375	49.9962124676037\\
0.77392578125	59.1427137030912\\
0.7744140625	57.3302805758982\\
0.77490234375	39.9378350769848\\
0.775390625	43.5612334807971\\
0.77587890625	57.3522753490763\\
0.7763671875	57.1728807698891\\
0.77685546875	43.2772950376118\\
0.77734375	39.5636261844844\\
0.77783203125	55.5224847990584\\
0.7783203125	55.8067776973477\\
0.77880859375	40.8062066452456\\
0.779296875	40.9814146856853\\
0.77978515625	56.9018492407611\\
0.7802734375	58.0331234829075\\
0.78076171875	47.2856338424429\\
0.78125	36.6037936269305\\
0.78173828125	55.1326628383632\\
0.7822265625	57.2087861677124\\
0.78271484375	47.012852039796\\
0.783203125	36.5023690357796\\
0.78369140625	55.5603813494673\\
0.7841796875	58.2320663783135\\
0.78466796875	49.5613702141368\\
0.78515625	37.2056183224373\\
0.78564453125	56.2148120166207\\
0.7861328125	60.1316607766106\\
0.78662109375	54.3690985044483\\
0.787109375	38.5562893635953\\
0.78759765625	52.8798079426582\\
0.7880859375	58.0026282491481\\
0.78857421875	52.1224517587992\\
0.7890625	25.2200262419031\\
0.78955078125	52.2320701725928\\
0.7900390625	58.6014755958152\\
0.79052734375	54.0380337524034\\
0.791015625	27.9827644309688\\
0.79150390625	50.1805660089406\\
0.7919921875	58.383247854756\\
0.79248046875	55.5876590161722\\
0.79296875	39.0769357567977\\
0.79345703125	46.8430642630583\\
0.7939453125	55.8565830216521\\
0.79443359375	52.714268175345\\
0.794921875	32.7327874313951\\
0.79541015625	50.080109392956\\
0.7958984375	58.394162953389\\
0.79638671875	55.782504386364\\
0.796875	36.2100820753291\\
0.79736328125	48.0159983812299\\
0.7978515625	58.9472673536995\\
0.79833984375	58.2864612236987\\
0.798828125	46.1088572837873\\
0.79931640625	44.9475088813185\\
0.7998046875	57.1567915834124\\
0.80029296875	57.2215983515111\\
0.80078125	44.1091836409895\\
0.80126953125	42.7870676483802\\
0.8017578125	57.6476174716065\\
0.80224609375	59.0701587895892\\
0.802734375	50.2574120491083\\
0.80322265625	44.1962973099881\\
0.8037109375	57.3095329535644\\
0.80419921875	59.6528553290896\\
0.8046875	52.2950167956601\\
0.80517578125	38.8642738555632\\
0.8056640625	53.6404671035249\\
0.80615234375	56.6372130153308\\
0.806640625	46.7618573056765\\
0.80712890625	34.1378174238037\\
0.8076171875	56.9976976446544\\
0.80810546875	61.3705562689678\\
0.80859375	56.7280442851695\\
0.80908203125	41.4764204563539\\
0.8095703125	50.125735197407\\
0.81005859375	56.105178580317\\
0.810546875	50.1559859882716\\
0.81103515625	23.9483034752848\\
0.8115234375	52.9762972284431\\
0.81201171875	59.1688165794477\\
0.8125	55.0068435394703\\
0.81298828125	33.3702792023252\\
0.8134765625	48.9877382411828\\
0.81396484375	57.3205817992888\\
0.814453125	53.8203648650737\\
0.81494140625	35.3704450476135\\
0.8154296875	51.857763826913\\
0.81591796875	59.8915041088045\\
0.81640625	57.7402036207182\\
0.81689453125	43.0995657293421\\
0.8173828125	48.6078563620422\\
0.81787109375	59.0616476307613\\
0.818359375	58.362937480568\\
0.81884765625	44.5553761738423\\
0.8193359375	42.2318056771906\\
0.81982421875	57.6561974916389\\
0.8203125	59.0516157955461\\
0.82080078125	49.5003111071294\\
0.8212890625	20.3521540567742\\
0.82177734375	50.9666026738734\\
0.822265625	53.4694523642718\\
0.82275390625	41.9043419545123\\
0.8232421875	42.5160447171166\\
0.82373046875	56.3553741311991\\
0.82421875	57.7481436284259\\
0.82470703125	47.8284618903961\\
0.8251953125	33.7315791191501\\
0.82568359375	54.3568290318361\\
0.826171875	57.2547392504058\\
0.82666015625	48.1448070498159\\
0.8271484375	27.977555820669\\
0.82763671875	54.5629653847049\\
0.828125	58.911648117791\\
0.82861328125	53.2020794097192\\
0.8291015625	37.2510356230004\\
0.82958984375	51.9988271894758\\
0.830078125	57.0793846071514\\
0.83056640625	51.078007619649\\
0.8310546875	29.4353299725857\\
0.83154296875	52.8333748506597\\
0.83203125	59.0480508049702\\
0.83251953125	55.3348295895813\\
0.8330078125	37.8765856986432\\
0.83349609375	47.669463568278\\
0.833984375	56.0545286993953\\
0.83447265625	52.9592293470138\\
0.8349609375	36.2456808478305\\
0.83544921875	48.8470206700534\\
0.8359375	56.7441863765519\\
0.83642578125	53.8779110197028\\
0.8369140625	36.8610666025875\\
0.83740234375	47.8022298494352\\
0.837890625	56.6420056566601\\
0.83837890625	54.1341409927289\\
0.8388671875	33.9594446596194\\
0.83935546875	46.4926094745846\\
0.83984375	57.1290175142254\\
0.84033203125	55.6977014469591\\
0.8408203125	40.1099443547011\\
0.84130859375	46.5719086392043\\
0.841796875	58.0590506967182\\
0.84228515625	57.799908292314\\
0.8427734375	44.5470312530222\\
0.84326171875	39.3008616248875\\
0.84375	56.3838766621473\\
0.84423828125	57.858898114234\\
0.8447265625	47.5697074441648\\
0.84521484375	38.4807400703254\\
0.845703125	55.8323158991894\\
0.84619140625	58.2444493859075\\
0.8466796875	49.0762380513162\\
0.84716796875	24.4613971337424\\
0.84765625	54.2278999001133\\
0.84814453125	58.5458013965374\\
0.8486328125	52.2663522712794\\
0.84912109375	28.6129019365405\\
0.849609375	50.342363944172\\
0.85009765625	55.5925152205242\\
0.8505859375	48.1165752510922\\
0.85107421875	20.5554689615695\\
0.8515625	53.8274960535374\\
0.85205078125	59.4224211782826\\
0.8525390625	55.1481563620925\\
0.85302734375	34.9263739442567\\
0.853515625	47.806430522795\\
0.85400390625	56.2852135513751\\
0.8544921875	53.1274448995444\\
0.85498046875	34.2677019774362\\
0.85546875	46.5768344577154\\
0.85595703125	54.9763564880975\\
0.8564453125	51.2191457885018\\
0.85693359375	24.6726221712806\\
0.857421875	48.3487411523812\\
0.85791015625	56.5599597602185\\
0.8583984375	53.0540677286146\\
0.85888671875	30.2457283979124\\
0.859375	50.8042960787726\\
0.85986328125	59.9436208622953\\
0.8603515625	58.8507865160712\\
0.86083984375	45.7681334370733\\
0.861328125	38.8498284429498\\
0.86181640625	55.3199250099371\\
0.8623046875	56.3309503882954\\
0.86279296875	45.2481666764759\\
0.86328125	38.5405648809359\\
0.86376953125	53.7971165007171\\
0.8642578125	54.4928296688578\\
0.86474609375	40.4224473709763\\
0.865234375	41.6559040288686\\
0.86572265625	56.9450653120085\\
0.8662109375	58.6143813576012\\
0.86669921875	49.3969348599845\\
0.8671875	28.4896757318356\\
0.86767578125	53.6515904172449\\
0.8681640625	57.6559368390684\\
0.86865234375	50.8458852962468\\
0.869140625	23.9123104840939\\
0.86962890625	50.0234713540816\\
0.8701171875	55.2175749308394\\
0.87060546875	48.586844512757\\
0.87109375	29.2637187875442\\
0.87158203125	51.2237239617884\\
0.8720703125	56.1398441112359\\
0.87255859375	50.6950377632013\\
0.873046875	39.7805944452816\\
0.87353515625	52.7581995569812\\
0.8740234375	56.8730590216271\\
0.87451171875	50.1080851485299\\
0.875	25.556339547249\\
0.87548828125	53.9221607264814\\
0.8759765625	60.2955495505245\\
0.87646484375	57.1362543762058\\
0.876953125	39.9221033086327\\
0.87744140625	45.4975198215686\\
0.8779296875	56.471678094049\\
0.87841796875	55.2623655916051\\
0.87890625	42.0740593905369\\
0.87939453125	45.1419825845073\\
0.8798828125	55.3881241156607\\
0.88037109375	54.3197343327254\\
0.880859375	39.8044867997201\\
0.88134765625	41.4298664626898\\
0.8818359375	53.5465957748819\\
0.88232421875	52.1052540062562\\
0.8828125	30.4549034866642\\
0.88330078125	44.9714905487449\\
0.8837890625	56.8390045314416\\
0.88427734375	56.5192515441822\\
0.884765625	43.2995749156726\\
0.88525390625	40.0515807699049\\
0.8857421875	55.9571626858881\\
0.88623046875	57.6295842207152\\
0.88671875	48.8401856370156\\
0.88720703125	37.7525811824899\\
0.8876953125	53.4161521689305\\
0.88818359375	55.9453944074399\\
0.888671875	46.747122261753\\
0.88916015625	25.4050070302493\\
0.8896484375	51.7949457206458\\
0.89013671875	55.2448364567476\\
0.890625	46.5402944138648\\
0.89111328125	32.2564968907961\\
0.8916015625	53.6244677946504\\
0.89208984375	57.6396771614726\\
0.892578125	51.4231921428708\\
0.89306640625	30.5877951223842\\
0.8935546875	50.802548565241\\
0.89404296875	56.6209833472496\\
0.89453125	51.878415728587\\
0.89501953125	33.8349964567216\\
0.8955078125	49.2106183413267\\
0.89599609375	55.6751000014021\\
0.896484375	51.2521131004214\\
0.89697265625	29.9428045384473\\
0.8974609375	48.1199326547603\\
0.89794921875	55.4125835681151\\
0.8984375	51.4299853774601\\
0.89892578125	27.6654339108595\\
0.8994140625	46.7868981548524\\
0.89990234375	54.7952334195619\\
0.900390625	50.5824657545643\\
0.90087890625	30.5856056086611\\
0.9013671875	51.987590387489\\
0.90185546875	59.5921205047397\\
0.90234375	57.5146260393235\\
0.90283203125	41.4249775609506\\
0.9033203125	40.4388773179097\\
0.90380859375	55.5281874740219\\
0.904296875	55.5994935273121\\
0.90478515625	42.3679379522857\\
0.9052734375	41.5848404899827\\
0.90576171875	55.6284753021904\\
0.90625	56.3765888327907\\
0.90673828125	45.2664250240157\\
0.9072265625	39.3967970627028\\
0.90771484375	54.9049501701691\\
0.908203125	56.5539604980364\\
0.90869140625	46.5777562127638\\
0.9091796875	37.8336115287087\\
0.90966796875	55.3307158620561\\
0.91015625	58.5205430255728\\
0.91064453125	52.1104282171718\\
0.9111328125	34.0254461621899\\
0.91162109375	49.9517161600543\\
0.912109375	54.8312071302776\\
0.91259765625	48.2576753501859\\
0.9130859375	19.3600451848867\\
0.91357421875	48.5071113562976\\
0.9140625	53.6133410139346\\
0.91455078125	46.5901352671491\\
0.9150390625	29.6396157198268\\
0.91552734375	51.1007490992783\\
0.916015625	55.5370925370013\\
0.91650390625	48.7768866306368\\
0.9169921875	24.8083680944277\\
0.91748046875	51.6039245391751\\
0.91796875	57.3571317412287\\
0.91845703125	52.7070943076116\\
0.9189453125	29.994967435579\\
0.91943359375	48.8337292445075\\
0.919921875	56.7702734439299\\
0.92041015625	53.8173395227466\\
0.9208984375	36.0149564145193\\
0.92138671875	45.4901005444441\\
0.921875	54.9007875975951\\
0.92236328125	52.2351739501846\\
0.9228515625	29.6649708450413\\
0.92333984375	45.0355065642554\\
0.923828125	55.3500609634261\\
0.92431640625	53.3925561976794\\
0.9248046875	36.8076419460423\\
0.92529296875	46.5208447255563\\
0.92578125	56.6096055029422\\
0.92626953125	55.3236387787002\\
0.9267578125	39.4759257857553\\
0.92724609375	43.4153908200508\\
0.927734375	56.5130040422184\\
0.92822265625	56.5930486393485\\
0.9287109375	43.2722785075131\\
0.92919921875	39.7446436625875\\
0.9296875	56.5740285912342\\
0.93017578125	58.5442321184873\\
0.9306640625	49.6521980361008\\
0.93115234375	21.6137142094722\\
0.931640625	52.2659158300967\\
0.93212890625	56.4592405808957\\
0.9326171875	49.5006765268705\\
0.93310546875	34.4613621666478\\
0.93359375	52.3530086911459\\
0.93408203125	56.8148046131265\\
0.9345703125	51.4273548650589\\
0.93505859375	37.1988382402091\\
0.935546875	49.5661266870821\\
0.93603515625	53.912628859697\\
0.9365234375	46.1866153091389\\
0.93701171875	26.1320680134492\\
0.9375	52.2090639278373\\
0.93798828125	57.0650840553271\\
0.9384765625	51.7368831706263\\
0.93896484375	37.567885522\\
0.939453125	52.4498912045414\\
0.93994140625	58.0492116521947\\
0.9404296875	53.7251596051564\\
0.94091796875	30.6225137258101\\
0.94140625	47.9674983794002\\
0.94189453125	56.634398092533\\
0.9423828125	53.7309363093136\\
0.94287109375	33.4693071325028\\
0.943359375	46.3375297073635\\
0.94384765625	56.2704053762278\\
0.9443359375	54.245052493009\\
0.94482421875	36.0303062013504\\
0.9453125	45.3684059736904\\
0.94580078125	56.7369943283008\\
0.9462890625	56.1683158415783\\
0.94677734375	43.4568701157208\\
0.947265625	40.8337324041267\\
0.94775390625	54.1761701116264\\
0.9482421875	53.9237482197664\\
0.94873046875	37.9885097537022\\
0.94921875	42.9744280587303\\
0.94970703125	56.8564869867101\\
0.9501953125	57.9119529315748\\
0.95068359375	48.2071839461221\\
0.951171875	33.6206722291856\\
0.95166015625	52.7827594585524\\
0.9521484375	55.2840733615383\\
0.95263671875	45.8979248997389\\
0.953125	40.0259900119771\\
};
\addplot [color=blue,solid]
  table[row sep=crcr]{0.953125	40.0259900119771\\
0.95361328125	55.4334584178517\\
0.9541015625	58.2128232206991\\
0.95458984375	51.0536802675657\\
0.955078125	30.2592375738934\\
0.95556640625	50.7926700560077\\
0.9560546875	55.3959763884952\\
0.95654296875	48.0364260753046\\
0.95703125	16.2200081486803\\
0.95751953125	51.0958882022726\\
0.9580078125	56.1377905200799\\
0.95849609375	49.8460304317489\\
0.958984375	30.3763115541596\\
0.95947265625	52.1438528800313\\
0.9599609375	57.6877510236431\\
0.96044921875	52.7733204273626\\
0.9609375	26.350840439487\\
0.96142578125	48.7733983137998\\
0.9619140625	56.4215982847581\\
0.96240234375	52.2327510944812\\
0.962890625	24.0610445408795\\
0.96337890625	49.8447072298583\\
0.9638671875	58.3044829532781\\
0.96435546875	55.8960558715602\\
0.96484375	36.9881384521847\\
0.96533203125	43.1040641071204\\
0.9658203125	55.8540758471102\\
0.96630859375	54.8811036378002\\
0.966796875	39.6962699780208\\
0.96728515625	45.3122326367133\\
0.9677734375	57.0839501982978\\
0.96826171875	57.076419692148\\
0.96875	44.9330708563706\\
0.96923828125	35.1625579003576\\
0.9697265625	53.3999674293199\\
0.97021484375	54.5274221208016\\
0.970703125	42.0975462180886\\
0.97119140625	40.3170173429863\\
0.9716796875	55.3600112796424\\
0.97216796875	56.6516607257251\\
0.97265625	45.9043301080298\\
0.97314453125	35.7426627244779\\
0.9736328125	54.9566091353663\\
0.97412109375	57.7677265574538\\
0.974609375	49.6340069704897\\
0.97509765625	27.3650366373428\\
0.9755859375	52.8194891005743\\
0.97607421875	57.5954545930468\\
0.9765625	52.3969160248261\\
0.97705078125	39.2120511712081\\
0.9775390625	50.8318975161185\\
0.97802734375	55.257750798976\\
0.978515625	48.3137825385711\\
0.97900390625	23.5738082282284\\
0.9794921875	51.8055566530012\\
0.97998046875	57.3301426523455\\
0.98046875	52.5286604874996\\
0.98095703125	35.8761184962632\\
0.9814453125	50.8461106528318\\
0.98193359375	57.2559639040886\\
0.982421875	53.2614676292987\\
0.98291015625	30.4933012359776\\
0.9833984375	47.0597099147624\\
0.98388671875	55.7166052687792\\
0.984375	52.2417330559808\\
0.98486328125	27.3727875491823\\
0.9853515625	48.4923309549828\\
0.98583984375	57.529461195891\\
0.986328125	55.4051519870759\\
0.98681640625	37.9107456536076\\
0.9873046875	44.0779898872455\\
0.98779296875	55.9400579070192\\
0.98828125	54.6718449948536\\
0.98876953125	36.7783851608077\\
0.9892578125	45.6964430313352\\
0.98974609375	58.1942794701978\\
0.990234375	58.6697652649785\\
0.99072265625	48.3005753350028\\
0.9912109375	37.3435825682477\\
0.99169921875	54.4740243136231\\
0.9921875	56.563042542526\\
0.99267578125	46.5826841726847\\
0.9931640625	32.1602005392821\\
0.99365234375	53.8502075854728\\
0.994140625	57.0900964830545\\
0.99462890625	49.5814448690386\\
0.9951171875	31.4533057049926\\
0.99560546875	51.4322599340343\\
0.99609375	55.4744048311516\\
0.99658203125	47.8615754958149\\
0.9970703125	20.0468058851133\\
0.99755859375	50.6950691583429\\
0.998046875	55.3684151402721\\
0.99853515625	48.6729390744488\\
0.9990234375	35.6293272090586\\
0.99951171875	53.3349444554309\\
};
\addlegendentry{Measured PSD};

\addplot [color=red,solid,forget plot]
  table[row sep=crcr]{-1	22.5717014495571\\
-0.99951171875	22.5717106255247\\
-0.9990234375	22.5717381534366\\
-0.99853515625	22.5717840333203\\
-0.998046875	22.571848265222\\
-0.99755859375	22.5719308492057\\
-0.9970703125	22.5720317853544\\
-0.99658203125	22.572151073769\\
-0.99609375	22.572288714569\\
-0.99560546875	22.5724447078923\\
-0.9951171875	22.5726190538952\\
-0.99462890625	22.5728117527522\\
-0.994140625	22.5730228046564\\
-0.99365234375	22.5732522098191\\
-0.9931640625	22.57349996847\\
-0.99267578125	22.5737660808575\\
-0.9921875	22.574050547248\\
-0.99169921875	22.5743533679264\\
-0.9912109375	22.5746745431962\\
-0.99072265625	22.5750140733789\\
-0.990234375	22.5753719588148\\
-0.98974609375	22.5757481998624\\
-0.9892578125	22.5761427968984\\
-0.98876953125	22.5765557503184\\
-0.98828125	22.5769870605359\\
-0.98779296875	22.5774367279832\\
-0.9873046875	22.5779047531107\\
-0.98681640625	22.5783911363873\\
-0.986328125	22.5788958783004\\
-0.98583984375	22.5794189793558\\
-0.9853515625	22.5799604400775\\
-0.98486328125	22.5805202610083\\
-0.984375	22.5810984427091\\
-0.98388671875	22.5816949857593\\
-0.9833984375	22.5823098907568\\
-0.98291015625	22.582943158318\\
-0.982421875	22.5835947890774\\
-0.98193359375	22.5842647836882\\
-0.9814453125	22.5849531428221\\
-0.98095703125	22.5856598671691\\
-0.98046875	22.5863849574377\\
-0.97998046875	22.5871284143547\\
-0.9794921875	22.5878902386657\\
-0.97900390625	22.5886704311343\\
-0.978515625	22.5894689925429\\
-0.97802734375	22.5902859236922\\
-0.9775390625	22.5911212254016\\
-0.97705078125	22.5919748985087\\
-0.9765625	22.5928469438696\\
-0.97607421875	22.593737362359\\
-0.9755859375	22.5946461548702\\
-0.97509765625	22.5955733223146\\
-0.974609375	22.5965188656224\\
-0.97412109375	22.5974827857423\\
-0.9736328125	22.5984650836414\\
-0.97314453125	22.5994657603053\\
-0.97265625	22.6004848167381\\
-0.97216796875	22.6015222539625\\
-0.9716796875	22.6025780730198\\
-0.97119140625	22.6036522749695\\
-0.970703125	22.6047448608901\\
-0.97021484375	22.6058558318781\\
-0.9697265625	22.6069851890491\\
-0.96923828125	22.6081329335368\\
-0.96875	22.6092990664938\\
-0.96826171875	22.6104835890909\\
-0.9677734375	22.6116865025179\\
-0.96728515625	22.6129078079827\\
-0.966796875	22.6141475067121\\
-0.96630859375	22.6154055999515\\
-0.9658203125	22.6166820889646\\
-0.96533203125	22.6179769750341\\
-0.96484375	22.619290259461\\
-0.96435546875	22.6206219435651\\
-0.9638671875	22.6219720286845\\
-0.96337890625	22.6233405161763\\
-0.962890625	22.6247274074161\\
-0.96240234375	22.6261327037981\\
-0.9619140625	22.6275564067351\\
-0.96142578125	22.6289985176586\\
-0.9609375	22.6304590380189\\
-0.96044921875	22.6319379692847\\
-0.9599609375	22.6334353129437\\
-0.95947265625	22.6349510705018\\
-0.958984375	22.6364852434841\\
-0.95849609375	22.6380378334342\\
-0.9580078125	22.6396088419142\\
-0.95751953125	22.6411982705053\\
-0.95703125	22.642806120807\\
-0.95654296875	22.6444323944379\\
-0.9560546875	22.6460770930352\\
-0.95556640625	22.6477402182547\\
-0.955078125	22.6494217717711\\
-0.95458984375	22.6511217552779\\
-0.9541015625	22.6528401704873\\
-0.95361328125	22.6545770191302\\
-0.953125	22.6563323029565\\
-0.95263671875	22.6581060237347\\
-0.9521484375	22.6598981832522\\
-0.95166015625	22.6617087833151\\
-0.951171875	22.6635378257485\\
-0.95068359375	22.6653853123962\\
-0.9501953125	22.667251245121\\
-0.94970703125	22.6691356258043\\
-0.94921875	22.6710384563465\\
-0.94873046875	22.672959738667\\
-0.9482421875	22.6748994747038\\
-0.94775390625	22.676857666414\\
-0.947265625	22.6788343157736\\
-0.94677734375	22.6808294247773\\
-0.9462890625	22.6828429954391\\
-0.94580078125	22.6848750297915\\
-0.9453125	22.6869255298862\\
-0.94482421875	22.6889944977939\\
-0.9443359375	22.691081935604\\
-0.94384765625	22.6931878454251\\
-0.943359375	22.6953122293847\\
-0.94287109375	22.6974550896293\\
-0.9423828125	22.6996164283244\\
-0.94189453125	22.7017962476546\\
-0.94140625	22.7039945498235\\
-0.94091796875	22.7062113370535\\
-0.9404296875	22.7084466115864\\
-0.93994140625	22.7107003756829\\
-0.939453125	22.7129726316227\\
-0.93896484375	22.7152633817047\\
-0.9384765625	22.7175726282469\\
-0.93798828125	22.7199003735865\\
-0.9375	22.7222466200794\\
-0.93701171875	22.7246113701011\\
-0.9365234375	22.7269946260462\\
-0.93603515625	22.729396390328\\
-0.935546875	22.7318166653796\\
-0.93505859375	22.7342554536529\\
-0.9345703125	22.7367127576191\\
-0.93408203125	22.7391885797685\\
-0.93359375	22.7416829226108\\
-0.93310546875	22.7441957886747\\
-0.9326171875	22.7467271805085\\
-0.93212890625	22.7492771006795\\
-0.931640625	22.7518455517742\\
-0.93115234375	22.7544325363987\\
-0.9306640625	22.7570380571781\\
-0.93017578125	22.7596621167569\\
-0.9296875	22.762304717799\\
-0.92919921875	22.7649658629877\\
-0.9287109375	22.7676455550255\\
-0.92822265625	22.7703437966342\\
-0.927734375	22.7730605905553\\
-0.92724609375	22.7757959395495\\
-0.9267578125	22.7785498463967\\
-0.92626953125	22.7813223138968\\
-0.92578125	22.7841133448685\\
-0.92529296875	22.7869229421504\\
-0.9248046875	22.7897511086004\\
-0.92431640625	22.7925978470959\\
-0.923828125	22.7954631605339\\
-0.92333984375	22.7983470518307\\
-0.9228515625	22.8012495239224\\
-0.92236328125	22.8041705797644\\
-0.921875	22.8071102223319\\
-0.92138671875	22.8100684546195\\
-0.9208984375	22.8130452796414\\
-0.92041015625	22.8160407004316\\
-0.919921875	22.8190547200435\\
-0.91943359375	22.8220873415503\\
-0.9189453125	22.8251385680448\\
-0.91845703125	22.8282084026394\\
-0.91796875	22.8312968484663\\
-0.91748046875	22.8344039086775\\
-0.9169921875	22.8375295864444\\
-0.91650390625	22.8406738849586\\
-0.916015625	22.843836807431\\
-0.91552734375	22.8470183570926\\
-0.9150390625	22.8502185371941\\
-0.91455078125	22.8534373510059\\
-0.9140625	22.8566748018184\\
-0.91357421875	22.8599308929418\\
-0.9130859375	22.8632056277061\\
-0.91259765625	22.8664990094612\\
-0.912109375	22.8698110415769\\
-0.91162109375	22.8731417274431\\
-0.9111328125	22.8764910704693\\
-0.91064453125	22.8798590740852\\
-0.91015625	22.8832457417405\\
-0.90966796875	22.8866510769046\\
-0.9091796875	22.8900750830674\\
-0.90869140625	22.8935177637382\\
-0.908203125	22.896979122447\\
-0.90771484375	22.9004591627434\\
-0.9072265625	22.9039578881973\\
-0.90673828125	22.9074753023987\\
-0.90625	22.9110114089575\\
-0.90576171875	22.9145662115042\\
-0.9052734375	22.918139713689\\
-0.90478515625	22.9217319191826\\
-0.904296875	22.9253428316757\\
-0.90380859375	22.9289724548793\\
-0.9033203125	22.9326207925247\\
-0.90283203125	22.9362878483635\\
-0.90234375	22.9399736261674\\
-0.90185546875	22.9436781297286\\
-0.9013671875	22.9474013628594\\
-0.90087890625	22.9511433293928\\
-0.900390625	22.9549040331819\\
-0.89990234375	22.9586834781002\\
-0.8994140625	22.9624816680418\\
-0.89892578125	22.966298606921\\
-0.8984375	22.9701342986726\\
-0.89794921875	22.9739887472522\\
-0.8974609375	22.9778619566353\\
-0.89697265625	22.9817539308186\\
-0.896484375	22.9856646738188\\
-0.89599609375	22.9895941896734\\
-0.8955078125	22.9935424824405\\
-0.89501953125	22.9975095561988\\
-0.89453125	23.0014954150475\\
-0.89404296875	23.0055000631068\\
-0.8935546875	23.0095235045171\\
-0.89306640625	23.0135657434398\\
-0.892578125	23.017626784057\\
-0.89208984375	23.0217066305716\\
-0.8916015625	23.0258052872071\\
-0.89111328125	23.0299227582079\\
-0.890625	23.0340590478393\\
-0.89013671875	23.0382141603873\\
-0.8896484375	23.0423881001588\\
-0.88916015625	23.0465808714816\\
-0.888671875	23.0507924787045\\
-0.88818359375	23.0550229261972\\
-0.8876953125	23.0592722183503\\
-0.88720703125	23.0635403595754\\
-0.88671875	23.0678273543052\\
-0.88623046875	23.0721332069935\\
-0.8857421875	23.0764579221149\\
-0.88525390625	23.0808015041653\\
-0.884765625	23.0851639576617\\
-0.88427734375	23.0895452871422\\
-0.8837890625	23.0939454971661\\
-0.88330078125	23.0983645923138\\
-0.8828125	23.1028025771871\\
-0.88232421875	23.107259456409\\
-0.8818359375	23.1117352346237\\
-0.88134765625	23.1162299164966\\
-0.880859375	23.1207435067147\\
-0.88037109375	23.1252760099861\\
-0.8798828125	23.1298274310405\\
-0.87939453125	23.1343977746289\\
-0.87890625	23.1389870455236\\
-0.87841796875	23.1435952485187\\
-0.8779296875	23.1482223884295\\
-0.87744140625	23.1528684700929\\
-0.876953125	23.1575334983674\\
-0.87646484375	23.1622174781331\\
-0.8759765625	23.1669204142916\\
-0.87548828125	23.1716423117661\\
-0.875	23.1763831755017\\
-0.87451171875	23.181143010465\\
-0.8740234375	23.1859218216443\\
-0.87353515625	23.1907196140497\\
-0.873046875	23.1955363927132\\
-0.87255859375	23.2003721626884\\
-0.8720703125	23.2052269290509\\
-0.87158203125	23.2101006968981\\
-0.87109375	23.2149934713492\\
-0.87060546875	23.2199052575455\\
-0.8701171875	23.2248360606503\\
-0.86962890625	23.2297858858487\\
-0.869140625	23.2347547383478\\
-0.86865234375	23.2397426233769\\
-0.8681640625	23.2447495461875\\
-0.86767578125	23.2497755120529\\
-0.8671875	23.2548205262687\\
-0.86669921875	23.2598845941527\\
-0.8662109375	23.2649677210449\\
-0.86572265625	23.2700699123074\\
-0.865234375	23.2751911733249\\
-0.86474609375	23.2803315095041\\
-0.8642578125	23.2854909262741\\
-0.86376953125	23.2906694290865\\
-0.86328125	23.2958670234151\\
-0.86279296875	23.3010837147563\\
-0.8623046875	23.3063195086289\\
-0.86181640625	23.3115744105743\\
-0.861328125	23.3168484261563\\
-0.86083984375	23.3221415609613\\
-0.8603515625	23.3274538205983\\
-0.85986328125	23.3327852106991\\
-0.859375	23.338135736918\\
-0.85888671875	23.343505404932\\
-0.8583984375	23.3488942204409\\
-0.85791015625	23.3543021891673\\
-0.857421875	23.3597293168565\\
-0.85693359375	23.3651756092768\\
-0.8564453125	23.3706410722194\\
-0.85595703125	23.3761257114982\\
-0.85546875	23.3816295329502\\
-0.85498046875	23.3871525424355\\
-0.8544921875	23.3926947458369\\
-0.85400390625	23.3982561490606\\
-0.853515625	23.4038367580357\\
-0.85302734375	23.4094365787146\\
-0.8525390625	23.4150556170727\\
-0.85205078125	23.4206938791087\\
-0.8515625	23.4263513708446\\
-0.85107421875	23.4320280983255\\
-0.8505859375	23.43772406762\\
-0.85009765625	23.4434392848201\\
-0.849609375	23.4491737560409\\
-0.84912109375	23.4549274874214\\
-0.8486328125	23.4607004851237\\
-0.84814453125	23.4664927553335\\
-0.84765625	23.4723043042602\\
-0.84716796875	23.4781351381366\\
-0.8466796875	23.4839852632193\\
-0.84619140625	23.4898546857885\\
-0.845703125	23.495743412148\\
-0.84521484375	23.5016514486256\\
-0.8447265625	23.5075788015728\\
-0.84423828125	23.5135254773647\\
-0.84375	23.5194914824007\\
-0.84326171875	23.5254768231038\\
-0.8427734375	23.531481505921\\
-0.84228515625	23.5375055373235\\
-0.841796875	23.5435489238062\\
-0.84130859375	23.5496116718884\\
-0.8408203125	23.5556937881134\\
-0.84033203125	23.5617952790486\\
-0.83984375	23.5679161512858\\
-0.83935546875	23.5740564114408\\
-0.8388671875	23.5802160661539\\
-0.83837890625	23.5863951220897\\
-0.837890625	23.5925935859372\\
-0.83740234375	23.5988114644096\\
-0.8369140625	23.6050487642449\\
-0.83642578125	23.6113054922054\\
-0.8359375	23.6175816550781\\
-0.83544921875	23.6238772596744\\
-0.8349609375	23.6301923128306\\
-0.83447265625	23.6365268214076\\
-0.833984375	23.6428807922909\\
-0.83349609375	23.6492542323909\\
-0.8330078125	23.6556471486429\\
-0.83251953125	23.662059548007\\
-0.83203125	23.6684914374683\\
-0.83154296875	23.6749428240366\\
-0.8310546875	23.6814137147471\\
-0.83056640625	23.6879041166597\\
-0.830078125	23.6944140368598\\
-0.82958984375	23.7009434824577\\
-0.8291015625	23.7074924605888\\
-0.82861328125	23.7140609784141\\
-0.828125	23.7206490431196\\
-0.82763671875	23.7272566619168\\
-0.8271484375	23.7338838420426\\
-0.82666015625	23.7405305907593\\
-0.826171875	23.7471969153548\\
-0.82568359375	23.7538828231422\\
-0.8251953125	23.7605883214607\\
-0.82470703125	23.7673134176749\\
-0.82421875	23.774058119175\\
-0.82373046875	23.7808224333772\\
-0.8232421875	23.7876063677232\\
-0.82275390625	23.7944099296808\\
-0.822265625	23.8012331267436\\
-0.82177734375	23.8080759664313\\
-0.8212890625	23.8149384562892\\
-0.82080078125	23.8218206038892\\
-0.8203125	23.8287224168289\\
-0.81982421875	23.8356439027322\\
-0.8193359375	23.8425850692494\\
-0.81884765625	23.8495459240567\\
-0.818359375	23.856526474857\\
-0.81787109375	23.8635267293792\\
-0.8173828125	23.8705466953791\\
-0.81689453125	23.8775863806385\\
-0.81640625	23.8846457929661\\
-0.81591796875	23.891724940197\\
-0.8154296875	23.8988238301931\\
-0.81494140625	23.9059424708428\\
-0.814453125	23.9130808700615\\
-0.81396484375	23.9202390357913\\
-0.8134765625	23.9274169760011\\
-0.81298828125	23.934614698687\\
-0.8125	23.9418322118717\\
-0.81201171875	23.9490695236053\\
-0.8115234375	23.9563266419649\\
-0.81103515625	23.9636035750546\\
-0.810546875	23.9709003310059\\
-0.81005859375	23.9782169179777\\
-0.8095703125	23.9855533441558\\
-0.80908203125	23.9929096177539\\
-0.80859375	24.0002857470129\\
-0.80810546875	24.0076817402012\\
-0.8076171875	24.0150976056149\\
-0.80712890625	24.0225333515776\\
-0.806640625	24.0299889864407\\
-0.80615234375	24.0374645185833\\
-0.8056640625	24.0449599564125\\
-0.80517578125	24.0524753083631\\
-0.8046875	24.0600105828978\\
-0.80419921875	24.0675657885074\\
-0.8037109375	24.0751409337109\\
-0.80322265625	24.0827360270553\\
-0.802734375	24.0903510771157\\
-0.80224609375	24.0979860924956\\
-0.8017578125	24.1056410818269\\
-0.80126953125	24.1133160537698\\
-0.80078125	24.1210110170128\\
-0.80029296875	24.1287259802731\\
-0.7998046875	24.1364609522966\\
-0.79931640625	24.1442159418575\\
-0.798828125	24.1519909577591\\
-0.79833984375	24.1597860088333\\
-0.7978515625	24.1676011039407\\
-0.79736328125	24.1754362519712\\
-0.796875	24.1832914618433\\
-0.79638671875	24.1911667425048\\
-0.7958984375	24.1990621029326\\
-0.79541015625	24.2069775521326\\
-0.794921875	24.2149130991402\\
-0.79443359375	24.2228687530199\\
-0.7939453125	24.230844522866\\
-0.79345703125	24.2388404178017\\
-0.79296875	24.2468564469802\\
-0.79248046875	24.2548926195841\\
-0.7919921875	24.2629489448257\\
-0.79150390625	24.2710254319471\\
-0.791015625	24.2791220902203\\
-0.79052734375	24.2872389289469\\
-0.7900390625	24.2953759574589\\
-0.78955078125	24.303533185118\\
-0.7890625	24.3117106213161\\
-0.78857421875	24.3199082754753\\
-0.7880859375	24.3281261570482\\
-0.78759765625	24.3363642755173\\
-0.787109375	24.3446226403958\\
-0.78662109375	24.3529012612274\\
-0.7861328125	24.3612001475862\\
-0.78564453125	24.3695193090772\\
-0.78515625	24.3778587553357\\
-0.78466796875	24.3862184960283\\
-0.7841796875	24.3945985408521\\
-0.78369140625	24.4029988995353\\
-0.783203125	24.411419581837\\
-0.78271484375	24.4198605975477\\
-0.7822265625	24.4283219564888\\
-0.78173828125	24.436803668513\\
-0.78125	24.4453057435045\\
-0.78076171875	24.4538281913788\\
-0.7802734375	24.4623710220829\\
-0.77978515625	24.4709342455954\\
-0.779296875	24.4795178719267\\
-0.77880859375	24.4881219111188\\
-0.7783203125	24.4967463732456\\
-0.77783203125	24.5053912684128\\
-0.77734375	24.5140566067584\\
-0.77685546875	24.522742398452\\
-0.7763671875	24.5314486536958\\
-0.77587890625	24.5401753827241\\
-0.775390625	24.5489225958036\\
-0.77490234375	24.5576903032332\\
-0.7744140625	24.5664785153445\\
-0.77392578125	24.5752872425018\\
-0.7734375	24.5841164951018\\
-0.77294921875	24.5929662835741\\
-0.7724609375	24.6018366183813\\
-0.77197265625	24.6107275100187\\
-0.771484375	24.6196389690148\\
-0.77099609375	24.6285710059312\\
-0.7705078125	24.6375236313626\\
-0.77001953125	24.6464968559371\\
-0.76953125	24.6554906903163\\
-0.76904296875	24.6645051451951\\
-0.7685546875	24.673540231302\\
-0.76806640625	24.6825959593992\\
-0.767578125	24.6916723402828\\
-0.76708984375	24.7007693847825\\
-0.7666015625	24.7098871037622\\
-0.76611328125	24.7190255081196\\
-0.765625	24.7281846087866\\
-0.76513671875	24.7373644167296\\
-0.7646484375	24.746564942949\\
-0.76416015625	24.7557861984797\\
-0.763671875	24.7650281943912\\
-0.76318359375	24.7742909417876\\
-0.7626953125	24.7835744518076\\
-0.76220703125	24.7928787356249\\
-0.76171875	24.8022038044479\\
-0.76123046875	24.8115496695202\\
-0.7607421875	24.8209163421205\\
-0.76025390625	24.8303038335625\\
-0.759765625	24.8397121551955\\
-0.75927734375	24.8491413184039\\
-0.7587890625	24.858591334608\\
-0.75830078125	24.8680622152633\\
-0.7578125	24.8775539718612\\
-0.75732421875	24.8870666159292\\
-0.7568359375	24.8966001590302\\
-0.75634765625	24.9061546127635\\
-0.755859375	24.9157299887645\\
-0.75537109375	24.9253262987046\\
-0.7548828125	24.934943554292\\
-0.75439453125	24.9445817672709\\
-0.75390625	24.9542409494223\\
-0.75341796875	24.9639211125639\\
-0.7529296875	24.9736222685499\\
-0.75244140625	24.9833444292718\\
-0.751953125	24.9930876066578\\
-0.75146484375	25.0028518126734\\
-0.7509765625	25.012637059321\\
-0.75048828125	25.0224433586408\\
-0.75	25.0322707227101\\
-0.74951171875	25.0421191636438\\
-0.7490234375	25.0519886935946\\
-0.74853515625	25.0618793247529\\
-0.748046875	25.0717910693471\\
-0.74755859375	25.0817239396433\\
-0.7470703125	25.0916779479462\\
-0.74658203125	25.1016531065984\\
-0.74609375	25.1116494279809\\
-0.74560546875	25.1216669245133\\
-0.7451171875	25.1317056086539\\
-0.74462890625	25.1417654928994\\
-0.744140625	25.1518465897855\\
-0.74365234375	25.1619489118869\\
-0.7431640625	25.1720724718174\\
-0.74267578125	25.1822172822299\\
-0.7421875	25.1923833558167\\
-0.74169921875	25.2025707053095\\
-0.7412109375	25.2127793434796\\
-0.74072265625	25.2230092831379\\
-0.740234375	25.2332605371354\\
-0.73974609375	25.2435331183626\\
-0.7392578125	25.2538270397505\\
-0.73876953125	25.2641423142701\\
-0.73828125	25.2744789549327\\
-0.73779296875	25.2848369747901\\
-0.7373046875	25.2952163869346\\
-0.73681640625	25.3056172044995\\
-0.736328125	25.3160394406586\\
-0.73583984375	25.3264831086269\\
-0.7353515625	25.3369482216603\\
-0.73486328125	25.3474347930561\\
-0.734375	25.3579428361531\\
-0.73388671875	25.3684723643312\\
-0.7333984375	25.3790233910123\\
-0.73291015625	25.38959592966\\
-0.732421875	25.4001899937797\\
-0.73193359375	25.410805596919\\
-0.7314453125	25.4214427526676\\
-0.73095703125	25.4321014746575\\
-0.73046875	25.4427817765632\\
-0.72998046875	25.4534836721018\\
-0.7294921875	25.4642071750333\\
-0.72900390625	25.4749522991603\\
-0.728515625	25.4857190583287\\
-0.72802734375	25.4965074664273\\
-0.7275390625	25.5073175373885\\
-0.72705078125	25.5181492851881\\
-0.7265625	25.5290027238453\\
-0.72607421875	25.5398778674233\\
-0.7255859375	25.550774730029\\
-0.72509765625	25.5616933258136\\
-0.724609375	25.5726336689723\\
-0.72412109375	25.5835957737447\\
-0.7236328125	25.5945796544148\\
-0.72314453125	25.6055853253115\\
-0.72265625	25.6166128008083\\
-0.72216796875	25.6276620953237\\
-0.7216796875	25.6387332233213\\
-0.72119140625	25.64982619931\\
-0.720703125	25.6609410378442\\
-0.72021484375	25.6720777535236\\
-0.7197265625	25.6832363609939\\
-0.71923828125	25.6944168749467\\
-0.71875	25.7056193101194\\
-0.71826171875	25.7168436812958\\
-0.7177734375	25.728090003306\\
-0.71728515625	25.7393582910267\\
-0.716796875	25.7506485593814\\
-0.71630859375	25.76196082334\\
-0.7158203125	25.77329509792\\
-0.71533203125	25.7846513981857\\
-0.71484375	25.796029739249\\
-0.71435546875	25.807430136269\\
-0.7138671875	25.8188526044528\\
-0.71337890625	25.8302971590553\\
-0.712890625	25.8417638153792\\
-0.71240234375	25.8532525887757\\
-0.7119140625	25.8647634946443\\
-0.71142578125	25.8762965484328\\
-0.7109375	25.887851765638\\
-0.71044921875	25.8994291618055\\
-0.7099609375	25.9110287525299\\
-0.70947265625	25.9226505534551\\
-0.708984375	25.9342945802745\\
-0.70849609375	25.9459608487309\\
-0.7080078125	25.9576493746171\\
-0.70751953125	25.9693601737757\\
-0.70703125	25.9810932620995\\
-0.70654296875	25.9928486555317\\
-0.7060546875	26.0046263700657\\
-0.70556640625	26.0164264217461\\
-0.705078125	26.0282488266679\\
-0.70458984375	26.0400936009775\\
-0.7041015625	26.0519607608724\\
-0.70361328125	26.0638503226015\\
-0.703125	26.0757623024653\\
-0.70263671875	26.0876967168164\\
-0.7021484375	26.0996535820592\\
-0.70166015625	26.1116329146502\\
-0.701171875	26.1236347310986\\
-0.70068359375	26.1356590479659\\
-0.7001953125	26.1477058818666\\
-0.69970703125	26.1597752494681\\
-0.69921875	26.171867167491\\
-0.69873046875	26.1839816527093\\
-0.6982421875	26.1961187219506\\
-0.69775390625	26.2082783920964\\
-0.697265625	26.2204606800819\\
-0.69677734375	26.2326656028968\\
-0.6962890625	26.244893177585\\
-0.69580078125	26.2571434212452\\
-0.6953125	26.2694163510307\\
-0.69482421875	26.2817119841501\\
-0.6943359375	26.294030337867\\
-0.69384765625	26.3063714295005\\
-0.693359375	26.3187352764255\\
-0.69287109375	26.3311218960727\\
-0.6923828125	26.3435313059289\\
-0.69189453125	26.3559635235371\\
-0.69140625	26.3684185664969\\
-0.69091796875	26.3808964524648\\
-0.6904296875	26.3933971991541\\
-0.68994140625	26.4059208243352\\
-0.689453125	26.4184673458361\\
-0.68896484375	26.4310367815424\\
-0.6884765625	26.4436291493975\\
-0.68798828125	26.4562444674029\\
-0.6875	26.4688827536184\\
-0.68701171875	26.4815440261623\\
-0.6865234375	26.4942283032119\\
-0.68603515625	26.5069356030033\\
-0.685546875	26.5196659438318\\
-0.68505859375	26.5324193440523\\
-0.6845703125	26.5451958220793\\
-0.68408203125	26.5579953963874\\
-0.68359375	26.5708180855112\\
-0.68310546875	26.5836639080458\\
-0.6826171875	26.5965328826469\\
-0.68212890625	26.6094250280314\\
-0.681640625	26.6223403629769\\
-0.68115234375	26.6352789063228\\
-0.6806640625	26.6482406769698\\
-0.68017578125	26.6612256938807\\
-0.6796875	26.6742339760804\\
-0.67919921875	26.6872655426562\\
-0.6787109375	26.700320412758\\
-0.67822265625	26.7133986055987\\
-0.677734375	26.7265001404541\\
-0.67724609375	26.7396250366637\\
-0.6767578125	26.7527733136304\\
-0.67626953125	26.7659449908215\\
-0.67578125	26.7791400877678\\
-0.67529296875	26.7923586240651\\
-0.6748046875	26.8056006193737\\
-0.67431640625	26.818866093419\\
-0.673828125	26.8321550659915\\
-0.67333984375	26.8454675569472\\
-0.6728515625	26.858803586208\\
-0.67236328125	26.872163173762\\
-0.671875	26.8855463396632\\
-0.67138671875	26.8989531040327\\
-0.6708984375	26.9123834870582\\
-0.67041015625	26.9258375089946\\
-0.669921875	26.9393151901642\\
-0.66943359375	26.9528165509573\\
-0.6689453125	26.966341611832\\
-0.66845703125	26.9798903933146\\
-0.66796875	26.9934629160001\\
-0.66748046875	27.0070592005524\\
-0.6669921875	27.0206792677046\\
-0.66650390625	27.0343231382592\\
-0.666015625	27.0479908330883\\
-0.66552734375	27.0616823731343\\
-0.6650390625	27.0753977794098\\
-0.66455078125	27.089137072998\\
-0.6640625	27.1029002750532\\
-0.66357421875	27.1166874068008\\
-0.6630859375	27.1304984895378\\
-0.66259765625	27.144333544633\\
-0.662109375	27.1581925935274\\
-0.66162109375	27.1720756577346\\
-0.6611328125	27.1859827588407\\
-0.66064453125	27.1999139185052\\
-0.66015625	27.2138691584607\\
-0.65966796875	27.2278485005138\\
-0.6591796875	27.2418519665449\\
-0.65869140625	27.2558795785091\\
-0.658203125	27.2699313584358\\
-0.65771484375	27.2840073284296\\
-0.6572265625	27.2981075106706\\
-0.65673828125	27.3122319274142\\
-0.65625	27.3263806009922\\
-0.65576171875	27.3405535538123\\
-0.6552734375	27.3547508083593\\
-0.65478515625	27.3689723871946\\
-0.654296875	27.3832183129574\\
-0.65380859375	27.397488608364\\
-0.6533203125	27.4117832962093\\
-0.65283203125	27.4261023993661\\
-0.65234375	27.4404459407862\\
-0.65185546875	27.4548139435004\\
-0.6513671875	27.4692064306188\\
-0.65087890625	27.4836234253314\\
-0.650390625	27.4980649509082\\
-0.64990234375	27.5125310306997\\
-0.6494140625	27.5270216881374\\
-0.64892578125	27.5415369467338\\
-0.6484375	27.5560768300829\\
-0.64794921875	27.5706413618609\\
-0.6474609375	27.5852305658261\\
-0.64697265625	27.5998444658195\\
-0.646484375	27.614483085765\\
-0.64599609375	27.6291464496701\\
-0.6455078125	27.643834581626\\
-0.64501953125	27.6585475058082\\
-0.64453125	27.6732852464764\\
-0.64404296875	27.6880478279756\\
-0.6435546875	27.7028352747358\\
-0.64306640625	27.717647611273\\
-0.642578125	27.7324848621891\\
-0.64208984375	27.7473470521725\\
-0.6416015625	27.7622342059984\\
-0.64111328125	27.7771463485296\\
-0.640625	27.7920835047161\\
-0.64013671875	27.8070456995964\\
-0.6396484375	27.8220329582973\\
-0.63916015625	27.8370453060343\\
-0.638671875	27.8520827681126\\
-0.63818359375	27.8671453699268\\
-0.6376953125	27.8822331369618\\
-0.63720703125	27.8973460947929\\
-0.63671875	27.9124842690864\\
-0.63623046875	27.9276476856\\
-0.6357421875	27.9428363701833\\
-0.63525390625	27.9580503487779\\
-0.634765625	27.9732896474184\\
-0.63427734375	27.9885542922322\\
-0.6337890625	28.0038443094403\\
-0.63330078125	28.0191597253578\\
-0.6328125	28.0345005663941\\
-0.63232421875	28.0498668590535\\
-0.6318359375	28.0652586299354\\
-0.63134765625	28.0806759057353\\
-0.630859375	28.0961187132447\\
-0.63037109375	28.1115870793516\\
-0.6298828125	28.1270810310415\\
-0.62939453125	28.1426005953972\\
-0.62890625	28.1581457995994\\
-0.62841796875	28.1737166709276\\
-0.6279296875	28.1893132367602\\
-0.62744140625	28.2049355245749\\
-0.626953125	28.2205835619493\\
-0.62646484375	28.2362573765616\\
-0.6259765625	28.2519569961907\\
-0.62548828125	28.2676824487168\\
-0.625	28.2834337621221\\
-0.62451171875	28.299210964491\\
-0.6240234375	28.3150140840108\\
-0.62353515625	28.330843148972\\
-0.623046875	28.346698187769\\
-0.62255859375	28.3625792289004\\
-0.6220703125	28.3784863009697\\
-0.62158203125	28.3944194326858\\
-0.62109375	28.4103786528632\\
-0.62060546875	28.4263639904229\\
-0.6201171875	28.4423754743928\\
-0.61962890625	28.4584131339079\\
-0.619140625	28.4744769982116\\
-0.61865234375	28.4905670966551\\
-0.6181640625	28.5066834586991\\
-0.61767578125	28.5228261139135\\
-0.6171875	28.5389950919783\\
-0.61669921875	28.5551904226841\\
-0.6162109375	28.5714121359324\\
-0.61572265625	28.5876602617366\\
-0.615234375	28.6039348302222\\
-0.61474609375	28.6202358716275\\
-0.6142578125	28.636563416304\\
-0.61376953125	28.6529174947173\\
-0.61328125	28.669298137447\\
-0.61279296875	28.6857053751882\\
-0.6123046875	28.7021392387512\\
-0.61181640625	28.7185997590627\\
-0.611328125	28.7350869671659\\
-0.61083984375	28.7516008942214\\
-0.6103515625	28.7681415715078\\
-0.60986328125	28.7847090304219\\
-0.609375	28.8013033024799\\
-0.60888671875	28.8179244193174\\
-0.6083984375	28.8345724126904\\
-0.60791015625	28.8512473144756\\
-0.607421875	28.8679491566715\\
-0.60693359375	28.8846779713982\\
-0.6064453125	28.901433790899\\
-0.60595703125	28.9182166475403\\
-0.60546875	28.9350265738122\\
-0.60498046875	28.9518636023297\\
-0.6044921875	28.968727765833\\
-0.60400390625	28.9856190971878\\
-0.603515625	29.0025376293867\\
-0.60302734375	29.019483395549\\
-0.6025390625	29.0364564289221\\
-0.60205078125	29.0534567628815\\
-0.6015625	29.0704844309321\\
-0.60107421875	29.0875394667082\\
-0.6005859375	29.1046219039746\\
-0.60009765625	29.1217317766273\\
-0.599609375	29.1388691186937\\
-0.59912109375	29.156033964334\\
-0.5986328125	29.1732263478411\\
-0.59814453125	29.190446303642\\
-0.59765625	29.2076938662978\\
-0.59716796875	29.2249690705049\\
-0.5966796875	29.2422719510958\\
-0.59619140625	29.2596025430391\\
-0.595703125	29.2769608814409\\
-0.59521484375	29.2943470015453\\
-0.5947265625	29.3117609387349\\
-0.59423828125	29.3292027285318\\
-0.59375	29.3466724065982\\
-0.59326171875	29.3641700087372\\
-0.5927734375	29.3816955708933\\
-0.59228515625	29.3992491291537\\
-0.591796875	29.4168307197482\\
-0.59130859375	29.4344403790508\\
-0.5908203125	29.4520781435798\\
-0.59033203125	29.469744049999\\
-0.58984375	29.4874381351182\\
-0.58935546875	29.505160435894\\
-0.5888671875	29.5229109894308\\
-0.58837890625	29.5406898329813\\
-0.587890625	29.5584970039472\\
-0.58740234375	29.5763325398807\\
-0.5869140625	29.5941964784841\\
-0.58642578125	29.6120888576119\\
-0.5859375	29.6300097152704\\
-0.58544921875	29.6479590896194\\
-0.5849609375	29.6659370189728\\
-0.58447265625	29.683943541799\\
-0.583984375	29.7019786967222\\
-0.58349609375	29.720042522523\\
-0.5830078125	29.7381350581394\\
-0.58251953125	29.7562563426673\\
-0.58203125	29.774406415362\\
-0.58154296875	29.7925853156382\\
-0.5810546875	29.8107930830716\\
-0.58056640625	29.8290297573991\\
-0.580078125	29.8472953785205\\
-0.57958984375	29.8655899864984\\
-0.5791015625	29.88391362156\\
-0.57861328125	29.9022663240972\\
-0.578125	29.9206481346682\\
-0.57763671875	29.9390590939976\\
-0.5771484375	29.9574992429782\\
-0.57666015625	29.9759686226712\\
-0.576171875	29.9944672743074\\
-0.57568359375	30.0129952392882\\
-0.5751953125	30.0315525591862\\
-0.57470703125	30.0501392757465\\
-0.57421875	30.0687554308875\\
-0.57373046875	30.0874010667018\\
-0.5732421875	30.1060762254572\\
-0.57275390625	30.1247809495975\\
-0.572265625	30.1435152817439\\
-0.57177734375	30.1622792646953\\
-0.5712890625	30.1810729414299\\
-0.57080078125	30.199896355106\\
-0.5703125	30.2187495490626\\
-0.56982421875	30.237632566821\\
-0.5693359375	30.2565454520855\\
-0.56884765625	30.2754882487442\\
-0.568359375	30.2944610008704\\
-0.56787109375	30.3134637527236\\
-0.5673828125	30.3324965487501\\
-0.56689453125	30.3515594335847\\
-0.56640625	30.3706524520509\\
-0.56591796875	30.3897756491629\\
-0.5654296875	30.4089290701258\\
-0.56494140625	30.4281127603372\\
-0.564453125	30.4473267653883\\
-0.56396484375	30.4665711310645\\
-0.5634765625	30.4858459033469\\
-0.56298828125	30.5051511284132\\
-0.5625	30.524486852639\\
-0.56201171875	30.5438531225988\\
-0.5615234375	30.5632499850668\\
-0.56103515625	30.5826774870187\\
-0.560546875	30.6021356756321\\
-0.56005859375	30.6216245982883\\
-0.5595703125	30.6411443025729\\
-0.55908203125	30.6606948362772\\
-0.55859375	30.6802762473994\\
-0.55810546875	30.6998885841458\\
-0.5576171875	30.7195318949317\\
-0.55712890625	30.7392062283827\\
-0.556640625	30.7589116333363\\
-0.55615234375	30.7786481588425\\
-0.5556640625	30.7984158541653\\
-0.55517578125	30.818214768784\\
-0.5546875	30.8380449523941\\
-0.55419921875	30.8579064549088\\
-0.5537109375	30.8777993264604\\
-0.55322265625	30.8977236174011\\
-0.552734375	30.9176793783044\\
-0.55224609375	30.9376666599669\\
-0.5517578125	30.9576855134086\\
-0.55126953125	30.9777359898751\\
-0.55078125	30.9978181408382\\
-0.55029296875	31.0179320179978\\
-0.5498046875	31.0380776732828\\
-0.54931640625	31.0582551588523\\
-0.548828125	31.0784645270976\\
-0.54833984375	31.0987058306428\\
-0.5478515625	31.1189791223465\\
-0.54736328125	31.1392844553032\\
-0.546875	31.1596218828445\\
-0.54638671875	31.1799914585405\\
-0.5458984375	31.2003932362013\\
-0.54541015625	31.2208272698783\\
-0.544921875	31.2412936138656\\
-0.54443359375	31.2617923227015\\
-0.5439453125	31.2823234511697\\
-0.54345703125	31.3028870543011\\
-0.54296875	31.3234831873749\\
-0.54248046875	31.3441119059202\\
-0.5419921875	31.3647732657175\\
-0.54150390625	31.3854673228\\
-0.541015625	31.4061941334552\\
-0.54052734375	31.4269537542267\\
-0.5400390625	31.4477462419149\\
-0.53955078125	31.4685716535794\\
-0.5390625	31.48943004654\\
-0.53857421875	31.5103214783783\\
-0.5380859375	31.5312460069394\\
-0.53759765625	31.5522036903332\\
-0.537109375	31.5731945869364\\
-0.53662109375	31.5942187553936\\
-0.5361328125	31.6152762546192\\
-0.53564453125	31.6363671437988\\
-0.53515625	31.657491482391\\
-0.53466796875	31.6786493301288\\
-0.5341796875	31.6998407470216\\
-0.53369140625	31.7210657933564\\
-0.533203125	31.7423245296996\\
-0.53271484375	31.763617016899\\
-0.5322265625	31.784943316085\\
-0.53173828125	31.8063034886726\\
-0.53125	31.8276975963628\\
-0.53076171875	31.8491257011449\\
-0.5302734375	31.8705878652974\\
-0.52978515625	31.8920841513906\\
-0.529296875	31.9136146222877\\
-0.52880859375	31.9351793411467\\
-0.5283203125	31.9567783714227\\
-0.52783203125	31.9784117768687\\
-0.52734375	32.0000796215385\\
-0.52685546875	32.0217819697876\\
-0.5263671875	32.0435188862755\\
-0.52587890625	32.0652904359677\\
-0.525390625	32.087096684137\\
-0.52490234375	32.1089376963658\\
-0.5244140625	32.1308135385478\\
-0.52392578125	32.1527242768899\\
-0.5234375	32.1746699779145\\
-0.52294921875	32.1966507084605\\
-0.5224609375	32.2186665356863\\
-0.52197265625	32.2407175270709\\
-0.521484375	32.2628037504165\\
-0.52099609375	32.2849252738501\\
-0.5205078125	32.3070821658254\\
-0.52001953125	32.3292744951253\\
-0.51953125	32.3515023308634\\
-0.51904296875	32.3737657424862\\
-0.5185546875	32.3960647997753\\
-0.51806640625	32.4183995728493\\
-0.517578125	32.4407701321661\\
-0.51708984375	32.4631765485245\\
-0.5166015625	32.4856188930669\\
-0.51611328125	32.508097237281\\
-0.515625	32.5306116530022\\
-0.51513671875	32.5531622124155\\
-0.5146484375	32.5757489880579\\
-0.51416015625	32.5983720528206\\
-0.513671875	32.6210314799509\\
-0.51318359375	32.6437273430548\\
-0.5126953125	32.666459716099\\
-0.51220703125	32.6892286734129\\
-0.51171875	32.7120342896918\\
-0.51123046875	32.7348766399979\\
-0.5107421875	32.7577557997637\\
-0.51025390625	32.7806718447936\\
-0.509765625	32.8036248512667\\
-0.50927734375	32.8266148957387\\
-0.5087890625	32.8496420551448\\
-0.50830078125	32.8727064068016\\
-0.5078125	32.8958080284097\\
-0.50732421875	32.9189469980563\\
-0.5068359375	32.9421233942172\\
-0.50634765625	32.9653372957598\\
-0.505859375	32.988588781945\\
-0.50537109375	33.0118779324303\\
-0.5048828125	33.0352048272717\\
-0.50439453125	33.0585695469267\\
-0.50390625	33.0819721722566\\
-0.50341796875	33.1054127845291\\
-0.5029296875	33.128891465421\\
-0.50244140625	33.1524082970209\\
-0.501953125	33.1759633618311\\
-0.50146484375	33.1995567427715\\
-0.5009765625	33.2231885231809\\
-0.50048828125	33.2468587868209\\
-0.5	33.2705676178776\\
-0.49951171875	33.2943151009652\\
-0.4990234375	33.3181013211279\\
-0.49853515625	33.3419263638433\\
-0.498046875	33.3657903150251\\
-0.49755859375	33.3896932610255\\
-0.4970703125	33.4136352886383\\
-0.49658203125	33.4376164851021\\
-0.49609375	33.4616369381023\\
-0.49560546875	33.4856967357751\\
-0.4951171875	33.5097959667093\\
-0.49462890625	33.53393471995\\
-0.494140625	33.5581130850015\\
-0.49365234375	33.5823311518296\\
-0.4931640625	33.6065890108657\\
-0.49267578125	33.6308867530088\\
-0.4921875	33.6552244696291\\
-0.49169921875	33.6796022525709\\
-0.4912109375	33.7040201941561\\
-0.49072265625	33.7284783871864\\
-0.490234375	33.7529769249474\\
-0.48974609375	33.7775159012112\\
-0.4892578125	33.80209541024\\
-0.48876953125	33.8267155467887\\
-0.48828125	33.8513764061088\\
-0.48779296875	33.8760780839515\\
-0.4873046875	33.9008206765704\\
-0.48681640625	33.9256042807259\\
-0.486328125	33.9504289936874\\
-0.48583984375	33.9752949132376\\
-0.4853515625	34.0002021376752\\
-0.48486328125	34.0251507658188\\
-0.484375	34.0501408970101\\
-0.48388671875	34.0751726311173\\
-0.4833984375	34.1002460685391\\
-0.48291015625	34.1253613102075\\
-0.482421875	34.1505184575918\\
-0.48193359375	34.1757176127021\\
-0.4814453125	34.2009588780929\\
-0.48095703125	34.2262423568669\\
-0.48046875	34.2515681526782\\
-0.47998046875	34.2769363697366\\
-0.4794921875	34.3023471128107\\
-0.47900390625	34.3278004872323\\
-0.478515625	34.3532965988996\\
-0.47802734375	34.3788355542815\\
-0.4775390625	34.4044174604208\\
-0.47705078125	34.4300424249391\\
-0.4765625	34.4557105560395\\
-0.47607421875	34.4814219625115\\
-0.4755859375	34.5071767537344\\
-0.47509765625	34.5329750396815\\
-0.474609375	34.558816930924\\
-0.47412109375	34.5847025386355\\
-0.4736328125	34.6106319745953\\
-0.47314453125	34.6366053511933\\
-0.47265625	34.6626227814338\\
-0.47216796875	34.6886843789396\\
-0.4716796875	34.7147902579563\\
-0.47119140625	34.7409405333569\\
-0.470703125	34.7671353206454\\
-0.47021484375	34.7933747359619\\
-0.4697265625	34.8196588960863\\
-0.46923828125	34.8459879184431\\
-0.46875	34.8723619211057\\
-0.46826171875	34.8987810228007\\
-0.4677734375	34.9252453429129\\
-0.46728515625	34.9517550014892\\
-0.466796875	34.9783101192437\\
-0.46630859375	35.0049108175619\\
-0.4658203125	35.0315572185058\\
-0.46533203125	35.0582494448181\\
-0.46484375	35.0849876199272\\
-0.46435546875	35.1117718679521\\
-0.4638671875	35.1386023137067\\
-0.46337890625	35.1654790827054\\
-0.462890625	35.1924023011671\\
-0.46240234375	35.2193720960209\\
-0.4619140625	35.2463885949105\\
-0.46142578125	35.2734519261996\\
-0.4609375	35.3005622189766\\
-0.46044921875	35.32771960306\\
-0.4599609375	35.3549242090034\\
-0.45947265625	35.3821761681005\\
-0.458984375	35.4094756123906\\
-0.45849609375	35.4368226746636\\
-0.4580078125	35.4642174884655\\
-0.45751953125	35.4916601881036\\
-0.45703125	35.519150908652\\
-0.45654296875	35.5466897859568\\
-0.4560546875	35.5742769566421\\
-0.45556640625	35.6019125581148\\
-0.455078125	35.6295967285706\\
-0.45458984375	35.6573296069997\\
-0.4541015625	35.6851113331922\\
-0.45361328125	35.7129420477439\\
-0.453125	35.7408218920621\\
-0.45263671875	35.7687510083714\\
-0.4521484375	35.7967295397194\\
-0.45166015625	35.8247576299829\\
-0.451171875	35.8528354238735\\
-0.45068359375	35.8809630669439\\
-0.4501953125	35.9091407055938\\
-0.44970703125	35.937368487076\\
-0.44921875	35.9656465595027\\
-0.44873046875	35.9939750718515\\
-0.4482421875	36.0223541739716\\
-0.44775390625	36.0507840165906\\
-0.447265625	36.0792647513204\\
-0.44677734375	36.1077965306637\\
-0.4462890625	36.1363795080205\\
-0.44580078125	36.1650138376949\\
-0.4453125	36.1936996749013\\
-0.44482421875	36.2224371757711\\
-0.4443359375	36.2512264973597\\
-0.44384765625	36.2800677976529\\
-0.443359375	36.308961235574\\
-0.44287109375	36.3379069709906\\
-0.4423828125	36.3669051647213\\
-0.44189453125	36.3959559785431\\
-0.44140625	36.4250595751982\\
-0.44091796875	36.454216118401\\
-0.4404296875	36.4834257728456\\
-0.43994140625	36.512688704213\\
-0.439453125	36.542005079178\\
-0.43896484375	36.5713750654171\\
-0.4384765625	36.6007988316156\\
-0.43798828125	36.6302765474753\\
-0.4375	36.6598083837218\\
-0.43701171875	36.6893945121125\\
-0.4365234375	36.7190351054439\\
-0.43603515625	36.7487303375597\\
-0.435546875	36.7784803833584\\
-0.43505859375	36.8082854188014\\
-0.4345703125	36.8381456209209\\
-0.43408203125	36.8680611678277\\
-0.43359375	36.8980322387198\\
-0.43310546875	36.9280590138902\\
-0.4326171875	36.9581416747353\\
-0.43212890625	36.9882804037633\\
-0.431640625	37.0184753846025\\
-0.43115234375	37.0487268020097\\
-0.4306640625	37.0790348418792\\
-0.43017578125	37.1093996912508\\
-0.4296875	37.139821538319\\
-0.42919921875	37.1703005724416\\
-0.4287109375	37.2008369841487\\
-0.42822265625	37.2314309651514\\
-0.427734375	37.2620827083511\\
-0.42724609375	37.2927924078486\\
-0.4267578125	37.323560258953\\
-0.42626953125	37.3543864581915\\
-0.42578125	37.3852712033183\\
-0.42529296875	37.4162146933242\\
-0.4248046875	37.4472171284462\\
-0.42431640625	37.4782787101771\\
-0.423828125	37.5093996412754\\
-0.42333984375	37.5405801257746\\
-0.4228515625	37.5718203689936\\
-0.42236328125	37.6031205775465\\
-0.421875	37.6344809593527\\
-0.42138671875	37.665901723647\\
-0.4208984375	37.6973830809899\\
-0.42041015625	37.7289252432782\\
-0.419921875	37.7605284237548\\
-0.41943359375	37.7921928370201\\
-0.4189453125	37.8239186990422\\
-0.41845703125	37.8557062271675\\
-0.41796875	37.887555640132\\
-0.41748046875	37.9194671580719\\
-0.4169921875	37.9514410025347\\
-0.41650390625	37.9834773964907\\
-0.416015625	38.0155765643436\\
-0.41552734375	38.0477387319426\\
-0.4150390625	38.0799641265936\\
-0.41455078125	38.1122529770705\\
-0.4140625	38.1446055136273\\
-0.41357421875	38.1770219680097\\
-0.4130859375	38.2095025734673\\
-0.41259765625	38.242047564765\\
-0.412109375	38.2746571781961\\
-0.41162109375	38.3073316515936\\
-0.4111328125	38.3400712243432\\
-0.41064453125	38.3728761373957\\
-0.41015625	38.4057466332796\\
-0.40966796875	38.4386829561136\\
-0.4091796875	38.4716853516198\\
-0.40869140625	38.5047540671366\\
-0.408203125	38.5378893516318\\
-0.40771484375	38.5710914557156\\
-0.4072265625	38.6043606316547\\
-0.40673828125	38.6376971333847\\
-0.40625	38.671101216525\\
-0.40576171875	38.7045731383915\\
-0.4052734375	38.738113158011\\
-0.40478515625	38.7717215361353\\
-0.404296875	38.8053985352551\\
-0.40380859375	38.8391444196146\\
-0.4033203125	38.8729594552259\\
-0.40283203125	38.9068439098832\\
-0.40234375	38.9407980531783\\
-0.40185546875	38.9748221565149\\
-0.4013671875	39.0089164931237\\
-0.40087890625	39.0430813380779\\
-0.400390625	39.0773169683083\\
-0.39990234375	39.1116236626189\\
-0.3994140625	39.1460017017022\\
-0.39892578125	39.1804513681558\\
-0.3984375	39.2149729464975\\
-0.39794921875	39.2495667231819\\
-0.3974609375	39.2842329866167\\
-0.39697265625	39.3189720271788\\
-0.396484375	39.3537841372315\\
-0.39599609375	39.3886696111406\\
-0.3955078125	39.4236287452919\\
-0.39501953125	39.4586618381081\\
-0.39453125	39.4937691900663\\
-0.39404296875	39.528951103715\\
-0.3935546875	39.5642078836923\\
-0.39306640625	39.5995398367436\\
-0.392578125	39.6349472717393\\
-0.39208984375	39.6704304996934\\
-0.3916015625	39.705989833782\\
-0.39111328125	39.7416255893612\\
-0.390625	39.7773380839865\\
-0.39013671875	39.8131276374318\\
-0.3896484375	39.8489945717081\\
-0.38916015625	39.8849392110834\\
-0.388671875	39.9209618821017\\
-0.38818359375	39.9570629136033\\
-0.3876953125	39.9932426367447\\
-0.38720703125	40.0295013850183\\
-0.38671875	40.0658394942736\\
-0.38623046875	40.102257302737\\
-0.3857421875	40.1387551510333\\
-0.38525390625	40.1753333822063\\
-0.384765625	40.2119923417404\\
-0.38427734375	40.2487323775821\\
-0.3837890625	40.2855538401613\\
-0.38330078125	40.3224570824138\\
-0.3828125	40.3594424598031\\
-0.38232421875	40.3965103303433\\
-0.3818359375	40.4336610546211\\
-0.38134765625	40.4708949958195\\
-0.380859375	40.5082125197402\\
-0.38037109375	40.5456139948278\\
-0.3798828125	40.5830997921928\\
-0.37939453125	40.6206702856358\\
-0.37890625	40.6583258516717\\
-0.37841796875	40.6960668695543\\
-0.3779296875	40.7338937213005\\
-0.37744140625	40.771806791716\\
-0.376953125	40.8098064684199\\
-0.37646484375	40.8478931418706\\
-0.3759765625	40.8860672053915\\
-0.37548828125	40.9243290551971\\
-0.375	40.9626790904197\\
-0.37451171875	41.0011177131353\\
-0.3740234375	41.0396453283913\\
-0.37353515625	41.0782623442335\\
-0.373046875	41.1169691717338\\
-0.37255859375	41.1557662250177\\
-0.3720703125	41.194653921293\\
-0.37158203125	41.2336326808779\\
-0.37109375	41.2727029272301\\
-0.37060546875	41.3118650869758\\
-0.3701171875	41.3511195899392\\
-0.36962890625	41.3904668691722\\
-0.369140625	41.4299073609849\\
-0.36865234375	41.4694415049755\\
-0.3681640625	41.5090697440619\\
-0.36767578125	41.5487925245121\\
-0.3671875	41.5886102959762\\
-0.36669921875	41.6285235115181\\
-0.3662109375	41.668532627648\\
-0.36572265625	41.7086381043547\\
-0.365234375	41.7488404051389\\
-0.36474609375	41.7891399970463\\
-0.3642578125	41.8295373507017\\
-0.36376953125	41.8700329403428\\
-0.36328125	41.910627243855\\
-0.36279296875	41.9513207428062\\
-0.3623046875	41.9921139224824\\
-0.36181640625	42.0330072719232\\
-0.361328125	42.0740012839582\\
-0.36083984375	42.1150964552432\\
-0.3603515625	42.1562932862979\\
-0.35986328125	42.1975922815428\\
-0.359375	42.2389939493377\\
-0.35888671875	42.2804988020192\\
-0.3583984375	42.3221073559404\\
-0.35791015625	42.3638201315098\\
-0.357421875	42.4056376532307\\
-0.35693359375	42.4475604497422\\
-0.3564453125	42.4895890538593\\
-0.35595703125	42.5317240026142\\
-0.35546875	42.5739658372981\\
-0.35498046875	42.6163151035033\\
-0.3544921875	42.658772351166\\
-0.35400390625	42.7013381346091\\
-0.353515625	42.7440130125867\\
-0.35302734375	42.7867975483273\\
-0.3525390625	42.8296923095797\\
-0.35205078125	42.8726978686573\\
-0.3515625	42.915814802485\\
-0.35107421875	42.9590436926446\\
-0.3505859375	43.0023851254228\\
-0.35009765625	43.045839691858\\
-0.349609375	43.0894079877889\\
-0.34912109375	43.1330906139034\\
-0.3486328125	43.1768881757874\\
-0.34814453125	43.2208012839753\\
-0.34765625	43.2648305540005\\
-0.34716796875	43.3089766064466\\
-0.3466796875	43.3532400669989\\
-0.34619140625	43.3976215664976\\
-0.345703125	43.4421217409908\\
-0.34521484375	43.4867412317876\\
-0.3447265625	43.5314806855137\\
-0.34423828125	43.5763407541658\\
-0.34375	43.6213220951682\\
-0.34326171875	43.6664253714285\\
-0.3427734375	43.7116512513958\\
-0.34228515625	43.7570004091183\\
-0.341796875	43.8024735243023\\
-0.34130859375	43.8480712823715\\
-0.3408203125	43.8937943745275\\
-0.34033203125	43.9396434978108\\
-0.33984375	43.9856193551629\\
-0.33935546875	44.0317226554885\\
-0.3388671875	44.0779541137195\\
-0.33837890625	44.1243144508788\\
-0.337890625	44.1708043941459\\
-0.33740234375	44.2174246769224\\
-0.3369140625	44.2641760388993\\
-0.33642578125	44.3110592261244\\
-0.3359375	44.3580749910708\\
-0.33544921875	44.4052240927068\\
-0.3349609375	44.4525072965662\\
-0.33447265625	44.4999253748189\\
-0.333984375	44.5474791063445\\
-0.33349609375	44.5951692768043\\
-0.3330078125	44.6429966787162\\
-0.33251953125	44.6909621115298\\
-0.33203125	44.7390663817022\\
-0.33154296875	44.7873103027758\\
-0.3310546875	44.8356946954561\\
-0.33056640625	44.884220387691\\
-0.330078125	44.9328882147515\\
-0.32958984375	44.9816990193127\\
-0.3291015625	45.0306536515368\\
-0.32861328125	45.0797529691562\\
-0.328125	45.1289978375588\\
-0.32763671875	45.1783891298737\\
-0.3271484375	45.2279277270584\\
-0.32666015625	45.2776145179868\\
-0.326171875	45.3274503995392\\
-0.32568359375	45.3774362766928\\
-0.3251953125	45.4275730626134\\
-0.32470703125	45.4778616787494\\
-0.32421875	45.5283030549252\\
-0.32373046875	45.5788981294384\\
-0.3232421875	45.6296478491559\\
-0.32275390625	45.6805531696128\\
-0.322265625	45.7316150551125\\
-0.32177734375	45.7828344788278\\
-0.3212890625	45.8342124229034\\
-0.32080078125	45.8857498785605\\
-0.3203125	45.9374478462022\\
-0.31982421875	45.9893073355204\\
-0.3193359375	46.0413293656048\\
-0.31884765625	46.093514965053\\
-0.318359375	46.1458651720817\\
-0.31787109375	46.1983810346407\\
-0.3173828125	46.2510636105274\\
-0.31689453125	46.3039139675035\\
-0.31640625	46.3569331834132\\
-0.31591796875	46.4101223463032\\
-0.3154296875	46.463482554544\\
-0.31494140625	46.517014916954\\
-0.314453125	46.5707205529238\\
-0.31396484375	46.624600592544\\
-0.3134765625	46.6786561767332\\
-0.31298828125	46.7328884573696\\
-0.3125	46.787298597423\\
-0.31201171875	46.8418877710894\\
-0.3115234375	46.8966571639279\\
-0.31103515625	46.9516079729985\\
-0.310546875	47.0067414070032\\
-0.31005859375	47.0620586864287\\
-0.3095703125	47.1175610436905\\
-0.30908203125	47.1732497232803\\
-0.30859375	47.2291259819149\\
-0.30810546875	47.2851910886871\\
-0.3076171875	47.3414463252196\\
-0.30712890625	47.3978929858208\\
-0.306640625	47.4545323776422\\
-0.30615234375	47.51136582084\\
-0.3056640625	47.5683946487368\\
-0.30517578125	47.6256202079876\\
-0.3046875	47.6830438587476\\
-0.30419921875	47.7406669748422\\
-0.3037109375	47.7984909439402\\
-0.30322265625	47.856517167729\\
-0.302734375	47.914747062093\\
-0.30224609375	47.9731820572948\\
-0.3017578125	48.0318235981581\\
-0.30126953125	48.0906731442548\\
-0.30078125	48.1497321700941\\
-0.30029296875	48.2090021653146\\
-0.2998046875	48.2684846348795\\
-0.29931640625	48.3281810992749\\
-0.298828125	48.3880930947108\\
-0.29833984375	48.4482221733255\\
-0.2978515625	48.5085699033933\\
-0.29736328125	48.5691378695346\\
-0.296875	48.6299276729304\\
-0.29638671875	48.6909409315393\\
-0.2958984375	48.7521792803185\\
-0.29541015625	48.8136443714472\\
-0.294921875	48.8753378745551\\
-0.29443359375	48.9372614769529\\
-0.2939453125	48.9994168838672\\
-0.29345703125	49.0618058186793\\
-0.29296875	49.1244300231669\\
-0.29248046875	49.1872912577508\\
-0.2919921875	49.2503913017441\\
-0.29150390625	49.3137319536065\\
-0.291015625	49.3773150312022\\
-0.29052734375	49.4411423720617\\
-0.2900390625	49.5052158336478\\
-0.28955078125	49.5695372936263\\
-0.2890625	49.6341086501405\\
-0.28857421875	49.6989318220898\\
-0.2880859375	49.7640087494139\\
-0.28759765625	49.8293413933803\\
-0.287109375	49.8949317368769\\
-0.28662109375	49.9607817847096\\
-0.2861328125	50.026893563904\\
-0.28564453125	50.0932691240123\\
-0.28515625	50.1599105374253\\
-0.28466796875	50.2268198996891\\
-0.2841796875	50.293999329827\\
-0.28369140625	50.3614509706665\\
-0.283203125	50.4291769891719\\
-0.28271484375	50.4971795767813\\
-0.2822265625	50.5654609497511\\
-0.28173828125	50.6340233495031\\
-0.28125	50.7028690429799\\
-0.28076171875	50.7720003230043\\
-0.2802734375	50.8414195086457\\
-0.27978515625	50.9111289455913\\
-0.279296875	50.9811310065243\\
-0.27880859375	51.0514280915072\\
-0.2783203125	51.1220226283726\\
-0.27783203125	51.1929170731187\\
-0.27734375	51.2641139103129\\
-0.27685546875	51.3356156535005\\
-0.2763671875	51.4074248456208\\
-0.27587890625	51.4795440594294\\
-0.275390625	51.551975897928\\
-0.27490234375	51.6247229948006\\
-0.2744140625	51.6977880148565\\
-0.27392578125	51.7711736544813\\
-0.2734375	51.8448826420939\\
-0.27294921875	51.9189177386124\\
-0.2724609375	51.9932817379258\\
-0.27197265625	52.0679774673745\\
-0.271484375	52.1430077882381\\
-0.27099609375	52.2183755962303\\
-0.2705078125	52.294083822003\\
-0.27001953125	52.3701354316571\\
-0.26953125	52.4465334272621\\
-0.26904296875	52.5232808473842\\
-0.2685546875	52.6003807676214\\
-0.26806640625	52.6778363011484\\
-0.267578125	52.7556505992697\\
-0.26708984375	52.8338268519802\\
-0.2666015625	52.9123682885366\\
-0.26611328125	52.9912781780352\\
-0.265625	53.0705598300006\\
-0.26513671875	53.1502165949821\\
-0.2646484375	53.2302518651604\\
-0.26416015625	53.3106690749616\\
-0.263671875	53.3914717016829\\
-0.26318359375	53.4726632661252\\
-0.2626953125	53.5542473332371\\
-0.26220703125	53.6362275127669\\
-0.26171875	53.7186074599252\\
-0.26123046875	53.8013908760563\\
-0.2607421875	53.8845815093199\\
-0.26025390625	53.9681831553817\\
-0.259765625	54.0521996581146\\
-0.25927734375	54.1366349103092\\
-0.2587890625	54.2214928543939\\
-0.25830078125	54.3067774831651\\
-0.2578125	54.3924928405269\\
-0.25732421875	54.4786430222407\\
-0.2568359375	54.5652321766846\\
-0.25634765625	54.6522645056221\\
-0.255859375	54.7397442649809\\
-0.25537109375	54.8276757656408\\
-0.2548828125	54.9160633742311\\
-0.25439453125	55.0049115139378\\
-0.25390625	55.0942246653191\\
-0.25341796875	55.1840073671307\\
-0.2529296875	55.2742642171594\\
-0.25244140625	55.3649998730662\\
-0.251953125	55.4562190532361\\
-0.25146484375	55.5479265376378\\
-0.2509765625	55.6401271686904\\
-0.25048828125	55.7328258521372\\
-0.25	55.8260275579274\\
-0.24951171875	55.9197373211041\\
-0.2490234375	56.013960242698\\
-0.24853515625	56.1087014906282\\
-0.248046875	56.2039663006061\\
-0.24755859375	56.299759977046\\
-0.2470703125	56.3960878939781\\
-0.24658203125	56.4929554959651\\
-0.24609375	56.590368299021\\
-0.24560546875	56.6883318915315\\
-0.2451171875	56.7868519351752\\
-0.24462890625	56.8859341658433\\
-0.244140625	56.9855843945587\\
-0.24365234375	57.0858085083917\\
-0.2431640625	57.1866124713717\\
-0.24267578125	57.288002325393\\
-0.2421875	57.3899841911139\\
-0.24169921875	57.4925642688466\\
-0.2412109375	57.5957488394368\\
-0.24072265625	57.6995442651307\\
-0.240234375	57.8039569904269\\
-0.23974609375	57.9089935429119\\
-0.2392578125	58.0146605340763\\
-0.23876953125	58.1209646601083\\
-0.23828125	58.2279127026641\\
-0.23779296875	58.3355115296078\\
-0.2373046875	58.4437680957228\\
-0.23681640625	58.5526894433869\\
-0.236328125	58.6622827032105\\
-0.23583984375	58.7725550946301\\
-0.2353515625	58.883513926458\\
-0.23486328125	58.9951665973766\\
-0.234375	59.1075205963809\\
-0.23388671875	59.2205835031543\\
-0.2333984375	59.3343629883802\\
-0.23291015625	59.448866813977\\
-0.232421875	59.5641028332547\\
-0.23193359375	59.6800789909816\\
-0.2314453125	59.7968033233562\\
-0.23095703125	59.9142839578756\\
-0.23046875	60.0325291130905\\
-0.22998046875	60.1515470982379\\
-0.2294921875	60.2713463127421\\
-0.22900390625	60.3919352455711\\
-0.228515625	60.5133224744388\\
-0.22802734375	60.6355166648385\\
-0.2275390625	60.7585265688961\\
-0.22705078125	60.8823610240272\\
-0.2265625	61.0070289513817\\
-0.22607421875	61.1325393540629\\
-0.2255859375	61.2589013150991\\
-0.22509765625	61.38612399515\\
-0.224609375	61.5142166299297\\
-0.22412109375	61.6431885273206\\
-0.2236328125	61.7730490641572\\
-0.22314453125	61.9038076826538\\
-0.22265625	62.0354738864491\\
-0.22216796875	62.1680572362383\\
-0.2216796875	62.3015673449629\\
-0.22119140625	62.4360138725238\\
-0.220703125	62.5714065199817\\
-0.22021484375	62.7077550232088\\
-0.2197265625	62.8450691459474\\
-0.21923828125	62.9833586722356\\
-0.21875	63.1226333981497\\
-0.21826171875	63.2629031228146\\
-0.2177734375	63.4041776386288\\
-0.21728515625	63.5464667206453\\
-0.216796875	63.6897801150464\\
-0.21630859375	63.834127526648\\
-0.2158203125	63.9795186053609\\
-0.21533203125	64.1259629315332\\
-0.21484375	64.2734700000953\\
-0.21435546875	64.4220492034167\\
-0.2138671875	64.5717098127859\\
-0.21337890625	64.7224609584106\\
-0.212890625	64.874311607836\\
-0.21240234375	65.0272705426627\\
-0.2119140625	65.1813463334493\\
-0.21142578125	65.3365473126639\\
-0.2109375	65.4928815455525\\
-0.21044921875	65.6503567987716\\
-0.2099609375	65.8089805066321\\
-0.20947265625	65.9687597347835\\
-0.208984375	66.1297011411635\\
-0.20849609375	66.291810934019\\
-0.2080078125	66.4550948268008\\
-0.20751953125	66.6195579897134\\
-0.20703125	66.7852049976943\\
-0.20654296875	66.9520397745797\\
-0.2060546875	67.1200655332012\\
-0.20556640625	67.2892847111407\\
-0.205078125	67.4596989018593\\
-0.20458984375	67.6313087808947\\
-0.2041015625	67.8041140268109\\
-0.20361328125	67.9781132365633\\
-0.203125	68.1533038349299\\
-0.20263671875	68.3296819776375\\
-0.2021484375	68.5072424478009\\
-0.20166015625	68.6859785452759\\
-0.201171875	68.8658819685088\\
-0.20068359375	69.0469426884565\\
-0.2001953125	69.2291488141346\\
-0.19970703125	69.4124864493442\\
-0.19921875	69.5969395401176\\
-0.19873046875	69.7824897124201\\
-0.1982421875	69.9691160996459\\
-0.19775390625	70.156795159446\\
-0.197265625	70.3455004794417\\
-0.19677734375	70.5352025713914\\
-0.1962890625	70.7258686534006\\
-0.19580078125	70.917462419812\\
-0.1953125	71.1099437984377\\
-0.19482421875	71.3032686948806\\
-0.1943359375	71.4973887237452\\
-0.19384765625	71.6922509266408\\
-0.193359375	71.8877974769922\\
-0.19287109375	72.083965371799\\
-0.1923828125	72.2806861106559\\
-0.19189453125	72.477885362524\\
-0.19140625	72.6754826209637\\
-0.19091796875	72.8733908488029\\
-0.1904296875	73.0715161134886\\
-0.18994140625	73.2697572147167\\
-0.189453125	73.4680053063075\\
-0.18896484375	73.6661435147035\\
-0.1884765625	73.8640465569585\\
-0.18798828125	74.0615803615844\\
-0.1875	74.2586016962087\\
-0.18701171875	74.4549578065961\\
-0.1865234375	74.6504860722645\\
-0.18603515625	74.8450136846233\\
-0.185546875	75.0383573542841\\
-0.18505859375	75.230323054977\\
-0.1845703125	75.4207058122387\\
-0.18408203125	75.6092895458143\\
-0.18359375	75.7958469754232\\
-0.18310546875	75.9801396001993\\
-0.1826171875	76.161917762683\\
-0.18212890625	76.3409208086756\\
-0.181640625	76.5168773545324\\
-0.18115234375	76.6895056735146\\
-0.1806640625	76.8585142126082\\
-0.18017578125	77.0236022506699\\
-0.1796875	77.1844607078775\\
-0.17919921875	77.3407731151392\\
-0.1787109375	77.4922167503652\\
-0.17822265625	77.638463946257\\
-0.177734375	77.7791835715438\\
-0.17724609375	77.9140426843521\\
-0.1767578125	78.0427083527129\\
-0.17626953125	78.1648496330777\\
-0.17578125	78.2801396932684\\
-0.17529296875	78.3882580615833\\
-0.1748046875	78.4888929790189\\
-0.17431640625	78.5817438268342\\
-0.173828125	78.6665235972512\\
-0.17333984375	78.7429613711028\\
-0.1728515625	78.8108047629275\\
-0.17236328125	78.8698222916017\\
-0.171875	78.9198056332353\\
-0.17138671875	78.9605717129446\\
-0.1708984375	78.991964593307\\
-0.17041015625	79.01385711991\\
-0.169921875	79.0261522883497\\
-0.16943359375	79.0287843023075\\
-0.1689453125	79.0217192987171\\
-0.16845703125	79.0049557233327\\
-0.16796875	78.978524348003\\
-0.16748046875	78.9424879292382\\
-0.1669921875	78.8969405159946\\
-0.16650390625	78.8420064226211\\
-0.166015625	78.7778388902799\\
-0.16552734375	78.7046184666616\\
-0.1650390625	78.6225511391792\\
-0.16455078125	78.5318662609072\\
-0.1640625	78.4328143112751\\
-0.16357421875	78.3256645348488\\
-0.1630859375	78.2107025015309\\
-0.16259765625	78.0882276302816\\
-0.162109375	77.95855071612\\
-0.16162109375	77.8219914969548\\
-0.1611328125	77.6788762928584\\
-0.16064453125	77.5295357459851\\
-0.16015625	77.3743026846447\\
-0.15966796875	77.2135101302636\\
-0.1591796875	77.0474894612503\\
-0.15869140625	76.8765687433025\\
-0.158203125	76.7010712315394\\
-0.15771484375	76.5213140461026\\
-0.1572265625	76.3376070195862\\
-0.15673828125	76.1502517118779\\
-0.15625	75.95954058571\\
-0.15576171875	75.765756334407\\
-0.1552734375	75.5691713519619\\
-0.15478515625	75.3700473346527\\
-0.154296875	75.1686350028243\\
-0.15380859375	74.9651739312291\\
-0.1533203125	74.7598924763291\\
-0.15283203125	74.5530077892218\\
-0.15234375	74.34472590326\\
-0.15185546875	74.1352418859994\\
-0.1513671875	73.9247400457521\\
-0.15087890625	73.7133941837445\\
-0.150390625	73.501367883622\\
-0.14990234375	73.2888148308151\\
-0.1494140625	73.0758791550313\\
-0.14892578125	72.8626957898749\\
-0.1484375	72.6493908443085\\
-0.14794921875	72.4360819813334\\
-0.1474609375	72.2228787998826\\
-0.14697265625	72.0098832164991\\
-0.146484375	71.7971898438999\\
-0.14599609375	71.5848863639879\\
-0.1455078125	71.3730538933158\\
-0.14501953125	71.1617673393722\\
-0.14453125	70.9510957464088\\
-0.14404296875	70.7411026298092\\
-0.1435546875	70.5318462982731\\
-0.14306640625	70.3233801632902\\
-0.142578125	70.1157530355954\\
-0.14208984375	69.9090094084315\\
-0.1416015625	69.7031897276035\\
-0.14111328125	69.4983306484105\\
-0.140625	69.2944652796388\\
-0.14013671875	69.0916234148761\\
-0.1396484375	68.8898317514676\\
-0.13916015625	68.689114097479\\
-0.138671875	68.4894915670731\\
-0.13818359375	68.290982764729\\
-0.1376953125	68.0936039587509\\
-0.13720703125	67.8973692445252\\
-0.13671875	67.7022906979896\\
-0.13623046875	67.5083785197778\\
-0.1357421875	67.3156411704975\\
-0.13525390625	67.1240854975947\\
-0.134765625	66.9337168542464\\
-0.13427734375	66.7445392107096\\
-0.1337890625	66.5565552585448\\
-0.13330078125	66.3697665081138\\
-0.1328125	66.1841733797385\\
-0.13232421875	65.9997752888883\\
-0.1318359375	65.8165707257534\\
-0.13134765625	65.6345573295348\\
-0.130859375	65.4537319577773\\
-0.13037109375	65.2740907510445\\
-0.1298828125	65.0956291932274\\
-0.12939453125	64.9183421677576\\
-0.12890625	64.7422240099831\\
-0.12841796875	64.5672685559518\\
-0.1279296875	64.393469187827\\
-0.12744140625	64.2208188761582\\
-0.126953125	64.0493102192047\\
-0.12646484375	63.8789354795049\\
-0.1259765625	63.7096866178709\\
-0.12548828125	63.5415553249776\\
-0.125	63.3745330507041\\
-0.12451171875	63.2086110313737\\
-0.1240234375	63.0437803150349\\
-0.12353515625	62.88003178491\\
-0.123046875	62.7173561811369\\
-0.12255859375	62.5557441209149\\
-0.1220703125	62.3951861171635\\
-0.12158203125	62.2356725957918\\
-0.12109375	62.0771939116744\\
-0.12060546875	61.9197403634207\\
-0.1201171875	61.7633022070165\\
-0.11962890625	61.6078696684173\\
-0.119140625	61.4534329551635\\
-0.11865234375	61.2999822670837\\
-0.1181640625	61.1475078061477\\
-0.11767578125	60.9959997855307\\
-0.1171875	60.845448437936\\
-0.11669921875	60.6958440232355\\
-0.1162109375	60.5471768354682\\
-0.11572265625	60.3994372092446\\
-0.115234375	60.2526155255965\\
-0.11474609375	60.1067022173116\\
-0.1142578125	59.961687773788\\
-0.11376953125	59.8175627454443\\
-0.11328125	59.6743177477108\\
-0.11279296875	59.5319434646394\\
-0.1123046875	59.3904306521525\\
-0.11181640625	59.2497701409584\\
-0.111328125	59.109952839159\\
-0.11083984375	58.9709697345682\\
-0.1103515625	58.8328118967645\\
-0.10986328125	58.6954704788968\\
-0.109375	58.5589367192589\\
-0.10888671875	58.4232019426526\\
-0.1083984375	58.2882575615527\\
-0.10791015625	58.1540950770901\\
-0.107421875	58.0207060798659\\
-0.10693359375	57.8880822506078\\
-0.1064453125	57.7562153606834\\
-0.10595703125	57.6250972724803\\
-0.10546875	57.4947199396607\\
-0.10498046875	57.3650754073047\\
-0.1044921875	57.236155811948\\
-0.10400390625	57.1079533815231\\
-0.103515625	56.9804604352117\\
-0.10302734375	56.8536693832163\\
-0.1025390625	56.7275727264565\\
-0.10205078125	56.6021630561983\\
-0.1015625	56.4774330536179\\
-0.10107421875	56.3533754893122\\
-0.1005859375	56.2299832227546\\
-0.10009765625	56.1072492017046\\
-0.099609375	55.9851664615747\\
-0.09912109375	55.8637281247573\\
-0.0986328125	55.7429273999184\\
-0.09814453125	55.6227575812587\\
-0.09765625	55.5032120477466\\
-0.09716796875	55.384284262327\\
-0.0966796875	55.2659677711064\\
-0.09619140625	55.1482562025199\\
-0.095703125	55.031143266481\\
-0.09521484375	54.9146227535141\\
-0.0947265625	54.7986885338782\\
-0.09423828125	54.6833345566766\\
-0.09375	54.5685548489592\\
-0.09326171875	54.4543435148167\\
-0.0927734375	54.3406947344699\\
-0.09228515625	54.2276027633539\\
-0.091796875	54.1150619311995\\
-0.09130859375	54.0030666411134\\
-0.0908203125	53.8916113686573\\
-0.09033203125	53.7806906609268\\
-0.08984375	53.6702991356327\\
-0.08935546875	53.5604314801851\\
-0.0888671875	53.4510824507774\\
-0.08837890625	53.3422468714788\\
-0.087890625	53.2339196333266\\
-0.08740234375	53.1260956934276\\
-0.0869140625	53.0187700740617\\
-0.08642578125	52.9119378617946\\
-0.0859375	52.8055942065949\\
-0.08544921875	52.6997343209596\\
-0.0849609375	52.5943534790465\\
-0.08447265625	52.4894470158145\\
-0.083984375	52.3850103261717\\
-0.08349609375	52.2810388641326\\
-0.0830078125	52.1775281419831\\
-0.08251953125	52.0744737294545\\
-0.08203125	51.971871252907\\
-0.08154296875	51.8697163945218\\
-0.0810546875	51.7680048915025\\
-0.08056640625	51.6667325352863\\
-0.080078125	51.5658951707648\\
-0.07958984375	51.4654886955133\\
-0.0791015625	51.3655090590316\\
-0.07861328125	51.2659522619929\\
-0.078125	51.1668143555029\\
-0.07763671875	51.0680914403693\\
-0.0771484375	50.9697796663801\\
-0.07666015625	50.8718752315924\\
-0.076171875	50.7743743816307\\
-0.07568359375	50.6772734089943\\
-0.0751953125	50.5805686523759\\
-0.07470703125	50.4842564959884\\
-0.07421875	50.3883333689016\\
-0.07373046875	50.2927957443891\\
-0.0732421875	50.1976401392842\\
-0.07275390625	50.1028631133452\\
-0.072265625	50.0084612686295\\
-0.07177734375	49.9144312488786\\
-0.0712890625	49.8207697389106\\
-0.07080078125	49.7274734640223\\
-0.0703125	49.6345391894011\\
-0.06982421875	49.5419637195443\\
-0.0693359375	49.4497438976886\\
-0.06884765625	49.3578766052477\\
-0.068359375	49.2663587612584\\
-0.06787109375	49.1751873218356\\
-0.0673828125	49.0843592796357\\
-0.06689453125	48.9938716633277\\
-0.06640625	48.9037215370732\\
-0.06591796875	48.8139060000145\\
-0.0654296875	48.7244221857699\\
-0.06494140625	48.6352672619382\\
-0.064453125	48.5464384296094\\
-0.06396484375	48.4579329228846\\
-0.0634765625	48.3697480084023\\
-0.06298828125	48.2818809848731\\
-0.0625	48.1943291826209\\
-0.06201171875	48.1070899631317\\
-0.0615234375	48.0201607186101\\
-0.06103515625	47.9335388715418\\
-0.060546875	47.8472218742637\\
-0.06005859375	47.7612072085406\\
-0.0595703125	47.6754923851489\\
-0.05908203125	47.5900749434665\\
-0.05859375	47.5049524510696\\
-0.05810546875	47.4201225033353\\
-0.0576171875	47.3355827230516\\
-0.05712890625	47.2513307600323\\
-0.056640625	47.1673642907392\\
-0.05615234375	47.0836810179092\\
-0.0556640625	47.0002786701887\\
-0.05517578125	46.917155001772\\
-0.0546875	46.8343077920475\\
-0.05419921875	46.7517348452476\\
-0.0537109375	46.6694339901056\\
-0.05322265625	46.5874030795171\\
-0.052734375	46.5056399902076\\
-0.05224609375	46.4241426224043\\
-0.0517578125	46.3429088995144\\
-0.05126953125	46.2619367678072\\
-0.05078125	46.181224196102\\
-0.05029296875	46.1007691754611\\
-0.0498046875	46.0205697188868\\
-0.04931640625	45.9406238610238\\
-0.048828125	45.8609296578667\\
-0.04833984375	45.7814851864704\\
-0.0478515625	45.7022885446676\\
-0.04736328125	45.6233378507884\\
-0.046875	45.5446312433853\\
-0.04638671875	45.466166880963\\
-0.0458984375	45.3879429417115\\
-0.04541015625	45.3099576232434\\
-0.044921875	45.2322091423364\\
-0.04443359375	45.1546957346786\\
-0.0439453125	45.0774156546181\\
-0.04345703125	45.0003671749168\\
-0.04296875	44.9235485865079\\
-0.04248046875	44.8469581982569\\
-0.0419921875	44.7705943367264\\
-0.04150390625	44.6944553459447\\
-0.041015625	44.6185395871781\\
-0.04052734375	44.5428454387059\\
-0.0400390625	44.4673712955998\\
-0.03955078125	44.3921155695065\\
-0.0390625	44.3170766884328\\
-0.03857421875	44.2422530965352\\
-0.0380859375	44.167643253912\\
-0.03759765625	44.093245636398\\
-0.037109375	44.0190587353644\\
-0.03662109375	43.9450810575189\\
-0.0361328125	43.8713111247112\\
-0.03564453125	43.7977474737403\\
-0.03515625	43.7243886561645\\
-0.03466796875	43.6512332381154\\
-0.0341796875	43.5782798001133\\
-0.03369140625	43.5055269368866\\
-0.033203125	43.432973257193\\
-0.03271484375	43.3606173836437\\
-0.0322265625	43.2884579525307\\
-0.03173828125	43.2164936136557\\
-0.03125	43.1447230301622\\
-0.03076171875	43.0731448783703\\
-0.0302734375	43.0017578476131\\
-0.02978515625	42.9305606400764\\
-0.029296875	42.8595519706402\\
-0.02880859375	42.7887305667226\\
-0.0283203125	42.7180951681265\\
-0.02783203125	42.6476445268879\\
-0.02734375	42.5773774071264\\
-0.02685546875	42.5072925848987\\
-0.0263671875	42.4373888480533\\
-0.02587890625	42.367664996088\\
-0.025390625	42.2981198400088\\
-0.02490234375	42.2287522021916\\
-0.0244140625	42.159560916245\\
-0.02392578125	42.0905448268762\\
-0.0234375	42.0217027897579\\
-0.02294921875	41.9530336713971\\
-0.0224609375	41.8845363490069\\
-0.02197265625	41.8162097103785\\
-0.021484375	41.7480526537565\\
-0.02099609375	41.6800640877148\\
-0.0205078125	41.6122429310353\\
-0.02001953125	41.5445881125875\\
-0.01953125	41.4770985712103\\
-0.01904296875	41.409773255595\\
-0.0185546875	41.3426111241705\\
-0.01806640625	41.2756111449899\\
-0.017578125	41.2087722956183\\
-0.01708984375	41.142093563023\\
-0.0166015625	41.0755739434642\\
-0.01611328125	41.0092124423881\\
-0.015625	40.9430080743212\\
-0.01513671875	40.8769598627656\\
-0.0146484375	40.8110668400967\\
-0.01416015625	40.7453280474615\\
-0.013671875	40.6797425346784\\
-0.01318359375	40.6143093601389\\
-0.0126953125	40.5490275907101\\
-0.01220703125	40.4838963016389\\
-0.01171875	40.4189145764569\\
-0.01123046875	40.3540815068877\\
-0.0107421875	40.2893961927544\\
-0.01025390625	40.2248577418886\\
-0.009765625	40.1604652700412\\
-0.00927734375	40.0962179007938\\
-0.0087890625	40.0321147654714\\
-0.00830078125	39.9681550030563\\
-0.0078125	39.9043377601037\\
-0.00732421875	39.8406621906573\\
-0.0068359375	39.7771274561671\\
-0.00634765625	39.7137327254076\\
-0.005859375	39.6504771743975\\
-0.00537109375	39.5873599863205\\
-0.0048828125	39.5243803514466\\
-0.00439453125	39.4615374670549\\
-0.00390625	39.3988305373577\\
-0.00341796875	39.3362587734245\\
-0.0029296875	39.2738213931084\\
-0.00244140625	39.2115176209727\\
-0.001953125	39.149346688218\\
-0.00146484375	39.0873078326113\\
-0.0009765625	39.0254002984156\\
-0.00048828125	38.9636233363197\\
0	38.9019762033704\\
0.00048828125	38.9636233363197\\
0.0009765625	39.0254002984156\\
0.00146484375	39.0873078326113\\
0.001953125	39.149346688218\\
0.00244140625	39.2115176209727\\
0.0029296875	39.2738213931084\\
0.00341796875	39.3362587734245\\
0.00390625	39.3988305373577\\
0.00439453125	39.4615374670549\\
0.0048828125	39.5243803514466\\
0.00537109375	39.5873599863205\\
0.005859375	39.6504771743975\\
0.00634765625	39.7137327254076\\
0.0068359375	39.7771274561671\\
0.00732421875	39.8406621906573\\
0.0078125	39.9043377601037\\
0.00830078125	39.9681550030563\\
0.0087890625	40.0321147654714\\
0.00927734375	40.0962179007938\\
0.009765625	40.1604652700412\\
0.01025390625	40.2248577418886\\
0.0107421875	40.2893961927544\\
0.01123046875	40.3540815068877\\
0.01171875	40.4189145764569\\
0.01220703125	40.4838963016389\\
0.0126953125	40.5490275907101\\
0.01318359375	40.6143093601389\\
0.013671875	40.6797425346784\\
0.01416015625	40.7453280474615\\
0.0146484375	40.8110668400967\\
0.01513671875	40.8769598627656\\
0.015625	40.9430080743212\\
0.01611328125	41.0092124423881\\
0.0166015625	41.0755739434642\\
0.01708984375	41.142093563023\\
0.017578125	41.2087722956183\\
0.01806640625	41.2756111449899\\
0.0185546875	41.3426111241705\\
0.01904296875	41.409773255595\\
0.01953125	41.4770985712103\\
0.02001953125	41.5445881125875\\
0.0205078125	41.6122429310353\\
0.02099609375	41.6800640877148\\
0.021484375	41.7480526537565\\
0.02197265625	41.8162097103785\\
0.0224609375	41.8845363490069\\
0.02294921875	41.9530336713971\\
0.0234375	42.0217027897579\\
0.02392578125	42.0905448268762\\
0.0244140625	42.159560916245\\
0.02490234375	42.2287522021916\\
0.025390625	42.2981198400088\\
0.02587890625	42.367664996088\\
0.0263671875	42.4373888480533\\
0.02685546875	42.5072925848987\\
0.02734375	42.5773774071264\\
0.02783203125	42.6476445268879\\
0.0283203125	42.7180951681265\\
0.02880859375	42.7887305667226\\
0.029296875	42.8595519706402\\
0.02978515625	42.9305606400764\\
0.0302734375	43.0017578476131\\
0.03076171875	43.0731448783703\\
0.03125	43.1447230301622\\
0.03173828125	43.2164936136557\\
0.0322265625	43.2884579525307\\
0.03271484375	43.3606173836437\\
0.033203125	43.432973257193\\
0.03369140625	43.5055269368866\\
0.0341796875	43.5782798001133\\
0.03466796875	43.6512332381154\\
0.03515625	43.7243886561645\\
0.03564453125	43.7977474737403\\
0.0361328125	43.8713111247112\\
0.03662109375	43.9450810575189\\
0.037109375	44.0190587353644\\
0.03759765625	44.093245636398\\
0.0380859375	44.167643253912\\
0.03857421875	44.2422530965352\\
0.0390625	44.3170766884328\\
0.03955078125	44.3921155695065\\
0.0400390625	44.4673712955998\\
0.04052734375	44.5428454387059\\
0.041015625	44.6185395871781\\
0.04150390625	44.6944553459447\\
0.0419921875	44.7705943367264\\
0.04248046875	44.8469581982569\\
0.04296875	44.9235485865079\\
0.04345703125	45.0003671749168\\
0.0439453125	45.0774156546181\\
0.04443359375	45.1546957346786\\
0.044921875	45.2322091423364\\
0.04541015625	45.3099576232434\\
0.0458984375	45.3879429417115\\
0.04638671875	45.466166880963\\
0.046875	45.5446312433853\\
0.04736328125	45.6233378507884\\
0.0478515625	45.7022885446676\\
0.04833984375	45.7814851864704\\
0.048828125	45.8609296578667\\
0.04931640625	45.9406238610238\\
0.0498046875	46.0205697188868\\
0.05029296875	46.1007691754611\\
0.05078125	46.181224196102\\
0.05126953125	46.2619367678072\\
0.0517578125	46.3429088995144\\
0.05224609375	46.4241426224043\\
0.052734375	46.5056399902076\\
0.05322265625	46.5874030795171\\
0.0537109375	46.6694339901056\\
0.05419921875	46.7517348452476\\
0.0546875	46.8343077920475\\
0.05517578125	46.917155001772\\
0.0556640625	47.0002786701887\\
0.05615234375	47.0836810179092\\
0.056640625	47.1673642907392\\
0.05712890625	47.2513307600323\\
0.0576171875	47.3355827230516\\
0.05810546875	47.4201225033353\\
0.05859375	47.5049524510696\\
0.05908203125	47.5900749434665\\
0.0595703125	47.6754923851489\\
0.06005859375	47.7612072085406\\
0.060546875	47.8472218742637\\
0.06103515625	47.9335388715418\\
0.0615234375	48.0201607186101\\
0.06201171875	48.1070899631317\\
0.0625	48.1943291826209\\
0.06298828125	48.2818809848731\\
0.0634765625	48.3697480084023\\
0.06396484375	48.4579329228846\\
0.064453125	48.5464384296094\\
0.06494140625	48.6352672619382\\
0.0654296875	48.7244221857699\\
0.06591796875	48.8139060000145\\
0.06640625	48.9037215370732\\
0.06689453125	48.9938716633277\\
0.0673828125	49.0843592796357\\
0.06787109375	49.1751873218356\\
0.068359375	49.2663587612584\\
0.06884765625	49.3578766052477\\
0.0693359375	49.4497438976886\\
0.06982421875	49.5419637195443\\
0.0703125	49.6345391894011\\
0.07080078125	49.7274734640223\\
0.0712890625	49.8207697389106\\
0.07177734375	49.9144312488786\\
0.072265625	50.0084612686295\\
0.07275390625	50.1028631133452\\
0.0732421875	50.1976401392842\\
0.07373046875	50.2927957443891\\
0.07421875	50.3883333689016\\
0.07470703125	50.4842564959884\\
0.0751953125	50.5805686523759\\
0.07568359375	50.6772734089943\\
0.076171875	50.7743743816307\\
0.07666015625	50.8718752315924\\
0.0771484375	50.9697796663801\\
0.07763671875	51.0680914403693\\
0.078125	51.1668143555029\\
0.07861328125	51.2659522619929\\
0.0791015625	51.3655090590316\\
0.07958984375	51.4654886955133\\
0.080078125	51.5658951707648\\
0.08056640625	51.6667325352863\\
0.0810546875	51.7680048915025\\
0.08154296875	51.8697163945218\\
0.08203125	51.971871252907\\
0.08251953125	52.0744737294545\\
0.0830078125	52.1775281419831\\
0.08349609375	52.2810388641326\\
0.083984375	52.3850103261717\\
0.08447265625	52.4894470158145\\
0.0849609375	52.5943534790465\\
0.08544921875	52.6997343209596\\
0.0859375	52.8055942065949\\
0.08642578125	52.9119378617946\\
0.0869140625	53.0187700740617\\
0.08740234375	53.1260956934276\\
0.087890625	53.2339196333266\\
0.08837890625	53.3422468714788\\
0.0888671875	53.4510824507774\\
0.08935546875	53.5604314801851\\
0.08984375	53.6702991356327\\
0.09033203125	53.7806906609268\\
0.0908203125	53.8916113686573\\
0.09130859375	54.0030666411134\\
0.091796875	54.1150619311995\\
0.09228515625	54.2276027633539\\
0.0927734375	54.3406947344699\\
0.09326171875	54.4543435148167\\
0.09375	54.5685548489592\\
0.09423828125	54.6833345566766\\
0.0947265625	54.7986885338782\\
0.09521484375	54.9146227535141\\
0.095703125	55.031143266481\\
0.09619140625	55.1482562025199\\
0.0966796875	55.2659677711064\\
0.09716796875	55.384284262327\\
0.09765625	55.5032120477466\\
0.09814453125	55.6227575812587\\
0.0986328125	55.7429273999184\\
0.09912109375	55.8637281247573\\
0.099609375	55.9851664615747\\
0.10009765625	56.1072492017046\\
0.1005859375	56.2299832227546\\
0.10107421875	56.3533754893122\\
0.1015625	56.4774330536179\\
0.10205078125	56.6021630561983\\
0.1025390625	56.7275727264565\\
0.10302734375	56.8536693832163\\
0.103515625	56.9804604352117\\
0.10400390625	57.1079533815231\\
0.1044921875	57.236155811948\\
0.10498046875	57.3650754073047\\
0.10546875	57.4947199396607\\
0.10595703125	57.6250972724803\\
0.1064453125	57.7562153606834\\
0.10693359375	57.8880822506078\\
0.107421875	58.0207060798659\\
0.10791015625	58.1540950770901\\
0.1083984375	58.2882575615527\\
0.10888671875	58.4232019426526\\
0.109375	58.5589367192589\\
0.10986328125	58.6954704788968\\
0.1103515625	58.8328118967645\\
0.11083984375	58.9709697345682\\
0.111328125	59.109952839159\\
0.11181640625	59.2497701409584\\
0.1123046875	59.3904306521525\\
0.11279296875	59.5319434646394\\
0.11328125	59.6743177477108\\
0.11376953125	59.8175627454443\\
0.1142578125	59.961687773788\\
0.11474609375	60.1067022173116\\
0.115234375	60.2526155255965\\
0.11572265625	60.3994372092446\\
0.1162109375	60.5471768354682\\
0.11669921875	60.6958440232355\\
0.1171875	60.845448437936\\
0.11767578125	60.9959997855307\\
0.1181640625	61.1475078061477\\
0.11865234375	61.2999822670837\\
0.119140625	61.4534329551635\\
0.11962890625	61.6078696684173\\
0.1201171875	61.7633022070165\\
0.12060546875	61.9197403634207\\
0.12109375	62.0771939116744\\
0.12158203125	62.2356725957918\\
0.1220703125	62.3951861171635\\
0.12255859375	62.5557441209149\\
0.123046875	62.7173561811369\\
0.12353515625	62.88003178491\\
0.1240234375	63.0437803150349\\
0.12451171875	63.2086110313737\\
0.125	63.3745330507041\\
0.12548828125	63.5415553249776\\
0.1259765625	63.7096866178709\\
0.12646484375	63.8789354795049\\
0.126953125	64.0493102192047\\
0.12744140625	64.2208188761582\\
0.1279296875	64.393469187827\\
0.12841796875	64.5672685559518\\
0.12890625	64.7422240099831\\
0.12939453125	64.9183421677576\\
0.1298828125	65.0956291932274\\
0.13037109375	65.2740907510445\\
0.130859375	65.4537319577773\\
0.13134765625	65.6345573295348\\
0.1318359375	65.8165707257534\\
0.13232421875	65.9997752888883\\
0.1328125	66.1841733797385\\
0.13330078125	66.3697665081138\\
0.1337890625	66.5565552585448\\
0.13427734375	66.7445392107096\\
0.134765625	66.9337168542464\\
0.13525390625	67.1240854975947\\
0.1357421875	67.3156411704975\\
0.13623046875	67.5083785197778\\
0.13671875	67.7022906979896\\
0.13720703125	67.8973692445252\\
0.1376953125	68.0936039587509\\
0.13818359375	68.290982764729\\
0.138671875	68.4894915670731\\
0.13916015625	68.689114097479\\
0.1396484375	68.8898317514676\\
0.14013671875	69.0916234148761\\
0.140625	69.2944652796388\\
0.14111328125	69.4983306484105\\
0.1416015625	69.7031897276035\\
0.14208984375	69.9090094084315\\
0.142578125	70.1157530355954\\
0.14306640625	70.3233801632902\\
0.1435546875	70.5318462982731\\
0.14404296875	70.7411026298092\\
0.14453125	70.9510957464088\\
0.14501953125	71.1617673393722\\
0.1455078125	71.3730538933158\\
0.14599609375	71.5848863639879\\
0.146484375	71.7971898438999\\
0.14697265625	72.0098832164991\\
0.1474609375	72.2228787998826\\
0.14794921875	72.4360819813334\\
0.1484375	72.6493908443085\\
0.14892578125	72.8626957898749\\
0.1494140625	73.0758791550313\\
0.14990234375	73.2888148308151\\
0.150390625	73.501367883622\\
0.15087890625	73.7133941837445\\
0.1513671875	73.9247400457521\\
0.15185546875	74.1352418859994\\
0.15234375	74.34472590326\\
0.15283203125	74.5530077892218\\
0.1533203125	74.7598924763291\\
0.15380859375	74.9651739312291\\
0.154296875	75.1686350028243\\
0.15478515625	75.3700473346527\\
0.1552734375	75.5691713519619\\
0.15576171875	75.765756334407\\
0.15625	75.95954058571\\
0.15673828125	76.1502517118779\\
0.1572265625	76.3376070195862\\
0.15771484375	76.5213140461026\\
0.158203125	76.7010712315394\\
0.15869140625	76.8765687433025\\
0.1591796875	77.0474894612503\\
0.15966796875	77.2135101302636\\
0.16015625	77.3743026846447\\
0.16064453125	77.5295357459851\\
0.1611328125	77.6788762928584\\
0.16162109375	77.8219914969548\\
0.162109375	77.95855071612\\
0.16259765625	78.0882276302816\\
0.1630859375	78.2107025015309\\
0.16357421875	78.3256645348488\\
0.1640625	78.4328143112751\\
0.16455078125	78.5318662609072\\
0.1650390625	78.6225511391792\\
0.16552734375	78.7046184666616\\
0.166015625	78.7778388902799\\
0.16650390625	78.8420064226211\\
0.1669921875	78.8969405159946\\
0.16748046875	78.9424879292382\\
0.16796875	78.978524348003\\
0.16845703125	79.0049557233327\\
0.1689453125	79.0217192987171\\
0.16943359375	79.0287843023075\\
0.169921875	79.0261522883497\\
0.17041015625	79.01385711991\\
0.1708984375	78.991964593307\\
0.17138671875	78.9605717129446\\
0.171875	78.9198056332353\\
0.17236328125	78.8698222916017\\
0.1728515625	78.8108047629275\\
0.17333984375	78.7429613711028\\
0.173828125	78.6665235972512\\
0.17431640625	78.5817438268342\\
0.1748046875	78.4888929790189\\
0.17529296875	78.3882580615833\\
0.17578125	78.2801396932684\\
0.17626953125	78.1648496330777\\
0.1767578125	78.0427083527129\\
0.17724609375	77.9140426843521\\
0.177734375	77.7791835715438\\
0.17822265625	77.638463946257\\
0.1787109375	77.4922167503652\\
0.17919921875	77.3407731151392\\
0.1796875	77.1844607078775\\
0.18017578125	77.0236022506699\\
0.1806640625	76.8585142126082\\
0.18115234375	76.6895056735146\\
0.181640625	76.5168773545324\\
0.18212890625	76.3409208086756\\
0.1826171875	76.161917762683\\
0.18310546875	75.9801396001993\\
0.18359375	75.7958469754232\\
0.18408203125	75.6092895458143\\
0.1845703125	75.4207058122387\\
0.18505859375	75.230323054977\\
0.185546875	75.0383573542841\\
0.18603515625	74.8450136846233\\
0.1865234375	74.6504860722645\\
0.18701171875	74.4549578065961\\
0.1875	74.2586016962087\\
0.18798828125	74.0615803615844\\
0.1884765625	73.8640465569585\\
0.18896484375	73.6661435147035\\
0.189453125	73.4680053063075\\
0.18994140625	73.2697572147167\\
0.1904296875	73.0715161134886\\
0.19091796875	72.8733908488029\\
0.19140625	72.6754826209637\\
0.19189453125	72.477885362524\\
0.1923828125	72.2806861106559\\
0.19287109375	72.083965371799\\
0.193359375	71.8877974769922\\
0.19384765625	71.6922509266408\\
0.1943359375	71.4973887237452\\
0.19482421875	71.3032686948806\\
0.1953125	71.1099437984377\\
0.19580078125	70.917462419812\\
0.1962890625	70.7258686534006\\
0.19677734375	70.5352025713914\\
0.197265625	70.3455004794417\\
0.19775390625	70.156795159446\\
0.1982421875	69.9691160996459\\
0.19873046875	69.7824897124201\\
0.19921875	69.5969395401176\\
0.19970703125	69.4124864493442\\
0.2001953125	69.2291488141346\\
0.20068359375	69.0469426884565\\
0.201171875	68.8658819685088\\
0.20166015625	68.6859785452759\\
0.2021484375	68.5072424478009\\
0.20263671875	68.3296819776375\\
0.203125	68.1533038349299\\
0.20361328125	67.9781132365633\\
0.2041015625	67.8041140268109\\
0.20458984375	67.6313087808947\\
0.205078125	67.4596989018593\\
0.20556640625	67.2892847111407\\
0.2060546875	67.1200655332012\\
0.20654296875	66.9520397745797\\
0.20703125	66.7852049976943\\
0.20751953125	66.6195579897134\\
0.2080078125	66.4550948268008\\
0.20849609375	66.291810934019\\
0.208984375	66.1297011411635\\
0.20947265625	65.9687597347835\\
0.2099609375	65.8089805066321\\
0.21044921875	65.6503567987716\\
0.2109375	65.4928815455525\\
0.21142578125	65.3365473126639\\
0.2119140625	65.1813463334493\\
0.21240234375	65.0272705426627\\
0.212890625	64.874311607836\\
0.21337890625	64.7224609584106\\
0.2138671875	64.5717098127859\\
0.21435546875	64.4220492034167\\
0.21484375	64.2734700000953\\
0.21533203125	64.1259629315332\\
0.2158203125	63.9795186053609\\
0.21630859375	63.834127526648\\
0.216796875	63.6897801150464\\
0.21728515625	63.5464667206453\\
0.2177734375	63.4041776386288\\
0.21826171875	63.2629031228146\\
0.21875	63.1226333981497\\
0.21923828125	62.9833586722356\\
0.2197265625	62.8450691459474\\
0.22021484375	62.7077550232088\\
0.220703125	62.5714065199817\\
0.22119140625	62.4360138725238\\
0.2216796875	62.3015673449629\\
0.22216796875	62.1680572362383\\
0.22265625	62.0354738864491\\
0.22314453125	61.9038076826538\\
0.2236328125	61.7730490641572\\
0.22412109375	61.6431885273206\\
0.224609375	61.5142166299297\\
0.22509765625	61.38612399515\\
0.2255859375	61.2589013150991\\
0.22607421875	61.1325393540629\\
0.2265625	61.0070289513817\\
0.22705078125	60.8823610240272\\
0.2275390625	60.7585265688961\\
0.22802734375	60.6355166648385\\
0.228515625	60.5133224744388\\
0.22900390625	60.3919352455711\\
0.2294921875	60.2713463127421\\
0.22998046875	60.1515470982379\\
0.23046875	60.0325291130905\\
0.23095703125	59.9142839578756\\
0.2314453125	59.7968033233562\\
0.23193359375	59.6800789909816\\
0.232421875	59.5641028332547\\
0.23291015625	59.448866813977\\
0.2333984375	59.3343629883802\\
0.23388671875	59.2205835031543\\
0.234375	59.1075205963809\\
0.23486328125	58.9951665973766\\
0.2353515625	58.883513926458\\
0.23583984375	58.7725550946301\\
0.236328125	58.6622827032105\\
0.23681640625	58.5526894433869\\
0.2373046875	58.4437680957228\\
0.23779296875	58.3355115296078\\
0.23828125	58.2279127026641\\
0.23876953125	58.1209646601083\\
0.2392578125	58.0146605340763\\
0.23974609375	57.9089935429119\\
0.240234375	57.8039569904269\\
0.24072265625	57.6995442651307\\
0.2412109375	57.5957488394368\\
0.24169921875	57.4925642688466\\
0.2421875	57.3899841911139\\
0.24267578125	57.288002325393\\
0.2431640625	57.1866124713717\\
0.24365234375	57.0858085083917\\
0.244140625	56.9855843945587\\
0.24462890625	56.8859341658433\\
0.2451171875	56.7868519351752\\
0.24560546875	56.6883318915315\\
0.24609375	56.590368299021\\
0.24658203125	56.4929554959651\\
0.2470703125	56.3960878939781\\
0.24755859375	56.299759977046\\
0.248046875	56.2039663006061\\
0.24853515625	56.1087014906282\\
0.2490234375	56.013960242698\\
0.24951171875	55.9197373211041\\
0.25	55.8260275579274\\
0.25048828125	55.7328258521372\\
0.2509765625	55.6401271686904\\
0.25146484375	55.5479265376378\\
0.251953125	55.4562190532361\\
0.25244140625	55.3649998730662\\
0.2529296875	55.2742642171594\\
0.25341796875	55.1840073671307\\
0.25390625	55.0942246653191\\
0.25439453125	55.0049115139378\\
0.2548828125	54.9160633742311\\
0.25537109375	54.8276757656408\\
0.255859375	54.7397442649809\\
0.25634765625	54.6522645056221\\
0.2568359375	54.5652321766846\\
0.25732421875	54.4786430222407\\
0.2578125	54.3924928405269\\
0.25830078125	54.3067774831651\\
0.2587890625	54.2214928543939\\
0.25927734375	54.1366349103092\\
0.259765625	54.0521996581146\\
0.26025390625	53.9681831553817\\
0.2607421875	53.8845815093199\\
0.26123046875	53.8013908760563\\
0.26171875	53.7186074599252\\
0.26220703125	53.6362275127669\\
0.2626953125	53.5542473332371\\
0.26318359375	53.4726632661252\\
0.263671875	53.3914717016829\\
0.26416015625	53.3106690749616\\
0.2646484375	53.2302518651604\\
0.26513671875	53.1502165949821\\
0.265625	53.0705598300006\\
0.26611328125	52.9912781780352\\
0.2666015625	52.9123682885366\\
0.26708984375	52.8338268519802\\
0.267578125	52.7556505992697\\
0.26806640625	52.6778363011484\\
0.2685546875	52.6003807676214\\
0.26904296875	52.5232808473842\\
0.26953125	52.4465334272621\\
0.27001953125	52.3701354316571\\
0.2705078125	52.294083822003\\
0.27099609375	52.2183755962303\\
0.271484375	52.1430077882381\\
0.27197265625	52.0679774673745\\
0.2724609375	51.9932817379258\\
0.27294921875	51.9189177386124\\
0.2734375	51.8448826420939\\
0.27392578125	51.7711736544813\\
0.2744140625	51.6977880148565\\
0.27490234375	51.6247229948006\\
0.275390625	51.551975897928\\
0.27587890625	51.4795440594294\\
0.2763671875	51.4074248456208\\
0.27685546875	51.3356156535005\\
0.27734375	51.2641139103129\\
0.27783203125	51.1929170731187\\
0.2783203125	51.1220226283726\\
0.27880859375	51.0514280915072\\
0.279296875	50.9811310065243\\
0.27978515625	50.9111289455913\\
0.2802734375	50.8414195086457\\
0.28076171875	50.7720003230043\\
0.28125	50.7028690429799\\
0.28173828125	50.6340233495031\\
0.2822265625	50.5654609497511\\
0.28271484375	50.4971795767813\\
0.283203125	50.4291769891719\\
0.28369140625	50.3614509706665\\
0.2841796875	50.293999329827\\
0.28466796875	50.2268198996891\\
0.28515625	50.1599105374253\\
0.28564453125	50.0932691240123\\
0.2861328125	50.026893563904\\
0.28662109375	49.9607817847096\\
0.287109375	49.8949317368769\\
0.28759765625	49.8293413933803\\
0.2880859375	49.7640087494139\\
0.28857421875	49.6989318220898\\
0.2890625	49.6341086501405\\
0.28955078125	49.5695372936263\\
0.2900390625	49.5052158336478\\
0.29052734375	49.4411423720617\\
0.291015625	49.3773150312022\\
0.29150390625	49.3137319536065\\
0.2919921875	49.2503913017441\\
0.29248046875	49.1872912577508\\
0.29296875	49.1244300231669\\
0.29345703125	49.0618058186793\\
0.2939453125	48.9994168838672\\
0.29443359375	48.9372614769529\\
0.294921875	48.8753378745551\\
0.29541015625	48.8136443714472\\
0.2958984375	48.7521792803185\\
0.29638671875	48.6909409315393\\
0.296875	48.6299276729304\\
0.29736328125	48.5691378695346\\
0.2978515625	48.5085699033933\\
0.29833984375	48.4482221733255\\
0.298828125	48.3880930947108\\
0.29931640625	48.3281810992749\\
0.2998046875	48.2684846348795\\
0.30029296875	48.2090021653146\\
0.30078125	48.1497321700941\\
0.30126953125	48.0906731442548\\
0.3017578125	48.0318235981581\\
0.30224609375	47.9731820572948\\
0.302734375	47.914747062093\\
0.30322265625	47.856517167729\\
0.3037109375	47.7984909439402\\
0.30419921875	47.7406669748422\\
0.3046875	47.6830438587476\\
0.30517578125	47.6256202079876\\
0.3056640625	47.5683946487368\\
0.30615234375	47.51136582084\\
0.306640625	47.4545323776422\\
0.30712890625	47.3978929858208\\
0.3076171875	47.3414463252196\\
0.30810546875	47.2851910886871\\
0.30859375	47.2291259819149\\
0.30908203125	47.1732497232803\\
0.3095703125	47.1175610436905\\
0.31005859375	47.0620586864287\\
0.310546875	47.0067414070032\\
0.31103515625	46.9516079729985\\
0.3115234375	46.8966571639279\\
0.31201171875	46.8418877710894\\
0.3125	46.787298597423\\
0.31298828125	46.7328884573696\\
0.3134765625	46.6786561767332\\
0.31396484375	46.624600592544\\
0.314453125	46.5707205529238\\
0.31494140625	46.517014916954\\
0.3154296875	46.463482554544\\
0.31591796875	46.4101223463032\\
0.31640625	46.3569331834132\\
0.31689453125	46.3039139675035\\
0.3173828125	46.2510636105274\\
0.31787109375	46.1983810346407\\
0.318359375	46.1458651720817\\
0.31884765625	46.093514965053\\
0.3193359375	46.0413293656048\\
0.31982421875	45.9893073355204\\
0.3203125	45.9374478462022\\
0.32080078125	45.8857498785605\\
0.3212890625	45.8342124229034\\
0.32177734375	45.7828344788278\\
0.322265625	45.7316150551125\\
0.32275390625	45.6805531696128\\
0.3232421875	45.6296478491559\\
0.32373046875	45.5788981294384\\
0.32421875	45.5283030549252\\
0.32470703125	45.4778616787494\\
0.3251953125	45.4275730626134\\
0.32568359375	45.3774362766928\\
0.326171875	45.3274503995392\\
0.32666015625	45.2776145179868\\
0.3271484375	45.2279277270584\\
0.32763671875	45.1783891298737\\
0.328125	45.1289978375588\\
0.32861328125	45.0797529691562\\
0.3291015625	45.0306536515368\\
0.32958984375	44.9816990193127\\
0.330078125	44.9328882147515\\
0.33056640625	44.884220387691\\
0.3310546875	44.8356946954561\\
0.33154296875	44.7873103027758\\
0.33203125	44.7390663817022\\
0.33251953125	44.6909621115298\\
0.3330078125	44.6429966787162\\
0.33349609375	44.5951692768043\\
0.333984375	44.5474791063445\\
0.33447265625	44.4999253748189\\
0.3349609375	44.4525072965662\\
0.33544921875	44.4052240927068\\
0.3359375	44.3580749910708\\
0.33642578125	44.3110592261244\\
0.3369140625	44.2641760388993\\
0.33740234375	44.2174246769224\\
0.337890625	44.1708043941459\\
0.33837890625	44.1243144508788\\
0.3388671875	44.0779541137195\\
0.33935546875	44.0317226554885\\
0.33984375	43.9856193551629\\
0.34033203125	43.9396434978108\\
0.3408203125	43.8937943745275\\
0.34130859375	43.8480712823715\\
0.341796875	43.8024735243023\\
0.34228515625	43.7570004091183\\
0.3427734375	43.7116512513958\\
0.34326171875	43.6664253714285\\
0.34375	43.6213220951682\\
0.34423828125	43.5763407541658\\
0.3447265625	43.5314806855137\\
0.34521484375	43.4867412317876\\
0.345703125	43.4421217409908\\
0.34619140625	43.3976215664976\\
0.3466796875	43.3532400669989\\
0.34716796875	43.3089766064466\\
0.34765625	43.2648305540005\\
0.34814453125	43.2208012839753\\
0.3486328125	43.1768881757874\\
0.34912109375	43.1330906139034\\
0.349609375	43.0894079877889\\
0.35009765625	43.045839691858\\
0.3505859375	43.0023851254228\\
0.35107421875	42.9590436926446\\
0.3515625	42.915814802485\\
0.35205078125	42.8726978686573\\
0.3525390625	42.8296923095797\\
0.35302734375	42.7867975483273\\
0.353515625	42.7440130125867\\
0.35400390625	42.7013381346091\\
0.3544921875	42.658772351166\\
0.35498046875	42.6163151035033\\
0.35546875	42.5739658372981\\
0.35595703125	42.5317240026142\\
0.3564453125	42.4895890538593\\
0.35693359375	42.4475604497422\\
0.357421875	42.4056376532307\\
0.35791015625	42.3638201315098\\
0.3583984375	42.3221073559404\\
0.35888671875	42.2804988020192\\
0.359375	42.2389939493377\\
0.35986328125	42.1975922815428\\
0.3603515625	42.1562932862979\\
0.36083984375	42.1150964552432\\
0.361328125	42.0740012839582\\
0.36181640625	42.0330072719232\\
0.3623046875	41.9921139224824\\
0.36279296875	41.9513207428062\\
0.36328125	41.910627243855\\
0.36376953125	41.8700329403428\\
0.3642578125	41.8295373507017\\
0.36474609375	41.7891399970463\\
0.365234375	41.7488404051389\\
0.36572265625	41.7086381043547\\
0.3662109375	41.668532627648\\
0.36669921875	41.6285235115181\\
0.3671875	41.5886102959762\\
0.36767578125	41.5487925245121\\
0.3681640625	41.5090697440619\\
0.36865234375	41.4694415049755\\
0.369140625	41.4299073609849\\
0.36962890625	41.3904668691722\\
0.3701171875	41.3511195899392\\
0.37060546875	41.3118650869758\\
0.37109375	41.2727029272301\\
0.37158203125	41.2336326808779\\
0.3720703125	41.194653921293\\
0.37255859375	41.1557662250177\\
0.373046875	41.1169691717338\\
0.37353515625	41.0782623442335\\
0.3740234375	41.0396453283913\\
0.37451171875	41.0011177131353\\
0.375	40.9626790904197\\
0.37548828125	40.9243290551971\\
0.3759765625	40.8860672053915\\
0.37646484375	40.8478931418706\\
0.376953125	40.8098064684199\\
0.37744140625	40.771806791716\\
0.3779296875	40.7338937213005\\
0.37841796875	40.6960668695543\\
0.37890625	40.6583258516717\\
0.37939453125	40.6206702856358\\
0.3798828125	40.5830997921928\\
0.38037109375	40.5456139948278\\
0.380859375	40.5082125197402\\
0.38134765625	40.4708949958195\\
0.3818359375	40.4336610546211\\
0.38232421875	40.3965103303433\\
0.3828125	40.3594424598031\\
0.38330078125	40.3224570824138\\
0.3837890625	40.2855538401613\\
0.38427734375	40.2487323775821\\
0.384765625	40.2119923417404\\
0.38525390625	40.1753333822063\\
0.3857421875	40.1387551510333\\
0.38623046875	40.102257302737\\
0.38671875	40.0658394942736\\
0.38720703125	40.0295013850183\\
0.3876953125	39.9932426367447\\
0.38818359375	39.9570629136033\\
0.388671875	39.9209618821017\\
0.38916015625	39.8849392110834\\
0.3896484375	39.8489945717081\\
0.39013671875	39.8131276374318\\
0.390625	39.7773380839865\\
0.39111328125	39.7416255893612\\
0.3916015625	39.705989833782\\
0.39208984375	39.6704304996934\\
0.392578125	39.6349472717393\\
0.39306640625	39.5995398367436\\
0.3935546875	39.5642078836923\\
0.39404296875	39.528951103715\\
0.39453125	39.4937691900663\\
0.39501953125	39.4586618381081\\
0.3955078125	39.4236287452919\\
0.39599609375	39.3886696111406\\
0.396484375	39.3537841372315\\
0.39697265625	39.3189720271788\\
0.3974609375	39.2842329866167\\
0.39794921875	39.2495667231819\\
0.3984375	39.2149729464975\\
0.39892578125	39.1804513681558\\
0.3994140625	39.1460017017022\\
0.39990234375	39.1116236626189\\
0.400390625	39.0773169683083\\
0.40087890625	39.0430813380779\\
0.4013671875	39.0089164931237\\
0.40185546875	38.9748221565149\\
0.40234375	38.9407980531783\\
0.40283203125	38.9068439098832\\
0.4033203125	38.8729594552259\\
0.40380859375	38.8391444196146\\
0.404296875	38.8053985352551\\
0.40478515625	38.7717215361353\\
0.4052734375	38.738113158011\\
0.40576171875	38.7045731383915\\
0.40625	38.671101216525\\
0.40673828125	38.6376971333847\\
0.4072265625	38.6043606316547\\
0.40771484375	38.5710914557156\\
0.408203125	38.5378893516318\\
0.40869140625	38.5047540671366\\
0.4091796875	38.4716853516198\\
0.40966796875	38.4386829561136\\
0.41015625	38.4057466332796\\
0.41064453125	38.3728761373957\\
0.4111328125	38.3400712243432\\
0.41162109375	38.3073316515936\\
0.412109375	38.2746571781961\\
0.41259765625	38.242047564765\\
0.4130859375	38.2095025734673\\
0.41357421875	38.1770219680097\\
0.4140625	38.1446055136273\\
0.41455078125	38.1122529770705\\
0.4150390625	38.0799641265936\\
0.41552734375	38.0477387319426\\
0.416015625	38.0155765643436\\
0.41650390625	37.9834773964907\\
0.4169921875	37.9514410025347\\
0.41748046875	37.9194671580719\\
0.41796875	37.887555640132\\
0.41845703125	37.8557062271675\\
0.4189453125	37.8239186990422\\
0.41943359375	37.7921928370201\\
0.419921875	37.7605284237548\\
0.42041015625	37.7289252432782\\
0.4208984375	37.6973830809899\\
0.42138671875	37.665901723647\\
0.421875	37.6344809593527\\
0.42236328125	37.6031205775465\\
0.4228515625	37.5718203689936\\
0.42333984375	37.5405801257746\\
0.423828125	37.5093996412754\\
0.42431640625	37.4782787101771\\
0.4248046875	37.4472171284462\\
0.42529296875	37.4162146933242\\
0.42578125	37.3852712033183\\
0.42626953125	37.3543864581915\\
0.4267578125	37.323560258953\\
0.42724609375	37.2927924078486\\
0.427734375	37.2620827083511\\
0.42822265625	37.2314309651514\\
0.4287109375	37.2008369841487\\
0.42919921875	37.1703005724416\\
0.4296875	37.139821538319\\
0.43017578125	37.1093996912508\\
0.4306640625	37.0790348418792\\
0.43115234375	37.0487268020097\\
0.431640625	37.0184753846025\\
0.43212890625	36.9882804037633\\
0.4326171875	36.9581416747353\\
0.43310546875	36.9280590138902\\
0.43359375	36.8980322387198\\
0.43408203125	36.8680611678277\\
0.4345703125	36.8381456209209\\
0.43505859375	36.8082854188014\\
0.435546875	36.7784803833584\\
0.43603515625	36.7487303375597\\
0.4365234375	36.7190351054439\\
0.43701171875	36.6893945121125\\
0.4375	36.6598083837218\\
0.43798828125	36.6302765474753\\
0.4384765625	36.6007988316156\\
0.43896484375	36.5713750654171\\
0.439453125	36.542005079178\\
0.43994140625	36.512688704213\\
0.4404296875	36.4834257728456\\
0.44091796875	36.454216118401\\
0.44140625	36.4250595751982\\
0.44189453125	36.3959559785431\\
0.4423828125	36.3669051647213\\
0.44287109375	36.3379069709906\\
0.443359375	36.308961235574\\
0.44384765625	36.2800677976529\\
0.4443359375	36.2512264973597\\
0.44482421875	36.2224371757711\\
0.4453125	36.1936996749013\\
0.44580078125	36.1650138376949\\
0.4462890625	36.1363795080205\\
0.44677734375	36.1077965306637\\
0.447265625	36.0792647513204\\
0.44775390625	36.0507840165906\\
0.4482421875	36.0223541739716\\
0.44873046875	35.9939750718515\\
0.44921875	35.9656465595027\\
0.44970703125	35.937368487076\\
0.4501953125	35.9091407055938\\
0.45068359375	35.8809630669439\\
0.451171875	35.8528354238735\\
0.45166015625	35.8247576299829\\
0.4521484375	35.7967295397194\\
0.45263671875	35.7687510083714\\
0.453125	35.7408218920621\\
0.45361328125	35.7129420477439\\
0.4541015625	35.6851113331922\\
0.45458984375	35.6573296069997\\
0.455078125	35.6295967285706\\
0.45556640625	35.6019125581148\\
0.4560546875	35.5742769566421\\
0.45654296875	35.5466897859568\\
0.45703125	35.519150908652\\
0.45751953125	35.4916601881036\\
0.4580078125	35.4642174884655\\
0.45849609375	35.4368226746636\\
0.458984375	35.4094756123906\\
0.45947265625	35.3821761681005\\
0.4599609375	35.3549242090034\\
0.46044921875	35.32771960306\\
0.4609375	35.3005622189766\\
0.46142578125	35.2734519261996\\
0.4619140625	35.2463885949105\\
0.46240234375	35.2193720960209\\
0.462890625	35.1924023011671\\
0.46337890625	35.1654790827054\\
0.4638671875	35.1386023137067\\
0.46435546875	35.1117718679521\\
0.46484375	35.0849876199272\\
0.46533203125	35.0582494448181\\
0.4658203125	35.0315572185058\\
0.46630859375	35.0049108175619\\
0.466796875	34.9783101192437\\
0.46728515625	34.9517550014892\\
0.4677734375	34.9252453429129\\
0.46826171875	34.8987810228007\\
0.46875	34.8723619211057\\
0.46923828125	34.8459879184431\\
0.4697265625	34.8196588960863\\
0.47021484375	34.7933747359619\\
0.470703125	34.7671353206454\\
0.47119140625	34.7409405333569\\
0.4716796875	34.7147902579563\\
0.47216796875	34.6886843789396\\
0.47265625	34.6626227814338\\
0.47314453125	34.6366053511933\\
0.4736328125	34.6106319745953\\
0.47412109375	34.5847025386355\\
0.474609375	34.558816930924\\
0.47509765625	34.5329750396815\\
0.4755859375	34.5071767537344\\
0.47607421875	34.4814219625115\\
0.4765625	34.4557105560395\\
0.47705078125	34.4300424249391\\
0.4775390625	34.4044174604208\\
0.47802734375	34.3788355542815\\
0.478515625	34.3532965988996\\
0.47900390625	34.3278004872323\\
0.4794921875	34.3023471128107\\
0.47998046875	34.2769363697366\\
0.48046875	34.2515681526782\\
0.48095703125	34.2262423568669\\
0.4814453125	34.2009588780929\\
0.48193359375	34.1757176127021\\
0.482421875	34.1505184575918\\
0.48291015625	34.1253613102075\\
0.4833984375	34.1002460685391\\
0.48388671875	34.0751726311173\\
0.484375	34.0501408970101\\
0.48486328125	34.0251507658188\\
0.4853515625	34.0002021376752\\
0.48583984375	33.9752949132376\\
0.486328125	33.9504289936874\\
0.48681640625	33.9256042807259\\
0.4873046875	33.9008206765704\\
0.48779296875	33.8760780839515\\
0.48828125	33.8513764061088\\
0.48876953125	33.8267155467887\\
0.4892578125	33.80209541024\\
0.48974609375	33.7775159012112\\
0.490234375	33.7529769249474\\
0.49072265625	33.7284783871864\\
0.4912109375	33.7040201941561\\
0.49169921875	33.6796022525709\\
0.4921875	33.6552244696291\\
0.49267578125	33.6308867530088\\
0.4931640625	33.6065890108657\\
0.49365234375	33.5823311518296\\
0.494140625	33.5581130850015\\
0.49462890625	33.53393471995\\
0.4951171875	33.5097959667093\\
0.49560546875	33.4856967357751\\
0.49609375	33.4616369381023\\
0.49658203125	33.4376164851021\\
0.4970703125	33.4136352886383\\
0.49755859375	33.3896932610255\\
0.498046875	33.3657903150251\\
0.49853515625	33.3419263638433\\
0.4990234375	33.3181013211279\\
0.49951171875	33.2943151009652\\
0.5	33.2705676178776\\
0.50048828125	33.2468587868209\\
0.5009765625	33.2231885231809\\
0.50146484375	33.1995567427715\\
0.501953125	33.1759633618311\\
0.50244140625	33.1524082970209\\
0.5029296875	33.128891465421\\
0.50341796875	33.1054127845291\\
0.50390625	33.0819721722566\\
0.50439453125	33.0585695469267\\
0.5048828125	33.0352048272717\\
0.50537109375	33.0118779324303\\
0.505859375	32.988588781945\\
0.50634765625	32.9653372957598\\
0.5068359375	32.9421233942172\\
0.50732421875	32.9189469980563\\
0.5078125	32.8958080284097\\
0.50830078125	32.8727064068016\\
0.5087890625	32.8496420551448\\
0.50927734375	32.8266148957387\\
0.509765625	32.8036248512667\\
0.51025390625	32.7806718447936\\
0.5107421875	32.7577557997637\\
0.51123046875	32.7348766399979\\
0.51171875	32.7120342896918\\
0.51220703125	32.6892286734129\\
0.5126953125	32.666459716099\\
0.51318359375	32.6437273430548\\
0.513671875	32.6210314799509\\
0.51416015625	32.5983720528206\\
0.5146484375	32.5757489880579\\
0.51513671875	32.5531622124155\\
0.515625	32.5306116530022\\
0.51611328125	32.508097237281\\
0.5166015625	32.4856188930669\\
0.51708984375	32.4631765485245\\
0.517578125	32.4407701321661\\
0.51806640625	32.4183995728493\\
0.5185546875	32.3960647997753\\
0.51904296875	32.3737657424862\\
0.51953125	32.3515023308634\\
0.52001953125	32.3292744951253\\
0.5205078125	32.3070821658254\\
0.52099609375	32.2849252738501\\
0.521484375	32.2628037504165\\
0.52197265625	32.2407175270709\\
0.5224609375	32.2186665356863\\
0.52294921875	32.1966507084605\\
0.5234375	32.1746699779145\\
0.52392578125	32.1527242768899\\
0.5244140625	32.1308135385478\\
0.52490234375	32.1089376963658\\
0.525390625	32.087096684137\\
0.52587890625	32.0652904359677\\
0.5263671875	32.0435188862755\\
0.52685546875	32.0217819697876\\
0.52734375	32.0000796215385\\
0.52783203125	31.9784117768687\\
0.5283203125	31.9567783714227\\
0.52880859375	31.9351793411467\\
0.529296875	31.9136146222877\\
0.52978515625	31.8920841513906\\
0.5302734375	31.8705878652974\\
0.53076171875	31.8491257011449\\
0.53125	31.8276975963628\\
0.53173828125	31.8063034886726\\
0.5322265625	31.784943316085\\
0.53271484375	31.763617016899\\
0.533203125	31.7423245296996\\
0.53369140625	31.7210657933564\\
0.5341796875	31.6998407470216\\
0.53466796875	31.6786493301288\\
0.53515625	31.657491482391\\
0.53564453125	31.6363671437988\\
0.5361328125	31.6152762546192\\
0.53662109375	31.5942187553936\\
0.537109375	31.5731945869364\\
0.53759765625	31.5522036903332\\
0.5380859375	31.5312460069394\\
0.53857421875	31.5103214783783\\
0.5390625	31.48943004654\\
0.53955078125	31.4685716535794\\
0.5400390625	31.4477462419149\\
0.54052734375	31.4269537542267\\
0.541015625	31.4061941334552\\
0.54150390625	31.3854673228\\
0.5419921875	31.3647732657175\\
0.54248046875	31.3441119059202\\
0.54296875	31.3234831873749\\
0.54345703125	31.3028870543011\\
0.5439453125	31.2823234511697\\
0.54443359375	31.2617923227015\\
0.544921875	31.2412936138656\\
0.54541015625	31.2208272698783\\
0.5458984375	31.2003932362013\\
0.54638671875	31.1799914585405\\
0.546875	31.1596218828445\\
0.54736328125	31.1392844553032\\
0.5478515625	31.1189791223465\\
0.54833984375	31.0987058306428\\
0.548828125	31.0784645270976\\
0.54931640625	31.0582551588523\\
0.5498046875	31.0380776732828\\
0.55029296875	31.0179320179978\\
0.55078125	30.9978181408382\\
0.55126953125	30.9777359898751\\
0.5517578125	30.9576855134086\\
0.55224609375	30.9376666599669\\
0.552734375	30.9176793783044\\
0.55322265625	30.8977236174011\\
0.5537109375	30.8777993264604\\
0.55419921875	30.8579064549088\\
0.5546875	30.8380449523941\\
0.55517578125	30.818214768784\\
0.5556640625	30.7984158541653\\
0.55615234375	30.7786481588425\\
0.556640625	30.7589116333363\\
0.55712890625	30.7392062283827\\
0.5576171875	30.7195318949317\\
0.55810546875	30.6998885841458\\
0.55859375	30.6802762473994\\
0.55908203125	30.6606948362772\\
0.5595703125	30.6411443025729\\
0.56005859375	30.6216245982883\\
0.560546875	30.6021356756321\\
0.56103515625	30.5826774870187\\
0.5615234375	30.5632499850668\\
0.56201171875	30.5438531225988\\
0.5625	30.524486852639\\
0.56298828125	30.5051511284132\\
0.5634765625	30.4858459033469\\
0.56396484375	30.4665711310645\\
0.564453125	30.4473267653883\\
0.56494140625	30.4281127603372\\
0.5654296875	30.4089290701258\\
0.56591796875	30.3897756491629\\
0.56640625	30.3706524520509\\
0.56689453125	30.3515594335847\\
0.5673828125	30.3324965487501\\
0.56787109375	30.3134637527236\\
0.568359375	30.2944610008704\\
0.56884765625	30.2754882487442\\
0.5693359375	30.2565454520855\\
0.56982421875	30.237632566821\\
0.5703125	30.2187495490626\\
0.57080078125	30.199896355106\\
0.5712890625	30.1810729414299\\
0.57177734375	30.1622792646953\\
0.572265625	30.1435152817439\\
0.57275390625	30.1247809495975\\
0.5732421875	30.1060762254572\\
0.57373046875	30.0874010667018\\
0.57421875	30.0687554308875\\
0.57470703125	30.0501392757465\\
0.5751953125	30.0315525591862\\
0.57568359375	30.0129952392882\\
0.576171875	29.9944672743074\\
0.57666015625	29.9759686226712\\
0.5771484375	29.9574992429782\\
0.57763671875	29.9390590939976\\
0.578125	29.9206481346682\\
0.57861328125	29.9022663240972\\
0.5791015625	29.88391362156\\
0.57958984375	29.8655899864984\\
0.580078125	29.8472953785205\\
0.58056640625	29.8290297573991\\
0.5810546875	29.8107930830716\\
0.58154296875	29.7925853156382\\
0.58203125	29.774406415362\\
0.58251953125	29.7562563426673\\
0.5830078125	29.7381350581394\\
0.58349609375	29.720042522523\\
0.583984375	29.7019786967222\\
0.58447265625	29.683943541799\\
0.5849609375	29.6659370189728\\
0.58544921875	29.6479590896194\\
0.5859375	29.6300097152704\\
0.58642578125	29.6120888576119\\
0.5869140625	29.5941964784841\\
0.58740234375	29.5763325398807\\
0.587890625	29.5584970039472\\
0.58837890625	29.5406898329813\\
0.5888671875	29.5229109894308\\
0.58935546875	29.505160435894\\
0.58984375	29.4874381351182\\
0.59033203125	29.469744049999\\
0.5908203125	29.4520781435798\\
0.59130859375	29.4344403790508\\
0.591796875	29.4168307197482\\
0.59228515625	29.3992491291537\\
0.5927734375	29.3816955708933\\
0.59326171875	29.3641700087372\\
0.59375	29.3466724065982\\
0.59423828125	29.3292027285318\\
0.5947265625	29.3117609387349\\
0.59521484375	29.2943470015453\\
0.595703125	29.2769608814409\\
0.59619140625	29.2596025430391\\
0.5966796875	29.2422719510958\\
0.59716796875	29.2249690705049\\
0.59765625	29.2076938662978\\
0.59814453125	29.190446303642\\
0.5986328125	29.1732263478411\\
0.59912109375	29.156033964334\\
0.599609375	29.1388691186937\\
0.60009765625	29.1217317766273\\
0.6005859375	29.1046219039746\\
0.60107421875	29.0875394667082\\
0.6015625	29.0704844309321\\
0.60205078125	29.0534567628815\\
0.6025390625	29.0364564289221\\
0.60302734375	29.019483395549\\
0.603515625	29.0025376293867\\
0.60400390625	28.9856190971878\\
0.6044921875	28.968727765833\\
0.60498046875	28.9518636023297\\
0.60546875	28.9350265738122\\
0.60595703125	28.9182166475403\\
0.6064453125	28.901433790899\\
0.60693359375	28.8846779713982\\
0.607421875	28.8679491566715\\
0.60791015625	28.8512473144756\\
0.6083984375	28.8345724126904\\
0.60888671875	28.8179244193174\\
0.609375	28.8013033024799\\
0.60986328125	28.7847090304219\\
0.6103515625	28.7681415715078\\
0.61083984375	28.7516008942214\\
0.611328125	28.7350869671659\\
0.61181640625	28.7185997590627\\
0.6123046875	28.7021392387512\\
0.61279296875	28.6857053751882\\
0.61328125	28.669298137447\\
0.61376953125	28.6529174947173\\
0.6142578125	28.636563416304\\
0.61474609375	28.6202358716275\\
0.615234375	28.6039348302222\\
0.61572265625	28.5876602617366\\
0.6162109375	28.5714121359324\\
0.61669921875	28.5551904226841\\
0.6171875	28.5389950919783\\
0.61767578125	28.5228261139135\\
0.6181640625	28.5066834586991\\
0.61865234375	28.4905670966551\\
0.619140625	28.4744769982116\\
0.61962890625	28.4584131339079\\
0.6201171875	28.4423754743928\\
0.62060546875	28.4263639904229\\
0.62109375	28.4103786528632\\
0.62158203125	28.3944194326858\\
0.6220703125	28.3784863009697\\
0.62255859375	28.3625792289004\\
0.623046875	28.346698187769\\
0.62353515625	28.330843148972\\
0.6240234375	28.3150140840108\\
0.62451171875	28.299210964491\\
0.625	28.2834337621221\\
0.62548828125	28.2676824487168\\
0.6259765625	28.2519569961907\\
0.62646484375	28.2362573765616\\
0.626953125	28.2205835619493\\
0.62744140625	28.2049355245749\\
0.6279296875	28.1893132367602\\
0.62841796875	28.1737166709276\\
0.62890625	28.1581457995994\\
0.62939453125	28.1426005953972\\
0.6298828125	28.1270810310415\\
0.63037109375	28.1115870793516\\
0.630859375	28.0961187132447\\
0.63134765625	28.0806759057353\\
0.6318359375	28.0652586299354\\
0.63232421875	28.0498668590535\\
0.6328125	28.0345005663941\\
0.63330078125	28.0191597253578\\
0.6337890625	28.0038443094403\\
0.63427734375	27.9885542922322\\
0.634765625	27.9732896474184\\
0.63525390625	27.9580503487779\\
0.6357421875	27.9428363701833\\
0.63623046875	27.9276476856\\
0.63671875	27.9124842690864\\
0.63720703125	27.8973460947929\\
0.6376953125	27.8822331369618\\
0.63818359375	27.8671453699268\\
0.638671875	27.8520827681126\\
0.63916015625	27.8370453060343\\
0.6396484375	27.8220329582973\\
0.64013671875	27.8070456995964\\
0.640625	27.7920835047161\\
0.64111328125	27.7771463485296\\
0.6416015625	27.7622342059984\\
0.64208984375	27.7473470521725\\
0.642578125	27.7324848621891\\
0.64306640625	27.717647611273\\
0.6435546875	27.7028352747358\\
0.64404296875	27.6880478279756\\
0.64453125	27.6732852464764\\
0.64501953125	27.6585475058082\\
0.6455078125	27.643834581626\\
0.64599609375	27.6291464496701\\
0.646484375	27.614483085765\\
0.64697265625	27.5998444658195\\
0.6474609375	27.5852305658261\\
0.64794921875	27.5706413618609\\
0.6484375	27.5560768300829\\
0.64892578125	27.5415369467338\\
0.6494140625	27.5270216881374\\
0.64990234375	27.5125310306997\\
0.650390625	27.4980649509082\\
0.65087890625	27.4836234253314\\
0.6513671875	27.4692064306188\\
0.65185546875	27.4548139435004\\
0.65234375	27.4404459407862\\
0.65283203125	27.4261023993661\\
0.6533203125	27.4117832962093\\
0.65380859375	27.397488608364\\
0.654296875	27.3832183129574\\
0.65478515625	27.3689723871946\\
0.6552734375	27.3547508083593\\
0.65576171875	27.3405535538123\\
0.65625	27.3263806009922\\
0.65673828125	27.3122319274142\\
0.6572265625	27.2981075106706\\
0.65771484375	27.2840073284296\\
0.658203125	27.2699313584358\\
0.65869140625	27.2558795785091\\
0.6591796875	27.2418519665449\\
0.65966796875	27.2278485005138\\
0.66015625	27.2138691584607\\
0.66064453125	27.1999139185052\\
0.6611328125	27.1859827588407\\
0.66162109375	27.1720756577346\\
0.662109375	27.1581925935274\\
0.66259765625	27.144333544633\\
0.6630859375	27.1304984895378\\
0.66357421875	27.1166874068008\\
0.6640625	27.1029002750532\\
0.66455078125	27.089137072998\\
0.6650390625	27.0753977794098\\
0.66552734375	27.0616823731343\\
0.666015625	27.0479908330883\\
0.66650390625	27.0343231382592\\
0.6669921875	27.0206792677046\\
0.66748046875	27.0070592005524\\
0.66796875	26.9934629160001\\
0.66845703125	26.9798903933146\\
0.6689453125	26.966341611832\\
0.66943359375	26.9528165509573\\
0.669921875	26.9393151901642\\
0.67041015625	26.9258375089946\\
0.6708984375	26.9123834870582\\
0.67138671875	26.8989531040327\\
0.671875	26.8855463396632\\
0.67236328125	26.872163173762\\
0.6728515625	26.858803586208\\
0.67333984375	26.8454675569472\\
0.673828125	26.8321550659915\\
0.67431640625	26.818866093419\\
0.6748046875	26.8056006193737\\
0.67529296875	26.7923586240651\\
0.67578125	26.7791400877678\\
0.67626953125	26.7659449908215\\
0.6767578125	26.7527733136304\\
0.67724609375	26.7396250366637\\
0.677734375	26.7265001404541\\
0.67822265625	26.7133986055987\\
0.6787109375	26.700320412758\\
0.67919921875	26.6872655426562\\
0.6796875	26.6742339760804\\
0.68017578125	26.6612256938807\\
0.6806640625	26.6482406769698\\
0.68115234375	26.6352789063228\\
0.681640625	26.6223403629769\\
0.68212890625	26.6094250280314\\
0.6826171875	26.5965328826469\\
0.68310546875	26.5836639080458\\
0.68359375	26.5708180855112\\
0.68408203125	26.5579953963874\\
0.6845703125	26.5451958220793\\
0.68505859375	26.5324193440523\\
0.685546875	26.5196659438318\\
0.68603515625	26.5069356030033\\
0.6865234375	26.4942283032119\\
0.68701171875	26.4815440261623\\
0.6875	26.4688827536184\\
0.68798828125	26.4562444674029\\
0.6884765625	26.4436291493975\\
0.68896484375	26.4310367815424\\
0.689453125	26.4184673458361\\
0.68994140625	26.4059208243352\\
0.6904296875	26.3933971991541\\
0.69091796875	26.3808964524648\\
0.69140625	26.3684185664969\\
0.69189453125	26.3559635235371\\
0.6923828125	26.3435313059289\\
0.69287109375	26.3311218960727\\
0.693359375	26.3187352764255\\
0.69384765625	26.3063714295005\\
0.6943359375	26.294030337867\\
0.69482421875	26.2817119841501\\
0.6953125	26.2694163510307\\
0.69580078125	26.2571434212452\\
0.6962890625	26.244893177585\\
0.69677734375	26.2326656028968\\
0.697265625	26.2204606800819\\
0.69775390625	26.2082783920964\\
0.6982421875	26.1961187219506\\
0.69873046875	26.1839816527093\\
0.69921875	26.171867167491\\
0.69970703125	26.1597752494681\\
0.7001953125	26.1477058818666\\
0.70068359375	26.1356590479659\\
0.701171875	26.1236347310986\\
0.70166015625	26.1116329146502\\
0.7021484375	26.0996535820592\\
0.70263671875	26.0876967168164\\
0.703125	26.0757623024653\\
0.70361328125	26.0638503226015\\
0.7041015625	26.0519607608724\\
0.70458984375	26.0400936009775\\
0.705078125	26.0282488266679\\
0.70556640625	26.0164264217461\\
0.7060546875	26.0046263700657\\
0.70654296875	25.9928486555317\\
0.70703125	25.9810932620995\\
0.70751953125	25.9693601737757\\
0.7080078125	25.9576493746171\\
0.70849609375	25.9459608487309\\
0.708984375	25.9342945802745\\
0.70947265625	25.9226505534551\\
0.7099609375	25.9110287525299\\
0.71044921875	25.8994291618055\\
0.7109375	25.887851765638\\
0.71142578125	25.8762965484328\\
0.7119140625	25.8647634946443\\
0.71240234375	25.8532525887757\\
0.712890625	25.8417638153792\\
0.71337890625	25.8302971590553\\
0.7138671875	25.8188526044528\\
0.71435546875	25.807430136269\\
0.71484375	25.796029739249\\
0.71533203125	25.7846513981857\\
0.7158203125	25.77329509792\\
0.71630859375	25.76196082334\\
0.716796875	25.7506485593814\\
0.71728515625	25.7393582910267\\
0.7177734375	25.728090003306\\
0.71826171875	25.7168436812958\\
0.71875	25.7056193101194\\
0.71923828125	25.6944168749467\\
0.7197265625	25.6832363609939\\
0.72021484375	25.6720777535236\\
0.720703125	25.6609410378442\\
0.72119140625	25.64982619931\\
0.7216796875	25.6387332233213\\
0.72216796875	25.6276620953237\\
0.72265625	25.6166128008083\\
0.72314453125	25.6055853253115\\
0.7236328125	25.5945796544148\\
0.72412109375	25.5835957737447\\
0.724609375	25.5726336689723\\
0.72509765625	25.5616933258136\\
0.7255859375	25.550774730029\\
0.72607421875	25.5398778674233\\
0.7265625	25.5290027238453\\
0.72705078125	25.5181492851881\\
0.7275390625	25.5073175373885\\
0.72802734375	25.4965074664273\\
0.728515625	25.4857190583287\\
0.72900390625	25.4749522991603\\
0.7294921875	25.4642071750333\\
0.72998046875	25.4534836721018\\
0.73046875	25.4427817765632\\
0.73095703125	25.4321014746575\\
0.7314453125	25.4214427526676\\
0.73193359375	25.410805596919\\
0.732421875	25.4001899937797\\
0.73291015625	25.38959592966\\
0.7333984375	25.3790233910123\\
0.73388671875	25.3684723643312\\
0.734375	25.3579428361531\\
0.73486328125	25.3474347930561\\
0.7353515625	25.3369482216603\\
0.73583984375	25.3264831086269\\
0.736328125	25.3160394406586\\
0.73681640625	25.3056172044995\\
0.7373046875	25.2952163869346\\
0.73779296875	25.2848369747901\\
0.73828125	25.2744789549327\\
0.73876953125	25.2641423142701\\
0.7392578125	25.2538270397505\\
0.73974609375	25.2435331183626\\
0.740234375	25.2332605371354\\
0.74072265625	25.2230092831379\\
0.7412109375	25.2127793434796\\
0.74169921875	25.2025707053095\\
0.7421875	25.1923833558167\\
0.74267578125	25.1822172822299\\
0.7431640625	25.1720724718174\\
0.74365234375	25.1619489118869\\
0.744140625	25.1518465897855\\
0.74462890625	25.1417654928994\\
0.7451171875	25.1317056086539\\
0.74560546875	25.1216669245133\\
0.74609375	25.1116494279809\\
0.74658203125	25.1016531065984\\
0.7470703125	25.0916779479462\\
0.74755859375	25.0817239396433\\
0.748046875	25.0717910693471\\
0.74853515625	25.0618793247529\\
0.7490234375	25.0519886935946\\
0.74951171875	25.0421191636438\\
0.75	25.0322707227101\\
0.75048828125	25.0224433586408\\
0.7509765625	25.012637059321\\
0.75146484375	25.0028518126734\\
0.751953125	24.9930876066578\\
0.75244140625	24.9833444292718\\
0.7529296875	24.9736222685499\\
0.75341796875	24.9639211125639\\
0.75390625	24.9542409494223\\
0.75439453125	24.9445817672709\\
0.7548828125	24.934943554292\\
0.75537109375	24.9253262987046\\
0.755859375	24.9157299887645\\
0.75634765625	24.9061546127635\\
0.7568359375	24.8966001590302\\
0.75732421875	24.8870666159292\\
0.7578125	24.8775539718612\\
0.75830078125	24.8680622152633\\
0.7587890625	24.858591334608\\
0.75927734375	24.8491413184039\\
0.759765625	24.8397121551955\\
0.76025390625	24.8303038335625\\
0.7607421875	24.8209163421205\\
0.76123046875	24.8115496695202\\
0.76171875	24.8022038044479\\
0.76220703125	24.7928787356249\\
0.7626953125	24.7835744518076\\
0.76318359375	24.7742909417876\\
0.763671875	24.7650281943912\\
0.76416015625	24.7557861984797\\
0.7646484375	24.746564942949\\
0.76513671875	24.7373644167296\\
0.765625	24.7281846087866\\
0.76611328125	24.7190255081196\\
0.7666015625	24.7098871037622\\
0.76708984375	24.7007693847825\\
0.767578125	24.6916723402828\\
0.76806640625	24.6825959593992\\
0.7685546875	24.673540231302\\
0.76904296875	24.6645051451951\\
0.76953125	24.6554906903163\\
0.77001953125	24.6464968559371\\
0.7705078125	24.6375236313626\\
0.77099609375	24.6285710059312\\
0.771484375	24.6196389690148\\
0.77197265625	24.6107275100187\\
0.7724609375	24.6018366183813\\
0.77294921875	24.5929662835741\\
0.7734375	24.5841164951018\\
0.77392578125	24.5752872425018\\
0.7744140625	24.5664785153445\\
0.77490234375	24.5576903032332\\
0.775390625	24.5489225958036\\
0.77587890625	24.5401753827241\\
0.7763671875	24.5314486536958\\
0.77685546875	24.522742398452\\
0.77734375	24.5140566067584\\
0.77783203125	24.5053912684128\\
0.7783203125	24.4967463732456\\
0.77880859375	24.4881219111188\\
0.779296875	24.4795178719267\\
0.77978515625	24.4709342455954\\
0.7802734375	24.4623710220829\\
0.78076171875	24.4538281913788\\
0.78125	24.4453057435045\\
0.78173828125	24.436803668513\\
0.7822265625	24.4283219564888\\
0.78271484375	24.4198605975477\\
0.783203125	24.411419581837\\
0.78369140625	24.4029988995353\\
0.7841796875	24.3945985408521\\
0.78466796875	24.3862184960283\\
0.78515625	24.3778587553357\\
0.78564453125	24.3695193090772\\
0.7861328125	24.3612001475862\\
0.78662109375	24.3529012612274\\
0.787109375	24.3446226403958\\
0.78759765625	24.3363642755173\\
0.7880859375	24.3281261570482\\
0.78857421875	24.3199082754753\\
0.7890625	24.3117106213161\\
0.78955078125	24.303533185118\\
0.7900390625	24.2953759574589\\
0.79052734375	24.2872389289469\\
0.791015625	24.2791220902203\\
0.79150390625	24.2710254319471\\
0.7919921875	24.2629489448257\\
0.79248046875	24.2548926195841\\
0.79296875	24.2468564469802\\
0.79345703125	24.2388404178017\\
0.7939453125	24.230844522866\\
0.79443359375	24.2228687530199\\
0.794921875	24.2149130991402\\
0.79541015625	24.2069775521326\\
0.7958984375	24.1990621029326\\
0.79638671875	24.1911667425048\\
0.796875	24.1832914618433\\
0.79736328125	24.1754362519712\\
0.7978515625	24.1676011039407\\
0.79833984375	24.1597860088333\\
0.798828125	24.1519909577591\\
0.79931640625	24.1442159418575\\
0.7998046875	24.1364609522966\\
0.80029296875	24.1287259802731\\
0.80078125	24.1210110170128\\
0.80126953125	24.1133160537698\\
0.8017578125	24.1056410818269\\
0.80224609375	24.0979860924956\\
0.802734375	24.0903510771157\\
0.80322265625	24.0827360270553\\
0.8037109375	24.0751409337109\\
0.80419921875	24.0675657885074\\
0.8046875	24.0600105828978\\
0.80517578125	24.0524753083631\\
0.8056640625	24.0449599564125\\
0.80615234375	24.0374645185833\\
0.806640625	24.0299889864407\\
0.80712890625	24.0225333515776\\
0.8076171875	24.0150976056149\\
0.80810546875	24.0076817402012\\
0.80859375	24.0002857470129\\
0.80908203125	23.9929096177539\\
0.8095703125	23.9855533441558\\
0.81005859375	23.9782169179777\\
0.810546875	23.9709003310059\\
0.81103515625	23.9636035750546\\
0.8115234375	23.9563266419649\\
0.81201171875	23.9490695236053\\
0.8125	23.9418322118717\\
0.81298828125	23.934614698687\\
0.8134765625	23.9274169760011\\
0.81396484375	23.9202390357913\\
0.814453125	23.9130808700615\\
0.81494140625	23.9059424708428\\
0.8154296875	23.8988238301931\\
0.81591796875	23.891724940197\\
0.81640625	23.8846457929661\\
0.81689453125	23.8775863806385\\
0.8173828125	23.8705466953791\\
0.81787109375	23.8635267293792\\
0.818359375	23.856526474857\\
0.81884765625	23.8495459240567\\
0.8193359375	23.8425850692494\\
0.81982421875	23.8356439027322\\
0.8203125	23.8287224168289\\
0.82080078125	23.8218206038892\\
0.8212890625	23.8149384562892\\
0.82177734375	23.8080759664313\\
0.822265625	23.8012331267436\\
0.82275390625	23.7944099296808\\
0.8232421875	23.7876063677232\\
0.82373046875	23.7808224333772\\
0.82421875	23.774058119175\\
0.82470703125	23.7673134176749\\
0.8251953125	23.7605883214607\\
0.82568359375	23.7538828231422\\
0.826171875	23.7471969153548\\
0.82666015625	23.7405305907593\\
0.8271484375	23.7338838420426\\
0.82763671875	23.7272566619168\\
0.828125	23.7206490431196\\
0.82861328125	23.7140609784141\\
0.8291015625	23.7074924605888\\
0.82958984375	23.7009434824577\\
0.830078125	23.6944140368598\\
0.83056640625	23.6879041166597\\
0.8310546875	23.6814137147471\\
0.83154296875	23.6749428240366\\
0.83203125	23.6684914374683\\
0.83251953125	23.662059548007\\
0.8330078125	23.6556471486429\\
0.83349609375	23.6492542323909\\
0.833984375	23.6428807922909\\
0.83447265625	23.6365268214076\\
0.8349609375	23.6301923128306\\
0.83544921875	23.6238772596744\\
0.8359375	23.6175816550781\\
0.83642578125	23.6113054922054\\
0.8369140625	23.6050487642449\\
0.83740234375	23.5988114644096\\
0.837890625	23.5925935859372\\
0.83837890625	23.5863951220897\\
0.8388671875	23.5802160661539\\
0.83935546875	23.5740564114408\\
0.83984375	23.5679161512858\\
0.84033203125	23.5617952790486\\
0.8408203125	23.5556937881134\\
0.84130859375	23.5496116718884\\
0.841796875	23.5435489238062\\
0.84228515625	23.5375055373235\\
0.8427734375	23.531481505921\\
0.84326171875	23.5254768231038\\
0.84375	23.5194914824007\\
0.84423828125	23.5135254773647\\
0.8447265625	23.5075788015728\\
0.84521484375	23.5016514486256\\
0.845703125	23.495743412148\\
0.84619140625	23.4898546857885\\
0.8466796875	23.4839852632193\\
0.84716796875	23.4781351381366\\
0.84765625	23.4723043042602\\
0.84814453125	23.4664927553335\\
0.8486328125	23.4607004851237\\
0.84912109375	23.4549274874214\\
0.849609375	23.4491737560409\\
0.85009765625	23.4434392848201\\
0.8505859375	23.43772406762\\
0.85107421875	23.4320280983255\\
0.8515625	23.4263513708446\\
0.85205078125	23.4206938791087\\
0.8525390625	23.4150556170727\\
0.85302734375	23.4094365787146\\
0.853515625	23.4038367580357\\
0.85400390625	23.3982561490606\\
0.8544921875	23.3926947458369\\
0.85498046875	23.3871525424355\\
0.85546875	23.3816295329502\\
0.85595703125	23.3761257114982\\
0.8564453125	23.3706410722194\\
0.85693359375	23.3651756092768\\
0.857421875	23.3597293168565\\
0.85791015625	23.3543021891673\\
0.8583984375	23.3488942204409\\
0.85888671875	23.343505404932\\
0.859375	23.338135736918\\
0.85986328125	23.3327852106991\\
0.8603515625	23.3274538205983\\
0.86083984375	23.3221415609613\\
0.861328125	23.3168484261563\\
0.86181640625	23.3115744105743\\
0.8623046875	23.3063195086289\\
0.86279296875	23.3010837147563\\
0.86328125	23.2958670234151\\
0.86376953125	23.2906694290865\\
0.8642578125	23.2854909262741\\
0.86474609375	23.2803315095041\\
0.865234375	23.2751911733249\\
0.86572265625	23.2700699123074\\
0.8662109375	23.2649677210449\\
0.86669921875	23.2598845941527\\
0.8671875	23.2548205262687\\
0.86767578125	23.2497755120529\\
0.8681640625	23.2447495461875\\
0.86865234375	23.2397426233769\\
0.869140625	23.2347547383478\\
0.86962890625	23.2297858858487\\
0.8701171875	23.2248360606503\\
0.87060546875	23.2199052575455\\
0.87109375	23.2149934713492\\
0.87158203125	23.2101006968981\\
0.8720703125	23.2052269290509\\
0.87255859375	23.2003721626884\\
0.873046875	23.1955363927132\\
0.87353515625	23.1907196140497\\
0.8740234375	23.1859218216443\\
0.87451171875	23.181143010465\\
0.875	23.1763831755017\\
0.87548828125	23.1716423117661\\
0.8759765625	23.1669204142916\\
0.87646484375	23.1622174781331\\
0.876953125	23.1575334983674\\
0.87744140625	23.1528684700929\\
0.8779296875	23.1482223884295\\
0.87841796875	23.1435952485187\\
0.87890625	23.1389870455236\\
0.87939453125	23.1343977746289\\
0.8798828125	23.1298274310405\\
0.88037109375	23.1252760099861\\
0.880859375	23.1207435067147\\
0.88134765625	23.1162299164966\\
0.8818359375	23.1117352346237\\
0.88232421875	23.107259456409\\
0.8828125	23.1028025771871\\
0.88330078125	23.0983645923138\\
0.8837890625	23.0939454971661\\
0.88427734375	23.0895452871422\\
0.884765625	23.0851639576617\\
0.88525390625	23.0808015041653\\
0.8857421875	23.0764579221149\\
0.88623046875	23.0721332069935\\
0.88671875	23.0678273543052\\
0.88720703125	23.0635403595754\\
0.8876953125	23.0592722183503\\
0.88818359375	23.0550229261972\\
0.888671875	23.0507924787045\\
0.88916015625	23.0465808714816\\
0.8896484375	23.0423881001588\\
0.89013671875	23.0382141603873\\
0.890625	23.0340590478393\\
0.89111328125	23.0299227582079\\
0.8916015625	23.0258052872071\\
0.89208984375	23.0217066305716\\
0.892578125	23.017626784057\\
0.89306640625	23.0135657434398\\
0.8935546875	23.0095235045171\\
0.89404296875	23.0055000631068\\
0.89453125	23.0014954150475\\
0.89501953125	22.9975095561988\\
0.8955078125	22.9935424824405\\
0.89599609375	22.9895941896734\\
0.896484375	22.9856646738188\\
0.89697265625	22.9817539308186\\
0.8974609375	22.9778619566353\\
0.89794921875	22.9739887472522\\
0.8984375	22.9701342986726\\
0.89892578125	22.966298606921\\
0.8994140625	22.9624816680418\\
0.89990234375	22.9586834781002\\
0.900390625	22.9549040331819\\
0.90087890625	22.9511433293928\\
0.9013671875	22.9474013628594\\
0.90185546875	22.9436781297286\\
0.90234375	22.9399736261674\\
0.90283203125	22.9362878483635\\
0.9033203125	22.9326207925247\\
0.90380859375	22.9289724548793\\
0.904296875	22.9253428316757\\
0.90478515625	22.9217319191826\\
0.9052734375	22.918139713689\\
0.90576171875	22.9145662115042\\
0.90625	22.9110114089575\\
0.90673828125	22.9074753023987\\
0.9072265625	22.9039578881973\\
0.90771484375	22.9004591627434\\
0.908203125	22.896979122447\\
0.90869140625	22.8935177637382\\
0.9091796875	22.8900750830674\\
0.90966796875	22.8866510769046\\
0.91015625	22.8832457417405\\
0.91064453125	22.8798590740852\\
0.9111328125	22.8764910704693\\
0.91162109375	22.8731417274431\\
0.912109375	22.8698110415769\\
0.91259765625	22.8664990094612\\
0.9130859375	22.8632056277061\\
0.91357421875	22.8599308929418\\
0.9140625	22.8566748018184\\
0.91455078125	22.8534373510059\\
0.9150390625	22.8502185371941\\
0.91552734375	22.8470183570926\\
0.916015625	22.843836807431\\
0.91650390625	22.8406738849586\\
0.9169921875	22.8375295864444\\
0.91748046875	22.8344039086775\\
0.91796875	22.8312968484663\\
0.91845703125	22.8282084026394\\
0.9189453125	22.8251385680448\\
0.91943359375	22.8220873415503\\
0.919921875	22.8190547200435\\
0.92041015625	22.8160407004316\\
0.9208984375	22.8130452796414\\
0.92138671875	22.8100684546195\\
0.921875	22.8071102223319\\
0.92236328125	22.8041705797644\\
0.9228515625	22.8012495239224\\
0.92333984375	22.7983470518307\\
0.923828125	22.7954631605339\\
0.92431640625	22.7925978470959\\
0.9248046875	22.7897511086004\\
0.92529296875	22.7869229421504\\
0.92578125	22.7841133448685\\
0.92626953125	22.7813223138968\\
0.9267578125	22.7785498463967\\
0.92724609375	22.7757959395495\\
0.927734375	22.7730605905553\\
0.92822265625	22.7703437966342\\
0.9287109375	22.7676455550255\\
0.92919921875	22.7649658629877\\
0.9296875	22.762304717799\\
0.93017578125	22.7596621167569\\
0.9306640625	22.7570380571781\\
0.93115234375	22.7544325363987\\
0.931640625	22.7518455517742\\
0.93212890625	22.7492771006795\\
0.9326171875	22.7467271805085\\
0.93310546875	22.7441957886747\\
0.93359375	22.7416829226108\\
0.93408203125	22.7391885797685\\
0.9345703125	22.7367127576191\\
0.93505859375	22.7342554536529\\
0.935546875	22.7318166653796\\
0.93603515625	22.729396390328\\
0.9365234375	22.7269946260462\\
0.93701171875	22.7246113701011\\
0.9375	22.7222466200794\\
0.93798828125	22.7199003735865\\
0.9384765625	22.7175726282469\\
0.93896484375	22.7152633817047\\
0.939453125	22.7129726316227\\
0.93994140625	22.7107003756829\\
0.9404296875	22.7084466115864\\
0.94091796875	22.7062113370535\\
0.94140625	22.7039945498235\\
0.94189453125	22.7017962476546\\
0.9423828125	22.6996164283244\\
0.94287109375	22.6974550896293\\
0.943359375	22.6953122293847\\
0.94384765625	22.6931878454251\\
0.9443359375	22.691081935604\\
0.94482421875	22.6889944977939\\
0.9453125	22.6869255298862\\
0.94580078125	22.6848750297915\\
0.9462890625	22.6828429954391\\
0.94677734375	22.6808294247773\\
0.947265625	22.6788343157736\\
0.94775390625	22.676857666414\\
0.9482421875	22.6748994747038\\
0.94873046875	22.672959738667\\
0.94921875	22.6710384563465\\
0.94970703125	22.6691356258043\\
0.9501953125	22.667251245121\\
0.95068359375	22.6653853123962\\
0.951171875	22.6635378257485\\
0.95166015625	22.6617087833151\\
0.9521484375	22.6598981832522\\
0.95263671875	22.6581060237347\\
0.953125	22.6563323029565\\
};
\addplot [color=red,solid]
  table[row sep=crcr]{0.953125	22.6563323029565\\
0.95361328125	22.6545770191302\\
0.9541015625	22.6528401704873\\
0.95458984375	22.6511217552779\\
0.955078125	22.6494217717711\\
0.95556640625	22.6477402182547\\
0.9560546875	22.6460770930352\\
0.95654296875	22.6444323944379\\
0.95703125	22.642806120807\\
0.95751953125	22.6411982705053\\
0.9580078125	22.6396088419142\\
0.95849609375	22.6380378334342\\
0.958984375	22.6364852434841\\
0.95947265625	22.6349510705018\\
0.9599609375	22.6334353129437\\
0.96044921875	22.6319379692847\\
0.9609375	22.6304590380189\\
0.96142578125	22.6289985176586\\
0.9619140625	22.6275564067351\\
0.96240234375	22.6261327037981\\
0.962890625	22.6247274074161\\
0.96337890625	22.6233405161763\\
0.9638671875	22.6219720286845\\
0.96435546875	22.6206219435651\\
0.96484375	22.619290259461\\
0.96533203125	22.6179769750341\\
0.9658203125	22.6166820889646\\
0.96630859375	22.6154055999515\\
0.966796875	22.6141475067121\\
0.96728515625	22.6129078079827\\
0.9677734375	22.6116865025179\\
0.96826171875	22.6104835890909\\
0.96875	22.6092990664938\\
0.96923828125	22.6081329335368\\
0.9697265625	22.6069851890491\\
0.97021484375	22.6058558318781\\
0.970703125	22.6047448608901\\
0.97119140625	22.6036522749695\\
0.9716796875	22.6025780730198\\
0.97216796875	22.6015222539625\\
0.97265625	22.6004848167381\\
0.97314453125	22.5994657603053\\
0.9736328125	22.5984650836414\\
0.97412109375	22.5974827857423\\
0.974609375	22.5965188656224\\
0.97509765625	22.5955733223146\\
0.9755859375	22.5946461548702\\
0.97607421875	22.593737362359\\
0.9765625	22.5928469438696\\
0.97705078125	22.5919748985087\\
0.9775390625	22.5911212254016\\
0.97802734375	22.5902859236922\\
0.978515625	22.5894689925429\\
0.97900390625	22.5886704311343\\
0.9794921875	22.5878902386657\\
0.97998046875	22.5871284143547\\
0.98046875	22.5863849574377\\
0.98095703125	22.5856598671691\\
0.9814453125	22.5849531428221\\
0.98193359375	22.5842647836882\\
0.982421875	22.5835947890774\\
0.98291015625	22.582943158318\\
0.9833984375	22.5823098907568\\
0.98388671875	22.5816949857593\\
0.984375	22.5810984427091\\
0.98486328125	22.5805202610083\\
0.9853515625	22.5799604400775\\
0.98583984375	22.5794189793558\\
0.986328125	22.5788958783004\\
0.98681640625	22.5783911363873\\
0.9873046875	22.5779047531107\\
0.98779296875	22.5774367279832\\
0.98828125	22.5769870605359\\
0.98876953125	22.5765557503184\\
0.9892578125	22.5761427968984\\
0.98974609375	22.5757481998624\\
0.990234375	22.5753719588148\\
0.99072265625	22.5750140733789\\
0.9912109375	22.5746745431962\\
0.99169921875	22.5743533679264\\
0.9921875	22.574050547248\\
0.99267578125	22.5737660808575\\
0.9931640625	22.57349996847\\
0.99365234375	22.5732522098191\\
0.994140625	22.5730228046564\\
0.99462890625	22.5728117527522\\
0.9951171875	22.5726190538952\\
0.99560546875	22.5724447078923\\
0.99609375	22.572288714569\\
0.99658203125	22.572151073769\\
0.9970703125	22.5720317853544\\
0.99755859375	22.5719308492057\\
0.998046875	22.571848265222\\
0.99853515625	22.5717840333203\\
0.9990234375	22.5717381534366\\
0.99951171875	22.5717106255247\\
};
\addlegendentry{AR(4) Model};

\addplot [color=green,solid,forget plot]
  table[row sep=crcr]{-1	12.3694656052897\\
-0.99951171875	12.3694799616387\\
-0.9990234375	12.3695230307156\\
-0.99853515625	12.3695948126094\\
-0.998046875	12.3696953074685\\
-0.99755859375	12.3698245155006\\
-0.9970703125	12.369982436973\\
-0.99658203125	12.3701690722121\\
-0.99609375	12.3703844216038\\
-0.99560546875	12.3706284855935\\
-0.9951171875	12.3709012646857\\
-0.99462890625	12.3712027594445\\
-0.994140625	12.3715329704934\\
-0.99365234375	12.3718918985151\\
-0.9931640625	12.3722795442519\\
-0.99267578125	12.3726959085053\\
-0.9921875	12.3731409921364\\
-0.99169921875	12.3736147960656\\
-0.9912109375	12.3741173212729\\
-0.99072265625	12.3746485687973\\
-0.990234375	12.3752085397378\\
-0.98974609375	12.3757972352524\\
-0.9892578125	12.3764146565588\\
-0.98876953125	12.3770608049341\\
-0.98828125	12.3777356817147\\
-0.98779296875	12.3784392882968\\
-0.9873046875	12.3791716261358\\
-0.98681640625	12.3799326967468\\
-0.986328125	12.3807225017043\\
-0.98583984375	12.3815410426423\\
-0.9853515625	12.3823883212545\\
-0.98486328125	12.383264339294\\
-0.984375	12.3841690985734\\
-0.98388671875	12.3851026009652\\
-0.9833984375	12.386064848401\\
-0.98291015625	12.3870558428725\\
-0.982421875	12.3880755864306\\
-0.98193359375	12.3891240811861\\
-0.9814453125	12.3902013293094\\
-0.98095703125	12.3913073330304\\
-0.98046875	12.3924420946388\\
-0.97998046875	12.3936056164842\\
-0.9794921875	12.3947979009755\\
-0.97900390625	12.3960189505816\\
-0.978515625	12.397268767831\\
-0.97802734375	12.3985473553123\\
-0.9775390625	12.3998547156734\\
-0.97705078125	12.4011908516223\\
-0.9765625	12.4025557659268\\
-0.97607421875	12.4039494614146\\
-0.9755859375	12.405371940973\\
-0.97509765625	12.4068232075495\\
-0.974609375	12.4083032641513\\
-0.97412109375	12.4098121138457\\
-0.9736328125	12.4113497597597\\
-0.97314453125	12.4129162050806\\
-0.97265625	12.4145114530555\\
-0.97216796875	12.4161355069915\\
-0.9716796875	12.4177883702558\\
-0.97119140625	12.4194700462757\\
-0.970703125	12.4211805385387\\
-0.97021484375	12.4229198505921\\
-0.9697265625	12.4246879860436\\
-0.96923828125	12.426484948561\\
-0.96875	12.4283107418724\\
-0.96826171875	12.430165369766\\
-0.9677734375	12.4320488360904\\
-0.96728515625	12.4339611447542\\
-0.966796875	12.4359022997267\\
-0.96630859375	12.4378723050372\\
-0.9658203125	12.4398711647756\\
-0.96533203125	12.4418988830922\\
-0.96484375	12.4439554641975\\
-0.96435546875	12.4460409123626\\
-0.9638671875	12.4481552319193\\
-0.96337890625	12.4502984272596\\
-0.962890625	12.4524705028363\\
-0.96240234375	12.4546714631626\\
-0.9619140625	12.4569013128124\\
-0.96142578125	12.4591600564204\\
-0.9609375	12.4614476986818\\
-0.96044921875	12.4637642443525\\
-0.9599609375	12.4661096982494\\
-0.95947265625	12.4684840652501\\
-0.958984375	12.4708873502928\\
-0.95849609375	12.4733195583769\\
-0.9580078125	12.4757806945624\\
-0.95751953125	12.4782707639707\\
-0.95703125	12.4807897717836\\
-0.95654296875	12.4833377232444\\
-0.9560546875	12.4859146236571\\
-0.95556640625	12.4885204783871\\
-0.955078125	12.4911552928608\\
-0.95458984375	12.4938190725657\\
-0.9541015625	12.4965118230506\\
-0.95361328125	12.4992335499257\\
-0.953125	12.5019842588621\\
-0.95263671875	12.5047639555927\\
-0.9521484375	12.5075726459115\\
-0.95166015625	12.510410335674\\
-0.951171875	12.5132770307971\\
-0.95068359375	12.5161727372595\\
-0.9501953125	12.519097461101\\
-0.94970703125	12.5220512084234\\
-0.94921875	12.5250339853899\\
-0.94873046875	12.5280457982256\\
-0.9482421875	12.5310866532171\\
-0.94775390625	12.5341565567129\\
-0.947265625	12.5372555151234\\
-0.94677734375	12.5403835349209\\
-0.9462890625	12.5435406226393\\
-0.94580078125	12.546726784875\\
-0.9453125	12.5499420282859\\
-0.94482421875	12.5531863595924\\
-0.9443359375	12.5564597855767\\
-0.94384765625	12.5597623130834\\
-0.943359375	12.5630939490192\\
-0.94287109375	12.5664547003531\\
-0.9423828125	12.5698445741165\\
-0.94189453125	12.5732635774031\\
-0.94140625	12.5767117173689\\
-0.94091796875	12.5801890012328\\
-0.9404296875	12.5836954362758\\
-0.93994140625	12.5872310298417\\
-0.939453125	12.5907957893369\\
-0.93896484375	12.5943897222305\\
-0.9384765625	12.5980128360544\\
-0.93798828125	12.6016651384033\\
-0.9375	12.6053466369347\\
-0.93701171875	12.6090573393692\\
-0.9365234375	12.6127972534901\\
-0.93603515625	12.616566387144\\
-0.935546875	12.6203647482405\\
-0.93505859375	12.6241923447523\\
-0.9345703125	12.6280491847155\\
-0.93408203125	12.6319352762293\\
-0.93359375	12.6358506274563\\
-0.93310546875	12.6397952466225\\
-0.9326171875	12.6437691420175\\
-0.93212890625	12.647772321994\\
-0.931640625	12.6518047949688\\
-0.93115234375	12.6558665694221\\
-0.9306640625	12.6599576538977\\
-0.93017578125	12.6640780570034\\
-0.9296875	12.6682277874108\\
-0.92919921875	12.6724068538553\\
-0.9287109375	12.6766152651362\\
-0.92822265625	12.6808530301172\\
-0.927734375	12.6851201577256\\
-0.92724609375	12.6894166569533\\
-0.9267578125	12.6937425368562\\
-0.92626953125	12.6980978065546\\
-0.92578125	12.7024824752332\\
-0.92529296875	12.7068965521409\\
-0.9248046875	12.7113400465913\\
-0.92431640625	12.7158129679628\\
-0.923828125	12.720315325698\\
-0.92333984375	12.7248471293045\\
-0.9228515625	12.7294083883547\\
-0.92236328125	12.7339991124857\\
-0.921875	12.7386193113997\\
-0.92138671875	12.7432689948639\\
-0.9208984375	12.7479481727103\\
-0.92041015625	12.7526568548366\\
-0.919921875	12.7573950512052\\
-0.91943359375	12.7621627718441\\
-0.9189453125	12.7669600268465\\
-0.91845703125	12.7717868263714\\
-0.91796875	12.7766431806429\\
-0.91748046875	12.781529099951\\
-0.9169921875	12.7864445946512\\
-0.91650390625	12.791389675165\\
-0.916015625	12.7963643519796\\
-0.91552734375	12.8013686356481\\
-0.9150390625	12.8064025367897\\
-0.91455078125	12.8114660660896\\
-0.9140625	12.8165592342991\\
-0.91357421875	12.8216820522361\\
-0.9130859375	12.8268345307843\\
-0.91259765625	12.8320166808943\\
-0.912109375	12.8372285135829\\
-0.91162109375	12.8424700399336\\
-0.9111328125	12.8477412710965\\
-0.91064453125	12.8530422182884\\
-0.91015625	12.8583728927932\\
-0.90966796875	12.8637333059614\\
-0.9091796875	12.8691234692106\\
-0.90869140625	12.8745433940256\\
-0.908203125	12.8799930919583\\
-0.90771484375	12.8854725746278\\
-0.9072265625	12.8909818537207\\
-0.90673828125	12.8965209409909\\
-0.90625	12.9020898482599\\
-0.90576171875	12.9076885874168\\
-0.9052734375	12.9133171704186\\
-0.90478515625	12.9189756092897\\
-0.904296875	12.9246639161228\\
-0.90380859375	12.9303821030783\\
-0.9033203125	12.936130182385\\
-0.90283203125	12.9419081663394\\
-0.90234375	12.9477160673068\\
-0.90185546875	12.9535538977205\\
-0.9013671875	12.9594216700824\\
-0.90087890625	12.965319396963\\
-0.900390625	12.9712470910012\\
-0.89990234375	12.9772047649049\\
-0.8994140625	12.9831924314508\\
-0.89892578125	12.9892101034845\\
-0.8984375	12.9952577939206\\
-0.89794921875	13.001335515743\\
-0.8974609375	13.0074432820047\\
-0.89697265625	13.013581105828\\
-0.896484375	13.0197490004048\\
-0.89599609375	13.0259469789964\\
-0.8955078125	13.0321750549338\\
-0.89501953125	13.0384332416178\\
-0.89453125	13.0447215525189\\
-0.89404296875	13.0510400011778\\
-0.8935546875	13.0573886012051\\
-0.89306640625	13.0637673662814\\
-0.892578125	13.0701763101579\\
-0.89208984375	13.076615446656\\
-0.8916015625	13.0830847896676\\
-0.89111328125	13.0895843531552\\
-0.890625	13.0961141511519\\
-0.89013671875	13.1026741977618\\
-0.8896484375	13.1092645071598\\
-0.88916015625	13.1158850935919\\
-0.888671875	13.122535971375\\
-0.88818359375	13.1292171548976\\
-0.8876953125	13.1359286586192\\
-0.88720703125	13.1426704970711\\
-0.88671875	13.149442684856\\
-0.88623046875	13.1562452366483\\
-0.8857421875	13.1630781671942\\
-0.88525390625	13.1699414913119\\
-0.884765625	13.1768352238917\\
-0.88427734375	13.1837593798958\\
-0.8837890625	13.190713974359\\
-0.88330078125	13.1976990223882\\
-0.8828125	13.204714539163\\
-0.88232421875	13.2117605399355\\
-0.8818359375	13.2188370400307\\
-0.88134765625	13.2259440548463\\
-0.880859375	13.2330815998531\\
-0.88037109375	13.2402496905949\\
-0.8798828125	13.2474483426889\\
-0.87939453125	13.2546775718255\\
-0.87890625	13.2619373937686\\
-0.87841796875	13.2692278243558\\
-0.8779296875	13.2765488794982\\
-0.87744140625	13.283900575181\\
-0.876953125	13.2912829274633\\
-0.87646484375	13.2986959524781\\
-0.8759765625	13.3061396664329\\
-0.87548828125	13.3136140856094\\
-0.875	13.3211192263638\\
-0.87451171875	13.328655105127\\
-0.8740234375	13.3362217384045\\
-0.87353515625	13.3438191427767\\
-0.873046875	13.3514473348992\\
-0.87255859375	13.3591063315024\\
-0.8720703125	13.3667961493922\\
-0.87158203125	13.3745168054498\\
-0.87109375	13.382268316632\\
-0.87060546875	13.3900506999712\\
-0.8701171875	13.3978639725756\\
-0.86962890625	13.4057081516294\\
-0.869140625	13.4135832543929\\
-0.86865234375	13.4214892982024\\
-0.8681640625	13.4294263004707\\
-0.86767578125	13.4373942786871\\
-0.8671875	13.4453932504176\\
-0.86669921875	13.4534232333047\\
-0.8662109375	13.461484245068\\
-0.86572265625	13.4695763035041\\
-0.865234375	13.4776994264868\\
-0.86474609375	13.4858536319672\\
-0.8642578125	13.494038937974\\
-0.86376953125	13.5022553626132\\
-0.86328125	13.510502924069\\
-0.86279296875	13.5187816406031\\
-0.8623046875	13.5270915305555\\
-0.86181640625	13.5354326123442\\
-0.861328125	13.5438049044658\\
-0.86083984375	13.5522084254953\\
-0.8603515625	13.5606431940862\\
-0.85986328125	13.5691092289709\\
-0.859375	13.5776065489608\\
-0.85888671875	13.5861351729463\\
-0.8583984375	13.5946951198972\\
-0.85791015625	13.6032864088624\\
-0.857421875	13.6119090589706\\
-0.85693359375	13.6205630894302\\
-0.8564453125	13.6292485195294\\
-0.85595703125	13.6379653686364\\
-0.85546875	13.6467136561996\\
-0.85498046875	13.6554934017477\\
-0.8544921875	13.66430462489\\
-0.85400390625	13.6731473453162\\
-0.853515625	13.6820215827972\\
-0.85302734375	13.6909273571845\\
-0.8525390625	13.6998646884108\\
-0.85205078125	13.7088335964903\\
-0.8515625	13.7178341015184\\
-0.85107421875	13.7268662236724\\
-0.8505859375	13.7359299832109\\
-0.85009765625	13.7450254004751\\
-0.849609375	13.7541524958876\\
-0.84912109375	13.7633112899539\\
-0.8486328125	13.7725018032616\\
-0.84814453125	13.781724056481\\
-0.84765625	13.7909780703651\\
-0.84716796875	13.8002638657499\\
-0.8466796875	13.8095814635546\\
-0.84619140625	13.8189308847817\\
-0.845703125	13.8283121505169\\
-0.84521484375	13.8377252819299\\
-0.8447265625	13.8471703002739\\
-0.84423828125	13.8566472268863\\
-0.84375	13.8661560831886\\
-0.84326171875	13.8756968906865\\
-0.8427734375	13.8852696709704\\
-0.84228515625	13.8948744457152\\
-0.841796875	13.904511236681\\
-0.84130859375	13.9141800657125\\
-0.8408203125	13.92388095474\\
-0.84033203125	13.933613925779\\
-0.83984375	13.9433790009307\\
-0.83935546875	13.9531762023819\\
-0.8388671875	13.9630055524056\\
-0.83837890625	13.9728670733607\\
-0.837890625	13.9827607876926\\
-0.83740234375	13.9926867179332\\
-0.8369140625	14.0026448867009\\
-0.83642578125	14.0126353167014\\
-0.8359375	14.022658030727\\
-0.83544921875	14.0327130516575\\
-0.8349609375	14.0428004024601\\
-0.83447265625	14.0529201061898\\
-0.833984375	14.0630721859893\\
-0.83349609375	14.0732566650892\\
-0.8330078125	14.0834735668086\\
-0.83251953125	14.0937229145548\\
-0.83203125	14.1040047318239\\
-0.83154296875	14.1143190422006\\
-0.8310546875	14.1246658693589\\
-0.83056640625	14.1350452370616\\
-0.830078125	14.1454571691612\\
-0.82958984375	14.1559016895999\\
-0.8291015625	14.1663788224093\\
-0.82861328125	14.1768885917115\\
-0.828125	14.1874310217184\\
-0.82763671875	14.1980061367326\\
-0.8271484375	14.2086139611471\\
-0.82666015625	14.2192545194459\\
-0.826171875	14.2299278362039\\
-0.82568359375	14.2406339360874\\
-0.8251953125	14.251372843854\\
-0.82470703125	14.262144584353\\
-0.82421875	14.2729491825256\\
-0.82373046875	14.2837866634051\\
-0.8232421875	14.294657052117\\
-0.82275390625	14.3055603738794\\
-0.822265625	14.316496654003\\
-0.82177734375	14.3274659178916\\
-0.8212890625	14.338468191042\\
-0.82080078125	14.3495034990444\\
-0.8203125	14.3605718675827\\
-0.81982421875	14.3716733224345\\
-0.8193359375	14.3828078894714\\
-0.81884765625	14.3939755946592\\
-0.818359375	14.4051764640585\\
-0.81787109375	14.4164105238241\\
-0.8173828125	14.427677800206\\
-0.81689453125	14.4389783195494\\
-0.81640625	14.4503121082947\\
-0.81591796875	14.4616791929779\\
-0.8154296875	14.4730796002309\\
-0.81494140625	14.4845133567816\\
-0.814453125	14.4959804894542\\
-0.81396484375	14.5074810251694\\
-0.8134765625	14.5190149909447\\
-0.81298828125	14.5305824138944\\
-0.8125	14.5421833212302\\
-0.81201171875	14.5538177402611\\
-0.8115234375	14.565485698394\\
-0.81103515625	14.5771872231333\\
-0.810546875	14.588922342082\\
-0.81005859375	14.6006910829412\\
-0.8095703125	14.6124934735108\\
-0.80908203125	14.6243295416895\\
-0.80859375	14.6361993154751\\
-0.80810546875	14.6481028229648\\
-0.8076171875	14.6600400923555\\
-0.80712890625	14.6720111519436\\
-0.806640625	14.6840160301261\\
-0.80615234375	14.6960547553999\\
-0.8056640625	14.7081273563628\\
-0.80517578125	14.7202338617133\\
-0.8046875	14.7323743002509\\
-0.80419921875	14.7445487008767\\
-0.8037109375	14.7567570925933\\
-0.80322265625	14.768999504505\\
-0.802734375	14.7812759658184\\
-0.80224609375	14.7935865058423\\
-0.8017578125	14.8059311539884\\
-0.80126953125	14.818309939771\\
-0.80078125	14.8307228928076\\
-0.80029296875	14.8431700428193\\
-0.7998046875	14.8556514196306\\
-0.79931640625	14.8681670531701\\
-0.798828125	14.8807169734705\\
-0.79833984375	14.8933012106691\\
-0.7978515625	14.9059197950078\\
-0.79736328125	14.9185727568335\\
-0.796875	14.9312601265984\\
-0.79638671875	14.9439819348603\\
-0.7958984375	14.9567382122826\\
-0.79541015625	14.9695289896351\\
-0.794921875	14.9823542977937\\
-0.79443359375	14.9952141677409\\
-0.7939453125	15.0081086305664\\
-0.79345703125	15.0210377174667\\
-0.79296875	15.034001459746\\
-0.79248046875	15.0469998888164\\
-0.7919921875	15.0600330361975\\
-0.79150390625	15.0731009335177\\
-0.791015625	15.0862036125138\\
-0.79052734375	15.0993411050315\\
-0.7900390625	15.1125134430256\\
-0.78955078125	15.1257206585603\\
-0.7890625	15.1389627838099\\
-0.78857421875	15.1522398510582\\
-0.7880859375	15.1655518926996\\
-0.78759765625	15.1788989412392\\
-0.787109375	15.1922810292928\\
-0.78662109375	15.2056981895875\\
-0.7861328125	15.2191504549619\\
-0.78564453125	15.2326378583663\\
-0.78515625	15.2461604328632\\
-0.78466796875	15.2597182116274\\
-0.7841796875	15.2733112279465\\
-0.78369140625	15.286939515221\\
-0.783203125	15.3006031069647\\
-0.78271484375	15.3143020368051\\
-0.7822265625	15.3280363384836\\
-0.78173828125	15.3418060458556\\
-0.78125	15.3556111928912\\
-0.78076171875	15.3694518136755\\
-0.7802734375	15.3833279424084\\
-0.77978515625	15.3972396134055\\
-0.779296875	15.4111868610982\\
-0.77880859375	15.4251697200338\\
-0.7783203125	15.4391882248762\\
-0.77783203125	15.4532424104059\\
-0.77734375	15.4673323115205\\
-0.77685546875	15.481457963235\\
-0.7763671875	15.4956194006821\\
-0.77587890625	15.5098166591125\\
-0.775390625	15.5240497738951\\
-0.77490234375	15.5383187805177\\
-0.7744140625	15.5526237145871\\
-0.77392578125	15.5669646118291\\
-0.7734375	15.5813415080897\\
-0.77294921875	15.5957544393343\\
-0.7724609375	15.6102034416491\\
-0.77197265625	15.6246885512407\\
-0.771484375	15.6392098044368\\
-0.77099609375	15.6537672376864\\
-0.7705078125	15.6683608875603\\
-0.77001953125	15.6829907907512\\
-0.76953125	15.6976569840741\\
-0.76904296875	15.7123595044668\\
-0.7685546875	15.7270983889902\\
-0.76806640625	15.7418736748286\\
-0.767578125	15.7566853992899\\
-0.76708984375	15.7715335998062\\
-0.7666015625	15.7864183139342\\
-0.76611328125	15.8013395793551\\
-0.765625	15.8162974338755\\
-0.76513671875	15.8312919154274\\
-0.7646484375	15.8463230620688\\
-0.76416015625	15.8613909119837\\
-0.763671875	15.876495503483\\
-0.76318359375	15.8916368750043\\
-0.7626953125	15.9068150651127\\
-0.76220703125	15.9220301125008\\
-0.76171875	15.9372820559895\\
-0.76123046875	15.9525709345279\\
-0.7607421875	15.967896787194\\
-0.76025390625	15.983259653195\\
-0.759765625	15.9986595718676\\
-0.75927734375	16.0140965826783\\
-0.7587890625	16.0295707252242\\
-0.75830078125	16.0450820392327\\
-0.7578125	16.0606305645625\\
-0.75732421875	16.0762163412036\\
-0.7568359375	16.0918394092779\\
-0.75634765625	16.1074998090396\\
-0.755859375	16.1231975808752\\
-0.75537109375	16.1389327653044\\
-0.7548828125	16.1547054029803\\
-0.75439453125	16.1705155346894\\
-0.75390625	16.1863632013529\\
-0.75341796875	16.2022484440261\\
-0.7529296875	16.2181713038995\\
-0.75244140625	16.2341318222988\\
-0.751953125	16.2501300406854\\
-0.75146484375	16.2661660006571\\
-0.7509765625	16.282239743948\\
-0.75048828125	16.2983513124291\\
-0.75	16.3145007481091\\
-0.74951171875	16.3306880931342\\
-0.7490234375	16.3469133897888\\
-0.74853515625	16.3631766804959\\
-0.748046875	16.3794780078175\\
-0.74755859375	16.3958174144552\\
-0.7470703125	16.4121949432502\\
-0.74658203125	16.428610637184\\
-0.74609375	16.4450645393789\\
-0.74560546875	16.4615566930983\\
-0.7451171875	16.4780871417469\\
-0.74462890625	16.4946559288715\\
-0.744140625	16.5112630981615\\
-0.74365234375	16.5279086934487\\
-0.7431640625	16.5445927587084\\
-0.74267578125	16.5613153380596\\
-0.7421875	16.5780764757653\\
-0.74169921875	16.594876216233\\
-0.7412109375	16.6117146040154\\
-0.74072265625	16.6285916838105\\
-0.740234375	16.6455075004622\\
-0.73974609375	16.6624620989607\\
-0.7392578125	16.679455524443\\
-0.73876953125	16.6964878221932\\
-0.73828125	16.7135590376433\\
-0.73779296875	16.7306692163733\\
-0.7373046875	16.7478184041117\\
-0.73681640625	16.7650066467361\\
-0.736328125	16.7822339902737\\
-0.73583984375	16.7995004809015\\
-0.7353515625	16.8168061649469\\
-0.73486328125	16.8341510888883\\
-0.734375	16.8515352993554\\
-0.73388671875	16.8689588431298\\
-0.7333984375	16.8864217671452\\
-0.73291015625	16.9039241184881\\
-0.732421875	16.9214659443984\\
-0.73193359375	16.9390472922696\\
-0.7314453125	16.9566682096494\\
-0.73095703125	16.9743287442401\\
-0.73046875	16.9920289438994\\
-0.72998046875	17.0097688566402\\
-0.7294921875	17.027548530632\\
-0.72900390625	17.0453680142006\\
-0.728515625	17.063227355829\\
-0.72802734375	17.0811266041579\\
-0.7275390625	17.099065807986\\
-0.72705078125	17.1170450162704\\
-0.7265625	17.1350642781278\\
-0.72607421875	17.153123642834\\
-0.7255859375	17.171223159825\\
-0.72509765625	17.1893628786978\\
-0.724609375	17.2075428492099\\
-0.72412109375	17.2257631212808\\
-0.7236328125	17.2440237449921\\
-0.72314453125	17.2623247705881\\
-0.72265625	17.2806662484761\\
-0.72216796875	17.2990482292272\\
-0.7216796875	17.3174707635765\\
-0.72119140625	17.3359339024243\\
-0.720703125	17.3544376968358\\
-0.72021484375	17.3729821980419\\
-0.7197265625	17.3915674574401\\
-0.71923828125	17.4101935265946\\
-0.71875	17.4288604572371\\
-0.71826171875	17.447568301267\\
-0.7177734375	17.4663171107524\\
-0.71728515625	17.4851069379302\\
-0.716796875	17.5039378352071\\
-0.71630859375	17.5228098551596\\
-0.7158203125	17.5417230505352\\
-0.71533203125	17.5606774742521\\
-0.71484375	17.5796731794008\\
-0.71435546875	17.5987102192438\\
-0.7138671875	17.6177886472163\\
-0.71337890625	17.6369085169274\\
-0.712890625	17.6560698821598\\
-0.71240234375	17.6752727968708\\
-0.7119140625	17.694517315193\\
-0.71142578125	17.7138034914346\\
-0.7109375	17.73313138008\\
-0.71044921875	17.7525010357905\\
-0.7099609375	17.7719125134048\\
-0.70947265625	17.7913658679396\\
-0.708984375	17.8108611545903\\
-0.70849609375	17.8303984287311\\
-0.7080078125	17.8499777459165\\
-0.70751953125	17.8695991618808\\
-0.70703125	17.8892627325397\\
-0.70654296875	17.90896851399\\
-0.7060546875	17.928716562511\\
-0.70556640625	17.9485069345645\\
-0.705078125	17.9683396867958\\
-0.70458984375	17.9882148760339\\
-0.7041015625	18.0081325592926\\
-0.70361328125	18.0280927937707\\
-0.703125	18.0480956368528\\
-0.70263671875	18.06814114611\\
-0.7021484375	18.0882293793002\\
-0.70166015625	18.1083603943691\\
-0.701171875	18.1285342494505\\
-0.70068359375	18.1487510028672\\
-0.7001953125	18.1690107131315\\
-0.69970703125	18.1893134389457\\
-0.69921875	18.2096592392031\\
-0.69873046875	18.2300481729882\\
-0.6982421875	18.2504802995775\\
-0.69775390625	18.2709556784405\\
-0.697265625	18.2914743692398\\
-0.69677734375	18.312036431832\\
-0.6962890625	18.3326419262683\\
-0.69580078125	18.3532909127954\\
-0.6953125	18.3739834518556\\
-0.69482421875	18.3947196040882\\
-0.6943359375	18.4154994303295\\
-0.69384765625	18.4363229916138\\
-0.693359375	18.4571903491739\\
-0.69287109375	18.4781015644421\\
-0.6923828125	18.4990566990504\\
-0.69189453125	18.5200558148315\\
-0.69140625	18.5410989738196\\
-0.69091796875	18.5621862382505\\
-0.6904296875	18.5833176705629\\
-0.68994140625	18.6044933333989\\
-0.689453125	18.6257132896045\\
-0.68896484375	18.6469776022304\\
-0.6884765625	18.668286334533\\
-0.68798828125	18.6896395499745\\
-0.6875	18.7110373122242\\
-0.68701171875	18.7324796851587\\
-0.6865234375	18.7539667328631\\
-0.68603515625	18.7754985196312\\
-0.685546875	18.7970751099667\\
-0.68505859375	18.8186965685835\\
-0.6845703125	18.8403629604068\\
-0.68408203125	18.8620743505735\\
-0.68359375	18.8838308044332\\
-0.68310546875	18.9056323875484\\
-0.6826171875	18.9274791656963\\
-0.68212890625	18.9493712048681\\
-0.681640625	18.9713085712711\\
-0.68115234375	18.9932913313286\\
-0.6806640625	19.0153195516808\\
-0.68017578125	19.0373932991857\\
-0.6796875	19.05951264092\\
-0.67919921875	19.0816776441792\\
-0.6787109375	19.1038883764793\\
-0.67822265625	19.1261449055566\\
-0.677734375	19.1484472993693\\
-0.67724609375	19.1707956260975\\
-0.6767578125	19.1931899541448\\
-0.67626953125	19.2156303521383\\
-0.67578125	19.2381168889298\\
-0.67529296875	19.2606496335965\\
-0.6748046875	19.2832286554419\\
-0.67431640625	19.3058540239963\\
-0.673828125	19.3285258090177\\
-0.67333984375	19.3512440804929\\
-0.6728515625	19.3740089086378\\
-0.67236328125	19.3968203638985\\
-0.671875	19.4196785169522\\
-0.67138671875	19.4425834387076\\
-0.6708984375	19.4655352003061\\
-0.67041015625	19.4885338731226\\
-0.669921875	19.5115795287659\\
-0.66943359375	19.5346722390802\\
-0.6689453125	19.557812076145\\
-0.66845703125	19.580999112277\\
-0.66796875	19.6042334200302\\
-0.66748046875	19.6275150721968\\
-0.6669921875	19.6508441418083\\
-0.66650390625	19.6742207021362\\
-0.666015625	19.6976448266928\\
-0.66552734375	19.7211165892322\\
-0.6650390625	19.744636063751\\
-0.66455078125	19.7682033244892\\
-0.6640625	19.791818445931\\
-0.66357421875	19.8154815028059\\
-0.6630859375	19.8391925700892\\
-0.66259765625	19.8629517230032\\
-0.662109375	19.8867590370179\\
-0.66162109375	19.9106145878519\\
-0.6611328125	19.9345184514734\\
-0.66064453125	19.9584707041009\\
-0.66015625	19.9824714222039\\
-0.65966796875	20.0065206825046\\
-0.6591796875	20.0306185619779\\
-0.65869140625	20.0547651378527\\
-0.658203125	20.0789604876127\\
-0.65771484375	20.1032046889975\\
-0.6572265625	20.1274978200033\\
-0.65673828125	20.1518399588839\\
-0.65625	20.1762311841516\\
-0.65576171875	20.2006715745781\\
-0.6552734375	20.2251612091954\\
-0.65478515625	20.2497001672968\\
-0.654296875	20.2742885284379\\
-0.65380859375	20.2989263724374\\
-0.6533203125	20.323613779378\\
-0.65283203125	20.3483508296075\\
-0.65234375	20.3731376037396\\
-0.65185546875	20.3979741826551\\
-0.6513671875	20.4228606475024\\
-0.65087890625	20.447797079699\\
-0.650390625	20.4727835609321\\
-0.64990234375	20.4978201731596\\
-0.6494140625	20.5229069986112\\
-0.64892578125	20.5480441197893\\
-0.6484375	20.5732316194702\\
-0.64794921875	20.5984695807044\\
-0.6474609375	20.6237580868185\\
-0.64697265625	20.6490972214155\\
-0.646484375	20.6744870683763\\
-0.64599609375	20.6999277118601\\
-0.6455078125	20.725419236306\\
-0.64501953125	20.7509617264337\\
-0.64453125	20.7765552672445\\
-0.64404296875	20.8021999440224\\
-0.6435546875	20.8278958423351\\
-0.64306640625	20.8536430480351\\
-0.642578125	20.8794416472605\\
-0.64208984375	20.9052917264363\\
-0.6416015625	20.9311933722752\\
-0.64111328125	20.9571466717788\\
-0.640625	20.9831517122386\\
-0.64013671875	21.009208581237\\
-0.6396484375	21.0353173666485\\
-0.63916015625	21.0614781566405\\
-0.638671875	21.0876910396745\\
-0.63818359375	21.1139561045072\\
-0.6376953125	21.1402734401917\\
-0.63720703125	21.1666431360781\\
-0.63671875	21.1930652818151\\
-0.63623046875	21.2195399673509\\
-0.6357421875	21.2460672829339\\
-0.63525390625	21.2726473191146\\
-0.634765625	21.2992801667458\\
-0.63427734375	21.3259659169845\\
-0.6337890625	21.3527046612923\\
-0.63330078125	21.3794964914371\\
-0.6328125	21.4063414994936\\
-0.63232421875	21.4332397778451\\
-0.6318359375	21.4601914191841\\
-0.63134765625	21.4871965165135\\
-0.630859375	21.5142551631481\\
-0.63037109375	21.5413674527152\\
-0.6298828125	21.5685334791559\\
-0.62939453125	21.5957533367267\\
-0.62890625	21.6230271199999\\
-0.62841796875	21.6503549238653\\
-0.6279296875	21.6777368435312\\
-0.62744140625	21.7051729745252\\
-0.626953125	21.732663412696\\
-0.62646484375	21.7602082542144\\
-0.6259765625	21.7878075955737\\
-0.62548828125	21.8154615335921\\
-0.625	21.843170165413\\
-0.62451171875	21.8709335885064\\
-0.6240234375	21.8987519006703\\
-0.62353515625	21.9266252000317\\
-0.623046875	21.9545535850475\\
-0.62255859375	21.9825371545065\\
-0.6220703125	22.0105760075299\\
-0.62158203125	22.0386702435727\\
-0.62109375	22.0668199624249\\
-0.62060546875	22.095025264213\\
-0.6201171875	22.1232862494007\\
-0.61962890625	22.1516030187907\\
-0.619140625	22.1799756735253\\
-0.61865234375	22.2084043150883\\
-0.6181640625	22.2368890453057\\
-0.61767578125	22.2654299663472\\
-0.6171875	22.2940271807275\\
-0.61669921875	22.3226807913074\\
-0.6162109375	22.351390901295\\
-0.61572265625	22.3801576142472\\
-0.615234375	22.4089810340709\\
-0.61474609375	22.437861265024\\
-0.6142578125	22.466798411717\\
-0.61376953125	22.4957925791142\\
-0.61328125	22.5248438725349\\
-0.61279296875	22.5539523976546\\
-0.6123046875	22.5831182605067\\
-0.61181640625	22.6123415674832\\
-0.611328125	22.6416224253366\\
-0.61083984375	22.6709609411807\\
-0.6103515625	22.7003572224922\\
-0.60986328125	22.7298113771119\\
-0.609375	22.7593235132462\\
-0.60888671875	22.7888937394681\\
-0.6083984375	22.8185221647188\\
-0.60791015625	22.8482088983089\\
-0.607421875	22.8779540499197\\
-0.60693359375	22.9077577296048\\
-0.6064453125	22.937620047791\\
-0.60595703125	22.9675411152801\\
-0.60546875	22.9975210432499\\
-0.60498046875	23.0275599432557\\
-0.6044921875	23.0576579272319\\
-0.60400390625	23.0878151074927\\
-0.603515625	23.1180315967343\\
-0.60302734375	23.1483075080356\\
-0.6025390625	23.1786429548599\\
-0.60205078125	23.2090380510563\\
-0.6015625	23.2394929108609\\
-0.60107421875	23.2700076488984\\
-0.6005859375	23.3005823801832\\
-0.60009765625	23.3312172201213\\
-0.599609375	23.361912284511\\
-0.59912109375	23.392667689545\\
-0.5986328125	23.4234835518112\\
-0.59814453125	23.4543599882947\\
-0.59765625	23.4852971163788\\
-0.59716796875	23.5162950538465\\
-0.5966796875	23.547353918882\\
-0.59619140625	23.5784738300721\\
-0.595703125	23.6096549064077\\
-0.59521484375	23.6408972672851\\
-0.5947265625	23.6722010325075\\
-0.59423828125	23.7035663222865\\
-0.59375	23.7349932572434\\
-0.59326171875	23.7664819584109\\
-0.5927734375	23.7980325472343\\
-0.59228515625	23.829645145573\\
-0.591796875	23.8613198757023\\
-0.59130859375	23.8930568603143\\
-0.5908203125	23.9248562225198\\
-0.59033203125	23.9567180858496\\
-0.58984375	23.988642574256\\
-0.58935546875	24.0206298121144\\
-0.5888671875	24.0526799242245\\
-0.58837890625	24.0847930358122\\
-0.587890625	24.1169692725306\\
-0.58740234375	24.149208760462\\
-0.5869140625	24.1815116261189\\
-0.58642578125	24.2138779964458\\
-0.5859375	24.2463079988209\\
-0.58544921875	24.2788017610571\\
-0.5849609375	24.3113594114038\\
-0.58447265625	24.3439810785485\\
-0.583984375	24.376666891618\\
-0.58349609375	24.4094169801804\\
-0.5830078125	24.442231474246\\
-0.58251953125	24.4751105042694\\
-0.58203125	24.5080542011507\\
-0.58154296875	24.5410626962371\\
-0.5810546875	24.5741361213246\\
-0.58056640625	24.6072746086592\\
-0.580078125	24.6404782909387\\
-0.57958984375	24.6737473013143\\
-0.5791015625	24.7070817733916\\
-0.57861328125	24.7404818412331\\
-0.578125	24.7739476393588\\
-0.57763671875	24.8074793027483\\
-0.5771484375	24.8410769668422\\
-0.57666015625	24.8747407675436\\
-0.576171875	24.9084708412198\\
-0.57568359375	24.9422673247037\\
-0.5751953125	24.9761303552956\\
-0.57470703125	25.0100600707643\\
-0.57421875	25.0440566093491\\
-0.57373046875	25.0781201097613\\
-0.5732421875	25.1122507111857\\
-0.57275390625	25.1464485532819\\
-0.572265625	25.1807137761865\\
-0.57177734375	25.2150465205139\\
-0.5712890625	25.2494469273586\\
-0.57080078125	25.2839151382964\\
-0.5703125	25.3184512953857\\
-0.56982421875	25.3530555411698\\
-0.5693359375	25.3877280186778\\
-0.56884765625	25.4224688714266\\
-0.568359375	25.4572782434221\\
-0.56787109375	25.4921562791613\\
-0.5673828125	25.5271031236334\\
-0.56689453125	25.5621189223216\\
-0.56640625	25.5972038212045\\
-0.56591796875	25.6323579667579\\
-0.5654296875	25.6675815059564\\
-0.56494140625	25.7028745862747\\
-0.564453125	25.7382373556895\\
-0.56396484375	25.7736699626807\\
-0.5634765625	25.8091725562333\\
-0.56298828125	25.844745285839\\
-0.5625	25.8803883014976\\
-0.56201171875	25.9161017537183\\
-0.5615234375	25.9518857935221\\
-0.56103515625	25.9877405724425\\
-0.560546875	26.0236662425275\\
-0.56005859375	26.0596629563412\\
-0.5595703125	26.0957308669653\\
-0.55908203125	26.1318701280004\\
-0.55859375	26.168080893568\\
-0.55810546875	26.2043633183118\\
-0.5576171875	26.2407175573993\\
-0.55712890625	26.2771437665234\\
-0.556640625	26.3136421019038\\
-0.55615234375	26.3502127202888\\
-0.5556640625	26.3868557789567\\
-0.55517578125	26.4235714357174\\
-0.5546875	26.4603598489137\\
-0.55419921875	26.4972211774232\\
-0.5537109375	26.5341555806597\\
-0.55322265625	26.5711632185746\\
-0.552734375	26.6082442516585\\
-0.55224609375	26.6453988409427\\
-0.5517578125	26.682627148001\\
-0.55126953125	26.7199293349506\\
-0.55078125	26.7573055644542\\
-0.55029296875	26.7947559997213\\
-0.5498046875	26.8322808045094\\
-0.54931640625	26.869880143126\\
-0.548828125	26.9075541804297\\
-0.54833984375	26.9453030818318\\
-0.5478515625	26.9831270132978\\
-0.54736328125	27.0210261413487\\
-0.546875	27.0590006330627\\
-0.54638671875	27.0970506560767\\
-0.5458984375	27.1351763785872\\
-0.54541015625	27.1733779693522\\
-0.544921875	27.2116555976928\\
-0.54443359375	27.2500094334939\\
-0.5439453125	27.2884396472064\\
-0.54345703125	27.3269464098481\\
-0.54296875	27.365529893005\\
-0.54248046875	27.4041902688334\\
-0.5419921875	27.4429277100603\\
-0.54150390625	27.4817423899854\\
-0.541015625	27.5206344824823\\
-0.54052734375	27.5596041619997\\
-0.5400390625	27.598651603563\\
-0.53955078125	27.6377769827753\\
-0.5390625	27.6769804758188\\
-0.53857421875	27.7162622594562\\
-0.5380859375	27.755622511032\\
-0.53759765625	27.7950614084734\\
-0.537109375	27.8345791302921\\
-0.53662109375	27.874175855585\\
-0.5361328125	27.9138517640359\\
-0.53564453125	27.9536070359162\\
-0.53515625	27.9934418520865\\
-0.53466796875	28.0333563939977\\
-0.5341796875	28.073350843692\\
-0.53369140625	28.113425383804\\
-0.533203125	28.1535801975623\\
-0.53271484375	28.1938154687898\\
-0.5322265625	28.2341313819057\\
-0.53173828125	28.2745281219259\\
-0.53125	28.3150058744643\\
-0.53076171875	28.3555648257336\\
-0.5302734375	28.3962051625469\\
-0.52978515625	28.4369270723182\\
-0.529296875	28.4777307430632\\
-0.52880859375	28.5186163634009\\
-0.5283203125	28.5595841225538\\
-0.52783203125	28.6006342103496\\
-0.52734375	28.6417668172213\\
-0.52685546875	28.6829821342086\\
-0.5263671875	28.7242803529584\\
-0.52587890625	28.7656616657258\\
-0.525390625	28.807126265375\\
-0.52490234375	28.8486743453797\\
-0.5244140625	28.8903060998241\\
-0.52392578125	28.9320217234036\\
-0.5234375	28.9738214114253\\
-0.52294921875	29.0157053598088\\
-0.5224609375	29.057673765087\\
-0.52197265625	29.0997268244061\\
-0.521484375	29.1418647355269\\
-0.52099609375	29.1840876968246\\
-0.5205078125	29.22639590729\\
-0.52001953125	29.2687895665294\\
-0.51953125	29.3112688747655\\
-0.51904296875	29.3538340328372\\
-0.5185546875	29.3964852422005\\
-0.51806640625	29.4392227049288\\
-0.517578125	29.4820466237127\\
-0.51708984375	29.5249572018608\\
-0.5166015625	29.5679546432996\\
-0.51611328125	29.6110391525738\\
-0.515625	29.6542109348464\\
-0.51513671875	29.6974701958986\\
-0.5146484375	29.7408171421304\\
-0.51416015625	29.7842519805599\\
-0.513671875	29.8277749188238\\
-0.51318359375	29.871386165177\\
-0.5126953125	29.9150859284928\\
-0.51220703125	29.9588744182623\\
-0.51171875	30.0027518445945\\
-0.51123046875	30.0467184182161\\
-0.5107421875	30.0907743504707\\
-0.51025390625	30.134919853319\\
-0.509765625	30.1791551393381\\
-0.50927734375	30.2234804217209\\
-0.5087890625	30.2678959142758\\
-0.50830078125	30.3124018314263\\
-0.5078125	30.3569983882096\\
-0.50732421875	30.4016858002769\\
-0.5068359375	30.4464642838918\\
-0.50634765625	30.49133405593\\
-0.505859375	30.5362953338783\\
-0.50537109375	30.5813483358335\\
-0.5048828125	30.6264932805016\\
-0.50439453125	30.6717303871965\\
-0.50390625	30.7170598758393\\
-0.50341796875	30.7624819669564\\
-0.5029296875	30.8079968816791\\
-0.50244140625	30.8536048417414\\
-0.501953125	30.8993060694792\\
-0.50146484375	30.9451007878284\\
-0.5009765625	30.9909892203238\\
-0.50048828125	31.036971591097\\
-0.5	31.0830481248748\\
-0.49951171875	31.1292190469777\\
-0.4990234375	31.1754845833177\\
-0.49853515625	31.2218449603962\\
-0.498046875	31.2683004053025\\
-0.49755859375	31.3148511457112\\
-0.4970703125	31.3614974098802\\
-0.49658203125	31.4082394266481\\
-0.49609375	31.4550774254324\\
-0.49560546875	31.5020116362262\\
-0.4951171875	31.5490422895964\\
-0.49462890625	31.5961696166802\\
-0.494140625	31.6433938491832\\
-0.49365234375	31.6907152193758\\
-0.4931640625	31.7381339600905\\
-0.49267578125	31.7856503047189\\
-0.4921875	31.8332644872081\\
-0.49169921875	31.8809767420578\\
-0.4912109375	31.9287873043169\\
-0.49072265625	31.9766964095794\\
-0.490234375	32.0247042939816\\
-0.48974609375	32.0728111941973\\
-0.4892578125	32.121017347435\\
-0.48876953125	32.169322991433\\
-0.48828125	32.2177283644558\\
-0.48779296875	32.2662337052895\\
-0.4873046875	32.3148392532375\\
-0.48681640625	32.3635452481162\\
-0.486328125	32.4123519302499\\
-0.48583984375	32.4612595404666\\
-0.4853515625	32.5102683200922\\
-0.48486328125	32.5593785109462\\
-0.484375	32.6085903553362\\
-0.48388671875	32.6579040960523\\
-0.4833984375	32.7073199763619\\
-0.48291015625	32.7568382400036\\
-0.482421875	32.8064591311819\\
-0.48193359375	32.8561828945608\\
-0.4814453125	32.9060097752577\\
-0.48095703125	32.955940018837\\
-0.48046875	33.0059738713039\\
-0.47998046875	33.0561115790973\\
-0.4794921875	33.1063533890832\\
-0.47900390625	33.1566995485478\\
-0.478515625	33.2071503051897\\
-0.47802734375	33.2577059071131\\
-0.4775390625	33.3083666028202\\
-0.47705078125	33.3591326412027\\
-0.4765625	33.4100042715347\\
-0.47607421875	33.460981743464\\
-0.4755859375	33.5120653070037\\
-0.47509765625	33.5632552125239\\
-0.474609375	33.6145517107425\\
-0.47412109375	33.6659550527167\\
-0.4736328125	33.7174654898329\\
-0.47314453125	33.7690832737983\\
-0.47265625	33.8208086566302\\
-0.47216796875	33.8726418906467\\
-0.4716796875	33.9245832284562\\
-0.47119140625	33.9766329229472\\
-0.470703125	34.0287912272775\\
-0.47021484375	34.0810583948631\\
-0.4697265625	34.1334346793675\\
-0.46923828125	34.1859203346898\\
-0.46875	34.2385156149533\\
-0.46826171875	34.2912207744934\\
-0.4677734375	34.3440360678452\\
-0.46728515625	34.3969617497314\\
-0.466796875	34.449998075049\\
-0.46630859375	34.5031452988566\\
-0.4658203125	34.5564036763605\\
-0.46533203125	34.6097734629017\\
-0.46484375	34.6632549139413\\
-0.46435546875	34.7168482850463\\
-0.4638671875	34.7705538318751\\
-0.46337890625	34.8243718101625\\
-0.462890625	34.8783024757042\\
-0.46240234375	34.9323460843415\\
-0.4619140625	34.986502891945\\
-0.46142578125	35.0407731543983\\
-0.4609375	35.0951571275817\\
-0.46044921875	35.1496550673546\\
-0.4599609375	35.2042672295385\\
-0.45947265625	35.2589938698989\\
-0.458984375	35.3138352441274\\
-0.45849609375	35.3687916078229\\
-0.4580078125	35.4238632164727\\
-0.45751953125	35.4790503254334\\
-0.45703125	35.5343531899106\\
-0.45654296875	35.5897720649391\\
-0.4560546875	35.6453072053623\\
-0.45556640625	35.7009588658107\\
-0.455078125	35.756727300681\\
-0.45458984375	35.8126127641136\\
-0.4541015625	35.8686155099706\\
-0.45361328125	35.9247357918127\\
-0.453125	35.980973862876\\
-0.45263671875	36.0373299760479\\
-0.4521484375	36.0938043838431\\
-0.45166015625	36.1503973383785\\
-0.451171875	36.207109091348\\
-0.45068359375	36.2639398939968\\
-0.4501953125	36.3208899970946\\
-0.44970703125	36.3779596509091\\
-0.44921875	36.4351491051782\\
-0.44873046875	36.4924586090821\\
-0.4482421875	36.5498884112149\\
-0.44775390625	36.607438759555\\
-0.447265625	36.6651099014358\\
-0.44677734375	36.722902083515\\
-0.4462890625	36.7808155517438\\
-0.44580078125	36.8388505513351\\
-0.4453125	36.8970073267318\\
-0.44482421875	36.9552861215732\\
-0.4443359375	37.0136871786621\\
-0.44384765625	37.0722107399303\\
-0.443359375	37.1308570464039\\
-0.44287109375	37.1896263381676\\
-0.4423828125	37.2485188543287\\
-0.44189453125	37.3075348329798\\
-0.44140625	37.3666745111615\\
-0.44091796875	37.4259381248238\\
-0.4404296875	37.4853259087866\\
-0.43994140625	37.5448380967004\\
-0.439453125	37.6044749210053\\
-0.43896484375	37.6642366128892\\
-0.4384765625	37.7241234022462\\
-0.43798828125	37.7841355176325\\
-0.4375	37.8442731862234\\
-0.43701171875	37.904536633768\\
-0.4365234375	37.9649260845435\\
-0.43603515625	38.0254417613088\\
-0.435546875	38.0860838852571\\
-0.43505859375	38.1468526759675\\
-0.4345703125	38.2077483513557\\
-0.43408203125	38.2687711276237\\
-0.43359375	38.3299212192089\\
-0.43310546875	38.3911988387319\\
-0.4326171875	38.4526041969433\\
-0.43212890625	38.5141375026694\\
-0.431640625	38.5757989627575\\
-0.43115234375	38.6375887820193\\
-0.4306640625	38.6995071631742\\
-0.43017578125	38.7615543067902\\
-0.4296875	38.8237304112251\\
-0.42919921875	38.8860356725662\\
-0.4287109375	38.9484702845676\\
-0.42822265625	39.0110344385887\\
-0.427734375	39.0737283235295\\
-0.42724609375	39.1365521257654\\
-0.4267578125	39.1995060290815\\
-0.42626953125	39.2625902146042\\
-0.42578125	39.3258048607331\\
-0.42529296875	39.3891501430703\\
-0.4248046875	39.4526262343498\\
-0.42431640625	39.5162333043639\\
-0.423828125	39.5799715198902\\
-0.42333984375	39.6438410446151\\
-0.4228515625	39.7078420390582\\
-0.42236328125	39.7719746604934\\
-0.421875	39.8362390628698\\
-0.42138671875	39.9006353967304\\
-0.4208984375	39.96516380913\\
-0.42041015625	40.0298244435509\\
-0.419921875	40.0946174398176\\
-0.41943359375	40.1595429340101\\
-0.4189453125	40.2246010583751\\
-0.41845703125	40.2897919412358\\
-0.41796875	40.3551157069003\\
-0.41748046875	40.4205724755684\\
-0.4169921875	40.4861623632364\\
-0.41650390625	40.5518854816005\\
-0.416015625	40.6177419379583\\
-0.41552734375	40.6837318351087\\
-0.4150390625	40.7498552712501\\
-0.41455078125	40.8161123398767\\
-0.4140625	40.8825031296728\\
-0.41357421875	40.9490277244058\\
-0.4130859375	41.0156862028165\\
-0.41259765625	41.0824786385086\\
-0.412109375	41.1494050998353\\
-0.41162109375	41.2164656497841\\
-0.4111328125	41.2836603458604\\
-0.41064453125	41.3509892399682\\
-0.41015625	41.4184523782888\\
-0.40966796875	41.4860498011585\\
-0.4091796875	41.5537815429423\\
-0.40869140625	41.6216476319079\\
-0.408203125	41.6896480900949\\
-0.40771484375	41.7577829331843\\
-0.4072265625	41.8260521703636\\
-0.40673828125	41.8944558041915\\
-0.40625	41.9629938304592\\
-0.40576171875	42.0316662380493\\
-0.4052734375	42.1004730087934\\
-0.40478515625	42.1694141173263\\
-0.404296875	42.238489530938\\
-0.40380859375	42.3076992094239\\
-0.4033203125	42.3770431049318\\
-0.40283203125	42.4465211618066\\
-0.40234375	42.5161333164327\\
-0.40185546875	42.5858794970736\\
-0.4013671875	42.6557596237091\\
-0.40087890625	42.7257736078695\\
-0.400390625	42.795921352468\\
-0.39990234375	42.8662027516289\\
-0.3994140625	42.9366176905151\\
-0.39892578125	43.0071660451506\\
-0.3984375	43.0778476822429\\
-0.39794921875	43.1486624589992\\
-0.3974609375	43.2196102229433\\
-0.39697265625	43.2906908117271\\
-0.396484375	43.3619040529401\\
-0.39599609375	43.4332497639162\\
-0.3955078125	43.5047277515371\\
-0.39501953125	43.5763378120331\\
-0.39453125	43.64807973078\\
-0.39404296875	43.7199532820945\\
-0.3935546875	43.791958229025\\
-0.39306640625	43.8640943231401\\
-0.392578125	43.9363613043136\\
-0.39208984375	44.0087589005069\\
-0.3916015625	44.0812868275474\\
-0.39111328125	44.1539447889046\\
-0.390625	44.2267324754619\\
-0.39013671875	44.2996495652859\\
-0.3896484375	44.3726957233926\\
-0.38916015625	44.4458706015093\\
-0.388671875	44.5191738378345\\
-0.38818359375	44.5926050567928\\
-0.3876953125	44.6661638687881\\
-0.38720703125	44.7398498699526\\
-0.38671875	44.8136626418918\\
-0.38623046875	44.8876017514275\\
-0.3857421875	44.961666750336\\
-0.38525390625	45.0358571750841\\
-0.384765625	45.1101725465601\\
-0.38427734375	45.1846123698028\\
-0.3837890625	45.2591761337261\\
-0.38330078125	45.3338633108407\\
-0.3828125	45.4086733569712\\
-0.38232421875	45.4836057109719\\
-0.3818359375	45.5586597944362\\
-0.38134765625	45.6338350114048\\
-0.380859375	45.7091307480695\\
-0.38037109375	45.7845463724731\\
-0.3798828125	45.8600812342069\\
-0.37939453125	45.9357346641035\\
-0.37890625	46.0115059739274\\
-0.37841796875	46.0873944560606\\
-0.3779296875	46.1633993831861\\
-0.37744140625	46.2395200079676\\
-0.376953125	46.3157555627252\\
-0.37646484375	46.3921052591088\\
-0.3759765625	46.468568287767\\
-0.37548828125	46.5451438180134\\
-0.375	46.6218309974897\\
-0.37451171875	46.6986289518253\\
-0.3740234375	46.7755367842934\\
-0.37353515625	46.8525535754645\\
-0.373046875	46.9296783828569\\
-0.37255859375	47.0069102405832\\
-0.3720703125	47.084248158995\\
-0.37158203125	47.1616911243244\\
-0.37109375	47.2392380983218\\
-0.37060546875	47.3168880178917\\
-0.3701171875	47.3946397947261\\
-0.36962890625	47.472492314935\\
-0.369140625	47.5504444386733\\
-0.36865234375	47.6284949997674\\
-0.3681640625	47.7066428053378\\
-0.36767578125	47.7848866354201\\
-0.3671875	47.863225242584\\
-0.36669921875	47.9416573515502\\
-0.3662109375	48.0201816588055\\
-0.36572265625	48.0987968322162\\
-0.365234375	48.1775015106396\\
-0.36474609375	48.256294303535\\
-0.3642578125	48.3351737905725\\
-0.36376953125	48.4141385212411\\
-0.36328125	48.4931870144557\\
-0.36279296875	48.5723177581641\\
-0.3623046875	48.6515292089524\\
-0.36181640625	48.7308197916504\\
-0.361328125	48.8101878989373\\
-0.36083984375	48.8896318909473\\
-0.3603515625	48.9691500948749\\
-0.35986328125	49.0487408045816\\
-0.359375	49.1284022802023\\
-0.35888671875	49.2081327477544\\
-0.3583984375	49.2879303987464\\
-0.35791015625	49.367793389789\\
-0.357421875	49.4477198422079\\
-0.35693359375	49.5277078416589\\
-0.3564453125	49.6077554377451\\
-0.35595703125	49.6878606436369\\
-0.35546875	49.7680214356959\\
-0.35498046875	49.8482357531003\\
-0.3544921875	49.9285014974769\\
-0.35400390625	50.0088165325349\\
-0.353515625	50.0891786837045\\
-0.35302734375	50.1695857377824\\
-0.3525390625	50.2500354425798\\
-0.35205078125	50.3305255065781\\
-0.3515625	50.4110535985904\\
-0.35107421875	50.491617347429\\
-0.3505859375	50.5722143415803\\
-0.35009765625	50.6528421288874\\
-0.349609375	50.7334982162405\\
-0.34912109375	50.8141800692758\\
-0.3486328125	50.8948851120833\\
-0.34814453125	50.9756107269241\\
-0.34765625	51.0563542539587\\
-0.34716796875	51.1371129909835\\
-0.3466796875	51.2178841931815\\
-0.34619140625	51.2986650728824\\
-0.345703125	51.3794527993359\\
-0.34521484375	51.4602444984981\\
-0.3447265625	51.5410372528309\\
-0.34423828125	51.621828101116\\
-0.34375	51.702614038283\\
-0.34326171875	51.7833920152543\\
-0.3427734375	51.8641589388042\\
-0.34228515625	51.9449116714365\\
-0.341796875	52.0256470312785\\
-0.34130859375	52.1063617919936\\
-0.3408203125	52.1870526827131\\
-0.34033203125	52.2677163879861\\
-0.33984375	52.3483495477521\\
-0.33935546875	52.4289487573326\\
-0.3388671875	52.5095105674453\\
-0.33837890625	52.5900314842414\\
-0.337890625	52.6705079693651\\
-0.33740234375	52.7509364400385\\
-0.3369140625	52.8313132691703\\
-0.33642578125	52.9116347854909\\
-0.3359375	52.9918972737135\\
-0.33544921875	53.0720969747232\\
-0.3349609375	53.1522300857934\\
-0.33447265625	53.232292760832\\
-0.333984375	53.3122811106558\\
-0.33349609375	53.3921912032972\\
-0.3330078125	53.4720190643416\\
-0.33251953125	53.5517606772957\\
-0.33203125	53.6314119839908\\
-0.33154296875	53.7109688850186\\
-0.3310546875	53.7904272402011\\
-0.33056640625	53.8697828690971\\
-0.330078125	53.9490315515443\\
-0.32958984375	54.028169028237\\
-0.3291015625	54.1071910013437\\
-0.32861328125	54.186093135161\\
-0.328125	54.2648710568077\\
-0.32763671875	54.3435203569585\\
-0.3271484375	54.4220365906175\\
-0.32666015625	54.5004152779343\\
-0.326171875	54.5786519050599\\
-0.32568359375	54.656741925047\\
-0.3251953125	54.7346807587919\\
-0.32470703125	54.8124637960203\\
-0.32421875	54.8900863963185\\
-0.32373046875	54.9675438902066\\
-0.3232421875	55.0448315802596\\
-0.32275390625	55.1219447422718\\
-0.322265625	55.1988786264693\\
-0.32177734375	55.2756284587662\\
-0.3212890625	55.3521894420701\\
-0.32080078125	55.4285567576321\\
-0.3203125	55.5047255664457\\
-0.31982421875	55.5806910106921\\
-0.3193359375	55.6564482152323\\
-0.31884765625	55.7319922891482\\
-0.318359375	55.80731832733\\
-0.31787109375	55.8824214121108\\
-0.3173828125	55.9572966149495\\
-0.31689453125	56.0319389981597\\
-0.31640625	56.1063436166863\\
-0.31591796875	56.180505519928\\
-0.3154296875	56.2544197536063\\
-0.31494140625	56.3280813616802\\
-0.314453125	56.4014853883058\\
-0.31396484375	56.4746268798411\\
-0.3134765625	56.5475008868936\\
-0.31298828125	56.6201024664125\\
-0.3125	56.692426683822\\
-0.31201171875	56.7644686151963\\
-0.3115234375	56.8362233494756\\
-0.31103515625	56.9076859907209\\
-0.310546875	56.978851660407\\
-0.31005859375	57.0497154997525\\
-0.3095703125	57.1202726720859\\
-0.30908203125	57.1905183652447\\
-0.30859375	57.2604477940089\\
-0.30810546875	57.3300562025638\\
-0.3076171875	57.3993388669942\\
-0.30712890625	57.4682910978042\\
-0.306640625	57.5369082424649\\
-0.30615234375	57.6051856879847\\
-0.3056640625	57.6731188635024\\
-0.30517578125	57.7407032428996\\
-0.3046875	57.8079343474311\\
-0.30419921875	57.8748077483713\\
-0.3037109375	57.9413190696729\\
-0.30322265625	58.0074639906374\\
-0.302734375	58.0732382485935\\
-0.30224609375	58.138637641581\\
-0.3017578125	58.2036580310394\\
-0.30126953125	58.2682953444958\\
-0.30078125	58.3325455782514\\
-0.30029296875	58.3964048000635\\
-0.2998046875	58.4598691518194\\
-0.29931640625	58.5229348522009\\
-0.298828125	58.5855981993341\\
-0.29833984375	58.6478555734249\\
-0.2978515625	58.7097034393737\\
-0.29736328125	58.7711383493695\\
-0.296875	58.8321569454579\\
-0.29638671875	58.8927559620821\\
-0.2958984375	58.9529322285913\\
-0.29541015625	59.0126826717174\\
-0.294921875	59.0720043180122\\
-0.29443359375	59.1308942962466\\
-0.2939453125	59.1893498397654\\
-0.29345703125	59.2473682887963\\
-0.29296875	59.3049470927111\\
-0.29248046875	59.3620838122337\\
-0.2919921875	59.4187761215933\\
-0.29150390625	59.4750218106224\\
-0.291015625	59.5308187867907\\
-0.29052734375	59.5861650771796\\
-0.2900390625	59.641058830389\\
-0.28955078125	59.6954983183763\\
-0.2890625	59.7494819382247\\
-0.28857421875	59.803008213839\\
-0.2880859375	59.8560757975646\\
-0.28759765625	59.9086834717301\\
-0.287109375	59.9608301501091\\
-0.28662109375	60.012514879301\\
-0.2861328125	60.0637368400267\\
-0.28564453125	60.1144953483406\\
-0.28515625	60.1647898567535\\
-0.28466796875	60.2146199552671\\
-0.2841796875	60.2639853723193\\
-0.28369140625	60.312885975636\\
-0.283203125	60.3613217729923\\
-0.28271484375	60.409292912878\\
-0.2822265625	60.4567996850688\\
-0.28173828125	60.5038425211026\\
-0.28125	60.55042199466\\
-0.28076171875	60.5965388218469\\
-0.2802734375	60.6421938613807\\
-0.27978515625	60.6873881146807\\
-0.279296875	60.7321227258595\\
-0.27880859375	60.7763989816188\\
-0.2783203125	60.8202183110475\\
-0.27783203125	60.8635822853247\\
-0.27734375	60.9064926173264\\
-0.27685546875	60.9489511611359\\
-0.2763671875	60.9909599114631\\
-0.27587890625	61.0325210029673\\
-0.275390625	61.0736367094901\\
-0.27490234375	61.1143094431971\\
-0.2744140625	61.1545417536295\\
-0.27392578125	61.1943363266688\\
-0.2734375	61.2336959834145\\
-0.27294921875	61.2726236789782\\
-0.2724609375	61.311122501194\\
-0.27197265625	61.3491956692499\\
-0.271484375	61.3868465322388\\
-0.27099609375	61.4240785676352\\
-0.2705078125	61.4608953796968\\
-0.27001953125	61.4973006977944\\
-0.26953125	61.5332983746742\\
-0.26904296875	61.5688923846508\\
-0.2685546875	61.6040868217398\\
-0.26806640625	61.6388858977263\\
-0.267578125	61.673293940176\\
-0.26708984375	61.7073153903911\\
-0.2666015625	61.740954801313\\
-0.26611328125	61.7742168353747\\
-0.265625	61.8071062623069\\
-0.26513671875	61.8396279568993\\
-0.2646484375	61.8717868967222\\
-0.26416015625	61.9035881598089\\
-0.263671875	61.9350369223031\\
-0.26318359375	61.9661384560751\\
-0.2626953125	61.9968981263084\\
-0.26220703125	62.0273213890609\\
-0.26171875	62.0574137888021\\
-0.26123046875	62.0871809559316\\
-0.2607421875	62.1166286042796\\
-0.26025390625	62.1457625285934\\
-0.259765625	62.1745886020137\\
-0.25927734375	62.2031127735406\\
-0.2587890625	62.231341065496\\
-0.25830078125	62.2592795709819\\
-0.2578125	62.286934451339\\
-0.25732421875	62.3143119336082\\
-0.2568359375	62.3414183079966\\
-0.25634765625	62.3682599253517\\
-0.255859375	62.3948431946454\\
-0.25537109375	62.4211745804708\\
-0.2548828125	62.4472606005539\\
-0.25439453125	62.4731078232818\\
-0.25390625	62.4987228652508\\
-0.25341796875	62.5241123888365\\
-0.2529296875	62.549283099785\\
-0.25244140625	62.5742417448327\\
-0.251953125	62.5989951093498\\
-0.25146484375	62.6235500150159\\
-0.2509765625	62.6479133175241\\
-0.25048828125	62.672091904318\\
-0.25	62.6960926923628\\
-0.24951171875	62.7199226259505\\
-0.2490234375	62.7435886745436\\
-0.24853515625	62.7670978306536\\
-0.248046875	62.7904571077612\\
-0.24755859375	62.8136735382745\\
-0.2470703125	62.8367541715299\\
-0.24658203125	62.8597060718328\\
-0.24609375	62.882536316543\\
-0.24560546875	62.9052519942023\\
-0.2451171875	62.927860202707\\
-0.24462890625	62.9503680475236\\
-0.244140625	62.9727826399527\\
-0.24365234375	62.9951110954345\\
-0.2431640625	63.017360531904\\
-0.24267578125	63.0395380681902\\
-0.2421875	63.0616508224623\\
-0.24169921875	63.0837059107228\\
-0.2412109375	63.1057104453475\\
-0.24072265625	63.1276715336711\\
-0.240234375	63.1495962766205\\
-0.23974609375	63.1714917673942\\
-0.2392578125	63.1933650901881\\
-0.23876953125	63.2152233189668\\
-0.23828125	63.2370735162819\\
-0.23779296875	63.2589227321347\\
-0.2373046875	63.2807780028835\\
-0.23681640625	63.3026463501979\\
-0.236328125	63.3245347800535\\
-0.23583984375	63.3464502817734\\
-0.2353515625	63.3683998271108\\
-0.23486328125	63.3903903693742\\
-0.234375	63.4124288425952\\
-0.23388671875	63.4345221607352\\
-0.2333984375	63.4566772169345\\
-0.23291015625	63.4789008827988\\
-0.232421875	63.5012000077261\\
-0.23193359375	63.5235814182691\\
-0.2314453125	63.5460519175364\\
-0.23095703125	63.568618284629\\
-0.23046875	63.5912872741104\\
-0.22998046875	63.6140656155127\\
-0.2294921875	63.6369600128747\\
-0.22900390625	63.6599771443107\\
-0.228515625	63.6831236616133\\
-0.22802734375	63.706406189883\\
-0.2275390625	63.7298313271879\\
-0.22705078125	63.7534056442511\\
-0.2265625	63.7771356841639\\
-0.22607421875	63.8010279621257\\
-0.2255859375	63.8250889652062\\
-0.22509765625	63.8493251521323\\
-0.224609375	63.8737429530958\\
-0.22412109375	63.8983487695821\\
-0.2236328125	63.9231489742178\\
-0.22314453125	63.9481499106371\\
-0.22265625	63.9733578933643\\
-0.22216796875	63.9987792077125\\
-0.2216796875	64.0244201096949\\
-0.22119140625	64.0502868259516\\
-0.220703125	64.0763855536851\\
-0.22021484375	64.1027224606082\\
-0.2197265625	64.1293036848988\\
-0.21923828125	64.1561353351638\\
-0.21875	64.1832234904071\\
-0.21826171875	64.2105742000031\\
-0.2177734375	64.2381934836736\\
-0.21728515625	64.2660873314647\\
-0.216796875	64.2942617037248\\
-0.21630859375	64.3227225310802\\
-0.2158203125	64.3514757144084\\
-0.21533203125	64.3805271248053\\
-0.21484375	64.4098826035468\\
-0.21435546875	64.4395479620419\\
-0.2138671875	64.469528981776\\
-0.21337890625	64.499831414242\\
-0.212890625	64.5304609808581\\
-0.21240234375	64.5614233728711\\
-0.2119140625	64.5927242512411\\
-0.21142578125	64.6243692465073\\
-0.2109375	64.6563639586336\\
-0.21044921875	64.6887139568301\\
-0.2099609375	64.721424779348\\
-0.20947265625	64.7545019332502\\
-0.208984375	64.7879508941481\\
-0.20849609375	64.82177710591\\
-0.2080078125	64.8559859803317\\
-0.20751953125	64.8905828967728\\
-0.20703125	64.9255732017525\\
-0.20654296875	64.9609622085011\\
-0.2060546875	64.9967551964712\\
-0.20556640625	65.0329574107971\\
-0.205078125	65.0695740617055\\
-0.20458984375	65.1066103238718\\
-0.2041015625	65.1440713357207\\
-0.20361328125	65.1819621986653\\
-0.203125	65.2202879762841\\
-0.20263671875	65.2590536934299\\
-0.2021484375	65.2982643352694\\
-0.20166015625	65.3379248462474\\
-0.201171875	65.3780401289726\\
-0.20068359375	65.4186150430222\\
-0.2001953125	65.4596544036591\\
-0.19970703125	65.501162980457\\
-0.19921875	65.5431454958313\\
-0.19873046875	65.5856066234691\\
-0.1982421875	65.6285509866531\\
-0.19775390625	65.6719831564741\\
-0.197265625	65.7159076499294\\
-0.19677734375	65.7603289278972\\
-0.1962890625	65.8052513929837\\
-0.19580078125	65.850679387237\\
-0.1953125	65.8966171897188\\
-0.19482421875	65.9430690139312\\
-0.1943359375	65.9900390050871\\
-0.19384765625	66.0375312372211\\
-0.193359375	66.0855497101295\\
-0.19287109375	66.1340983461351\\
-0.1923828125	66.1831809866647\\
-0.19189453125	66.2328013886355\\
-0.19140625	66.2829632206359\\
-0.19091796875	66.3336700588958\\
-0.1904296875	66.3849253830344\\
-0.18994140625	66.4367325715757\\
-0.189453125	66.4890948972218\\
-0.18896484375	66.5420155218729\\
-0.1884765625	66.5954974913804\\
-0.18798828125	66.6495437300248\\
-0.1875	66.7041570347038\\
-0.18701171875	66.759340068816\\
-0.1865234375	66.815095355832\\
-0.18603515625	66.8714252725328\\
-0.185546875	66.9283320419056\\
-0.18505859375	66.9858177256799\\
-0.1845703125	67.0438842164873\\
-0.18408203125	67.1025332296331\\
-0.18359375	67.1617662944569\\
-0.18310546875	67.2215847452693\\
-0.1826171875	67.2819897118449\\
-0.18212890625	67.3429821094549\\
-0.181640625	67.4045626284151\\
-0.18115234375	67.4667317231365\\
-0.1806640625	67.5294896006529\\
-0.18017578125	67.5928362086053\\
-0.1796875	67.6567712226636\\
-0.17919921875	67.72129403336\\
-0.1787109375	67.7864037323145\\
-0.17822265625	67.8520990978271\\
-0.177734375	67.9183785798122\\
-0.17724609375	67.9852402840543\\
-0.1767578125	68.0526819557533\\
-0.17626953125	68.120700962342\\
-0.17578125	68.1892942755446\\
-0.17529296875	68.258458452652\\
-0.1748046875	68.3281896169869\\
-0.17431640625	68.3984834375336\\
-0.173828125	68.469335107706\\
-0.17333984375	68.5407393232246\\
-0.1728515625	68.6126902590808\\
-0.17236328125	68.6851815455596\\
-0.171875	68.7582062432964\\
-0.17138671875	68.8317568173434\\
-0.1708984375	68.9058251102218\\
-0.17041015625	68.9804023139387\\
-0.169921875	69.055478940944\\
-0.16943359375	69.1310447940133\\
-0.1689453125	69.2070889350346\\
-0.16845703125	69.283599652687\\
-0.16796875	69.3605644289968\\
-0.16748046875	69.4379699047652\\
-0.1669921875	69.5158018438572\\
-0.16650390625	69.5940450963575\\
-0.166015625	69.6726835605889\\
-0.16552734375	69.7517001440065\\
-0.1650390625	69.8310767229836\\
-0.16455078125	69.9107941015042\\
-0.1640625	69.990831968798\\
-0.16357421875	70.071168855949\\
-0.1630859375	70.1517820915245\\
-0.16259765625	70.232647756282\\
-0.162109375	70.3137406370169\\
-0.16162109375	70.3950341796307\\
-0.1611328125	70.4765004415111\\
-0.16064453125	70.5581100433258\\
-0.16015625	70.6398321203532\\
-0.15966796875	70.7216342734843\\
-0.1591796875	70.8034825200482\\
-0.15869140625	70.8853412446375\\
-0.158203125	70.9671731501198\\
-0.15771484375	71.048939209054\\
-0.1572265625	71.1305986157461\\
-0.15673828125	71.2121087391998\\
-0.15625	71.2934250772546\\
-0.15576171875	71.374501212212\\
-0.1552734375	71.4552887682934\\
-0.15478515625	71.5357373712946\\
-0.154296875	71.6157946108224\\
-0.15380859375	71.6954060055451\\
-0.1533203125	71.7745149719\\
-0.15283203125	71.853062796741\\
-0.15234375	71.9309886144334\\
-0.15185546875	72.0082293889317\\
-0.1513671875	72.0847199014018\\
-0.15087890625	72.1603927439757\\
-0.150390625	72.2351783202467\\
-0.14990234375	72.3090048531343\\
-0.1494140625	72.3817984007624\\
-0.14892578125	72.4534828810099\\
-0.1484375	72.5239801053926\\
-0.14794921875	72.5932098229417\\
-0.1474609375	72.6610897747415\\
-0.14697265625	72.727535759764\\
-0.146484375	72.7924617126284\\
-0.14599609375	72.8557797938784\\
-0.1455078125	72.9174004933302\\
-0.14501953125	72.9772327469892\\
-0.14453125	73.0351840679806\\
-0.14404296875	73.0911606918543\\
-0.1435546875	73.1450677365444\\
-0.14306640625	73.1968093771625\\
-0.142578125	73.246289035693\\
-0.14208984375	73.2934095855351\\
-0.1416015625	73.3380735706981\\
-0.14111328125	73.3801834393123\\
-0.140625	73.4196417909577\\
-0.14013671875	73.4563516371483\\
-0.1396484375	73.4902166741382\\
-0.13916015625	73.5211415670288\\
-0.138671875	73.5490322439863\\
-0.13818359375	73.5737961991882\\
-0.1376953125	73.595342802944\\
-0.13720703125	73.6135836172588\\
-0.13671875	73.6284327149471\\
-0.13623046875	73.6398070002547\\
-0.1357421875	73.6476265288089\\
-0.13525390625	73.6518148246077\\
-0.134765625	73.6522991916646\\
-0.13427734375	73.6490110178602\\
-0.1337890625	73.6418860685283\\
-0.13330078125	73.630864767279\\
-0.1328125	73.6158924616158\\
-0.13232421875	73.5969196709538\\
-0.1318359375	73.573902314746\\
-0.13134765625	73.5468019185726\\
-0.130859375	73.5155857961998\\
-0.13037109375	73.4802272058299\\
-0.1298828125	73.4407054789865\\
-0.12939453125	73.3970061207408\\
-0.12890625	73.3491208802577\\
-0.12841796875	73.2970477909447\\
-0.1279296875	73.2407911797904\\
-0.12744140625	73.1803616457941\\
-0.126953125	73.1157760077139\\
-0.12646484375	73.0470572216672\\
-0.1259765625	72.9742342694254\\
-0.12548828125	72.8973420185345\\
-0.125	72.8164210556617\\
-0.12451171875	72.7315174948242\\
-0.1240234375	72.6426827623574\\
-0.12353515625	72.5499733606848\\
-0.123046875	72.4534506131052\\
-0.12255859375	72.3531803919228\\
-0.1220703125	72.2492328323424\\
-0.12158203125	72.1416820346053\\
-0.12109375	72.0306057568416\\
-0.12060546875	71.9160851011225\\
-0.1201171875	71.7982041951277\\
-0.11962890625	71.6770498717809\\
-0.119140625	71.5527113490997\\
-0.11865234375	71.4252799123883\\
-0.1181640625	71.2948486007591\\
-0.11767578125	71.1615118998056\\
-0.1171875	71.0253654420988\\
-0.11669921875	70.8865057169817\\
-0.1162109375	70.7450297909765\\
-0.11572265625	70.6010350399248\\
-0.115234375	70.4546188938006\\
-0.11474609375	70.3058785949678\\
-0.1142578125	70.1549109704763\\
-0.11376953125	70.0018122188256\\
-0.11328125	69.8466777114772\\
-0.11279296875	69.6896018092551\\
-0.1123046875	69.5306776936342\\
-0.11181640625	69.3699972128037\\
-0.111328125	69.2076507422889\\
-0.11083984375	69.0437270598104\\
-0.1103515625	68.8783132339915\\
-0.10986328125	68.7114945264412\\
-0.109375	68.5433543066951\\
-0.10888671875	68.3739739794419\\
-0.1083984375	68.2034329234283\\
-0.10791015625	68.0318084414107\\
-0.107421875	67.8591757205011\\
-0.10693359375	67.6856078022479\\
-0.1064453125	67.5111755617837\\
-0.10595703125	67.3359476953854\\
-0.10546875	67.1599907157857\\
-0.10498046875	66.9833689546073\\
-0.1044921875	66.8061445712892\\
-0.10400390625	66.6283775679125\\
-0.103515625	66.4501258093431\\
-0.10302734375	66.2714450481427\\
-0.1025390625	66.0923889537225\\
-0.10205078125	65.9130091452409\\
-0.1015625	65.733355227783\\
-0.10107421875	65.5534748313746\\
-0.1005859375	65.3734136524283\\
-0.10009765625	65.1932154972354\\
-0.099609375	65.0129223271513\\
-0.09912109375	64.8325743051512\\
-0.0986328125	64.6522098434481\\
-0.09814453125	64.4718656519124\\
-0.09765625	64.291576787028\\
-0.09716796875	64.1113767011709\\
-0.0966796875	63.9312972919968\\
-0.09619140625	63.751368951759\\
-0.095703125	63.5716206163892\\
-0.09521484375	63.3920798141966\\
-0.0947265625	63.2127727140533\\
-0.09423828125	63.0337241729562\\
-0.09375	62.8549577828635\\
-0.09326171875	62.6764959167213\\
-0.0927734375	62.4983597736076\\
-0.09228515625	62.3205694229315\\
-0.091796875	62.1431438476374\\
-0.09130859375	61.9661009863702\\
-0.0908203125	61.7894577745726\\
-0.09033203125	61.6132301844825\\
-0.08984375	61.4374332640204\\
-0.08935546875	61.2620811745471\\
-0.0888671875	61.0871872274903\\
-0.08837890625	60.9127639198373\\
-0.087890625	60.7388229684944\\
-0.08740234375	60.5653753435251\\
-0.0869140625	60.3924313002724\\
-0.08642578125	60.2200004103814\\
-0.0859375	60.048091591738\\
-0.08544921875	59.8767131373399\\
-0.0849609375	59.7058727431235\\
-0.08447265625	59.5355775347643\\
-0.083984375	59.365834093478\\
-0.08349609375	59.1966484808413\\
-0.0830078125	59.028026262663\\
-0.08251953125	58.8599725319258\\
-0.08203125	58.6924919308281\\
-0.08154296875	58.5255886719503\\
-0.0810546875	58.3592665585725\\
-0.08056640625	58.1935290041706\\
-0.080078125	58.0283790511164\\
-0.07958984375	57.8638193886078\\
-0.0791015625	57.699852369855\\
-0.07861328125	57.5364800285509\\
-0.078125	57.3737040946452\\
-0.07763671875	57.2115260094535\\
-0.0771484375	57.0499469401208\\
-0.07666015625	56.8889677934656\\
-0.076171875	56.7285892292275\\
-0.07568359375	56.56881167274\\
-0.0751953125	56.4096353270521\\
-0.07470703125	56.2510601845176\\
-0.07421875	56.0930860378768\\
-0.07373046875	55.9357124908468\\
-0.0732421875	55.7789389682426\\
-0.07275390625	55.6227647256464\\
-0.072265625	55.4671888586456\\
-0.07177734375	55.3122103116533\\
-0.0712890625	55.1578278863335\\
-0.07080078125	55.0040402496435\\
-0.0703125	54.8508459415104\\
-0.06982421875	54.6982433821584\\
-0.0693359375	54.546230879101\\
-0.06884765625	54.3948066338101\\
-0.068359375	54.2439687480798\\
-0.06787109375	54.0937152300951\\
-0.0673828125	53.9440440002183\\
-0.06689453125	53.7949528965069\\
-0.06640625	53.6464396799732\\
-0.06591796875	53.4985020395969\\
-0.0654296875	53.3511375971027\\
-0.06494140625	53.2043439115115\\
-0.064453125	53.0581184834765\\
-0.06396484375	52.9124587594124\\
-0.0634765625	52.7673621354287\\
-0.06298828125	52.622825961073\\
-0.0625	52.4788475428963\\
-0.06201171875	52.3354241478438\\
-0.0615234375	52.1925530064834\\
-0.06103515625	52.0502313160766\\
-0.060546875	51.9084562434981\\
-0.06005859375	51.7672249280143\\
-0.0595703125	51.6265344839227\\
-0.05908203125	51.4863820030622\\
-0.05859375	51.3467645571979\\
-0.05810546875	51.2076792002861\\
-0.0576171875	51.0691229706267\\
-0.05712890625	50.9310928929058\\
-0.056640625	50.7935859801355\\
-0.05615234375	50.6565992354934\\
-0.0556640625	50.5201296540685\\
-0.05517578125	50.384174224517\\
-0.0546875	50.248729930631\\
-0.05419921875	50.1137937528256\\
-0.0537109375	49.9793626695477\\
-0.05322265625	49.8454336586093\\
-0.052734375	49.7120036984505\\
-0.05224609375	49.5790697693322\\
-0.0517578125	49.4466288544669\\
-0.05126953125	49.3146779410842\\
-0.05078125	49.1832140214398\\
-0.05029296875	49.0522340937659\\
-0.0498046875	48.9217351631698\\
-0.04931640625	48.7917142424796\\
-0.048828125	48.6621683530416\\
-0.04833984375	48.5330945254712\\
-0.0478515625	48.4044898003585\\
-0.04736328125	48.2763512289331\\
-0.046875	48.148675873686\\
-0.04638671875	48.021460808956\\
-0.0458984375	47.8947031214761\\
-0.04541015625	47.7683999108881\\
-0.044921875	47.6425482902213\\
-0.04443359375	47.5171453863409\\
-0.0439453125	47.3921883403653\\
-0.04345703125	47.2676743080557\\
-0.04296875	47.1436004601768\\
-0.04248046875	47.0199639828328\\
-0.0419921875	46.896762077777\\
-0.04150390625	46.7739919626993\\
-0.041015625	46.65165087149\\
-0.04052734375	46.5297360544827\\
-0.0400390625	46.4082447786766\\
-0.03955078125	46.2871743279397\\
-0.0390625	46.1665220031927\\
-0.03857421875	46.0462851225768\\
-0.0380859375	45.9264610216032\\
-0.03759765625	45.8070470532879\\
-0.037109375	45.6880405882711\\
-0.03662109375	45.5694390149219\\
-0.0361328125	45.4512397394304\\
-0.03564453125	45.3334401858856\\
-0.03515625	45.2160377963417\\
-0.03466796875	45.0990300308737\\
-0.0341796875	44.98241436762\\
-0.03369140625	44.8661883028163\\
-0.033203125	44.7503493508188\\
-0.03271484375	44.634895044118\\
-0.0322265625	44.519822933344\\
-0.03173828125	44.4051305872631\\
-0.03125	44.2908155927666\\
-0.03076171875	44.1768755548516\\
-0.0302734375	44.0633080965959\\
-0.02978515625	43.9501108591246\\
-0.029296875	43.8372815015716\\
-0.02880859375	43.7248177010343\\
-0.0283203125	43.6127171525235\\
-0.02783203125	43.5009775689072\\
-0.02734375	43.3895966808496\\
-0.02685546875	43.2785722367458\\
-0.0263671875	43.1679020026514\\
-0.02587890625	43.057583762209\\
-0.025390625	42.9476153165694\\
-0.02490234375	42.8379944843101\\
-0.0244140625	42.7287191013505\\
-0.02392578125	42.6197870208634\\
-0.0234375	42.5111961131831\\
-0.02294921875	42.4029442657125\\
-0.0224609375	42.2950293828254\\
-0.02197265625	42.187449385768\\
-0.021484375	42.0802022125574\\
-0.02099609375	41.9732858178783\\
-0.0205078125	41.8666981729777\\
-0.02001953125	41.7604372655582\\
-0.01953125	41.6545010996689\\
-0.01904296875	41.548887695596\\
-0.0185546875	41.4435950897507\\
-0.01806640625	41.338621334557\\
-0.017578125	41.2339644983383\\
-0.01708984375	41.1296226652018\\
-0.0166015625	41.0255939349241\\
-0.01611328125	40.9218764228341\\
-0.015625	40.8184682596959\\
-0.01513671875	40.7153675915921\\
-0.0146484375	40.6125725798048\\
-0.01416015625	40.5100814006976\\
-0.013671875	40.4078922455966\\
-0.01318359375	40.3060033206715\\
-0.0126953125	40.2044128468159\\
-0.01220703125	40.1031190595287\\
-0.01171875	40.0021202087935\\
-0.01123046875	39.9014145589602\\
-0.0107421875	39.8010003886251\\
-0.01025390625	39.7008759905116\\
-0.009765625	39.6010396713512\\
-0.00927734375	39.5014897517644\\
-0.0087890625	39.4022245661423\\
-0.00830078125	39.3032424625279\\
-0.0078125	39.2045418024984\\
-0.00732421875	39.1061209610472\\
-0.0068359375	39.0079783264669\\
-0.00634765625	38.9101123002327\\
-0.005859375	38.8125212968852\\
-0.00537109375	38.7152037439158\\
-0.0048828125	38.6181580816502\\
-0.00439453125	38.5213827631349\\
-0.00390625	38.424876254022\\
-0.00341796875	38.3286370324562\\
-0.0029296875	38.2326635889619\\
-0.00244140625	38.1369544263309\\
-0.001953125	38.0415080595106\\
-0.00146484375	37.9463230154936\\
-0.0009765625	37.8513978332068\\
-0.00048828125	37.7567310634028\\
0	37.6623212685503\\
0.00048828125	37.7567310634028\\
0.0009765625	37.8513978332068\\
0.00146484375	37.9463230154936\\
0.001953125	38.0415080595106\\
0.00244140625	38.1369544263309\\
0.0029296875	38.2326635889619\\
0.00341796875	38.3286370324562\\
0.00390625	38.424876254022\\
0.00439453125	38.5213827631349\\
0.0048828125	38.6181580816502\\
0.00537109375	38.7152037439158\\
0.005859375	38.8125212968852\\
0.00634765625	38.9101123002327\\
0.0068359375	39.0079783264669\\
0.00732421875	39.1061209610472\\
0.0078125	39.2045418024984\\
0.00830078125	39.3032424625279\\
0.0087890625	39.4022245661423\\
0.00927734375	39.5014897517644\\
0.009765625	39.6010396713512\\
0.01025390625	39.7008759905116\\
0.0107421875	39.8010003886251\\
0.01123046875	39.9014145589602\\
0.01171875	40.0021202087935\\
0.01220703125	40.1031190595287\\
0.0126953125	40.2044128468159\\
0.01318359375	40.3060033206715\\
0.013671875	40.4078922455966\\
0.01416015625	40.5100814006976\\
0.0146484375	40.6125725798048\\
0.01513671875	40.7153675915921\\
0.015625	40.8184682596959\\
0.01611328125	40.9218764228341\\
0.0166015625	41.0255939349241\\
0.01708984375	41.1296226652018\\
0.017578125	41.2339644983383\\
0.01806640625	41.338621334557\\
0.0185546875	41.4435950897507\\
0.01904296875	41.548887695596\\
0.01953125	41.6545010996689\\
0.02001953125	41.7604372655582\\
0.0205078125	41.8666981729777\\
0.02099609375	41.9732858178783\\
0.021484375	42.0802022125574\\
0.02197265625	42.187449385768\\
0.0224609375	42.2950293828254\\
0.02294921875	42.4029442657125\\
0.0234375	42.5111961131831\\
0.02392578125	42.6197870208634\\
0.0244140625	42.7287191013505\\
0.02490234375	42.8379944843101\\
0.025390625	42.9476153165694\\
0.02587890625	43.057583762209\\
0.0263671875	43.1679020026514\\
0.02685546875	43.2785722367458\\
0.02734375	43.3895966808496\\
0.02783203125	43.5009775689072\\
0.0283203125	43.6127171525235\\
0.02880859375	43.7248177010343\\
0.029296875	43.8372815015716\\
0.02978515625	43.9501108591246\\
0.0302734375	44.0633080965959\\
0.03076171875	44.1768755548516\\
0.03125	44.2908155927666\\
0.03173828125	44.4051305872631\\
0.0322265625	44.519822933344\\
0.03271484375	44.634895044118\\
0.033203125	44.7503493508188\\
0.03369140625	44.8661883028163\\
0.0341796875	44.98241436762\\
0.03466796875	45.0990300308737\\
0.03515625	45.2160377963417\\
0.03564453125	45.3334401858856\\
0.0361328125	45.4512397394304\\
0.03662109375	45.5694390149219\\
0.037109375	45.6880405882711\\
0.03759765625	45.8070470532879\\
0.0380859375	45.9264610216032\\
0.03857421875	46.0462851225768\\
0.0390625	46.1665220031927\\
0.03955078125	46.2871743279397\\
0.0400390625	46.4082447786766\\
0.04052734375	46.5297360544827\\
0.041015625	46.65165087149\\
0.04150390625	46.7739919626993\\
0.0419921875	46.896762077777\\
0.04248046875	47.0199639828328\\
0.04296875	47.1436004601768\\
0.04345703125	47.2676743080557\\
0.0439453125	47.3921883403653\\
0.04443359375	47.5171453863409\\
0.044921875	47.6425482902213\\
0.04541015625	47.7683999108881\\
0.0458984375	47.8947031214761\\
0.04638671875	48.021460808956\\
0.046875	48.148675873686\\
0.04736328125	48.2763512289331\\
0.0478515625	48.4044898003585\\
0.04833984375	48.5330945254712\\
0.048828125	48.6621683530416\\
0.04931640625	48.7917142424796\\
0.0498046875	48.9217351631698\\
0.05029296875	49.0522340937659\\
0.05078125	49.1832140214398\\
0.05126953125	49.3146779410842\\
0.0517578125	49.4466288544669\\
0.05224609375	49.5790697693322\\
0.052734375	49.7120036984505\\
0.05322265625	49.8454336586093\\
0.0537109375	49.9793626695477\\
0.05419921875	50.1137937528256\\
0.0546875	50.248729930631\\
0.05517578125	50.384174224517\\
0.0556640625	50.5201296540685\\
0.05615234375	50.6565992354934\\
0.056640625	50.7935859801355\\
0.05712890625	50.9310928929058\\
0.0576171875	51.0691229706267\\
0.05810546875	51.2076792002861\\
0.05859375	51.3467645571979\\
0.05908203125	51.4863820030622\\
0.0595703125	51.6265344839227\\
0.06005859375	51.7672249280143\\
0.060546875	51.9084562434981\\
0.06103515625	52.0502313160766\\
0.0615234375	52.1925530064834\\
0.06201171875	52.3354241478438\\
0.0625	52.4788475428963\\
0.06298828125	52.622825961073\\
0.0634765625	52.7673621354287\\
0.06396484375	52.9124587594124\\
0.064453125	53.0581184834765\\
0.06494140625	53.2043439115115\\
0.0654296875	53.3511375971027\\
0.06591796875	53.4985020395969\\
0.06640625	53.6464396799732\\
0.06689453125	53.7949528965069\\
0.0673828125	53.9440440002183\\
0.06787109375	54.0937152300951\\
0.068359375	54.2439687480798\\
0.06884765625	54.3948066338101\\
0.0693359375	54.546230879101\\
0.06982421875	54.6982433821584\\
0.0703125	54.8508459415104\\
0.07080078125	55.0040402496435\\
0.0712890625	55.1578278863335\\
0.07177734375	55.3122103116533\\
0.072265625	55.4671888586456\\
0.07275390625	55.6227647256464\\
0.0732421875	55.7789389682426\\
0.07373046875	55.9357124908468\\
0.07421875	56.0930860378768\\
0.07470703125	56.2510601845176\\
0.0751953125	56.4096353270521\\
0.07568359375	56.56881167274\\
0.076171875	56.7285892292275\\
0.07666015625	56.8889677934656\\
0.0771484375	57.0499469401208\\
0.07763671875	57.2115260094535\\
0.078125	57.3737040946452\\
0.07861328125	57.5364800285509\\
0.0791015625	57.699852369855\\
0.07958984375	57.8638193886078\\
0.080078125	58.0283790511164\\
0.08056640625	58.1935290041706\\
0.0810546875	58.3592665585725\\
0.08154296875	58.5255886719503\\
0.08203125	58.6924919308281\\
0.08251953125	58.8599725319258\\
0.0830078125	59.028026262663\\
0.08349609375	59.1966484808413\\
0.083984375	59.365834093478\\
0.08447265625	59.5355775347643\\
0.0849609375	59.7058727431235\\
0.08544921875	59.8767131373399\\
0.0859375	60.048091591738\\
0.08642578125	60.2200004103814\\
0.0869140625	60.3924313002724\\
0.08740234375	60.5653753435251\\
0.087890625	60.7388229684944\\
0.08837890625	60.9127639198373\\
0.0888671875	61.0871872274903\\
0.08935546875	61.2620811745471\\
0.08984375	61.4374332640204\\
0.09033203125	61.6132301844825\\
0.0908203125	61.7894577745726\\
0.09130859375	61.9661009863702\\
0.091796875	62.1431438476374\\
0.09228515625	62.3205694229315\\
0.0927734375	62.4983597736076\\
0.09326171875	62.6764959167213\\
0.09375	62.8549577828635\\
0.09423828125	63.0337241729562\\
0.0947265625	63.2127727140533\\
0.09521484375	63.3920798141966\\
0.095703125	63.5716206163892\\
0.09619140625	63.751368951759\\
0.0966796875	63.9312972919968\\
0.09716796875	64.1113767011709\\
0.09765625	64.291576787028\\
0.09814453125	64.4718656519124\\
0.0986328125	64.6522098434481\\
0.09912109375	64.8325743051512\\
0.099609375	65.0129223271513\\
0.10009765625	65.1932154972354\\
0.1005859375	65.3734136524283\\
0.10107421875	65.5534748313746\\
0.1015625	65.733355227783\\
0.10205078125	65.9130091452409\\
0.1025390625	66.0923889537225\\
0.10302734375	66.2714450481427\\
0.103515625	66.4501258093431\\
0.10400390625	66.6283775679125\\
0.1044921875	66.8061445712892\\
0.10498046875	66.9833689546073\\
0.10546875	67.1599907157857\\
0.10595703125	67.3359476953854\\
0.1064453125	67.5111755617837\\
0.10693359375	67.6856078022479\\
0.107421875	67.8591757205011\\
0.10791015625	68.0318084414107\\
0.1083984375	68.2034329234283\\
0.10888671875	68.3739739794419\\
0.109375	68.5433543066951\\
0.10986328125	68.7114945264412\\
0.1103515625	68.8783132339915\\
0.11083984375	69.0437270598104\\
0.111328125	69.2076507422889\\
0.11181640625	69.3699972128037\\
0.1123046875	69.5306776936342\\
0.11279296875	69.6896018092551\\
0.11328125	69.8466777114772\\
0.11376953125	70.0018122188256\\
0.1142578125	70.1549109704763\\
0.11474609375	70.3058785949678\\
0.115234375	70.4546188938006\\
0.11572265625	70.6010350399248\\
0.1162109375	70.7450297909765\\
0.11669921875	70.8865057169817\\
0.1171875	71.0253654420988\\
0.11767578125	71.1615118998056\\
0.1181640625	71.2948486007591\\
0.11865234375	71.4252799123883\\
0.119140625	71.5527113490997\\
0.11962890625	71.6770498717809\\
0.1201171875	71.7982041951277\\
0.12060546875	71.9160851011225\\
0.12109375	72.0306057568416\\
0.12158203125	72.1416820346053\\
0.1220703125	72.2492328323424\\
0.12255859375	72.3531803919228\\
0.123046875	72.4534506131052\\
0.12353515625	72.5499733606848\\
0.1240234375	72.6426827623574\\
0.12451171875	72.7315174948242\\
0.125	72.8164210556617\\
0.12548828125	72.8973420185345\\
0.1259765625	72.9742342694254\\
0.12646484375	73.0470572216672\\
0.126953125	73.1157760077139\\
0.12744140625	73.1803616457941\\
0.1279296875	73.2407911797904\\
0.12841796875	73.2970477909447\\
0.12890625	73.3491208802577\\
0.12939453125	73.3970061207408\\
0.1298828125	73.4407054789865\\
0.13037109375	73.4802272058299\\
0.130859375	73.5155857961998\\
0.13134765625	73.5468019185726\\
0.1318359375	73.573902314746\\
0.13232421875	73.5969196709538\\
0.1328125	73.6158924616158\\
0.13330078125	73.630864767279\\
0.1337890625	73.6418860685283\\
0.13427734375	73.6490110178602\\
0.134765625	73.6522991916646\\
0.13525390625	73.6518148246077\\
0.1357421875	73.6476265288089\\
0.13623046875	73.6398070002547\\
0.13671875	73.6284327149471\\
0.13720703125	73.6135836172588\\
0.1376953125	73.595342802944\\
0.13818359375	73.5737961991882\\
0.138671875	73.5490322439863\\
0.13916015625	73.5211415670288\\
0.1396484375	73.4902166741382\\
0.14013671875	73.4563516371483\\
0.140625	73.4196417909577\\
0.14111328125	73.3801834393123\\
0.1416015625	73.3380735706981\\
0.14208984375	73.2934095855351\\
0.142578125	73.246289035693\\
0.14306640625	73.1968093771625\\
0.1435546875	73.1450677365444\\
0.14404296875	73.0911606918543\\
0.14453125	73.0351840679806\\
0.14501953125	72.9772327469892\\
0.1455078125	72.9174004933302\\
0.14599609375	72.8557797938784\\
0.146484375	72.7924617126284\\
0.14697265625	72.727535759764\\
0.1474609375	72.6610897747415\\
0.14794921875	72.5932098229417\\
0.1484375	72.5239801053926\\
0.14892578125	72.4534828810099\\
0.1494140625	72.3817984007624\\
0.14990234375	72.3090048531343\\
0.150390625	72.2351783202467\\
0.15087890625	72.1603927439757\\
0.1513671875	72.0847199014018\\
0.15185546875	72.0082293889317\\
0.15234375	71.9309886144334\\
0.15283203125	71.853062796741\\
0.1533203125	71.7745149719\\
0.15380859375	71.6954060055451\\
0.154296875	71.6157946108224\\
0.15478515625	71.5357373712946\\
0.1552734375	71.4552887682934\\
0.15576171875	71.374501212212\\
0.15625	71.2934250772546\\
0.15673828125	71.2121087391998\\
0.1572265625	71.1305986157461\\
0.15771484375	71.048939209054\\
0.158203125	70.9671731501198\\
0.15869140625	70.8853412446375\\
0.1591796875	70.8034825200482\\
0.15966796875	70.7216342734843\\
0.16015625	70.6398321203532\\
0.16064453125	70.5581100433258\\
0.1611328125	70.4765004415111\\
0.16162109375	70.3950341796307\\
0.162109375	70.3137406370169\\
0.16259765625	70.232647756282\\
0.1630859375	70.1517820915245\\
0.16357421875	70.071168855949\\
0.1640625	69.990831968798\\
0.16455078125	69.9107941015042\\
0.1650390625	69.8310767229836\\
0.16552734375	69.7517001440065\\
0.166015625	69.6726835605889\\
0.16650390625	69.5940450963575\\
0.1669921875	69.5158018438572\\
0.16748046875	69.4379699047652\\
0.16796875	69.3605644289968\\
0.16845703125	69.283599652687\\
0.1689453125	69.2070889350346\\
0.16943359375	69.1310447940133\\
0.169921875	69.055478940944\\
0.17041015625	68.9804023139387\\
0.1708984375	68.9058251102218\\
0.17138671875	68.8317568173434\\
0.171875	68.7582062432964\\
0.17236328125	68.6851815455596\\
0.1728515625	68.6126902590808\\
0.17333984375	68.5407393232246\\
0.173828125	68.469335107706\\
0.17431640625	68.3984834375336\\
0.1748046875	68.3281896169869\\
0.17529296875	68.258458452652\\
0.17578125	68.1892942755446\\
0.17626953125	68.120700962342\\
0.1767578125	68.0526819557533\\
0.17724609375	67.9852402840543\\
0.177734375	67.9183785798122\\
0.17822265625	67.8520990978271\\
0.1787109375	67.7864037323145\\
0.17919921875	67.72129403336\\
0.1796875	67.6567712226636\\
0.18017578125	67.5928362086053\\
0.1806640625	67.5294896006529\\
0.18115234375	67.4667317231365\\
0.181640625	67.4045626284151\\
0.18212890625	67.3429821094549\\
0.1826171875	67.2819897118449\\
0.18310546875	67.2215847452693\\
0.18359375	67.1617662944569\\
0.18408203125	67.1025332296331\\
0.1845703125	67.0438842164873\\
0.18505859375	66.9858177256799\\
0.185546875	66.9283320419056\\
0.18603515625	66.8714252725328\\
0.1865234375	66.815095355832\\
0.18701171875	66.759340068816\\
0.1875	66.7041570347038\\
0.18798828125	66.6495437300248\\
0.1884765625	66.5954974913804\\
0.18896484375	66.5420155218729\\
0.189453125	66.4890948972218\\
0.18994140625	66.4367325715757\\
0.1904296875	66.3849253830344\\
0.19091796875	66.3336700588958\\
0.19140625	66.2829632206359\\
0.19189453125	66.2328013886355\\
0.1923828125	66.1831809866647\\
0.19287109375	66.1340983461351\\
0.193359375	66.0855497101295\\
0.19384765625	66.0375312372211\\
0.1943359375	65.9900390050871\\
0.19482421875	65.9430690139312\\
0.1953125	65.8966171897188\\
0.19580078125	65.850679387237\\
0.1962890625	65.8052513929837\\
0.19677734375	65.7603289278972\\
0.197265625	65.7159076499294\\
0.19775390625	65.6719831564741\\
0.1982421875	65.6285509866531\\
0.19873046875	65.5856066234691\\
0.19921875	65.5431454958313\\
0.19970703125	65.501162980457\\
0.2001953125	65.4596544036591\\
0.20068359375	65.4186150430222\\
0.201171875	65.3780401289726\\
0.20166015625	65.3379248462474\\
0.2021484375	65.2982643352694\\
0.20263671875	65.2590536934299\\
0.203125	65.2202879762841\\
0.20361328125	65.1819621986653\\
0.2041015625	65.1440713357207\\
0.20458984375	65.1066103238718\\
0.205078125	65.0695740617055\\
0.20556640625	65.0329574107971\\
0.2060546875	64.9967551964712\\
0.20654296875	64.9609622085011\\
0.20703125	64.9255732017525\\
0.20751953125	64.8905828967728\\
0.2080078125	64.8559859803317\\
0.20849609375	64.82177710591\\
0.208984375	64.7879508941481\\
0.20947265625	64.7545019332502\\
0.2099609375	64.721424779348\\
0.21044921875	64.6887139568301\\
0.2109375	64.6563639586336\\
0.21142578125	64.6243692465073\\
0.2119140625	64.5927242512411\\
0.21240234375	64.5614233728711\\
0.212890625	64.5304609808581\\
0.21337890625	64.499831414242\\
0.2138671875	64.469528981776\\
0.21435546875	64.4395479620419\\
0.21484375	64.4098826035468\\
0.21533203125	64.3805271248053\\
0.2158203125	64.3514757144084\\
0.21630859375	64.3227225310802\\
0.216796875	64.2942617037248\\
0.21728515625	64.2660873314647\\
0.2177734375	64.2381934836736\\
0.21826171875	64.2105742000031\\
0.21875	64.1832234904071\\
0.21923828125	64.1561353351638\\
0.2197265625	64.1293036848988\\
0.22021484375	64.1027224606082\\
0.220703125	64.0763855536851\\
0.22119140625	64.0502868259516\\
0.2216796875	64.0244201096949\\
0.22216796875	63.9987792077125\\
0.22265625	63.9733578933643\\
0.22314453125	63.9481499106371\\
0.2236328125	63.9231489742178\\
0.22412109375	63.8983487695821\\
0.224609375	63.8737429530958\\
0.22509765625	63.8493251521323\\
0.2255859375	63.8250889652062\\
0.22607421875	63.8010279621257\\
0.2265625	63.7771356841639\\
0.22705078125	63.7534056442511\\
0.2275390625	63.7298313271879\\
0.22802734375	63.706406189883\\
0.228515625	63.6831236616133\\
0.22900390625	63.6599771443107\\
0.2294921875	63.6369600128747\\
0.22998046875	63.6140656155127\\
0.23046875	63.5912872741104\\
0.23095703125	63.568618284629\\
0.2314453125	63.5460519175364\\
0.23193359375	63.5235814182691\\
0.232421875	63.5012000077261\\
0.23291015625	63.4789008827988\\
0.2333984375	63.4566772169345\\
0.23388671875	63.4345221607352\\
0.234375	63.4124288425952\\
0.23486328125	63.3903903693742\\
0.2353515625	63.3683998271108\\
0.23583984375	63.3464502817734\\
0.236328125	63.3245347800535\\
0.23681640625	63.3026463501979\\
0.2373046875	63.2807780028835\\
0.23779296875	63.2589227321347\\
0.23828125	63.2370735162819\\
0.23876953125	63.2152233189668\\
0.2392578125	63.1933650901881\\
0.23974609375	63.1714917673942\\
0.240234375	63.1495962766205\\
0.24072265625	63.1276715336711\\
0.2412109375	63.1057104453475\\
0.24169921875	63.0837059107228\\
0.2421875	63.0616508224623\\
0.24267578125	63.0395380681902\\
0.2431640625	63.017360531904\\
0.24365234375	62.9951110954345\\
0.244140625	62.9727826399527\\
0.24462890625	62.9503680475236\\
0.2451171875	62.927860202707\\
0.24560546875	62.9052519942023\\
0.24609375	62.882536316543\\
0.24658203125	62.8597060718328\\
0.2470703125	62.8367541715299\\
0.24755859375	62.8136735382745\\
0.248046875	62.7904571077612\\
0.24853515625	62.7670978306536\\
0.2490234375	62.7435886745436\\
0.24951171875	62.7199226259505\\
0.25	62.6960926923628\\
0.25048828125	62.672091904318\\
0.2509765625	62.6479133175241\\
0.25146484375	62.6235500150159\\
0.251953125	62.5989951093498\\
0.25244140625	62.5742417448327\\
0.2529296875	62.549283099785\\
0.25341796875	62.5241123888365\\
0.25390625	62.4987228652508\\
0.25439453125	62.4731078232818\\
0.2548828125	62.4472606005539\\
0.25537109375	62.4211745804708\\
0.255859375	62.3948431946454\\
0.25634765625	62.3682599253517\\
0.2568359375	62.3414183079966\\
0.25732421875	62.3143119336082\\
0.2578125	62.286934451339\\
0.25830078125	62.2592795709819\\
0.2587890625	62.231341065496\\
0.25927734375	62.2031127735406\\
0.259765625	62.1745886020137\\
0.26025390625	62.1457625285934\\
0.2607421875	62.1166286042796\\
0.26123046875	62.0871809559316\\
0.26171875	62.0574137888021\\
0.26220703125	62.0273213890609\\
0.2626953125	61.9968981263084\\
0.26318359375	61.9661384560751\\
0.263671875	61.9350369223031\\
0.26416015625	61.9035881598089\\
0.2646484375	61.8717868967222\\
0.26513671875	61.8396279568993\\
0.265625	61.8071062623069\\
0.26611328125	61.7742168353747\\
0.2666015625	61.740954801313\\
0.26708984375	61.7073153903911\\
0.267578125	61.673293940176\\
0.26806640625	61.6388858977263\\
0.2685546875	61.6040868217398\\
0.26904296875	61.5688923846508\\
0.26953125	61.5332983746742\\
0.27001953125	61.4973006977944\\
0.2705078125	61.4608953796968\\
0.27099609375	61.4240785676352\\
0.271484375	61.3868465322388\\
0.27197265625	61.3491956692499\\
0.2724609375	61.311122501194\\
0.27294921875	61.2726236789782\\
0.2734375	61.2336959834145\\
0.27392578125	61.1943363266688\\
0.2744140625	61.1545417536295\\
0.27490234375	61.1143094431971\\
0.275390625	61.0736367094901\\
0.27587890625	61.0325210029673\\
0.2763671875	60.9909599114631\\
0.27685546875	60.9489511611359\\
0.27734375	60.9064926173264\\
0.27783203125	60.8635822853247\\
0.2783203125	60.8202183110475\\
0.27880859375	60.7763989816188\\
0.279296875	60.7321227258595\\
0.27978515625	60.6873881146807\\
0.2802734375	60.6421938613807\\
0.28076171875	60.5965388218469\\
0.28125	60.55042199466\\
0.28173828125	60.5038425211026\\
0.2822265625	60.4567996850688\\
0.28271484375	60.409292912878\\
0.283203125	60.3613217729923\\
0.28369140625	60.312885975636\\
0.2841796875	60.2639853723193\\
0.28466796875	60.2146199552671\\
0.28515625	60.1647898567535\\
0.28564453125	60.1144953483406\\
0.2861328125	60.0637368400267\\
0.28662109375	60.012514879301\\
0.287109375	59.9608301501091\\
0.28759765625	59.9086834717301\\
0.2880859375	59.8560757975646\\
0.28857421875	59.803008213839\\
0.2890625	59.7494819382247\\
0.28955078125	59.6954983183763\\
0.2900390625	59.641058830389\\
0.29052734375	59.5861650771796\\
0.291015625	59.5308187867907\\
0.29150390625	59.4750218106224\\
0.2919921875	59.4187761215933\\
0.29248046875	59.3620838122337\\
0.29296875	59.3049470927111\\
0.29345703125	59.2473682887963\\
0.2939453125	59.1893498397654\\
0.29443359375	59.1308942962466\\
0.294921875	59.0720043180122\\
0.29541015625	59.0126826717174\\
0.2958984375	58.9529322285913\\
0.29638671875	58.8927559620821\\
0.296875	58.8321569454579\\
0.29736328125	58.7711383493695\\
0.2978515625	58.7097034393737\\
0.29833984375	58.6478555734249\\
0.298828125	58.5855981993341\\
0.29931640625	58.5229348522009\\
0.2998046875	58.4598691518194\\
0.30029296875	58.3964048000635\\
0.30078125	58.3325455782514\\
0.30126953125	58.2682953444958\\
0.3017578125	58.2036580310394\\
0.30224609375	58.138637641581\\
0.302734375	58.0732382485935\\
0.30322265625	58.0074639906374\\
0.3037109375	57.9413190696729\\
0.30419921875	57.8748077483713\\
0.3046875	57.8079343474311\\
0.30517578125	57.7407032428996\\
0.3056640625	57.6731188635024\\
0.30615234375	57.6051856879847\\
0.306640625	57.5369082424649\\
0.30712890625	57.4682910978042\\
0.3076171875	57.3993388669942\\
0.30810546875	57.3300562025638\\
0.30859375	57.2604477940089\\
0.30908203125	57.1905183652447\\
0.3095703125	57.1202726720859\\
0.31005859375	57.0497154997525\\
0.310546875	56.978851660407\\
0.31103515625	56.9076859907209\\
0.3115234375	56.8362233494756\\
0.31201171875	56.7644686151963\\
0.3125	56.692426683822\\
0.31298828125	56.6201024664125\\
0.3134765625	56.5475008868936\\
0.31396484375	56.4746268798411\\
0.314453125	56.4014853883058\\
0.31494140625	56.3280813616802\\
0.3154296875	56.2544197536063\\
0.31591796875	56.180505519928\\
0.31640625	56.1063436166863\\
0.31689453125	56.0319389981597\\
0.3173828125	55.9572966149495\\
0.31787109375	55.8824214121108\\
0.318359375	55.80731832733\\
0.31884765625	55.7319922891482\\
0.3193359375	55.6564482152323\\
0.31982421875	55.5806910106921\\
0.3203125	55.5047255664457\\
0.32080078125	55.4285567576321\\
0.3212890625	55.3521894420701\\
0.32177734375	55.2756284587662\\
0.322265625	55.1988786264693\\
0.32275390625	55.1219447422718\\
0.3232421875	55.0448315802596\\
0.32373046875	54.9675438902066\\
0.32421875	54.8900863963185\\
0.32470703125	54.8124637960203\\
0.3251953125	54.7346807587919\\
0.32568359375	54.656741925047\\
0.326171875	54.5786519050599\\
0.32666015625	54.5004152779343\\
0.3271484375	54.4220365906175\\
0.32763671875	54.3435203569585\\
0.328125	54.2648710568077\\
0.32861328125	54.186093135161\\
0.3291015625	54.1071910013437\\
0.32958984375	54.028169028237\\
0.330078125	53.9490315515443\\
0.33056640625	53.8697828690971\\
0.3310546875	53.7904272402011\\
0.33154296875	53.7109688850186\\
0.33203125	53.6314119839908\\
0.33251953125	53.5517606772957\\
0.3330078125	53.4720190643416\\
0.33349609375	53.3921912032972\\
0.333984375	53.3122811106558\\
0.33447265625	53.232292760832\\
0.3349609375	53.1522300857934\\
0.33544921875	53.0720969747232\\
0.3359375	52.9918972737135\\
0.33642578125	52.9116347854909\\
0.3369140625	52.8313132691703\\
0.33740234375	52.7509364400385\\
0.337890625	52.6705079693651\\
0.33837890625	52.5900314842414\\
0.3388671875	52.5095105674453\\
0.33935546875	52.4289487573326\\
0.33984375	52.3483495477521\\
0.34033203125	52.2677163879861\\
0.3408203125	52.1870526827131\\
0.34130859375	52.1063617919936\\
0.341796875	52.0256470312785\\
0.34228515625	51.9449116714365\\
0.3427734375	51.8641589388042\\
0.34326171875	51.7833920152543\\
0.34375	51.702614038283\\
0.34423828125	51.621828101116\\
0.3447265625	51.5410372528309\\
0.34521484375	51.4602444984981\\
0.345703125	51.3794527993359\\
0.34619140625	51.2986650728824\\
0.3466796875	51.2178841931815\\
0.34716796875	51.1371129909835\\
0.34765625	51.0563542539587\\
0.34814453125	50.9756107269241\\
0.3486328125	50.8948851120833\\
0.34912109375	50.8141800692758\\
0.349609375	50.7334982162405\\
0.35009765625	50.6528421288874\\
0.3505859375	50.5722143415803\\
0.35107421875	50.491617347429\\
0.3515625	50.4110535985904\\
0.35205078125	50.3305255065781\\
0.3525390625	50.2500354425798\\
0.35302734375	50.1695857377824\\
0.353515625	50.0891786837045\\
0.35400390625	50.0088165325349\\
0.3544921875	49.9285014974769\\
0.35498046875	49.8482357531003\\
0.35546875	49.7680214356959\\
0.35595703125	49.6878606436369\\
0.3564453125	49.6077554377451\\
0.35693359375	49.5277078416589\\
0.357421875	49.4477198422079\\
0.35791015625	49.367793389789\\
0.3583984375	49.2879303987464\\
0.35888671875	49.2081327477544\\
0.359375	49.1284022802023\\
0.35986328125	49.0487408045816\\
0.3603515625	48.9691500948749\\
0.36083984375	48.8896318909473\\
0.361328125	48.8101878989373\\
0.36181640625	48.7308197916504\\
0.3623046875	48.6515292089524\\
0.36279296875	48.5723177581641\\
0.36328125	48.4931870144557\\
0.36376953125	48.4141385212411\\
0.3642578125	48.3351737905725\\
0.36474609375	48.256294303535\\
0.365234375	48.1775015106396\\
0.36572265625	48.0987968322162\\
0.3662109375	48.0201816588055\\
0.36669921875	47.9416573515502\\
0.3671875	47.863225242584\\
0.36767578125	47.7848866354201\\
0.3681640625	47.7066428053378\\
0.36865234375	47.6284949997674\\
0.369140625	47.5504444386733\\
0.36962890625	47.472492314935\\
0.3701171875	47.3946397947261\\
0.37060546875	47.3168880178917\\
0.37109375	47.2392380983218\\
0.37158203125	47.1616911243244\\
0.3720703125	47.084248158995\\
0.37255859375	47.0069102405832\\
0.373046875	46.9296783828569\\
0.37353515625	46.8525535754645\\
0.3740234375	46.7755367842934\\
0.37451171875	46.6986289518253\\
0.375	46.6218309974897\\
0.37548828125	46.5451438180134\\
0.3759765625	46.468568287767\\
0.37646484375	46.3921052591088\\
0.376953125	46.3157555627252\\
0.37744140625	46.2395200079676\\
0.3779296875	46.1633993831861\\
0.37841796875	46.0873944560606\\
0.37890625	46.0115059739274\\
0.37939453125	45.9357346641035\\
0.3798828125	45.8600812342069\\
0.38037109375	45.7845463724731\\
0.380859375	45.7091307480695\\
0.38134765625	45.6338350114048\\
0.3818359375	45.5586597944362\\
0.38232421875	45.4836057109719\\
0.3828125	45.4086733569712\\
0.38330078125	45.3338633108407\\
0.3837890625	45.2591761337261\\
0.38427734375	45.1846123698028\\
0.384765625	45.1101725465601\\
0.38525390625	45.0358571750841\\
0.3857421875	44.961666750336\\
0.38623046875	44.8876017514275\\
0.38671875	44.8136626418918\\
0.38720703125	44.7398498699526\\
0.3876953125	44.6661638687881\\
0.38818359375	44.5926050567928\\
0.388671875	44.5191738378345\\
0.38916015625	44.4458706015093\\
0.3896484375	44.3726957233926\\
0.39013671875	44.2996495652859\\
0.390625	44.2267324754619\\
0.39111328125	44.1539447889046\\
0.3916015625	44.0812868275474\\
0.39208984375	44.0087589005069\\
0.392578125	43.9363613043136\\
0.39306640625	43.8640943231401\\
0.3935546875	43.791958229025\\
0.39404296875	43.7199532820945\\
0.39453125	43.64807973078\\
0.39501953125	43.5763378120331\\
0.3955078125	43.5047277515371\\
0.39599609375	43.4332497639162\\
0.396484375	43.3619040529401\\
0.39697265625	43.2906908117271\\
0.3974609375	43.2196102229433\\
0.39794921875	43.1486624589992\\
0.3984375	43.0778476822429\\
0.39892578125	43.0071660451506\\
0.3994140625	42.9366176905151\\
0.39990234375	42.8662027516289\\
0.400390625	42.795921352468\\
0.40087890625	42.7257736078695\\
0.4013671875	42.6557596237091\\
0.40185546875	42.5858794970736\\
0.40234375	42.5161333164327\\
0.40283203125	42.4465211618066\\
0.4033203125	42.3770431049318\\
0.40380859375	42.3076992094239\\
0.404296875	42.238489530938\\
0.40478515625	42.1694141173263\\
0.4052734375	42.1004730087934\\
0.40576171875	42.0316662380493\\
0.40625	41.9629938304592\\
0.40673828125	41.8944558041915\\
0.4072265625	41.8260521703636\\
0.40771484375	41.7577829331843\\
0.408203125	41.6896480900949\\
0.40869140625	41.6216476319079\\
0.4091796875	41.5537815429423\\
0.40966796875	41.4860498011585\\
0.41015625	41.4184523782888\\
0.41064453125	41.3509892399682\\
0.4111328125	41.2836603458604\\
0.41162109375	41.2164656497841\\
0.412109375	41.1494050998353\\
0.41259765625	41.0824786385086\\
0.4130859375	41.0156862028165\\
0.41357421875	40.9490277244058\\
0.4140625	40.8825031296728\\
0.41455078125	40.8161123398767\\
0.4150390625	40.7498552712501\\
0.41552734375	40.6837318351087\\
0.416015625	40.6177419379583\\
0.41650390625	40.5518854816005\\
0.4169921875	40.4861623632364\\
0.41748046875	40.4205724755684\\
0.41796875	40.3551157069003\\
0.41845703125	40.2897919412358\\
0.4189453125	40.2246010583751\\
0.41943359375	40.1595429340101\\
0.419921875	40.0946174398176\\
0.42041015625	40.0298244435509\\
0.4208984375	39.96516380913\\
0.42138671875	39.9006353967304\\
0.421875	39.8362390628698\\
0.42236328125	39.7719746604934\\
0.4228515625	39.7078420390582\\
0.42333984375	39.6438410446151\\
0.423828125	39.5799715198902\\
0.42431640625	39.5162333043639\\
0.4248046875	39.4526262343498\\
0.42529296875	39.3891501430703\\
0.42578125	39.3258048607331\\
0.42626953125	39.2625902146042\\
0.4267578125	39.1995060290815\\
0.42724609375	39.1365521257654\\
0.427734375	39.0737283235295\\
0.42822265625	39.0110344385887\\
0.4287109375	38.9484702845676\\
0.42919921875	38.8860356725662\\
0.4296875	38.8237304112251\\
0.43017578125	38.7615543067902\\
0.4306640625	38.6995071631742\\
0.43115234375	38.6375887820193\\
0.431640625	38.5757989627575\\
0.43212890625	38.5141375026694\\
0.4326171875	38.4526041969433\\
0.43310546875	38.3911988387319\\
0.43359375	38.3299212192089\\
0.43408203125	38.2687711276237\\
0.4345703125	38.2077483513557\\
0.43505859375	38.1468526759675\\
0.435546875	38.0860838852571\\
0.43603515625	38.0254417613088\\
0.4365234375	37.9649260845435\\
0.43701171875	37.904536633768\\
0.4375	37.8442731862234\\
0.43798828125	37.7841355176325\\
0.4384765625	37.7241234022462\\
0.43896484375	37.6642366128892\\
0.439453125	37.6044749210053\\
0.43994140625	37.5448380967004\\
0.4404296875	37.4853259087866\\
0.44091796875	37.4259381248238\\
0.44140625	37.3666745111615\\
0.44189453125	37.3075348329798\\
0.4423828125	37.2485188543287\\
0.44287109375	37.1896263381676\\
0.443359375	37.1308570464039\\
0.44384765625	37.0722107399303\\
0.4443359375	37.0136871786621\\
0.44482421875	36.9552861215732\\
0.4453125	36.8970073267318\\
0.44580078125	36.8388505513351\\
0.4462890625	36.7808155517438\\
0.44677734375	36.722902083515\\
0.447265625	36.6651099014358\\
0.44775390625	36.607438759555\\
0.4482421875	36.5498884112149\\
0.44873046875	36.4924586090821\\
0.44921875	36.4351491051782\\
0.44970703125	36.3779596509091\\
0.4501953125	36.3208899970946\\
0.45068359375	36.2639398939968\\
0.451171875	36.207109091348\\
0.45166015625	36.1503973383785\\
0.4521484375	36.0938043838431\\
0.45263671875	36.0373299760479\\
0.453125	35.980973862876\\
0.45361328125	35.9247357918127\\
0.4541015625	35.8686155099706\\
0.45458984375	35.8126127641136\\
0.455078125	35.756727300681\\
0.45556640625	35.7009588658107\\
0.4560546875	35.6453072053623\\
0.45654296875	35.5897720649391\\
0.45703125	35.5343531899106\\
0.45751953125	35.4790503254334\\
0.4580078125	35.4238632164727\\
0.45849609375	35.3687916078229\\
0.458984375	35.3138352441274\\
0.45947265625	35.2589938698989\\
0.4599609375	35.2042672295385\\
0.46044921875	35.1496550673546\\
0.4609375	35.0951571275817\\
0.46142578125	35.0407731543983\\
0.4619140625	34.986502891945\\
0.46240234375	34.9323460843415\\
0.462890625	34.8783024757042\\
0.46337890625	34.8243718101625\\
0.4638671875	34.7705538318751\\
0.46435546875	34.7168482850463\\
0.46484375	34.6632549139413\\
0.46533203125	34.6097734629017\\
0.4658203125	34.5564036763605\\
0.46630859375	34.5031452988566\\
0.466796875	34.449998075049\\
0.46728515625	34.3969617497314\\
0.4677734375	34.3440360678452\\
0.46826171875	34.2912207744934\\
0.46875	34.2385156149533\\
0.46923828125	34.1859203346898\\
0.4697265625	34.1334346793675\\
0.47021484375	34.0810583948631\\
0.470703125	34.0287912272775\\
0.47119140625	33.9766329229472\\
0.4716796875	33.9245832284562\\
0.47216796875	33.8726418906467\\
0.47265625	33.8208086566302\\
0.47314453125	33.7690832737983\\
0.4736328125	33.7174654898329\\
0.47412109375	33.6659550527167\\
0.474609375	33.6145517107425\\
0.47509765625	33.5632552125239\\
0.4755859375	33.5120653070037\\
0.47607421875	33.460981743464\\
0.4765625	33.4100042715347\\
0.47705078125	33.3591326412027\\
0.4775390625	33.3083666028202\\
0.47802734375	33.2577059071131\\
0.478515625	33.2071503051897\\
0.47900390625	33.1566995485478\\
0.4794921875	33.1063533890832\\
0.47998046875	33.0561115790973\\
0.48046875	33.0059738713039\\
0.48095703125	32.955940018837\\
0.4814453125	32.9060097752577\\
0.48193359375	32.8561828945608\\
0.482421875	32.8064591311819\\
0.48291015625	32.7568382400036\\
0.4833984375	32.7073199763619\\
0.48388671875	32.6579040960523\\
0.484375	32.6085903553362\\
0.48486328125	32.5593785109462\\
0.4853515625	32.5102683200922\\
0.48583984375	32.4612595404666\\
0.486328125	32.4123519302499\\
0.48681640625	32.3635452481162\\
0.4873046875	32.3148392532375\\
0.48779296875	32.2662337052895\\
0.48828125	32.2177283644558\\
0.48876953125	32.169322991433\\
0.4892578125	32.121017347435\\
0.48974609375	32.0728111941973\\
0.490234375	32.0247042939816\\
0.49072265625	31.9766964095794\\
0.4912109375	31.9287873043169\\
0.49169921875	31.8809767420578\\
0.4921875	31.8332644872081\\
0.49267578125	31.7856503047189\\
0.4931640625	31.7381339600905\\
0.49365234375	31.6907152193758\\
0.494140625	31.6433938491832\\
0.49462890625	31.5961696166802\\
0.4951171875	31.5490422895964\\
0.49560546875	31.5020116362262\\
0.49609375	31.4550774254324\\
0.49658203125	31.4082394266481\\
0.4970703125	31.3614974098802\\
0.49755859375	31.3148511457112\\
0.498046875	31.2683004053025\\
0.49853515625	31.2218449603962\\
0.4990234375	31.1754845833177\\
0.49951171875	31.1292190469777\\
0.5	31.0830481248748\\
0.50048828125	31.036971591097\\
0.5009765625	30.9909892203238\\
0.50146484375	30.9451007878284\\
0.501953125	30.8993060694792\\
0.50244140625	30.8536048417414\\
0.5029296875	30.8079968816791\\
0.50341796875	30.7624819669564\\
0.50390625	30.7170598758393\\
0.50439453125	30.6717303871965\\
0.5048828125	30.6264932805016\\
0.50537109375	30.5813483358335\\
0.505859375	30.5362953338783\\
0.50634765625	30.49133405593\\
0.5068359375	30.4464642838918\\
0.50732421875	30.4016858002769\\
0.5078125	30.3569983882096\\
0.50830078125	30.3124018314263\\
0.5087890625	30.2678959142758\\
0.50927734375	30.2234804217209\\
0.509765625	30.1791551393381\\
0.51025390625	30.134919853319\\
0.5107421875	30.0907743504707\\
0.51123046875	30.0467184182161\\
0.51171875	30.0027518445945\\
0.51220703125	29.9588744182623\\
0.5126953125	29.9150859284928\\
0.51318359375	29.871386165177\\
0.513671875	29.8277749188238\\
0.51416015625	29.7842519805599\\
0.5146484375	29.7408171421304\\
0.51513671875	29.6974701958986\\
0.515625	29.6542109348464\\
0.51611328125	29.6110391525738\\
0.5166015625	29.5679546432996\\
0.51708984375	29.5249572018608\\
0.517578125	29.4820466237127\\
0.51806640625	29.4392227049288\\
0.5185546875	29.3964852422005\\
0.51904296875	29.3538340328372\\
0.51953125	29.3112688747655\\
0.52001953125	29.2687895665294\\
0.5205078125	29.22639590729\\
0.52099609375	29.1840876968246\\
0.521484375	29.1418647355269\\
0.52197265625	29.0997268244061\\
0.5224609375	29.057673765087\\
0.52294921875	29.0157053598088\\
0.5234375	28.9738214114253\\
0.52392578125	28.9320217234036\\
0.5244140625	28.8903060998241\\
0.52490234375	28.8486743453797\\
0.525390625	28.807126265375\\
0.52587890625	28.7656616657258\\
0.5263671875	28.7242803529584\\
0.52685546875	28.6829821342086\\
0.52734375	28.6417668172213\\
0.52783203125	28.6006342103496\\
0.5283203125	28.5595841225538\\
0.52880859375	28.5186163634009\\
0.529296875	28.4777307430632\\
0.52978515625	28.4369270723182\\
0.5302734375	28.3962051625469\\
0.53076171875	28.3555648257336\\
0.53125	28.3150058744643\\
0.53173828125	28.2745281219259\\
0.5322265625	28.2341313819057\\
0.53271484375	28.1938154687898\\
0.533203125	28.1535801975623\\
0.53369140625	28.113425383804\\
0.5341796875	28.073350843692\\
0.53466796875	28.0333563939977\\
0.53515625	27.9934418520865\\
0.53564453125	27.9536070359162\\
0.5361328125	27.9138517640359\\
0.53662109375	27.874175855585\\
0.537109375	27.8345791302921\\
0.53759765625	27.7950614084734\\
0.5380859375	27.755622511032\\
0.53857421875	27.7162622594562\\
0.5390625	27.6769804758188\\
0.53955078125	27.6377769827753\\
0.5400390625	27.598651603563\\
0.54052734375	27.5596041619997\\
0.541015625	27.5206344824823\\
0.54150390625	27.4817423899854\\
0.5419921875	27.4429277100603\\
0.54248046875	27.4041902688334\\
0.54296875	27.365529893005\\
0.54345703125	27.3269464098481\\
0.5439453125	27.2884396472064\\
0.54443359375	27.2500094334939\\
0.544921875	27.2116555976928\\
0.54541015625	27.1733779693522\\
0.5458984375	27.1351763785872\\
0.54638671875	27.0970506560767\\
0.546875	27.0590006330627\\
0.54736328125	27.0210261413487\\
0.5478515625	26.9831270132978\\
0.54833984375	26.9453030818318\\
0.548828125	26.9075541804297\\
0.54931640625	26.869880143126\\
0.5498046875	26.8322808045094\\
0.55029296875	26.7947559997213\\
0.55078125	26.7573055644542\\
0.55126953125	26.7199293349506\\
0.5517578125	26.682627148001\\
0.55224609375	26.6453988409427\\
0.552734375	26.6082442516585\\
0.55322265625	26.5711632185746\\
0.5537109375	26.5341555806597\\
0.55419921875	26.4972211774232\\
0.5546875	26.4603598489137\\
0.55517578125	26.4235714357174\\
0.5556640625	26.3868557789567\\
0.55615234375	26.3502127202888\\
0.556640625	26.3136421019038\\
0.55712890625	26.2771437665234\\
0.5576171875	26.2407175573993\\
0.55810546875	26.2043633183118\\
0.55859375	26.168080893568\\
0.55908203125	26.1318701280004\\
0.5595703125	26.0957308669653\\
0.56005859375	26.0596629563412\\
0.560546875	26.0236662425275\\
0.56103515625	25.9877405724425\\
0.5615234375	25.9518857935221\\
0.56201171875	25.9161017537183\\
0.5625	25.8803883014976\\
0.56298828125	25.844745285839\\
0.5634765625	25.8091725562333\\
0.56396484375	25.7736699626807\\
0.564453125	25.7382373556895\\
0.56494140625	25.7028745862747\\
0.5654296875	25.6675815059564\\
0.56591796875	25.6323579667579\\
0.56640625	25.5972038212045\\
0.56689453125	25.5621189223216\\
0.5673828125	25.5271031236334\\
0.56787109375	25.4921562791613\\
0.568359375	25.4572782434221\\
0.56884765625	25.4224688714266\\
0.5693359375	25.3877280186778\\
0.56982421875	25.3530555411698\\
0.5703125	25.3184512953857\\
0.57080078125	25.2839151382964\\
0.5712890625	25.2494469273586\\
0.57177734375	25.2150465205139\\
0.572265625	25.1807137761865\\
0.57275390625	25.1464485532819\\
0.5732421875	25.1122507111857\\
0.57373046875	25.0781201097613\\
0.57421875	25.0440566093491\\
0.57470703125	25.0100600707643\\
0.5751953125	24.9761303552956\\
0.57568359375	24.9422673247037\\
0.576171875	24.9084708412198\\
0.57666015625	24.8747407675436\\
0.5771484375	24.8410769668422\\
0.57763671875	24.8074793027483\\
0.578125	24.7739476393588\\
0.57861328125	24.7404818412331\\
0.5791015625	24.7070817733916\\
0.57958984375	24.6737473013143\\
0.580078125	24.6404782909387\\
0.58056640625	24.6072746086592\\
0.5810546875	24.5741361213246\\
0.58154296875	24.5410626962371\\
0.58203125	24.5080542011507\\
0.58251953125	24.4751105042694\\
0.5830078125	24.442231474246\\
0.58349609375	24.4094169801804\\
0.583984375	24.376666891618\\
0.58447265625	24.3439810785485\\
0.5849609375	24.3113594114038\\
0.58544921875	24.2788017610571\\
0.5859375	24.2463079988209\\
0.58642578125	24.2138779964458\\
0.5869140625	24.1815116261189\\
0.58740234375	24.149208760462\\
0.587890625	24.1169692725306\\
0.58837890625	24.0847930358122\\
0.5888671875	24.0526799242245\\
0.58935546875	24.0206298121144\\
0.58984375	23.988642574256\\
0.59033203125	23.9567180858496\\
0.5908203125	23.9248562225198\\
0.59130859375	23.8930568603143\\
0.591796875	23.8613198757023\\
0.59228515625	23.829645145573\\
0.5927734375	23.7980325472343\\
0.59326171875	23.7664819584109\\
0.59375	23.7349932572434\\
0.59423828125	23.7035663222865\\
0.5947265625	23.6722010325075\\
0.59521484375	23.6408972672851\\
0.595703125	23.6096549064077\\
0.59619140625	23.5784738300721\\
0.5966796875	23.547353918882\\
0.59716796875	23.5162950538465\\
0.59765625	23.4852971163788\\
0.59814453125	23.4543599882947\\
0.5986328125	23.4234835518112\\
0.59912109375	23.392667689545\\
0.599609375	23.361912284511\\
0.60009765625	23.3312172201213\\
0.6005859375	23.3005823801832\\
0.60107421875	23.2700076488984\\
0.6015625	23.2394929108609\\
0.60205078125	23.2090380510563\\
0.6025390625	23.1786429548599\\
0.60302734375	23.1483075080356\\
0.603515625	23.1180315967343\\
0.60400390625	23.0878151074927\\
0.6044921875	23.0576579272319\\
0.60498046875	23.0275599432557\\
0.60546875	22.9975210432499\\
0.60595703125	22.9675411152801\\
0.6064453125	22.937620047791\\
0.60693359375	22.9077577296048\\
0.607421875	22.8779540499197\\
0.60791015625	22.8482088983089\\
0.6083984375	22.8185221647188\\
0.60888671875	22.7888937394681\\
0.609375	22.7593235132462\\
0.60986328125	22.7298113771119\\
0.6103515625	22.7003572224922\\
0.61083984375	22.6709609411807\\
0.611328125	22.6416224253366\\
0.61181640625	22.6123415674832\\
0.6123046875	22.5831182605067\\
0.61279296875	22.5539523976546\\
0.61328125	22.5248438725349\\
0.61376953125	22.4957925791142\\
0.6142578125	22.466798411717\\
0.61474609375	22.437861265024\\
0.615234375	22.4089810340709\\
0.61572265625	22.3801576142472\\
0.6162109375	22.351390901295\\
0.61669921875	22.3226807913074\\
0.6171875	22.2940271807275\\
0.61767578125	22.2654299663472\\
0.6181640625	22.2368890453057\\
0.61865234375	22.2084043150883\\
0.619140625	22.1799756735253\\
0.61962890625	22.1516030187907\\
0.6201171875	22.1232862494007\\
0.62060546875	22.095025264213\\
0.62109375	22.0668199624249\\
0.62158203125	22.0386702435727\\
0.6220703125	22.0105760075299\\
0.62255859375	21.9825371545065\\
0.623046875	21.9545535850475\\
0.62353515625	21.9266252000317\\
0.6240234375	21.8987519006703\\
0.62451171875	21.8709335885064\\
0.625	21.843170165413\\
0.62548828125	21.8154615335921\\
0.6259765625	21.7878075955737\\
0.62646484375	21.7602082542144\\
0.626953125	21.732663412696\\
0.62744140625	21.7051729745252\\
0.6279296875	21.6777368435312\\
0.62841796875	21.6503549238653\\
0.62890625	21.6230271199999\\
0.62939453125	21.5957533367267\\
0.6298828125	21.5685334791559\\
0.63037109375	21.5413674527152\\
0.630859375	21.5142551631481\\
0.63134765625	21.4871965165135\\
0.6318359375	21.4601914191841\\
0.63232421875	21.4332397778451\\
0.6328125	21.4063414994936\\
0.63330078125	21.3794964914371\\
0.6337890625	21.3527046612923\\
0.63427734375	21.3259659169845\\
0.634765625	21.2992801667458\\
0.63525390625	21.2726473191146\\
0.6357421875	21.2460672829339\\
0.63623046875	21.2195399673509\\
0.63671875	21.1930652818151\\
0.63720703125	21.1666431360781\\
0.6376953125	21.1402734401917\\
0.63818359375	21.1139561045072\\
0.638671875	21.0876910396745\\
0.63916015625	21.0614781566405\\
0.6396484375	21.0353173666485\\
0.64013671875	21.009208581237\\
0.640625	20.9831517122386\\
0.64111328125	20.9571466717788\\
0.6416015625	20.9311933722752\\
0.64208984375	20.9052917264363\\
0.642578125	20.8794416472605\\
0.64306640625	20.8536430480351\\
0.6435546875	20.8278958423351\\
0.64404296875	20.8021999440224\\
0.64453125	20.7765552672445\\
0.64501953125	20.7509617264337\\
0.6455078125	20.725419236306\\
0.64599609375	20.6999277118601\\
0.646484375	20.6744870683763\\
0.64697265625	20.6490972214155\\
0.6474609375	20.6237580868185\\
0.64794921875	20.5984695807044\\
0.6484375	20.5732316194702\\
0.64892578125	20.5480441197893\\
0.6494140625	20.5229069986112\\
0.64990234375	20.4978201731596\\
0.650390625	20.4727835609321\\
0.65087890625	20.447797079699\\
0.6513671875	20.4228606475024\\
0.65185546875	20.3979741826551\\
0.65234375	20.3731376037396\\
0.65283203125	20.3483508296075\\
0.6533203125	20.323613779378\\
0.65380859375	20.2989263724374\\
0.654296875	20.2742885284379\\
0.65478515625	20.2497001672968\\
0.6552734375	20.2251612091954\\
0.65576171875	20.2006715745781\\
0.65625	20.1762311841516\\
0.65673828125	20.1518399588839\\
0.6572265625	20.1274978200033\\
0.65771484375	20.1032046889975\\
0.658203125	20.0789604876127\\
0.65869140625	20.0547651378527\\
0.6591796875	20.0306185619779\\
0.65966796875	20.0065206825046\\
0.66015625	19.9824714222039\\
0.66064453125	19.9584707041009\\
0.6611328125	19.9345184514734\\
0.66162109375	19.9106145878519\\
0.662109375	19.8867590370179\\
0.66259765625	19.8629517230032\\
0.6630859375	19.8391925700892\\
0.66357421875	19.8154815028059\\
0.6640625	19.791818445931\\
0.66455078125	19.7682033244892\\
0.6650390625	19.744636063751\\
0.66552734375	19.7211165892322\\
0.666015625	19.6976448266928\\
0.66650390625	19.6742207021362\\
0.6669921875	19.6508441418083\\
0.66748046875	19.6275150721968\\
0.66796875	19.6042334200302\\
0.66845703125	19.580999112277\\
0.6689453125	19.557812076145\\
0.66943359375	19.5346722390802\\
0.669921875	19.5115795287659\\
0.67041015625	19.4885338731226\\
0.6708984375	19.4655352003061\\
0.67138671875	19.4425834387076\\
0.671875	19.4196785169522\\
0.67236328125	19.3968203638985\\
0.6728515625	19.3740089086378\\
0.67333984375	19.3512440804929\\
0.673828125	19.3285258090177\\
0.67431640625	19.3058540239963\\
0.6748046875	19.2832286554419\\
0.67529296875	19.2606496335965\\
0.67578125	19.2381168889298\\
0.67626953125	19.2156303521383\\
0.6767578125	19.1931899541448\\
0.67724609375	19.1707956260975\\
0.677734375	19.1484472993693\\
0.67822265625	19.1261449055566\\
0.6787109375	19.1038883764793\\
0.67919921875	19.0816776441792\\
0.6796875	19.05951264092\\
0.68017578125	19.0373932991857\\
0.6806640625	19.0153195516808\\
0.68115234375	18.9932913313286\\
0.681640625	18.9713085712711\\
0.68212890625	18.9493712048681\\
0.6826171875	18.9274791656963\\
0.68310546875	18.9056323875484\\
0.68359375	18.8838308044332\\
0.68408203125	18.8620743505735\\
0.6845703125	18.8403629604068\\
0.68505859375	18.8186965685835\\
0.685546875	18.7970751099667\\
0.68603515625	18.7754985196312\\
0.6865234375	18.7539667328631\\
0.68701171875	18.7324796851587\\
0.6875	18.7110373122242\\
0.68798828125	18.6896395499745\\
0.6884765625	18.668286334533\\
0.68896484375	18.6469776022304\\
0.689453125	18.6257132896045\\
0.68994140625	18.6044933333989\\
0.6904296875	18.5833176705629\\
0.69091796875	18.5621862382505\\
0.69140625	18.5410989738196\\
0.69189453125	18.5200558148315\\
0.6923828125	18.4990566990504\\
0.69287109375	18.4781015644421\\
0.693359375	18.4571903491739\\
0.69384765625	18.4363229916138\\
0.6943359375	18.4154994303295\\
0.69482421875	18.3947196040882\\
0.6953125	18.3739834518556\\
0.69580078125	18.3532909127954\\
0.6962890625	18.3326419262683\\
0.69677734375	18.312036431832\\
0.697265625	18.2914743692398\\
0.69775390625	18.2709556784405\\
0.6982421875	18.2504802995775\\
0.69873046875	18.2300481729882\\
0.69921875	18.2096592392031\\
0.69970703125	18.1893134389457\\
0.7001953125	18.1690107131315\\
0.70068359375	18.1487510028672\\
0.701171875	18.1285342494505\\
0.70166015625	18.1083603943691\\
0.7021484375	18.0882293793002\\
0.70263671875	18.06814114611\\
0.703125	18.0480956368528\\
0.70361328125	18.0280927937707\\
0.7041015625	18.0081325592926\\
0.70458984375	17.9882148760339\\
0.705078125	17.9683396867958\\
0.70556640625	17.9485069345645\\
0.7060546875	17.928716562511\\
0.70654296875	17.90896851399\\
0.70703125	17.8892627325397\\
0.70751953125	17.8695991618808\\
0.7080078125	17.8499777459165\\
0.70849609375	17.8303984287311\\
0.708984375	17.8108611545903\\
0.70947265625	17.7913658679396\\
0.7099609375	17.7719125134048\\
0.71044921875	17.7525010357905\\
0.7109375	17.73313138008\\
0.71142578125	17.7138034914346\\
0.7119140625	17.694517315193\\
0.71240234375	17.6752727968708\\
0.712890625	17.6560698821598\\
0.71337890625	17.6369085169274\\
0.7138671875	17.6177886472163\\
0.71435546875	17.5987102192438\\
0.71484375	17.5796731794008\\
0.71533203125	17.5606774742521\\
0.7158203125	17.5417230505352\\
0.71630859375	17.5228098551596\\
0.716796875	17.5039378352071\\
0.71728515625	17.4851069379302\\
0.7177734375	17.4663171107524\\
0.71826171875	17.447568301267\\
0.71875	17.4288604572371\\
0.71923828125	17.4101935265946\\
0.7197265625	17.3915674574401\\
0.72021484375	17.3729821980419\\
0.720703125	17.3544376968358\\
0.72119140625	17.3359339024243\\
0.7216796875	17.3174707635765\\
0.72216796875	17.2990482292272\\
0.72265625	17.2806662484761\\
0.72314453125	17.2623247705881\\
0.7236328125	17.2440237449921\\
0.72412109375	17.2257631212808\\
0.724609375	17.2075428492099\\
0.72509765625	17.1893628786978\\
0.7255859375	17.171223159825\\
0.72607421875	17.153123642834\\
0.7265625	17.1350642781278\\
0.72705078125	17.1170450162704\\
0.7275390625	17.099065807986\\
0.72802734375	17.0811266041579\\
0.728515625	17.063227355829\\
0.72900390625	17.0453680142006\\
0.7294921875	17.027548530632\\
0.72998046875	17.0097688566402\\
0.73046875	16.9920289438994\\
0.73095703125	16.9743287442401\\
0.7314453125	16.9566682096494\\
0.73193359375	16.9390472922696\\
0.732421875	16.9214659443984\\
0.73291015625	16.9039241184881\\
0.7333984375	16.8864217671452\\
0.73388671875	16.8689588431298\\
0.734375	16.8515352993554\\
0.73486328125	16.8341510888883\\
0.7353515625	16.8168061649469\\
0.73583984375	16.7995004809015\\
0.736328125	16.7822339902737\\
0.73681640625	16.7650066467361\\
0.7373046875	16.7478184041117\\
0.73779296875	16.7306692163733\\
0.73828125	16.7135590376433\\
0.73876953125	16.6964878221932\\
0.7392578125	16.679455524443\\
0.73974609375	16.6624620989607\\
0.740234375	16.6455075004622\\
0.74072265625	16.6285916838105\\
0.7412109375	16.6117146040154\\
0.74169921875	16.594876216233\\
0.7421875	16.5780764757653\\
0.74267578125	16.5613153380596\\
0.7431640625	16.5445927587084\\
0.74365234375	16.5279086934487\\
0.744140625	16.5112630981615\\
0.74462890625	16.4946559288715\\
0.7451171875	16.4780871417469\\
0.74560546875	16.4615566930983\\
0.74609375	16.4450645393789\\
0.74658203125	16.428610637184\\
0.7470703125	16.4121949432502\\
0.74755859375	16.3958174144552\\
0.748046875	16.3794780078175\\
0.74853515625	16.3631766804959\\
0.7490234375	16.3469133897888\\
0.74951171875	16.3306880931342\\
0.75	16.3145007481091\\
0.75048828125	16.2983513124291\\
0.7509765625	16.282239743948\\
0.75146484375	16.2661660006571\\
0.751953125	16.2501300406854\\
0.75244140625	16.2341318222988\\
0.7529296875	16.2181713038995\\
0.75341796875	16.2022484440261\\
0.75390625	16.1863632013529\\
0.75439453125	16.1705155346894\\
0.7548828125	16.1547054029803\\
0.75537109375	16.1389327653044\\
0.755859375	16.1231975808752\\
0.75634765625	16.1074998090396\\
0.7568359375	16.0918394092779\\
0.75732421875	16.0762163412036\\
0.7578125	16.0606305645625\\
0.75830078125	16.0450820392327\\
0.7587890625	16.0295707252242\\
0.75927734375	16.0140965826783\\
0.759765625	15.9986595718676\\
0.76025390625	15.983259653195\\
0.7607421875	15.967896787194\\
0.76123046875	15.9525709345279\\
0.76171875	15.9372820559895\\
0.76220703125	15.9220301125008\\
0.7626953125	15.9068150651127\\
0.76318359375	15.8916368750043\\
0.763671875	15.876495503483\\
0.76416015625	15.8613909119837\\
0.7646484375	15.8463230620688\\
0.76513671875	15.8312919154274\\
0.765625	15.8162974338755\\
0.76611328125	15.8013395793551\\
0.7666015625	15.7864183139342\\
0.76708984375	15.7715335998062\\
0.767578125	15.7566853992899\\
0.76806640625	15.7418736748286\\
0.7685546875	15.7270983889902\\
0.76904296875	15.7123595044668\\
0.76953125	15.6976569840741\\
0.77001953125	15.6829907907512\\
0.7705078125	15.6683608875603\\
0.77099609375	15.6537672376864\\
0.771484375	15.6392098044368\\
0.77197265625	15.6246885512407\\
0.7724609375	15.6102034416491\\
0.77294921875	15.5957544393343\\
0.7734375	15.5813415080897\\
0.77392578125	15.5669646118291\\
0.7744140625	15.5526237145871\\
0.77490234375	15.5383187805177\\
0.775390625	15.5240497738951\\
0.77587890625	15.5098166591125\\
0.7763671875	15.4956194006821\\
0.77685546875	15.481457963235\\
0.77734375	15.4673323115205\\
0.77783203125	15.4532424104059\\
0.7783203125	15.4391882248762\\
0.77880859375	15.4251697200338\\
0.779296875	15.4111868610982\\
0.77978515625	15.3972396134055\\
0.7802734375	15.3833279424084\\
0.78076171875	15.3694518136755\\
0.78125	15.3556111928912\\
0.78173828125	15.3418060458556\\
0.7822265625	15.3280363384836\\
0.78271484375	15.3143020368051\\
0.783203125	15.3006031069647\\
0.78369140625	15.286939515221\\
0.7841796875	15.2733112279465\\
0.78466796875	15.2597182116274\\
0.78515625	15.2461604328632\\
0.78564453125	15.2326378583663\\
0.7861328125	15.2191504549619\\
0.78662109375	15.2056981895875\\
0.787109375	15.1922810292928\\
0.78759765625	15.1788989412392\\
0.7880859375	15.1655518926996\\
0.78857421875	15.1522398510582\\
0.7890625	15.1389627838099\\
0.78955078125	15.1257206585603\\
0.7900390625	15.1125134430256\\
0.79052734375	15.0993411050315\\
0.791015625	15.0862036125138\\
0.79150390625	15.0731009335177\\
0.7919921875	15.0600330361975\\
0.79248046875	15.0469998888164\\
0.79296875	15.034001459746\\
0.79345703125	15.0210377174667\\
0.7939453125	15.0081086305664\\
0.79443359375	14.9952141677409\\
0.794921875	14.9823542977937\\
0.79541015625	14.9695289896351\\
0.7958984375	14.9567382122826\\
0.79638671875	14.9439819348603\\
0.796875	14.9312601265984\\
0.79736328125	14.9185727568335\\
0.7978515625	14.9059197950078\\
0.79833984375	14.8933012106691\\
0.798828125	14.8807169734705\\
0.79931640625	14.8681670531701\\
0.7998046875	14.8556514196306\\
0.80029296875	14.8431700428193\\
0.80078125	14.8307228928076\\
0.80126953125	14.818309939771\\
0.8017578125	14.8059311539884\\
0.80224609375	14.7935865058423\\
0.802734375	14.7812759658184\\
0.80322265625	14.768999504505\\
0.8037109375	14.7567570925933\\
0.80419921875	14.7445487008767\\
0.8046875	14.7323743002509\\
0.80517578125	14.7202338617133\\
0.8056640625	14.7081273563628\\
0.80615234375	14.6960547553999\\
0.806640625	14.6840160301261\\
0.80712890625	14.6720111519436\\
0.8076171875	14.6600400923555\\
0.80810546875	14.6481028229648\\
0.80859375	14.6361993154751\\
0.80908203125	14.6243295416895\\
0.8095703125	14.6124934735108\\
0.81005859375	14.6006910829412\\
0.810546875	14.588922342082\\
0.81103515625	14.5771872231333\\
0.8115234375	14.565485698394\\
0.81201171875	14.5538177402611\\
0.8125	14.5421833212302\\
0.81298828125	14.5305824138944\\
0.8134765625	14.5190149909447\\
0.81396484375	14.5074810251694\\
0.814453125	14.4959804894542\\
0.81494140625	14.4845133567816\\
0.8154296875	14.4730796002309\\
0.81591796875	14.4616791929779\\
0.81640625	14.4503121082947\\
0.81689453125	14.4389783195494\\
0.8173828125	14.427677800206\\
0.81787109375	14.4164105238241\\
0.818359375	14.4051764640585\\
0.81884765625	14.3939755946592\\
0.8193359375	14.3828078894714\\
0.81982421875	14.3716733224345\\
0.8203125	14.3605718675827\\
0.82080078125	14.3495034990444\\
0.8212890625	14.338468191042\\
0.82177734375	14.3274659178916\\
0.822265625	14.316496654003\\
0.82275390625	14.3055603738794\\
0.8232421875	14.294657052117\\
0.82373046875	14.2837866634051\\
0.82421875	14.2729491825256\\
0.82470703125	14.262144584353\\
0.8251953125	14.251372843854\\
0.82568359375	14.2406339360874\\
0.826171875	14.2299278362039\\
0.82666015625	14.2192545194459\\
0.8271484375	14.2086139611471\\
0.82763671875	14.1980061367326\\
0.828125	14.1874310217184\\
0.82861328125	14.1768885917115\\
0.8291015625	14.1663788224093\\
0.82958984375	14.1559016895999\\
0.830078125	14.1454571691612\\
0.83056640625	14.1350452370616\\
0.8310546875	14.1246658693589\\
0.83154296875	14.1143190422006\\
0.83203125	14.1040047318239\\
0.83251953125	14.0937229145548\\
0.8330078125	14.0834735668086\\
0.83349609375	14.0732566650892\\
0.833984375	14.0630721859893\\
0.83447265625	14.0529201061898\\
0.8349609375	14.0428004024601\\
0.83544921875	14.0327130516575\\
0.8359375	14.022658030727\\
0.83642578125	14.0126353167014\\
0.8369140625	14.0026448867009\\
0.83740234375	13.9926867179332\\
0.837890625	13.9827607876926\\
0.83837890625	13.9728670733607\\
0.8388671875	13.9630055524056\\
0.83935546875	13.9531762023819\\
0.83984375	13.9433790009307\\
0.84033203125	13.933613925779\\
0.8408203125	13.92388095474\\
0.84130859375	13.9141800657125\\
0.841796875	13.904511236681\\
0.84228515625	13.8948744457152\\
0.8427734375	13.8852696709704\\
0.84326171875	13.8756968906865\\
0.84375	13.8661560831886\\
0.84423828125	13.8566472268863\\
0.8447265625	13.8471703002739\\
0.84521484375	13.8377252819299\\
0.845703125	13.8283121505169\\
0.84619140625	13.8189308847817\\
0.8466796875	13.8095814635546\\
0.84716796875	13.8002638657499\\
0.84765625	13.7909780703651\\
0.84814453125	13.781724056481\\
0.8486328125	13.7725018032616\\
0.84912109375	13.7633112899539\\
0.849609375	13.7541524958876\\
0.85009765625	13.7450254004751\\
0.8505859375	13.7359299832109\\
0.85107421875	13.7268662236724\\
0.8515625	13.7178341015184\\
0.85205078125	13.7088335964903\\
0.8525390625	13.6998646884108\\
0.85302734375	13.6909273571845\\
0.853515625	13.6820215827972\\
0.85400390625	13.6731473453162\\
0.8544921875	13.66430462489\\
0.85498046875	13.6554934017477\\
0.85546875	13.6467136561996\\
0.85595703125	13.6379653686364\\
0.8564453125	13.6292485195294\\
0.85693359375	13.6205630894302\\
0.857421875	13.6119090589706\\
0.85791015625	13.6032864088624\\
0.8583984375	13.5946951198972\\
0.85888671875	13.5861351729463\\
0.859375	13.5776065489608\\
0.85986328125	13.5691092289709\\
0.8603515625	13.5606431940862\\
0.86083984375	13.5522084254953\\
0.861328125	13.5438049044658\\
0.86181640625	13.5354326123442\\
0.8623046875	13.5270915305555\\
0.86279296875	13.5187816406031\\
0.86328125	13.510502924069\\
0.86376953125	13.5022553626132\\
0.8642578125	13.494038937974\\
0.86474609375	13.4858536319672\\
0.865234375	13.4776994264868\\
0.86572265625	13.4695763035041\\
0.8662109375	13.461484245068\\
0.86669921875	13.4534232333047\\
0.8671875	13.4453932504176\\
0.86767578125	13.4373942786871\\
0.8681640625	13.4294263004707\\
0.86865234375	13.4214892982024\\
0.869140625	13.4135832543929\\
0.86962890625	13.4057081516294\\
0.8701171875	13.3978639725756\\
0.87060546875	13.3900506999712\\
0.87109375	13.382268316632\\
0.87158203125	13.3745168054498\\
0.8720703125	13.3667961493922\\
0.87255859375	13.3591063315024\\
0.873046875	13.3514473348992\\
0.87353515625	13.3438191427767\\
0.8740234375	13.3362217384045\\
0.87451171875	13.328655105127\\
0.875	13.3211192263638\\
0.87548828125	13.3136140856094\\
0.8759765625	13.3061396664329\\
0.87646484375	13.2986959524781\\
0.876953125	13.2912829274633\\
0.87744140625	13.283900575181\\
0.8779296875	13.2765488794982\\
0.87841796875	13.2692278243558\\
0.87890625	13.2619373937686\\
0.87939453125	13.2546775718255\\
0.8798828125	13.2474483426889\\
0.88037109375	13.2402496905949\\
0.880859375	13.2330815998531\\
0.88134765625	13.2259440548463\\
0.8818359375	13.2188370400307\\
0.88232421875	13.2117605399355\\
0.8828125	13.204714539163\\
0.88330078125	13.1976990223882\\
0.8837890625	13.190713974359\\
0.88427734375	13.1837593798958\\
0.884765625	13.1768352238917\\
0.88525390625	13.1699414913119\\
0.8857421875	13.1630781671942\\
0.88623046875	13.1562452366483\\
0.88671875	13.149442684856\\
0.88720703125	13.1426704970711\\
0.8876953125	13.1359286586192\\
0.88818359375	13.1292171548976\\
0.888671875	13.122535971375\\
0.88916015625	13.1158850935919\\
0.8896484375	13.1092645071598\\
0.89013671875	13.1026741977618\\
0.890625	13.0961141511519\\
0.89111328125	13.0895843531552\\
0.8916015625	13.0830847896676\\
0.89208984375	13.076615446656\\
0.892578125	13.0701763101579\\
0.89306640625	13.0637673662814\\
0.8935546875	13.0573886012051\\
0.89404296875	13.0510400011778\\
0.89453125	13.0447215525189\\
0.89501953125	13.0384332416178\\
0.8955078125	13.0321750549338\\
0.89599609375	13.0259469789964\\
0.896484375	13.0197490004048\\
0.89697265625	13.013581105828\\
0.8974609375	13.0074432820047\\
0.89794921875	13.001335515743\\
0.8984375	12.9952577939206\\
0.89892578125	12.9892101034845\\
0.8994140625	12.9831924314508\\
0.89990234375	12.9772047649049\\
0.900390625	12.9712470910012\\
0.90087890625	12.965319396963\\
0.9013671875	12.9594216700824\\
0.90185546875	12.9535538977205\\
0.90234375	12.9477160673068\\
0.90283203125	12.9419081663394\\
0.9033203125	12.936130182385\\
0.90380859375	12.9303821030783\\
0.904296875	12.9246639161228\\
0.90478515625	12.9189756092897\\
0.9052734375	12.9133171704186\\
0.90576171875	12.9076885874168\\
0.90625	12.9020898482599\\
0.90673828125	12.8965209409909\\
0.9072265625	12.8909818537207\\
0.90771484375	12.8854725746278\\
0.908203125	12.8799930919583\\
0.90869140625	12.8745433940256\\
0.9091796875	12.8691234692106\\
0.90966796875	12.8637333059614\\
0.91015625	12.8583728927932\\
0.91064453125	12.8530422182884\\
0.9111328125	12.8477412710965\\
0.91162109375	12.8424700399336\\
0.912109375	12.8372285135829\\
0.91259765625	12.8320166808943\\
0.9130859375	12.8268345307843\\
0.91357421875	12.8216820522361\\
0.9140625	12.8165592342991\\
0.91455078125	12.8114660660896\\
0.9150390625	12.8064025367897\\
0.91552734375	12.8013686356481\\
0.916015625	12.7963643519796\\
0.91650390625	12.791389675165\\
0.9169921875	12.7864445946512\\
0.91748046875	12.781529099951\\
0.91796875	12.7766431806429\\
0.91845703125	12.7717868263714\\
0.9189453125	12.7669600268465\\
0.91943359375	12.7621627718441\\
0.919921875	12.7573950512052\\
0.92041015625	12.7526568548366\\
0.9208984375	12.7479481727103\\
0.92138671875	12.7432689948639\\
0.921875	12.7386193113997\\
0.92236328125	12.7339991124857\\
0.9228515625	12.7294083883547\\
0.92333984375	12.7248471293045\\
0.923828125	12.720315325698\\
0.92431640625	12.7158129679628\\
0.9248046875	12.7113400465913\\
0.92529296875	12.7068965521409\\
0.92578125	12.7024824752332\\
0.92626953125	12.6980978065546\\
0.9267578125	12.6937425368562\\
0.92724609375	12.6894166569533\\
0.927734375	12.6851201577256\\
0.92822265625	12.6808530301172\\
0.9287109375	12.6766152651362\\
0.92919921875	12.6724068538553\\
0.9296875	12.6682277874108\\
0.93017578125	12.6640780570034\\
0.9306640625	12.6599576538977\\
0.93115234375	12.6558665694221\\
0.931640625	12.6518047949688\\
0.93212890625	12.647772321994\\
0.9326171875	12.6437691420175\\
0.93310546875	12.6397952466225\\
0.93359375	12.6358506274563\\
0.93408203125	12.6319352762293\\
0.9345703125	12.6280491847155\\
0.93505859375	12.6241923447523\\
0.935546875	12.6203647482405\\
0.93603515625	12.616566387144\\
0.9365234375	12.6127972534901\\
0.93701171875	12.6090573393692\\
0.9375	12.6053466369347\\
0.93798828125	12.6016651384033\\
0.9384765625	12.5980128360544\\
0.93896484375	12.5943897222305\\
0.939453125	12.5907957893369\\
0.93994140625	12.5872310298417\\
0.9404296875	12.5836954362758\\
0.94091796875	12.5801890012328\\
0.94140625	12.5767117173689\\
0.94189453125	12.5732635774031\\
0.9423828125	12.5698445741165\\
0.94287109375	12.5664547003531\\
0.943359375	12.5630939490192\\
0.94384765625	12.5597623130834\\
0.9443359375	12.5564597855767\\
0.94482421875	12.5531863595924\\
0.9453125	12.5499420282859\\
0.94580078125	12.546726784875\\
0.9462890625	12.5435406226393\\
0.94677734375	12.5403835349209\\
0.947265625	12.5372555151234\\
0.94775390625	12.5341565567129\\
0.9482421875	12.5310866532171\\
0.94873046875	12.5280457982256\\
0.94921875	12.5250339853899\\
0.94970703125	12.5220512084234\\
0.9501953125	12.519097461101\\
0.95068359375	12.5161727372595\\
0.951171875	12.5132770307971\\
0.95166015625	12.510410335674\\
0.9521484375	12.5075726459115\\
0.95263671875	12.5047639555927\\
0.953125	12.5019842588621\\
};
\addplot [color=green,solid]
  table[row sep=crcr]{0.953125	12.5019842588621\\
0.95361328125	12.4992335499257\\
0.9541015625	12.4965118230506\\
0.95458984375	12.4938190725657\\
0.955078125	12.4911552928608\\
0.95556640625	12.4885204783871\\
0.9560546875	12.4859146236571\\
0.95654296875	12.4833377232444\\
0.95703125	12.4807897717836\\
0.95751953125	12.4782707639707\\
0.9580078125	12.4757806945624\\
0.95849609375	12.4733195583769\\
0.958984375	12.4708873502928\\
0.95947265625	12.4684840652501\\
0.9599609375	12.4661096982494\\
0.96044921875	12.4637642443525\\
0.9609375	12.4614476986818\\
0.96142578125	12.4591600564204\\
0.9619140625	12.4569013128124\\
0.96240234375	12.4546714631626\\
0.962890625	12.4524705028363\\
0.96337890625	12.4502984272596\\
0.9638671875	12.4481552319193\\
0.96435546875	12.4460409123626\\
0.96484375	12.4439554641975\\
0.96533203125	12.4418988830922\\
0.9658203125	12.4398711647756\\
0.96630859375	12.4378723050372\\
0.966796875	12.4359022997267\\
0.96728515625	12.4339611447542\\
0.9677734375	12.4320488360904\\
0.96826171875	12.430165369766\\
0.96875	12.4283107418724\\
0.96923828125	12.426484948561\\
0.9697265625	12.4246879860436\\
0.97021484375	12.4229198505921\\
0.970703125	12.4211805385387\\
0.97119140625	12.4194700462757\\
0.9716796875	12.4177883702558\\
0.97216796875	12.4161355069915\\
0.97265625	12.4145114530555\\
0.97314453125	12.4129162050806\\
0.9736328125	12.4113497597597\\
0.97412109375	12.4098121138457\\
0.974609375	12.4083032641513\\
0.97509765625	12.4068232075495\\
0.9755859375	12.405371940973\\
0.97607421875	12.4039494614146\\
0.9765625	12.4025557659268\\
0.97705078125	12.4011908516223\\
0.9775390625	12.3998547156734\\
0.97802734375	12.3985473553123\\
0.978515625	12.397268767831\\
0.97900390625	12.3960189505816\\
0.9794921875	12.3947979009755\\
0.97998046875	12.3936056164842\\
0.98046875	12.3924420946388\\
0.98095703125	12.3913073330304\\
0.9814453125	12.3902013293094\\
0.98193359375	12.3891240811861\\
0.982421875	12.3880755864306\\
0.98291015625	12.3870558428725\\
0.9833984375	12.386064848401\\
0.98388671875	12.3851026009652\\
0.984375	12.3841690985734\\
0.98486328125	12.383264339294\\
0.9853515625	12.3823883212545\\
0.98583984375	12.3815410426423\\
0.986328125	12.3807225017043\\
0.98681640625	12.3799326967468\\
0.9873046875	12.3791716261358\\
0.98779296875	12.3784392882968\\
0.98828125	12.3777356817147\\
0.98876953125	12.3770608049341\\
0.9892578125	12.3764146565588\\
0.98974609375	12.3757972352524\\
0.990234375	12.3752085397378\\
0.99072265625	12.3746485687973\\
0.9912109375	12.3741173212729\\
0.99169921875	12.3736147960656\\
0.9921875	12.3731409921364\\
0.99267578125	12.3726959085053\\
0.9931640625	12.3722795442519\\
0.99365234375	12.3718918985151\\
0.994140625	12.3715329704934\\
0.99462890625	12.3712027594445\\
0.9951171875	12.3709012646857\\
0.99560546875	12.3706284855935\\
0.99609375	12.3703844216038\\
0.99658203125	12.3701690722121\\
0.9970703125	12.369982436973\\
0.99755859375	12.3698245155006\\
0.998046875	12.3696953074685\\
0.99853515625	12.3695948126094\\
0.9990234375	12.3695230307156\\
0.99951171875	12.3694799616387\\
};
\addlegendentry{AR(5) Model};

\addplot [color=mycolor1,solid,forget plot]
  table[row sep=crcr]{-1	6.06134845614974\\
-0.99951171875	6.06136444283627\\
-0.9990234375	6.06141240293221\\
-0.99853515625	6.06149233654665\\
-0.998046875	6.06160424386135\\
-0.99755859375	6.06174812513078\\
-0.9970703125	6.06192398068215\\
-0.99658203125	6.06213181091537\\
-0.99609375	6.06237161630308\\
-0.99560546875	6.06264339739062\\
-0.9951171875	6.0629471547961\\
-0.99462890625	6.06328288921033\\
-0.994140625	6.06365060139686\\
-0.99365234375	6.064050292192\\
-0.9931640625	6.06448196250484\\
-0.99267578125	6.06494561331718\\
-0.9921875	6.0654412456836\\
-0.99169921875	6.06596886073151\\
-0.9912109375	6.06652845966103\\
-0.99072265625	6.06712004374517\\
-0.990234375	6.06774361432966\\
-0.98974609375	6.06839917283312\\
-0.9892578125	6.06908672074699\\
-0.98876953125	6.06980625963554\\
-0.98828125	6.07055779113592\\
-0.98779296875	6.07134131695812\\
-0.9873046875	6.07215683888508\\
-0.98681640625	6.0730043587726\\
-0.986328125	6.07388387854945\\
-0.98583984375	6.07479540021726\\
-0.9853515625	6.07573892585069\\
-0.98486328125	6.07671445759736\\
-0.984375	6.07772199767785\\
-0.98388671875	6.0787615483858\\
-0.9833984375	6.07983311208785\\
-0.98291015625	6.0809366912237\\
-0.982421875	6.08207228830614\\
-0.98193359375	6.08323990592104\\
-0.9814453125	6.0844395467274\\
-0.98095703125	6.08567121345736\\
-0.98046875	6.08693490891623\\
-0.97998046875	6.08823063598249\\
-0.9794921875	6.08955839760787\\
-0.97900390625	6.09091819681732\\
-0.978515625	6.09231003670907\\
-0.97802734375	6.09373392045463\\
-0.9775390625	6.09518985129885\\
-0.97705078125	6.09667783255992\\
-0.9765625	6.09819786762942\\
-0.97607421875	6.09974995997235\\
-0.9755859375	6.1013341131271\\
-0.97509765625	6.10295033070561\\
-0.974609375	6.10459861639325\\
-0.97412109375	6.10627897394898\\
-0.9736328125	6.10799140720528\\
-0.97314453125	6.10973592006828\\
-0.97265625	6.11151251651772\\
-0.97216796875	6.11332120060699\\
-0.9716796875	6.11516197646326\\
-0.97119140625	6.11703484828734\\
-0.970703125	6.1189398203539\\
-0.97021484375	6.1208768970114\\
-0.9697265625	6.12284608268213\\
-0.96923828125	6.12484738186233\\
-0.96875	6.12688079912212\\
-0.96826171875	6.12894633910561\\
-0.9677734375	6.13104400653095\\
-0.96728515625	6.13317380619029\\
-0.966796875	6.13533574294994\\
-0.96630859375	6.1375298217503\\
-0.9658203125	6.13975604760598\\
-0.96533203125	6.14201442560581\\
-0.96484375	6.14430496091289\\
-0.96435546875	6.14662765876464\\
-0.9638671875	6.14898252447284\\
-0.96337890625	6.15136956342368\\
-0.962890625	6.15378878107782\\
-0.96240234375	6.15624018297042\\
-0.9619140625	6.15872377471116\\
-0.96142578125	6.16123956198438\\
-0.9609375	6.16378755054904\\
-0.96044921875	6.16636774623881\\
-0.9599609375	6.16898015496211\\
-0.95947265625	6.17162478270217\\
-0.958984375	6.17430163551709\\
-0.95849609375	6.17701071953987\\
-0.9580078125	6.17975204097849\\
-0.95751953125	6.18252560611595\\
-0.95703125	6.18533142131031\\
-0.95654296875	6.18816949299479\\
-0.9560546875	6.19103982767777\\
-0.95556640625	6.19394243194293\\
-0.955078125	6.19687731244917\\
-0.95458984375	6.19984447593085\\
-0.9541015625	6.2028439291977\\
-0.95361328125	6.20587567913492\\
-0.953125	6.20893973270332\\
-0.95263671875	6.21203609693924\\
-0.9521484375	6.21516477895476\\
-0.95166015625	6.21832578593762\\
-0.951171875	6.22151912515143\\
-0.95068359375	6.2247448039356\\
-0.9501953125	6.2280028297055\\
-0.94970703125	6.23129320995245\\
-0.94921875	6.2346159522439\\
-0.94873046875	6.23797106422335\\
-0.9482421875	6.24135855361053\\
-0.94775390625	6.24477842820142\\
-0.947265625	6.24823069586832\\
-0.94677734375	6.25171536455996\\
-0.9462890625	6.25523244230149\\
-0.94580078125	6.25878193719465\\
-0.9453125	6.26236385741778\\
-0.94482421875	6.26597821122588\\
-0.9443359375	6.26962500695075\\
-0.94384765625	6.27330425300098\\
-0.943359375	6.27701595786209\\
-0.94287109375	6.28076013009662\\
-0.9423828125	6.28453677834409\\
-0.94189453125	6.2883459113212\\
-0.94140625	6.29218753782189\\
-0.94091796875	6.29606166671735\\
-0.9404296875	6.29996830695616\\
-0.93994140625	6.30390746756436\\
-0.939453125	6.3078791576455\\
-0.93896484375	6.31188338638077\\
-0.9384765625	6.31592016302902\\
-0.93798828125	6.31998949692695\\
-0.9375	6.32409139748901\\
-0.93701171875	6.32822587420769\\
-0.9365234375	6.33239293665349\\
-0.93603515625	6.336592594475\\
-0.935546875	6.34082485739904\\
-0.93505859375	6.34508973523069\\
-0.9345703125	6.34938723785346\\
-0.93408203125	6.35371737522927\\
-0.93359375	6.35808015739863\\
-0.93310546875	6.36247559448068\\
-0.9326171875	6.36690369667331\\
-0.93212890625	6.37136447425324\\
-0.931640625	6.3758579375761\\
-0.93115234375	6.38038409707653\\
-0.9306640625	6.38494296326829\\
-0.93017578125	6.38953454674434\\
-0.9296875	6.39415885817695\\
-0.92919921875	6.39881590831775\\
-0.9287109375	6.4035057079979\\
-0.92822265625	6.40822826812812\\
-0.927734375	6.41298359969883\\
-0.92724609375	6.41777171378024\\
-0.9267578125	6.42259262152243\\
-0.92626953125	6.42744633415549\\
-0.92578125	6.43233286298958\\
-0.92529296875	6.43725221941505\\
-0.9248046875	6.44220441490257\\
-0.92431640625	6.44718946100317\\
-0.923828125	6.45220736934842\\
-0.92333984375	6.45725815165045\\
-0.9228515625	6.46234181970217\\
-0.92236328125	6.46745838537723\\
-0.921875	6.47260786063029\\
-0.92138671875	6.47779025749699\\
-0.9208984375	6.48300558809414\\
-0.92041015625	6.48825386461979\\
-0.919921875	6.49353509935339\\
-0.91943359375	6.49884930465584\\
-0.9189453125	6.50419649296961\\
-0.91845703125	6.50957667681896\\
-0.91796875	6.51498986880989\\
-0.91748046875	6.52043608163035\\
-0.9169921875	6.52591532805036\\
-0.91650390625	6.53142762092208\\
-0.916015625	6.53697297317999\\
-0.91552734375	6.54255139784094\\
-0.9150390625	6.54816290800431\\
-0.91455078125	6.55380751685213\\
-0.9140625	6.55948523764918\\
-0.91357421875	6.56519608374312\\
-0.9130859375	6.57094006856465\\
-0.91259765625	6.57671720562754\\
-0.912109375	6.58252750852886\\
-0.91162109375	6.58837099094906\\
-0.9111328125	6.59424766665204\\
-0.91064453125	6.6001575494854\\
-0.91015625	6.60610065338044\\
-0.90966796875	6.61207699235236\\
-0.9091796875	6.6180865805004\\
-0.90869140625	6.62412943200793\\
-0.908203125	6.63020556114257\\
-0.90771484375	6.63631498225637\\
-0.9072265625	6.64245770978594\\
-0.90673828125	6.64863375825251\\
-0.90625	6.65484314226217\\
-0.90576171875	6.66108587650591\\
-0.9052734375	6.66736197575983\\
-0.90478515625	6.6736714548852\\
-0.904296875	6.6800143288287\\
-0.90380859375	6.68639061262248\\
-0.9033203125	6.69280032138429\\
-0.90283203125	6.69924347031767\\
-0.90234375	6.70572007471208\\
-0.90185546875	6.71223014994303\\
-0.9013671875	6.71877371147224\\
-0.90087890625	6.72535077484774\\
-0.900390625	6.73196135570407\\
-0.89990234375	6.73860546976238\\
-0.8994140625	6.74528313283064\\
-0.89892578125	6.75199436080368\\
-0.8984375	6.75873916966347\\
-0.89794921875	6.76551757547916\\
-0.8974609375	6.77232959440728\\
-0.89697265625	6.77917524269192\\
-0.896484375	6.7860545366648\\
-0.89599609375	6.79296749274549\\
-0.8955078125	6.79991412744155\\
-0.89501953125	6.80689445734868\\
-0.89453125	6.81390849915084\\
-0.89404296875	6.8209562696205\\
-0.8935546875	6.82803778561868\\
-0.89306640625	6.8351530640952\\
-0.892578125	6.84230212208882\\
-0.89208984375	6.84948497672738\\
-0.8916015625	6.85670164522792\\
-0.89111328125	6.86395214489696\\
-0.890625	6.87123649313057\\
-0.89013671875	6.87855470741455\\
-0.8896484375	6.88590680532461\\
-0.88916015625	6.89329280452654\\
-0.888671875	6.90071272277636\\
-0.88818359375	6.9081665779205\\
-0.8876953125	6.915654387896\\
-0.88720703125	6.92317617073057\\
-0.88671875	6.9307319445429\\
-0.88623046875	6.93832172754276\\
-0.8857421875	6.94594553803117\\
-0.88525390625	6.95360339440058\\
-0.884765625	6.96129531513511\\
-0.88427734375	6.96902131881058\\
-0.8837890625	6.97678142409487\\
-0.88330078125	6.98457564974791\\
-0.8828125	6.99240401462205\\
-0.88232421875	7.00026653766208\\
-0.8818359375	7.00816323790548\\
-0.88134765625	7.01609413448264\\
-0.880859375	7.02405924661696\\
-0.88037109375	7.03205859362508\\
-0.8798828125	7.04009219491711\\
-0.87939453125	7.04816006999671\\
-0.87890625	7.05626223846135\\
-0.87841796875	7.06439872000252\\
-0.8779296875	7.07256953440583\\
-0.87744140625	7.0807747015513\\
-0.876953125	7.08901424141348\\
-0.87646484375	7.09728817406169\\
-0.8759765625	7.1055965196602\\
-0.87548828125	7.11393929846837\\
-0.875	7.12231653084095\\
-0.87451171875	7.1307282372282\\
-0.8740234375	7.13917443817612\\
-0.87353515625	7.14765515432662\\
-0.873046875	7.15617040641776\\
-0.87255859375	7.16472021528394\\
-0.8720703125	7.17330460185607\\
-0.87158203125	7.1819235871618\\
-0.87109375	7.19057719232578\\
-0.87060546875	7.19926543856972\\
-0.8701171875	7.20798834721277\\
-0.86962890625	7.2167459396716\\
-0.869140625	7.22553823746065\\
-0.86865234375	7.23436526219238\\
-0.8681640625	7.24322703557742\\
-0.86767578125	7.25212357942484\\
-0.8671875	7.26105491564229\\
-0.86669921875	7.2700210662363\\
-0.8662109375	7.27902205331245\\
-0.86572265625	7.28805789907559\\
-0.865234375	7.29712862583004\\
-0.86474609375	7.30623425597986\\
-0.8642578125	7.31537481202905\\
-0.86376953125	7.32455031658174\\
-0.86328125	7.33376079234246\\
-0.86279296875	7.34300626211634\\
-0.8623046875	7.35228674880933\\
-0.86181640625	7.36160227542845\\
-0.861328125	7.37095286508199\\
-0.86083984375	7.38033854097979\\
-0.8603515625	7.38975932643338\\
-0.85986328125	7.39921524485634\\
-0.859375	7.40870631976439\\
-0.85888671875	7.41823257477573\\
-0.8583984375	7.42779403361126\\
-0.85791015625	7.43739072009476\\
-0.857421875	7.44702265815319\\
-0.85693359375	7.45668987181689\\
-0.8564453125	7.46639238521984\\
-0.85595703125	7.47613022259993\\
-0.85546875	7.48590340829913\\
-0.85498046875	7.49571196676379\\
-0.8544921875	7.50555592254487\\
-0.85400390625	7.5154353002982\\
-0.853515625	7.52535012478472\\
-0.85302734375	7.53530042087071\\
-0.8525390625	7.54528621352806\\
-0.85205078125	7.55530752783453\\
-0.8515625	7.56536438897401\\
-0.85107421875	7.57545682223673\\
-0.8505859375	7.58558485301954\\
-0.85009765625	7.5957485068262\\
-0.849609375	7.60594780926762\\
-0.84912109375	7.61618278606206\\
-0.8486328125	7.62645346303551\\
-0.84814453125	7.63675986612185\\
-0.84765625	7.64710202136313\\
-0.84716796875	7.65747995490994\\
-0.8466796875	7.66789369302148\\
-0.84619140625	7.67834326206603\\
-0.845703125	7.68882868852112\\
-0.84521484375	7.69934999897377\\
-0.8447265625	7.70990722012085\\
-0.84423828125	7.7205003787693\\
-0.84375	7.73112950183641\\
-0.84326171875	7.74179461635012\\
-0.8427734375	7.75249574944926\\
-0.84228515625	7.76323292838387\\
-0.841796875	7.77400618051544\\
-0.84130859375	7.78481553331725\\
-0.8408203125	7.79566101437459\\
-0.84033203125	7.8065426513851\\
-0.83984375	7.817460472159\\
-0.83935546875	7.82841450461947\\
-0.8388671875	7.83940477680279\\
-0.83837890625	7.85043131685883\\
-0.837890625	7.86149415305115\\
-0.83740234375	7.87259331375741\\
-0.8369140625	7.88372882746966\\
-0.83642578125	7.89490072279458\\
-0.8359375	7.90610902845384\\
-0.83544921875	7.91735377328434\\
-0.8349609375	7.92863498623859\\
-0.83447265625	7.93995269638494\\
-0.833984375	7.95130693290793\\
-0.83349609375	7.96269772510855\\
-0.8330078125	7.97412510240462\\
-0.83251953125	7.98558909433106\\
-0.83203125	7.99708973054016\\
-0.83154296875	8.00862704080198\\
-0.8310546875	8.02020105500458\\
-0.83056640625	8.03181180315443\\
-0.830078125	8.0434593153766\\
-0.82958984375	8.05514362191523\\
-0.8291015625	8.0668647531337\\
-0.82861328125	8.0786227395151\\
-0.828125	8.09041761166243\\
-0.82763671875	8.10224940029898\\
-0.8271484375	8.11411813626869\\
-0.82666015625	8.12602385053642\\
-0.826171875	8.13796657418831\\
-0.82568359375	8.14994633843211\\
-0.8251953125	8.1619631745975\\
-0.82470703125	8.17401711413646\\
-0.82421875	8.18610818862359\\
-0.82373046875	8.19823642975641\\
-0.8232421875	8.21040186935577\\
-0.82275390625	8.22260453936616\\
-0.822265625	8.23484447185605\\
-0.82177734375	8.24712169901822\\
-0.8212890625	8.25943625317018\\
-0.82080078125	8.27178816675442\\
-0.8203125	8.28417747233885\\
-0.81982421875	8.29660420261707\\
-0.8193359375	8.30906839040883\\
-0.81884765625	8.32157006866027\\
-0.818359375	8.33410927044439\\
-0.81787109375	8.34668602896132\\
-0.8173828125	8.35930037753873\\
-0.81689453125	8.37195234963218\\
-0.81640625	8.38464197882552\\
-0.81591796875	8.39736929883119\\
-0.8154296875	8.41013434349063\\
-0.81494140625	8.4229371467747\\
-0.814453125	8.43577774278397\\
-0.81396484375	8.44865616574909\\
-0.8134765625	8.46157245003129\\
-0.81298828125	8.47452663012263\\
-0.8125	8.48751874064642\\
-0.81201171875	8.50054881635763\\
-0.8115234375	8.51361689214325\\
-0.81103515625	8.52672300302269\\
-0.810546875	8.53986718414817\\
-0.81005859375	8.55304947080509\\
-0.8095703125	8.56626989841243\\
-0.80908203125	8.57952850252316\\
-0.80859375	8.59282531882463\\
-0.80810546875	8.606160383139\\
-0.8076171875	8.61953373142354\\
-0.80712890625	8.63294539977118\\
-0.806640625	8.64639542441076\\
-0.80615234375	8.65988384170759\\
-0.8056640625	8.6734106881637\\
-0.80517578125	8.68697600041842\\
-0.8046875	8.70057981524864\\
-0.80419921875	8.71422216956936\\
-0.8037109375	8.72790310043399\\
-0.80322265625	8.74162264503481\\
-0.802734375	8.75538084070345\\
-0.80224609375	8.76917772491123\\
-0.8017578125	8.78301333526966\\
-0.80126953125	8.79688770953079\\
-0.80078125	8.81080088558769\\
-0.80029296875	8.82475290147489\\
-0.7998046875	8.83874379536879\\
-0.79931640625	8.8527736055881\\
-0.798828125	8.86684237059429\\
-0.79833984375	8.88095012899204\\
-0.7978515625	8.89509691952962\\
-0.79736328125	8.90928278109944\\
-0.796875	8.92350775273841\\
-0.79638671875	8.93777187362845\\
-0.7958984375	8.95207518309688\\
-0.79541015625	8.96641772061695\\
-0.794921875	8.98079952580822\\
-0.79443359375	8.9952206384371\\
-0.7939453125	9.00968109841722\\
-0.79345703125	9.02418094580999\\
-0.79296875	9.038720220825\\
-0.79248046875	9.05329896382049\\
-0.7919921875	9.06791721530388\\
-0.79150390625	9.08257501593218\\
-0.791015625	9.09727240651249\\
-0.79052734375	9.11200942800248\\
-0.7900390625	9.12678612151089\\
-0.78955078125	9.14160252829796\\
-0.7890625	9.15645868977597\\
-0.78857421875	9.1713546475097\\
-0.7880859375	9.18629044321694\\
-0.78759765625	9.20126611876895\\
-0.787109375	9.21628171619101\\
-0.78662109375	9.23133727766286\\
-0.7861328125	9.24643284551921\\
-0.78564453125	9.2615684622503\\
-0.78515625	9.27674417050234\\
-0.78466796875	9.29196001307807\\
-0.7841796875	9.30721603293721\\
-0.78369140625	9.32251227319703\\
-0.783203125	9.33784877713286\\
-0.78271484375	9.35322558817854\\
-0.7822265625	9.36864274992707\\
-0.78173828125	9.38410030613101\\
-0.78125	9.39959830070307\\
-0.78076171875	9.41513677771663\\
-0.7802734375	9.43071578140628\\
-0.77978515625	9.44633535616833\\
-0.779296875	9.46199554656136\\
-0.77880859375	9.47769639730677\\
-0.7783203125	9.49343795328933\\
-0.77783203125	9.50922025955768\\
-0.77734375	9.52504336132496\\
-0.77685546875	9.54090730396925\\
-0.7763671875	9.55681213303425\\
-0.77587890625	9.57275789422975\\
-0.775390625	9.58874463343219\\
-0.77490234375	9.60477239668531\\
-0.7744140625	9.62084123020059\\
-0.77392578125	9.63695118035792\\
-0.7734375	9.65310229370613\\
-0.77294921875	9.66929461696358\\
-0.7724609375	9.6855281970187\\
-0.77197265625	9.70180308093064\\
-0.771484375	9.71811931592977\\
-0.77099609375	9.73447694941834\\
-0.7705078125	9.75087602897103\\
-0.77001953125	9.76731660233556\\
-0.76953125	9.78379871743326\\
-0.76904296875	9.8003224223597\\
-0.7685546875	9.81688776538528\\
-0.76806640625	9.83349479495584\\
-0.767578125	9.85014355969324\\
-0.76708984375	9.86683410839603\\
-0.7666015625	9.88356649003999\\
-0.76611328125	9.9003407537788\\
-0.765625	9.91715694894467\\
-0.76513671875	9.93401512504892\\
-0.7646484375	9.95091533178263\\
-0.76416015625	9.96785761901727\\
-0.763671875	9.98484203680534\\
-0.76318359375	10.001868635381\\
-0.7626953125	10.0189374651607\\
-0.76220703125	10.0360485767438\\
-0.76171875	10.0532020209134\\
-0.76123046875	10.0703978486367\\
-0.7607421875	10.0876361110658\\
-0.76025390625	10.1049168595384\\
-0.759765625	10.1222401455785\\
-0.75927734375	10.1396060208967\\
-0.7587890625	10.1570145373915\\
-0.75830078125	10.1744657471494\\
-0.7578125	10.1919597024459\\
-0.75732421875	10.2094964557459\\
-0.7568359375	10.2270760597047\\
-0.75634765625	10.2446985671686\\
-0.755859375	10.2623640311753\\
-0.75537109375	10.2800725049551\\
-0.7548828125	10.2978240419311\\
-0.75439453125	10.3156186957201\\
-0.75390625	10.3334565201336\\
-0.75341796875	10.3513375691779\\
-0.7529296875	10.3692618970554\\
-0.75244140625	10.3872295581647\\
-0.751953125	10.4052406071021\\
-0.75146484375	10.4232950986617\\
-0.7509765625	10.4413930878361\\
-0.75048828125	10.4595346298178\\
-0.75	10.477719779999\\
-0.74951171875	10.4959485939733\\
-0.7490234375	10.5142211275355\\
-0.74853515625	10.5325374366832\\
-0.748046875	10.5508975776168\\
-0.74755859375	10.569301606741\\
-0.7470703125	10.5877495806648\\
-0.74658203125	10.6062415562029\\
-0.74609375	10.624777590376\\
-0.74560546875	10.6433577404118\\
-0.7451171875	10.6619820637459\\
-0.74462890625	10.6806506180222\\
-0.744140625	10.699363461094\\
-0.74365234375	10.7181206510246\\
-0.7431640625	10.7369222460884\\
-0.74267578125	10.7557683047711\\
-0.7421875	10.7746588857712\\
-0.74169921875	10.7935940480003\\
-0.7412109375	10.8125738505841\\
-0.74072265625	10.8315983528631\\
-0.740234375	10.8506676143937\\
-0.73974609375	10.8697816949486\\
-0.7392578125	10.8889406545181\\
-0.73876953125	10.9081445533103\\
-0.73828125	10.9273934517527\\
-0.73779296875	10.9466874104924\\
-0.7373046875	10.9660264903973\\
-0.73681640625	10.9854107525568\\
-0.736328125	11.0048402582825\\
-0.73583984375	11.0243150691096\\
-0.7353515625	11.0438352467971\\
-0.73486328125	11.0634008533292\\
-0.734375	11.0830119509157\\
-0.73388671875	11.1026686019932\\
-0.7333984375	11.1223708692261\\
-0.73291015625	11.1421188155068\\
-0.732421875	11.1619125039575\\
-0.73193359375	11.1817519979305\\
-0.7314453125	11.2016373610092\\
-0.73095703125	11.2215686570091\\
-0.73046875	11.2415459499787\\
-0.72998046875	11.2615693042004\\
-0.7294921875	11.2816387841912\\
-0.72900390625	11.301754454704\\
-0.728515625	11.3219163807283\\
-0.72802734375	11.3421246274913\\
-0.7275390625	11.3623792604584\\
-0.72705078125	11.3826803453349\\
-0.7265625	11.403027948066\\
-0.72607421875	11.4234221348387\\
-0.7255859375	11.443862972082\\
-0.72509765625	11.4643505264685\\
-0.724609375	11.4848848649148\\
-0.72412109375	11.5054660545828\\
-0.7236328125	11.5260941628807\\
-0.72314453125	11.5467692574637\\
-0.72265625	11.5674914062353\\
-0.72216796875	11.5882606773483\\
-0.7216796875	11.6090771392054\\
-0.72119140625	11.6299408604608\\
-0.720703125	11.6508519100206\\
-0.72021484375	11.6718103570444\\
-0.7197265625	11.6928162709459\\
-0.71923828125	11.713869721394\\
-0.71875	11.734970778314\\
-0.71826171875	11.7561195118886\\
-0.7177734375	11.7773159925588\\
-0.71728515625	11.798560291025\\
-0.716796875	11.8198524782484\\
-0.71630859375	11.8411926254514\\
-0.7158203125	11.8625808041193\\
-0.71533203125	11.8840170860009\\
-0.71484375	11.9055015431099\\
-0.71435546875	11.9270342477261\\
-0.7138671875	11.9486152723959\\
-0.71337890625	11.970244689934\\
-0.712890625	11.9919225734243\\
-0.71240234375	12.0136489962209\\
-0.7119140625	12.0354240319494\\
-0.71142578125	12.0572477545078\\
-0.7109375	12.0791202380681\\
-0.71044921875	12.1010415570767\\
-0.7099609375	12.1230117862562\\
-0.70947265625	12.1450310006063\\
-0.708984375	12.1670992754049\\
-0.70849609375	12.1892166862093\\
-0.7080078125	12.2113833088575\\
-0.70751953125	12.2335992194692\\
-0.70703125	12.2558644944471\\
-0.70654296875	12.278179210478\\
-0.7060546875	12.3005434445339\\
-0.70556640625	12.3229572738737\\
-0.705078125	12.3454207760439\\
-0.70458984375	12.3679340288797\\
-0.7041015625	12.3904971105069\\
-0.70361328125	12.4131100993425\\
-0.703125	12.4357730740962\\
-0.70263671875	12.4584861137716\\
-0.7021484375	12.4812492976676\\
-0.70166015625	12.5040627053792\\
-0.701171875	12.5269264167996\\
-0.70068359375	12.5498405121204\\
-0.7001953125	12.5728050718337\\
-0.69970703125	12.5958201767333\\
-0.69921875	12.6188859079155\\
-0.69873046875	12.6420023467809\\
-0.6982421875	12.6651695750354\\
-0.69775390625	12.688387674692\\
-0.697265625	12.7116567280713\\
-0.69677734375	12.7349768178037\\
-0.6962890625	12.7583480268301\\
-0.69580078125	12.7817704384037\\
-0.6953125	12.805244136091\\
-0.69482421875	12.8287692037734\\
-0.6943359375	12.8523457256485\\
-0.69384765625	12.8759737862313\\
-0.693359375	12.8996534703561\\
-0.69287109375	12.9233848631771\\
-0.6923828125	12.9471680501708\\
-0.69189453125	12.9710031171364\\
-0.69140625	12.994890150198\\
-0.69091796875	13.0188292358055\\
-0.6904296875	13.0428204607363\\
-0.68994140625	13.0668639120969\\
-0.689453125	13.0909596773238\\
-0.68896484375	13.1151078441857\\
-0.6884765625	13.1393085007843\\
-0.68798828125	13.1635617355562\\
-0.6875	13.1878676372742\\
-0.68701171875	13.2122262950489\\
-0.6865234375	13.2366377983301\\
-0.68603515625	13.2611022369083\\
-0.685546875	13.2856197009164\\
-0.68505859375	13.3101902808312\\
-0.6845703125	13.3348140674746\\
-0.68408203125	13.3594911520156\\
-0.68359375	13.3842216259717\\
-0.68310546875	13.4090055812103\\
-0.6826171875	13.4338431099506\\
-0.68212890625	13.4587343047647\\
-0.681640625	13.4836792585798\\
-0.68115234375	13.5086780646793\\
-0.6806640625	13.5337308167049\\
-0.68017578125	13.5588376086575\\
-0.6796875	13.5839985348996\\
-0.67919921875	13.6092136901566\\
-0.6787109375	13.6344831695184\\
-0.67822265625	13.6598070684411\\
-0.677734375	13.6851854827487\\
-0.67724609375	13.710618508635\\
-0.6767578125	13.7361062426649\\
-0.67626953125	13.7616487817762\\
-0.67578125	13.7872462232817\\
-0.67529296875	13.8128986648703\\
-0.6748046875	13.8386062046094\\
-0.67431640625	13.864368940946\\
-0.673828125	13.8901869727089\\
-0.67333984375	13.9160603991103\\
-0.6728515625	13.9419893197477\\
-0.67236328125	13.9679738346054\\
-0.671875	13.9940140440567\\
-0.67138671875	14.0201100488653\\
-0.6708984375	14.0462619501875\\
-0.67041015625	14.0724698495737\\
-0.669921875	14.0987338489704\\
-0.66943359375	14.1250540507221\\
-0.6689453125	14.1514305575731\\
-0.66845703125	14.1778634726694\\
-0.66796875	14.2043528995605\\
-0.66748046875	14.2308989422015\\
-0.6669921875	14.2575017049545\\
-0.66650390625	14.2841612925914\\
-0.666015625	14.3108778102951\\
-0.66552734375	14.3376513636615\\
-0.6650390625	14.3644820587019\\
-0.66455078125	14.3913700018446\\
-0.6640625	14.4183152999369\\
-0.66357421875	14.4453180602473\\
-0.6630859375	14.4723783904673\\
-0.66259765625	14.4994963987136\\
-0.662109375	14.5266721935298\\
-0.66162109375	14.5539058838888\\
-0.6611328125	14.5811975791949\\
-0.66064453125	14.6085473892853\\
-0.66015625	14.635955424433\\
-0.65966796875	14.663421795348\\
-0.6591796875	14.6909466131803\\
-0.65869140625	14.7185299895214\\
-0.658203125	14.7461720364064\\
-0.65771484375	14.7738728663169\\
-0.6572265625	14.801632592182\\
-0.65673828125	14.8294513273816\\
-0.65625	14.8573291857478\\
-0.65576171875	14.8852662815675\\
-0.6552734375	14.9132627295844\\
-0.65478515625	14.9413186450015\\
-0.654296875	14.9694341434829\\
-0.65380859375	14.9976093411565\\
-0.6533203125	15.0258443546161\\
-0.65283203125	15.0541393009235\\
-0.65234375	15.0824942976111\\
-0.65185546875	15.1109094626842\\
-0.6513671875	15.1393849146228\\
-0.65087890625	15.1679207723848\\
-0.650390625	15.1965171554076\\
-0.64990234375	15.2251741836109\\
-0.6494140625	15.2538919773989\\
-0.64892578125	15.2826706576628\\
-0.6484375	15.3115103457832\\
-0.64794921875	15.3404111636323\\
-0.6474609375	15.369373233577\\
-0.64697265625	15.3983966784803\\
-0.646484375	15.427481621705\\
-0.64599609375	15.456628187115\\
-0.6455078125	15.485836499079\\
-0.64501953125	15.5151066824719\\
-0.64453125	15.5444388626781\\
-0.64404296875	15.5738331655939\\
-0.6435546875	15.6032897176297\\
-0.64306640625	15.6328086457132\\
-0.642578125	15.6623900772915\\
-0.64208984375	15.692034140334\\
-0.6416015625	15.7217409633348\\
-0.64111328125	15.7515106753157\\
-0.640625	15.7813434058287\\
-0.64013671875	15.8112392849584\\
-0.6396484375	15.8411984433252\\
-0.63916015625	15.871221012088\\
-0.638671875	15.9013071229463\\
-0.63818359375	15.9314569081437\\
-0.6376953125	15.9616705004705\\
-0.63720703125	15.9919480332661\\
-0.63671875	16.0222896404223\\
-0.63623046875	16.052695456386\\
-0.6357421875	16.0831656161618\\
-0.63525390625	16.1137002553152\\
-0.634765625	16.1442995099754\\
-0.63427734375	16.1749635168381\\
-0.6337890625	16.2056924131684\\
-0.63330078125	16.2364863368041\\
-0.6328125	16.2673454261581\\
-0.63232421875	16.298269820222\\
-0.6318359375	16.3292596585685\\
-0.63134765625	16.3603150813551\\
-0.630859375	16.3914362293263\\
-0.63037109375	16.4226232438175\\
-0.6298828125	16.4538762667576\\
-0.62939453125	16.4851954406722\\
-0.62890625	16.5165809086868\\
-0.62841796875	16.5480328145298\\
-0.6279296875	16.5795513025358\\
-0.62744140625	16.6111365176488\\
-0.626953125	16.6427886054254\\
-0.62646484375	16.6745077120379\\
-0.6259765625	16.7062939842775\\
-0.62548828125	16.7381475695581\\
-0.625	16.7700686159188\\
-0.62451171875	16.8020572720279\\
-0.6240234375	16.8341136871858\\
-0.62353515625	16.8662380113286\\
-0.623046875	16.8984303950315\\
-0.62255859375	16.9306909895118\\
-0.6220703125	16.9630199466331\\
-0.62158203125	16.9954174189078\\
-0.62109375	17.0278835595015\\
-0.62060546875	17.0604185222358\\
-0.6201171875	17.0930224615922\\
-0.61962890625	17.1256955327155\\
-0.619140625	17.1584378914174\\
-0.61865234375	17.1912496941802\\
-0.6181640625	17.2241310981601\\
-0.61767578125	17.2570822611912\\
-0.6171875	17.2901033417891\\
-0.61669921875	17.3231944991544\\
-0.6162109375	17.3563558931765\\
-0.61572265625	17.3895876844375\\
-0.615234375	17.4228900342157\\
-0.61474609375	17.4562631044896\\
-0.6142578125	17.4897070579416\\
-0.61376953125	17.5232220579621\\
-0.61328125	17.5568082686529\\
-0.61279296875	17.5904658548315\\
-0.6123046875	17.6241949820348\\
-0.61181640625	17.6579958165232\\
-0.611328125	17.6918685252846\\
-0.61083984375	17.7258132760381\\
-0.6103515625	17.7598302372384\\
-0.60986328125	17.7939195780798\\
-0.609375	17.8280814684998\\
-0.60888671875	17.8623160791841\\
-0.6083984375	17.8966235815699\\
-0.60791015625	17.9310041478507\\
-0.607421875	17.9654579509799\\
-0.60693359375	17.9999851646756\\
-0.6064453125	18.0345859634247\\
-0.60595703125	18.0692605224867\\
-0.60546875	18.1040090178989\\
-0.60498046875	18.1388316264799\\
-0.6044921875	18.1737285258345\\
-0.60400390625	18.2086998943581\\
-0.603515625	18.2437459112407\\
-0.60302734375	18.2788667564718\\
-0.6025390625	18.3140626108449\\
-0.60205078125	18.3493336559617\\
-0.6015625	18.384680074237\\
-0.60107421875	18.4201020489031\\
-0.6005859375	18.4555997640144\\
-0.60009765625	18.4911734044523\\
-0.599609375	18.5268231559296\\
-0.59912109375	18.5625492049956\\
-0.5986328125	18.5983517390404\\
-0.59814453125	18.6342309463\\
-0.59765625	18.6701870158609\\
-0.59716796875	18.7062201376655\\
-0.5966796875	18.7423305025161\\
-0.59619140625	18.7785183020807\\
-0.595703125	18.8147837288974\\
-0.59521484375	18.8511269763798\\
-0.5947265625	18.8875482388214\\
-0.59423828125	18.9240477114015\\
-0.59375	18.9606255901898\\
-0.59326171875	18.9972820721516\\
-0.5927734375	19.034017355153\\
-0.59228515625	19.0708316379661\\
-0.591796875	19.1077251202744\\
-0.59130859375	19.1446980026779\\
-0.5908203125	19.1817504866984\\
-0.59033203125	19.2188827747853\\
-0.58984375	19.2560950703203\\
-0.58935546875	19.2933875776233\\
-0.5888671875	19.330760501958\\
-0.58837890625	19.3682140495369\\
-0.587890625	19.4057484275274\\
-0.58740234375	19.4433638440571\\
-0.5869140625	19.4810605082194\\
-0.58642578125	19.5188386300795\\
-0.5859375	19.5566984206798\\
-0.58544921875	19.5946400920456\\
-0.5849609375	19.6326638571913\\
-0.58447265625	19.670769930126\\
-0.583984375	19.7089585258593\\
-0.58349609375	19.7472298604073\\
-0.5830078125	19.7855841507988\\
-0.58251953125	19.8240216150808\\
-0.58203125	19.8625424723253\\
-0.58154296875	19.9011469426346\\
-0.5810546875	19.939835247148\\
-0.58056640625	19.9786076080478\\
-0.580078125	20.0174642485655\\
-0.57958984375	20.0564053929882\\
-0.5791015625	20.0954312666649\\
-0.57861328125	20.1345420960127\\
-0.578125	20.1737381085233\\
-0.57763671875	20.2130195327699\\
-0.5771484375	20.2523865984129\\
-0.57666015625	20.291839536207\\
-0.576171875	20.331378578008\\
-0.57568359375	20.3710039567787\\
-0.5751953125	20.4107159065964\\
-0.57470703125	20.4505146626591\\
-0.57421875	20.4904004612928\\
-0.57373046875	20.5303735399578\\
-0.5732421875	20.570434137256\\
-0.57275390625	20.6105824929377\\
-0.572265625	20.6508188479087\\
-0.57177734375	20.6911434442373\\
-0.5712890625	20.7315565251612\\
-0.57080078125	20.7720583350952\\
-0.5703125	20.8126491196379\\
-0.56982421875	20.8533291255792\\
-0.5693359375	20.8940986009074\\
-0.56884765625	20.934957794817\\
-0.568359375	20.9759069577158\\
-0.56787109375	21.0169463412324\\
-0.5673828125	21.0580761982238\\
-0.56689453125	21.0992967827831\\
-0.56640625	21.1406083502468\\
-0.56591796875	21.1820111572032\\
-0.5654296875	21.2235054614992\\
-0.56494140625	21.2650915222491\\
-0.564453125	21.3067695998417\\
-0.56396484375	21.3485399559487\\
-0.5634765625	21.3904028535326\\
-0.56298828125	21.4323585568547\\
-0.5625	21.4744073314832\\
-0.56201171875	21.5165494443013\\
-0.5615234375	21.5587851635159\\
-0.56103515625	21.6011147586653\\
-0.560546875	21.6435385006277\\
-0.56005859375	21.6860566616302\\
-0.5595703125	21.7286695152563\\
-0.55908203125	21.7713773364554\\
-0.55859375	21.8141804015507\\
-0.55810546875	21.8570789882485\\
-0.5576171875	21.9000733756462\\
-0.55712890625	21.943163844242\\
-0.556640625	21.9863506759429\\
-0.55615234375	22.0296341540743\\
-0.5556640625	22.0730145633887\\
-0.55517578125	22.1164921900747\\
-0.5546875	22.1600673217665\\
-0.55419921875	22.2037402475528\\
-0.5537109375	22.247511257986\\
-0.55322265625	22.291380645092\\
-0.552734375	22.3353487023791\\
-0.55224609375	22.379415724848\\
-0.5517578125	22.4235820090008\\
-0.55126953125	22.4678478528512\\
-0.55078125	22.5122135559337\\
-0.55029296875	22.5566794193139\\
-0.5498046875	22.6012457455977\\
-0.54931640625	22.645912838942\\
-0.548828125	22.690681005064\\
-0.54833984375	22.7355505512518\\
-0.5478515625	22.7805217863741\\
-0.54736328125	22.825595020891\\
-0.546875	22.8707705668635\\
-0.54638671875	22.9160487379648\\
-0.5458984375	22.9614298494902\\
-0.54541015625	23.0069142183675\\
-0.544921875	23.0525021631681\\
-0.54443359375	23.0981940041176\\
-0.5439453125	23.1439900631061\\
-0.54345703125	23.1898906636997\\
-0.54296875	23.2358961311511\\
-0.54248046875	23.2820067924106\\
-0.5419921875	23.3282229761373\\
-0.54150390625	23.3745450127104\\
-0.541015625	23.4209732342402\\
-0.54052734375	23.4675079745797\\
-0.5400390625	23.5141495693361\\
-0.53955078125	23.5608983558819\\
-0.5390625	23.607754673367\\
-0.53857421875	23.6547188627301\\
-0.5380859375	23.7017912667108\\
-0.53759765625	23.7489722298612\\
-0.537109375	23.7962620985578\\
-0.53662109375	23.8436612210139\\
-0.5361328125	23.8911699472915\\
-0.53564453125	23.9387886293137\\
-0.53515625	23.9865176208768\\
-0.53466796875	24.034357277663\\
-0.5341796875	24.0823079572527\\
-0.53369140625	24.1303700191371\\
-0.533203125	24.1785438247314\\
-0.53271484375	24.2268297373866\\
-0.5322265625	24.2752281224036\\
-0.53173828125	24.3237393470452\\
-0.53125	24.3723637805496\\
-0.53076171875	24.4211017941438\\
-0.5302734375	24.4699537610565\\
-0.52978515625	24.5189200565315\\
-0.529296875	24.5680010578416\\
-0.52880859375	24.6171971443016\\
-0.5283203125	24.6665086972825\\
-0.52783203125	24.715936100225\\
-0.52734375	24.7654797386533\\
-0.52685546875	24.8151400001893\\
-0.5263671875	24.8649172745669\\
-0.52587890625	24.9148119536454\\
-0.525390625	24.9648244314247\\
-0.52490234375	25.0149551040593\\
-0.5244140625	25.0652043698726\\
-0.52392578125	25.1155726293718\\
-0.5234375	25.1660602852628\\
-0.52294921875	25.2166677424645\\
-0.5224609375	25.2673954081242\\
-0.52197265625	25.3182436916324\\
-0.521484375	25.369213004638\\
-0.52099609375	25.4203037610638\\
-0.5205078125	25.4715163771212\\
-0.52001953125	25.5228512713268\\
-0.51953125	25.5743088645168\\
-0.51904296875	25.6258895798635\\
-0.5185546875	25.6775938428909\\
-0.51806640625	25.7294220814906\\
-0.517578125	25.7813747259378\\
-0.51708984375	25.8334522089076\\
-0.5166015625	25.8856549654912\\
-0.51611328125	25.9379834332122\\
-0.515625	25.9904380520434\\
-0.51513671875	26.0430192644232\\
-0.5146484375	26.0957275152722\\
-0.51416015625	26.1485632520104\\
-0.513671875	26.2015269245739\\
-0.51318359375	26.2546189854324\\
-0.5126953125	26.3078398896058\\
-0.51220703125	26.361190094682\\
-0.51171875	26.4146700608342\\
-0.51123046875	26.4682802508385\\
-0.5107421875	26.5220211300916\\
-0.51025390625	26.5758931666285\\
-0.509765625	26.6298968311406\\
-0.50927734375	26.6840325969937\\
-0.5087890625	26.7383009402461\\
-0.50830078125	26.792702339667\\
-0.5078125	26.8472372767548\\
-0.50732421875	26.9019062357559\\
-0.5068359375	26.9567097036829\\
-0.50634765625	27.0116481703339\\
-0.505859375	27.0667221283112\\
-0.50537109375	27.1219320730402\\
-0.5048828125	27.1772785027886\\
-0.50439453125	27.232761918686\\
-0.50390625	27.2883828247427\\
-0.50341796875	27.3441417278701\\
-0.5029296875	27.4000391378993\\
-0.50244140625	27.4560755676017\\
-0.501953125	27.5122515327085\\
-0.50146484375	27.5685675519308\\
-0.5009765625	27.6250241469799\\
-0.50048828125	27.6816218425873\\
-0.5	27.7383611665252\\
-0.49951171875	27.795242649627\\
-0.4990234375	27.852266825808\\
-0.49853515625	27.9094342320862\\
-0.498046875	27.9667454086027\\
-0.49755859375	28.0242008986436\\
-0.4970703125	28.0818012486601\\
-0.49658203125	28.1395470082905\\
-0.49609375	28.197438730381\\
-0.49560546875	28.2554769710078\\
-0.4951171875	28.3136622894979\\
-0.49462890625	28.3719952484515\\
-0.494140625	28.4304764137633\\
-0.49365234375	28.489106354645\\
-0.4931640625	28.5478856436468\\
-0.49267578125	28.6068148566798\\
-0.4921875	28.6658945730382\\
-0.49169921875	28.7251253754219\\
-0.4912109375	28.7845078499586\\
-0.49072265625	28.8440425862269\\
-0.490234375	28.9037301772785\\
-0.48974609375	28.9635712196614\\
-0.4892578125	29.0235663134427\\
-0.48876953125	29.0837160622312\\
-0.48828125	29.1440210732015\\
-0.48779296875	29.2044819571161\\
-0.4873046875	29.2650993283496\\
-0.48681640625	29.3258738049113\\
-0.486328125	29.3868060084696\\
-0.48583984375	29.4478965643747\\
-0.4853515625	29.509146101683\\
-0.48486328125	29.5705552531803\\
-0.484375	29.6321246554061\\
-0.48388671875	29.6938549486771\\
-0.4833984375	29.7557467771115\\
-0.48291015625	29.8178007886528\\
-0.482421875	29.8800176350943\\
-0.48193359375	29.942397972103\\
-0.4814453125	30.0049424592439\\
-0.48095703125	30.0676517600044\\
-0.48046875	30.1305265418187\\
-0.47998046875	30.1935674760923\\
-0.4794921875	30.256775238226\\
-0.47900390625	30.3201505076412\\
-0.478515625	30.3836939678037\\
-0.47802734375	30.4474063062489\\
-0.4775390625	30.511288214606\\
-0.47705078125	30.5753403886227\\
-0.4765625	30.6395635281901\\
-0.47607421875	30.7039583373673\\
-0.4755859375	30.7685255244057\\
-0.47509765625	30.8332658017743\\
-0.474609375	30.8981798861838\\
-0.47412109375	30.9632684986116\\
-0.4736328125	31.0285323643263\\
-0.47314453125	31.0939722129126\\
-0.47265625	31.1595887782952\\
-0.47216796875	31.2253827987642\\
-0.4716796875	31.2913550169988\\
-0.47119140625	31.3575061800926\\
-0.470703125	31.4238370395769\\
-0.47021484375	31.4903483514458\\
-0.4697265625	31.5570408761804\\
-0.46923828125	31.6239153787722\\
-0.46875	31.690972628748\\
-0.46826171875	31.7582134001931\\
-0.4677734375	31.8256384717755\\
-0.46728515625	31.8932486267694\\
-0.466796875	31.9610446530785\\
-0.46630859375	32.0290273432596\\
-0.4658203125	32.0971974945458\\
-0.46533203125	32.1655559088689\\
-0.46484375	32.2341033928832\\
-0.46435546875	32.302840757987\\
-0.4638671875	32.3717688203455\\
-0.46337890625	32.4408884009129\\
-0.462890625	32.5102003254541\\
-0.46240234375	32.5797054245662\\
-0.4619140625	32.6494045337001\\
-0.46142578125	32.719298493181\\
-0.4609375	32.7893881482295\\
-0.46044921875	32.8596743489818\\
-0.4599609375	32.9301579505094\\
-0.45947265625	33.0008398128389\\
-0.458984375	33.0717208009713\\
-0.45849609375	33.1428017849005\\
-0.4580078125	33.2140836396316\\
-0.45751953125	33.2855672451989\\
-0.45703125	33.3572534866834\\
-0.45654296875	33.4291432542292\\
-0.4560546875	33.5012374430602\\
-0.45556640625	33.5735369534957\\
-0.455078125	33.6460426909657\\
-0.45458984375	33.7187555660251\\
-0.4541015625	33.7916764943679\\
-0.45361328125	33.8648063968404\\
-0.453125	33.9381461994537\\
-0.45263671875	34.0116968333953\\
-0.4521484375	34.0854592350402\\
-0.45166015625	34.1594343459614\\
-0.451171875	34.2336231129387\\
-0.45068359375	34.3080264879679\\
-0.4501953125	34.3826454282675\\
-0.44970703125	34.4574808962863\\
-0.44921875	34.532533859708\\
-0.44873046875	34.607805291457\\
-0.4482421875	34.6832961697008\\
-0.44775390625	34.759007477853\\
-0.447265625	34.8349402045744\\
-0.44677734375	34.9110953437732\\
-0.4462890625	34.9874738946035\\
-0.44580078125	35.0640768614629\\
-0.4453125	35.1409052539887\\
-0.44482421875	35.2179600870521\\
-0.4443359375	35.2952423807521\\
-0.44384765625	35.3727531604065\\
-0.443359375	35.4504934565424\\
-0.44287109375	35.5284643048844\\
-0.4423828125	35.6066667463415\\
-0.44189453125	35.6851018269916\\
-0.44140625	35.7637705980651\\
-0.44091796875	35.8426741159254\\
-0.4404296875	35.9218134420487\\
-0.43994140625	36.0011896430005\\
-0.439453125	36.0808037904108\\
-0.43896484375	36.1606569609466\\
-0.4384765625	36.2407502362826\\
-0.43798828125	36.3210847030685\\
-0.4375	36.4016614528952\\
-0.43701171875	36.4824815822572\\
-0.4365234375	36.563546192513\\
-0.43603515625	36.6448563898421\\
-0.435546875	36.7264132852004\\
-0.43505859375	36.808217994271\\
-0.4345703125	36.8902716374127\\
-0.43408203125	36.9725753396056\\
-0.43359375	37.0551302303929\\
-0.43310546875	37.1379374438194\\
-0.4326171875	37.2209981183661\\
-0.43212890625	37.3043133968818\\
-0.431640625	37.3878844265106\\
-0.43115234375	37.4717123586146\\
-0.4306640625	37.5557983486941\\
-0.43017578125	37.6401435563021\\
-0.4296875	37.7247491449549\\
-0.42919921875	37.8096162820389\\
-0.4287109375	37.8947461387107\\
-0.42822265625	37.980139889795\\
-0.427734375	38.065798713675\\
-0.42724609375	38.1517237921787\\
-0.4267578125	38.23791631046\\
-0.42626953125	38.3243774568737\\
-0.42578125	38.4111084228444\\
-0.42529296875	38.4981104027302\\
-0.4248046875	38.5853845936796\\
-0.42431640625	38.672932195482\\
-0.423828125	38.7607544104117\\
-0.42333984375	38.8488524430649\\
-0.4228515625	38.9372275001894\\
-0.42236328125	39.0258807905074\\
-0.421875	39.11481352453\\
-0.42138671875	39.2040269143641\\
-0.4208984375	39.2935221735114\\
-0.42041015625	39.383300516658\\
-0.419921875	39.4733631594567\\
-0.41943359375	39.5637113182986\\
-0.4189453125	39.6543462100765\\
-0.41845703125	39.7452690519385\\
-0.41796875	39.8364810610303\\
-0.41748046875	39.9279834542291\\
-0.4169921875	40.0197774478652\\
-0.41650390625	40.1118642574331\\
-0.416015625	40.2042450972912\\
-0.41552734375	40.2969211803494\\
-0.4150390625	40.3898937177444\\
-0.41455078125	40.4831639185025\\
-0.4140625	40.5767329891892\\
-0.41357421875	40.6706021335453\\
-0.4130859375	40.7647725521083\\
-0.41259765625	40.8592454418207\\
-0.412109375	40.9540219956217\\
-0.41162109375	41.0491034020245\\
-0.4111328125	41.144490844677\\
-0.41064453125	41.2401855019063\\
-0.41015625	41.3361885462458\\
-0.40966796875	41.4325011439445\\
-0.4091796875	41.5291244544585\\
-0.40869140625	41.6260596299229\\
-0.408203125	41.7233078146047\\
-0.40771484375	41.8208701443352\\
-0.4072265625	41.9187477459217\\
-0.40673828125	42.0169417365372\\
-0.40625	42.1154532230886\\
-0.40576171875	42.214283301561\\
-0.4052734375	42.3134330563388\\
-0.40478515625	42.4129035595016\\
-0.404296875	42.5126958700958\\
-0.40380859375	42.6128110333785\\
-0.4033203125	42.7132500800354\\
-0.40283203125	42.8140140253702\\
-0.40234375	42.9151038684653\\
-0.40185546875	43.0165205913124\\
-0.4013671875	43.1182651579125\\
-0.40087890625	43.220338513344\\
-0.400390625	43.3227415827974\\
-0.39990234375	43.4254752705772\\
-0.3994140625	43.5285404590674\\
-0.39892578125	43.6319380076619\\
-0.3984375	43.7356687516568\\
-0.39794921875	43.8397335011044\\
-0.3974609375	43.944133039628\\
-0.39697265625	44.0488681231945\\
-0.396484375	44.1539394788453\\
-0.39599609375	44.2593478033838\\
-0.3955078125	44.3650937620173\\
-0.39501953125	44.4711779869523\\
-0.39453125	44.5776010759422\\
-0.39404296875	44.6843635907858\\
-0.3935546875	44.7914660557736\\
-0.39306640625	44.8989089560832\\
-0.392578125	45.0066927361189\\
-0.39208984375	45.1148177977968\\
-0.3916015625	45.223284498771\\
-0.39111328125	45.3320931506018\\
-0.390625	45.4412440168619\\
-0.39013671875	45.5507373111798\\
-0.3896484375	45.6605731952185\\
-0.38916015625	45.7707517765877\\
-0.388671875	45.8812731066871\\
-0.38818359375	45.9921371784784\\
-0.3876953125	46.1033439241855\\
-0.38720703125	46.2148932129196\\
-0.38671875	46.3267848482267\\
-0.38623046875	46.4390185655563\\
-0.3857421875	46.5515940296492\\
-0.38525390625	46.6645108318409\\
-0.384765625	46.7777684872795\\
-0.38427734375	46.8913664320548\\
-0.3837890625	47.0053040202378\\
-0.38330078125	47.119580520826\\
-0.3828125	47.2341951145936\\
-0.38232421875	47.3491468908443\\
-0.3818359375	47.4644348440625\\
-0.38134765625	47.5800578704629\\
-0.380859375	47.6960147644338\\
-0.38037109375	47.812304214872\\
-0.3798828125	47.928924801408\\
-0.37939453125	48.0458749905177\\
-0.37890625	48.163153131518\\
-0.37841796875	48.2807574524443\\
-0.3779296875	48.3986860558077\\
-0.37744140625	48.5169369142275\\
-0.376953125	48.6355078659394\\
-0.37646484375	48.7543966101737\\
-0.3759765625	48.873600702404\\
-0.37548828125	48.9931175494617\\
-0.375	49.1129444045154\\
-0.37451171875	49.2330783619128\\
-0.3740234375	49.3535163518812\\
-0.37353515625	49.4742551350883\\
-0.373046875	49.5952912970563\\
-0.37255859375	49.7166212424324\\
-0.3720703125	49.8382411891102\\
-0.37158203125	49.9601471622028\\
-0.37109375	50.0823349878657\\
-0.37060546875	50.2048002869666\\
-0.3701171875	50.3275384686052\\
-0.36962890625	50.4505447234768\\
-0.369140625	50.5738140170851\\
-0.36865234375	50.6973410827991\\
-0.3681640625	50.8211204147576\\
-0.36767578125	50.9451462606204\\
-0.3671875	51.0694126141667\\
-0.36669921875	51.1939132077441\\
-0.3662109375	51.3186415045671\\
-0.36572265625	51.4435906908698\\
-0.365234375	51.5687536679139\\
-0.36474609375	51.6941230438567\\
-0.3642578125	51.819691125482\\
-0.36376953125	51.9454499097987\\
-0.36328125	52.0713910755127\\
-0.36279296875	52.1975059743774\\
-0.3623046875	52.3237856224313\\
-0.36181640625	52.4502206911268\\
-0.361328125	52.5768014983639\\
-0.36083984375	52.7035179994324\\
-0.3603515625	52.8303597778779\\
-0.35986328125	52.9573160363\\
-0.359375	53.0843755870978\\
-0.35888671875	53.2115268431729\\
-0.3583984375	53.3387578086097\\
-0.35791015625	53.4660560693444\\
-0.357421875	53.5934087838448\\
-0.35693359375	53.7208026738163\\
-0.3564453125	53.8482240149572\\
-0.35595703125	53.9756586277843\\
-0.35546875	54.1030918685529\\
-0.35498046875	54.2305086202961\\
-0.3544921875	54.3578932840117\\
-0.35400390625	54.485229770024\\
-0.353515625	54.6125014895525\\
-0.35302734375	54.73969134652\\
-0.3525390625	54.8667817296328\\
-0.35205078125	54.993754504774\\
-0.3515625	55.1205910077417\\
-0.35107421875	55.2472720373793\\
-0.3505859375	55.3737778491352\\
-0.35009765625	55.5000881491007\\
-0.349609375	55.6261820885683\\
-0.34912109375	55.7520382591622\\
-0.3486328125	55.8776346885913\\
-0.34814453125	56.0029488370743\\
-0.34765625	56.1279575944946\\
-0.34716796875	56.2526372783393\\
-0.3466796875	56.3769636324794\\
-0.34619140625	56.5009118268522\\
-0.345703125	56.6244564581049\\
-0.34521484375	56.7475715512625\\
-0.3447265625	56.8702305624817\\
-0.34423828125	56.9924063829559\\
-0.34375	57.1140713440343\\
-0.34326171875	57.235197223621\\
-0.3427734375	57.3557552539173\\
-0.34228515625	57.4757161305728\\
-0.341796875	57.5950500233087\\
-0.34130859375	57.7137265880754\\
-0.3408203125	57.8317149808049\\
-0.34033203125	57.94898387282\\
-0.33984375	58.065501467953\\
-0.33935546875	58.1812355214309\\
-0.3388671875	58.2961533605758\\
-0.33837890625	58.4102219073676\\
-0.337890625	58.5234077029107\\
-0.33740234375	58.6356769338403\\
-0.3369140625	58.7469954607001\\
-0.33642578125	58.8573288483165\\
-0.3359375	58.9666423981835\\
-0.33544921875	59.074901182872\\
-0.3349609375	59.1820700824609\\
-0.33447265625	59.2881138229851\\
-0.333984375	59.3929970168827\\
-0.33349609375	59.4966842054124\\
-0.3330078125	59.5991399030059\\
-0.33251953125	59.7003286435041\\
-0.33203125	59.8002150282177\\
-0.33154296875	59.8987637757402\\
-0.3310546875	59.9959397734257\\
-0.33056640625	60.0917081304392\\
-0.330078125	60.1860342322644\\
-0.32958984375	60.2788837965506\\
-0.3291015625	60.3702229301587\\
-0.32861328125	60.4600181872648\\
-0.328125	60.5482366283551\\
-0.32763671875	60.6348458799445\\
-0.3271484375	60.7198141948332\\
-0.32666015625	60.8031105127089\\
-0.326171875	60.8847045208892\\
-0.32568359375	60.9645667149922\\
-0.3251953125	61.0426684593134\\
-0.32470703125	61.1189820466823\\
-0.32421875	61.1934807575614\\
-0.32373046875	61.266138918155\\
-0.3232421875	61.3369319572821\\
-0.32275390625	61.405836461773\\
-0.322265625	61.4728302301486\\
-0.32177734375	61.5378923243407\\
-0.3212890625	61.6010031192192\\
-0.32080078125	61.6621443496936\\
-0.3203125	61.7212991551671\\
-0.31982421875	61.778452121127\\
-0.3193359375	61.8335893176714\\
-0.31884765625	61.8866983347761\\
-0.318359375	61.937768314128\\
-0.31787109375	61.9867899773636\\
-0.3173828125	62.0337556505651\\
-0.31689453125	62.0786592848889\\
-0.31640625	62.1214964732213\\
-0.31591796875	62.162264462769\\
-0.3154296875	62.2009621635254\\
-0.31494140625	62.2375901525607\\
-0.314453125	62.2721506741201\\
-0.31396484375	62.3046476355247\\
-0.3134765625	62.3350865989\\
-0.31298828125	62.3634747687746\\
-0.3125	62.3898209756168\\
-0.31201171875	62.4141356553912\\
-0.3115234375	62.4364308252485\\
-0.31103515625	62.4567200554682\\
-0.310546875	62.4750184378037\\
-0.31005859375	62.4913425503884\\
-0.3095703125	62.5057104193805\\
-0.30908203125	62.5181414775368\\
-0.30859375	62.5286565199197\\
-0.30810546875	62.5372776569503\\
-0.3076171875	62.5440282650331\\
-0.30712890625	62.5489329349794\\
-0.306640625	62.5520174184686\\
-0.30615234375	62.5533085727848\\
-0.3056640625	62.5528343040728\\
-0.30517578125	62.5506235093531\\
-0.3046875	62.5467060175398\\
-0.30419921875	62.5411125296972\\
-0.3037109375	62.5338745587698\\
-0.30322265625	62.5250243690145\\
-0.302734375	62.5145949153554\\
-0.30224609375	62.5026197828752\\
-0.3017578125	62.489133126647\\
-0.30126953125	62.4741696121006\\
-0.30078125	62.457764356108\\
-0.30029296875	62.439952868959\\
-0.2998046875	62.4207709973877\\
-0.29931640625	62.4002548687984\\
-0.298828125	62.3784408368257\\
-0.29833984375	62.355365428353\\
-0.2978515625	62.3310652920983\\
-0.29736328125	62.3055771488649\\
-0.296875	62.2789377435436\\
-0.29638671875	62.2511837989374\\
-0.2958984375	62.2223519714707\\
-0.29541015625	62.1924788088315\\
-0.294921875	62.161600709585\\
-0.29443359375	62.1297538847876\\
-0.2939453125	62.0969743216147\\
-0.29345703125	62.063297749016\\
-0.29296875	62.0287596053915\\
-0.29248046875	61.9933950082829\\
-0.2919921875	61.9572387260614\\
-0.29150390625	61.9203251515876\\
-0.291015625	61.882688277815\\
-0.29052734375	61.8443616752976\\
-0.2900390625	61.8053784715629\\
-0.28955078125	61.7657713323011\\
-0.2890625	61.7255724443245\\
-0.28857421875	61.6848135002381\\
-0.2880859375	61.6435256847695\\
-0.28759765625	61.601739662696\\
-0.287109375	61.5594855683096\\
-0.28662109375	61.5167929963552\\
-0.2861328125	61.4736909943827\\
-0.28564453125	61.4302080564426\\
-0.28515625	61.3863721180656\\
-0.28466796875	61.3422105524603\\
-0.2841796875	61.2977501678631\\
-0.28369140625	61.2530172059774\\
-0.283203125	61.2080373414404\\
-0.28271484375	61.1628356822544\\
-0.2822265625	61.1174367711219\\
-0.28173828125	61.071864587626\\
-0.28125	61.0261425511977\\
-0.28076171875	60.9802935248147\\
-0.2802734375	60.9343398193772\\
-0.27978515625	60.8883031987076\\
-0.279296875	60.8422048851248\\
-0.27880859375	60.7960655655439\\
-0.2783203125	60.7499053980549\\
-0.27783203125	60.7037440189352\\
-0.27734375	60.6576005500547\\
-0.27685546875	60.6114936066324\\
-0.2763671875	60.5654413053046\\
-0.27587890625	60.5194612724715\\
-0.275390625	60.4735706528844\\
-0.27490234375	60.4277861184422\\
-0.2744140625	60.3821238771674\\
-0.27392578125	60.33659968233\\
-0.2734375	60.2912288416964\\
-0.27294921875	60.2460262268728\\
-0.2724609375	60.2010062827241\\
-0.27197265625	60.1561830368422\\
-0.271484375	60.1115701090472\\
-0.27099609375	60.0671807208978\\
-0.2705078125	60.0230277051973\\
-0.27001953125	59.979123515477\\
-0.26953125	59.9354802354414\\
-0.26904296875	59.8921095883637\\
-0.2685546875	59.849022946416\\
-0.26806640625	59.8062313399257\\
-0.267578125	59.7637454665459\\
-0.26708984375	59.7215757003308\\
-0.2666015625	59.6797321007069\\
-0.26611328125	59.6382244213352\\
-0.265625	59.597062118854\\
-0.26513671875	59.5562543614981\\
-0.2646484375	59.5158100375902\\
-0.26416015625	59.4757377638982\\
-0.263671875	59.436045893855\\
-0.26318359375	59.3967425256387\\
-0.2626953125	59.3578355101084\\
-0.26220703125	59.319332458596\\
-0.26171875	59.2812407505489\\
-0.26123046875	59.2435675410264\\
-0.2607421875	59.2063197680459\\
-0.26025390625	59.1695041597791\\
-0.259765625	59.1331272415987\\
-0.25927734375	59.0971953429759\\
-0.2587890625	59.0617146042284\\
-0.25830078125	59.0266909831198\\
-0.2578125	58.9921302613124\\
-0.25732421875	58.9580380506734\\
-0.2568359375	58.9244197994367\\
-0.25634765625	58.8912807982218\\
-0.255859375	58.858626185911\\
-0.25537109375	58.8264609553882\\
-0.2548828125	58.7947899591395\\
-0.25439453125	58.7636179147198\\
-0.25390625	58.7329494100855\\
-0.25341796875	58.7027889087979\\
-0.2529296875	58.6731407550983\\
-0.25244140625	58.6440091788571\\
-0.251953125	58.6153983004006\\
-0.25146484375	58.5873121352179\\
-0.2509765625	58.5597545985494\\
-0.25048828125	58.5327295098605\\
-0.25	58.5062405972032\\
-0.24951171875	58.4802915014688\\
-0.2490234375	58.454885780532\\
-0.24853515625	58.4300269132925\\
-0.248046875	58.4057183036151\\
-0.24755859375	58.3819632841691\\
-0.2470703125	58.3587651201748\\
-0.24658203125	58.3361270130537\\
-0.24609375	58.3140521039899\\
-0.24560546875	58.292543477402\\
-0.2451171875	58.2716041643311\\
-0.24462890625	58.2512371457444\\
-0.244140625	58.231445355759\\
-0.24365234375	58.2122316847886\\
-0.2431640625	58.1935989826131\\
-0.24267578125	58.1755500613773\\
-0.2421875	58.1580876985175\\
-0.24169921875	58.141214639621\\
-0.2412109375	58.1249336012191\\
-0.24072265625	58.1092472735174\\
-0.240234375	58.0941583230643\\
-0.23974609375	58.0796693953607\\
-0.2392578125	58.0657831174131\\
-0.23876953125	58.0525021002318\\
-0.23828125	58.0398289412769\\
-0.23779296875	58.0277662268539\\
-0.2373046875	58.0163165344608\\
-0.23681640625	58.0054824350881\\
-0.236328125	57.9952664954765\\
-0.23583984375	57.9856712803297\\
-0.2353515625	57.9766993544885\\
-0.23486328125	57.9683532850649\\
-0.234375	57.9606356435406\\
-0.23388671875	57.9535490078293\\
-0.2333984375	57.9470959643068\\
-0.23291015625	57.9412791098087\\
-0.232421875	57.9361010535987\\
-0.23193359375	57.9315644193087\\
-0.2314453125	57.9276718468529\\
-0.23095703125	57.9244259943158\\
-0.23046875	57.9218295398184\\
-0.22998046875	57.9198851833619\\
-0.2294921875	57.9185956486513\\
-0.22900390625	57.9179636849009\\
-0.228515625	57.9179920686221\\
-0.22802734375	57.918683605396\\
-0.2275390625	57.9200411316315\\
-0.22705078125	57.9220675163116\\
-0.2265625	57.9247656627271\\
-0.22607421875	57.9281385102016\\
-0.2255859375	57.9321890358074\\
-0.22509765625	57.9369202560748\\
-0.224609375	57.942335228695\\
-0.22412109375	57.9484370542195\\
-0.2236328125	57.955228877756\\
-0.22314453125	57.9627138906629\\
-0.22265625	57.9708953322427\\
-0.22216796875	57.9797764914379\\
-0.2216796875	57.9893607085276\\
-0.22119140625	57.9996513768285\\
-0.220703125	58.0106519444009\\
-0.22021484375	58.0223659157604\\
-0.2197265625	58.0347968535977\\
-0.21923828125	58.0479483805061\\
-0.21875	58.0618241807206\\
-0.21826171875	58.076428001867\\
-0.2177734375	58.0917636567243\\
-0.21728515625	58.1078350250008\\
-0.216796875	58.1246460551258\\
-0.21630859375	58.1422007660573\\
-0.2158203125	58.1605032491075\\
-0.21533203125	58.1795576697883\\
-0.21484375	58.1993682696764\\
-0.21435546875	58.2199393683008\\
-0.2138671875	58.2412753650522\\
-0.21337890625	58.2633807411189\\
-0.212890625	58.2862600614465\\
-0.21240234375	58.3099179767266\\
-0.2119140625	58.3343592254122\\
-0.21142578125	58.3595886357646\\
-0.2109375	58.3856111279312\\
-0.21044921875	58.412431716056\\
-0.2099609375	58.4400555104233\\
-0.20947265625	58.4684877196404\\
-0.208984375	58.4977336528529\\
-0.20849609375	58.5277987220025\\
-0.2080078125	58.5586884441227\\
-0.20751953125	58.5904084436774\\
-0.20703125	58.6229644549419\\
-0.20654296875	58.6563623244296\\
-0.2060546875	58.6906080133643\\
-0.20556640625	58.7257076002006\\
-0.205078125	58.7616672831948\\
-0.20458984375	58.7984933830262\\
-0.2041015625	58.8361923454713\\
-0.20361328125	58.8747707441336\\
-0.203125	58.9142352832285\\
-0.20263671875	58.9545928004277\\
-0.2021484375	58.995850269762\\
-0.20166015625	59.0380148045874\\
-0.201171875	59.0810936606126\\
-0.20068359375	59.1250942389935\\
-0.2001953125	59.1700240894942\\
-0.19970703125	59.2158909137164\\
-0.19921875	59.2627025683997\\
-0.19873046875	59.3104670687955\\
-0.1982421875	59.3591925921137\\
-0.19775390625	59.408887481047\\
-0.197265625	59.4595602473726\\
-0.19677734375	59.511219575635\\
-0.1962890625	59.5638743269092\\
-0.19580078125	59.6175335426487\\
-0.1953125	59.6722064486189\\
-0.19482421875	59.7279024589177\\
-0.1943359375	59.7846311800851\\
-0.19384765625	59.8424024153038\\
-0.193359375	59.901226168692\\
-0.19287109375	59.9611126496912\\
-0.1923828125	60.0220722775479\\
-0.19189453125	60.0841156858943\\
-0.19140625	60.1472537274254\\
-0.19091796875	60.2114974786779\\
-0.1904296875	60.2768582449077\\
-0.18994140625	60.3433475650701\\
-0.189453125	60.4109772169026\\
-0.18896484375	60.4797592221098\\
-0.1884765625	60.549705851652\\
-0.18798828125	60.6208296311371\\
-0.1875	60.6931433463161\\
-0.18701171875	60.7666600486793\\
-0.1865234375	60.8413930611563\\
-0.18603515625	60.9173559839157\\
-0.185546875	60.9945627002619\\
-0.18505859375	61.0730273826299\\
-0.1845703125	61.1527644986723\\
-0.18408203125	61.2337888174361\\
-0.18359375	61.3161154156262\\
-0.18310546875	61.3997596839472\\
-0.1826171875	61.4847373335217\\
-0.18212890625	61.5710644023752\\
-0.181640625	61.6587572619806\\
-0.18115234375	61.7478326238537\\
-0.1806640625	61.8383075461881\\
-0.18017578125	61.9301994405168\\
-0.1796875	62.0235260783896\\
-0.17919921875	62.118305598047\\
-0.1787109375	62.2145565110764\\
-0.17822265625	62.3122977090271\\
-0.177734375	62.4115484699658\\
-0.17724609375	62.5123284649436\\
-0.1767578125	62.6146577643485\\
-0.17626953125	62.7185568441108\\
-0.17578125	62.8240465917271\\
-0.17529296875	62.9311483120602\\
-0.1748046875	63.0398837328768\\
-0.17431640625	63.150275010064\\
-0.173828125	63.2623447324801\\
-0.17333984375	63.3761159263715\\
-0.1728515625	63.4916120592891\\
-0.17236328125	63.6088570434284\\
-0.171875	63.7278752383104\\
-0.17138671875	63.8486914527061\\
-0.1708984375	63.971330945703\\
-0.17041015625	64.0958194267971\\
-0.169921875	64.2221830548813\\
-0.16943359375	64.3504484359862\\
-0.1689453125	64.4806426196186\\
-0.16845703125	64.6127930935169\\
-0.16796875	64.746927776632\\
-0.16748046875	64.8830750101154\\
-0.1669921875	65.0212635460746\\
-0.16650390625	65.1615225338303\\
-0.166015625	65.3038815033798\\
-0.16552734375	65.4483703457381\\
-0.1650390625	65.5950192897961\\
-0.16455078125	65.7438588752923\\
-0.1640625	65.8949199214517\\
-0.16357421875	66.0482334907986\\
-0.1630859375	66.2038308475959\\
-0.16259765625	66.3617434103035\\
-0.162109375	66.5220026973839\\
-0.16162109375	66.6846402657096\\
-0.1611328125	66.8496876407492\\
-0.16064453125	67.0171762376189\\
-0.16015625	67.1871372719849\\
-0.15966796875	67.3596016597045\\
-0.1591796875	67.534599903962\\
-0.15869140625	67.712161968536\\
-0.158203125	67.892317135682\\
-0.15771484375	68.0750938469616\\
-0.1572265625	68.2605195251747\\
-0.15673828125	68.4486203753561\\
-0.15625	68.6394211626036\\
-0.15576171875	68.8329449642576\\
-0.1552734375	69.0292128937301\\
-0.15478515625	69.2282437929966\\
-0.154296875	69.4300538904846\\
-0.15380859375	69.6346564207895\\
-0.1533203125	69.8420612023101\\
-0.15283203125	70.0522741685548\\
-0.15234375	70.2652968485077\\
-0.15185546875	70.4811257910533\\
-0.1513671875	70.6997519280804\\
-0.15087890625	70.9211598704933\\
-0.150390625	71.1453271309617\\
-0.14990234375	71.3722232668899\\
-0.1494140625	71.601808936742\\
-0.14892578125	71.8340348625874\\
-0.1484375	72.0688406915304\\
-0.14794921875	72.3061537486107\\
-0.1474609375	72.5458876738325\\
-0.14697265625	72.7879409362513\\
-0.146484375	73.0321952186075\\
-0.14599609375	73.2785136668597\\
-0.1455078125	73.5267390003071\\
-0.14501953125	73.7766914798093\\
-0.14453125	74.028166734124\\
-0.14404296875	74.280933447645\\
-0.1435546875	74.5347309170285\\
-0.14306640625	74.7892664894988\\
-0.142578125	75.0442129021695\\
-0.14208984375	75.2992055496968\\
-0.1416015625	75.5538397171142\\
-0.14111328125	75.8076678259263\\
-0.140625	76.0601967544955\\
-0.14013671875	76.3108853084119\\
-0.1396484375	76.5591419327286\\
-0.13916015625	76.8043227753231\\
-0.138671875	77.0457302286688\\
-0.13818359375	77.2826120950492\\
-0.1376953125	77.5141615366339\\
-0.13720703125	77.7395179852403\\
-0.13671875	77.9577691951226\\
-0.13623046875	78.1679546235609\\
-0.1357421875	78.3690703157796\\
-0.13525390625	78.5600754503013\\
-0.134765625	78.7399006658641\\
-0.13427734375	78.907458239581\\
-0.1337890625	79.0616541173061\\
-0.13330078125	79.2014017117251\\
-0.1328125	79.3256372838491\\
-0.13232421875	79.4333366140917\\
-0.1318359375	79.5235325567213\\
-0.13134765625	79.5953329651727\\
-0.130859375	79.6479383859131\\
-0.13037109375	79.6806588559585\\
-0.1298828125	79.6929291136764\\
-0.12939453125	79.684321551554\\
-0.12890625	79.6545563066245\\
-0.12841796875	79.6035079979456\\
-0.1279296875	79.5312087741746\\
-0.12744140625	79.4378475164129\\
-0.126953125	79.3237652369616\\
-0.12646484375	79.1894469065781\\
-0.1259765625	79.0355101148577\\
-0.12548828125	78.8626911066538\\
-0.125	78.6718288323639\\
-0.12451171875	78.4638476970003\\
-0.1240234375	78.2397396929331\\
-0.12353515625	78.0005465593245\\
-0.123046875	77.7473425361824\\
-0.12255859375	77.4812181830025\\
-0.1220703125	77.2032656219995\\
-0.12158203125	76.9145654537127\\
-0.12109375	76.616175486589\\
-0.12060546875	76.3091213278155\\
-0.1201171875	75.9943888039478\\
-0.11962890625	75.672918118337\\
-0.119140625	75.3455996078834\\
-0.11865234375	75.013270932929\\
-0.1181640625	74.6767155190216\\
-0.11767578125	74.3366620653596\\
-0.1171875	73.9937849393485\\
-0.11669921875	73.6487052875221\\
-0.1162109375	73.3019927079115\\
-0.11572265625	72.9541673460578\\
-0.115234375	72.6057022948669\\
-0.11474609375	72.2570261963292\\
-0.1142578125	71.9085259600825\\
-0.11376953125	71.560549529374\\
-0.11328125	71.2134086389577\\
-0.11279296875	70.8673815216903\\
-0.1123046875	70.522715531112\\
-0.11181640625	70.1796296561762\\
-0.111328125	69.838316911656\\
-0.11083984375	69.4989465937712\\
-0.1103515625	69.1616663954063\\
-0.10986328125	68.8266043790896\\
-0.109375	68.493870808836\\
-0.10888671875	68.1635598441692\\
-0.1083984375	67.8357511012436\\
-0.10791015625	67.5105110871149\\
-0.107421875	67.1878945139422\\
-0.10693359375	66.8679455003277\\
-0.1064453125	66.5506986671906\\
-0.10595703125	66.236180135575\\
-0.10546875	65.9244084336514\\
-0.10498046875	65.6153953199478\\
-0.1044921875	65.3091465295348\\
-0.10400390625	65.005662449544\\
-0.103515625	64.7049387300237\\
-0.10302734375	64.406966835744\\
-0.1025390625	64.1117345441729\\
-0.10205078125	63.8192263944617\\
-0.1015625	63.529424091899\\
-0.10107421875	63.242306871941\\
-0.1005859375	62.9578518275804\\
-0.10009765625	62.6760342034971\\
-0.099609375	62.3968276601417\\
-0.09912109375	62.1202045106109\\
-0.0986328125	61.8461359329329\\
-0.09814453125	61.574592160128\\
-0.09765625	61.3055426502039\\
-0.09716796875	61.0389562380343\\
-0.0966796875	60.7748012708966\\
-0.09619140625	60.5130457292673\\
-0.095703125	60.2536573343305\\
-0.09521484375	59.9966036435086\\
-0.0947265625	59.7418521352066\\
-0.09423828125	59.4893702838381\\
-0.09375	59.2391256261054\\
-0.09326171875	58.99108581941\\
-0.0927734375	58.7452186931798\\
-0.09228515625	58.5014922938321\\
-0.091796875	58.2598749240134\\
-0.09130859375	58.0203351766983\\
-0.0908203125	57.7828419646733\\
-0.09033203125	57.5473645458787\\
-0.08984375	57.3138725450339\\
-0.08935546875	57.082335971936\\
-0.0888671875	56.8527252367743\\
-0.08837890625	56.6250111627791\\
-0.087890625	56.3991649964834\\
-0.08740234375	56.1751584158571\\
-0.0869140625	55.9529635365384\\
-0.08642578125	55.7325529163764\\
-0.0859375	55.5138995584654\\
-0.08544921875	55.296976912843\\
-0.0849609375	55.0817588770049\\
-0.08447265625	54.8682197953694\\
-0.083984375	54.6563344578179\\
-0.08349609375	54.4460780974218\\
-0.0830078125	54.237426387455\\
-0.08251953125	54.0303554377824\\
-0.08203125	53.8248417907053\\
-0.08154296875	53.6208624163363\\
-0.0810546875	53.4183947075687\\
-0.08056640625	53.2174164747002\\
-0.080078125	53.0179059397624\\
-0.07958984375	52.8198417306032\\
-0.0791015625	52.6232028747662\\
-0.07861328125	52.4279687932028\\
-0.078125	52.2341192938519\\
-0.07763671875	52.0416345651184\\
-0.0771484375	51.8504951692748\\
-0.07666015625	51.660682035814\\
-0.076171875	51.4721764547701\\
-0.07568359375	51.2849600700299\\
-0.0751953125	51.0990148726501\\
-0.07470703125	50.9143231941955\\
-0.07421875	50.7308677001107\\
-0.07373046875	50.5486313831384\\
-0.0732421875	50.367597556792\\
-0.07275390625	50.1877498488941\\
-0.072265625	50.009072195185\\
-0.07177734375	49.8315488330116\\
-0.0712890625	49.6551642950996\\
-0.07080078125	49.479903403414\\
-0.0703125	49.3057512631138\\
-0.06982421875	49.1326932566014\\
-0.0693359375	48.9607150376724\\
-0.06884765625	48.7898025257654\\
-0.068359375	48.6199419003148\\
-0.06787109375	48.4511195952076\\
-0.0673828125	48.2833222933451\\
-0.06689453125	48.1165369213093\\
-0.06640625	47.9507506441359\\
-0.06591796875	47.7859508601907\\
-0.0654296875	47.6221251961536\\
-0.06494140625	47.4592615021043\\
-0.064453125	47.2973478467135\\
-0.06396484375	47.1363725125366\\
-0.0634765625	46.9763239914084\\
-0.06298828125	46.8171909799394\\
-0.0625	46.65896237511\\
-0.06201171875	46.5016272699643\\
-0.0615234375	46.3451749493986\\
-0.06103515625	46.1895948860464\\
-0.060546875	46.0348767362553\\
-0.06005859375	45.8810103361567\\
-0.0595703125	45.7279856978251\\
-0.05908203125	45.5757930055258\\
-0.05859375	45.4244226120496\\
-0.05810546875	45.2738650351315\\
-0.0576171875	45.1241109539543\\
-0.05712890625	44.9751512057321\\
-0.056640625	44.8269767823747\\
-0.05615234375	44.6795788272292\\
-0.0556640625	44.532948631899\\
-0.05517578125	44.3870776331369\\
-0.0546875	44.2419574098106\\
-0.05419921875	44.097579679941\\
-0.0537109375	43.9539362978091\\
-0.05322265625	43.8110192511308\\
-0.052734375	43.6688206582998\\
-0.05224609375	43.527332765693\\
-0.0517578125	43.386547945042\\
-0.05126953125	43.2464586908637\\
-0.05078125	43.1070576179533\\
-0.05029296875	42.9683374589358\\
-0.0498046875	42.8302910618738\\
-0.04931640625	42.6929113879331\\
-0.048828125	42.5561915091017\\
-0.04833984375	42.4201246059625\\
-0.0478515625	42.2847039655186\\
-0.04736328125	42.1499229790679\\
-0.046875	42.015775140129\\
-0.04638671875	41.8822540424138\\
-0.0458984375	41.749353377848\\
-0.04541015625	41.6170669346371\\
-0.044921875	41.4853885953768\\
-0.04443359375	41.354312335208\\
-0.0439453125	41.2238322200133\\
-0.04345703125	41.0939424046549\\
-0.04296875	40.9646371312538\\
-0.04248046875	40.8359107275072\\
-0.0419921875	40.7077576050455\\
-0.04150390625	40.5801722578251\\
-0.041015625	40.4531492605595\\
-0.04052734375	40.3266832671846\\
-0.0400390625	40.200769009359\\
-0.03955078125	40.0754012949985\\
-0.0390625	39.9505750068427\\
-0.03857421875	39.8262851010548\\
-0.0380859375	39.7025266058514\\
-0.03759765625	39.5792946201644\\
-0.037109375	39.4565843123308\\
-0.03662109375	39.3343909188133\\
-0.0361328125	39.2127097429479\\
-0.03564453125	39.0915361537197\\
-0.03515625	38.9708655845654\\
-0.03466796875	38.8506935322022\\
-0.0341796875	38.7310155554819\\
-0.03369140625	38.6118272742706\\
-0.033203125	38.4931243683519\\
-0.03271484375	38.3749025763546\\
-0.0322265625	38.2571576947031\\
-0.03173828125	38.1398855765903\\
-0.03125	38.0230821309731\\
-0.03076171875	37.9067433215884\\
-0.0302734375	37.7908651659914\\
-0.02978515625	37.6754437346129\\
-0.029296875	37.5604751498377\\
-0.02880859375	37.4459555851018\\
-0.0283203125	37.331881264009\\
-0.02783203125	37.2182484594656\\
-0.02734375	37.1050534928336\\
-0.02685546875	36.9922927331016\\
-0.0263671875	36.879962596072\\
-0.02587890625	36.7680595435668\\
-0.025390625	36.6565800826479\\
-0.02490234375	36.5455207648549\\
-0.0244140625	36.4348781854574\\
-0.02392578125	36.3246489827236\\
-0.0234375	36.2148298372032\\
-0.02294921875	36.1054174710248\\
-0.0224609375	35.9964086472081\\
-0.02197265625	35.8878001689892\\
-0.021484375	35.7795888791604\\
-0.02099609375	35.6717716594221\\
-0.0205078125	35.564345429749\\
-0.02001953125	35.4573071477676\\
-0.01953125	35.3506538081471\\
-0.01904296875	35.2443824420015\\
-0.0185546875	35.1384901163045\\
-0.01806640625	35.0329739333147\\
-0.017578125	34.9278310300131\\
-0.01708984375	34.8230585775512\\
-0.0166015625	34.7186537807095\\
-0.01611328125	34.6146138773673\\
-0.015625	34.5109361379817\\
-0.01513671875	34.4076178650779\\
-0.0146484375	34.304656392748\\
-0.01416015625	34.2020490861606\\
-0.013671875	34.0997933410789\\
-0.01318359375	33.997886583388\\
-0.0126953125	33.896326268632\\
-0.01220703125	33.795109881559\\
-0.01171875	33.694234935675\\
-0.01123046875	33.5936989728058\\
-0.0107421875	33.4934995626682\\
-0.01025390625	33.3936343024475\\
-0.009765625	33.2941008163847\\
-0.00927734375	33.1948967553697\\
-0.0087890625	33.0960197965432\\
-0.00830078125	32.9974676429056\\
-0.0078125	32.8992380229324\\
-0.00732421875	32.8013286901978\\
-0.0068359375	32.7037374230044\\
-0.00634765625	32.6064620240198\\
-0.005859375	32.5095003199196\\
-0.00537109375	32.4128501610375\\
-0.0048828125	32.3165094210209\\
-0.00439453125	32.2204759964931\\
-0.00390625	32.1247478067216\\
-0.00341796875	32.0293227932921\\
-0.0029296875	31.9341989197883\\
-0.00244140625	31.8393741714778\\
-0.001953125	31.7448465550028\\
-0.00146484375	31.6506140980769\\
-0.0009765625	31.556674849187\\
-0.00048828125	31.4630268773002\\
0	31.3696682715762\\
0.00048828125	31.4630268773002\\
0.0009765625	31.556674849187\\
0.00146484375	31.6506140980769\\
0.001953125	31.7448465550028\\
0.00244140625	31.8393741714778\\
0.0029296875	31.9341989197883\\
0.00341796875	32.0293227932921\\
0.00390625	32.1247478067216\\
0.00439453125	32.2204759964931\\
0.0048828125	32.3165094210209\\
0.00537109375	32.4128501610375\\
0.005859375	32.5095003199196\\
0.00634765625	32.6064620240198\\
0.0068359375	32.7037374230044\\
0.00732421875	32.8013286901978\\
0.0078125	32.8992380229324\\
0.00830078125	32.9974676429056\\
0.0087890625	33.0960197965432\\
0.00927734375	33.1948967553697\\
0.009765625	33.2941008163847\\
0.01025390625	33.3936343024475\\
0.0107421875	33.4934995626682\\
0.01123046875	33.5936989728058\\
0.01171875	33.694234935675\\
0.01220703125	33.795109881559\\
0.0126953125	33.896326268632\\
0.01318359375	33.997886583388\\
0.013671875	34.0997933410789\\
0.01416015625	34.2020490861606\\
0.0146484375	34.304656392748\\
0.01513671875	34.4076178650779\\
0.015625	34.5109361379817\\
0.01611328125	34.6146138773673\\
0.0166015625	34.7186537807095\\
0.01708984375	34.8230585775512\\
0.017578125	34.9278310300131\\
0.01806640625	35.0329739333147\\
0.0185546875	35.1384901163045\\
0.01904296875	35.2443824420015\\
0.01953125	35.3506538081471\\
0.02001953125	35.4573071477676\\
0.0205078125	35.564345429749\\
0.02099609375	35.6717716594221\\
0.021484375	35.7795888791604\\
0.02197265625	35.8878001689892\\
0.0224609375	35.9964086472081\\
0.02294921875	36.1054174710248\\
0.0234375	36.2148298372032\\
0.02392578125	36.3246489827236\\
0.0244140625	36.4348781854574\\
0.02490234375	36.5455207648549\\
0.025390625	36.6565800826479\\
0.02587890625	36.7680595435668\\
0.0263671875	36.879962596072\\
0.02685546875	36.9922927331016\\
0.02734375	37.1050534928336\\
0.02783203125	37.2182484594656\\
0.0283203125	37.331881264009\\
0.02880859375	37.4459555851018\\
0.029296875	37.5604751498377\\
0.02978515625	37.6754437346129\\
0.0302734375	37.7908651659914\\
0.03076171875	37.9067433215884\\
0.03125	38.0230821309731\\
0.03173828125	38.1398855765903\\
0.0322265625	38.2571576947031\\
0.03271484375	38.3749025763546\\
0.033203125	38.4931243683519\\
0.03369140625	38.6118272742706\\
0.0341796875	38.7310155554819\\
0.03466796875	38.8506935322022\\
0.03515625	38.9708655845654\\
0.03564453125	39.0915361537197\\
0.0361328125	39.2127097429479\\
0.03662109375	39.3343909188133\\
0.037109375	39.4565843123308\\
0.03759765625	39.5792946201644\\
0.0380859375	39.7025266058514\\
0.03857421875	39.8262851010548\\
0.0390625	39.9505750068427\\
0.03955078125	40.0754012949985\\
0.0400390625	40.200769009359\\
0.04052734375	40.3266832671846\\
0.041015625	40.4531492605595\\
0.04150390625	40.5801722578251\\
0.0419921875	40.7077576050455\\
0.04248046875	40.8359107275072\\
0.04296875	40.9646371312538\\
0.04345703125	41.0939424046549\\
0.0439453125	41.2238322200133\\
0.04443359375	41.354312335208\\
0.044921875	41.4853885953768\\
0.04541015625	41.6170669346371\\
0.0458984375	41.749353377848\\
0.04638671875	41.8822540424138\\
0.046875	42.015775140129\\
0.04736328125	42.1499229790679\\
0.0478515625	42.2847039655186\\
0.04833984375	42.4201246059625\\
0.048828125	42.5561915091017\\
0.04931640625	42.6929113879331\\
0.0498046875	42.8302910618738\\
0.05029296875	42.9683374589358\\
0.05078125	43.1070576179533\\
0.05126953125	43.2464586908637\\
0.0517578125	43.386547945042\\
0.05224609375	43.527332765693\\
0.052734375	43.6688206582998\\
0.05322265625	43.8110192511308\\
0.0537109375	43.9539362978091\\
0.05419921875	44.097579679941\\
0.0546875	44.2419574098106\\
0.05517578125	44.3870776331369\\
0.0556640625	44.532948631899\\
0.05615234375	44.6795788272292\\
0.056640625	44.8269767823747\\
0.05712890625	44.9751512057321\\
0.0576171875	45.1241109539543\\
0.05810546875	45.2738650351315\\
0.05859375	45.4244226120496\\
0.05908203125	45.5757930055258\\
0.0595703125	45.7279856978251\\
0.06005859375	45.8810103361567\\
0.060546875	46.0348767362553\\
0.06103515625	46.1895948860464\\
0.0615234375	46.3451749493986\\
0.06201171875	46.5016272699643\\
0.0625	46.65896237511\\
0.06298828125	46.8171909799394\\
0.0634765625	46.9763239914084\\
0.06396484375	47.1363725125366\\
0.064453125	47.2973478467135\\
0.06494140625	47.4592615021043\\
0.0654296875	47.6221251961536\\
0.06591796875	47.7859508601907\\
0.06640625	47.9507506441359\\
0.06689453125	48.1165369213093\\
0.0673828125	48.2833222933451\\
0.06787109375	48.4511195952076\\
0.068359375	48.6199419003148\\
0.06884765625	48.7898025257654\\
0.0693359375	48.9607150376724\\
0.06982421875	49.1326932566014\\
0.0703125	49.3057512631138\\
0.07080078125	49.479903403414\\
0.0712890625	49.6551642950996\\
0.07177734375	49.8315488330116\\
0.072265625	50.009072195185\\
0.07275390625	50.1877498488941\\
0.0732421875	50.367597556792\\
0.07373046875	50.5486313831384\\
0.07421875	50.7308677001107\\
0.07470703125	50.9143231941955\\
0.0751953125	51.0990148726501\\
0.07568359375	51.2849600700299\\
0.076171875	51.4721764547701\\
0.07666015625	51.660682035814\\
0.0771484375	51.8504951692748\\
0.07763671875	52.0416345651184\\
0.078125	52.2341192938519\\
0.07861328125	52.4279687932028\\
0.0791015625	52.6232028747662\\
0.07958984375	52.8198417306032\\
0.080078125	53.0179059397624\\
0.08056640625	53.2174164747002\\
0.0810546875	53.4183947075687\\
0.08154296875	53.6208624163363\\
0.08203125	53.8248417907053\\
0.08251953125	54.0303554377824\\
0.0830078125	54.237426387455\\
0.08349609375	54.4460780974218\\
0.083984375	54.6563344578179\\
0.08447265625	54.8682197953694\\
0.0849609375	55.0817588770049\\
0.08544921875	55.296976912843\\
0.0859375	55.5138995584654\\
0.08642578125	55.7325529163764\\
0.0869140625	55.9529635365384\\
0.08740234375	56.1751584158571\\
0.087890625	56.3991649964834\\
0.08837890625	56.6250111627791\\
0.0888671875	56.8527252367743\\
0.08935546875	57.082335971936\\
0.08984375	57.3138725450339\\
0.09033203125	57.5473645458787\\
0.0908203125	57.7828419646733\\
0.09130859375	58.0203351766983\\
0.091796875	58.2598749240134\\
0.09228515625	58.5014922938321\\
0.0927734375	58.7452186931798\\
0.09326171875	58.99108581941\\
0.09375	59.2391256261054\\
0.09423828125	59.4893702838381\\
0.0947265625	59.7418521352066\\
0.09521484375	59.9966036435086\\
0.095703125	60.2536573343305\\
0.09619140625	60.5130457292673\\
0.0966796875	60.7748012708966\\
0.09716796875	61.0389562380343\\
0.09765625	61.3055426502039\\
0.09814453125	61.574592160128\\
0.0986328125	61.8461359329329\\
0.09912109375	62.1202045106109\\
0.099609375	62.3968276601417\\
0.10009765625	62.6760342034971\\
0.1005859375	62.9578518275804\\
0.10107421875	63.242306871941\\
0.1015625	63.529424091899\\
0.10205078125	63.8192263944617\\
0.1025390625	64.1117345441729\\
0.10302734375	64.406966835744\\
0.103515625	64.7049387300237\\
0.10400390625	65.005662449544\\
0.1044921875	65.3091465295348\\
0.10498046875	65.6153953199478\\
0.10546875	65.9244084336514\\
0.10595703125	66.236180135575\\
0.1064453125	66.5506986671906\\
0.10693359375	66.8679455003277\\
0.107421875	67.1878945139422\\
0.10791015625	67.5105110871149\\
0.1083984375	67.8357511012436\\
0.10888671875	68.1635598441692\\
0.109375	68.493870808836\\
0.10986328125	68.8266043790896\\
0.1103515625	69.1616663954063\\
0.11083984375	69.4989465937712\\
0.111328125	69.838316911656\\
0.11181640625	70.1796296561762\\
0.1123046875	70.522715531112\\
0.11279296875	70.8673815216903\\
0.11328125	71.2134086389577\\
0.11376953125	71.560549529374\\
0.1142578125	71.9085259600825\\
0.11474609375	72.2570261963292\\
0.115234375	72.6057022948669\\
0.11572265625	72.9541673460578\\
0.1162109375	73.3019927079115\\
0.11669921875	73.6487052875221\\
0.1171875	73.9937849393485\\
0.11767578125	74.3366620653596\\
0.1181640625	74.6767155190216\\
0.11865234375	75.013270932929\\
0.119140625	75.3455996078834\\
0.11962890625	75.672918118337\\
0.1201171875	75.9943888039478\\
0.12060546875	76.3091213278155\\
0.12109375	76.616175486589\\
0.12158203125	76.9145654537127\\
0.1220703125	77.2032656219995\\
0.12255859375	77.4812181830025\\
0.123046875	77.7473425361824\\
0.12353515625	78.0005465593245\\
0.1240234375	78.2397396929331\\
0.12451171875	78.4638476970003\\
0.125	78.6718288323639\\
0.12548828125	78.8626911066538\\
0.1259765625	79.0355101148577\\
0.12646484375	79.1894469065781\\
0.126953125	79.3237652369616\\
0.12744140625	79.4378475164129\\
0.1279296875	79.5312087741746\\
0.12841796875	79.6035079979456\\
0.12890625	79.6545563066245\\
0.12939453125	79.684321551554\\
0.1298828125	79.6929291136764\\
0.13037109375	79.6806588559585\\
0.130859375	79.6479383859131\\
0.13134765625	79.5953329651727\\
0.1318359375	79.5235325567213\\
0.13232421875	79.4333366140917\\
0.1328125	79.3256372838491\\
0.13330078125	79.2014017117251\\
0.1337890625	79.0616541173061\\
0.13427734375	78.907458239581\\
0.134765625	78.7399006658641\\
0.13525390625	78.5600754503013\\
0.1357421875	78.3690703157796\\
0.13623046875	78.1679546235609\\
0.13671875	77.9577691951226\\
0.13720703125	77.7395179852403\\
0.1376953125	77.5141615366339\\
0.13818359375	77.2826120950492\\
0.138671875	77.0457302286688\\
0.13916015625	76.8043227753231\\
0.1396484375	76.5591419327286\\
0.14013671875	76.3108853084119\\
0.140625	76.0601967544955\\
0.14111328125	75.8076678259263\\
0.1416015625	75.5538397171142\\
0.14208984375	75.2992055496968\\
0.142578125	75.0442129021695\\
0.14306640625	74.7892664894988\\
0.1435546875	74.5347309170285\\
0.14404296875	74.280933447645\\
0.14453125	74.028166734124\\
0.14501953125	73.7766914798093\\
0.1455078125	73.5267390003071\\
0.14599609375	73.2785136668597\\
0.146484375	73.0321952186075\\
0.14697265625	72.7879409362513\\
0.1474609375	72.5458876738325\\
0.14794921875	72.3061537486107\\
0.1484375	72.0688406915304\\
0.14892578125	71.8340348625874\\
0.1494140625	71.601808936742\\
0.14990234375	71.3722232668899\\
0.150390625	71.1453271309617\\
0.15087890625	70.9211598704933\\
0.1513671875	70.6997519280804\\
0.15185546875	70.4811257910533\\
0.15234375	70.2652968485077\\
0.15283203125	70.0522741685548\\
0.1533203125	69.8420612023101\\
0.15380859375	69.6346564207895\\
0.154296875	69.4300538904846\\
0.15478515625	69.2282437929966\\
0.1552734375	69.0292128937301\\
0.15576171875	68.8329449642576\\
0.15625	68.6394211626036\\
0.15673828125	68.4486203753561\\
0.1572265625	68.2605195251747\\
0.15771484375	68.0750938469616\\
0.158203125	67.892317135682\\
0.15869140625	67.712161968536\\
0.1591796875	67.534599903962\\
0.15966796875	67.3596016597045\\
0.16015625	67.1871372719849\\
0.16064453125	67.0171762376189\\
0.1611328125	66.8496876407492\\
0.16162109375	66.6846402657096\\
0.162109375	66.5220026973839\\
0.16259765625	66.3617434103035\\
0.1630859375	66.2038308475959\\
0.16357421875	66.0482334907986\\
0.1640625	65.8949199214517\\
0.16455078125	65.7438588752923\\
0.1650390625	65.5950192897961\\
0.16552734375	65.4483703457381\\
0.166015625	65.3038815033798\\
0.16650390625	65.1615225338303\\
0.1669921875	65.0212635460746\\
0.16748046875	64.8830750101154\\
0.16796875	64.746927776632\\
0.16845703125	64.6127930935169\\
0.1689453125	64.4806426196186\\
0.16943359375	64.3504484359862\\
0.169921875	64.2221830548813\\
0.17041015625	64.0958194267971\\
0.1708984375	63.971330945703\\
0.17138671875	63.8486914527061\\
0.171875	63.7278752383104\\
0.17236328125	63.6088570434284\\
0.1728515625	63.4916120592891\\
0.17333984375	63.3761159263715\\
0.173828125	63.2623447324801\\
0.17431640625	63.150275010064\\
0.1748046875	63.0398837328768\\
0.17529296875	62.9311483120602\\
0.17578125	62.8240465917271\\
0.17626953125	62.7185568441108\\
0.1767578125	62.6146577643485\\
0.17724609375	62.5123284649436\\
0.177734375	62.4115484699658\\
0.17822265625	62.3122977090271\\
0.1787109375	62.2145565110764\\
0.17919921875	62.118305598047\\
0.1796875	62.0235260783896\\
0.18017578125	61.9301994405168\\
0.1806640625	61.8383075461881\\
0.18115234375	61.7478326238537\\
0.181640625	61.6587572619806\\
0.18212890625	61.5710644023752\\
0.1826171875	61.4847373335217\\
0.18310546875	61.3997596839472\\
0.18359375	61.3161154156262\\
0.18408203125	61.2337888174361\\
0.1845703125	61.1527644986723\\
0.18505859375	61.0730273826299\\
0.185546875	60.9945627002619\\
0.18603515625	60.9173559839157\\
0.1865234375	60.8413930611563\\
0.18701171875	60.7666600486793\\
0.1875	60.6931433463161\\
0.18798828125	60.6208296311371\\
0.1884765625	60.549705851652\\
0.18896484375	60.4797592221098\\
0.189453125	60.4109772169026\\
0.18994140625	60.3433475650701\\
0.1904296875	60.2768582449077\\
0.19091796875	60.2114974786779\\
0.19140625	60.1472537274254\\
0.19189453125	60.0841156858943\\
0.1923828125	60.0220722775479\\
0.19287109375	59.9611126496912\\
0.193359375	59.901226168692\\
0.19384765625	59.8424024153038\\
0.1943359375	59.7846311800851\\
0.19482421875	59.7279024589177\\
0.1953125	59.6722064486189\\
0.19580078125	59.6175335426487\\
0.1962890625	59.5638743269092\\
0.19677734375	59.511219575635\\
0.197265625	59.4595602473726\\
0.19775390625	59.408887481047\\
0.1982421875	59.3591925921137\\
0.19873046875	59.3104670687955\\
0.19921875	59.2627025683997\\
0.19970703125	59.2158909137164\\
0.2001953125	59.1700240894942\\
0.20068359375	59.1250942389935\\
0.201171875	59.0810936606126\\
0.20166015625	59.0380148045874\\
0.2021484375	58.995850269762\\
0.20263671875	58.9545928004277\\
0.203125	58.9142352832285\\
0.20361328125	58.8747707441336\\
0.2041015625	58.8361923454713\\
0.20458984375	58.7984933830262\\
0.205078125	58.7616672831948\\
0.20556640625	58.7257076002006\\
0.2060546875	58.6906080133643\\
0.20654296875	58.6563623244296\\
0.20703125	58.6229644549419\\
0.20751953125	58.5904084436774\\
0.2080078125	58.5586884441227\\
0.20849609375	58.5277987220025\\
0.208984375	58.4977336528529\\
0.20947265625	58.4684877196404\\
0.2099609375	58.4400555104233\\
0.21044921875	58.412431716056\\
0.2109375	58.3856111279312\\
0.21142578125	58.3595886357646\\
0.2119140625	58.3343592254122\\
0.21240234375	58.3099179767266\\
0.212890625	58.2862600614465\\
0.21337890625	58.2633807411189\\
0.2138671875	58.2412753650522\\
0.21435546875	58.2199393683008\\
0.21484375	58.1993682696764\\
0.21533203125	58.1795576697883\\
0.2158203125	58.1605032491075\\
0.21630859375	58.1422007660573\\
0.216796875	58.1246460551258\\
0.21728515625	58.1078350250008\\
0.2177734375	58.0917636567243\\
0.21826171875	58.076428001867\\
0.21875	58.0618241807206\\
0.21923828125	58.0479483805061\\
0.2197265625	58.0347968535977\\
0.22021484375	58.0223659157604\\
0.220703125	58.0106519444009\\
0.22119140625	57.9996513768285\\
0.2216796875	57.9893607085276\\
0.22216796875	57.9797764914379\\
0.22265625	57.9708953322427\\
0.22314453125	57.9627138906629\\
0.2236328125	57.955228877756\\
0.22412109375	57.9484370542195\\
0.224609375	57.942335228695\\
0.22509765625	57.9369202560748\\
0.2255859375	57.9321890358074\\
0.22607421875	57.9281385102016\\
0.2265625	57.9247656627271\\
0.22705078125	57.9220675163116\\
0.2275390625	57.9200411316315\\
0.22802734375	57.918683605396\\
0.228515625	57.9179920686221\\
0.22900390625	57.9179636849009\\
0.2294921875	57.9185956486513\\
0.22998046875	57.9198851833619\\
0.23046875	57.9218295398184\\
0.23095703125	57.9244259943158\\
0.2314453125	57.9276718468529\\
0.23193359375	57.9315644193087\\
0.232421875	57.9361010535987\\
0.23291015625	57.9412791098087\\
0.2333984375	57.9470959643068\\
0.23388671875	57.9535490078293\\
0.234375	57.9606356435406\\
0.23486328125	57.9683532850649\\
0.2353515625	57.9766993544885\\
0.23583984375	57.9856712803297\\
0.236328125	57.9952664954765\\
0.23681640625	58.0054824350881\\
0.2373046875	58.0163165344608\\
0.23779296875	58.0277662268539\\
0.23828125	58.0398289412769\\
0.23876953125	58.0525021002318\\
0.2392578125	58.0657831174131\\
0.23974609375	58.0796693953607\\
0.240234375	58.0941583230643\\
0.24072265625	58.1092472735174\\
0.2412109375	58.1249336012191\\
0.24169921875	58.141214639621\\
0.2421875	58.1580876985175\\
0.24267578125	58.1755500613773\\
0.2431640625	58.1935989826131\\
0.24365234375	58.2122316847886\\
0.244140625	58.231445355759\\
0.24462890625	58.2512371457444\\
0.2451171875	58.2716041643311\\
0.24560546875	58.292543477402\\
0.24609375	58.3140521039899\\
0.24658203125	58.3361270130537\\
0.2470703125	58.3587651201748\\
0.24755859375	58.3819632841691\\
0.248046875	58.4057183036151\\
0.24853515625	58.4300269132925\\
0.2490234375	58.454885780532\\
0.24951171875	58.4802915014688\\
0.25	58.5062405972032\\
0.25048828125	58.5327295098605\\
0.2509765625	58.5597545985494\\
0.25146484375	58.5873121352179\\
0.251953125	58.6153983004006\\
0.25244140625	58.6440091788571\\
0.2529296875	58.6731407550983\\
0.25341796875	58.7027889087979\\
0.25390625	58.7329494100855\\
0.25439453125	58.7636179147198\\
0.2548828125	58.7947899591395\\
0.25537109375	58.8264609553882\\
0.255859375	58.858626185911\\
0.25634765625	58.8912807982218\\
0.2568359375	58.9244197994367\\
0.25732421875	58.9580380506734\\
0.2578125	58.9921302613124\\
0.25830078125	59.0266909831198\\
0.2587890625	59.0617146042284\\
0.25927734375	59.0971953429759\\
0.259765625	59.1331272415987\\
0.26025390625	59.1695041597791\\
0.2607421875	59.2063197680459\\
0.26123046875	59.2435675410264\\
0.26171875	59.2812407505489\\
0.26220703125	59.319332458596\\
0.2626953125	59.3578355101084\\
0.26318359375	59.3967425256387\\
0.263671875	59.436045893855\\
0.26416015625	59.4757377638982\\
0.2646484375	59.5158100375902\\
0.26513671875	59.5562543614981\\
0.265625	59.597062118854\\
0.26611328125	59.6382244213352\\
0.2666015625	59.6797321007069\\
0.26708984375	59.7215757003308\\
0.267578125	59.7637454665459\\
0.26806640625	59.8062313399257\\
0.2685546875	59.849022946416\\
0.26904296875	59.8921095883637\\
0.26953125	59.9354802354414\\
0.27001953125	59.979123515477\\
0.2705078125	60.0230277051973\\
0.27099609375	60.0671807208978\\
0.271484375	60.1115701090472\\
0.27197265625	60.1561830368422\\
0.2724609375	60.2010062827241\\
0.27294921875	60.2460262268728\\
0.2734375	60.2912288416964\\
0.27392578125	60.33659968233\\
0.2744140625	60.3821238771674\\
0.27490234375	60.4277861184422\\
0.275390625	60.4735706528844\\
0.27587890625	60.5194612724715\\
0.2763671875	60.5654413053046\\
0.27685546875	60.6114936066324\\
0.27734375	60.6576005500547\\
0.27783203125	60.7037440189352\\
0.2783203125	60.7499053980549\\
0.27880859375	60.7960655655439\\
0.279296875	60.8422048851248\\
0.27978515625	60.8883031987076\\
0.2802734375	60.9343398193772\\
0.28076171875	60.9802935248147\\
0.28125	61.0261425511977\\
0.28173828125	61.071864587626\\
0.2822265625	61.1174367711219\\
0.28271484375	61.1628356822544\\
0.283203125	61.2080373414404\\
0.28369140625	61.2530172059774\\
0.2841796875	61.2977501678631\\
0.28466796875	61.3422105524603\\
0.28515625	61.3863721180656\\
0.28564453125	61.4302080564426\\
0.2861328125	61.4736909943827\\
0.28662109375	61.5167929963552\\
0.287109375	61.5594855683096\\
0.28759765625	61.601739662696\\
0.2880859375	61.6435256847695\\
0.28857421875	61.6848135002381\\
0.2890625	61.7255724443245\\
0.28955078125	61.7657713323011\\
0.2900390625	61.8053784715629\\
0.29052734375	61.8443616752976\\
0.291015625	61.882688277815\\
0.29150390625	61.9203251515876\\
0.2919921875	61.9572387260614\\
0.29248046875	61.9933950082829\\
0.29296875	62.0287596053915\\
0.29345703125	62.063297749016\\
0.2939453125	62.0969743216147\\
0.29443359375	62.1297538847876\\
0.294921875	62.161600709585\\
0.29541015625	62.1924788088315\\
0.2958984375	62.2223519714707\\
0.29638671875	62.2511837989374\\
0.296875	62.2789377435436\\
0.29736328125	62.3055771488649\\
0.2978515625	62.3310652920983\\
0.29833984375	62.355365428353\\
0.298828125	62.3784408368257\\
0.29931640625	62.4002548687984\\
0.2998046875	62.4207709973877\\
0.30029296875	62.439952868959\\
0.30078125	62.457764356108\\
0.30126953125	62.4741696121006\\
0.3017578125	62.489133126647\\
0.30224609375	62.5026197828752\\
0.302734375	62.5145949153554\\
0.30322265625	62.5250243690145\\
0.3037109375	62.5338745587698\\
0.30419921875	62.5411125296972\\
0.3046875	62.5467060175398\\
0.30517578125	62.5506235093531\\
0.3056640625	62.5528343040728\\
0.30615234375	62.5533085727848\\
0.306640625	62.5520174184686\\
0.30712890625	62.5489329349794\\
0.3076171875	62.5440282650331\\
0.30810546875	62.5372776569503\\
0.30859375	62.5286565199197\\
0.30908203125	62.5181414775368\\
0.3095703125	62.5057104193805\\
0.31005859375	62.4913425503884\\
0.310546875	62.4750184378037\\
0.31103515625	62.4567200554682\\
0.3115234375	62.4364308252485\\
0.31201171875	62.4141356553912\\
0.3125	62.3898209756168\\
0.31298828125	62.3634747687746\\
0.3134765625	62.3350865989\\
0.31396484375	62.3046476355247\\
0.314453125	62.2721506741201\\
0.31494140625	62.2375901525607\\
0.3154296875	62.2009621635254\\
0.31591796875	62.162264462769\\
0.31640625	62.1214964732213\\
0.31689453125	62.0786592848889\\
0.3173828125	62.0337556505651\\
0.31787109375	61.9867899773636\\
0.318359375	61.937768314128\\
0.31884765625	61.8866983347761\\
0.3193359375	61.8335893176714\\
0.31982421875	61.778452121127\\
0.3203125	61.7212991551671\\
0.32080078125	61.6621443496936\\
0.3212890625	61.6010031192192\\
0.32177734375	61.5378923243407\\
0.322265625	61.4728302301486\\
0.32275390625	61.405836461773\\
0.3232421875	61.3369319572821\\
0.32373046875	61.266138918155\\
0.32421875	61.1934807575614\\
0.32470703125	61.1189820466823\\
0.3251953125	61.0426684593134\\
0.32568359375	60.9645667149922\\
0.326171875	60.8847045208892\\
0.32666015625	60.8031105127089\\
0.3271484375	60.7198141948332\\
0.32763671875	60.6348458799445\\
0.328125	60.5482366283551\\
0.32861328125	60.4600181872648\\
0.3291015625	60.3702229301587\\
0.32958984375	60.2788837965506\\
0.330078125	60.1860342322644\\
0.33056640625	60.0917081304392\\
0.3310546875	59.9959397734257\\
0.33154296875	59.8987637757402\\
0.33203125	59.8002150282177\\
0.33251953125	59.7003286435041\\
0.3330078125	59.5991399030059\\
0.33349609375	59.4966842054124\\
0.333984375	59.3929970168827\\
0.33447265625	59.2881138229851\\
0.3349609375	59.1820700824609\\
0.33544921875	59.074901182872\\
0.3359375	58.9666423981835\\
0.33642578125	58.8573288483165\\
0.3369140625	58.7469954607001\\
0.33740234375	58.6356769338403\\
0.337890625	58.5234077029107\\
0.33837890625	58.4102219073676\\
0.3388671875	58.2961533605758\\
0.33935546875	58.1812355214309\\
0.33984375	58.065501467953\\
0.34033203125	57.94898387282\\
0.3408203125	57.8317149808049\\
0.34130859375	57.7137265880754\\
0.341796875	57.5950500233087\\
0.34228515625	57.4757161305728\\
0.3427734375	57.3557552539173\\
0.34326171875	57.235197223621\\
0.34375	57.1140713440343\\
0.34423828125	56.9924063829559\\
0.3447265625	56.8702305624817\\
0.34521484375	56.7475715512625\\
0.345703125	56.6244564581049\\
0.34619140625	56.5009118268522\\
0.3466796875	56.3769636324794\\
0.34716796875	56.2526372783393\\
0.34765625	56.1279575944946\\
0.34814453125	56.0029488370743\\
0.3486328125	55.8776346885913\\
0.34912109375	55.7520382591622\\
0.349609375	55.6261820885683\\
0.35009765625	55.5000881491007\\
0.3505859375	55.3737778491352\\
0.35107421875	55.2472720373793\\
0.3515625	55.1205910077417\\
0.35205078125	54.993754504774\\
0.3525390625	54.8667817296328\\
0.35302734375	54.73969134652\\
0.353515625	54.6125014895525\\
0.35400390625	54.485229770024\\
0.3544921875	54.3578932840117\\
0.35498046875	54.2305086202961\\
0.35546875	54.1030918685529\\
0.35595703125	53.9756586277843\\
0.3564453125	53.8482240149572\\
0.35693359375	53.7208026738163\\
0.357421875	53.5934087838448\\
0.35791015625	53.4660560693444\\
0.3583984375	53.3387578086097\\
0.35888671875	53.2115268431729\\
0.359375	53.0843755870978\\
0.35986328125	52.9573160363\\
0.3603515625	52.8303597778779\\
0.36083984375	52.7035179994324\\
0.361328125	52.5768014983639\\
0.36181640625	52.4502206911268\\
0.3623046875	52.3237856224313\\
0.36279296875	52.1975059743774\\
0.36328125	52.0713910755127\\
0.36376953125	51.9454499097987\\
0.3642578125	51.819691125482\\
0.36474609375	51.6941230438567\\
0.365234375	51.5687536679139\\
0.36572265625	51.4435906908698\\
0.3662109375	51.3186415045671\\
0.36669921875	51.1939132077441\\
0.3671875	51.0694126141667\\
0.36767578125	50.9451462606204\\
0.3681640625	50.8211204147576\\
0.36865234375	50.6973410827991\\
0.369140625	50.5738140170851\\
0.36962890625	50.4505447234768\\
0.3701171875	50.3275384686052\\
0.37060546875	50.2048002869666\\
0.37109375	50.0823349878657\\
0.37158203125	49.9601471622028\\
0.3720703125	49.8382411891102\\
0.37255859375	49.7166212424324\\
0.373046875	49.5952912970563\\
0.37353515625	49.4742551350883\\
0.3740234375	49.3535163518812\\
0.37451171875	49.2330783619128\\
0.375	49.1129444045154\\
0.37548828125	48.9931175494617\\
0.3759765625	48.873600702404\\
0.37646484375	48.7543966101737\\
0.376953125	48.6355078659394\\
0.37744140625	48.5169369142275\\
0.3779296875	48.3986860558077\\
0.37841796875	48.2807574524443\\
0.37890625	48.163153131518\\
0.37939453125	48.0458749905177\\
0.3798828125	47.928924801408\\
0.38037109375	47.812304214872\\
0.380859375	47.6960147644338\\
0.38134765625	47.5800578704629\\
0.3818359375	47.4644348440625\\
0.38232421875	47.3491468908443\\
0.3828125	47.2341951145936\\
0.38330078125	47.119580520826\\
0.3837890625	47.0053040202378\\
0.38427734375	46.8913664320548\\
0.384765625	46.7777684872795\\
0.38525390625	46.6645108318409\\
0.3857421875	46.5515940296492\\
0.38623046875	46.4390185655563\\
0.38671875	46.3267848482267\\
0.38720703125	46.2148932129196\\
0.3876953125	46.1033439241855\\
0.38818359375	45.9921371784784\\
0.388671875	45.8812731066871\\
0.38916015625	45.7707517765877\\
0.3896484375	45.6605731952185\\
0.39013671875	45.5507373111798\\
0.390625	45.4412440168619\\
0.39111328125	45.3320931506018\\
0.3916015625	45.223284498771\\
0.39208984375	45.1148177977968\\
0.392578125	45.0066927361189\\
0.39306640625	44.8989089560832\\
0.3935546875	44.7914660557736\\
0.39404296875	44.6843635907858\\
0.39453125	44.5776010759422\\
0.39501953125	44.4711779869523\\
0.3955078125	44.3650937620173\\
0.39599609375	44.2593478033838\\
0.396484375	44.1539394788453\\
0.39697265625	44.0488681231945\\
0.3974609375	43.944133039628\\
0.39794921875	43.8397335011044\\
0.3984375	43.7356687516568\\
0.39892578125	43.6319380076619\\
0.3994140625	43.5285404590674\\
0.39990234375	43.4254752705772\\
0.400390625	43.3227415827974\\
0.40087890625	43.220338513344\\
0.4013671875	43.1182651579125\\
0.40185546875	43.0165205913124\\
0.40234375	42.9151038684653\\
0.40283203125	42.8140140253702\\
0.4033203125	42.7132500800354\\
0.40380859375	42.6128110333785\\
0.404296875	42.5126958700958\\
0.40478515625	42.4129035595016\\
0.4052734375	42.3134330563388\\
0.40576171875	42.214283301561\\
0.40625	42.1154532230886\\
0.40673828125	42.0169417365372\\
0.4072265625	41.9187477459217\\
0.40771484375	41.8208701443352\\
0.408203125	41.7233078146047\\
0.40869140625	41.6260596299229\\
0.4091796875	41.5291244544585\\
0.40966796875	41.4325011439445\\
0.41015625	41.3361885462458\\
0.41064453125	41.2401855019063\\
0.4111328125	41.144490844677\\
0.41162109375	41.0491034020245\\
0.412109375	40.9540219956217\\
0.41259765625	40.8592454418207\\
0.4130859375	40.7647725521083\\
0.41357421875	40.6706021335453\\
0.4140625	40.5767329891892\\
0.41455078125	40.4831639185025\\
0.4150390625	40.3898937177444\\
0.41552734375	40.2969211803494\\
0.416015625	40.2042450972912\\
0.41650390625	40.1118642574331\\
0.4169921875	40.0197774478652\\
0.41748046875	39.9279834542291\\
0.41796875	39.8364810610303\\
0.41845703125	39.7452690519385\\
0.4189453125	39.6543462100765\\
0.41943359375	39.5637113182986\\
0.419921875	39.4733631594567\\
0.42041015625	39.383300516658\\
0.4208984375	39.2935221735114\\
0.42138671875	39.2040269143641\\
0.421875	39.11481352453\\
0.42236328125	39.0258807905074\\
0.4228515625	38.9372275001894\\
0.42333984375	38.8488524430649\\
0.423828125	38.7607544104117\\
0.42431640625	38.672932195482\\
0.4248046875	38.5853845936796\\
0.42529296875	38.4981104027302\\
0.42578125	38.4111084228444\\
0.42626953125	38.3243774568737\\
0.4267578125	38.23791631046\\
0.42724609375	38.1517237921787\\
0.427734375	38.065798713675\\
0.42822265625	37.980139889795\\
0.4287109375	37.8947461387107\\
0.42919921875	37.8096162820389\\
0.4296875	37.7247491449549\\
0.43017578125	37.6401435563021\\
0.4306640625	37.5557983486941\\
0.43115234375	37.4717123586146\\
0.431640625	37.3878844265106\\
0.43212890625	37.3043133968818\\
0.4326171875	37.2209981183661\\
0.43310546875	37.1379374438194\\
0.43359375	37.0551302303929\\
0.43408203125	36.9725753396056\\
0.4345703125	36.8902716374127\\
0.43505859375	36.808217994271\\
0.435546875	36.7264132852004\\
0.43603515625	36.6448563898421\\
0.4365234375	36.563546192513\\
0.43701171875	36.4824815822572\\
0.4375	36.4016614528952\\
0.43798828125	36.3210847030685\\
0.4384765625	36.2407502362826\\
0.43896484375	36.1606569609466\\
0.439453125	36.0808037904108\\
0.43994140625	36.0011896430005\\
0.4404296875	35.9218134420487\\
0.44091796875	35.8426741159254\\
0.44140625	35.7637705980651\\
0.44189453125	35.6851018269916\\
0.4423828125	35.6066667463415\\
0.44287109375	35.5284643048844\\
0.443359375	35.4504934565424\\
0.44384765625	35.3727531604065\\
0.4443359375	35.2952423807521\\
0.44482421875	35.2179600870521\\
0.4453125	35.1409052539887\\
0.44580078125	35.0640768614629\\
0.4462890625	34.9874738946035\\
0.44677734375	34.9110953437732\\
0.447265625	34.8349402045744\\
0.44775390625	34.759007477853\\
0.4482421875	34.6832961697008\\
0.44873046875	34.607805291457\\
0.44921875	34.532533859708\\
0.44970703125	34.4574808962863\\
0.4501953125	34.3826454282675\\
0.45068359375	34.3080264879679\\
0.451171875	34.2336231129387\\
0.45166015625	34.1594343459614\\
0.4521484375	34.0854592350402\\
0.45263671875	34.0116968333953\\
0.453125	33.9381461994537\\
0.45361328125	33.8648063968404\\
0.4541015625	33.7916764943679\\
0.45458984375	33.7187555660251\\
0.455078125	33.6460426909657\\
0.45556640625	33.5735369534957\\
0.4560546875	33.5012374430602\\
0.45654296875	33.4291432542292\\
0.45703125	33.3572534866834\\
0.45751953125	33.2855672451989\\
0.4580078125	33.2140836396316\\
0.45849609375	33.1428017849005\\
0.458984375	33.0717208009713\\
0.45947265625	33.0008398128389\\
0.4599609375	32.9301579505094\\
0.46044921875	32.8596743489818\\
0.4609375	32.7893881482295\\
0.46142578125	32.719298493181\\
0.4619140625	32.6494045337001\\
0.46240234375	32.5797054245662\\
0.462890625	32.5102003254541\\
0.46337890625	32.4408884009129\\
0.4638671875	32.3717688203455\\
0.46435546875	32.302840757987\\
0.46484375	32.2341033928832\\
0.46533203125	32.1655559088689\\
0.4658203125	32.0971974945458\\
0.46630859375	32.0290273432596\\
0.466796875	31.9610446530785\\
0.46728515625	31.8932486267694\\
0.4677734375	31.8256384717755\\
0.46826171875	31.7582134001931\\
0.46875	31.690972628748\\
0.46923828125	31.6239153787722\\
0.4697265625	31.5570408761804\\
0.47021484375	31.4903483514458\\
0.470703125	31.4238370395769\\
0.47119140625	31.3575061800926\\
0.4716796875	31.2913550169988\\
0.47216796875	31.2253827987642\\
0.47265625	31.1595887782952\\
0.47314453125	31.0939722129126\\
0.4736328125	31.0285323643263\\
0.47412109375	30.9632684986116\\
0.474609375	30.8981798861838\\
0.47509765625	30.8332658017743\\
0.4755859375	30.7685255244057\\
0.47607421875	30.7039583373673\\
0.4765625	30.6395635281901\\
0.47705078125	30.5753403886227\\
0.4775390625	30.511288214606\\
0.47802734375	30.4474063062489\\
0.478515625	30.3836939678037\\
0.47900390625	30.3201505076412\\
0.4794921875	30.256775238226\\
0.47998046875	30.1935674760923\\
0.48046875	30.1305265418187\\
0.48095703125	30.0676517600044\\
0.4814453125	30.0049424592439\\
0.48193359375	29.942397972103\\
0.482421875	29.8800176350943\\
0.48291015625	29.8178007886528\\
0.4833984375	29.7557467771115\\
0.48388671875	29.6938549486771\\
0.484375	29.6321246554061\\
0.48486328125	29.5705552531803\\
0.4853515625	29.509146101683\\
0.48583984375	29.4478965643747\\
0.486328125	29.3868060084696\\
0.48681640625	29.3258738049113\\
0.4873046875	29.2650993283496\\
0.48779296875	29.2044819571161\\
0.48828125	29.1440210732015\\
0.48876953125	29.0837160622312\\
0.4892578125	29.0235663134427\\
0.48974609375	28.9635712196614\\
0.490234375	28.9037301772785\\
0.49072265625	28.8440425862269\\
0.4912109375	28.7845078499586\\
0.49169921875	28.7251253754219\\
0.4921875	28.6658945730382\\
0.49267578125	28.6068148566798\\
0.4931640625	28.5478856436468\\
0.49365234375	28.489106354645\\
0.494140625	28.4304764137633\\
0.49462890625	28.3719952484515\\
0.4951171875	28.3136622894979\\
0.49560546875	28.2554769710078\\
0.49609375	28.197438730381\\
0.49658203125	28.1395470082905\\
0.4970703125	28.0818012486601\\
0.49755859375	28.0242008986436\\
0.498046875	27.9667454086027\\
0.49853515625	27.9094342320862\\
0.4990234375	27.852266825808\\
0.49951171875	27.795242649627\\
0.5	27.7383611665252\\
0.50048828125	27.6816218425873\\
0.5009765625	27.6250241469799\\
0.50146484375	27.5685675519308\\
0.501953125	27.5122515327085\\
0.50244140625	27.4560755676017\\
0.5029296875	27.4000391378993\\
0.50341796875	27.3441417278701\\
0.50390625	27.2883828247427\\
0.50439453125	27.232761918686\\
0.5048828125	27.1772785027886\\
0.50537109375	27.1219320730402\\
0.505859375	27.0667221283112\\
0.50634765625	27.0116481703339\\
0.5068359375	26.9567097036829\\
0.50732421875	26.9019062357559\\
0.5078125	26.8472372767548\\
0.50830078125	26.792702339667\\
0.5087890625	26.7383009402461\\
0.50927734375	26.6840325969937\\
0.509765625	26.6298968311406\\
0.51025390625	26.5758931666285\\
0.5107421875	26.5220211300916\\
0.51123046875	26.4682802508385\\
0.51171875	26.4146700608342\\
0.51220703125	26.361190094682\\
0.5126953125	26.3078398896058\\
0.51318359375	26.2546189854324\\
0.513671875	26.2015269245739\\
0.51416015625	26.1485632520104\\
0.5146484375	26.0957275152722\\
0.51513671875	26.0430192644232\\
0.515625	25.9904380520434\\
0.51611328125	25.9379834332122\\
0.5166015625	25.8856549654912\\
0.51708984375	25.8334522089076\\
0.517578125	25.7813747259378\\
0.51806640625	25.7294220814906\\
0.5185546875	25.6775938428909\\
0.51904296875	25.6258895798635\\
0.51953125	25.5743088645168\\
0.52001953125	25.5228512713268\\
0.5205078125	25.4715163771212\\
0.52099609375	25.4203037610638\\
0.521484375	25.369213004638\\
0.52197265625	25.3182436916324\\
0.5224609375	25.2673954081242\\
0.52294921875	25.2166677424645\\
0.5234375	25.1660602852628\\
0.52392578125	25.1155726293718\\
0.5244140625	25.0652043698726\\
0.52490234375	25.0149551040593\\
0.525390625	24.9648244314247\\
0.52587890625	24.9148119536454\\
0.5263671875	24.8649172745669\\
0.52685546875	24.8151400001893\\
0.52734375	24.7654797386533\\
0.52783203125	24.715936100225\\
0.5283203125	24.6665086972825\\
0.52880859375	24.6171971443016\\
0.529296875	24.5680010578416\\
0.52978515625	24.5189200565315\\
0.5302734375	24.4699537610565\\
0.53076171875	24.4211017941438\\
0.53125	24.3723637805496\\
0.53173828125	24.3237393470452\\
0.5322265625	24.2752281224036\\
0.53271484375	24.2268297373866\\
0.533203125	24.1785438247314\\
0.53369140625	24.1303700191371\\
0.5341796875	24.0823079572527\\
0.53466796875	24.034357277663\\
0.53515625	23.9865176208768\\
0.53564453125	23.9387886293137\\
0.5361328125	23.8911699472915\\
0.53662109375	23.8436612210139\\
0.537109375	23.7962620985578\\
0.53759765625	23.7489722298612\\
0.5380859375	23.7017912667108\\
0.53857421875	23.6547188627301\\
0.5390625	23.607754673367\\
0.53955078125	23.5608983558819\\
0.5400390625	23.5141495693361\\
0.54052734375	23.4675079745797\\
0.541015625	23.4209732342402\\
0.54150390625	23.3745450127104\\
0.5419921875	23.3282229761373\\
0.54248046875	23.2820067924106\\
0.54296875	23.2358961311511\\
0.54345703125	23.1898906636997\\
0.5439453125	23.1439900631061\\
0.54443359375	23.0981940041176\\
0.544921875	23.0525021631681\\
0.54541015625	23.0069142183675\\
0.5458984375	22.9614298494902\\
0.54638671875	22.9160487379648\\
0.546875	22.8707705668635\\
0.54736328125	22.825595020891\\
0.5478515625	22.7805217863741\\
0.54833984375	22.7355505512518\\
0.548828125	22.690681005064\\
0.54931640625	22.645912838942\\
0.5498046875	22.6012457455977\\
0.55029296875	22.5566794193139\\
0.55078125	22.5122135559337\\
0.55126953125	22.4678478528512\\
0.5517578125	22.4235820090008\\
0.55224609375	22.379415724848\\
0.552734375	22.3353487023791\\
0.55322265625	22.291380645092\\
0.5537109375	22.247511257986\\
0.55419921875	22.2037402475528\\
0.5546875	22.1600673217665\\
0.55517578125	22.1164921900747\\
0.5556640625	22.0730145633887\\
0.55615234375	22.0296341540743\\
0.556640625	21.9863506759429\\
0.55712890625	21.943163844242\\
0.5576171875	21.9000733756462\\
0.55810546875	21.8570789882485\\
0.55859375	21.8141804015507\\
0.55908203125	21.7713773364554\\
0.5595703125	21.7286695152563\\
0.56005859375	21.6860566616302\\
0.560546875	21.6435385006277\\
0.56103515625	21.6011147586653\\
0.5615234375	21.5587851635159\\
0.56201171875	21.5165494443013\\
0.5625	21.4744073314832\\
0.56298828125	21.4323585568547\\
0.5634765625	21.3904028535326\\
0.56396484375	21.3485399559487\\
0.564453125	21.3067695998417\\
0.56494140625	21.2650915222491\\
0.5654296875	21.2235054614992\\
0.56591796875	21.1820111572032\\
0.56640625	21.1406083502468\\
0.56689453125	21.0992967827831\\
0.5673828125	21.0580761982238\\
0.56787109375	21.0169463412324\\
0.568359375	20.9759069577158\\
0.56884765625	20.934957794817\\
0.5693359375	20.8940986009074\\
0.56982421875	20.8533291255792\\
0.5703125	20.8126491196379\\
0.57080078125	20.7720583350952\\
0.5712890625	20.7315565251612\\
0.57177734375	20.6911434442373\\
0.572265625	20.6508188479087\\
0.57275390625	20.6105824929377\\
0.5732421875	20.570434137256\\
0.57373046875	20.5303735399578\\
0.57421875	20.4904004612928\\
0.57470703125	20.4505146626591\\
0.5751953125	20.4107159065964\\
0.57568359375	20.3710039567787\\
0.576171875	20.331378578008\\
0.57666015625	20.291839536207\\
0.5771484375	20.2523865984129\\
0.57763671875	20.2130195327699\\
0.578125	20.1737381085233\\
0.57861328125	20.1345420960127\\
0.5791015625	20.0954312666649\\
0.57958984375	20.0564053929882\\
0.580078125	20.0174642485655\\
0.58056640625	19.9786076080478\\
0.5810546875	19.939835247148\\
0.58154296875	19.9011469426346\\
0.58203125	19.8625424723253\\
0.58251953125	19.8240216150808\\
0.5830078125	19.7855841507988\\
0.58349609375	19.7472298604073\\
0.583984375	19.7089585258593\\
0.58447265625	19.670769930126\\
0.5849609375	19.6326638571913\\
0.58544921875	19.5946400920456\\
0.5859375	19.5566984206798\\
0.58642578125	19.5188386300795\\
0.5869140625	19.4810605082194\\
0.58740234375	19.4433638440571\\
0.587890625	19.4057484275274\\
0.58837890625	19.3682140495369\\
0.5888671875	19.330760501958\\
0.58935546875	19.2933875776233\\
0.58984375	19.2560950703203\\
0.59033203125	19.2188827747853\\
0.5908203125	19.1817504866984\\
0.59130859375	19.1446980026779\\
0.591796875	19.1077251202744\\
0.59228515625	19.0708316379661\\
0.5927734375	19.034017355153\\
0.59326171875	18.9972820721516\\
0.59375	18.9606255901898\\
0.59423828125	18.9240477114015\\
0.5947265625	18.8875482388214\\
0.59521484375	18.8511269763798\\
0.595703125	18.8147837288974\\
0.59619140625	18.7785183020807\\
0.5966796875	18.7423305025161\\
0.59716796875	18.7062201376655\\
0.59765625	18.6701870158609\\
0.59814453125	18.6342309463\\
0.5986328125	18.5983517390404\\
0.59912109375	18.5625492049956\\
0.599609375	18.5268231559296\\
0.60009765625	18.4911734044523\\
0.6005859375	18.4555997640144\\
0.60107421875	18.4201020489031\\
0.6015625	18.384680074237\\
0.60205078125	18.3493336559617\\
0.6025390625	18.3140626108449\\
0.60302734375	18.2788667564718\\
0.603515625	18.2437459112407\\
0.60400390625	18.2086998943581\\
0.6044921875	18.1737285258345\\
0.60498046875	18.1388316264799\\
0.60546875	18.1040090178989\\
0.60595703125	18.0692605224867\\
0.6064453125	18.0345859634247\\
0.60693359375	17.9999851646756\\
0.607421875	17.9654579509799\\
0.60791015625	17.9310041478507\\
0.6083984375	17.8966235815699\\
0.60888671875	17.8623160791841\\
0.609375	17.8280814684998\\
0.60986328125	17.7939195780798\\
0.6103515625	17.7598302372384\\
0.61083984375	17.7258132760381\\
0.611328125	17.6918685252846\\
0.61181640625	17.6579958165232\\
0.6123046875	17.6241949820348\\
0.61279296875	17.5904658548315\\
0.61328125	17.5568082686529\\
0.61376953125	17.5232220579621\\
0.6142578125	17.4897070579416\\
0.61474609375	17.4562631044896\\
0.615234375	17.4228900342157\\
0.61572265625	17.3895876844375\\
0.6162109375	17.3563558931765\\
0.61669921875	17.3231944991544\\
0.6171875	17.2901033417891\\
0.61767578125	17.2570822611912\\
0.6181640625	17.2241310981601\\
0.61865234375	17.1912496941802\\
0.619140625	17.1584378914174\\
0.61962890625	17.1256955327155\\
0.6201171875	17.0930224615922\\
0.62060546875	17.0604185222358\\
0.62109375	17.0278835595015\\
0.62158203125	16.9954174189078\\
0.6220703125	16.9630199466331\\
0.62255859375	16.9306909895118\\
0.623046875	16.8984303950315\\
0.62353515625	16.8662380113286\\
0.6240234375	16.8341136871858\\
0.62451171875	16.8020572720279\\
0.625	16.7700686159188\\
0.62548828125	16.7381475695581\\
0.6259765625	16.7062939842775\\
0.62646484375	16.6745077120379\\
0.626953125	16.6427886054254\\
0.62744140625	16.6111365176488\\
0.6279296875	16.5795513025358\\
0.62841796875	16.5480328145298\\
0.62890625	16.5165809086868\\
0.62939453125	16.4851954406722\\
0.6298828125	16.4538762667576\\
0.63037109375	16.4226232438175\\
0.630859375	16.3914362293263\\
0.63134765625	16.3603150813551\\
0.6318359375	16.3292596585685\\
0.63232421875	16.298269820222\\
0.6328125	16.2673454261581\\
0.63330078125	16.2364863368041\\
0.6337890625	16.2056924131684\\
0.63427734375	16.1749635168381\\
0.634765625	16.1442995099754\\
0.63525390625	16.1137002553152\\
0.6357421875	16.0831656161618\\
0.63623046875	16.052695456386\\
0.63671875	16.0222896404223\\
0.63720703125	15.9919480332661\\
0.6376953125	15.9616705004705\\
0.63818359375	15.9314569081437\\
0.638671875	15.9013071229463\\
0.63916015625	15.871221012088\\
0.6396484375	15.8411984433252\\
0.64013671875	15.8112392849584\\
0.640625	15.7813434058287\\
0.64111328125	15.7515106753157\\
0.6416015625	15.7217409633348\\
0.64208984375	15.692034140334\\
0.642578125	15.6623900772915\\
0.64306640625	15.6328086457132\\
0.6435546875	15.6032897176297\\
0.64404296875	15.5738331655939\\
0.64453125	15.5444388626781\\
0.64501953125	15.5151066824719\\
0.6455078125	15.485836499079\\
0.64599609375	15.456628187115\\
0.646484375	15.427481621705\\
0.64697265625	15.3983966784803\\
0.6474609375	15.369373233577\\
0.64794921875	15.3404111636323\\
0.6484375	15.3115103457832\\
0.64892578125	15.2826706576628\\
0.6494140625	15.2538919773989\\
0.64990234375	15.2251741836109\\
0.650390625	15.1965171554076\\
0.65087890625	15.1679207723848\\
0.6513671875	15.1393849146228\\
0.65185546875	15.1109094626842\\
0.65234375	15.0824942976111\\
0.65283203125	15.0541393009235\\
0.6533203125	15.0258443546161\\
0.65380859375	14.9976093411565\\
0.654296875	14.9694341434829\\
0.65478515625	14.9413186450015\\
0.6552734375	14.9132627295844\\
0.65576171875	14.8852662815675\\
0.65625	14.8573291857478\\
0.65673828125	14.8294513273816\\
0.6572265625	14.801632592182\\
0.65771484375	14.7738728663169\\
0.658203125	14.7461720364064\\
0.65869140625	14.7185299895214\\
0.6591796875	14.6909466131803\\
0.65966796875	14.663421795348\\
0.66015625	14.635955424433\\
0.66064453125	14.6085473892853\\
0.6611328125	14.5811975791949\\
0.66162109375	14.5539058838888\\
0.662109375	14.5266721935298\\
0.66259765625	14.4994963987136\\
0.6630859375	14.4723783904673\\
0.66357421875	14.4453180602473\\
0.6640625	14.4183152999369\\
0.66455078125	14.3913700018446\\
0.6650390625	14.3644820587019\\
0.66552734375	14.3376513636615\\
0.666015625	14.3108778102951\\
0.66650390625	14.2841612925914\\
0.6669921875	14.2575017049545\\
0.66748046875	14.2308989422015\\
0.66796875	14.2043528995605\\
0.66845703125	14.1778634726694\\
0.6689453125	14.1514305575731\\
0.66943359375	14.1250540507221\\
0.669921875	14.0987338489704\\
0.67041015625	14.0724698495737\\
0.6708984375	14.0462619501875\\
0.67138671875	14.0201100488653\\
0.671875	13.9940140440567\\
0.67236328125	13.9679738346054\\
0.6728515625	13.9419893197477\\
0.67333984375	13.9160603991103\\
0.673828125	13.8901869727089\\
0.67431640625	13.864368940946\\
0.6748046875	13.8386062046094\\
0.67529296875	13.8128986648703\\
0.67578125	13.7872462232817\\
0.67626953125	13.7616487817762\\
0.6767578125	13.7361062426649\\
0.67724609375	13.710618508635\\
0.677734375	13.6851854827487\\
0.67822265625	13.6598070684411\\
0.6787109375	13.6344831695184\\
0.67919921875	13.6092136901566\\
0.6796875	13.5839985348996\\
0.68017578125	13.5588376086575\\
0.6806640625	13.5337308167049\\
0.68115234375	13.5086780646793\\
0.681640625	13.4836792585798\\
0.68212890625	13.4587343047647\\
0.6826171875	13.4338431099506\\
0.68310546875	13.4090055812103\\
0.68359375	13.3842216259717\\
0.68408203125	13.3594911520156\\
0.6845703125	13.3348140674746\\
0.68505859375	13.3101902808312\\
0.685546875	13.2856197009164\\
0.68603515625	13.2611022369083\\
0.6865234375	13.2366377983301\\
0.68701171875	13.2122262950489\\
0.6875	13.1878676372742\\
0.68798828125	13.1635617355562\\
0.6884765625	13.1393085007843\\
0.68896484375	13.1151078441857\\
0.689453125	13.0909596773238\\
0.68994140625	13.0668639120969\\
0.6904296875	13.0428204607363\\
0.69091796875	13.0188292358055\\
0.69140625	12.994890150198\\
0.69189453125	12.9710031171364\\
0.6923828125	12.9471680501708\\
0.69287109375	12.9233848631771\\
0.693359375	12.8996534703561\\
0.69384765625	12.8759737862313\\
0.6943359375	12.8523457256485\\
0.69482421875	12.8287692037734\\
0.6953125	12.805244136091\\
0.69580078125	12.7817704384037\\
0.6962890625	12.7583480268301\\
0.69677734375	12.7349768178037\\
0.697265625	12.7116567280713\\
0.69775390625	12.688387674692\\
0.6982421875	12.6651695750354\\
0.69873046875	12.6420023467809\\
0.69921875	12.6188859079155\\
0.69970703125	12.5958201767333\\
0.7001953125	12.5728050718337\\
0.70068359375	12.5498405121204\\
0.701171875	12.5269264167996\\
0.70166015625	12.5040627053792\\
0.7021484375	12.4812492976676\\
0.70263671875	12.4584861137716\\
0.703125	12.4357730740962\\
0.70361328125	12.4131100993425\\
0.7041015625	12.3904971105069\\
0.70458984375	12.3679340288797\\
0.705078125	12.3454207760439\\
0.70556640625	12.3229572738737\\
0.7060546875	12.3005434445339\\
0.70654296875	12.278179210478\\
0.70703125	12.2558644944471\\
0.70751953125	12.2335992194692\\
0.7080078125	12.2113833088575\\
0.70849609375	12.1892166862093\\
0.708984375	12.1670992754049\\
0.70947265625	12.1450310006063\\
0.7099609375	12.1230117862562\\
0.71044921875	12.1010415570767\\
0.7109375	12.0791202380681\\
0.71142578125	12.0572477545078\\
0.7119140625	12.0354240319494\\
0.71240234375	12.0136489962209\\
0.712890625	11.9919225734243\\
0.71337890625	11.970244689934\\
0.7138671875	11.9486152723959\\
0.71435546875	11.9270342477261\\
0.71484375	11.9055015431099\\
0.71533203125	11.8840170860009\\
0.7158203125	11.8625808041193\\
0.71630859375	11.8411926254514\\
0.716796875	11.8198524782484\\
0.71728515625	11.798560291025\\
0.7177734375	11.7773159925588\\
0.71826171875	11.7561195118886\\
0.71875	11.734970778314\\
0.71923828125	11.713869721394\\
0.7197265625	11.6928162709459\\
0.72021484375	11.6718103570444\\
0.720703125	11.6508519100206\\
0.72119140625	11.6299408604608\\
0.7216796875	11.6090771392054\\
0.72216796875	11.5882606773483\\
0.72265625	11.5674914062353\\
0.72314453125	11.5467692574637\\
0.7236328125	11.5260941628807\\
0.72412109375	11.5054660545828\\
0.724609375	11.4848848649148\\
0.72509765625	11.4643505264685\\
0.7255859375	11.443862972082\\
0.72607421875	11.4234221348387\\
0.7265625	11.403027948066\\
0.72705078125	11.3826803453349\\
0.7275390625	11.3623792604584\\
0.72802734375	11.3421246274913\\
0.728515625	11.3219163807283\\
0.72900390625	11.301754454704\\
0.7294921875	11.2816387841912\\
0.72998046875	11.2615693042004\\
0.73046875	11.2415459499787\\
0.73095703125	11.2215686570091\\
0.7314453125	11.2016373610092\\
0.73193359375	11.1817519979305\\
0.732421875	11.1619125039575\\
0.73291015625	11.1421188155068\\
0.7333984375	11.1223708692261\\
0.73388671875	11.1026686019932\\
0.734375	11.0830119509157\\
0.73486328125	11.0634008533292\\
0.7353515625	11.0438352467971\\
0.73583984375	11.0243150691096\\
0.736328125	11.0048402582825\\
0.73681640625	10.9854107525568\\
0.7373046875	10.9660264903973\\
0.73779296875	10.9466874104924\\
0.73828125	10.9273934517527\\
0.73876953125	10.9081445533103\\
0.7392578125	10.8889406545181\\
0.73974609375	10.8697816949486\\
0.740234375	10.8506676143937\\
0.74072265625	10.8315983528631\\
0.7412109375	10.8125738505841\\
0.74169921875	10.7935940480003\\
0.7421875	10.7746588857712\\
0.74267578125	10.7557683047711\\
0.7431640625	10.7369222460884\\
0.74365234375	10.7181206510246\\
0.744140625	10.699363461094\\
0.74462890625	10.6806506180222\\
0.7451171875	10.6619820637459\\
0.74560546875	10.6433577404118\\
0.74609375	10.624777590376\\
0.74658203125	10.6062415562029\\
0.7470703125	10.5877495806648\\
0.74755859375	10.569301606741\\
0.748046875	10.5508975776168\\
0.74853515625	10.5325374366832\\
0.7490234375	10.5142211275355\\
0.74951171875	10.4959485939733\\
0.75	10.477719779999\\
0.75048828125	10.4595346298178\\
0.7509765625	10.4413930878361\\
0.75146484375	10.4232950986617\\
0.751953125	10.4052406071021\\
0.75244140625	10.3872295581647\\
0.7529296875	10.3692618970554\\
0.75341796875	10.3513375691779\\
0.75390625	10.3334565201336\\
0.75439453125	10.3156186957201\\
0.7548828125	10.2978240419311\\
0.75537109375	10.2800725049551\\
0.755859375	10.2623640311753\\
0.75634765625	10.2446985671686\\
0.7568359375	10.2270760597047\\
0.75732421875	10.2094964557459\\
0.7578125	10.1919597024459\\
0.75830078125	10.1744657471494\\
0.7587890625	10.1570145373915\\
0.75927734375	10.1396060208967\\
0.759765625	10.1222401455785\\
0.76025390625	10.1049168595384\\
0.7607421875	10.0876361110658\\
0.76123046875	10.0703978486367\\
0.76171875	10.0532020209134\\
0.76220703125	10.0360485767438\\
0.7626953125	10.0189374651607\\
0.76318359375	10.001868635381\\
0.763671875	9.98484203680534\\
0.76416015625	9.96785761901727\\
0.7646484375	9.95091533178263\\
0.76513671875	9.93401512504892\\
0.765625	9.91715694894467\\
0.76611328125	9.9003407537788\\
0.7666015625	9.88356649003999\\
0.76708984375	9.86683410839603\\
0.767578125	9.85014355969324\\
0.76806640625	9.83349479495584\\
0.7685546875	9.81688776538528\\
0.76904296875	9.8003224223597\\
0.76953125	9.78379871743326\\
0.77001953125	9.76731660233556\\
0.7705078125	9.75087602897103\\
0.77099609375	9.73447694941834\\
0.771484375	9.71811931592977\\
0.77197265625	9.70180308093064\\
0.7724609375	9.6855281970187\\
0.77294921875	9.66929461696358\\
0.7734375	9.65310229370613\\
0.77392578125	9.63695118035792\\
0.7744140625	9.62084123020059\\
0.77490234375	9.60477239668531\\
0.775390625	9.58874463343219\\
0.77587890625	9.57275789422975\\
0.7763671875	9.55681213303425\\
0.77685546875	9.54090730396925\\
0.77734375	9.52504336132496\\
0.77783203125	9.50922025955768\\
0.7783203125	9.49343795328933\\
0.77880859375	9.47769639730677\\
0.779296875	9.46199554656136\\
0.77978515625	9.44633535616833\\
0.7802734375	9.43071578140628\\
0.78076171875	9.41513677771663\\
0.78125	9.39959830070307\\
0.78173828125	9.38410030613101\\
0.7822265625	9.36864274992707\\
0.78271484375	9.35322558817854\\
0.783203125	9.33784877713286\\
0.78369140625	9.32251227319703\\
0.7841796875	9.30721603293721\\
0.78466796875	9.29196001307807\\
0.78515625	9.27674417050234\\
0.78564453125	9.2615684622503\\
0.7861328125	9.24643284551921\\
0.78662109375	9.23133727766286\\
0.787109375	9.21628171619101\\
0.78759765625	9.20126611876895\\
0.7880859375	9.18629044321694\\
0.78857421875	9.1713546475097\\
0.7890625	9.15645868977597\\
0.78955078125	9.14160252829796\\
0.7900390625	9.12678612151089\\
0.79052734375	9.11200942800248\\
0.791015625	9.09727240651249\\
0.79150390625	9.08257501593218\\
0.7919921875	9.06791721530388\\
0.79248046875	9.05329896382049\\
0.79296875	9.038720220825\\
0.79345703125	9.02418094580999\\
0.7939453125	9.00968109841722\\
0.79443359375	8.9952206384371\\
0.794921875	8.98079952580822\\
0.79541015625	8.96641772061695\\
0.7958984375	8.95207518309688\\
0.79638671875	8.93777187362845\\
0.796875	8.92350775273841\\
0.79736328125	8.90928278109944\\
0.7978515625	8.89509691952962\\
0.79833984375	8.88095012899204\\
0.798828125	8.86684237059429\\
0.79931640625	8.8527736055881\\
0.7998046875	8.83874379536879\\
0.80029296875	8.82475290147489\\
0.80078125	8.81080088558769\\
0.80126953125	8.79688770953079\\
0.8017578125	8.78301333526966\\
0.80224609375	8.76917772491123\\
0.802734375	8.75538084070345\\
0.80322265625	8.74162264503481\\
0.8037109375	8.72790310043399\\
0.80419921875	8.71422216956936\\
0.8046875	8.70057981524864\\
0.80517578125	8.68697600041842\\
0.8056640625	8.6734106881637\\
0.80615234375	8.65988384170759\\
0.806640625	8.64639542441076\\
0.80712890625	8.63294539977118\\
0.8076171875	8.61953373142354\\
0.80810546875	8.606160383139\\
0.80859375	8.59282531882463\\
0.80908203125	8.57952850252316\\
0.8095703125	8.56626989841243\\
0.81005859375	8.55304947080509\\
0.810546875	8.53986718414817\\
0.81103515625	8.52672300302269\\
0.8115234375	8.51361689214325\\
0.81201171875	8.50054881635763\\
0.8125	8.48751874064642\\
0.81298828125	8.47452663012263\\
0.8134765625	8.46157245003129\\
0.81396484375	8.44865616574909\\
0.814453125	8.43577774278397\\
0.81494140625	8.4229371467747\\
0.8154296875	8.41013434349063\\
0.81591796875	8.39736929883119\\
0.81640625	8.38464197882552\\
0.81689453125	8.37195234963218\\
0.8173828125	8.35930037753873\\
0.81787109375	8.34668602896132\\
0.818359375	8.33410927044439\\
0.81884765625	8.32157006866027\\
0.8193359375	8.30906839040883\\
0.81982421875	8.29660420261707\\
0.8203125	8.28417747233885\\
0.82080078125	8.27178816675442\\
0.8212890625	8.25943625317018\\
0.82177734375	8.24712169901822\\
0.822265625	8.23484447185605\\
0.82275390625	8.22260453936616\\
0.8232421875	8.21040186935577\\
0.82373046875	8.19823642975641\\
0.82421875	8.18610818862359\\
0.82470703125	8.17401711413646\\
0.8251953125	8.1619631745975\\
0.82568359375	8.14994633843211\\
0.826171875	8.13796657418831\\
0.82666015625	8.12602385053642\\
0.8271484375	8.11411813626869\\
0.82763671875	8.10224940029898\\
0.828125	8.09041761166243\\
0.82861328125	8.0786227395151\\
0.8291015625	8.0668647531337\\
0.82958984375	8.05514362191523\\
0.830078125	8.0434593153766\\
0.83056640625	8.03181180315443\\
0.8310546875	8.02020105500458\\
0.83154296875	8.00862704080198\\
0.83203125	7.99708973054016\\
0.83251953125	7.98558909433106\\
0.8330078125	7.97412510240462\\
0.83349609375	7.96269772510855\\
0.833984375	7.95130693290793\\
0.83447265625	7.93995269638494\\
0.8349609375	7.92863498623859\\
0.83544921875	7.91735377328434\\
0.8359375	7.90610902845384\\
0.83642578125	7.89490072279458\\
0.8369140625	7.88372882746966\\
0.83740234375	7.87259331375741\\
0.837890625	7.86149415305115\\
0.83837890625	7.85043131685883\\
0.8388671875	7.83940477680279\\
0.83935546875	7.82841450461947\\
0.83984375	7.817460472159\\
0.84033203125	7.8065426513851\\
0.8408203125	7.79566101437459\\
0.84130859375	7.78481553331725\\
0.841796875	7.77400618051544\\
0.84228515625	7.76323292838387\\
0.8427734375	7.75249574944926\\
0.84326171875	7.74179461635012\\
0.84375	7.73112950183641\\
0.84423828125	7.7205003787693\\
0.8447265625	7.70990722012085\\
0.84521484375	7.69934999897377\\
0.845703125	7.68882868852112\\
0.84619140625	7.67834326206603\\
0.8466796875	7.66789369302148\\
0.84716796875	7.65747995490994\\
0.84765625	7.64710202136313\\
0.84814453125	7.63675986612185\\
0.8486328125	7.62645346303551\\
0.84912109375	7.61618278606206\\
0.849609375	7.60594780926762\\
0.85009765625	7.5957485068262\\
0.8505859375	7.58558485301954\\
0.85107421875	7.57545682223673\\
0.8515625	7.56536438897401\\
0.85205078125	7.55530752783453\\
0.8525390625	7.54528621352806\\
0.85302734375	7.53530042087071\\
0.853515625	7.52535012478472\\
0.85400390625	7.5154353002982\\
0.8544921875	7.50555592254487\\
0.85498046875	7.49571196676379\\
0.85546875	7.48590340829913\\
0.85595703125	7.47613022259993\\
0.8564453125	7.46639238521984\\
0.85693359375	7.45668987181689\\
0.857421875	7.44702265815319\\
0.85791015625	7.43739072009476\\
0.8583984375	7.42779403361126\\
0.85888671875	7.41823257477573\\
0.859375	7.40870631976439\\
0.85986328125	7.39921524485634\\
0.8603515625	7.38975932643338\\
0.86083984375	7.38033854097979\\
0.861328125	7.37095286508199\\
0.86181640625	7.36160227542845\\
0.8623046875	7.35228674880933\\
0.86279296875	7.34300626211634\\
0.86328125	7.33376079234246\\
0.86376953125	7.32455031658174\\
0.8642578125	7.31537481202905\\
0.86474609375	7.30623425597986\\
0.865234375	7.29712862583004\\
0.86572265625	7.28805789907559\\
0.8662109375	7.27902205331245\\
0.86669921875	7.2700210662363\\
0.8671875	7.26105491564229\\
0.86767578125	7.25212357942484\\
0.8681640625	7.24322703557742\\
0.86865234375	7.23436526219238\\
0.869140625	7.22553823746065\\
0.86962890625	7.2167459396716\\
0.8701171875	7.20798834721277\\
0.87060546875	7.19926543856972\\
0.87109375	7.19057719232578\\
0.87158203125	7.1819235871618\\
0.8720703125	7.17330460185607\\
0.87255859375	7.16472021528394\\
0.873046875	7.15617040641776\\
0.87353515625	7.14765515432662\\
0.8740234375	7.13917443817612\\
0.87451171875	7.1307282372282\\
0.875	7.12231653084095\\
0.87548828125	7.11393929846837\\
0.8759765625	7.1055965196602\\
0.87646484375	7.09728817406169\\
0.876953125	7.08901424141348\\
0.87744140625	7.0807747015513\\
0.8779296875	7.07256953440583\\
0.87841796875	7.06439872000252\\
0.87890625	7.05626223846135\\
0.87939453125	7.04816006999671\\
0.8798828125	7.04009219491711\\
0.88037109375	7.03205859362508\\
0.880859375	7.02405924661696\\
0.88134765625	7.01609413448264\\
0.8818359375	7.00816323790548\\
0.88232421875	7.00026653766208\\
0.8828125	6.99240401462205\\
0.88330078125	6.98457564974791\\
0.8837890625	6.97678142409487\\
0.88427734375	6.96902131881058\\
0.884765625	6.96129531513511\\
0.88525390625	6.95360339440058\\
0.8857421875	6.94594553803117\\
0.88623046875	6.93832172754276\\
0.88671875	6.9307319445429\\
0.88720703125	6.92317617073057\\
0.8876953125	6.915654387896\\
0.88818359375	6.9081665779205\\
0.888671875	6.90071272277636\\
0.88916015625	6.89329280452654\\
0.8896484375	6.88590680532461\\
0.89013671875	6.87855470741455\\
0.890625	6.87123649313057\\
0.89111328125	6.86395214489696\\
0.8916015625	6.85670164522792\\
0.89208984375	6.84948497672738\\
0.892578125	6.84230212208882\\
0.89306640625	6.8351530640952\\
0.8935546875	6.82803778561868\\
0.89404296875	6.8209562696205\\
0.89453125	6.81390849915084\\
0.89501953125	6.80689445734868\\
0.8955078125	6.79991412744155\\
0.89599609375	6.79296749274549\\
0.896484375	6.7860545366648\\
0.89697265625	6.77917524269192\\
0.8974609375	6.77232959440728\\
0.89794921875	6.76551757547916\\
0.8984375	6.75873916966347\\
0.89892578125	6.75199436080368\\
0.8994140625	6.74528313283064\\
0.89990234375	6.73860546976238\\
0.900390625	6.73196135570407\\
0.90087890625	6.72535077484774\\
0.9013671875	6.71877371147224\\
0.90185546875	6.71223014994303\\
0.90234375	6.70572007471208\\
0.90283203125	6.69924347031767\\
0.9033203125	6.69280032138429\\
0.90380859375	6.68639061262248\\
0.904296875	6.6800143288287\\
0.90478515625	6.6736714548852\\
0.9052734375	6.66736197575983\\
0.90576171875	6.66108587650591\\
0.90625	6.65484314226217\\
0.90673828125	6.64863375825251\\
0.9072265625	6.64245770978594\\
0.90771484375	6.63631498225637\\
0.908203125	6.63020556114257\\
0.90869140625	6.62412943200793\\
0.9091796875	6.6180865805004\\
0.90966796875	6.61207699235236\\
0.91015625	6.60610065338044\\
0.91064453125	6.6001575494854\\
0.9111328125	6.59424766665204\\
0.91162109375	6.58837099094906\\
0.912109375	6.58252750852886\\
0.91259765625	6.57671720562754\\
0.9130859375	6.57094006856465\\
0.91357421875	6.56519608374312\\
0.9140625	6.55948523764918\\
0.91455078125	6.55380751685213\\
0.9150390625	6.54816290800431\\
0.91552734375	6.54255139784094\\
0.916015625	6.53697297317999\\
0.91650390625	6.53142762092208\\
0.9169921875	6.52591532805036\\
0.91748046875	6.52043608163035\\
0.91796875	6.51498986880989\\
0.91845703125	6.50957667681896\\
0.9189453125	6.50419649296961\\
0.91943359375	6.49884930465584\\
0.919921875	6.49353509935339\\
0.92041015625	6.48825386461979\\
0.9208984375	6.48300558809414\\
0.92138671875	6.47779025749699\\
0.921875	6.47260786063029\\
0.92236328125	6.46745838537723\\
0.9228515625	6.46234181970217\\
0.92333984375	6.45725815165045\\
0.923828125	6.45220736934842\\
0.92431640625	6.44718946100317\\
0.9248046875	6.44220441490257\\
0.92529296875	6.43725221941505\\
0.92578125	6.43233286298958\\
0.92626953125	6.42744633415549\\
0.9267578125	6.42259262152243\\
0.92724609375	6.41777171378024\\
0.927734375	6.41298359969883\\
0.92822265625	6.40822826812812\\
0.9287109375	6.4035057079979\\
0.92919921875	6.39881590831775\\
0.9296875	6.39415885817695\\
0.93017578125	6.38953454674434\\
0.9306640625	6.38494296326829\\
0.93115234375	6.38038409707653\\
0.931640625	6.3758579375761\\
0.93212890625	6.37136447425324\\
0.9326171875	6.36690369667331\\
0.93310546875	6.36247559448068\\
0.93359375	6.35808015739863\\
0.93408203125	6.35371737522927\\
0.9345703125	6.34938723785346\\
0.93505859375	6.34508973523069\\
0.935546875	6.34082485739904\\
0.93603515625	6.336592594475\\
0.9365234375	6.33239293665349\\
0.93701171875	6.32822587420769\\
0.9375	6.32409139748901\\
0.93798828125	6.31998949692695\\
0.9384765625	6.31592016302902\\
0.93896484375	6.31188338638077\\
0.939453125	6.3078791576455\\
0.93994140625	6.30390746756436\\
0.9404296875	6.29996830695616\\
0.94091796875	6.29606166671735\\
0.94140625	6.29218753782189\\
0.94189453125	6.2883459113212\\
0.9423828125	6.28453677834409\\
0.94287109375	6.28076013009662\\
0.943359375	6.27701595786209\\
0.94384765625	6.27330425300098\\
0.9443359375	6.26962500695075\\
0.94482421875	6.26597821122588\\
0.9453125	6.26236385741778\\
0.94580078125	6.25878193719465\\
0.9462890625	6.25523244230149\\
0.94677734375	6.25171536455996\\
0.947265625	6.24823069586832\\
0.94775390625	6.24477842820142\\
0.9482421875	6.24135855361053\\
0.94873046875	6.23797106422335\\
0.94921875	6.2346159522439\\
0.94970703125	6.23129320995245\\
0.9501953125	6.2280028297055\\
0.95068359375	6.2247448039356\\
0.951171875	6.22151912515143\\
0.95166015625	6.21832578593762\\
0.9521484375	6.21516477895476\\
0.95263671875	6.21203609693924\\
0.953125	6.20893973270332\\
};
\addplot [color=mycolor1,solid]
  table[row sep=crcr]{0.953125	6.20893973270332\\
0.95361328125	6.20587567913492\\
0.9541015625	6.2028439291977\\
0.95458984375	6.19984447593085\\
0.955078125	6.19687731244917\\
0.95556640625	6.19394243194293\\
0.9560546875	6.19103982767777\\
0.95654296875	6.18816949299479\\
0.95703125	6.18533142131031\\
0.95751953125	6.18252560611595\\
0.9580078125	6.17975204097849\\
0.95849609375	6.17701071953987\\
0.958984375	6.17430163551709\\
0.95947265625	6.17162478270217\\
0.9599609375	6.16898015496211\\
0.96044921875	6.16636774623881\\
0.9609375	6.16378755054904\\
0.96142578125	6.16123956198438\\
0.9619140625	6.15872377471116\\
0.96240234375	6.15624018297042\\
0.962890625	6.15378878107782\\
0.96337890625	6.15136956342368\\
0.9638671875	6.14898252447284\\
0.96435546875	6.14662765876464\\
0.96484375	6.14430496091289\\
0.96533203125	6.14201442560581\\
0.9658203125	6.13975604760598\\
0.96630859375	6.1375298217503\\
0.966796875	6.13533574294994\\
0.96728515625	6.13317380619029\\
0.9677734375	6.13104400653095\\
0.96826171875	6.12894633910561\\
0.96875	6.12688079912212\\
0.96923828125	6.12484738186233\\
0.9697265625	6.12284608268213\\
0.97021484375	6.1208768970114\\
0.970703125	6.1189398203539\\
0.97119140625	6.11703484828734\\
0.9716796875	6.11516197646326\\
0.97216796875	6.11332120060699\\
0.97265625	6.11151251651772\\
0.97314453125	6.10973592006828\\
0.9736328125	6.10799140720528\\
0.97412109375	6.10627897394898\\
0.974609375	6.10459861639325\\
0.97509765625	6.10295033070561\\
0.9755859375	6.1013341131271\\
0.97607421875	6.09974995997235\\
0.9765625	6.09819786762942\\
0.97705078125	6.09667783255992\\
0.9775390625	6.09518985129885\\
0.97802734375	6.09373392045463\\
0.978515625	6.09231003670907\\
0.97900390625	6.09091819681732\\
0.9794921875	6.08955839760787\\
0.97998046875	6.08823063598249\\
0.98046875	6.08693490891623\\
0.98095703125	6.08567121345736\\
0.9814453125	6.0844395467274\\
0.98193359375	6.08323990592104\\
0.982421875	6.08207228830614\\
0.98291015625	6.0809366912237\\
0.9833984375	6.07983311208785\\
0.98388671875	6.0787615483858\\
0.984375	6.07772199767785\\
0.98486328125	6.07671445759736\\
0.9853515625	6.07573892585069\\
0.98583984375	6.07479540021726\\
0.986328125	6.07388387854945\\
0.98681640625	6.0730043587726\\
0.9873046875	6.07215683888508\\
0.98779296875	6.07134131695812\\
0.98828125	6.07055779113592\\
0.98876953125	6.06980625963554\\
0.9892578125	6.06908672074699\\
0.98974609375	6.06839917283312\\
0.990234375	6.06774361432966\\
0.99072265625	6.06712004374517\\
0.9912109375	6.06652845966103\\
0.99169921875	6.06596886073151\\
0.9921875	6.0654412456836\\
0.99267578125	6.06494561331718\\
0.9931640625	6.06448196250484\\
0.99365234375	6.064050292192\\
0.994140625	6.06365060139686\\
0.99462890625	6.06328288921033\\
0.9951171875	6.0629471547961\\
0.99560546875	6.06264339739062\\
0.99609375	6.06237161630308\\
0.99658203125	6.06213181091537\\
0.9970703125	6.06192398068215\\
0.99755859375	6.06174812513078\\
0.998046875	6.06160424386135\\
0.99853515625	6.06149233654665\\
0.9990234375	6.06141240293221\\
0.99951171875	6.06136444283627\\
};
\addlegendentry{AR(6) Model};

\addplot [color=mycolor2,solid,forget plot]
  table[row sep=crcr]{-1	5.84850287090141\\
-0.99951171875	5.84851819003698\\
-0.9990234375	5.848564147482\\
-0.99853515625	5.84864074335139\\
-0.998046875	5.84874797783672\\
-0.99755859375	5.84888585120612\\
-0.9970703125	5.84905436380443\\
-0.99658203125	5.84925351605302\\
-0.99609375	5.84948330844995\\
-0.99560546875	5.84974374156992\\
-0.9951171875	5.85003481606423\\
-0.99462890625	5.85035653266084\\
-0.994140625	5.85070889216437\\
-0.99365234375	5.85109189545608\\
-0.9931640625	5.85150554349392\\
-0.99267578125	5.85194983731247\\
-0.9921875	5.85242477802302\\
-0.99169921875	5.85293036681356\\
-0.9912109375	5.85346660494875\\
-0.99072265625	5.85403349376996\\
-0.990234375	5.85463103469532\\
-0.98974609375	5.85525922921963\\
-0.9892578125	5.8559180789145\\
-0.98876953125	5.85660758542826\\
-0.98828125	5.85732775048601\\
-0.98779296875	5.85807857588966\\
-0.9873046875	5.8588600635179\\
-0.98681640625	5.85967221532622\\
-0.986328125	5.86051503334701\\
-0.98583984375	5.86138851968942\\
-0.9853515625	5.86229267653955\\
-0.98486328125	5.86322750616031\\
-0.984375	5.86419301089156\\
-0.98388671875	5.86518919315008\\
-0.9833984375	5.86621605542959\\
-0.98291015625	5.86727360030075\\
-0.982421875	5.86836183041124\\
-0.98193359375	5.86948074848572\\
-0.9814453125	5.87063035732589\\
-0.98095703125	5.87181065981052\\
-0.98046875	5.87302165889543\\
-0.97998046875	5.87426335761353\\
-0.9794921875	5.8755357590749\\
-0.97900390625	5.87683886646675\\
-0.978515625	5.87817268305345\\
-0.97802734375	5.8795372121766\\
-0.9775390625	5.88093245725504\\
-0.97705078125	5.88235842178485\\
-0.9765625	5.88381510933939\\
-0.97607421875	5.88530252356938\\
-0.9755859375	5.88682066820287\\
-0.97509765625	5.88836954704528\\
-0.974609375	5.88994916397946\\
-0.97412109375	5.89155952296569\\
-0.9736328125	5.89320062804175\\
-0.97314453125	5.89487248332292\\
-0.97265625	5.89657509300202\\
-0.97216796875	5.89830846134945\\
-0.9716796875	5.90007259271326\\
-0.97119140625	5.90186749151911\\
-0.970703125	5.90369316227036\\
-0.97021484375	5.90554960954812\\
-0.9697265625	5.90743683801122\\
-0.96923828125	5.90935485239636\\
-0.96875	5.91130365751802\\
-0.96826171875	5.9132832582686\\
-0.9677734375	5.9152936596184\\
-0.96728515625	5.91733486661571\\
-0.966796875	5.91940688438682\\
-0.96630859375	5.92150971813606\\
-0.9658203125	5.92364337314586\\
-0.96533203125	5.92580785477679\\
-0.96484375	5.92800316846761\\
-0.96435546875	5.93022931973529\\
-0.9638671875	5.93248631417509\\
-0.96337890625	5.93477415746058\\
-0.962890625	5.93709285534371\\
-0.96240234375	5.93944241365484\\
-0.9619140625	5.94182283830277\\
-0.96142578125	5.94423413527487\\
-0.9609375	5.94667631063702\\
-0.96044921875	5.94914937053376\\
-0.9599609375	5.95165332118824\\
-0.95947265625	5.95418816890241\\
-0.958984375	5.95675392005691\\
-0.95849609375	5.95935058111125\\
-0.9580078125	5.96197815860381\\
-0.95751953125	5.96463665915189\\
-0.95703125	5.96732608945181\\
-0.95654296875	5.97004645627891\\
-0.9560546875	5.97279776648763\\
-0.95556640625	5.9755800270116\\
-0.955078125	5.97839324486362\\
-0.95458984375	5.98123742713581\\
-0.9541015625	5.9841125809996\\
-0.95361328125	5.98701871370583\\
-0.953125	5.9899558325848\\
-0.95263671875	5.99292394504631\\
-0.9521484375	5.99592305857977\\
-0.95166015625	5.9989531807542\\
-0.951171875	6.00201431921835\\
-0.95068359375	6.00510648170076\\
-0.9501953125	6.00822967600978\\
-0.94970703125	6.01138391003364\\
-0.94921875	6.01456919174059\\
-0.94873046875	6.01778552917891\\
-0.9482421875	6.02103293047696\\
-0.94775390625	6.02431140384328\\
-0.947265625	6.02762095756666\\
-0.94677734375	6.03096160001621\\
-0.9462890625	6.03433333964141\\
-0.94580078125	6.03773618497218\\
-0.9453125	6.04117014461901\\
-0.94482421875	6.04463522727296\\
-0.9443359375	6.04813144170577\\
-0.94384765625	6.05165879676993\\
-0.943359375	6.05521730139873\\
-0.94287109375	6.05880696460641\\
-0.9423828125	6.06242779548812\\
-0.94189453125	6.06607980322012\\
-0.94140625	6.06976299705975\\
-0.94091796875	6.0734773863456\\
-0.9404296875	6.07722298049753\\
-0.93994140625	6.08099978901675\\
-0.939453125	6.08480782148595\\
-0.93896484375	6.08864708756933\\
-0.9384765625	6.0925175970127\\
-0.93798828125	6.09641935964359\\
-0.9375	6.10035238537125\\
-0.93701171875	6.10431668418688\\
-0.9365234375	6.10831226616355\\
-0.93603515625	6.11233914145641\\
-0.935546875	6.1163973203027\\
-0.93505859375	6.12048681302188\\
-0.9345703125	6.12460763001573\\
-0.93408203125	6.12875978176838\\
-0.93359375	6.13294327884645\\
-0.93310546875	6.13715813189911\\
-0.9326171875	6.14140435165821\\
-0.93212890625	6.14568194893834\\
-0.931640625	6.14999093463689\\
-0.93115234375	6.15433131973427\\
-0.9306640625	6.15870311529384\\
-0.93017578125	6.1631063324621\\
-0.9296875	6.16754098246879\\
-0.92919921875	6.17200707662696\\
-0.9287109375	6.17650462633304\\
-0.92822265625	6.18103364306702\\
-0.927734375	6.18559413839244\\
-0.92724609375	6.19018612395661\\
-0.9267578125	6.19480961149059\\
-0.92626953125	6.1994646128094\\
-0.92578125	6.20415113981203\\
-0.92529296875	6.20886920448162\\
-0.9248046875	6.21361881888548\\
-0.92431640625	6.21839999517529\\
-0.923828125	6.22321274558713\\
-0.92333984375	6.22805708244161\\
-0.9228515625	6.23293301814399\\
-0.92236328125	6.23784056518427\\
-0.921875	6.24277973613731\\
-0.92138671875	6.24775054366291\\
-0.9208984375	6.25275300050598\\
-0.92041015625	6.25778711949655\\
-0.919921875	6.26285291355004\\
-0.91943359375	6.26795039566718\\
-0.9189453125	6.27307957893426\\
-0.91845703125	6.27824047652324\\
-0.91796875	6.28343310169176\\
-0.91748046875	6.28865746778337\\
-0.9169921875	6.29391358822756\\
-0.91650390625	6.29920147653995\\
-0.916015625	6.30452114632238\\
-0.91552734375	6.30987261126297\\
-0.9150390625	6.31525588513634\\
-0.91455078125	6.32067098180368\\
-0.9140625	6.32611791521284\\
-0.91357421875	6.33159669939853\\
-0.9130859375	6.33710734848236\\
-0.91259765625	6.34264987667302\\
-0.912109375	6.34822429826638\\
-0.91162109375	6.35383062764564\\
-0.9111328125	6.3594688792814\\
-0.91064453125	6.36513906773185\\
-0.91015625	6.37084120764289\\
-0.90966796875	6.3765753137482\\
-0.9091796875	6.38234140086944\\
-0.90869140625	6.38813948391635\\
-0.908203125	6.39396957788688\\
-0.90771484375	6.3998316978673\\
-0.9072265625	6.4057258590324\\
-0.90673828125	6.41165207664556\\
-0.90625	6.4176103660589\\
-0.90576171875	6.42360074271345\\
-0.9052734375	6.42962322213923\\
-0.90478515625	6.4356778199554\\
-0.904296875	6.44176455187051\\
-0.90380859375	6.44788343368242\\
-0.9033203125	6.45403448127865\\
-0.90283203125	6.4602177106364\\
-0.90234375	6.46643313782273\\
-0.90185546875	6.47268077899469\\
-0.9013671875	6.47896065039953\\
-0.90087890625	6.48527276837472\\
-0.900390625	6.4916171493482\\
-0.89990234375	6.49799380983846\\
-0.8994140625	6.50440276645478\\
-0.89892578125	6.51084403589724\\
-0.8984375	6.51731763495701\\
-0.89794921875	6.52382358051638\\
-0.8974609375	6.53036188954902\\
-0.89697265625	6.53693257912004\\
-0.896484375	6.54353566638619\\
-0.89599609375	6.55017116859602\\
-0.8955078125	6.55683910309\\
-0.89501953125	6.56353948730073\\
-0.89453125	6.570272338753\\
-0.89404296875	6.57703767506409\\
-0.8935546875	6.58383551394379\\
-0.89306640625	6.59066587319463\\
-0.892578125	6.59752877071206\\
-0.89208984375	6.60442422448454\\
-0.8916015625	6.61135225259376\\
-0.89111328125	6.61831287321481\\
-0.890625	6.62530610461626\\
-0.89013671875	6.63233196516047\\
-0.8896484375	6.63939047330359\\
-0.88916015625	6.64648164759588\\
-0.888671875	6.65360550668175\\
-0.88818359375	6.66076206930004\\
-0.8876953125	6.6679513542841\\
-0.88720703125	6.67517338056201\\
-0.88671875	6.68242816715677\\
-0.88623046875	6.68971573318641\\
-0.8857421875	6.6970360978642\\
-0.88525390625	6.70438928049883\\
-0.884765625	6.71177530049462\\
-0.88427734375	6.7191941773516\\
-0.8837890625	6.7266459306658\\
-0.88330078125	6.73413058012933\\
-0.8828125	6.74164814553065\\
-0.88232421875	6.74919864675468\\
-0.8818359375	6.75678210378302\\
-0.88134765625	6.76439853669413\\
-0.880859375	6.77204796566351\\
-0.88037109375	6.77973041096387\\
-0.8798828125	6.78744589296537\\
-0.87939453125	6.79519443213573\\
-0.87890625	6.8029760490405\\
-0.87841796875	6.81079076434317\\
-0.8779296875	6.81863859880543\\
-0.87744140625	6.82651957328731\\
-0.876953125	6.8344337087474\\
-0.87646484375	6.84238102624309\\
-0.8759765625	6.85036154693063\\
-0.87548828125	6.85837529206549\\
-0.875	6.86642228300243\\
-0.87451171875	6.87450254119578\\
-0.8740234375	6.88261608819961\\
-0.87353515625	6.89076294566789\\
-0.873046875	6.89894313535479\\
-0.87255859375	6.90715667911477\\
-0.8720703125	6.91540359890289\\
-0.87158203125	6.92368391677494\\
-0.87109375	6.93199765488767\\
-0.87060546875	6.94034483549899\\
-0.8701171875	6.94872548096823\\
-0.86962890625	6.95713961375626\\
-0.869140625	6.96558725642579\\
-0.86865234375	6.97406843164152\\
-0.8681640625	6.98258316217038\\
-0.86767578125	6.99113147088173\\
-0.8671875	6.99971338074762\\
-0.86669921875	7.00832891484296\\
-0.8662109375	7.01697809634572\\
-0.86572265625	7.02566094853724\\
-0.865234375	7.03437749480234\\
-0.86474609375	7.04312775862961\\
-0.8642578125	7.05191176361162\\
-0.86376953125	7.06072953344517\\
-0.86328125	7.06958109193145\\
-0.86279296875	7.07846646297629\\
-0.8623046875	7.08738567059047\\
-0.86181640625	7.09633873888978\\
-0.861328125	7.10532569209547\\
-0.86083984375	7.11434655453428\\
-0.8603515625	7.12340135063875\\
-0.85986328125	7.13249010494754\\
-0.859375	7.1416128421055\\
-0.85888671875	7.15076958686402\\
-0.8583984375	7.15996036408128\\
-0.85791015625	7.16918519872238\\
-0.857421875	7.17844411585972\\
-0.85693359375	7.18773714067312\\
-0.8564453125	7.19706429845012\\
-0.85595703125	7.20642561458624\\
-0.85546875	7.21582111458521\\
-0.85498046875	7.22525082405921\\
-0.8544921875	7.23471476872908\\
-0.85400390625	7.24421297442464\\
-0.853515625	7.25374546708492\\
-0.85302734375	7.26331227275839\\
-0.8525390625	7.27291341760322\\
-0.85205078125	7.28254892788757\\
-0.8515625	7.29221882998979\\
-0.85107421875	7.3019231503987\\
-0.8505859375	7.31166191571388\\
-0.85009765625	7.32143515264588\\
-0.849609375	7.33124288801654\\
-0.84912109375	7.34108514875919\\
-0.8486328125	7.35096196191894\\
-0.84814453125	7.36087335465298\\
-0.84765625	7.37081935423079\\
-0.84716796875	7.38079998803447\\
-0.8466796875	7.39081528355891\\
-0.84619140625	7.40086526841221\\
-0.845703125	7.41094997031581\\
-0.84521484375	7.42106941710486\\
-0.8447265625	7.43122363672845\\
-0.84423828125	7.44141265724988\\
-0.84375	7.45163650684699\\
-0.84326171875	7.46189521381238\\
-0.8427734375	7.47218880655375\\
-0.84228515625	7.48251731359409\\
-0.841796875	7.4928807635721\\
-0.84130859375	7.50327918524236\\
-0.8408203125	7.51371260747567\\
-0.84033203125	7.52418105925933\\
-0.83984375	7.53468456969744\\
-0.83935546875	7.54522316801116\\
-0.8388671875	7.55579688353903\\
-0.83837890625	7.56640574573727\\
-0.837890625	7.57704978418008\\
-0.83740234375	7.58772902855989\\
-0.8369140625	7.59844350868772\\
-0.83642578125	7.60919325449344\\
-0.8359375	7.61997829602612\\
-0.83544921875	7.63079866345428\\
-0.8349609375	7.6416543870662\\
-0.83447265625	7.6525454972703\\
-0.833984375	7.66347202459534\\
-0.83349609375	7.67443399969083\\
-0.8330078125	7.68543145332727\\
-0.83251953125	7.6964644163965\\
-0.83203125	7.70753291991205\\
-0.83154296875	7.71863699500935\\
-0.8310546875	7.72977667294614\\
-0.83056640625	7.7409519851028\\
-0.830078125	7.75216296298258\\
-0.82958984375	7.76340963821202\\
-0.8291015625	7.77469204254125\\
-0.82861328125	7.78601020784425\\
-0.828125	7.79736416611928\\
-0.82763671875	7.80875394948913\\
-0.8271484375	7.82017959020154\\
-0.82666015625	7.83164112062942\\
-0.826171875	7.84313857327128\\
-0.82568359375	7.85467198075151\\
-0.8251953125	7.86624137582077\\
-0.82470703125	7.87784679135628\\
-0.82421875	7.88948826036219\\
-0.82373046875	7.90116581596994\\
-0.8232421875	7.91287949143855\\
-0.82275390625	7.92462932015504\\
-0.822265625	7.93641533563471\\
-0.82177734375	7.94823757152157\\
-0.8212890625	7.96009606158858\\
-0.82080078125	7.97199083973817\\
-0.8203125	7.9839219400024\\
-0.81982421875	7.9958893965435\\
-0.8193359375	8.00789324365409\\
-0.81884765625	8.01993351575766\\
-0.818359375	8.03201024740882\\
-0.81787109375	8.04412347329376\\
-0.8173828125	8.05627322823056\\
-0.81689453125	8.06845954716963\\
-0.81640625	8.08068246519396\\
-0.81591796875	8.09294201751962\\
-0.8154296875	8.10523823949606\\
-0.81494140625	8.11757116660653\\
-0.814453125	8.12994083446841\\
-0.81396484375	8.14234727883366\\
-0.8134765625	8.15479053558913\\
-0.81298828125	8.167270640757\\
-0.8125	8.17978763049514\\
-0.81201171875	8.19234154109751\\
-0.8115234375	8.20493240899456\\
-0.81103515625	8.2175602707536\\
-0.810546875	8.23022516307918\\
-0.81005859375	8.2429271228136\\
-0.8095703125	8.25566618693713\\
-0.80908203125	8.26844239256854\\
-0.80859375	8.28125577696549\\
-0.80810546875	8.29410637752491\\
-0.8076171875	8.3069942317834\\
-0.80712890625	8.31991937741764\\
-0.806640625	8.33288185224484\\
-0.80615234375	8.34588169422314\\
-0.8056640625	8.35891894145196\\
-0.80517578125	8.37199363217254\\
-0.8046875	8.38510580476824\\
-0.80419921875	8.39825549776507\\
-0.8037109375	8.41144274983201\\
-0.80322265625	8.4246675997815\\
-0.802734375	8.43793008656989\\
-0.80224609375	8.45123024929781\\
-0.8017578125	8.46456812721064\\
-0.80126953125	8.47794375969891\\
-0.80078125	8.49135718629883\\
-0.80029296875	8.5048084466926\\
-0.7998046875	8.51829758070898\\
-0.79931640625	8.53182462832362\\
-0.798828125	8.54538962965958\\
-0.79833984375	8.55899262498778\\
-0.7978515625	8.5726336547274\\
-0.79736328125	8.5863127594464\\
-0.796875	8.6000299798619\\
-0.79638671875	8.61378535684073\\
-0.7958984375	8.6275789313998\\
-0.79541015625	8.64141074470663\\
-0.794921875	8.65528083807978\\
-0.79443359375	8.66918925298935\\
-0.7939453125	8.6831360310574\\
-0.79345703125	8.69712121405847\\
-0.79296875	8.71114484392006\\
-0.79248046875	8.72520696272304\\
-0.7919921875	8.7393076127022\\
-0.79150390625	8.75344683624673\\
-0.791015625	8.76762467590066\\
-0.79052734375	8.78184117436338\\
-0.7900390625	8.79609637449014\\
-0.78955078125	8.81039031929249\\
-0.7890625	8.82472305193886\\
-0.78857421875	8.83909461575496\\
-0.7880859375	8.85350505422436\\
-0.78759765625	8.86795441098899\\
-0.787109375	8.88244272984957\\
-0.78662109375	8.89697005476618\\
-0.7861328125	8.91153642985878\\
-0.78564453125	8.92614189940771\\
-0.78515625	8.94078650785415\\
-0.78466796875	8.95547029980074\\
-0.7841796875	8.97019332001203\\
-0.78369140625	8.98495561341501\\
-0.783203125	8.99975722509969\\
-0.78271484375	9.01459820031956\\
-0.7822265625	9.02947858449215\\
-0.78173828125	9.04439842319961\\
-0.78125	9.05935776218917\\
-0.78076171875	9.07435664737372\\
-0.7802734375	9.08939512483238\\
-0.77978515625	9.10447324081098\\
-0.779296875	9.1195910417227\\
-0.77880859375	9.13474857414852\\
-0.7783203125	9.14994588483786\\
-0.77783203125	9.16518302070908\\
-0.77734375	9.1804600288501\\
-0.77685546875	9.19577695651889\\
-0.7763671875	9.21113385114409\\
-0.77587890625	9.22653076032559\\
-0.775390625	9.24196773183503\\
-0.77490234375	9.25744481361647\\
-0.7744140625	9.27296205378689\\
-0.77392578125	9.28851950063684\\
-0.7734375	9.30411720263093\\
-0.77294921875	9.31975520840858\\
-0.7724609375	9.33543356678438\\
-0.77197265625	9.35115232674892\\
-0.771484375	9.36691153746922\\
-0.77099609375	9.3827112482894\\
-0.7705078125	9.3985515087313\\
-0.77001953125	9.41443236849503\\
-0.76953125	9.43035387745961\\
-0.76904296875	9.44631608568361\\
-0.7685546875	9.46231904340572\\
-0.76806640625	9.47836280104537\\
-0.767578125	9.49444740920341\\
-0.76708984375	9.51057291866268\\
-0.7666015625	9.52673938038865\\
-0.76611328125	9.54294684553006\\
-0.765625	9.55919536541957\\
-0.76513671875	9.57548499157436\\
-0.7646484375	9.59181577569682\\
-0.76416015625	9.60818776967515\\
-0.763671875	9.62460102558405\\
-0.76318359375	9.64105559568537\\
-0.7626953125	9.6575515324287\\
-0.76220703125	9.67408888845213\\
-0.76171875	9.69066771658285\\
-0.76123046875	9.70728806983782\\
-0.7607421875	9.72395000142446\\
-0.76025390625	9.74065356474133\\
-0.759765625	9.75739881337877\\
-0.75927734375	9.77418580111961\\
-0.7587890625	9.79101458193985\\
-0.75830078125	9.80788521000936\\
-0.7578125	9.82479773969254\\
-0.75732421875	9.84175222554907\\
-0.7568359375	9.85874872233451\\
-0.75634765625	9.87578728500113\\
-0.755859375	9.89286796869852\\
-0.75537109375	9.90999082877433\\
-0.7548828125	9.92715592077504\\
-0.75439453125	9.94436330044656\\
-0.75390625	9.96161302373506\\
-0.75341796875	9.97890514678765\\
-0.7529296875	9.9962397259531\\
-0.75244140625	10.0136168177826\\
-0.751953125	10.0310364790305\\
-0.75146484375	10.048498766655\\
-0.7509765625	10.066003737819\\
-0.75048828125	10.0835514498907\\
-0.75	10.1011419604445\\
-0.74951171875	10.1187753272617\\
-0.7490234375	10.1364516083311\\
-0.74853515625	10.1541708618502\\
-0.748046875	10.1719331462255\\
-0.74755859375	10.1897385200735\\
-0.7470703125	10.2075870422214\\
-0.74658203125	10.2254787717081\\
-0.74609375	10.2434137677846\\
-0.74560546875	10.2613920899151\\
-0.7451171875	10.2794137977777\\
-0.74462890625	10.2974789512652\\
-0.744140625	10.315587610486\\
-0.74365234375	10.3337398357645\\
-0.7431640625	10.3519356876427\\
-0.74267578125	10.3701752268801\\
-0.7421875	10.3884585144551\\
-0.74169921875	10.4067856115659\\
-0.7412109375	10.4251565796308\\
-0.74072265625	10.4435714802896\\
-0.740234375	10.462030375404\\
-0.73974609375	10.4805333270588\\
-0.7392578125	10.4990803975625\\
-0.73876953125	10.5176716494483\\
-0.73828125	10.5363071454747\\
-0.73779296875	10.5549869486268\\
-0.7373046875	10.5737111221169\\
-0.73681640625	10.5924797293853\\
-0.736328125	10.6112928341013\\
-0.73583984375	10.6301505001641\\
-0.7353515625	10.6490527917035\\
-0.73486328125	10.6679997730812\\
-0.734375	10.6869915088913\\
-0.73388671875	10.7060280639611\\
-0.7333984375	10.7251095033527\\
-0.73291015625	10.7442358923631\\
-0.732421875	10.7634072965256\\
-0.73193359375	10.7826237816105\\
-0.7314453125	10.8018854136263\\
-0.73095703125	10.8211922588202\\
-0.73046875	10.8405443836794\\
-0.72998046875	10.8599418549321\\
-0.7294921875	10.8793847395479\\
-0.72900390625	10.8988731047394\\
-0.728515625	10.9184070179629\\
-0.72802734375	10.9379865469191\\
-0.7275390625	10.9576117595545\\
-0.72705078125	10.9772827240622\\
-0.7265625	10.9969995088828\\
-0.72607421875	11.0167621827055\\
-0.7255859375	11.0365708144689\\
-0.72509765625	11.0564254733623\\
-0.724609375	11.0763262288264\\
-0.72412109375	11.0962731505546\\
-0.7236328125	11.1162663084938\\
-0.72314453125	11.1363057728453\\
-0.72265625	11.1563916140663\\
-0.72216796875	11.1765239028703\\
-0.7216796875	11.1967027102288\\
-0.72119140625	11.2169281073716\\
-0.720703125	11.2372001657887\\
-0.72021484375	11.2575189572306\\
-0.7197265625	11.2778845537097\\
-0.71923828125	11.2982970275014\\
-0.71875	11.318756451145\\
-0.71826171875	11.3392628974449\\
-0.7177734375	11.3598164394718\\
-0.71728515625	11.3804171505635\\
-0.716796875	11.4010651043261\\
-0.71630859375	11.4217603746353\\
-0.7158203125	11.4425030356371\\
-0.71533203125	11.4632931617494\\
-0.71484375	11.4841308276629\\
-0.71435546875	11.5050161083418\\
-0.7138671875	11.525949079026\\
-0.71337890625	11.546929815231\\
-0.712890625	11.5679583927498\\
-0.71240234375	11.5890348876541\\
-0.7119140625	11.610159376295\\
-0.71142578125	11.6313319353044\\
-0.7109375	11.6525526415962\\
-0.71044921875	11.6738215723676\\
-0.7099609375	11.6951388051\\
-0.70947265625	11.7165044175604\\
-0.708984375	11.7379184878024\\
-0.70849609375	11.7593810941676\\
-0.7080078125	11.7808923152869\\
-0.70751953125	11.8024522300813\\
-0.70703125	11.8240609177637\\
-0.70654296875	11.8457184578396\\
-0.7060546875	11.8674249301087\\
-0.70556640625	11.8891804146658\\
-0.705078125	11.9109849919027\\
-0.70458984375	11.9328387425086\\
-0.7041015625	11.954741747472\\
-0.70361328125	11.9766940880819\\
-0.703125	11.9986958459288\\
-0.70263671875	12.0207471029063\\
-0.7021484375	12.0428479412121\\
-0.70166015625	12.0649984433495\\
-0.701171875	12.087198692129\\
-0.70068359375	12.1094487706688\\
-0.7001953125	12.1317487623971\\
-0.69970703125	12.1540987510527\\
-0.69921875	12.1764988206867\\
-0.69873046875	12.1989490556639\\
-0.6982421875	12.2214495406639\\
-0.69775390625	12.2440003606827\\
-0.697265625	12.2666016010338\\
-0.69677734375	12.2892533473501\\
-0.6962890625	12.3119556855847\\
-0.69580078125	12.3347087020128\\
-0.6953125	12.3575124832327\\
-0.69482421875	12.3803671161674\\
-0.6943359375	12.4032726880662\\
-0.69384765625	12.4262292865058\\
-0.693359375	12.4492369993921\\
-0.69287109375	12.4722959149612\\
-0.6923828125	12.4954061217815\\
-0.69189453125	12.5185677087543\\
-0.69140625	12.5417807651163\\
-0.69091796875	12.5650453804402\\
-0.6904296875	12.5883616446368\\
-0.68994140625	12.611729647956\\
-0.689453125	12.6351494809889\\
-0.68896484375	12.6586212346688\\
-0.6884765625	12.6821450002729\\
-0.68798828125	12.705720869424\\
-0.6875	12.729348934092\\
-0.68701171875	12.7530292865952\\
-0.6865234375	12.7767620196021\\
-0.68603515625	12.8005472261332\\
-0.685546875	12.8243849995619\\
-0.68505859375	12.848275433617\\
-0.6845703125	12.8722186223833\\
-0.68408203125	12.8962146603043\\
-0.68359375	12.920263642183\\
-0.68310546875	12.9443656631837\\
-0.6826171875	12.9685208188341\\
-0.68212890625	12.9927292050263\\
-0.681640625	13.0169909180191\\
-0.68115234375	13.041306054439\\
-0.6806640625	13.0656747112826\\
-0.68017578125	13.0900969859178\\
-0.6796875	13.1145729760857\\
-0.67919921875	13.1391027799021\\
-0.6787109375	13.1636864958598\\
-0.67822265625	13.1883242228295\\
-0.677734375	13.2130160600623\\
-0.67724609375	13.2377621071911\\
-0.6767578125	13.2625624642324\\
-0.67626953125	13.2874172315881\\
-0.67578125	13.3123265100473\\
-0.67529296875	13.3372904007881\\
-0.6748046875	13.3623090053796\\
-0.67431640625	13.3873824257833\\
-0.673828125	13.4125107643553\\
-0.67333984375	13.4376941238481\\
-0.6728515625	13.4629326074124\\
-0.67236328125	13.4882263185989\\
-0.671875	13.5135753613602\\
-0.67138671875	13.5389798400529\\
-0.6708984375	13.5644398594393\\
-0.67041015625	13.5899555246894\\
-0.669921875	13.6155269413827\\
-0.66943359375	13.6411542155104\\
-0.6689453125	13.6668374534773\\
-0.66845703125	13.6925767621033\\
-0.66796875	13.718372248626\\
-0.66748046875	13.7442240207026\\
-0.6669921875	13.7701321864115\\
-0.66650390625	13.7960968542546\\
-0.666015625	13.8221181331595\\
-0.66552734375	13.848196132481\\
-0.6650390625	13.8743309620038\\
-0.66455078125	13.9005227319442\\
-0.6640625	13.9267715529521\\
-0.66357421875	13.9530775361135\\
-0.6630859375	13.979440792952\\
-0.66259765625	14.0058614354315\\
-0.662109375	14.0323395759582\\
-0.66162109375	14.0588753273824\\
-0.6611328125	14.085468803001\\
-0.66064453125	14.1121201165597\\
-0.66015625	14.1388293822549\\
-0.65966796875	14.1655967147362\\
-0.6591796875	14.1924222291085\\
-0.65869140625	14.2193060409341\\
-0.658203125	14.2462482662352\\
-0.65771484375	14.273249021496\\
-0.6572265625	14.300308423665\\
-0.65673828125	14.3274265901573\\
-0.65625	14.3546036388567\\
-0.65576171875	14.3818396881187\\
-0.6552734375	14.4091348567717\\
-0.65478515625	14.4364892641206\\
-0.654296875	14.4639030299482\\
-0.65380859375	14.491376274518\\
-0.6533203125	14.5189091185766\\
-0.65283203125	14.546501683356\\
-0.65234375	14.5741540905759\\
-0.65185546875	14.6018664624467\\
-0.6513671875	14.6296389216712\\
-0.65087890625	14.6574715914475\\
-0.650390625	14.6853645954716\\
-0.64990234375	14.7133180579394\\
-0.6494140625	14.7413321035499\\
-0.64892578125	14.7694068575072\\
-0.6484375	14.7975424455231\\
-0.64794921875	14.8257389938201\\
-0.6474609375	14.8539966291334\\
-0.64697265625	14.8823154787141\\
-0.646484375	14.9106956703312\\
-0.64599609375	14.9391373322748\\
-0.6455078125	14.9676405933583\\
-0.64501953125	14.9962055829215\\
-0.64453125	15.0248324308329\\
-0.64404296875	15.0535212674926\\
-0.6435546875	15.0822722238352\\
-0.64306640625	15.111085431332\\
-0.642578125	15.1399610219945\\
-0.64208984375	15.1688991283765\\
-0.6416015625	15.1978998835773\\
-0.64111328125	15.2269634212444\\
-0.640625	15.2560898755763\\
-0.64013671875	15.2852793813256\\
-0.6396484375	15.3145320738014\\
-0.63916015625	15.3438480888726\\
-0.638671875	15.3732275629708\\
-0.63818359375	15.4026706330928\\
-0.6376953125	15.4321774368041\\
-0.63720703125	15.4617481122416\\
-0.63671875	15.4913827981167\\
-0.63623046875	15.5210816337181\\
-0.6357421875	15.5508447589149\\
-0.63525390625	15.58067231416\\
-0.634765625	15.6105644404927\\
-0.63427734375	15.6405212795419\\
-0.6337890625	15.6705429735296\\
-0.63330078125	15.7006296652735\\
-0.6328125	15.7307814981904\\
-0.63232421875	15.7609986162995\\
-0.6318359375	15.7912811642254\\
-0.63134765625	15.8216292872014\\
-0.630859375	15.8520431310727\\
-0.63037109375	15.8825228422997\\
-0.6298828125	15.9130685679613\\
-0.62939453125	15.9436804557583\\
-0.62890625	15.9743586540165\\
-0.62841796875	16.0051033116903\\
-0.6279296875	16.0359145783658\\
-0.62744140625	16.0667926042645\\
-0.626953125	16.0977375402466\\
-0.62646484375	16.1287495378143\\
-0.6259765625	16.1598287491155\\
-0.62548828125	16.1909753269473\\
-0.625	16.2221894247592\\
-0.62451171875	16.2534711966571\\
-0.6240234375	16.2848207974065\\
-0.62353515625	16.3162383824363\\
-0.623046875	16.3477241078423\\
-0.62255859375	16.3792781303908\\
-0.6220703125	16.4109006075227\\
-0.62158203125	16.4425916973565\\
-0.62109375	16.4743515586926\\
-0.62060546875	16.5061803510166\\
-0.6201171875	16.5380782345035\\
-0.61962890625	16.5700453700213\\
-0.619140625	16.6020819191346\\
-0.61865234375	16.634188044109\\
-0.6181640625	16.6663639079143\\
-0.61767578125	16.698609674229\\
-0.6171875	16.7309255074438\\
-0.61669921875	16.763311572666\\
-0.6162109375	16.7957680357231\\
-0.61572265625	16.8282950631668\\
-0.615234375	16.8608928222775\\
-0.61474609375	16.8935614810679\\
-0.6142578125	16.9263012082872\\
-0.61376953125	16.9591121734256\\
-0.61328125	16.9919945467179\\
-0.61279296875	17.024948499148\\
-0.6123046875	17.0579742024532\\
-0.61181640625	17.0910718291283\\
-0.611328125	17.1242415524299\\
-0.61083984375	17.1574835463806\\
-0.6103515625	17.1907979857737\\
-0.60986328125	17.2241850461772\\
-0.609375	17.2576449039384\\
-0.60888671875	17.2911777361882\\
-0.6083984375	17.3247837208458\\
-0.60791015625	17.3584630366229\\
-0.607421875	17.3922158630285\\
-0.60693359375	17.4260423803733\\
-0.6064453125	17.4599427697743\\
-0.60595703125	17.4939172131597\\
-0.60546875	17.527965893273\\
-0.60498046875	17.5620889936784\\
-0.6044921875	17.5962866987649\\
-0.60400390625	17.6305591937514\\
-0.603515625	17.6649066646917\\
-0.60302734375	17.6993292984788\\
-0.6025390625	17.7338272828501\\
-0.60205078125	17.7684008063924\\
-0.6015625	17.8030500585466\\
-0.60107421875	17.8377752296127\\
-0.6005859375	17.8725765107552\\
-0.60009765625	17.9074540940076\\
-0.599609375	17.9424081722778\\
-0.59912109375	17.9774389393531\\
-0.5986328125	18.0125465899057\\
-0.59814453125	18.0477313194975\\
-0.59765625	18.0829933245853\\
-0.59716796875	18.1183328025265\\
-0.5966796875	18.153749951584\\
-0.59619140625	18.1892449709321\\
-0.595703125	18.224818060661\\
-0.59521484375	18.2604694217833\\
-0.5947265625	18.2961992562386\\
-0.59423828125	18.3320077668995\\
-0.59375	18.367895157577\\
-0.59326171875	18.4038616330263\\
-0.5927734375	18.439907398952\\
-0.59228515625	18.4760326620143\\
-0.591796875	18.5122376298344\\
-0.59130859375	18.5485225110003\\
-0.5908203125	18.5848875150729\\
-0.59033203125	18.6213328525913\\
-0.58984375	18.6578587350794\\
-0.58935546875	18.6944653750513\\
-0.5888671875	18.7311529860174\\
-0.58837890625	18.7679217824907\\
-0.587890625	18.8047719799925\\
-0.58740234375	18.841703795059\\
-0.5869140625	18.8787174452468\\
-0.58642578125	18.91581314914\\
-0.5859375	18.9529911263555\\
-0.58544921875	18.9902515975502\\
-0.5849609375	19.0275947844267\\
-0.58447265625	19.0650209097401\\
-0.583984375	19.1025301973044\\
-0.58349609375	19.1401228719987\\
-0.5830078125	19.1777991597743\\
-0.58251953125	19.2155592876608\\
-0.58203125	19.2534034837732\\
-0.58154296875	19.2913319773182\\
-0.5810546875	19.3293449986011\\
-0.58056640625	19.367442779033\\
-0.580078125	19.4056255511371\\
-0.57958984375	19.4438935485557\\
-0.5791015625	19.4822470060577\\
-0.57861328125	19.5206861595451\\
-0.578125	19.5592112460601\\
-0.57763671875	19.5978225037925\\
-0.5771484375	19.6365201720869\\
-0.57666015625	19.6753044914497\\
-0.576171875	19.7141757035563\\
-0.57568359375	19.7531340512591\\
-0.5751953125	19.7921797785943\\
-0.57470703125	19.8313131307895\\
-0.57421875	19.8705343542714\\
-0.57373046875	19.9098436966733\\
-0.5732421875	19.9492414068428\\
-0.57275390625	19.9887277348494\\
-0.572265625	20.0283029319923\\
-0.57177734375	20.0679672508085\\
-0.5712890625	20.1077209450801\\
-0.57080078125	20.147564269843\\
-0.5703125	20.1874974813943\\
-0.56982421875	20.2275208373008\\
-0.5693359375	20.2676345964068\\
-0.56884765625	20.3078390188426\\
-0.568359375	20.3481343660326\\
-0.56787109375	20.3885209007038\\
-0.5673828125	20.428998886894\\
-0.56689453125	20.4695685899605\\
-0.56640625	20.5102302765884\\
-0.56591796875	20.5509842147995\\
-0.5654296875	20.5918306739605\\
-0.56494140625	20.6327699247924\\
-0.564453125	20.6738022393789\\
-0.56396484375	20.7149278911751\\
-0.5634765625	20.7561471550171\\
-0.56298828125	20.7974603071304\\
-0.5625	20.8388676251393\\
-0.56201171875	20.880369388076\\
-0.5615234375	20.92196587639\\
-0.56103515625	20.9636573719571\\
-0.560546875	21.0054441580891\\
-0.56005859375	21.0473265195431\\
-0.5595703125	21.0893047425314\\
-0.55908203125	21.1313791147303\\
-0.55859375	21.1735499252908\\
-0.55810546875	21.2158174648477\\
-0.5576171875	21.2581820255298\\
-0.55712890625	21.3006439009696\\
-0.556640625	21.3432033863136\\
-0.55615234375	21.3858607782321\\
-0.5556640625	21.4286163749297\\
-0.55517578125	21.4714704761551\\
-0.5546875	21.5144233832122\\
-0.55419921875	21.5574753989696\\
-0.5537109375	21.6006268278721\\
-0.55322265625	21.6438779759504\\
-0.552734375	21.6872291508324\\
-0.55224609375	21.7306806617538\\
-0.5517578125	21.774232819569\\
-0.55126953125	21.8178859367618\\
-0.55078125	21.8616403274568\\
-0.55029296875	21.9054963074305\\
-0.5498046875	21.9494541941223\\
-0.54931640625	21.993514306646\\
-0.548828125	22.0376769658011\\
-0.54833984375	22.0819424940845\\
-0.5478515625	22.1263112157017\\
-0.54736328125	22.1707834565793\\
-0.546875	22.2153595443757\\
-0.54638671875	22.260039808494\\
-0.5458984375	22.3048245800933\\
-0.54541015625	22.3497141921011\\
-0.544921875	22.3947089792256\\
-0.54443359375	22.4398092779677\\
-0.5439453125	22.4850154266334\\
-0.54345703125	22.5303277653468\\
-0.54296875	22.5757466360621\\
-0.54248046875	22.6212723825764\\
-0.5419921875	22.6669053505428\\
-0.54150390625	22.7126458874833\\
-0.541015625	22.7584943428015\\
-0.54052734375	22.804451067796\\
-0.5400390625	22.8505164156735\\
-0.53955078125	22.8966907415623\\
-0.5390625	22.9429744025257\\
-0.53857421875	22.9893677575756\\
-0.5380859375	23.0358711676862\\
-0.53759765625	23.0824849958075\\
-0.537109375	23.1292096068798\\
-0.53662109375	23.1760453678473\\
-0.5361328125	23.2229926476726\\
-0.53564453125	23.2700518173505\\
-0.53515625	23.3172232499229\\
-0.53466796875	23.3645073204929\\
-0.5341796875	23.4119044062399\\
-0.53369140625	23.4594148864337\\
-0.533203125	23.5070391424503\\
-0.53271484375	23.5547775577859\\
-0.5322265625	23.6026305180729\\
-0.53173828125	23.6505984110948\\
-0.53125	23.6986816268013\\
-0.53076171875	23.7468805573245\\
-0.5302734375	23.795195596994\\
-0.52978515625	23.843627142353\\
-0.529296875	23.8921755921739\\
-0.52880859375	23.940841347475\\
-0.5283203125	23.9896248115357\\
-0.52783203125	24.038526389914\\
-0.52734375	24.087546490462\\
-0.52685546875	24.1366855233429\\
-0.5263671875	24.1859439010479\\
-0.52587890625	24.2353220384128\\
-0.525390625	24.2848203526353\\
-0.52490234375	24.334439263292\\
-0.5244140625	24.3841791923557\\
-0.52392578125	24.434040564213\\
-0.5234375	24.4840238056821\\
-0.52294921875	24.53412934603\\
-0.5224609375	24.584357616991\\
-0.52197265625	24.6347090527845\\
-0.521484375	24.6851840901333\\
-0.52099609375	24.735783168282\\
-0.5205078125	24.7865067290154\\
-0.52001953125	24.8373552166775\\
-0.51953125	24.88832907819\\
-0.51904296875	24.9394287630717\\
-0.5185546875	24.9906547234571\\
-0.51806640625	25.0420074141166\\
-0.517578125	25.0934872924751\\
-0.51708984375	25.1450948186323\\
-0.5166015625	25.1968304553824\\
-0.51611328125	25.2486946682339\\
-0.515625	25.30068792543\\
-0.51513671875	25.3528106979688\\
-0.5146484375	25.4050634596241\\
-0.51416015625	25.4574466869657\\
-0.513671875	25.5099608593805\\
-0.51318359375	25.5626064590936\\
-0.5126953125	25.6153839711895\\
-0.51220703125	25.6682938836332\\
-0.51171875	25.7213366872924\\
-0.51123046875	25.7745128759587\\
-0.5107421875	25.8278229463699\\
-0.51025390625	25.8812673982318\\
-0.509765625	25.9348467342411\\
-0.50927734375	25.9885614601074\\
-0.5087890625	26.042412084576\\
-0.50830078125	26.096399119451\\
-0.5078125	26.1505230796182\\
-0.50732421875	26.2047844830687\\
-0.5068359375	26.2591838509217\\
-0.50634765625	26.3137217074491\\
-0.505859375	26.3683985800987\\
-0.50537109375	26.4232149995184\\
-0.5048828125	26.4781714995807\\
-0.50439453125	26.5332686174071\\
-0.50390625	26.5885068933926\\
-0.50341796875	26.6438868712308\\
-0.5029296875	26.699409097939\\
-0.50244140625	26.7550741238835\\
-0.501953125	26.8108825028048\\
-0.50146484375	26.866834791844\\
-0.5009765625	26.922931551568\\
-0.50048828125	26.9791733459963\\
-0.5	27.0355607426269\\
-0.49951171875	27.0920943124633\\
-0.4990234375	27.1487746300409\\
-0.49853515625	27.2056022734543\\
-0.498046875	27.2625778243845\\
-0.49755859375	27.3197018681265\\
-0.4970703125	27.3769749936166\\
-0.49658203125	27.4343977934609\\
-0.49609375	27.4919708639629\\
-0.49560546875	27.5496948051522\\
-0.4951171875	27.6075702208132\\
-0.49462890625	27.6655977185136\\
-0.494140625	27.7237779096337\\
-0.49365234375	27.7821114093956\\
-0.4931640625	27.840598836893\\
-0.49267578125	27.8992408151206\\
-0.4921875	27.9580379710042\\
-0.49169921875	28.0169909354312\\
-0.4912109375	28.076100343281\\
-0.49072265625	28.1353668334552\\
-0.490234375	28.1947910489094\\
-0.48974609375	28.2543736366839\\
-0.4892578125	28.3141152479351\\
-0.48876953125	28.3740165379675\\
-0.48828125	28.4340781662654\\
-0.48779296875	28.494300796525\\
-0.4873046875	28.5546850966873\\
-0.48681640625	28.6152317389701\\
-0.486328125	28.6759413999012\\
-0.48583984375	28.7368147603519\\
-0.4853515625	28.7978525055699\\
-0.48486328125	28.8590553252131\\
-0.484375	28.9204239133836\\
-0.48388671875	28.9819589686618\\
-0.4833984375	29.0436611941409\\
-0.48291015625	29.1055312974613\\
-0.482421875	29.1675699908455\\
-0.48193359375	29.2297779911336\\
-0.4814453125	29.2921560198184\\
-0.48095703125	29.354704803081\\
-0.48046875	29.4174250718268\\
-0.47998046875	29.4803175617216\\
-0.4794921875	29.5433830132279\\
-0.47900390625	29.6066221716419\\
-0.478515625	29.6700357871296\\
-0.47802734375	29.7336246147648\\
-0.4775390625	29.7973894145657\\
-0.47705078125	29.861330951533\\
-0.4765625	29.9254499956876\\
-0.47607421875	29.9897473221085\\
-0.4755859375	30.0542237109714\\
-0.47509765625	30.1188799475869\\
-0.474609375	30.1837168224397\\
-0.47412109375	30.2487351312274\\
-0.4736328125	30.3139356748996\\
-0.47314453125	30.3793192596976\\
-0.47265625	30.4448866971941\\
-0.47216796875	30.5106388043331\\
-0.4716796875	30.5765764034697\\
-0.47119140625	30.6427003224113\\
-0.470703125	30.709011394457\\
-0.47021484375	30.7755104584396\\
-0.4697265625	30.8421983587657\\
-0.46923828125	30.9090759454571\\
-0.46875	30.9761440741927\\
-0.46826171875	31.0434036063493\\
-0.4677734375	31.1108554090439\\
-0.46728515625	31.1785003551755\\
-0.466796875	31.2463393234672\\
-0.46630859375	31.3143731985087\\
-0.4658203125	31.3826028707983\\
-0.46533203125	31.4510292367861\\
-0.46484375	31.5196531989165\\
-0.46435546875	31.588475665671\\
-0.4638671875	31.6574975516115\\
-0.46337890625	31.7267197774236\\
-0.462890625	31.7961432699595\\
-0.46240234375	31.8657689622818\\
-0.4619140625	31.935597793707\\
-0.46142578125	32.0056307098488\\
-0.4609375	32.0758686626624\\
-0.46044921875	32.146312610488\\
-0.4599609375	32.2169635180943\\
-0.45947265625	32.2878223567231\\
-0.458984375	32.358890104133\\
-0.45849609375	32.4301677446431\\
-0.4580078125	32.5016562691771\\
-0.45751953125	32.573356675308\\
-0.45703125	32.6452699673011\\
-0.45654296875	32.7173971561584\\
-0.4560546875	32.7897392596627\\
-0.45556640625	32.8622973024213\\
-0.455078125	32.9350723159098\\
-0.45458984375	33.0080653385156\\
-0.4541015625	33.0812774155819\\
-0.45361328125	33.1547095994508\\
-0.453125	33.228362949507\\
-0.45263671875	33.3022385322202\\
-0.4521484375	33.376337421189\\
-0.45166015625	33.450660697183\\
-0.451171875	33.5252094481851\\
-0.45068359375	33.5999847694344\\
-0.4501953125	33.6749877634678\\
-0.44970703125	33.7502195401611\\
-0.44921875	33.8256812167712\\
-0.44873046875	33.9013739179763\\
-0.4482421875	33.9772987759165\\
-0.44775390625	34.053456930234\\
-0.447265625	34.1298495281128\\
-0.44677734375	34.206477724317\\
-0.4462890625	34.2833426812305\\
-0.44580078125	34.3604455688938\\
-0.4453125	34.4377875650422\\
-0.44482421875	34.5153698551419\\
-0.4443359375	34.5931936324262\\
-0.44384765625	34.6712600979308\\
-0.443359375	34.7495704605282\\
-0.44287109375	34.8281259369612\\
-0.4423828125	34.9069277518755\\
-0.44189453125	34.9859771378521\\
-0.44140625	35.0652753354374\\
-0.44091796875	35.1448235931733\\
-0.4404296875	35.2246231676258\\
-0.43994140625	35.3046753234127\\
-0.439453125	35.3849813332294\\
-0.43896484375	35.4655424778745\\
-0.4384765625	35.5463600462731\\
-0.43798828125	35.6274353354989\\
-0.4375	35.7087696507953\\
-0.43701171875	35.7903643055944\\
-0.4365234375	35.8722206215346\\
-0.43603515625	35.9543399284759\\
-0.435546875	36.0367235645149\\
-0.43505859375	36.1193728759958\\
-0.4345703125	36.2022892175214\\
-0.43408203125	36.2854739519605\\
-0.43359375	36.3689284504541\\
-0.43310546875	36.4526540924187\\
-0.4326171875	36.5366522655478\\
-0.43212890625	36.6209243658103\\
-0.431640625	36.7054717974469\\
-0.43115234375	36.790295972963\\
-0.4306640625	36.8753983131196\\
-0.43017578125	36.9607802469207\\
-0.4296875	37.0464432115976\\
-0.42919921875	37.1323886525903\\
-0.4287109375	37.2186180235249\\
-0.42822265625	37.3051327861886\\
-0.427734375	37.3919344104993\\
-0.42724609375	37.479024374473\\
-0.4267578125	37.5664041641864\\
-0.42626953125	37.654075273735\\
-0.42578125	37.7420392051881\\
-0.42529296875	37.8302974685372\\
-0.4248046875	37.918851581642\\
-0.42431640625	38.0077030701697\\
-0.423828125	38.0968534675299\\
-0.42333984375	38.1863043148041\\
-0.4228515625	38.2760571606693\\
-0.42236328125	38.3661135613164\\
-0.421875	38.4564750803616\\
-0.42138671875	38.5471432887527\\
-0.4208984375	38.6381197646675\\
-0.42041015625	38.7294060934064\\
-0.419921875	38.8210038672777\\
-0.41943359375	38.9129146854749\\
-0.4189453125	39.0051401539467\\
-0.41845703125	39.0976818852596\\
-0.41796875	39.1905414984507\\
-0.41748046875	39.2837206188734\\
-0.4169921875	39.3772208780327\\
-0.41650390625	39.4710439134124\\
-0.416015625	39.5651913682912\\
-0.41552734375	39.6596648915499\\
-0.4150390625	39.754466137467\\
-0.41455078125	39.8495967655042\\
-0.4140625	39.9450584400796\\
-0.41357421875	40.0408528303298\\
-0.4130859375	40.1369816098593\\
-0.41259765625	40.2334464564768\\
-0.412109375	40.3302490519191\\
-0.41162109375	40.4273910815597\\
-0.4111328125	40.5248742341043\\
-0.41064453125	40.622700201271\\
-0.41015625	40.7208706774539\\
-0.40966796875	40.8193873593726\\
-0.4091796875	40.9182519457025\\
-0.40869140625	41.0174661366896\\
-0.408203125	41.1170316337462\\
-0.40771484375	41.2169501390278\\
-0.4072265625	41.3172233549907\\
-0.40673828125	41.417852983929\\
-0.40625	41.5188407274907\\
-0.40576171875	41.6201882861718\\
-0.4052734375	41.7218973587876\\
-0.40478515625	41.8239696419199\\
-0.404296875	41.9264068293406\\
-0.40380859375	42.0292106114087\\
-0.4033203125	42.1323826744415\\
-0.40283203125	42.2359247000579\\
-0.40234375	42.3398383644934\\
-0.40185546875	42.444125337885\\
-0.4013671875	42.548787283526\\
-0.40087890625	42.6538258570876\\
-0.400390625	42.7592427058087\\
-0.39990234375	42.8650394676503\\
-0.3994140625	42.9712177704143\\
-0.39892578125	43.0777792308249\\
-0.3984375	43.1847254535725\\
-0.39794921875	43.2920580303159\\
-0.3974609375	43.3997785386445\\
-0.39697265625	43.5078885409965\\
-0.396484375	43.6163895835323\\
-0.39599609375	43.7252831949625\\
-0.3955078125	43.8345708853264\\
-0.39501953125	43.9442541447216\\
-0.39453125	44.054334441981\\
-0.39404296875	44.1648132232965\\
-0.3935546875	44.2756919107871\\
-0.39306640625	44.3869719010082\\
-0.392578125	44.4986545634021\\
-0.39208984375	44.6107412386857\\
-0.3916015625	44.7232332371735\\
-0.39111328125	44.836131837034\\
-0.390625	44.9494382824768\\
-0.39013671875	45.0631537818678\\
-0.3896484375	45.1772795057693\\
-0.38916015625	45.2918165849044\\
-0.388671875	45.4067661080393\\
-0.38818359375	45.5221291197844\\
-0.3876953125	45.6379066183084\\
-0.38720703125	45.7540995529646\\
-0.38671875	45.8707088218232\\
-0.38623046875	45.9877352691102\\
-0.3857421875	46.1051796825458\\
-0.38525390625	46.2230427905817\\
-0.384765625	46.3413252595314\\
-0.38427734375	46.4600276905916\\
-0.3837890625	46.5791506167499\\
-0.38330078125	46.6986944995755\\
-0.3828125	46.8186597258872\\
-0.38232421875	46.9390466042979\\
-0.3818359375	47.0598553616274\\
-0.38134765625	47.1810861391821\\
-0.380859375	47.3027389888957\\
-0.38037109375	47.4248138693264\\
-0.3798828125	47.5473106415063\\
-0.37939453125	47.6702290646382\\
-0.37890625	47.7935687916339\\
-0.37841796875	47.9173293644908\\
-0.3779296875	48.0415102094987\\
-0.37744140625	48.1661106322758\\
-0.376953125	48.2911298126234\\
-0.37646484375	48.4165667991985\\
-0.3759765625	48.5424205039958\\
-0.37548828125	48.6686896966346\\
-0.375	48.7953729984447\\
-0.37451171875	48.9224688763456\\
-0.3740234375	49.0499756365124\\
-0.37353515625	49.177891417824\\
-0.373046875	49.3062141850861\\
-0.37255859375	49.4349417220242\\
-0.3720703125	49.5640716240394\\
-0.37158203125	49.6936012907221\\
-0.37109375	49.823527918118\\
-0.37060546875	49.9538484907377\\
-0.3701171875	50.0845597733085\\
-0.36962890625	50.2156583022581\\
-0.369140625	50.347140376928\\
-0.36865234375	50.4790020505087\\
-0.3681640625	50.6112391206931\\
-0.36767578125	50.7438471200416\\
-0.3671875	50.8768213060549\\
-0.36669921875	51.0101566509493\\
-0.3662109375	51.1438478311312\\
-0.36572265625	51.2778892163646\\
-0.365234375	51.4122748586319\\
-0.36474609375	51.546998480681\\
-0.3642578125	51.6820534642587\\
-0.36376953125	51.8174328380283\\
-0.36328125	51.9531292651698\\
-0.36279296875	52.0891350306613\\
-0.3623046875	52.2254420282456\\
-0.36181640625	52.3620417470779\\
-0.361328125	52.4989252580628\\
-0.36083984375	52.6360831998796\\
-0.3603515625	52.7735057647028\\
-0.35986328125	52.9111826836239\\
-0.359375	53.0491032117803\\
-0.35888671875	53.1872561132028\\
-0.3583984375	53.3256296453896\\
-0.35791015625	53.4642115436226\\
-0.357421875	53.6029890050368\\
-0.35693359375	53.7419486724629\\
-0.3564453125	53.8810766180591\\
-0.35595703125	54.0203583267558\\
-0.35546875	54.1597786795351\\
-0.35498046875	54.2993219365734\\
-0.3544921875	54.4389717202747\\
-0.35400390625	54.5787109982299\\
-0.353515625	54.7185220661353\\
-0.35302734375	54.8583865307121\\
-0.3525390625	54.998285292668\\
-0.35205078125	55.13819852975\\
-0.3515625	55.278105679937\\
-0.35107421875	55.41798542483\\
-0.3505859375	55.5578156732963\\
-0.35009765625	55.6975735454357\\
-0.349609375	55.8372353569342\\
-0.34912109375	55.9767766038807\\
-0.3486328125	56.1161719481247\\
-0.34814453125	56.25539520326\\
-0.34765625	56.3944193213213\\
-0.34716796875	56.5332163802902\\
-0.3466796875	56.6717575725087\\
-0.34619140625	56.8100131941069\\
-0.345703125	56.9479526355519\\
-0.34521484375	57.0855443734372\\
-0.3447265625	57.2227559636305\\
-0.34423828125	57.3595540359064\\
-0.34375	57.4959042901932\\
-0.34326171875	57.6317714945702\\
-0.3427734375	57.7671194851518\\
-0.34228515625	57.9019111680012\\
-0.341796875	58.0361085232207\\
-0.34130859375	58.1696726113632\\
-0.3408203125	58.3025635823174\\
-0.34033203125	58.4347406868145\\
-0.33984375	58.5661622907094\\
-0.33935546875	58.6967858921847\\
-0.3388671875	58.8265681420252\\
-0.33837890625	58.9554648671098\\
-0.337890625	59.0834310972587\\
-0.33740234375	59.210421095572\\
-0.3369140625	59.3363883923874\\
-0.33642578125	59.4612858229754\\
-0.3359375	59.5850655690799\\
-0.33544921875	59.7076792044018\\
-0.3349609375	59.8290777441081\\
-0.33447265625	59.9492116984298\\
-0.333984375	60.0680311304038\\
-0.33349609375	60.1854857177831\\
-0.3330078125	60.301524819127\\
-0.33251953125	60.4160975440544\\
-0.33203125	60.5291528276233\\
-0.33154296875	60.6406395087687\\
-0.3310546875	60.7505064127025\\
-0.33056640625	60.8587024371579\\
-0.330078125	60.9651766423158\\
-0.32958984375	61.0698783442327\\
-0.3291015625	61.1727572115473\\
-0.32861328125	61.273763365215\\
-0.328125	61.3728474809848\\
-0.32763671875	61.4699608942959\\
-0.3271484375	61.5650557072494\\
-0.32666015625	61.6580848972616\\
-0.326171875	61.7490024269934\\
-0.32568359375	61.8377633551064\\
-0.3251953125	61.9243239473786\\
-0.32470703125	62.0086417876834\\
-0.32421875	62.0906758883171\\
-0.32373046875	62.1703867991416\\
-0.3232421875	62.2477367149927\\
-0.32275390625	62.3226895807992\\
-0.322265625	62.3952111938461\\
-0.32177734375	62.4652693026201\\
-0.3212890625	62.532833701677\\
-0.32080078125	62.5978763219794\\
-0.3203125	62.6603713161692\\
-0.31982421875	62.7202951382581\\
-0.3193359375	62.7776266172435\\
-0.31884765625	62.8323470241897\\
-0.318359375	62.8844401323459\\
-0.31787109375	62.9338922699125\\
-0.3173828125	62.9806923651151\\
-0.31689453125	63.0248319832845\\
-0.31640625	63.0663053557019\\
-0.31591796875	63.1051094000114\\
-0.3154296875	63.1412437320698\\
-0.31494140625	63.1747106691456\\
-0.314453125	63.2055152244553\\
-0.31396484375	63.2336650930662\\
-0.3134765625	63.2591706292663\\
-0.31298828125	63.2820448155505\\
-0.3125	63.3023032234317\\
-0.31201171875	63.3199639663354\\
-0.3115234375	63.3350476448914\\
-0.31103515625	63.3475772849733\\
-0.310546875	63.3575782688902\\
-0.31005859375	63.3650782601645\\
-0.3095703125	63.3701071223669\\
-0.30908203125	63.3726968325084\\
-0.30859375	63.3728813895106\\
-0.30810546875	63.3706967182965\\
-0.3076171875	63.3661805700548\\
-0.30712890625	63.3593724192387\\
-0.306640625	63.3503133578654\\
-0.30615234375	63.339045987677\\
-0.3056640625	63.3256143107188\\
-0.30517578125	63.31006361888\\
-0.3046875	63.2924403829244\\
-0.30419921875	63.2727921415231\\
-0.3037109375	63.251167390777\\
-0.30322265625	63.2276154746909\\
-0.302734375	63.202186477037\\
-0.30224609375	63.1749311150126\\
-0.3017578125	63.1459006350679\\
-0.30126953125	63.1151467112476\\
-0.30078125	63.0827213463544\\
-0.30029296875	63.048676776214\\
-0.2998046875	63.013065377282\\
-0.29931640625	62.9759395778065\\
-0.298828125	62.9373517727204\\
-0.29833984375	62.8973542424126\\
-0.2978515625	62.8559990754925\\
-0.29736328125	62.8133380956345\\
-0.296875	62.7694227925608\\
-0.29638671875	62.7243042571971\\
-0.2958984375	62.6780331210053\\
-0.29541015625	62.6306594994843\\
-0.294921875	62.5822329397987\\
-0.29443359375	62.5328023724837\\
-0.2939453125	62.4824160671556\\
-0.29345703125	62.4311215921396\\
-0.29296875	62.3789657779162\\
-0.29248046875	62.3259946842753\\
-0.2919921875	62.2722535710553\\
-0.29150390625	62.2177868723394\\
-0.291015625	62.16263817397\\
-0.29052734375	62.1068501942428\\
-0.2900390625	62.0504647676319\\
-0.28955078125	61.9935228313988\\
-0.2890625	61.9360644149357\\
-0.28857421875	61.8781286316904\\
-0.2880859375	61.8197536735251\\
-0.28759765625	61.7609768073577\\
-0.287109375	61.7018343739409\\
-0.28662109375	61.6423617886319\\
-0.2861328125	61.582593544015\\
-0.28564453125	61.5225632142355\\
-0.28515625	61.4623034609161\\
-0.28466796875	61.4018460405226\\
-0.2841796875	61.3412218130584\\
-0.28369140625	61.280460751967\\
-0.283203125	61.2195919551274\\
-0.28271484375	61.158643656837\\
-0.2822265625	61.0976432406743\\
-0.28173828125	61.0366172531473\\
-0.28125	60.9755914180319\\
-0.28076171875	60.914590651314\\
-0.2802734375	60.8536390766524\\
-0.27978515625	60.7927600412863\\
-0.279296875	60.731976132313\\
-0.27880859375	60.67130919327\\
-0.2783203125	60.6107803409587\\
-0.27783203125	60.5504099824497\\
-0.27734375	60.4902178322171\\
-0.27685546875	60.4302229293525\\
-0.2763671875	60.3704436548107\\
-0.27587890625	60.3108977486472\\
-0.275390625	60.2516023272079\\
-0.27490234375	60.1925739002361\\
-0.2744140625	60.1338283878656\\
-0.27392578125	60.0753811374708\\
-0.2734375	60.0172469403474\\
-0.27294921875	59.9594400482021\\
-0.2724609375	59.9019741894285\\
-0.27197265625	59.8448625851542\\
-0.271484375	59.7881179650394\\
-0.27099609375	59.7317525828158\\
-0.2705078125	59.6757782315535\\
-0.27001953125	59.620206258644\\
-0.26953125	59.5650475804928\\
-0.26904296875	59.5103126969139\\
-0.2685546875	59.4560117052198\\
-0.26806640625	59.402154314004\\
-0.267578125	59.3487498566115\\
-0.26708984375	59.2958073042974\\
-0.2666015625	59.2433352790691\\
-0.26611328125	59.1913420662162\\
-0.265625	59.1398356265249\\
-0.26513671875	59.0888236081808\\
-0.2646484375	59.0383133583614\\
-0.26416015625	58.9883119345206\\
-0.263671875	58.938826115369\\
-0.26318359375	58.8898624115533\\
-0.2626953125	58.8414270760385\\
-0.26220703125	58.7935261141992\\
-0.26171875	58.7461652936219\\
-0.26123046875	58.6993501536257\\
-0.2607421875	58.6530860145061\\
-0.26025390625	58.6073779865062\\
-0.259765625	58.5622309785228\\
-0.25927734375	58.517649706552\\
-0.2587890625	58.4736387018805\\
-0.25830078125	58.4302023190288\\
-0.2578125	58.3873447434514\\
-0.25732421875	58.345069999002\\
-0.2568359375	58.3033819551674\\
-0.25634765625	58.2622843340783\\
-0.255859375	58.2217807173019\\
-0.25537109375	58.181874552422\\
-0.2548828125	58.1425691594141\\
-0.25439453125	58.1038677368207\\
-0.25390625	58.0657733677314\\
-0.25341796875	58.0282890255758\\
-0.2529296875	57.9914175797342\\
-0.25244140625	57.9551618009704\\
-0.251953125	57.9195243666942\\
-0.25146484375	57.884507866058\\
-0.2509765625	57.8501148048928\\
-0.25048828125	57.8163476104896\\
-0.25	57.7832086362299\\
-0.24951171875	57.7507001660726\\
-0.2490234375	57.7188244189003\\
-0.24853515625	57.6875835527307\\
-0.248046875	57.6569796687974\\
-0.24755859375	57.6270148155056\\
-0.2470703125	57.5976909922664\\
-0.24658203125	57.5690101532139\\
-0.24609375	57.5409742108103\\
-0.24560546875	57.513585039343\\
-0.2451171875	57.4868444783167\\
-0.24462890625	57.4607543357456\\
-0.244140625	57.43531639135\\
-0.24365234375	57.4105323996589\\
-0.2431640625	57.3864040930248\\
-0.24267578125	57.3629331845532\\
-0.2421875	57.3401213709491\\
-0.24169921875	57.3179703352864\\
-0.2412109375	57.2964817497014\\
-0.24072265625	57.275657278014\\
-0.240234375	57.255498578281\\
-0.23974609375	57.2360073052827\\
-0.2392578125	57.2171851129469\\
-0.23876953125	57.199033656713\\
-0.23828125	57.1815545958387\\
-0.23779296875	57.1647495956519\\
-0.2373046875	57.1486203297515\\
-0.23681640625	57.133168482158\\
-0.236328125	57.1183957494179\\
-0.23583984375	57.1043038426635\\
-0.2353515625	57.0908944896305\\
-0.23486328125	57.0781694366365\\
-0.234375	57.0661304505214\\
-0.23388671875	57.0547793205529\\
-0.2333984375	57.0441178602992\\
-0.23291015625	57.0341479094698\\
-0.232421875	57.0248713357288\\
-0.23193359375	57.0162900364795\\
-0.2314453125	57.0084059406261\\
-0.23095703125	57.0012210103099\\
-0.23046875	56.9947372426267\\
-0.22998046875	56.9889566713227\\
-0.2294921875	56.983881368475\\
-0.22900390625	56.9795134461538\\
-0.228515625	56.975855058073\\
-0.22802734375	56.9729084012264\\
-0.2275390625	56.9706757175149\\
-0.22705078125	56.9691592953629\\
-0.2265625	56.9683614713286\\
-0.22607421875	56.968284631708\\
-0.2255859375	56.9689312141347\\
-0.22509765625	56.9703037091764\\
-0.224609375	56.972404661931\\
-0.22412109375	56.9752366736231\\
-0.2236328125	56.9788024032012\\
-0.22314453125	56.9831045689399\\
-0.22265625	56.9881459500453\\
-0.22216796875	56.993929388268\\
-0.2216796875	57.0004577895233\\
-0.22119140625	57.0077341255204\\
-0.220703125	57.0157614354032\\
-0.22021484375	57.024542827402\\
-0.2197265625	57.0340814805001\\
-0.21923828125	57.0443806461147\\
-0.21875	57.0554436497946\\
-0.21826171875	57.0672738929358\\
-0.2177734375	57.0798748545171\\
-0.21728515625	57.0932500928555\\
-0.216796875	57.1074032473859\\
-0.21630859375	57.1223380404627\\
-0.2158203125	57.1380582791887\\
-0.21533203125	57.1545678572701\\
-0.21484375	57.1718707568997\\
-0.21435546875	57.1899710506716\\
-0.2138671875	57.2088729035261\\
-0.21337890625	57.2285805747291\\
-0.212890625	57.2490984198856\\
-0.21240234375	57.2704308929909\\
-0.2119140625	57.2925825485184\\
-0.21142578125	57.3155580435493\\
-0.2109375	57.3393621399424\\
-0.21044921875	57.3639997065494\\
-0.2099609375	57.3894757214727\\
-0.20947265625	57.4157952743734\\
-0.208984375	57.4429635688261\\
-0.20849609375	57.4709859247257\\
-0.2080078125	57.4998677807474\\
-0.20751953125	57.5296146968607\\
-0.20703125	57.5602323569003\\
-0.20654296875	57.5917265711983\\
-0.2060546875	57.6241032792746\\
-0.20556640625	57.6573685525923\\
-0.205078125	57.6915285973789\\
-0.20458984375	57.7265897575154\\
-0.2041015625	57.7625585174948\\
-0.20361328125	57.7994415054546\\
-0.203125	57.8372454962842\\
-0.20263671875	57.8759774148097\\
-0.2021484375	57.9156443390592\\
-0.20166015625	57.9562535036122\\
-0.201171875	57.9978123030334\\
-0.20068359375	58.0403282953958\\
-0.2001953125	58.0838092058962\\
-0.19970703125	58.1282629305641\\
-0.19921875	58.1736975400686\\
-0.19873046875	58.2201212836261\\
-0.1982421875	58.2675425930121\\
-0.19775390625	58.3159700866803\\
-0.197265625	58.3654125739917\\
-0.19677734375	58.4158790595585\\
-0.1962890625	58.4673787477055\\
-0.19580078125	58.5199210470522\\
-0.1953125	58.5735155752209\\
-0.19482421875	58.6281721636726\\
-0.1943359375	58.6839008626762\\
-0.19384765625	58.7407119464144\\
-0.193359375	58.7986159182294\\
-0.19287109375	58.8576235160155\\
-0.1923828125	58.917745717758\\
-0.19189453125	58.9789937472285\\
-0.19140625	59.0413790798355\\
-0.19091796875	59.1049134486392\\
-0.1904296875	59.1696088505315\\
-0.18994140625	59.2354775525889\\
-0.189453125	59.3025320985998\\
-0.18896484375	59.3707853157741\\
-0.1884765625	59.4402503216369\\
-0.18798828125	59.5109405311128\\
-0.1875	59.5828696638056\\
-0.18701171875	59.6560517514768\\
-0.1865234375	59.7305011457283\\
-0.18603515625	59.806232525895\\
-0.185546875	59.8832609071496\\
-0.18505859375	59.9616016488271\\
-0.1845703125	60.0412704629694\\
-0.18408203125	60.1222834230981\\
-0.18359375	60.2046569732177\\
-0.18310546875	60.2884079370514\\
-0.1826171875	60.373553527516\\
-0.18212890625	60.4601113564373\\
-0.181640625	60.5480994445075\\
-0.18115234375	60.6375362314885\\
-0.1806640625	60.7284405866629\\
-0.18017578125	60.8208318195312\\
-0.1796875	60.9147296907565\\
-0.17919921875	61.0101544233577\\
-0.1787109375	61.1071267141453\\
-0.17822265625	61.2056677454008\\
-0.177734375	61.3057991967919\\
-0.17724609375	61.4075432575195\\
-0.1767578125	61.5109226386877\\
-0.17626953125	61.6159605858885\\
-0.17578125	61.722680891986\\
-0.17529296875	61.8311079100899\\
-0.1748046875	61.9412665666964\\
-0.17431640625	62.0531823749782\\
-0.173828125	62.1668814481996\\
-0.17333984375	62.282390513225\\
-0.1728515625	62.3997369240907\\
-0.17236328125	62.5189486756003\\
-0.171875	62.640054416896\\
-0.17138671875	62.7630834649598\\
-0.1708984375	62.8880658179816\\
-0.17041015625	63.0150321685272\\
-0.169921875	63.1440139164298\\
-0.16943359375	63.2750431813177\\
-0.1689453125	63.4081528146758\\
-0.16845703125	63.5433764113299\\
-0.16796875	63.6807483202224\\
-0.16748046875	63.8203036543351\\
-0.1669921875	63.9620782995911\\
-0.16650390625	64.1061089225501\\
-0.166015625	64.2524329766851\\
-0.16552734375	64.4010887069987\\
-0.1650390625	64.55211515271\\
-0.16455078125	64.7055521477066\\
-0.1640625	64.8614403184166\\
-0.16357421875	65.0198210787103\\
-0.1630859375	65.1807366213948\\
-0.16259765625	65.3442299058053\\
-0.162109375	65.5103446409384\\
-0.16162109375	65.6791252634943\\
-0.1611328125	65.8506169101261\\
-0.16064453125	66.0248653830936\\
-0.16015625	66.2019171084275\\
-0.15966796875	66.38181908559\\
-0.1591796875	66.5646188274917\\
-0.15869140625	66.7503642895889\\
-0.158203125	66.9391037866054\\
-0.15771484375	67.1308858952666\\
-0.1572265625	67.3257593412063\\
-0.15673828125	67.5237728679928\\
-0.15625	67.7249750859588\\
-0.15576171875	67.929414298227\\
-0.1552734375	68.1371383010109\\
-0.15478515625	68.3481941548939\\
-0.154296875	68.5626279233984\\
-0.15380859375	68.7804843746967\\
-0.1533203125	69.0018066418135\\
-0.15283203125	69.2266358361138\\
-0.15234375	69.4550106082411\\
-0.15185546875	69.6869666499966\\
-0.1513671875	69.9225361298788\\
-0.15087890625	70.1617470541898\\
-0.150390625	70.4046225446986\\
-0.14990234375	70.6511800228798\\
-0.1494140625	70.9014302896902\\
-0.14892578125	71.1553764887308\\
-0.1484375	71.4130129394635\\
-0.14794921875	71.6743238259423\\
-0.1474609375	71.9392817252907\\
-0.14697265625	72.2078459589697\\
-0.146484375	72.4799607487485\\
-0.14599609375	72.755553158349\\
-0.1455078125	73.0345308009973\\
-0.14501953125	73.3167792927975\\
-0.14453125	73.602159432035\\
-0.14404296875	73.8905040854655\\
-0.1435546875	74.181614764604\\
-0.14306640625	74.4752578783064\\
-0.142578125	74.7711606529327\\
-0.14208984375	75.0690067185628\\
-0.1416015625	75.3684313696136\\
-0.14111328125	75.6690165214373\\
-0.140625	75.9702854016689\\
-0.14013671875	76.2716970369775\\
-0.1396484375	76.5726406230886\\
-0.13916015625	76.8724298990271\\
-0.138671875	77.1702976858456\\
-0.13818359375	77.4653907955145\\
-0.1376953125	77.7567655665816\\
-0.13720703125	78.0433843380295\\
-0.13671875	78.324113228968\\
-0.13623046875	78.5977216451579\\
-0.1357421875	78.862883978308\\
-0.13525390625	79.1181839929719\\
-0.134765625	79.362122399685\\
-0.13427734375	79.5931280813078\\
-0.1337890625	79.8095733619536\\
-0.13330078125	80.0097935750845\\
-0.1328125	80.1921109935893\\
-0.13232421875	80.3548629298514\\
-0.1318359375	80.4964335061862\\
-0.13134765625	80.6152882540698\\
-0.130859375	80.7100103533068\\
-0.13037109375	80.779337008247\\
-0.1298828125	80.8221942210467\\
-0.12939453125	80.8377281047066\\
-0.12890625	80.8253309150405\\
-0.12841796875	80.7846601880345\\
-0.1279296875	80.7156497415658\\
-0.12744140625	80.6185118077055\\
-0.126953125	80.4937301517266\\
-0.12646484375	80.3420446401959\\
-0.1259765625	80.1644282743316\\
-0.12548828125	79.9620581459945\\
-0.125	79.7362820610536\\
-0.12451171875	79.4885826909062\\
-0.1240234375	79.2205410646869\\
-0.12353515625	78.9338010290495\\
-0.123046875	78.6300360190239\\
-0.12255859375	78.310919146605\\
-0.1220703125	77.9780972649865\\
-0.12158203125	77.6331693396276\\
-0.12109375	77.2776691760248\\
-0.12060546875	76.9130523310994\\
-0.1201171875	76.5406868740336\\
-0.11962890625	76.161847559793\\
-0.119140625	75.7777129265427\\
-0.11865234375	75.3893648165961\\
-0.1181640625	74.9977898388326\\
-0.11767578125	74.6038823289443\\
-0.1171875	74.2084484140919\\
-0.11669921875	73.8122108440177\\
-0.1162109375	73.4158143065443\\
-0.11572265625	73.0198309984163\\
-0.115234375	72.6247662705629\\
-0.11474609375	72.2310642091013\\
-0.1142578125	71.8391130493882\\
-0.11376953125	71.4492503503516\\
-0.11328125	71.0617678806499\\
-0.11279296875	70.6769161875236\\
-0.1123046875	70.2949088342089\\
-0.11181640625	69.9159263031675\\
-0.111328125	69.540119570763\\
-0.11083984375	69.1676133649763\\
-0.1103515625	68.7985091217914\\
-0.10986328125	68.4328876584153\\
-0.109375	68.0708115828832\\
-0.10888671875	67.712327460129\\
-0.1083984375	67.3574677544854\\
-0.10791015625	67.0062525680324\\
-0.107421875	66.6586911933497\\
-0.10693359375	66.3147834981734\\
-0.1064453125	65.9745211582942\\
-0.10595703125	65.6378887538134\\
-0.10546875	65.3048647426549\\
-0.10498046875	64.9754223240346\\
-0.1044921875	64.6495302034413\\
-0.10400390625	64.3271532696016\\
-0.103515625	64.0082531928829\\
-0.10302734375	63.6927889536544\\
-0.1025390625	63.3807173082573\\
-0.10205078125	63.0719931994493\\
-0.1015625	62.7665701174687\\
-0.10107421875	62.4644004172083\\
-0.1005859375	62.1654355964114\\
-0.10009765625	61.8696265392585\\
-0.099609375	61.5769237292512\\
-0.09912109375	61.2872774348666\\
-0.0986328125	61.0006378710712\\
-0.09814453125	60.7169553394512\\
-0.09765625	60.4361803494033\\
-0.09716796875	60.1582637225637\\
-0.0966796875	59.8831566824058\\
-0.09619140625	59.6108109307305\\
-0.095703125	59.3411787125712\\
-0.09521484375	59.0742128708724\\
-0.0947265625	58.8098668921453\\
-0.09423828125	58.5480949441703\\
-0.09375	58.2888519066973\\
-0.09326171875	58.0320933959861\\
-0.0927734375	57.7777757839346\\
-0.09228515625	57.5258562124625\\
-0.091796875	57.2762926037371\\
-0.09130859375	57.0290436667647\\
-0.0908203125	56.7840689008149\\
-0.09033203125	56.5413285960853\\
-0.08984375	56.3007838319742\\
-0.08935546875	56.0623964732842\\
-0.0888671875	55.8261291646436\\
-0.08837890625	55.5919453233989\\
-0.087890625	55.3598091312028\\
-0.08740234375	55.1296855244984\\
-0.0869140625	54.901540184073\\
-0.08642578125	54.6753395238377\\
-0.0859375	54.4510506789703\\
-0.08544921875	54.2286414935412\\
-0.0849609375	54.0080805077298\\
-0.08447265625	53.7893369447225\\
-0.083984375	53.5723806973791\\
-0.08349609375	53.3571823147335\\
-0.0830078125	53.1437129883967\\
-0.08251953125	52.9319445389144\\
-0.08203125	52.7218494021273\\
-0.08154296875	52.5134006155762\\
-0.0810546875	52.306571804988\\
-0.08056640625	52.1013371708719\\
-0.080078125	51.8976714752538\\
-0.07958984375	51.695550028571\\
-0.0791015625	51.4949486767458\\
-0.07861328125	51.2958437884549\\
-0.078125	51.0982122426064\\
-0.07763671875	50.9020314160366\\
-0.0771484375	50.7072791714356\\
-0.07666015625	50.513933845506\\
-0.076171875	50.3219742373634\\
-0.07568359375	50.1313795971801\\
-0.0751953125	49.9421296150748\\
-0.07470703125	49.7542044102507\\
-0.07421875	49.5675845203814\\
-0.07373046875	49.3822508912447\\
-0.0732421875	49.1981848666024\\
-0.07275390625	49.015368178326\\
-0.072265625	48.8337829367638\\
-0.07177734375	48.6534116213476\\
-0.0712890625	48.4742370714369\\
-0.07080078125	48.2962424773951\\
-0.0703125	48.1194113718946\\
-0.06982421875	47.943727621448\\
-0.0693359375	47.7691754181591\\
-0.06884765625	47.5957392716912\\
-0.068359375	47.4234040014462\\
-0.06787109375	47.2521547289531\\
-0.0673828125	47.081976870457\\
-0.06689453125	46.9128561297083\\
-0.06640625	46.7447784909434\\
-0.06591796875	46.577730212056\\
-0.0654296875	46.4116978179513\\
-0.06494140625	46.2466680940801\\
-0.064453125	46.0826280801475\\
-0.06396484375	45.9195650639929\\
-0.0634765625	45.7574665756343\\
-0.06298828125	45.5963203814751\\
-0.0625	45.4361144786681\\
-0.06201171875	45.2768370896317\\
-0.0615234375	45.1184766567149\\
-0.06103515625	44.9610218370071\\
-0.060546875	44.8044614972882\\
-0.06005859375	44.6487847091155\\
-0.0595703125	44.4939807440424\\
-0.05908203125	44.3400390689676\\
-0.05859375	44.1869493416087\\
-0.05810546875	44.034701406097\\
-0.0576171875	43.8832852886918\\
-0.05712890625	43.7326911936089\\
-0.056640625	43.5829094989604\\
-0.05615234375	43.4339307528034\\
-0.0556640625	43.2857456692936\\
-0.05517578125	43.1383451249414\\
-0.0546875	42.9917201549664\\
-0.05419921875	42.8458619497496\\
-0.0537109375	42.7007618513779\\
-0.05322265625	42.5564113502799\\
-0.052734375	42.4128020819505\\
-0.05224609375	42.26992582376\\
-0.0517578125	42.1277744918475\\
-0.05126953125	41.9863401380945\\
-0.05078125	41.845614947177\\
-0.05029296875	41.7055912336935\\
-0.0498046875	41.566261439368\\
-0.04931640625	41.4276181303234\\
-0.048828125	41.2896539944261\\
-0.04833984375	41.1523618386972\\
-0.0478515625	41.0157345867899\\
-0.04736328125	40.8797652765315\\
-0.046875	40.7444470575265\\
-0.04638671875	40.6097731888206\\
-0.0458984375	40.4757370366233\\
-0.04541015625	40.3423320720871\\
-0.044921875	40.2095518691425\\
-0.04443359375	40.0773901023865\\
-0.0439453125	39.9458405450231\\
-0.04345703125	39.8148970668553\\
-0.04296875	39.6845536323257\\
-0.04248046875	39.5548042986052\\
-0.0419921875	39.425643213729\\
-0.04150390625	39.2970646147767\\
-0.041015625	39.1690628260978\\
-0.04052734375	39.0416322575787\\
-0.0400390625	38.9147674029525\\
-0.03955078125	38.7884628381486\\
-0.0390625	38.6627132196821\\
-0.03857421875	38.5375132830806\\
-0.0380859375	38.4128578413499\\
-0.03759765625	38.2887417834741\\
-0.037109375	38.1651600729524\\
-0.03662109375	38.042107746369\\
-0.0361328125	37.9195799119978\\
-0.03564453125	37.7975717484375\\
-0.03515625	37.6760785032809\\
-0.03466796875	37.5550954918123\\
-0.0341796875	37.4346180957367\\
-0.03369140625	37.3146417619375\\
-0.033203125	37.1951620012622\\
-0.03271484375	37.0761743873362\\
-0.0322265625	36.9576745554034\\
-0.03173828125	36.8396582011925\\
-0.03125	36.7221210798096\\
-0.03076171875	36.6050590046547\\
-0.0302734375	36.4884678463631\\
-0.02978515625	36.37234353177\\
-0.029296875	36.2566820428977\\
-0.02880859375	36.141479415966\\
-0.0283203125	36.0267317404236\\
-0.02783203125	35.9124351580008\\
-0.02734375	35.7985858617836\\
-0.02685546875	35.6851800953064\\
-0.0263671875	35.5722141516662\\
-0.02587890625	35.4596843726546\\
-0.025390625	35.3475871479086\\
-0.02490234375	35.2359189140805\\
-0.0244140625	35.1246761540239\\
-0.02392578125	35.0138553959987\\
-0.0234375	34.903453212892\\
-0.02294921875	34.7934662214553\\
-0.0224609375	34.6838910815582\\
-0.02197265625	34.5747244954576\\
-0.021484375	34.4659632070821\\
-0.02099609375	34.3576040013312\\
-0.0205078125	34.2496437033888\\
-0.02001953125	34.1420791780517\\
-0.01953125	34.0349073290711\\
-0.01904296875	33.9281250985075\\
-0.0185546875	33.8217294660995\\
-0.01806640625	33.7157174486449\\
-0.017578125	33.6100860993945\\
-0.01708984375	33.5048325074578\\
-0.0166015625	33.3999537972215\\
-0.01611328125	33.2954471277784\\
-0.015625	33.1913096923689\\
-0.01513671875	33.0875387178332\\
-0.0146484375	32.9841314640737\\
-0.01416015625	32.8810852235289\\
-0.013671875	32.7783973206574\\
-0.01318359375	32.6760651114321\\
-0.0126953125	32.5740859828434\\
-0.01220703125	32.4724573524137\\
-0.01171875	32.3711766677199\\
-0.01123046875	32.2702414059256\\
-0.0107421875	32.1696490733227\\
-0.01025390625	32.0693972048813\\
-0.009765625	31.9694833638081\\
-0.00927734375	31.8699051411139\\
-0.0087890625	31.7706601551886\\
-0.00830078125	31.6717460513847\\
-0.0078125	31.5731605016083\\
-0.00732421875	31.4749012039181\\
-0.0068359375	31.3769658821319\\
-0.00634765625	31.2793522854402\\
-0.005859375	31.1820581880267\\
-0.00537109375	31.0850813886972\\
-0.0048828125	30.9884197105134\\
-0.00439453125	30.8920710004352\\
-0.00390625	30.7960331289682\\
-0.00341796875	30.7003039898186\\
-0.0029296875	30.604881499554\\
-0.00244140625	30.5097635972702\\
-0.001953125	30.4149482442643\\
-0.00146484375	30.3204334237138\\
-0.0009765625	30.2262171403607\\
-0.00048828125	30.1322974202027\\
0	30.0386723101882\\
0.00048828125	30.1322974202027\\
0.0009765625	30.2262171403607\\
0.00146484375	30.3204334237138\\
0.001953125	30.4149482442643\\
0.00244140625	30.5097635972702\\
0.0029296875	30.604881499554\\
0.00341796875	30.7003039898186\\
0.00390625	30.7960331289682\\
0.00439453125	30.8920710004352\\
0.0048828125	30.9884197105134\\
0.00537109375	31.0850813886972\\
0.005859375	31.1820581880267\\
0.00634765625	31.2793522854402\\
0.0068359375	31.3769658821319\\
0.00732421875	31.4749012039181\\
0.0078125	31.5731605016083\\
0.00830078125	31.6717460513847\\
0.0087890625	31.7706601551886\\
0.00927734375	31.8699051411139\\
0.009765625	31.9694833638081\\
0.01025390625	32.0693972048813\\
0.0107421875	32.1696490733227\\
0.01123046875	32.2702414059256\\
0.01171875	32.3711766677199\\
0.01220703125	32.4724573524137\\
0.0126953125	32.5740859828434\\
0.01318359375	32.6760651114321\\
0.013671875	32.7783973206574\\
0.01416015625	32.8810852235289\\
0.0146484375	32.9841314640737\\
0.01513671875	33.0875387178332\\
0.015625	33.1913096923689\\
0.01611328125	33.2954471277784\\
0.0166015625	33.3999537972215\\
0.01708984375	33.5048325074578\\
0.017578125	33.6100860993945\\
0.01806640625	33.7157174486449\\
0.0185546875	33.8217294660995\\
0.01904296875	33.9281250985075\\
0.01953125	34.0349073290711\\
0.02001953125	34.1420791780517\\
0.0205078125	34.2496437033888\\
0.02099609375	34.3576040013312\\
0.021484375	34.4659632070821\\
0.02197265625	34.5747244954576\\
0.0224609375	34.6838910815582\\
0.02294921875	34.7934662214553\\
0.0234375	34.903453212892\\
0.02392578125	35.0138553959987\\
0.0244140625	35.1246761540239\\
0.02490234375	35.2359189140805\\
0.025390625	35.3475871479086\\
0.02587890625	35.4596843726546\\
0.0263671875	35.5722141516662\\
0.02685546875	35.6851800953064\\
0.02734375	35.7985858617836\\
0.02783203125	35.9124351580008\\
0.0283203125	36.0267317404236\\
0.02880859375	36.141479415966\\
0.029296875	36.2566820428977\\
0.02978515625	36.37234353177\\
0.0302734375	36.4884678463631\\
0.03076171875	36.6050590046547\\
0.03125	36.7221210798096\\
0.03173828125	36.8396582011925\\
0.0322265625	36.9576745554034\\
0.03271484375	37.0761743873362\\
0.033203125	37.1951620012622\\
0.03369140625	37.3146417619375\\
0.0341796875	37.4346180957367\\
0.03466796875	37.5550954918123\\
0.03515625	37.6760785032809\\
0.03564453125	37.7975717484375\\
0.0361328125	37.9195799119978\\
0.03662109375	38.042107746369\\
0.037109375	38.1651600729524\\
0.03759765625	38.2887417834741\\
0.0380859375	38.4128578413499\\
0.03857421875	38.5375132830806\\
0.0390625	38.6627132196821\\
0.03955078125	38.7884628381486\\
0.0400390625	38.9147674029525\\
0.04052734375	39.0416322575787\\
0.041015625	39.1690628260978\\
0.04150390625	39.2970646147767\\
0.0419921875	39.425643213729\\
0.04248046875	39.5548042986052\\
0.04296875	39.6845536323257\\
0.04345703125	39.8148970668553\\
0.0439453125	39.9458405450231\\
0.04443359375	40.0773901023865\\
0.044921875	40.2095518691425\\
0.04541015625	40.3423320720871\\
0.0458984375	40.4757370366233\\
0.04638671875	40.6097731888206\\
0.046875	40.7444470575265\\
0.04736328125	40.8797652765315\\
0.0478515625	41.0157345867899\\
0.04833984375	41.1523618386972\\
0.048828125	41.2896539944261\\
0.04931640625	41.4276181303234\\
0.0498046875	41.566261439368\\
0.05029296875	41.7055912336935\\
0.05078125	41.845614947177\\
0.05126953125	41.9863401380945\\
0.0517578125	42.1277744918475\\
0.05224609375	42.26992582376\\
0.052734375	42.4128020819505\\
0.05322265625	42.5564113502799\\
0.0537109375	42.7007618513779\\
0.05419921875	42.8458619497496\\
0.0546875	42.9917201549664\\
0.05517578125	43.1383451249414\\
0.0556640625	43.2857456692936\\
0.05615234375	43.4339307528034\\
0.056640625	43.5829094989604\\
0.05712890625	43.7326911936089\\
0.0576171875	43.8832852886918\\
0.05810546875	44.034701406097\\
0.05859375	44.1869493416087\\
0.05908203125	44.3400390689676\\
0.0595703125	44.4939807440424\\
0.06005859375	44.6487847091155\\
0.060546875	44.8044614972882\\
0.06103515625	44.9610218370071\\
0.0615234375	45.1184766567149\\
0.06201171875	45.2768370896317\\
0.0625	45.4361144786681\\
0.06298828125	45.5963203814751\\
0.0634765625	45.7574665756343\\
0.06396484375	45.9195650639929\\
0.064453125	46.0826280801475\\
0.06494140625	46.2466680940801\\
0.0654296875	46.4116978179513\\
0.06591796875	46.577730212056\\
0.06640625	46.7447784909434\\
0.06689453125	46.9128561297083\\
0.0673828125	47.081976870457\\
0.06787109375	47.2521547289531\\
0.068359375	47.4234040014462\\
0.06884765625	47.5957392716912\\
0.0693359375	47.7691754181591\\
0.06982421875	47.943727621448\\
0.0703125	48.1194113718946\\
0.07080078125	48.2962424773951\\
0.0712890625	48.4742370714369\\
0.07177734375	48.6534116213476\\
0.072265625	48.8337829367638\\
0.07275390625	49.015368178326\\
0.0732421875	49.1981848666024\\
0.07373046875	49.3822508912447\\
0.07421875	49.5675845203814\\
0.07470703125	49.7542044102507\\
0.0751953125	49.9421296150748\\
0.07568359375	50.1313795971801\\
0.076171875	50.3219742373634\\
0.07666015625	50.513933845506\\
0.0771484375	50.7072791714356\\
0.07763671875	50.9020314160366\\
0.078125	51.0982122426064\\
0.07861328125	51.2958437884549\\
0.0791015625	51.4949486767458\\
0.07958984375	51.695550028571\\
0.080078125	51.8976714752538\\
0.08056640625	52.1013371708719\\
0.0810546875	52.306571804988\\
0.08154296875	52.5134006155762\\
0.08203125	52.7218494021273\\
0.08251953125	52.9319445389144\\
0.0830078125	53.1437129883967\\
0.08349609375	53.3571823147335\\
0.083984375	53.5723806973791\\
0.08447265625	53.7893369447225\\
0.0849609375	54.0080805077298\\
0.08544921875	54.2286414935412\\
0.0859375	54.4510506789703\\
0.08642578125	54.6753395238377\\
0.0869140625	54.901540184073\\
0.08740234375	55.1296855244984\\
0.087890625	55.3598091312028\\
0.08837890625	55.5919453233989\\
0.0888671875	55.8261291646436\\
0.08935546875	56.0623964732842\\
0.08984375	56.3007838319742\\
0.09033203125	56.5413285960853\\
0.0908203125	56.7840689008149\\
0.09130859375	57.0290436667647\\
0.091796875	57.2762926037371\\
0.09228515625	57.5258562124625\\
0.0927734375	57.7777757839346\\
0.09326171875	58.0320933959861\\
0.09375	58.2888519066973\\
0.09423828125	58.5480949441703\\
0.0947265625	58.8098668921453\\
0.09521484375	59.0742128708724\\
0.095703125	59.3411787125712\\
0.09619140625	59.6108109307305\\
0.0966796875	59.8831566824058\\
0.09716796875	60.1582637225637\\
0.09765625	60.4361803494033\\
0.09814453125	60.7169553394512\\
0.0986328125	61.0006378710712\\
0.09912109375	61.2872774348666\\
0.099609375	61.5769237292512\\
0.10009765625	61.8696265392585\\
0.1005859375	62.1654355964114\\
0.10107421875	62.4644004172083\\
0.1015625	62.7665701174687\\
0.10205078125	63.0719931994493\\
0.1025390625	63.3807173082573\\
0.10302734375	63.6927889536544\\
0.103515625	64.0082531928829\\
0.10400390625	64.3271532696016\\
0.1044921875	64.6495302034413\\
0.10498046875	64.9754223240346\\
0.10546875	65.3048647426549\\
0.10595703125	65.6378887538134\\
0.1064453125	65.9745211582942\\
0.10693359375	66.3147834981734\\
0.107421875	66.6586911933497\\
0.10791015625	67.0062525680324\\
0.1083984375	67.3574677544854\\
0.10888671875	67.712327460129\\
0.109375	68.0708115828832\\
0.10986328125	68.4328876584153\\
0.1103515625	68.7985091217914\\
0.11083984375	69.1676133649763\\
0.111328125	69.540119570763\\
0.11181640625	69.9159263031675\\
0.1123046875	70.2949088342089\\
0.11279296875	70.6769161875236\\
0.11328125	71.0617678806499\\
0.11376953125	71.4492503503516\\
0.1142578125	71.8391130493882\\
0.11474609375	72.2310642091013\\
0.115234375	72.6247662705629\\
0.11572265625	73.0198309984163\\
0.1162109375	73.4158143065443\\
0.11669921875	73.8122108440177\\
0.1171875	74.2084484140919\\
0.11767578125	74.6038823289443\\
0.1181640625	74.9977898388326\\
0.11865234375	75.3893648165961\\
0.119140625	75.7777129265427\\
0.11962890625	76.161847559793\\
0.1201171875	76.5406868740336\\
0.12060546875	76.9130523310994\\
0.12109375	77.2776691760248\\
0.12158203125	77.6331693396276\\
0.1220703125	77.9780972649865\\
0.12255859375	78.310919146605\\
0.123046875	78.6300360190239\\
0.12353515625	78.9338010290495\\
0.1240234375	79.2205410646869\\
0.12451171875	79.4885826909062\\
0.125	79.7362820610536\\
0.12548828125	79.9620581459945\\
0.1259765625	80.1644282743316\\
0.12646484375	80.3420446401959\\
0.126953125	80.4937301517266\\
0.12744140625	80.6185118077055\\
0.1279296875	80.7156497415658\\
0.12841796875	80.7846601880345\\
0.12890625	80.8253309150405\\
0.12939453125	80.8377281047066\\
0.1298828125	80.8221942210467\\
0.13037109375	80.779337008247\\
0.130859375	80.7100103533068\\
0.13134765625	80.6152882540698\\
0.1318359375	80.4964335061862\\
0.13232421875	80.3548629298514\\
0.1328125	80.1921109935893\\
0.13330078125	80.0097935750845\\
0.1337890625	79.8095733619536\\
0.13427734375	79.5931280813078\\
0.134765625	79.362122399685\\
0.13525390625	79.1181839929719\\
0.1357421875	78.862883978308\\
0.13623046875	78.5977216451579\\
0.13671875	78.324113228968\\
0.13720703125	78.0433843380295\\
0.1376953125	77.7567655665816\\
0.13818359375	77.4653907955145\\
0.138671875	77.1702976858456\\
0.13916015625	76.8724298990271\\
0.1396484375	76.5726406230886\\
0.14013671875	76.2716970369775\\
0.140625	75.9702854016689\\
0.14111328125	75.6690165214373\\
0.1416015625	75.3684313696136\\
0.14208984375	75.0690067185628\\
0.142578125	74.7711606529327\\
0.14306640625	74.4752578783064\\
0.1435546875	74.181614764604\\
0.14404296875	73.8905040854655\\
0.14453125	73.602159432035\\
0.14501953125	73.3167792927975\\
0.1455078125	73.0345308009973\\
0.14599609375	72.755553158349\\
0.146484375	72.4799607487485\\
0.14697265625	72.2078459589697\\
0.1474609375	71.9392817252907\\
0.14794921875	71.6743238259423\\
0.1484375	71.4130129394635\\
0.14892578125	71.1553764887308\\
0.1494140625	70.9014302896902\\
0.14990234375	70.6511800228798\\
0.150390625	70.4046225446986\\
0.15087890625	70.1617470541898\\
0.1513671875	69.9225361298788\\
0.15185546875	69.6869666499966\\
0.15234375	69.4550106082411\\
0.15283203125	69.2266358361138\\
0.1533203125	69.0018066418135\\
0.15380859375	68.7804843746967\\
0.154296875	68.5626279233984\\
0.15478515625	68.3481941548939\\
0.1552734375	68.1371383010109\\
0.15576171875	67.929414298227\\
0.15625	67.7249750859588\\
0.15673828125	67.5237728679928\\
0.1572265625	67.3257593412063\\
0.15771484375	67.1308858952666\\
0.158203125	66.9391037866054\\
0.15869140625	66.7503642895889\\
0.1591796875	66.5646188274917\\
0.15966796875	66.38181908559\\
0.16015625	66.2019171084275\\
0.16064453125	66.0248653830936\\
0.1611328125	65.8506169101261\\
0.16162109375	65.6791252634943\\
0.162109375	65.5103446409384\\
0.16259765625	65.3442299058053\\
0.1630859375	65.1807366213948\\
0.16357421875	65.0198210787103\\
0.1640625	64.8614403184166\\
0.16455078125	64.7055521477066\\
0.1650390625	64.55211515271\\
0.16552734375	64.4010887069987\\
0.166015625	64.2524329766851\\
0.16650390625	64.1061089225501\\
0.1669921875	63.9620782995911\\
0.16748046875	63.8203036543351\\
0.16796875	63.6807483202224\\
0.16845703125	63.5433764113299\\
0.1689453125	63.4081528146758\\
0.16943359375	63.2750431813177\\
0.169921875	63.1440139164298\\
0.17041015625	63.0150321685272\\
0.1708984375	62.8880658179816\\
0.17138671875	62.7630834649598\\
0.171875	62.640054416896\\
0.17236328125	62.5189486756003\\
0.1728515625	62.3997369240907\\
0.17333984375	62.282390513225\\
0.173828125	62.1668814481996\\
0.17431640625	62.0531823749782\\
0.1748046875	61.9412665666964\\
0.17529296875	61.8311079100899\\
0.17578125	61.722680891986\\
0.17626953125	61.6159605858885\\
0.1767578125	61.5109226386877\\
0.17724609375	61.4075432575195\\
0.177734375	61.3057991967919\\
0.17822265625	61.2056677454008\\
0.1787109375	61.1071267141453\\
0.17919921875	61.0101544233577\\
0.1796875	60.9147296907565\\
0.18017578125	60.8208318195312\\
0.1806640625	60.7284405866629\\
0.18115234375	60.6375362314885\\
0.181640625	60.5480994445075\\
0.18212890625	60.4601113564373\\
0.1826171875	60.373553527516\\
0.18310546875	60.2884079370514\\
0.18359375	60.2046569732177\\
0.18408203125	60.1222834230981\\
0.1845703125	60.0412704629694\\
0.18505859375	59.9616016488271\\
0.185546875	59.8832609071496\\
0.18603515625	59.806232525895\\
0.1865234375	59.7305011457283\\
0.18701171875	59.6560517514768\\
0.1875	59.5828696638056\\
0.18798828125	59.5109405311128\\
0.1884765625	59.4402503216369\\
0.18896484375	59.3707853157741\\
0.189453125	59.3025320985998\\
0.18994140625	59.2354775525889\\
0.1904296875	59.1696088505315\\
0.19091796875	59.1049134486392\\
0.19140625	59.0413790798355\\
0.19189453125	58.9789937472285\\
0.1923828125	58.917745717758\\
0.19287109375	58.8576235160155\\
0.193359375	58.7986159182294\\
0.19384765625	58.7407119464144\\
0.1943359375	58.6839008626762\\
0.19482421875	58.6281721636726\\
0.1953125	58.5735155752209\\
0.19580078125	58.5199210470522\\
0.1962890625	58.4673787477055\\
0.19677734375	58.4158790595585\\
0.197265625	58.3654125739917\\
0.19775390625	58.3159700866803\\
0.1982421875	58.2675425930121\\
0.19873046875	58.2201212836261\\
0.19921875	58.1736975400686\\
0.19970703125	58.1282629305641\\
0.2001953125	58.0838092058962\\
0.20068359375	58.0403282953958\\
0.201171875	57.9978123030334\\
0.20166015625	57.9562535036122\\
0.2021484375	57.9156443390592\\
0.20263671875	57.8759774148097\\
0.203125	57.8372454962842\\
0.20361328125	57.7994415054546\\
0.2041015625	57.7625585174948\\
0.20458984375	57.7265897575154\\
0.205078125	57.6915285973789\\
0.20556640625	57.6573685525923\\
0.2060546875	57.6241032792746\\
0.20654296875	57.5917265711983\\
0.20703125	57.5602323569003\\
0.20751953125	57.5296146968607\\
0.2080078125	57.4998677807474\\
0.20849609375	57.4709859247257\\
0.208984375	57.4429635688261\\
0.20947265625	57.4157952743734\\
0.2099609375	57.3894757214727\\
0.21044921875	57.3639997065494\\
0.2109375	57.3393621399424\\
0.21142578125	57.3155580435493\\
0.2119140625	57.2925825485184\\
0.21240234375	57.2704308929909\\
0.212890625	57.2490984198856\\
0.21337890625	57.2285805747291\\
0.2138671875	57.2088729035261\\
0.21435546875	57.1899710506716\\
0.21484375	57.1718707568997\\
0.21533203125	57.1545678572701\\
0.2158203125	57.1380582791887\\
0.21630859375	57.1223380404627\\
0.216796875	57.1074032473859\\
0.21728515625	57.0932500928555\\
0.2177734375	57.0798748545171\\
0.21826171875	57.0672738929358\\
0.21875	57.0554436497946\\
0.21923828125	57.0443806461147\\
0.2197265625	57.0340814805001\\
0.22021484375	57.024542827402\\
0.220703125	57.0157614354032\\
0.22119140625	57.0077341255204\\
0.2216796875	57.0004577895233\\
0.22216796875	56.993929388268\\
0.22265625	56.9881459500453\\
0.22314453125	56.9831045689399\\
0.2236328125	56.9788024032012\\
0.22412109375	56.9752366736231\\
0.224609375	56.972404661931\\
0.22509765625	56.9703037091764\\
0.2255859375	56.9689312141347\\
0.22607421875	56.968284631708\\
0.2265625	56.9683614713286\\
0.22705078125	56.9691592953629\\
0.2275390625	56.9706757175149\\
0.22802734375	56.9729084012264\\
0.228515625	56.975855058073\\
0.22900390625	56.9795134461538\\
0.2294921875	56.983881368475\\
0.22998046875	56.9889566713227\\
0.23046875	56.9947372426267\\
0.23095703125	57.0012210103099\\
0.2314453125	57.0084059406261\\
0.23193359375	57.0162900364795\\
0.232421875	57.0248713357288\\
0.23291015625	57.0341479094698\\
0.2333984375	57.0441178602992\\
0.23388671875	57.0547793205529\\
0.234375	57.0661304505214\\
0.23486328125	57.0781694366365\\
0.2353515625	57.0908944896305\\
0.23583984375	57.1043038426635\\
0.236328125	57.1183957494179\\
0.23681640625	57.133168482158\\
0.2373046875	57.1486203297515\\
0.23779296875	57.1647495956519\\
0.23828125	57.1815545958387\\
0.23876953125	57.199033656713\\
0.2392578125	57.2171851129469\\
0.23974609375	57.2360073052827\\
0.240234375	57.255498578281\\
0.24072265625	57.275657278014\\
0.2412109375	57.2964817497014\\
0.24169921875	57.3179703352864\\
0.2421875	57.3401213709491\\
0.24267578125	57.3629331845532\\
0.2431640625	57.3864040930248\\
0.24365234375	57.4105323996589\\
0.244140625	57.43531639135\\
0.24462890625	57.4607543357456\\
0.2451171875	57.4868444783167\\
0.24560546875	57.513585039343\\
0.24609375	57.5409742108103\\
0.24658203125	57.5690101532139\\
0.2470703125	57.5976909922664\\
0.24755859375	57.6270148155056\\
0.248046875	57.6569796687974\\
0.24853515625	57.6875835527307\\
0.2490234375	57.7188244189003\\
0.24951171875	57.7507001660726\\
0.25	57.7832086362299\\
0.25048828125	57.8163476104896\\
0.2509765625	57.8501148048928\\
0.25146484375	57.884507866058\\
0.251953125	57.9195243666942\\
0.25244140625	57.9551618009704\\
0.2529296875	57.9914175797342\\
0.25341796875	58.0282890255758\\
0.25390625	58.0657733677314\\
0.25439453125	58.1038677368207\\
0.2548828125	58.1425691594141\\
0.25537109375	58.181874552422\\
0.255859375	58.2217807173019\\
0.25634765625	58.2622843340783\\
0.2568359375	58.3033819551674\\
0.25732421875	58.345069999002\\
0.2578125	58.3873447434514\\
0.25830078125	58.4302023190288\\
0.2587890625	58.4736387018805\\
0.25927734375	58.517649706552\\
0.259765625	58.5622309785228\\
0.26025390625	58.6073779865062\\
0.2607421875	58.6530860145061\\
0.26123046875	58.6993501536257\\
0.26171875	58.7461652936219\\
0.26220703125	58.7935261141992\\
0.2626953125	58.8414270760385\\
0.26318359375	58.8898624115533\\
0.263671875	58.938826115369\\
0.26416015625	58.9883119345206\\
0.2646484375	59.0383133583614\\
0.26513671875	59.0888236081808\\
0.265625	59.1398356265249\\
0.26611328125	59.1913420662162\\
0.2666015625	59.2433352790691\\
0.26708984375	59.2958073042974\\
0.267578125	59.3487498566115\\
0.26806640625	59.402154314004\\
0.2685546875	59.4560117052198\\
0.26904296875	59.5103126969139\\
0.26953125	59.5650475804928\\
0.27001953125	59.620206258644\\
0.2705078125	59.6757782315535\\
0.27099609375	59.7317525828158\\
0.271484375	59.7881179650394\\
0.27197265625	59.8448625851542\\
0.2724609375	59.9019741894285\\
0.27294921875	59.9594400482021\\
0.2734375	60.0172469403474\\
0.27392578125	60.0753811374708\\
0.2744140625	60.1338283878656\\
0.27490234375	60.1925739002361\\
0.275390625	60.2516023272079\\
0.27587890625	60.3108977486472\\
0.2763671875	60.3704436548107\\
0.27685546875	60.4302229293525\\
0.27734375	60.4902178322171\\
0.27783203125	60.5504099824497\\
0.2783203125	60.6107803409587\\
0.27880859375	60.67130919327\\
0.279296875	60.731976132313\\
0.27978515625	60.7927600412863\\
0.2802734375	60.8536390766524\\
0.28076171875	60.914590651314\\
0.28125	60.9755914180319\\
0.28173828125	61.0366172531473\\
0.2822265625	61.0976432406743\\
0.28271484375	61.158643656837\\
0.283203125	61.2195919551274\\
0.28369140625	61.280460751967\\
0.2841796875	61.3412218130584\\
0.28466796875	61.4018460405226\\
0.28515625	61.4623034609161\\
0.28564453125	61.5225632142355\\
0.2861328125	61.582593544015\\
0.28662109375	61.6423617886319\\
0.287109375	61.7018343739409\\
0.28759765625	61.7609768073577\\
0.2880859375	61.8197536735251\\
0.28857421875	61.8781286316904\\
0.2890625	61.9360644149357\\
0.28955078125	61.9935228313988\\
0.2900390625	62.0504647676319\\
0.29052734375	62.1068501942428\\
0.291015625	62.16263817397\\
0.29150390625	62.2177868723394\\
0.2919921875	62.2722535710553\\
0.29248046875	62.3259946842753\\
0.29296875	62.3789657779162\\
0.29345703125	62.4311215921396\\
0.2939453125	62.4824160671556\\
0.29443359375	62.5328023724837\\
0.294921875	62.5822329397987\\
0.29541015625	62.6306594994843\\
0.2958984375	62.6780331210053\\
0.29638671875	62.7243042571971\\
0.296875	62.7694227925608\\
0.29736328125	62.8133380956345\\
0.2978515625	62.8559990754925\\
0.29833984375	62.8973542424126\\
0.298828125	62.9373517727204\\
0.29931640625	62.9759395778065\\
0.2998046875	63.013065377282\\
0.30029296875	63.048676776214\\
0.30078125	63.0827213463544\\
0.30126953125	63.1151467112476\\
0.3017578125	63.1459006350679\\
0.30224609375	63.1749311150126\\
0.302734375	63.202186477037\\
0.30322265625	63.2276154746909\\
0.3037109375	63.251167390777\\
0.30419921875	63.2727921415231\\
0.3046875	63.2924403829244\\
0.30517578125	63.31006361888\\
0.3056640625	63.3256143107188\\
0.30615234375	63.339045987677\\
0.306640625	63.3503133578654\\
0.30712890625	63.3593724192387\\
0.3076171875	63.3661805700548\\
0.30810546875	63.3706967182965\\
0.30859375	63.3728813895106\\
0.30908203125	63.3726968325084\\
0.3095703125	63.3701071223669\\
0.31005859375	63.3650782601645\\
0.310546875	63.3575782688902\\
0.31103515625	63.3475772849733\\
0.3115234375	63.3350476448914\\
0.31201171875	63.3199639663354\\
0.3125	63.3023032234317\\
0.31298828125	63.2820448155505\\
0.3134765625	63.2591706292663\\
0.31396484375	63.2336650930662\\
0.314453125	63.2055152244553\\
0.31494140625	63.1747106691456\\
0.3154296875	63.1412437320698\\
0.31591796875	63.1051094000114\\
0.31640625	63.0663053557019\\
0.31689453125	63.0248319832845\\
0.3173828125	62.9806923651151\\
0.31787109375	62.9338922699125\\
0.318359375	62.8844401323459\\
0.31884765625	62.8323470241897\\
0.3193359375	62.7776266172435\\
0.31982421875	62.7202951382581\\
0.3203125	62.6603713161692\\
0.32080078125	62.5978763219794\\
0.3212890625	62.532833701677\\
0.32177734375	62.4652693026201\\
0.322265625	62.3952111938461\\
0.32275390625	62.3226895807992\\
0.3232421875	62.2477367149927\\
0.32373046875	62.1703867991416\\
0.32421875	62.0906758883171\\
0.32470703125	62.0086417876834\\
0.3251953125	61.9243239473786\\
0.32568359375	61.8377633551064\\
0.326171875	61.7490024269934\\
0.32666015625	61.6580848972616\\
0.3271484375	61.5650557072494\\
0.32763671875	61.4699608942959\\
0.328125	61.3728474809848\\
0.32861328125	61.273763365215\\
0.3291015625	61.1727572115473\\
0.32958984375	61.0698783442327\\
0.330078125	60.9651766423158\\
0.33056640625	60.8587024371579\\
0.3310546875	60.7505064127025\\
0.33154296875	60.6406395087687\\
0.33203125	60.5291528276233\\
0.33251953125	60.4160975440544\\
0.3330078125	60.301524819127\\
0.33349609375	60.1854857177831\\
0.333984375	60.0680311304038\\
0.33447265625	59.9492116984298\\
0.3349609375	59.8290777441081\\
0.33544921875	59.7076792044018\\
0.3359375	59.5850655690799\\
0.33642578125	59.4612858229754\\
0.3369140625	59.3363883923874\\
0.33740234375	59.210421095572\\
0.337890625	59.0834310972587\\
0.33837890625	58.9554648671098\\
0.3388671875	58.8265681420252\\
0.33935546875	58.6967858921847\\
0.33984375	58.5661622907094\\
0.34033203125	58.4347406868145\\
0.3408203125	58.3025635823174\\
0.34130859375	58.1696726113632\\
0.341796875	58.0361085232207\\
0.34228515625	57.9019111680012\\
0.3427734375	57.7671194851518\\
0.34326171875	57.6317714945702\\
0.34375	57.4959042901932\\
0.34423828125	57.3595540359064\\
0.3447265625	57.2227559636305\\
0.34521484375	57.0855443734372\\
0.345703125	56.9479526355519\\
0.34619140625	56.8100131941069\\
0.3466796875	56.6717575725087\\
0.34716796875	56.5332163802902\\
0.34765625	56.3944193213213\\
0.34814453125	56.25539520326\\
0.3486328125	56.1161719481247\\
0.34912109375	55.9767766038807\\
0.349609375	55.8372353569342\\
0.35009765625	55.6975735454357\\
0.3505859375	55.5578156732963\\
0.35107421875	55.41798542483\\
0.3515625	55.278105679937\\
0.35205078125	55.13819852975\\
0.3525390625	54.998285292668\\
0.35302734375	54.8583865307121\\
0.353515625	54.7185220661353\\
0.35400390625	54.5787109982299\\
0.3544921875	54.4389717202747\\
0.35498046875	54.2993219365734\\
0.35546875	54.1597786795351\\
0.35595703125	54.0203583267558\\
0.3564453125	53.8810766180591\\
0.35693359375	53.7419486724629\\
0.357421875	53.6029890050368\\
0.35791015625	53.4642115436226\\
0.3583984375	53.3256296453896\\
0.35888671875	53.1872561132028\\
0.359375	53.0491032117803\\
0.35986328125	52.9111826836239\\
0.3603515625	52.7735057647028\\
0.36083984375	52.6360831998796\\
0.361328125	52.4989252580628\\
0.36181640625	52.3620417470779\\
0.3623046875	52.2254420282456\\
0.36279296875	52.0891350306613\\
0.36328125	51.9531292651698\\
0.36376953125	51.8174328380283\\
0.3642578125	51.6820534642587\\
0.36474609375	51.546998480681\\
0.365234375	51.4122748586319\\
0.36572265625	51.2778892163646\\
0.3662109375	51.1438478311312\\
0.36669921875	51.0101566509493\\
0.3671875	50.8768213060549\\
0.36767578125	50.7438471200416\\
0.3681640625	50.6112391206931\\
0.36865234375	50.4790020505087\\
0.369140625	50.347140376928\\
0.36962890625	50.2156583022581\\
0.3701171875	50.0845597733085\\
0.37060546875	49.9538484907377\\
0.37109375	49.823527918118\\
0.37158203125	49.6936012907221\\
0.3720703125	49.5640716240394\\
0.37255859375	49.4349417220242\\
0.373046875	49.3062141850861\\
0.37353515625	49.177891417824\\
0.3740234375	49.0499756365124\\
0.37451171875	48.9224688763456\\
0.375	48.7953729984447\\
0.37548828125	48.6686896966346\\
0.3759765625	48.5424205039958\\
0.37646484375	48.4165667991985\\
0.376953125	48.2911298126234\\
0.37744140625	48.1661106322758\\
0.3779296875	48.0415102094987\\
0.37841796875	47.9173293644908\\
0.37890625	47.7935687916339\\
0.37939453125	47.6702290646382\\
0.3798828125	47.5473106415063\\
0.38037109375	47.4248138693264\\
0.380859375	47.3027389888957\\
0.38134765625	47.1810861391821\\
0.3818359375	47.0598553616274\\
0.38232421875	46.9390466042979\\
0.3828125	46.8186597258872\\
0.38330078125	46.6986944995755\\
0.3837890625	46.5791506167499\\
0.38427734375	46.4600276905916\\
0.384765625	46.3413252595314\\
0.38525390625	46.2230427905817\\
0.3857421875	46.1051796825458\\
0.38623046875	45.9877352691102\\
0.38671875	45.8707088218232\\
0.38720703125	45.7540995529646\\
0.3876953125	45.6379066183084\\
0.38818359375	45.5221291197844\\
0.388671875	45.4067661080393\\
0.38916015625	45.2918165849044\\
0.3896484375	45.1772795057693\\
0.39013671875	45.0631537818678\\
0.390625	44.9494382824768\\
0.39111328125	44.836131837034\\
0.3916015625	44.7232332371735\\
0.39208984375	44.6107412386857\\
0.392578125	44.4986545634021\\
0.39306640625	44.3869719010082\\
0.3935546875	44.2756919107871\\
0.39404296875	44.1648132232965\\
0.39453125	44.054334441981\\
0.39501953125	43.9442541447216\\
0.3955078125	43.8345708853264\\
0.39599609375	43.7252831949625\\
0.396484375	43.6163895835323\\
0.39697265625	43.5078885409965\\
0.3974609375	43.3997785386445\\
0.39794921875	43.2920580303159\\
0.3984375	43.1847254535725\\
0.39892578125	43.0777792308249\\
0.3994140625	42.9712177704143\\
0.39990234375	42.8650394676503\\
0.400390625	42.7592427058087\\
0.40087890625	42.6538258570876\\
0.4013671875	42.548787283526\\
0.40185546875	42.444125337885\\
0.40234375	42.3398383644934\\
0.40283203125	42.2359247000579\\
0.4033203125	42.1323826744415\\
0.40380859375	42.0292106114087\\
0.404296875	41.9264068293406\\
0.40478515625	41.8239696419199\\
0.4052734375	41.7218973587876\\
0.40576171875	41.6201882861718\\
0.40625	41.5188407274907\\
0.40673828125	41.417852983929\\
0.4072265625	41.3172233549907\\
0.40771484375	41.2169501390278\\
0.408203125	41.1170316337462\\
0.40869140625	41.0174661366896\\
0.4091796875	40.9182519457025\\
0.40966796875	40.8193873593726\\
0.41015625	40.7208706774539\\
0.41064453125	40.622700201271\\
0.4111328125	40.5248742341043\\
0.41162109375	40.4273910815597\\
0.412109375	40.3302490519191\\
0.41259765625	40.2334464564768\\
0.4130859375	40.1369816098593\\
0.41357421875	40.0408528303298\\
0.4140625	39.9450584400796\\
0.41455078125	39.8495967655042\\
0.4150390625	39.754466137467\\
0.41552734375	39.6596648915499\\
0.416015625	39.5651913682912\\
0.41650390625	39.4710439134124\\
0.4169921875	39.3772208780327\\
0.41748046875	39.2837206188734\\
0.41796875	39.1905414984507\\
0.41845703125	39.0976818852596\\
0.4189453125	39.0051401539467\\
0.41943359375	38.9129146854749\\
0.419921875	38.8210038672777\\
0.42041015625	38.7294060934064\\
0.4208984375	38.6381197646675\\
0.42138671875	38.5471432887527\\
0.421875	38.4564750803616\\
0.42236328125	38.3661135613164\\
0.4228515625	38.2760571606693\\
0.42333984375	38.1863043148041\\
0.423828125	38.0968534675299\\
0.42431640625	38.0077030701697\\
0.4248046875	37.918851581642\\
0.42529296875	37.8302974685372\\
0.42578125	37.7420392051881\\
0.42626953125	37.654075273735\\
0.4267578125	37.5664041641864\\
0.42724609375	37.479024374473\\
0.427734375	37.3919344104993\\
0.42822265625	37.3051327861886\\
0.4287109375	37.2186180235249\\
0.42919921875	37.1323886525903\\
0.4296875	37.0464432115976\\
0.43017578125	36.9607802469207\\
0.4306640625	36.8753983131196\\
0.43115234375	36.790295972963\\
0.431640625	36.7054717974469\\
0.43212890625	36.6209243658103\\
0.4326171875	36.5366522655478\\
0.43310546875	36.4526540924187\\
0.43359375	36.3689284504541\\
0.43408203125	36.2854739519605\\
0.4345703125	36.2022892175214\\
0.43505859375	36.1193728759958\\
0.435546875	36.0367235645149\\
0.43603515625	35.9543399284759\\
0.4365234375	35.8722206215346\\
0.43701171875	35.7903643055944\\
0.4375	35.7087696507953\\
0.43798828125	35.6274353354989\\
0.4384765625	35.5463600462731\\
0.43896484375	35.4655424778745\\
0.439453125	35.3849813332294\\
0.43994140625	35.3046753234127\\
0.4404296875	35.2246231676258\\
0.44091796875	35.1448235931733\\
0.44140625	35.0652753354374\\
0.44189453125	34.9859771378521\\
0.4423828125	34.9069277518755\\
0.44287109375	34.8281259369612\\
0.443359375	34.7495704605282\\
0.44384765625	34.6712600979308\\
0.4443359375	34.5931936324262\\
0.44482421875	34.5153698551419\\
0.4453125	34.4377875650422\\
0.44580078125	34.3604455688938\\
0.4462890625	34.2833426812305\\
0.44677734375	34.206477724317\\
0.447265625	34.1298495281128\\
0.44775390625	34.053456930234\\
0.4482421875	33.9772987759165\\
0.44873046875	33.9013739179763\\
0.44921875	33.8256812167712\\
0.44970703125	33.7502195401611\\
0.4501953125	33.6749877634678\\
0.45068359375	33.5999847694344\\
0.451171875	33.5252094481851\\
0.45166015625	33.450660697183\\
0.4521484375	33.376337421189\\
0.45263671875	33.3022385322202\\
0.453125	33.228362949507\\
0.45361328125	33.1547095994508\\
0.4541015625	33.0812774155819\\
0.45458984375	33.0080653385156\\
0.455078125	32.9350723159098\\
0.45556640625	32.8622973024213\\
0.4560546875	32.7897392596627\\
0.45654296875	32.7173971561584\\
0.45703125	32.6452699673011\\
0.45751953125	32.573356675308\\
0.4580078125	32.5016562691771\\
0.45849609375	32.4301677446431\\
0.458984375	32.358890104133\\
0.45947265625	32.2878223567231\\
0.4599609375	32.2169635180943\\
0.46044921875	32.146312610488\\
0.4609375	32.0758686626624\\
0.46142578125	32.0056307098488\\
0.4619140625	31.935597793707\\
0.46240234375	31.8657689622818\\
0.462890625	31.7961432699595\\
0.46337890625	31.7267197774236\\
0.4638671875	31.6574975516115\\
0.46435546875	31.588475665671\\
0.46484375	31.5196531989165\\
0.46533203125	31.4510292367861\\
0.4658203125	31.3826028707983\\
0.46630859375	31.3143731985087\\
0.466796875	31.2463393234672\\
0.46728515625	31.1785003551755\\
0.4677734375	31.1108554090439\\
0.46826171875	31.0434036063493\\
0.46875	30.9761440741927\\
0.46923828125	30.9090759454571\\
0.4697265625	30.8421983587657\\
0.47021484375	30.7755104584396\\
0.470703125	30.709011394457\\
0.47119140625	30.6427003224113\\
0.4716796875	30.5765764034697\\
0.47216796875	30.5106388043331\\
0.47265625	30.4448866971941\\
0.47314453125	30.3793192596976\\
0.4736328125	30.3139356748996\\
0.47412109375	30.2487351312274\\
0.474609375	30.1837168224397\\
0.47509765625	30.1188799475869\\
0.4755859375	30.0542237109714\\
0.47607421875	29.9897473221085\\
0.4765625	29.9254499956876\\
0.47705078125	29.861330951533\\
0.4775390625	29.7973894145657\\
0.47802734375	29.7336246147648\\
0.478515625	29.6700357871296\\
0.47900390625	29.6066221716419\\
0.4794921875	29.5433830132279\\
0.47998046875	29.4803175617216\\
0.48046875	29.4174250718268\\
0.48095703125	29.354704803081\\
0.4814453125	29.2921560198184\\
0.48193359375	29.2297779911336\\
0.482421875	29.1675699908455\\
0.48291015625	29.1055312974613\\
0.4833984375	29.0436611941409\\
0.48388671875	28.9819589686618\\
0.484375	28.9204239133836\\
0.48486328125	28.8590553252131\\
0.4853515625	28.7978525055699\\
0.48583984375	28.7368147603519\\
0.486328125	28.6759413999012\\
0.48681640625	28.6152317389701\\
0.4873046875	28.5546850966873\\
0.48779296875	28.494300796525\\
0.48828125	28.4340781662654\\
0.48876953125	28.3740165379675\\
0.4892578125	28.3141152479351\\
0.48974609375	28.2543736366839\\
0.490234375	28.1947910489094\\
0.49072265625	28.1353668334552\\
0.4912109375	28.076100343281\\
0.49169921875	28.0169909354312\\
0.4921875	27.9580379710042\\
0.49267578125	27.8992408151206\\
0.4931640625	27.840598836893\\
0.49365234375	27.7821114093956\\
0.494140625	27.7237779096337\\
0.49462890625	27.6655977185136\\
0.4951171875	27.6075702208132\\
0.49560546875	27.5496948051522\\
0.49609375	27.4919708639629\\
0.49658203125	27.4343977934609\\
0.4970703125	27.3769749936166\\
0.49755859375	27.3197018681265\\
0.498046875	27.2625778243845\\
0.49853515625	27.2056022734543\\
0.4990234375	27.1487746300409\\
0.49951171875	27.0920943124633\\
0.5	27.0355607426269\\
0.50048828125	26.9791733459963\\
0.5009765625	26.922931551568\\
0.50146484375	26.866834791844\\
0.501953125	26.8108825028048\\
0.50244140625	26.7550741238835\\
0.5029296875	26.699409097939\\
0.50341796875	26.6438868712308\\
0.50390625	26.5885068933926\\
0.50439453125	26.5332686174071\\
0.5048828125	26.4781714995807\\
0.50537109375	26.4232149995184\\
0.505859375	26.3683985800987\\
0.50634765625	26.3137217074491\\
0.5068359375	26.2591838509217\\
0.50732421875	26.2047844830687\\
0.5078125	26.1505230796182\\
0.50830078125	26.096399119451\\
0.5087890625	26.042412084576\\
0.50927734375	25.9885614601074\\
0.509765625	25.9348467342411\\
0.51025390625	25.8812673982318\\
0.5107421875	25.8278229463699\\
0.51123046875	25.7745128759587\\
0.51171875	25.7213366872924\\
0.51220703125	25.6682938836332\\
0.5126953125	25.6153839711895\\
0.51318359375	25.5626064590936\\
0.513671875	25.5099608593805\\
0.51416015625	25.4574466869657\\
0.5146484375	25.4050634596241\\
0.51513671875	25.3528106979688\\
0.515625	25.30068792543\\
0.51611328125	25.2486946682339\\
0.5166015625	25.1968304553824\\
0.51708984375	25.1450948186323\\
0.517578125	25.0934872924751\\
0.51806640625	25.0420074141166\\
0.5185546875	24.9906547234571\\
0.51904296875	24.9394287630717\\
0.51953125	24.88832907819\\
0.52001953125	24.8373552166775\\
0.5205078125	24.7865067290154\\
0.52099609375	24.735783168282\\
0.521484375	24.6851840901333\\
0.52197265625	24.6347090527845\\
0.5224609375	24.584357616991\\
0.52294921875	24.53412934603\\
0.5234375	24.4840238056821\\
0.52392578125	24.434040564213\\
0.5244140625	24.3841791923557\\
0.52490234375	24.334439263292\\
0.525390625	24.2848203526353\\
0.52587890625	24.2353220384128\\
0.5263671875	24.1859439010479\\
0.52685546875	24.1366855233429\\
0.52734375	24.087546490462\\
0.52783203125	24.038526389914\\
0.5283203125	23.9896248115357\\
0.52880859375	23.940841347475\\
0.529296875	23.8921755921739\\
0.52978515625	23.843627142353\\
0.5302734375	23.795195596994\\
0.53076171875	23.7468805573245\\
0.53125	23.6986816268013\\
0.53173828125	23.6505984110948\\
0.5322265625	23.6026305180729\\
0.53271484375	23.5547775577859\\
0.533203125	23.5070391424503\\
0.53369140625	23.4594148864337\\
0.5341796875	23.4119044062399\\
0.53466796875	23.3645073204929\\
0.53515625	23.3172232499229\\
0.53564453125	23.2700518173505\\
0.5361328125	23.2229926476726\\
0.53662109375	23.1760453678473\\
0.537109375	23.1292096068798\\
0.53759765625	23.0824849958075\\
0.5380859375	23.0358711676862\\
0.53857421875	22.9893677575756\\
0.5390625	22.9429744025257\\
0.53955078125	22.8966907415623\\
0.5400390625	22.8505164156735\\
0.54052734375	22.804451067796\\
0.541015625	22.7584943428015\\
0.54150390625	22.7126458874833\\
0.5419921875	22.6669053505428\\
0.54248046875	22.6212723825764\\
0.54296875	22.5757466360621\\
0.54345703125	22.5303277653468\\
0.5439453125	22.4850154266334\\
0.54443359375	22.4398092779677\\
0.544921875	22.3947089792256\\
0.54541015625	22.3497141921011\\
0.5458984375	22.3048245800933\\
0.54638671875	22.260039808494\\
0.546875	22.2153595443757\\
0.54736328125	22.1707834565793\\
0.5478515625	22.1263112157017\\
0.54833984375	22.0819424940845\\
0.548828125	22.0376769658011\\
0.54931640625	21.993514306646\\
0.5498046875	21.9494541941223\\
0.55029296875	21.9054963074305\\
0.55078125	21.8616403274568\\
0.55126953125	21.8178859367618\\
0.5517578125	21.774232819569\\
0.55224609375	21.7306806617538\\
0.552734375	21.6872291508324\\
0.55322265625	21.6438779759504\\
0.5537109375	21.6006268278721\\
0.55419921875	21.5574753989696\\
0.5546875	21.5144233832122\\
0.55517578125	21.4714704761551\\
0.5556640625	21.4286163749297\\
0.55615234375	21.3858607782321\\
0.556640625	21.3432033863136\\
0.55712890625	21.3006439009696\\
0.5576171875	21.2581820255298\\
0.55810546875	21.2158174648477\\
0.55859375	21.1735499252908\\
0.55908203125	21.1313791147303\\
0.5595703125	21.0893047425314\\
0.56005859375	21.0473265195431\\
0.560546875	21.0054441580891\\
0.56103515625	20.9636573719571\\
0.5615234375	20.92196587639\\
0.56201171875	20.880369388076\\
0.5625	20.8388676251393\\
0.56298828125	20.7974603071304\\
0.5634765625	20.7561471550171\\
0.56396484375	20.7149278911751\\
0.564453125	20.6738022393789\\
0.56494140625	20.6327699247924\\
0.5654296875	20.5918306739605\\
0.56591796875	20.5509842147995\\
0.56640625	20.5102302765884\\
0.56689453125	20.4695685899605\\
0.5673828125	20.428998886894\\
0.56787109375	20.3885209007038\\
0.568359375	20.3481343660326\\
0.56884765625	20.3078390188426\\
0.5693359375	20.2676345964068\\
0.56982421875	20.2275208373008\\
0.5703125	20.1874974813943\\
0.57080078125	20.147564269843\\
0.5712890625	20.1077209450801\\
0.57177734375	20.0679672508085\\
0.572265625	20.0283029319923\\
0.57275390625	19.9887277348494\\
0.5732421875	19.9492414068428\\
0.57373046875	19.9098436966733\\
0.57421875	19.8705343542714\\
0.57470703125	19.8313131307895\\
0.5751953125	19.7921797785943\\
0.57568359375	19.7531340512591\\
0.576171875	19.7141757035563\\
0.57666015625	19.6753044914497\\
0.5771484375	19.6365201720869\\
0.57763671875	19.5978225037925\\
0.578125	19.5592112460601\\
0.57861328125	19.5206861595451\\
0.5791015625	19.4822470060577\\
0.57958984375	19.4438935485557\\
0.580078125	19.4056255511371\\
0.58056640625	19.367442779033\\
0.5810546875	19.3293449986011\\
0.58154296875	19.2913319773182\\
0.58203125	19.2534034837732\\
0.58251953125	19.2155592876608\\
0.5830078125	19.1777991597743\\
0.58349609375	19.1401228719987\\
0.583984375	19.1025301973044\\
0.58447265625	19.0650209097401\\
0.5849609375	19.0275947844267\\
0.58544921875	18.9902515975502\\
0.5859375	18.9529911263555\\
0.58642578125	18.91581314914\\
0.5869140625	18.8787174452468\\
0.58740234375	18.841703795059\\
0.587890625	18.8047719799925\\
0.58837890625	18.7679217824907\\
0.5888671875	18.7311529860174\\
0.58935546875	18.6944653750513\\
0.58984375	18.6578587350794\\
0.59033203125	18.6213328525913\\
0.5908203125	18.5848875150729\\
0.59130859375	18.5485225110003\\
0.591796875	18.5122376298344\\
0.59228515625	18.4760326620143\\
0.5927734375	18.439907398952\\
0.59326171875	18.4038616330263\\
0.59375	18.367895157577\\
0.59423828125	18.3320077668995\\
0.5947265625	18.2961992562386\\
0.59521484375	18.2604694217833\\
0.595703125	18.224818060661\\
0.59619140625	18.1892449709321\\
0.5966796875	18.153749951584\\
0.59716796875	18.1183328025265\\
0.59765625	18.0829933245853\\
0.59814453125	18.0477313194975\\
0.5986328125	18.0125465899057\\
0.59912109375	17.9774389393531\\
0.599609375	17.9424081722778\\
0.60009765625	17.9074540940076\\
0.6005859375	17.8725765107552\\
0.60107421875	17.8377752296127\\
0.6015625	17.8030500585466\\
0.60205078125	17.7684008063924\\
0.6025390625	17.7338272828501\\
0.60302734375	17.6993292984788\\
0.603515625	17.6649066646917\\
0.60400390625	17.6305591937514\\
0.6044921875	17.5962866987649\\
0.60498046875	17.5620889936784\\
0.60546875	17.527965893273\\
0.60595703125	17.4939172131597\\
0.6064453125	17.4599427697743\\
0.60693359375	17.4260423803733\\
0.607421875	17.3922158630285\\
0.60791015625	17.3584630366229\\
0.6083984375	17.3247837208458\\
0.60888671875	17.2911777361882\\
0.609375	17.2576449039384\\
0.60986328125	17.2241850461772\\
0.6103515625	17.1907979857737\\
0.61083984375	17.1574835463806\\
0.611328125	17.1242415524299\\
0.61181640625	17.0910718291283\\
0.6123046875	17.0579742024532\\
0.61279296875	17.024948499148\\
0.61328125	16.9919945467179\\
0.61376953125	16.9591121734256\\
0.6142578125	16.9263012082872\\
0.61474609375	16.8935614810679\\
0.615234375	16.8608928222775\\
0.61572265625	16.8282950631668\\
0.6162109375	16.7957680357231\\
0.61669921875	16.763311572666\\
0.6171875	16.7309255074438\\
0.61767578125	16.698609674229\\
0.6181640625	16.6663639079143\\
0.61865234375	16.634188044109\\
0.619140625	16.6020819191346\\
0.61962890625	16.5700453700213\\
0.6201171875	16.5380782345035\\
0.62060546875	16.5061803510166\\
0.62109375	16.4743515586926\\
0.62158203125	16.4425916973565\\
0.6220703125	16.4109006075227\\
0.62255859375	16.3792781303908\\
0.623046875	16.3477241078423\\
0.62353515625	16.3162383824363\\
0.6240234375	16.2848207974065\\
0.62451171875	16.2534711966571\\
0.625	16.2221894247592\\
0.62548828125	16.1909753269473\\
0.6259765625	16.1598287491155\\
0.62646484375	16.1287495378143\\
0.626953125	16.0977375402466\\
0.62744140625	16.0667926042645\\
0.6279296875	16.0359145783658\\
0.62841796875	16.0051033116903\\
0.62890625	15.9743586540165\\
0.62939453125	15.9436804557583\\
0.6298828125	15.9130685679613\\
0.63037109375	15.8825228422997\\
0.630859375	15.8520431310727\\
0.63134765625	15.8216292872014\\
0.6318359375	15.7912811642254\\
0.63232421875	15.7609986162995\\
0.6328125	15.7307814981904\\
0.63330078125	15.7006296652735\\
0.6337890625	15.6705429735296\\
0.63427734375	15.6405212795419\\
0.634765625	15.6105644404927\\
0.63525390625	15.58067231416\\
0.6357421875	15.5508447589149\\
0.63623046875	15.5210816337181\\
0.63671875	15.4913827981167\\
0.63720703125	15.4617481122416\\
0.6376953125	15.4321774368041\\
0.63818359375	15.4026706330928\\
0.638671875	15.3732275629708\\
0.63916015625	15.3438480888726\\
0.6396484375	15.3145320738014\\
0.64013671875	15.2852793813256\\
0.640625	15.2560898755763\\
0.64111328125	15.2269634212444\\
0.6416015625	15.1978998835773\\
0.64208984375	15.1688991283765\\
0.642578125	15.1399610219945\\
0.64306640625	15.111085431332\\
0.6435546875	15.0822722238352\\
0.64404296875	15.0535212674926\\
0.64453125	15.0248324308329\\
0.64501953125	14.9962055829215\\
0.6455078125	14.9676405933583\\
0.64599609375	14.9391373322748\\
0.646484375	14.9106956703312\\
0.64697265625	14.8823154787141\\
0.6474609375	14.8539966291334\\
0.64794921875	14.8257389938201\\
0.6484375	14.7975424455231\\
0.64892578125	14.7694068575072\\
0.6494140625	14.7413321035499\\
0.64990234375	14.7133180579394\\
0.650390625	14.6853645954716\\
0.65087890625	14.6574715914475\\
0.6513671875	14.6296389216712\\
0.65185546875	14.6018664624467\\
0.65234375	14.5741540905759\\
0.65283203125	14.546501683356\\
0.6533203125	14.5189091185766\\
0.65380859375	14.491376274518\\
0.654296875	14.4639030299482\\
0.65478515625	14.4364892641206\\
0.6552734375	14.4091348567717\\
0.65576171875	14.3818396881187\\
0.65625	14.3546036388567\\
0.65673828125	14.3274265901573\\
0.6572265625	14.300308423665\\
0.65771484375	14.273249021496\\
0.658203125	14.2462482662352\\
0.65869140625	14.2193060409341\\
0.6591796875	14.1924222291085\\
0.65966796875	14.1655967147362\\
0.66015625	14.1388293822549\\
0.66064453125	14.1121201165597\\
0.6611328125	14.085468803001\\
0.66162109375	14.0588753273824\\
0.662109375	14.0323395759582\\
0.66259765625	14.0058614354315\\
0.6630859375	13.979440792952\\
0.66357421875	13.9530775361135\\
0.6640625	13.9267715529521\\
0.66455078125	13.9005227319442\\
0.6650390625	13.8743309620038\\
0.66552734375	13.848196132481\\
0.666015625	13.8221181331595\\
0.66650390625	13.7960968542546\\
0.6669921875	13.7701321864115\\
0.66748046875	13.7442240207026\\
0.66796875	13.718372248626\\
0.66845703125	13.6925767621033\\
0.6689453125	13.6668374534773\\
0.66943359375	13.6411542155104\\
0.669921875	13.6155269413827\\
0.67041015625	13.5899555246894\\
0.6708984375	13.5644398594393\\
0.67138671875	13.5389798400529\\
0.671875	13.5135753613602\\
0.67236328125	13.4882263185989\\
0.6728515625	13.4629326074124\\
0.67333984375	13.4376941238481\\
0.673828125	13.4125107643553\\
0.67431640625	13.3873824257833\\
0.6748046875	13.3623090053796\\
0.67529296875	13.3372904007881\\
0.67578125	13.3123265100473\\
0.67626953125	13.2874172315881\\
0.6767578125	13.2625624642324\\
0.67724609375	13.2377621071911\\
0.677734375	13.2130160600623\\
0.67822265625	13.1883242228295\\
0.6787109375	13.1636864958598\\
0.67919921875	13.1391027799021\\
0.6796875	13.1145729760857\\
0.68017578125	13.0900969859178\\
0.6806640625	13.0656747112826\\
0.68115234375	13.041306054439\\
0.681640625	13.0169909180191\\
0.68212890625	12.9927292050263\\
0.6826171875	12.9685208188341\\
0.68310546875	12.9443656631837\\
0.68359375	12.920263642183\\
0.68408203125	12.8962146603043\\
0.6845703125	12.8722186223833\\
0.68505859375	12.848275433617\\
0.685546875	12.8243849995619\\
0.68603515625	12.8005472261332\\
0.6865234375	12.7767620196021\\
0.68701171875	12.7530292865952\\
0.6875	12.729348934092\\
0.68798828125	12.705720869424\\
0.6884765625	12.6821450002729\\
0.68896484375	12.6586212346688\\
0.689453125	12.6351494809889\\
0.68994140625	12.611729647956\\
0.6904296875	12.5883616446368\\
0.69091796875	12.5650453804402\\
0.69140625	12.5417807651163\\
0.69189453125	12.5185677087543\\
0.6923828125	12.4954061217815\\
0.69287109375	12.4722959149612\\
0.693359375	12.4492369993921\\
0.69384765625	12.4262292865058\\
0.6943359375	12.4032726880662\\
0.69482421875	12.3803671161674\\
0.6953125	12.3575124832327\\
0.69580078125	12.3347087020128\\
0.6962890625	12.3119556855847\\
0.69677734375	12.2892533473501\\
0.697265625	12.2666016010338\\
0.69775390625	12.2440003606827\\
0.6982421875	12.2214495406639\\
0.69873046875	12.1989490556639\\
0.69921875	12.1764988206867\\
0.69970703125	12.1540987510527\\
0.7001953125	12.1317487623971\\
0.70068359375	12.1094487706688\\
0.701171875	12.087198692129\\
0.70166015625	12.0649984433495\\
0.7021484375	12.0428479412121\\
0.70263671875	12.0207471029063\\
0.703125	11.9986958459288\\
0.70361328125	11.9766940880819\\
0.7041015625	11.954741747472\\
0.70458984375	11.9328387425086\\
0.705078125	11.9109849919027\\
0.70556640625	11.8891804146658\\
0.7060546875	11.8674249301087\\
0.70654296875	11.8457184578396\\
0.70703125	11.8240609177637\\
0.70751953125	11.8024522300813\\
0.7080078125	11.7808923152869\\
0.70849609375	11.7593810941676\\
0.708984375	11.7379184878024\\
0.70947265625	11.7165044175604\\
0.7099609375	11.6951388051\\
0.71044921875	11.6738215723676\\
0.7109375	11.6525526415962\\
0.71142578125	11.6313319353044\\
0.7119140625	11.610159376295\\
0.71240234375	11.5890348876541\\
0.712890625	11.5679583927498\\
0.71337890625	11.546929815231\\
0.7138671875	11.525949079026\\
0.71435546875	11.5050161083418\\
0.71484375	11.4841308276629\\
0.71533203125	11.4632931617494\\
0.7158203125	11.4425030356371\\
0.71630859375	11.4217603746353\\
0.716796875	11.4010651043261\\
0.71728515625	11.3804171505635\\
0.7177734375	11.3598164394718\\
0.71826171875	11.3392628974449\\
0.71875	11.318756451145\\
0.71923828125	11.2982970275014\\
0.7197265625	11.2778845537097\\
0.72021484375	11.2575189572306\\
0.720703125	11.2372001657887\\
0.72119140625	11.2169281073716\\
0.7216796875	11.1967027102288\\
0.72216796875	11.1765239028703\\
0.72265625	11.1563916140663\\
0.72314453125	11.1363057728453\\
0.7236328125	11.1162663084938\\
0.72412109375	11.0962731505546\\
0.724609375	11.0763262288264\\
0.72509765625	11.0564254733623\\
0.7255859375	11.0365708144689\\
0.72607421875	11.0167621827055\\
0.7265625	10.9969995088828\\
0.72705078125	10.9772827240622\\
0.7275390625	10.9576117595545\\
0.72802734375	10.9379865469191\\
0.728515625	10.9184070179629\\
0.72900390625	10.8988731047394\\
0.7294921875	10.8793847395479\\
0.72998046875	10.8599418549321\\
0.73046875	10.8405443836794\\
0.73095703125	10.8211922588202\\
0.7314453125	10.8018854136263\\
0.73193359375	10.7826237816105\\
0.732421875	10.7634072965256\\
0.73291015625	10.7442358923631\\
0.7333984375	10.7251095033527\\
0.73388671875	10.7060280639611\\
0.734375	10.6869915088913\\
0.73486328125	10.6679997730812\\
0.7353515625	10.6490527917035\\
0.73583984375	10.6301505001641\\
0.736328125	10.6112928341013\\
0.73681640625	10.5924797293853\\
0.7373046875	10.5737111221169\\
0.73779296875	10.5549869486268\\
0.73828125	10.5363071454747\\
0.73876953125	10.5176716494483\\
0.7392578125	10.4990803975625\\
0.73974609375	10.4805333270588\\
0.740234375	10.462030375404\\
0.74072265625	10.4435714802896\\
0.7412109375	10.4251565796308\\
0.74169921875	10.4067856115659\\
0.7421875	10.3884585144551\\
0.74267578125	10.3701752268801\\
0.7431640625	10.3519356876427\\
0.74365234375	10.3337398357645\\
0.744140625	10.315587610486\\
0.74462890625	10.2974789512652\\
0.7451171875	10.2794137977777\\
0.74560546875	10.2613920899151\\
0.74609375	10.2434137677846\\
0.74658203125	10.2254787717081\\
0.7470703125	10.2075870422214\\
0.74755859375	10.1897385200735\\
0.748046875	10.1719331462255\\
0.74853515625	10.1541708618502\\
0.7490234375	10.1364516083311\\
0.74951171875	10.1187753272617\\
0.75	10.1011419604445\\
0.75048828125	10.0835514498907\\
0.7509765625	10.066003737819\\
0.75146484375	10.048498766655\\
0.751953125	10.0310364790305\\
0.75244140625	10.0136168177826\\
0.7529296875	9.9962397259531\\
0.75341796875	9.97890514678765\\
0.75390625	9.96161302373506\\
0.75439453125	9.94436330044656\\
0.7548828125	9.92715592077504\\
0.75537109375	9.90999082877433\\
0.755859375	9.89286796869852\\
0.75634765625	9.87578728500113\\
0.7568359375	9.85874872233451\\
0.75732421875	9.84175222554907\\
0.7578125	9.82479773969254\\
0.75830078125	9.80788521000936\\
0.7587890625	9.79101458193985\\
0.75927734375	9.77418580111961\\
0.759765625	9.75739881337877\\
0.76025390625	9.74065356474133\\
0.7607421875	9.72395000142446\\
0.76123046875	9.70728806983782\\
0.76171875	9.69066771658285\\
0.76220703125	9.67408888845213\\
0.7626953125	9.6575515324287\\
0.76318359375	9.64105559568537\\
0.763671875	9.62460102558405\\
0.76416015625	9.60818776967515\\
0.7646484375	9.59181577569682\\
0.76513671875	9.57548499157436\\
0.765625	9.55919536541957\\
0.76611328125	9.54294684553006\\
0.7666015625	9.52673938038865\\
0.76708984375	9.51057291866268\\
0.767578125	9.49444740920341\\
0.76806640625	9.47836280104537\\
0.7685546875	9.46231904340572\\
0.76904296875	9.44631608568361\\
0.76953125	9.43035387745961\\
0.77001953125	9.41443236849503\\
0.7705078125	9.3985515087313\\
0.77099609375	9.3827112482894\\
0.771484375	9.36691153746922\\
0.77197265625	9.35115232674892\\
0.7724609375	9.33543356678438\\
0.77294921875	9.31975520840858\\
0.7734375	9.30411720263093\\
0.77392578125	9.28851950063684\\
0.7744140625	9.27296205378689\\
0.77490234375	9.25744481361647\\
0.775390625	9.24196773183503\\
0.77587890625	9.22653076032559\\
0.7763671875	9.21113385114409\\
0.77685546875	9.19577695651889\\
0.77734375	9.1804600288501\\
0.77783203125	9.16518302070908\\
0.7783203125	9.14994588483786\\
0.77880859375	9.13474857414852\\
0.779296875	9.1195910417227\\
0.77978515625	9.10447324081098\\
0.7802734375	9.08939512483238\\
0.78076171875	9.07435664737372\\
0.78125	9.05935776218917\\
0.78173828125	9.04439842319961\\
0.7822265625	9.02947858449215\\
0.78271484375	9.01459820031956\\
0.783203125	8.99975722509969\\
0.78369140625	8.98495561341501\\
0.7841796875	8.97019332001203\\
0.78466796875	8.95547029980074\\
0.78515625	8.94078650785415\\
0.78564453125	8.92614189940771\\
0.7861328125	8.91153642985878\\
0.78662109375	8.89697005476618\\
0.787109375	8.88244272984957\\
0.78759765625	8.86795441098899\\
0.7880859375	8.85350505422436\\
0.78857421875	8.83909461575496\\
0.7890625	8.82472305193886\\
0.78955078125	8.81039031929249\\
0.7900390625	8.79609637449014\\
0.79052734375	8.78184117436338\\
0.791015625	8.76762467590066\\
0.79150390625	8.75344683624673\\
0.7919921875	8.7393076127022\\
0.79248046875	8.72520696272304\\
0.79296875	8.71114484392006\\
0.79345703125	8.69712121405847\\
0.7939453125	8.6831360310574\\
0.79443359375	8.66918925298935\\
0.794921875	8.65528083807978\\
0.79541015625	8.64141074470663\\
0.7958984375	8.6275789313998\\
0.79638671875	8.61378535684073\\
0.796875	8.6000299798619\\
0.79736328125	8.5863127594464\\
0.7978515625	8.5726336547274\\
0.79833984375	8.55899262498778\\
0.798828125	8.54538962965958\\
0.79931640625	8.53182462832362\\
0.7998046875	8.51829758070898\\
0.80029296875	8.5048084466926\\
0.80078125	8.49135718629883\\
0.80126953125	8.47794375969891\\
0.8017578125	8.46456812721064\\
0.80224609375	8.45123024929781\\
0.802734375	8.43793008656989\\
0.80322265625	8.4246675997815\\
0.8037109375	8.41144274983201\\
0.80419921875	8.39825549776507\\
0.8046875	8.38510580476824\\
0.80517578125	8.37199363217254\\
0.8056640625	8.35891894145196\\
0.80615234375	8.34588169422314\\
0.806640625	8.33288185224484\\
0.80712890625	8.31991937741764\\
0.8076171875	8.3069942317834\\
0.80810546875	8.29410637752491\\
0.80859375	8.28125577696549\\
0.80908203125	8.26844239256854\\
0.8095703125	8.25566618693713\\
0.81005859375	8.2429271228136\\
0.810546875	8.23022516307918\\
0.81103515625	8.2175602707536\\
0.8115234375	8.20493240899456\\
0.81201171875	8.19234154109751\\
0.8125	8.17978763049514\\
0.81298828125	8.167270640757\\
0.8134765625	8.15479053558913\\
0.81396484375	8.14234727883366\\
0.814453125	8.12994083446841\\
0.81494140625	8.11757116660653\\
0.8154296875	8.10523823949606\\
0.81591796875	8.09294201751962\\
0.81640625	8.08068246519396\\
0.81689453125	8.06845954716963\\
0.8173828125	8.05627322823056\\
0.81787109375	8.04412347329376\\
0.818359375	8.03201024740882\\
0.81884765625	8.01993351575766\\
0.8193359375	8.00789324365409\\
0.81982421875	7.9958893965435\\
0.8203125	7.9839219400024\\
0.82080078125	7.97199083973817\\
0.8212890625	7.96009606158858\\
0.82177734375	7.94823757152157\\
0.822265625	7.93641533563471\\
0.82275390625	7.92462932015504\\
0.8232421875	7.91287949143855\\
0.82373046875	7.90116581596994\\
0.82421875	7.88948826036219\\
0.82470703125	7.87784679135628\\
0.8251953125	7.86624137582077\\
0.82568359375	7.85467198075151\\
0.826171875	7.84313857327128\\
0.82666015625	7.83164112062942\\
0.8271484375	7.82017959020154\\
0.82763671875	7.80875394948913\\
0.828125	7.79736416611928\\
0.82861328125	7.78601020784425\\
0.8291015625	7.77469204254125\\
0.82958984375	7.76340963821202\\
0.830078125	7.75216296298258\\
0.83056640625	7.7409519851028\\
0.8310546875	7.72977667294614\\
0.83154296875	7.71863699500935\\
0.83203125	7.70753291991205\\
0.83251953125	7.6964644163965\\
0.8330078125	7.68543145332727\\
0.83349609375	7.67443399969083\\
0.833984375	7.66347202459534\\
0.83447265625	7.6525454972703\\
0.8349609375	7.6416543870662\\
0.83544921875	7.63079866345428\\
0.8359375	7.61997829602612\\
0.83642578125	7.60919325449344\\
0.8369140625	7.59844350868772\\
0.83740234375	7.58772902855989\\
0.837890625	7.57704978418008\\
0.83837890625	7.56640574573727\\
0.8388671875	7.55579688353903\\
0.83935546875	7.54522316801116\\
0.83984375	7.53468456969744\\
0.84033203125	7.52418105925933\\
0.8408203125	7.51371260747567\\
0.84130859375	7.50327918524236\\
0.841796875	7.4928807635721\\
0.84228515625	7.48251731359409\\
0.8427734375	7.47218880655375\\
0.84326171875	7.46189521381238\\
0.84375	7.45163650684699\\
0.84423828125	7.44141265724988\\
0.8447265625	7.43122363672845\\
0.84521484375	7.42106941710486\\
0.845703125	7.41094997031581\\
0.84619140625	7.40086526841221\\
0.8466796875	7.39081528355891\\
0.84716796875	7.38079998803447\\
0.84765625	7.37081935423079\\
0.84814453125	7.36087335465298\\
0.8486328125	7.35096196191894\\
0.84912109375	7.34108514875919\\
0.849609375	7.33124288801654\\
0.85009765625	7.32143515264588\\
0.8505859375	7.31166191571388\\
0.85107421875	7.3019231503987\\
0.8515625	7.29221882998979\\
0.85205078125	7.28254892788757\\
0.8525390625	7.27291341760322\\
0.85302734375	7.26331227275839\\
0.853515625	7.25374546708492\\
0.85400390625	7.24421297442464\\
0.8544921875	7.23471476872908\\
0.85498046875	7.22525082405921\\
0.85546875	7.21582111458521\\
0.85595703125	7.20642561458624\\
0.8564453125	7.19706429845012\\
0.85693359375	7.18773714067312\\
0.857421875	7.17844411585972\\
0.85791015625	7.16918519872238\\
0.8583984375	7.15996036408128\\
0.85888671875	7.15076958686402\\
0.859375	7.1416128421055\\
0.85986328125	7.13249010494754\\
0.8603515625	7.12340135063875\\
0.86083984375	7.11434655453428\\
0.861328125	7.10532569209547\\
0.86181640625	7.09633873888978\\
0.8623046875	7.08738567059047\\
0.86279296875	7.07846646297629\\
0.86328125	7.06958109193145\\
0.86376953125	7.06072953344517\\
0.8642578125	7.05191176361162\\
0.86474609375	7.04312775862961\\
0.865234375	7.03437749480234\\
0.86572265625	7.02566094853724\\
0.8662109375	7.01697809634572\\
0.86669921875	7.00832891484296\\
0.8671875	6.99971338074762\\
0.86767578125	6.99113147088173\\
0.8681640625	6.98258316217038\\
0.86865234375	6.97406843164152\\
0.869140625	6.96558725642579\\
0.86962890625	6.95713961375626\\
0.8701171875	6.94872548096823\\
0.87060546875	6.94034483549899\\
0.87109375	6.93199765488767\\
0.87158203125	6.92368391677494\\
0.8720703125	6.91540359890289\\
0.87255859375	6.90715667911477\\
0.873046875	6.89894313535479\\
0.87353515625	6.89076294566789\\
0.8740234375	6.88261608819961\\
0.87451171875	6.87450254119578\\
0.875	6.86642228300243\\
0.87548828125	6.85837529206549\\
0.8759765625	6.85036154693063\\
0.87646484375	6.84238102624309\\
0.876953125	6.8344337087474\\
0.87744140625	6.82651957328731\\
0.8779296875	6.81863859880543\\
0.87841796875	6.81079076434317\\
0.87890625	6.8029760490405\\
0.87939453125	6.79519443213573\\
0.8798828125	6.78744589296537\\
0.88037109375	6.77973041096387\\
0.880859375	6.77204796566351\\
0.88134765625	6.76439853669413\\
0.8818359375	6.75678210378302\\
0.88232421875	6.74919864675468\\
0.8828125	6.74164814553065\\
0.88330078125	6.73413058012933\\
0.8837890625	6.7266459306658\\
0.88427734375	6.7191941773516\\
0.884765625	6.71177530049462\\
0.88525390625	6.70438928049883\\
0.8857421875	6.6970360978642\\
0.88623046875	6.68971573318641\\
0.88671875	6.68242816715677\\
0.88720703125	6.67517338056201\\
0.8876953125	6.6679513542841\\
0.88818359375	6.66076206930004\\
0.888671875	6.65360550668175\\
0.88916015625	6.64648164759588\\
0.8896484375	6.63939047330359\\
0.89013671875	6.63233196516047\\
0.890625	6.62530610461626\\
0.89111328125	6.61831287321481\\
0.8916015625	6.61135225259376\\
0.89208984375	6.60442422448454\\
0.892578125	6.59752877071206\\
0.89306640625	6.59066587319463\\
0.8935546875	6.58383551394379\\
0.89404296875	6.57703767506409\\
0.89453125	6.570272338753\\
0.89501953125	6.56353948730073\\
0.8955078125	6.55683910309\\
0.89599609375	6.55017116859602\\
0.896484375	6.54353566638619\\
0.89697265625	6.53693257912004\\
0.8974609375	6.53036188954902\\
0.89794921875	6.52382358051638\\
0.8984375	6.51731763495701\\
0.89892578125	6.51084403589724\\
0.8994140625	6.50440276645478\\
0.89990234375	6.49799380983846\\
0.900390625	6.4916171493482\\
0.90087890625	6.48527276837472\\
0.9013671875	6.47896065039953\\
0.90185546875	6.47268077899469\\
0.90234375	6.46643313782273\\
0.90283203125	6.4602177106364\\
0.9033203125	6.45403448127865\\
0.90380859375	6.44788343368242\\
0.904296875	6.44176455187051\\
0.90478515625	6.4356778199554\\
0.9052734375	6.42962322213923\\
0.90576171875	6.42360074271345\\
0.90625	6.4176103660589\\
0.90673828125	6.41165207664556\\
0.9072265625	6.4057258590324\\
0.90771484375	6.3998316978673\\
0.908203125	6.39396957788688\\
0.90869140625	6.38813948391635\\
0.9091796875	6.38234140086944\\
0.90966796875	6.3765753137482\\
0.91015625	6.37084120764289\\
0.91064453125	6.36513906773185\\
0.9111328125	6.3594688792814\\
0.91162109375	6.35383062764564\\
0.912109375	6.34822429826638\\
0.91259765625	6.34264987667302\\
0.9130859375	6.33710734848236\\
0.91357421875	6.33159669939853\\
0.9140625	6.32611791521284\\
0.91455078125	6.32067098180368\\
0.9150390625	6.31525588513634\\
0.91552734375	6.30987261126297\\
0.916015625	6.30452114632238\\
0.91650390625	6.29920147653995\\
0.9169921875	6.29391358822756\\
0.91748046875	6.28865746778337\\
0.91796875	6.28343310169176\\
0.91845703125	6.27824047652324\\
0.9189453125	6.27307957893426\\
0.91943359375	6.26795039566718\\
0.919921875	6.26285291355004\\
0.92041015625	6.25778711949655\\
0.9208984375	6.25275300050598\\
0.92138671875	6.24775054366291\\
0.921875	6.24277973613731\\
0.92236328125	6.23784056518427\\
0.9228515625	6.23293301814399\\
0.92333984375	6.22805708244161\\
0.923828125	6.22321274558713\\
0.92431640625	6.21839999517529\\
0.9248046875	6.21361881888548\\
0.92529296875	6.20886920448162\\
0.92578125	6.20415113981203\\
0.92626953125	6.1994646128094\\
0.9267578125	6.19480961149059\\
0.92724609375	6.19018612395661\\
0.927734375	6.18559413839244\\
0.92822265625	6.18103364306702\\
0.9287109375	6.17650462633304\\
0.92919921875	6.17200707662696\\
0.9296875	6.16754098246879\\
0.93017578125	6.1631063324621\\
0.9306640625	6.15870311529384\\
0.93115234375	6.15433131973427\\
0.931640625	6.14999093463689\\
0.93212890625	6.14568194893834\\
0.9326171875	6.14140435165821\\
0.93310546875	6.13715813189911\\
0.93359375	6.13294327884645\\
0.93408203125	6.12875978176838\\
0.9345703125	6.12460763001573\\
0.93505859375	6.12048681302188\\
0.935546875	6.1163973203027\\
0.93603515625	6.11233914145641\\
0.9365234375	6.10831226616355\\
0.93701171875	6.10431668418688\\
0.9375	6.10035238537125\\
0.93798828125	6.09641935964359\\
0.9384765625	6.0925175970127\\
0.93896484375	6.08864708756933\\
0.939453125	6.08480782148595\\
0.93994140625	6.08099978901675\\
0.9404296875	6.07722298049753\\
0.94091796875	6.0734773863456\\
0.94140625	6.06976299705975\\
0.94189453125	6.06607980322012\\
0.9423828125	6.06242779548812\\
0.94287109375	6.05880696460641\\
0.943359375	6.05521730139873\\
0.94384765625	6.05165879676993\\
0.9443359375	6.04813144170577\\
0.94482421875	6.04463522727296\\
0.9453125	6.04117014461901\\
0.94580078125	6.03773618497218\\
0.9462890625	6.03433333964141\\
0.94677734375	6.03096160001621\\
0.947265625	6.02762095756666\\
0.94775390625	6.02431140384328\\
0.9482421875	6.02103293047696\\
0.94873046875	6.01778552917891\\
0.94921875	6.01456919174059\\
0.94970703125	6.01138391003364\\
0.9501953125	6.00822967600978\\
0.95068359375	6.00510648170076\\
0.951171875	6.00201431921835\\
0.95166015625	5.9989531807542\\
0.9521484375	5.99592305857977\\
0.95263671875	5.99292394504631\\
0.953125	5.9899558325848\\
};
\addplot [color=mycolor2,solid]
  table[row sep=crcr]{0.953125	5.9899558325848\\
0.95361328125	5.98701871370583\\
0.9541015625	5.9841125809996\\
0.95458984375	5.98123742713581\\
0.955078125	5.97839324486362\\
0.95556640625	5.9755800270116\\
0.9560546875	5.97279776648763\\
0.95654296875	5.97004645627891\\
0.95703125	5.96732608945181\\
0.95751953125	5.96463665915189\\
0.9580078125	5.96197815860381\\
0.95849609375	5.95935058111125\\
0.958984375	5.95675392005691\\
0.95947265625	5.95418816890241\\
0.9599609375	5.95165332118824\\
0.96044921875	5.94914937053376\\
0.9609375	5.94667631063702\\
0.96142578125	5.94423413527487\\
0.9619140625	5.94182283830277\\
0.96240234375	5.93944241365484\\
0.962890625	5.93709285534371\\
0.96337890625	5.93477415746058\\
0.9638671875	5.93248631417509\\
0.96435546875	5.93022931973529\\
0.96484375	5.92800316846761\\
0.96533203125	5.92580785477679\\
0.9658203125	5.92364337314586\\
0.96630859375	5.92150971813606\\
0.966796875	5.91940688438682\\
0.96728515625	5.91733486661571\\
0.9677734375	5.9152936596184\\
0.96826171875	5.9132832582686\\
0.96875	5.91130365751802\\
0.96923828125	5.90935485239636\\
0.9697265625	5.90743683801122\\
0.97021484375	5.90554960954812\\
0.970703125	5.90369316227036\\
0.97119140625	5.90186749151911\\
0.9716796875	5.90007259271326\\
0.97216796875	5.89830846134945\\
0.97265625	5.89657509300202\\
0.97314453125	5.89487248332292\\
0.9736328125	5.89320062804175\\
0.97412109375	5.89155952296569\\
0.974609375	5.88994916397946\\
0.97509765625	5.88836954704528\\
0.9755859375	5.88682066820287\\
0.97607421875	5.88530252356938\\
0.9765625	5.88381510933939\\
0.97705078125	5.88235842178485\\
0.9775390625	5.88093245725504\\
0.97802734375	5.8795372121766\\
0.978515625	5.87817268305345\\
0.97900390625	5.87683886646675\\
0.9794921875	5.8755357590749\\
0.97998046875	5.87426335761353\\
0.98046875	5.87302165889543\\
0.98095703125	5.87181065981052\\
0.9814453125	5.87063035732589\\
0.98193359375	5.86948074848572\\
0.982421875	5.86836183041124\\
0.98291015625	5.86727360030075\\
0.9833984375	5.86621605542959\\
0.98388671875	5.86518919315008\\
0.984375	5.86419301089156\\
0.98486328125	5.86322750616031\\
0.9853515625	5.86229267653955\\
0.98583984375	5.86138851968942\\
0.986328125	5.86051503334701\\
0.98681640625	5.85967221532622\\
0.9873046875	5.8588600635179\\
0.98779296875	5.85807857588966\\
0.98828125	5.85732775048601\\
0.98876953125	5.85660758542826\\
0.9892578125	5.8559180789145\\
0.98974609375	5.85525922921963\\
0.990234375	5.85463103469532\\
0.99072265625	5.85403349376996\\
0.9912109375	5.85346660494875\\
0.99169921875	5.85293036681356\\
0.9921875	5.85242477802302\\
0.99267578125	5.85194983731247\\
0.9931640625	5.85150554349392\\
0.99365234375	5.85109189545608\\
0.994140625	5.85070889216437\\
0.99462890625	5.85035653266084\\
0.9951171875	5.85003481606423\\
0.99560546875	5.84974374156992\\
0.99609375	5.84948330844995\\
0.99658203125	5.84925351605302\\
0.9970703125	5.84905436380443\\
0.99755859375	5.84888585120612\\
0.998046875	5.84874797783672\\
0.99853515625	5.84864074335139\\
0.9990234375	5.848564147482\\
0.99951171875	5.84851819003698\\
};
\addlegendentry{AR(8) Model};

\addplot [color=mycolor3,solid,forget plot]
  table[row sep=crcr]{-1	6.26903218010927\\
-0.99951171875	6.26905863069305\\
-0.9990234375	6.26913798228735\\
-0.99853515625	6.26927023442106\\
-0.998046875	6.26945538630899\\
-0.99755859375	6.26969343685183\\
-0.9970703125	6.26998438463616\\
-0.99658203125	6.2703282279344\\
-0.99609375	6.27072496470476\\
-0.99560546875	6.27117459259128\\
-0.9951171875	6.27167710892365\\
-0.99462890625	6.27223251071726\\
-0.994140625	6.2728407946731\\
-0.99365234375	6.2735019571777\\
-0.9931640625	6.27421599430304\\
-0.99267578125	6.27498290180647\\
-0.9921875	6.27580267513066\\
-0.99169921875	6.27667530940347\\
-0.9912109375	6.27760079943785\\
-0.99072265625	6.27857913973175\\
-0.990234375	6.27961032446799\\
-0.98974609375	6.28069434751417\\
-0.9892578125	6.28183120242251\\
-0.98876953125	6.28302088242972\\
-0.98828125	6.2842633804569\\
-0.98779296875	6.28555868910934\\
-0.9873046875	6.28690680067643\\
-0.98681640625	6.28830770713145\\
-0.986328125	6.28976140013144\\
-0.98583984375	6.29126787101701\\
-0.9853515625	6.29282711081215\\
-0.98486328125	6.2944391102241\\
-0.984375	6.29610385964309\\
-0.98388671875	6.29782134914225\\
-0.9833984375	6.29959156847725\\
-0.98291015625	6.30141450708625\\
-0.982421875	6.30329015408959\\
-0.98193359375	6.30521849828963\\
-0.9814453125	6.30719952817047\\
-0.98095703125	6.3092332318977\\
-0.98046875	6.31131959731832\\
-0.97998046875	6.31345861196025\\
-0.9794921875	6.3156502630323\\
-0.97900390625	6.31789453742374\\
-0.978515625	6.32019142170419\\
-0.97802734375	6.32254090212321\\
-0.9775390625	6.32494296461018\\
-0.97705078125	6.32739759477387\\
-0.9765625	6.32990477790222\\
-0.97607421875	6.33246449896204\\
-0.9755859375	6.33507674259875\\
-0.97509765625	6.33774149313602\\
-0.974609375	6.34045873457545\\
-0.97412109375	6.34322845059635\\
-0.9736328125	6.34605062455526\\
-0.97314453125	6.34892523948581\\
-0.97265625	6.35185227809823\\
-0.97216796875	6.35483172277907\\
-0.9716796875	6.3578635555909\\
-0.97119140625	6.36094775827185\\
-0.970703125	6.36408431223538\\
-0.97021484375	6.36727319856981\\
-0.9697265625	6.37051439803804\\
-0.96923828125	6.37380789107711\\
-0.96875	6.37715365779784\\
-0.96826171875	6.38055167798448\\
-0.9677734375	6.38400193109425\\
-0.96728515625	6.38750439625705\\
-0.966796875	6.39105905227497\\
-0.96630859375	6.3946658776219\\
-0.9658203125	6.39832485044315\\
-0.96533203125	6.40203594855501\\
-0.96484375	6.40579914944431\\
-0.96435546875	6.40961443026809\\
-0.9638671875	6.41348176785303\\
-0.96337890625	6.41740113869512\\
-0.962890625	6.42137251895913\\
-0.96240234375	6.42539588447828\\
-0.9619140625	6.42947121075364\\
-0.96142578125	6.43359847295383\\
-0.9609375	6.43777764591445\\
-0.96044921875	6.44200870413764\\
-0.9599609375	6.44629162179159\\
-0.95947265625	6.45062637271016\\
-0.958984375	6.45501293039225\\
-0.95849609375	6.45945126800145\\
-0.9580078125	6.46394135836546\\
-0.95751953125	6.46848317397562\\
-0.95703125	6.47307668698645\\
-0.95654296875	6.47772186921508\\
-0.9560546875	6.48241869214081\\
-0.95556640625	6.48716712690454\\
-0.955078125	6.49196714430829\\
-0.95458984375	6.49681871481468\\
-0.9541015625	6.50172180854633\\
-0.95361328125	6.50667639528546\\
-0.953125	6.51168244447329\\
-0.95263671875	6.51673992520942\\
-0.9521484375	6.5218488062515\\
-0.95166015625	6.5270090560144\\
-0.951171875	6.53222064256995\\
-0.95068359375	6.53748353364619\\
-0.9501953125	6.54279769662686\\
-0.94970703125	6.54816309855089\\
-0.94921875	6.5535797061118\\
-0.94873046875	6.5590474856571\\
-0.9482421875	6.56456640318781\\
-0.94775390625	6.57013642435774\\
-0.947265625	6.57575751447311\\
-0.94677734375	6.58142963849174\\
-0.9462890625	6.58715276102266\\
-0.94580078125	6.59292684632543\\
-0.9453125	6.59875185830955\\
-0.94482421875	6.60462776053392\\
-0.9443359375	6.61055451620614\\
-0.94384765625	6.61653208818201\\
-0.943359375	6.62256043896491\\
-0.94287109375	6.62863953070513\\
-0.9423828125	6.63476932519934\\
-0.94189453125	6.64094978388994\\
-0.94140625	6.64718086786444\\
-0.94091796875	6.6534625378549\\
-0.9404296875	6.6597947542372\\
-0.93994140625	6.66617747703057\\
-0.939453125	6.67261066589679\\
-0.93896484375	6.67909428013976\\
-0.9384765625	6.6856282787047\\
-0.93798828125	6.69221262017767\\
-0.9375	6.69884726278477\\
-0.93701171875	6.7055321643917\\
-0.9365234375	6.71226728250297\\
-0.93603515625	6.71905257426132\\
-0.935546875	6.72588799644714\\
-0.93505859375	6.73277350547769\\
-0.9345703125	6.73970905740662\\
-0.93408203125	6.74669460792323\\
-0.93359375	6.75373011235186\\
-0.93310546875	6.7608155256512\\
-0.9326171875	6.76795080241371\\
-0.93212890625	6.77513589686496\\
-0.931640625	6.78237076286296\\
-0.93115234375	6.78965535389751\\
-0.9306640625	6.79698962308957\\
-0.93017578125	6.80437352319058\\
-0.9296875	6.8118070065819\\
-0.92919921875	6.81929002527406\\
-0.9287109375	6.8268225309061\\
-0.92822265625	6.83440447474506\\
-0.927734375	6.84203580768517\\
-0.92724609375	6.84971648024728\\
-0.9267578125	6.8574464425782\\
-0.92626953125	6.86522564445007\\
-0.92578125	6.87305403525966\\
-0.92529296875	6.88093156402777\\
-0.9248046875	6.88885817939853\\
-0.92431640625	6.89683382963883\\
-0.923828125	6.90485846263759\\
-0.92333984375	6.91293202590519\\
-0.9228515625	6.92105446657276\\
-0.92236328125	6.92922573139158\\
-0.921875	6.93744576673239\\
-0.92138671875	6.94571451858485\\
-0.9208984375	6.95403193255677\\
-0.92041015625	6.96239795387358\\
-0.919921875	6.97081252737762\\
-0.91943359375	6.97927559752755\\
-0.9189453125	6.9877871083977\\
-0.91845703125	6.99634700367746\\
-0.91796875	7.00495522667065\\
-0.91748046875	7.01361172029482\\
-0.9169921875	7.02231642708075\\
-0.91650390625	7.03106928917175\\
-0.916015625	7.03987024832309\\
-0.91552734375	7.04871924590129\\
-0.9150390625	7.05761622288368\\
-0.91455078125	7.06656111985759\\
-0.9140625	7.07555387701992\\
-0.91357421875	7.08459443417642\\
-0.9130859375	7.09368273074116\\
-0.91259765625	7.10281870573589\\
-0.912109375	7.11200229778951\\
-0.91162109375	7.1212334451374\\
-0.9111328125	7.13051208562093\\
-0.91064453125	7.13983815668681\\
-0.91015625	7.14921159538656\\
-0.90966796875	7.15863233837589\\
-0.9091796875	7.1681003219142\\
-0.90869140625	7.17761548186403\\
-0.908203125	7.18717775369038\\
-0.90771484375	7.19678707246031\\
-0.9072265625	7.20644337284235\\
-0.90673828125	7.21614658910589\\
-0.90625	7.22589665512075\\
-0.90576171875	7.23569350435658\\
-0.9052734375	7.24553706988243\\
-0.90478515625	7.25542728436608\\
-0.904296875	7.26536408007372\\
-0.90380859375	7.2753473888693\\
-0.9033203125	7.28537714221409\\
-0.90283203125	7.2954532711662\\
-0.90234375	7.30557570638012\\
-0.90185546875	7.31574437810615\\
-0.9013671875	7.32595921619007\\
-0.90087890625	7.33622015007259\\
-0.900390625	7.34652710878887\\
-0.89990234375	7.35688002096816\\
-0.8994140625	7.36727881483335\\
-0.89892578125	7.3777234182005\\
-0.8984375	7.38821375847841\\
-0.89794921875	7.39874976266828\\
-0.8974609375	7.40933135736324\\
-0.89697265625	7.41995846874803\\
-0.896484375	7.43063102259848\\
-0.89599609375	7.44134894428131\\
-0.8955078125	7.45211215875364\\
-0.89501953125	7.46292059056267\\
-0.89453125	7.47377416384533\\
-0.89404296875	7.48467280232797\\
-0.8935546875	7.49561642932601\\
-0.89306640625	7.50660496774358\\
-0.892578125	7.5176383400733\\
-0.89208984375	7.52871646839599\\
-0.8916015625	7.53983927438024\\
-0.89111328125	7.55100667928232\\
-0.890625	7.56221860394579\\
-0.89013671875	7.57347496880134\\
-0.8896484375	7.58477569386648\\
-0.88916015625	7.59612069874536\\
-0.888671875	7.60750990262855\\
-0.88818359375	7.61894322429278\\
-0.8876953125	7.6304205821009\\
-0.88720703125	7.64194189400149\\
-0.88671875	7.65350707752889\\
-0.88623046875	7.66511604980294\\
-0.8857421875	7.67676872752889\\
-0.88525390625	7.68846502699724\\
-0.884765625	7.70020486408369\\
-0.88427734375	7.71198815424896\\
-0.8837890625	7.72381481253882\\
-0.88330078125	7.73568475358389\\
-0.8828125	7.74759789159971\\
-0.88232421875	7.75955414038665\\
-0.8818359375	7.77155341332987\\
-0.88134765625	7.78359562339935\\
-0.880859375	7.79568068314987\\
-0.88037109375	7.80780850472108\\
-0.8798828125	7.81997899983747\\
-0.87939453125	7.83219207980852\\
-0.87890625	7.84444765552866\\
-0.87841796875	7.85674563747749\\
-0.8779296875	7.86908593571978\\
-0.87744140625	7.88146845990565\\
-0.876953125	7.89389311927067\\
-0.87646484375	7.90635982263611\\
-0.8759765625	7.918868478409\\
-0.87548828125	7.93141899458241\\
-0.875	7.94401127873563\\
-0.87451171875	7.95664523803441\\
-0.8740234375	7.96932077923124\\
-0.87353515625	7.98203780866553\\
-0.873046875	7.99479623226401\\
-0.87255859375	8.00759595554099\\
-0.8720703125	8.02043688359869\\
-0.87158203125	8.03331892112758\\
-0.87109375	8.04624197240678\\
-0.87060546875	8.05920594130441\\
-0.8701171875	8.07221073127807\\
-0.86962890625	8.08525624537517\\
-0.869140625	8.09834238623347\\
-0.86865234375	8.11146905608154\\
-0.8681640625	8.1246361567392\\
-0.86767578125	8.13784358961814\\
-0.8671875	8.15109125572237\\
-0.86669921875	8.16437905564879\\
-0.8662109375	8.17770688958786\\
-0.86572265625	8.19107465732415\\
-0.865234375	8.20448225823693\\
-0.86474609375	8.21792959130091\\
-0.8642578125	8.23141655508689\\
-0.86376953125	8.24494304776243\\
-0.86328125	8.25850896709264\\
-0.86279296875	8.2721142104409\\
-0.8623046875	8.28575867476963\\
-0.86181640625	8.29944225664109\\
-0.861328125	8.31316485221827\\
-0.86083984375	8.32692635726567\\
-0.8603515625	8.3407266671502\\
-0.85986328125	8.35456567684212\\
-0.859375	8.36844328091594\\
-0.85888671875	8.38235937355138\\
-0.8583984375	8.39631384853442\\
-0.85791015625	8.41030659925818\\
-0.857421875	8.4243375187241\\
-0.85693359375	8.43840649954295\\
-0.8564453125	8.45251343393589\\
-0.85595703125	8.46665821373571\\
-0.85546875	8.48084073038785\\
-0.85498046875	8.49506087495166\\
-0.8544921875	8.50931853810161\\
-0.85400390625	8.52361361012851\\
-0.853515625	8.53794598094078\\
-0.85302734375	8.55231554006579\\
-0.8525390625	8.56672217665114\\
-0.85205078125	8.58116577946605\\
-0.8515625	8.59564623690276\\
-0.85107421875	8.61016343697793\\
-0.8505859375	8.62471726733409\\
-0.85009765625	8.63930761524117\\
-0.849609375	8.653934367598\\
-0.84912109375	8.6685974109338\\
-0.8486328125	8.68329663140988\\
-0.84814453125	8.69803191482113\\
-0.84765625	8.71280314659774\\
-0.84716796875	8.7276102118069\\
-0.8466796875	8.74245299515443\\
-0.84619140625	8.75733138098662\\
-0.845703125	8.77224525329196\\
-0.84521484375	8.78719449570293\\
-0.8447265625	8.80217899149792\\
-0.84423828125	8.8171986236031\\
-0.84375	8.83225327459427\\
-0.84326171875	8.84734282669889\\
-0.8427734375	8.86246716179807\\
-0.84228515625	8.87762616142852\\
-0.841796875	8.89281970678468\\
-0.84130859375	8.90804767872084\\
-0.8408203125	8.92330995775318\\
-0.84033203125	8.93860642406202\\
-0.83984375	8.95393695749403\\
-0.83935546875	8.96930143756439\\
-0.8388671875	8.98469974345921\\
-0.83837890625	9.00013175403769\\
-0.837890625	9.01559734783464\\
-0.83740234375	9.03109640306275\\
-0.8369140625	9.04662879761506\\
-0.83642578125	9.0621944090675\\
-0.8359375	9.07779311468132\\
-0.83544921875	9.09342479140568\\
-0.8349609375	9.10908931588021\\
-0.83447265625	9.1247865644377\\
-0.833984375	9.1405164131067\\
-0.83349609375	9.15627873761431\\
-0.8330078125	9.17207341338882\\
-0.83251953125	9.18790031556263\\
-0.83203125	9.20375931897499\\
-0.83154296875	9.2196502981749\\
-0.8310546875	9.23557312742408\\
-0.83056640625	9.2515276806998\\
-0.830078125	9.26751383169799\\
-0.82958984375	9.28353145383625\\
-0.8291015625	9.29958042025691\\
-0.82861328125	9.3156606038302\\
-0.828125	9.33177187715733\\
-0.82763671875	9.34791411257379\\
-0.8271484375	9.36408718215258\\
-0.82666015625	9.38029095770747\\
-0.826171875	9.39652531079636\\
-0.82568359375	9.41279011272467\\
-0.8251953125	9.42908523454878\\
-0.82470703125	9.44541054707944\\
-0.82421875	9.46176592088537\\
-0.82373046875	9.47815122629674\\
-0.8232421875	9.49456633340879\\
-0.82275390625	9.51101111208557\\
-0.822265625	9.52748543196345\\
-0.82177734375	9.54398916245506\\
-0.8212890625	9.56052217275291\\
-0.82080078125	9.57708433183331\\
-0.8203125	9.5936755084602\\
-0.81982421875	9.61029557118911\\
-0.8193359375	9.62694438837109\\
-0.81884765625	9.64362182815671\\
-0.818359375	9.66032775850016\\
-0.81787109375	9.67706204716332\\
-0.8173828125	9.69382456171994\\
-0.81689453125	9.71061516955979\\
-0.81640625	9.72743373789294\\
-0.81591796875	9.744280133754\\
-0.8154296875	9.76115422400658\\
-0.81494140625	9.7780558753475\\
-0.814453125	9.79498495431137\\
-0.81396484375	9.81194132727493\\
-0.8134765625	9.82892486046171\\
-0.81298828125	9.84593541994654\\
-0.8125	9.86297287166013\\
-0.81201171875	9.88003708139384\\
-0.8115234375	9.89712791480425\\
-0.81103515625	9.91424523741807\\
-0.810546875	9.93138891463689\\
-0.81005859375	9.94855881174201\\
-0.8095703125	9.96575479389935\\
-0.80908203125	9.98297672616446\\
-0.80859375	10.0002244734875\\
-0.80810546875	10.0174979007182\\
-0.8076171875	10.0347968726112\\
-0.80712890625	10.0521212538309\\
-0.806640625	10.0694709089569\\
-0.80615234375	10.0868457024889\\
-0.8056640625	10.1042454988525\\
-0.80517578125	10.1216701624042\\
-0.8046875	10.1391195574368\\
-0.80419921875	10.1565935481849\\
-0.8037109375	10.1740919988304\\
-0.80322265625	10.1916147735081\\
-0.802734375	10.209161736311\\
-0.80224609375	10.2267327512964\\
-0.8017578125	10.2443276824911\\
-0.80126953125	10.2619463938974\\
-0.80078125	10.2795887494988\\
-0.80029296875	10.2972546132659\\
-0.7998046875	10.3149438491621\\
-0.79931640625	10.3326563211497\\
-0.798828125	10.3503918931958\\
-0.79833984375	10.3681504292784\\
-0.7978515625	10.3859317933921\\
-0.79736328125	10.4037358495548\\
-0.796875	10.4215624618134\\
-0.79638671875	10.4394114942502\\
-0.7958984375	10.4572828109891\\
-0.79541015625	10.4751762762018\\
-0.794921875	10.4930917541145\\
-0.79443359375	10.5110291090139\\
-0.7939453125	10.5289882052538\\
-0.79345703125	10.5469689072617\\
-0.79296875	10.5649710795452\\
-0.79248046875	10.5829945866986\\
-0.7919921875	10.6010392934094\\
-0.79150390625	10.6191050644655\\
-0.791015625	10.6371917647611\\
-0.79052734375	10.6552992593043\\
-0.7900390625	10.6734274132233\\
-0.78955078125	10.6915760917736\\
-0.7890625	10.7097451603448\\
-0.78857421875	10.7279344844676\\
-0.7880859375	10.7461439298205\\
-0.78759765625	10.7643733622374\\
-0.787109375	10.7826226477142\\
-0.78662109375	10.8008916524162\\
-0.7861328125	10.819180242685\\
-0.78564453125	10.8374882850461\\
-0.78515625	10.8558156462159\\
-0.78466796875	10.874162193109\\
-0.7841796875	10.8925277928457\\
-0.78369140625	10.9109123127591\\
-0.783203125	10.929315620403\\
-0.78271484375	10.9477375835589\\
-0.7822265625	10.9661780702438\\
-0.78173828125	10.9846369487176\\
-0.78125	11.0031140874909\\
-0.78076171875	11.0216093553324\\
-0.7802734375	11.0401226212767\\
-0.77978515625	11.0586537546321\\
-0.779296875	11.077202624988\\
-0.77880859375	11.0957691022234\\
-0.7783203125	11.1143530565139\\
-0.77783203125	11.1329543583402\\
-0.77734375	11.1515728784957\\
-0.77685546875	11.1702084880945\\
-0.7763671875	11.1888610585795\\
-0.77587890625	11.2075304617305\\
-0.775390625	11.2262165696717\\
-0.77490234375	11.2449192548806\\
-0.7744140625	11.2636383901956\\
-0.77392578125	11.2823738488244\\
-0.7734375	11.3011255043518\\
-0.77294921875	11.3198932307487\\
-0.7724609375	11.3386769023797\\
-0.77197265625	11.3574763940115\\
-0.771484375	11.3762915808217\\
-0.77099609375	11.3951223384065\\
-0.7705078125	11.4139685427896\\
-0.77001953125	11.4328300704304\\
-0.76953125	11.4517067982327\\
-0.76904296875	11.4705986035527\\
-0.7685546875	11.4895053642082\\
-0.76806640625	11.5084269584865\\
-0.767578125	11.5273632651534\\
-0.76708984375	11.5463141634618\\
-0.7666015625	11.5652795331599\\
-0.76611328125	11.5842592545005\\
-0.765625	11.6032532082492\\
-0.76513671875	11.6222612756932\\
-0.7646484375	11.6412833386503\\
-0.76416015625	11.6603192794774\\
-0.763671875	11.6793689810793\\
-0.76318359375	11.6984323269177\\
-0.7626953125	11.7175092010198\\
-0.76220703125	11.7365994879871\\
-0.76171875	11.7557030730049\\
-0.76123046875	11.7748198418502\\
-0.7607421875	11.7939496809014\\
-0.76025390625	11.813092477147\\
-0.759765625	11.8322481181944\\
-0.75927734375	11.8514164922791\\
-0.7587890625	11.8705974882735\\
-0.75830078125	11.889790995696\\
-0.7578125	11.9089969047199\\
-0.75732421875	11.9282151061828\\
-0.7568359375	11.9474454915951\\
-0.75634765625	11.9666879531495\\
-0.755859375	11.9859423837298\\
-0.75537109375	12.0052086769202\\
-0.7548828125	12.0244867270143\\
-0.75439453125	12.043776429024\\
-0.75390625	12.063077678689\\
-0.75341796875	12.0823903724859\\
-0.7529296875	12.1017144076367\\
-0.75244140625	12.1210496821189\\
-0.751953125	12.1403960946738\\
-0.75146484375	12.1597535448163\\
-0.7509765625	12.1791219328436\\
-0.75048828125	12.1985011598445\\
-0.75	12.2178911277087\\
-0.74951171875	12.2372917391359\\
-0.7490234375	12.2567028976448\\
-0.74853515625	12.2761245075825\\
-0.748046875	12.2955564741335\\
-0.74755859375	12.314998703329\\
-0.7470703125	12.3344511020561\\
-0.74658203125	12.3539135780666\\
-0.74609375	12.3733860399867\\
-0.74560546875	12.3928683973258\\
-0.7451171875	12.4123605604857\\
-0.74462890625	12.4318624407698\\
-0.744140625	12.4513739503923\\
-0.74365234375	12.4708950024872\\
-0.7431640625	12.4904255111174\\
-0.74267578125	12.5099653912841\\
-0.7421875	12.5295145589355\\
-0.74169921875	12.5490729309763\\
-0.7412109375	12.5686404252762\\
-0.74072265625	12.5882169606797\\
-0.740234375	12.6078024570146\\
-0.73974609375	12.6273968351013\\
-0.7392578125	12.6470000167616\\
-0.73876953125	12.6666119248281\\
-0.73828125	12.6862324831526\\
-0.73779296875	12.7058616166157\\
-0.7373046875	12.7254992511353\\
-0.73681640625	12.7451453136757\\
-0.736328125	12.7647997322565\\
-0.73583984375	12.7844624359614\\
-0.7353515625	12.8041333549472\\
-0.73486328125	12.8238124204527\\
-0.734375	12.8434995648072\\
-0.73388671875	12.8631947214394\\
-0.7333984375	12.8828978248866\\
-0.73291015625	12.9026088108026\\
-0.732421875	12.9223276159672\\
-0.73193359375	12.9420541782945\\
-0.7314453125	12.9617884368414\\
-0.73095703125	12.9815303318168\\
-0.73046875	13.0012798045893\\
-0.72998046875	13.0210367976965\\
-0.7294921875	13.0408012548534\\
-0.72900390625	13.0605731209605\\
-0.728515625	13.0803523421125\\
-0.72802734375	13.1001388656067\\
-0.7275390625	13.1199326399514\\
-0.72705078125	13.1397336148742\\
-0.7265625	13.1595417413302\\
-0.72607421875	13.1793569715103\\
-0.7255859375	13.1991792588495\\
-0.72509765625	13.2190085580351\\
-0.724609375	13.2388448250145\\
-0.72412109375	13.2586880170036\\
-0.7236328125	13.2785380924949\\
-0.72314453125	13.2983950112651\\
-0.72265625	13.3182587343834\\
-0.72216796875	13.3381292242192\\
-0.7216796875	13.3580064444502\\
-0.72119140625	13.3778903600697\\
-0.720703125	13.397780937395\\
-0.72021484375	13.4176781440746\\
-0.7197265625	13.4375819490959\\
-0.71923828125	13.4574923227932\\
-0.71875	13.4774092368548\\
-0.71826171875	13.4973326643305\\
-0.7177734375	13.5172625796393\\
-0.71728515625	13.5371989585766\\
-0.716796875	13.5571417783217\\
-0.71630859375	13.5770910174447\\
-0.7158203125	13.597046655914\\
-0.71533203125	13.6170086751033\\
-0.71484375	13.636977057799\\
-0.71435546875	13.6569517882068\\
-0.7138671875	13.6769328519587\\
-0.71337890625	13.6969202361202\\
-0.712890625	13.7169139291969\\
-0.71240234375	13.7369139211415\\
-0.7119140625	13.7569202033601\\
-0.71142578125	13.7769327687193\\
-0.7109375	13.7969516115525\\
-0.71044921875	13.8169767276664\\
-0.7099609375	13.8370081143478\\
-0.70947265625	13.8570457703693\\
-0.708984375	13.8770896959964\\
-0.70849609375	13.8971398929931\\
-0.7080078125	13.9171963646284\\
-0.70751953125	13.9372591156824\\
-0.70703125	13.9573281524521\\
-0.70654296875	13.9774034827577\\
-0.7060546875	13.9974851159479\\
-0.70556640625	14.0175730629066\\
-0.705078125	14.0376673360577\\
-0.70458984375	14.0577679493715\\
-0.7041015625	14.0778749183697\\
-0.70361328125	14.0979882601313\\
-0.703125	14.1181079932981\\
-0.70263671875	14.1382341380796\\
-0.7021484375	14.1583667162587\\
-0.70166015625	14.1785057511966\\
-0.701171875	14.1986512678384\\
-0.70068359375	14.2188032927177\\
-0.7001953125	14.2389618539614\\
-0.69970703125	14.2591269812954\\
-0.69921875	14.2792987060485\\
-0.69873046875	14.2994770611577\\
-0.6982421875	14.3196620811727\\
-0.69775390625	14.3398538022603\\
-0.697265625	14.3600522622092\\
-0.69677734375	14.3802575004341\\
-0.6962890625	14.4004695579801\\
-0.69580078125	14.4206884775271\\
-0.6953125	14.4409143033939\\
-0.69482421875	14.4611470815419\\
-0.6943359375	14.4813868595797\\
-0.69384765625	14.5016336867666\\
-0.693359375	14.5218876140163\\
-0.69287109375	14.5421486939011\\
-0.6923828125	14.5624169806552\\
-0.69189453125	14.5826925301784\\
-0.69140625	14.6029754000393\\
-0.69091796875	14.6232656494793\\
-0.6904296875	14.643563339415\\
-0.68994140625	14.6638685324425\\
-0.689453125	14.6841812928395\\
-0.68896484375	14.7045016865691\\
-0.6884765625	14.7248297812823\\
-0.68798828125	14.7451656463212\\
-0.6875	14.7655093527215\\
-0.68701171875	14.7858609732154\\
-0.6865234375	14.8062205822341\\
-0.68603515625	14.8265882559103\\
-0.685546875	14.8469640720809\\
-0.68505859375	14.8673481102888\\
-0.6845703125	14.8877404517859\\
-0.68408203125	14.9081411795346\\
-0.68359375	14.9285503782102\\
-0.68310546875	14.948968134203\\
-0.6826171875	14.9693945356199\\
-0.68212890625	14.9898296722866\\
-0.681640625	15.010273635749\\
-0.68115234375	15.0307265192749\\
-0.6806640625	15.051188417856\\
-0.68017578125	15.0716594282085\\
-0.6796875	15.0921396487756\\
-0.67919921875	15.1126291797277\\
-0.6787109375	15.1331281229645\\
-0.67822265625	15.1536365821156\\
-0.677734375	15.1741546625416\\
-0.67724609375	15.1946824713355\\
-0.6767578125	15.2152201173229\\
-0.67626953125	15.2357677110632\\
-0.67578125	15.2563253648504\\
-0.67529296875	15.2768931927134\\
-0.6748046875	15.2974713104165\\
-0.67431640625	15.3180598354604\\
-0.673828125	15.3386588870818\\
-0.67333984375	15.3592685862544\\
-0.6728515625	15.3798890556883\\
-0.67236328125	15.4005204198311\\
-0.671875	15.4211628048669\\
-0.67138671875	15.441816338717\\
-0.6708984375	15.4624811510395\\
-0.67041015625	15.4831573732289\\
-0.669921875	15.5038451384162\\
-0.66943359375	15.5245445814682\\
-0.6689453125	15.545255838987\\
-0.66845703125	15.5659790493099\\
-0.66796875	15.5867143525082\\
-0.66748046875	15.607461890387\\
-0.6669921875	15.6282218064841\\
-0.66650390625	15.6489942460691\\
-0.666015625	15.669779356143\\
-0.66552734375	15.6905772854364\\
-0.6650390625	15.7113881844091\\
-0.66455078125	15.7322122052486\\
-0.6640625	15.7530495018689\\
-0.66357421875	15.7739002299093\\
-0.6630859375	15.7947645467327\\
-0.66259765625	15.8156426114248\\
-0.662109375	15.8365345847917\\
-0.66162109375	15.857440629359\\
-0.6611328125	15.8783609093699\\
-0.66064453125	15.8992955907833\\
-0.66015625	15.9202448412722\\
-0.65966796875	15.9412088302218\\
-0.6591796875	15.9621877287274\\
-0.65869140625	15.9831817095925\\
-0.658203125	16.0041909473267\\
-0.65771484375	16.0252156181435\\
-0.6572265625	16.0462558999581\\
-0.65673828125	16.0673119723853\\
-0.65625	16.0883840167369\\
-0.65576171875	16.1094722160193\\
-0.6552734375	16.1305767549312\\
-0.65478515625	16.151697819861\\
-0.654296875	16.1728355988842\\
-0.65380859375	16.1939902817608\\
-0.6533203125	16.2151620599323\\
-0.65283203125	16.2363511265196\\
-0.65234375	16.2575576763192\\
-0.65185546875	16.2787819058013\\
-0.6513671875	16.3000240131063\\
-0.65087890625	16.3212841980418\\
-0.650390625	16.3425626620799\\
-0.64990234375	16.3638596083537\\
-0.6494140625	16.3851752416545\\
-0.64892578125	16.4065097684284\\
-0.6484375	16.4278633967731\\
-0.64794921875	16.4492363364345\\
-0.6474609375	16.4706287988036\\
-0.64697265625	16.492040996913\\
-0.646484375	16.5134731454331\\
-0.64599609375	16.5349254606695\\
-0.6455078125	16.5563981605583\\
-0.64501953125	16.5778914646637\\
-0.64453125	16.5994055941734\\
-0.64404296875	16.6209407718956\\
-0.6435546875	16.6424972222552\\
-0.64306640625	16.6640751712897\\
-0.642578125	16.6856748466459\\
-0.64208984375	16.7072964775759\\
-0.6416015625	16.728940294933\\
-0.64111328125	16.7506065311684\\
-0.640625	16.7722954203267\\
-0.64013671875	16.7940071980425\\
-0.6396484375	16.815742101536\\
-0.63916015625	16.8375003696092\\
-0.638671875	16.8592822426418\\
-0.63818359375	16.8810879625872\\
-0.6376953125	16.9029177729683\\
-0.63720703125	16.9247719188737\\
-0.63671875	16.946650646953\\
-0.63623046875	16.9685542054134\\
-0.6357421875	16.9904828440148\\
-0.63525390625	17.0124368140659\\
-0.634765625	17.0344163684201\\
-0.63427734375	17.0564217614711\\
-0.6337890625	17.0784532491487\\
-0.63330078125	17.1005110889144\\
-0.6328125	17.1225955397573\\
-0.63232421875	17.1447068621896\\
-0.6318359375	17.1668453182425\\
-0.63134765625	17.1890111714617\\
-0.630859375	17.2112046869031\\
-0.63037109375	17.2334261311284\\
-0.6298828125	17.2556757722008\\
-0.62939453125	17.2779538796807\\
-0.62890625	17.3002607246213\\
-0.62841796875	17.3225965795639\\
-0.6279296875	17.3449617185341\\
-0.62744140625	17.3673564170369\\
-0.626953125	17.3897809520524\\
-0.62646484375	17.4122356020318\\
-0.6259765625	17.4347206468926\\
-0.62548828125	17.4572363680143\\
-0.625	17.479783048234\\
-0.62451171875	17.5023609718421\\
-0.6240234375	17.5249704245781\\
-0.62353515625	17.5476116936257\\
-0.623046875	17.5702850676091\\
-0.62255859375	17.592990836588\\
-0.6220703125	17.6157292920537\\
-0.62158203125	17.6385007269247\\
-0.62109375	17.6613054355423\\
-0.62060546875	17.6841437136662\\
-0.6201171875	17.7070158584705\\
-0.61962890625	17.7299221685391\\
-0.619140625	17.7528629438617\\
-0.61865234375	17.7758384858292\\
-0.6181640625	17.7988490972302\\
-0.61767578125	17.8218950822458\\
-0.6171875	17.8449767464463\\
-0.61669921875	17.8680943967867\\
-0.6162109375	17.8912483416022\\
-0.61572265625	17.9144388906049\\
-0.615234375	17.937666354879\\
-0.61474609375	17.9609310468771\\
-0.6142578125	17.9842332804161\\
-0.61376953125	18.0075733706731\\
-0.61328125	18.0309516341817\\
-0.61279296875	18.0543683888277\\
-0.6123046875	18.0778239538457\\
-0.61181640625	18.1013186498146\\
-0.611328125	18.1248527986542\\
-0.61083984375	18.1484267236215\\
-0.6103515625	18.1720407493066\\
-0.60986328125	18.1956952016291\\
-0.609375	18.2193904078345\\
-0.60888671875	18.2431266964905\\
-0.6083984375	18.2669043974835\\
-0.60791015625	18.290723842015\\
-0.607421875	18.314585362598\\
-0.60693359375	18.3384892930535\\
-0.6064453125	18.3624359685075\\
-0.60595703125	18.386425725387\\
-0.60546875	18.4104589014171\\
-0.60498046875	18.4345358356178\\
-0.6044921875	18.4586568683005\\
-0.60400390625	18.482822341065\\
-0.603515625	18.5070325967962\\
-0.60302734375	18.5312879796613\\
-0.6025390625	18.5555888351068\\
-0.60205078125	18.5799355098551\\
-0.6015625	18.6043283519024\\
-0.60107421875	18.628767710515\\
-0.6005859375	18.6532539362271\\
-0.60009765625	18.6777873808381\\
-0.599609375	18.7023683974095\\
-0.59912109375	18.7269973402627\\
-0.5986328125	18.7516745649764\\
-0.59814453125	18.7764004283841\\
-0.59765625	18.8011752885718\\
-0.59716796875	18.8259995048754\\
-0.5966796875	18.8508734378789\\
-0.59619140625	18.8757974494117\\
-0.595703125	18.9007719025469\\
-0.59521484375	18.9257971615989\\
-0.5947265625	18.9508735921219\\
-0.59423828125	18.9760015609074\\
-0.59375	19.001181435983\\
-0.59326171875	19.0264135866099\\
-0.5927734375	19.051698383282\\
-0.59228515625	19.0770361977239\\
-0.591796875	19.1024274028892\\
-0.59130859375	19.1278723729595\\
-0.5908203125	19.1533714833429\\
-0.59033203125	19.1789251106723\\
-0.58984375	19.2045336328047\\
-0.58935546875	19.2301974288201\\
-0.5888671875	19.2559168790201\\
-0.58837890625	19.2816923649271\\
-0.587890625	19.3075242692835\\
-0.58740234375	19.3334129760511\\
-0.5869140625	19.3593588704099\\
-0.58642578125	19.3853623387581\\
-0.5859375	19.4114237687111\\
-0.58544921875	19.4375435491014\\
-0.5849609375	19.4637220699781\\
-0.58447265625	19.4899597226069\\
-0.583984375	19.5162568994695\\
-0.58349609375	19.5426139942641\\
-0.5830078125	19.5690314019051\\
-0.58251953125	19.5955095185234\\
-0.58203125	19.6220487414665\\
-0.58154296875	19.648649469299\\
-0.5810546875	19.675312101803\\
-0.58056640625	19.7020370399786\\
-0.580078125	19.7288246860445\\
-0.57958984375	19.7556754434389\\
-0.5791015625	19.7825897168204\\
-0.57861328125	19.8095679120684\\
-0.578125	19.8366104362851\\
-0.57763671875	19.8637176977958\\
-0.5771484375	19.8908901061506\\
-0.57666015625	19.9181280721258\\
-0.576171875	19.9454320077251\\
-0.57568359375	19.9728023261816\\
-0.5751953125	20.0002394419592\\
-0.57470703125	20.0277437707546\\
-0.57421875	20.0553157294989\\
-0.57373046875	20.0829557363602\\
-0.5732421875	20.1106642107454\\
-0.57275390625	20.1384415733025\\
-0.572265625	20.1662882459231\\
-0.57177734375	20.1942046517446\\
-0.5712890625	20.2221912151536\\
-0.57080078125	20.2502483617876\\
-0.5703125	20.2783765185387\\
-0.56982421875	20.3065761135562\\
-0.5693359375	20.3348475762499\\
-0.56884765625	20.3631913372931\\
-0.568359375	20.3916078286263\\
-0.56787109375	20.4200974834604\\
-0.5673828125	20.4486607362802\\
-0.56689453125	20.4772980228486\\
-0.56640625	20.5060097802102\\
-0.56591796875	20.5347964466952\\
-0.5654296875	20.5636584619236\\
-0.56494140625	20.5925962668095\\
-0.564453125	20.6216103035657\\
-0.56396484375	20.6507010157076\\
-0.5634765625	20.6798688480585\\
-0.56298828125	20.7091142467541\\
-0.5625	20.7384376592474\\
-0.56201171875	20.7678395343139\\
-0.5615234375	20.7973203220566\\
-0.56103515625	20.8268804739116\\
-0.560546875	20.8565204426534\\
-0.56005859375	20.8862406824003\\
-0.5595703125	20.9160416486209\\
-0.55908203125	20.9459237981391\\
-0.55859375	20.9758875891408\\
-0.55810546875	21.0059334811798\\
-0.5576171875	21.0360619351841\\
-0.55712890625	21.0662734134626\\
-0.556640625	21.0965683797116\\
-0.55615234375	21.1269472990217\\
-0.5556640625	21.1574106378844\\
-0.55517578125	21.1879588641997\\
-0.5546875	21.2185924472829\\
-0.55419921875	21.2493118578723\\
-0.5537109375	21.2801175681363\\
-0.55322265625	21.3110100516818\\
-0.552734375	21.3419897835616\\
-0.55224609375	21.3730572402822\\
-0.5517578125	21.4042128998125\\
-0.55126953125	21.435457241592\\
-0.55078125	21.4667907465391\\
-0.55029296875	21.4982138970599\\
-0.5498046875	21.529727177057\\
-0.54931640625	21.5613310719385\\
-0.548828125	21.5930260686273\\
-0.54833984375	21.6248126555701\\
-0.5478515625	21.6566913227471\\
-0.54736328125	21.6886625616821\\
-0.546875	21.7207268654513\\
-0.54638671875	21.7528847286943\\
-0.5458984375	21.785136647624\\
-0.54541015625	21.8174831200366\\
-0.544921875	21.8499246453227\\
-0.54443359375	21.8824617244776\\
-0.5439453125	21.9150948601123\\
-0.54345703125	21.9478245564649\\
-0.54296875	21.9806513194114\\
-0.54248046875	22.0135756564774\\
-0.5419921875	22.0465980768499\\
-0.54150390625	22.0797190913885\\
-0.541015625	22.1129392126383\\
-0.54052734375	22.1462589548413\\
-0.5400390625	22.1796788339494\\
-0.53955078125	22.2131993676366\\
-0.5390625	22.2468210753119\\
-0.53857421875	22.2805444781322\\
-0.5380859375	22.3143700990158\\
-0.53759765625	22.3482984626554\\
-0.537109375	22.3823300955318\\
-0.53662109375	22.4164655259277\\
-0.5361328125	22.450705283942\\
-0.53564453125	22.4850499015031\\
-0.53515625	22.5194999123844\\
-0.53466796875	22.5540558522182\\
-0.5341796875	22.5887182585107\\
-0.53369140625	22.6234876706571\\
-0.533203125	22.6583646299567\\
-0.53271484375	22.6933496796286\\
-0.5322265625	22.7284433648273\\
-0.53173828125	22.7636462326583\\
-0.53125	22.7989588321949\\
-0.53076171875	22.8343817144937\\
-0.5302734375	22.8699154326121\\
-0.52978515625	22.9055605416242\\
-0.529296875	22.9413175986387\\
-0.52880859375	22.9771871628154\\
-0.5283203125	23.0131697953833\\
-0.52783203125	23.049266059658\\
-0.52734375	23.0854765210599\\
-0.52685546875	23.1218017471323\\
-0.5263671875	23.1582423075598\\
-0.52587890625	23.1947987741873\\
-0.525390625	23.2314717210386\\
-0.52490234375	23.2682617243359\\
-0.5244140625	23.3051693625192\\
-0.52392578125	23.342195216266\\
-0.5234375	23.379339868511\\
-0.52294921875	23.4166039044671\\
-0.5224609375	23.4539879116449\\
-0.52197265625	23.4914924798744\\
-0.521484375	23.5291182013255\\
-0.52099609375	23.5668656705294\\
-0.5205078125	23.6047354844\\
-0.52001953125	23.6427282422565\\
-0.51953125	23.6808445458446\\
-0.51904296875	23.7190849993594\\
-0.5185546875	23.7574502094681\\
-0.51806640625	23.7959407853329\\
-0.517578125	23.8345573386342\\
-0.51708984375	23.8733004835944\\
-0.5166015625	23.9121708370017\\
-0.51611328125	23.9511690182339\\
-0.515625	23.9902956492834\\
-0.51513671875	24.0295513547818\\
-0.5146484375	24.0689367620246\\
-0.51416015625	24.1084525009973\\
-0.513671875	24.1480992044004\\
-0.51318359375	24.187877507676\\
-0.5126953125	24.2277880490338\\
-0.51220703125	24.2678314694777\\
-0.51171875	24.3080084128334\\
-0.51123046875	24.3483195257747\\
-0.5107421875	24.3887654578524\\
-0.51025390625	24.4293468615213\\
-0.509765625	24.4700643921689\\
-0.50927734375	24.5109187081444\\
-0.5087890625	24.5519104707872\\
-0.50830078125	24.5930403444568\\
-0.5078125	24.6343089965623\\
-0.50732421875	24.6757170975925\\
-0.5068359375	24.7172653211465\\
-0.50634765625	24.7589543439645\\
-0.505859375	24.8007848459591\\
-0.50537109375	24.8427575102469\\
-0.5048828125	24.8848730231804\\
-0.50439453125	24.9271320743806\\
-0.50390625	24.9695353567696\\
-0.50341796875	25.0120835666037\\
-0.5029296875	25.0547774035073\\
-0.50244140625	25.0976175705066\\
-0.501953125	25.1406047740642\\
-0.50146484375	25.1837397241136\\
-0.5009765625	25.2270231340948\\
-0.50048828125	25.2704557209899\\
-0.5	25.3140382053588\\
-0.49951171875	25.3577713113764\\
-0.4990234375	25.4016557668689\\
-0.49853515625	25.4456923033518\\
-0.498046875	25.4898816560671\\
-0.49755859375	25.5342245640224\\
-0.4970703125	25.5787217700291\\
-0.49658203125	25.623374020742\\
-0.49609375	25.6681820666988\\
-0.49560546875	25.7131466623605\\
-0.4951171875	25.7582685661519\\
-0.49462890625	25.8035485405031\\
-0.494140625	25.848987351891\\
-0.49365234375	25.8945857708814\\
-0.4931640625	25.9403445721719\\
-0.49267578125	25.9862645346352\\
-0.4921875	26.0323464413625\\
-0.49169921875	26.0785910797082\\
-0.4912109375	26.1249992413344\\
-0.49072265625	26.1715717222565\\
-0.490234375	26.2183093228891\\
-0.48974609375	26.2652128480923\\
-0.4892578125	26.312283107219\\
-0.48876953125	26.3595209141622\\
-0.48828125	26.4069270874036\\
-0.48779296875	26.4545024500622\\
-0.4873046875	26.5022478299434\\
-0.48681640625	26.5501640595897\\
-0.486328125	26.5982519763307\\
-0.48583984375	26.6465124223344\\
-0.4853515625	26.6949462446594\\
-0.48486328125	26.7435542953071\\
-0.484375	26.7923374312749\\
-0.48388671875	26.8412965146101\\
-0.4833984375	26.8904324124641\\
-0.48291015625	26.939745997148\\
-0.482421875	26.9892381461879\\
-0.48193359375	27.0389097423816\\
-0.4814453125	27.0887616738562\\
-0.48095703125	27.1387948341252\\
-0.48046875	27.1890101221479\\
-0.47998046875	27.2394084423884\\
-0.4794921875	27.2899907048755\\
-0.47900390625	27.3407578252643\\
-0.478515625	27.3917107248967\\
-0.47802734375	27.442850330865\\
-0.4775390625	27.4941775760741\\
-0.47705078125	27.545693399306\\
-0.4765625	27.5973987452842\\
-0.47607421875	27.6492945647396\\
-0.4755859375	27.7013818144766\\
-0.47509765625	27.7536614574407\\
-0.474609375	27.8061344627861\\
-0.47412109375	27.8588018059452\\
-0.4736328125	27.9116644686978\\
-0.47314453125	27.9647234392424\\
-0.47265625	28.0179797122674\\
-0.47216796875	28.0714342890238\\
-0.4716796875	28.1250881773984\\
-0.47119140625	28.1789423919888\\
-0.470703125	28.2329979541776\\
-0.47021484375	28.2872558922099\\
-0.4697265625	28.3417172412699\\
-0.46923828125	28.3963830435592\\
-0.46875	28.4512543483762\\
-0.46826171875	28.5063322121962\\
-0.4677734375	28.5616176987531\\
-0.46728515625	28.6171118791215\\
-0.466796875	28.6728158317998\\
-0.46630859375	28.7287306427956\\
-0.4658203125	28.7848574057105\\
-0.46533203125	28.8411972218272\\
-0.46484375	28.8977512001973\\
-0.46435546875	28.9545204577305\\
-0.4638671875	29.0115061192841\\
-0.46337890625	29.0687093177555\\
-0.462890625	29.1261311941735\\
-0.46240234375	29.183772897793\\
-0.4619140625	29.2416355861896\\
-0.46142578125	29.2997204253558\\
-0.4609375	29.3580285897988\\
-0.46044921875	29.4165612626392\\
-0.4599609375	29.4753196357108\\
-0.45947265625	29.5343049096627\\
-0.458984375	29.5935182940613\\
-0.45849609375	29.6529610074954\\
-0.4580078125	29.7126342776809\\
-0.45751953125	29.7725393415684\\
-0.45703125	29.832677445451\\
-0.45654296875	29.8930498450749\\
-0.4560546875	29.9536578057501\\
-0.45556640625	30.0145026024633\\
-0.455078125	30.0755855199927\\
-0.45458984375	30.1369078530234\\
-0.4541015625	30.1984709062651\\
-0.45361328125	30.260275994571\\
-0.453125	30.3223244430584\\
-0.45263671875	30.3846175872315\\
-0.4521484375	30.4471567731045\\
-0.45166015625	30.5099433573281\\
-0.451171875	30.5729787073163\\
-0.45068359375	30.6362642013755\\
-0.4501953125	30.6998012288357\\
-0.44970703125	30.7635911901831\\
-0.44921875	30.8276354971941\\
-0.44873046875	30.8919355730723\\
-0.4482421875	30.956492852586\\
-0.44775390625	31.021308782209\\
-0.447265625	31.0863848202621\\
-0.44677734375	31.1517224370573\\
-0.4462890625	31.2173231150436\\
-0.44580078125	31.2831883489556\\
-0.4453125	31.3493196459628\\
-0.44482421875	31.4157185258219\\
-0.4443359375	31.4823865210317\\
-0.44384765625	31.5493251769887\\
-0.443359375	31.6165360521461\\
-0.44287109375	31.6840207181745\\
-0.4423828125	31.7517807601252\\
-0.44189453125	31.8198177765948\\
-0.44140625	31.8881333798938\\
-0.44091796875	31.956729196216\\
-0.4404296875	32.025606865811\\
-0.43994140625	32.0947680431588\\
-0.439453125	32.1642143971476\\
-0.43896484375	32.233947611253\\
-0.4384765625	32.3039693837206\\
-0.43798828125	32.3742814277507\\
-0.4375	32.4448854716858\\
-0.43701171875	32.5157832592007\\
-0.4365234375	32.5869765494953\\
-0.43603515625	32.6584671174902\\
-0.435546875	32.7302567540248\\
-0.43505859375	32.8023472660587\\
-0.4345703125	32.874740476875\\
-0.43408203125	32.9474382262879\\
-0.43359375	33.020442370852\\
-0.43310546875	33.0937547840752\\
-0.4326171875	33.1673773566343\\
-0.43212890625	33.241311996594\\
-0.431640625	33.3155606296284\\
-0.43115234375	33.3901251992467\\
-0.4306640625	33.4650076670209\\
-0.43017578125	33.5402100128175\\
-0.4296875	33.6157342350323\\
-0.42919921875	33.6915823508284\\
-0.4287109375	33.7677563963778\\
-0.42822265625	33.8442584271061\\
-0.427734375	33.921090517941\\
-0.42724609375	33.9982547635641\\
-0.4267578125	34.0757532786665\\
-0.42626953125	34.1535881982076\\
-0.42578125	34.2317616776779\\
-0.42529296875	34.3102758933655\\
-0.4248046875	34.389133042626\\
-0.42431640625	34.4683353441569\\
-0.423828125	34.5478850382747\\
-0.42333984375	34.6277843871975\\
-0.4228515625	34.7080356753296\\
-0.42236328125	34.7886412095523\\
-0.421875	34.8696033195164\\
-0.42138671875	34.9509243579408\\
-0.4208984375	35.0326067009139\\
-0.42041015625	35.1146527482001\\
-0.419921875	35.1970649235495\\
-0.41943359375	35.279845675013\\
-0.4189453125	35.362997475261\\
-0.41845703125	35.4465228219066\\
-0.41796875	35.5304242378332\\
-0.41748046875	35.6147042715266\\
-0.4169921875	35.6993654974117\\
-0.41650390625	35.7844105161937\\
-0.416015625	35.8698419552035\\
-0.41552734375	35.955662468748\\
-0.4150390625	36.0418747384654\\
-0.41455078125	36.1284814736845\\
-0.4140625	36.2154854117891\\
-0.41357421875	36.3028893185872\\
-0.4130859375	36.3906959886847\\
-0.41259765625	36.4789082458641\\
-0.412109375	36.567528943468\\
-0.41162109375	36.6565609647875\\
-0.4111328125	36.7460072234555\\
-0.41064453125	36.8358706638443\\
-0.41015625	36.9261542614691\\
-0.40966796875	37.0168610233959\\
-0.4091796875	37.1079939886537\\
-0.40869140625	37.1995562286531\\
-0.408203125	37.291550847608\\
-0.40771484375	37.3839809829637\\
-0.4072265625	37.4768498058294\\
-0.40673828125	37.5701605214146\\
-0.40625	37.6639163694724\\
-0.40576171875	37.7581206247453\\
-0.4052734375	37.8527765974176\\
-0.40478515625	37.9478876335711\\
-0.404296875	38.0434571156469\\
-0.40380859375	38.1394884629102\\
-0.4033203125	38.2359851319212\\
-0.40283203125	38.332950617009\\
-0.40234375	38.4303884507507\\
-0.40185546875	38.5283022044549\\
-0.4013671875	38.6266954886484\\
-0.40087890625	38.7255719535679\\
-0.400390625	38.8249352896549\\
-0.39990234375	38.9247892280541\\
-0.3994140625	39.0251375411164\\
-0.39892578125	39.1259840429041\\
-0.3984375	39.2273325896994\\
-0.39794921875	39.3291870805162\\
-0.3974609375	39.4315514576148\\
-0.39697265625	39.5344297070173\\
-0.396484375	39.6378258590276\\
-0.39599609375	39.7417439887507\\
-0.3955078125	39.8461882166155\\
-0.39501953125	39.9511627088969\\
-0.39453125	40.0566716782399\\
-0.39404296875	40.1627193841829\\
-0.3935546875	40.2693101336816\\
-0.39306640625	40.3764482816317\\
-0.392578125	40.4841382313905\\
-0.39208984375	40.5923844352962\\
-0.3916015625	40.7011913951855\\
-0.39111328125	40.8105636629079\\
-0.390625	40.9205058408353\\
-0.39013671875	41.0310225823683\\
-0.3896484375	41.1421185924365\\
-0.38916015625	41.2537986279921\\
-0.388671875	41.3660674984974\\
-0.38818359375	41.4789300664026\\
-0.3876953125	41.5923912476162\\
-0.38720703125	41.7064560119633\\
-0.38671875	41.8211293836335\\
-0.38623046875	41.936416441615\\
-0.3857421875	42.0523223201154\\
-0.38525390625	42.1688522089658\\
-0.384765625	42.2860113540081\\
-0.38427734375	42.4038050574631\\
-0.3837890625	42.5222386782782\\
-0.38330078125	42.6413176324517\\
-0.3828125	42.7610473933328\\
-0.38232421875	42.8814334918949\\
-0.3818359375	43.0024815169791\\
-0.38134765625	43.1241971155062\\
-0.380859375	43.246585992655\\
-0.38037109375	43.3696539120023\\
-0.3798828125	43.4934066956241\\
-0.37939453125	43.6178502241523\\
-0.37890625	43.7429904367862\\
-0.37841796875	43.8688333312521\\
-0.3779296875	43.9953849637104\\
-0.37744140625	44.1226514486036\\
-0.376953125	44.250638958442\\
-0.37646484375	44.3793537235228\\
-0.3759765625	44.5088020315767\\
-0.37548828125	44.6389902273373\\
-0.375	44.7699247120284\\
-0.37451171875	44.9016119427614\\
-0.3740234375	45.0340584318393\\
-0.37353515625	45.1672707459576\\
-0.373046875	45.3012555052971\\
-0.37255859375	45.4360193825002\\
-0.3720703125	45.571569101523\\
-0.37158203125	45.707911436354\\
-0.37109375	45.8450532095908\\
-0.37060546875	45.9830012908653\\
-0.3701171875	46.1217625951057\\
-0.36962890625	46.261344080627\\
-0.369140625	46.4017527470354\\
-0.36865234375	46.5429956329357\\
-0.3681640625	46.6850798134288\\
-0.36767578125	46.8280123973837\\
-0.3671875	46.9718005244716\\
-0.36669921875	47.116451361943\\
-0.3662109375	47.2619721011339\\
-0.36572265625	47.4083699536812\\
-0.365234375	47.5556521474292\\
-0.36474609375	47.7038259220056\\
-0.3642578125	47.8528985240484\\
-0.36376953125	48.0028772020568\\
-0.36328125	48.1537692008438\\
-0.36279296875	48.3055817555661\\
-0.3623046875	48.4583220853001\\
-0.36181640625	48.611997386138\\
-0.361328125	48.766614823771\\
-0.36083984375	48.9221815255273\\
-0.3603515625	49.0787045718302\\
-0.35986328125	49.2361909870374\\
-0.359375	49.3946477296248\\
-0.35888671875	49.5540816816685\\
-0.3583984375	49.7144996375862\\
-0.35791015625	49.8759082920839\\
-0.357421875	50.038314227264\\
-0.35693359375	50.2017238988374\\
-0.3564453125	50.3661436213842\\
-0.35595703125	50.5315795526045\\
-0.35546875	50.6980376764939\\
-0.35498046875	50.8655237853773\\
-0.3544921875	51.0340434607304\\
-0.35400390625	51.203602052713\\
-0.353515625	51.3742046583345\\
-0.35302734375	51.5458560981673\\
-0.3525390625	51.7185608915194\\
-0.35205078125	51.8923232299733\\
-0.3515625	52.0671469491888\\
-0.35107421875	52.2430354988696\\
-0.3505859375	52.4199919107802\\
-0.35009765625	52.5980187646992\\
-0.349609375	52.7771181521862\\
-0.34912109375	52.9572916380346\\
-0.3486328125	53.1385402192772\\
-0.34814453125	53.3208642816017\\
-0.34765625	53.5042635530325\\
-0.34716796875	53.6887370547223\\
-0.3466796875	53.8742830486946\\
-0.34619140625	54.0608989823717\\
-0.345703125	54.2485814297145\\
-0.34521484375	54.4373260287945\\
-0.3447265625	54.627127415616\\
-0.34423828125	54.8179791539959\\
-0.34375	55.0098736613074\\
-0.34326171875	55.2028021298902\\
-0.3427734375	55.3967544439207\\
-0.34228515625	55.5917190915437\\
-0.341796875	55.7876830720547\\
-0.34130859375	55.9846317979321\\
-0.3408203125	56.1825489915139\\
-0.34033203125	56.3814165761231\\
-0.33984375	56.5812145614502\\
-0.33935546875	56.7819209230114\\
-0.3388671875	56.9835114755166\\
-0.33837890625	57.1859597399941\\
-0.337890625	57.3892368045466\\
-0.33740234375	57.5933111786334\\
-0.3369140625	57.7981486408107\\
-0.33642578125	58.0037120798978\\
-0.3359375	58.2099613295855\\
-0.33544921875	58.4168529965535\\
-0.3349609375	58.6243402822307\\
-0.33447265625	58.8323727984002\\
-0.333984375	59.0408963769399\\
-0.33349609375	59.2498528740792\\
-0.3330078125	59.459179969665\\
-0.33251953125	59.66881096205\\
-0.33203125	59.878674559352\\
-0.33154296875	60.0886946679905\\
-0.3310546875	60.2987901795647\\
-0.33056640625	60.5088747573381\\
-0.330078125	60.7188566237805\\
-0.32958984375	60.9286383508557\\
-0.3291015625	61.1381166549647\\
-0.32861328125	61.34718219872\\
-0.328125	61.5557194019902\\
-0.32763671875	61.7636062649355\\
-0.3271484375	61.9707142060488\\
-0.32666015625	62.1769079185166\\
-0.326171875	62.3820452485132\\
-0.32568359375	62.5859770993399\\
-0.3251953125	62.7885473656113\\
-0.32470703125	62.9895929019521\\
-0.32421875	63.1889435309181\\
-0.32373046875	63.3864220950436\\
-0.3232421875	63.5818445580791\\
-0.32275390625	63.7750201605563\\
-0.322265625	63.9657516348323\\
-0.32177734375	64.1538354846628\\
-0.3212890625	64.3390623341661\\
-0.32080078125	64.5212173507032\\
-0.3203125	64.7000807457349\\
-0.31982421875	64.8754283570999\\
-0.3193359375	65.0470323153639\\
-0.31884765625	65.2146617959421\\
-0.318359375	65.3780838575557\\
-0.31787109375	65.5370643662852\\
-0.3173828125	65.6913690030132\\
-0.31689453125	65.8407643504287\\
-0.31640625	65.9850190540328\\
-0.31591796875	66.1239050497474\\
-0.3154296875	66.2571988488541\\
-0.31494140625	66.3846828690989\\
-0.314453125	66.5061467989778\\
-0.31396484375	66.6213889805023\\
-0.3134765625	66.7302177942051\\
-0.31298828125	66.8324530288722\\
-0.3125	66.9279272175031\\
-0.31201171875	67.0164869204007\\
-0.3115234375	67.0979939361063\\
-0.31103515625	67.1723264211713\\
-0.310546875	67.2393799005117\\
-0.31005859375	67.2990681513369\\
-0.3095703125	67.3513239453672\\
-0.30908203125	67.3960996362138\\
-0.30859375	67.4333675813549\\
-0.30810546875	67.4631203910181\\
-0.3076171875	67.485370999395\\
-0.30712890625	67.5001525568754\\
-0.306640625	67.5075181452907\\
-0.30615234375	67.5075403213922\\
-0.3056640625	67.5003104968695\\
-0.30517578125	67.4859381660283\\
-0.3046875	67.4645499947296\\
-0.30419921875	67.4362887862704\\
-0.3037109375	67.4013123415049\\
-0.30322265625	67.3597922316583\\
-0.302734375	67.3119125029357\\
-0.30224609375	67.2578683322095\\
-0.3017578125	67.1978646527928\\
-0.30126953125	67.1321147686214\\
-0.30078125	67.0608389741276\\
-0.30029296875	66.9842631957369\\
-0.2998046875	66.9026176693577\\
-0.29931640625	66.8161356664793\\
-0.298828125	66.7250522796574\\
-0.29833984375	66.6296032762733\\
-0.2978515625	66.5300240275823\\
-0.29736328125	66.4265485182492\\
-0.296875	66.3194084398526\\
-0.29638671875	66.2088323702526\\
-0.2958984375	66.09504503928\\
-0.29541015625	65.978266679935\\
-0.294921875	65.8587124631887\\
-0.29443359375	65.7365920135551\\
-0.2939453125	65.6121090018554\\
-0.29345703125	65.4854608110038\\
-0.29296875	65.3568382702074\\
-0.29248046875	65.2264254526755\\
-0.2919921875	65.0943995317529\\
-0.29150390625	64.9609306903265\\
-0.291015625	64.8261820783728\\
-0.29052734375	64.6903098136151\\
-0.2900390625	64.5534630204187\\
-0.28955078125	64.4157839022623\\
-0.2890625	64.277407843373\\
-0.28857421875	64.1384635353792\\
-0.2880859375	63.9990731251342\\
-0.28759765625	63.8593523801544\\
-0.287109375	63.7194108684224\\
-0.28662109375	63.5793521495998\\
-0.2861328125	63.4392739749896\\
-0.28564453125	63.2992684938609\\
-0.28515625	63.1594224640209\\
-0.28466796875	63.0198174647675\\
-0.2841796875	62.8805301105856\\
-0.28369140625	62.7416322641751\\
-0.283203125	62.6031912475907\\
-0.28271484375	62.4652700504606\\
-0.2822265625	62.3279275344106\\
-0.28173828125	62.1912186329781\\
-0.28125	62.0551945464247\\
-0.28076171875	61.9199029309812\\
-0.2802734375	61.7853880821611\\
-0.27978515625	61.6516911118754\\
-0.279296875	61.5188501191562\\
-0.27880859375	61.3869003543745\\
-0.2783203125	61.2558743768944\\
-0.27783203125	61.1258022061567\\
-0.27734375	60.9967114662356\\
-0.27685546875	60.8686275239419\\
-0.2763671875	60.7415736205833\\
-0.27587890625	60.6155709975125\\
-0.275390625	60.4906390156222\\
-0.27490234375	60.3667952689522\\
-0.2744140625	60.2440556925954\\
-0.27392578125	60.1224346650943\\
-0.2734375	60.0019451055265\\
-0.27294921875	59.8825985654831\\
-0.2724609375	59.7644053161446\\
-0.27197265625	59.6473744306598\\
-0.271484375	59.5315138620321\\
-0.27099609375	59.4168305167139\\
-0.2705078125	59.3033303241091\\
-0.27001953125	59.1910183021754\\
-0.26953125	59.0798986193174\\
-0.26904296875	58.9699746527523\\
-0.2685546875	58.8612490435265\\
-0.26806640625	58.7537237483557\\
-0.267578125	58.6474000884512\\
-0.26708984375	58.5422787954932\\
-0.2666015625	58.4383600549023\\
-0.26611328125	58.3356435465553\\
-0.265625	58.2341284830844\\
-0.26513671875	58.1338136458926\\
-0.2646484375	58.0346974190125\\
-0.26416015625	57.9367778209284\\
-0.263671875	57.8400525344772\\
-0.26318359375	57.7445189349363\\
-0.2626953125	57.6501741164027\\
-0.26220703125	57.5570149165609\\
-0.26171875	57.465037939933\\
-0.26123046875	57.3742395796984\\
-0.2607421875	57.2846160381669\\
-0.26025390625	57.1961633459843\\
-0.259765625	57.1088773801448\\
-0.25927734375	57.0227538808801\\
-0.2587890625	56.9377884674922\\
-0.25830078125	56.8539766531926\\
-0.2578125	56.7713138590075\\
-0.25732421875	56.689795426805\\
-0.2568359375	56.6094166314955\\
-0.25634765625	56.5301726924569\\
-0.255859375	56.452058784231\\
-0.25537109375	56.3750700465349\\
-0.2548828125	56.2992015936287\\
-0.25439453125	56.22444852308\\
-0.25390625	56.1508059239611\\
-0.25341796875	56.0782688845136\\
-0.2529296875	56.0068324993141\\
-0.25244140625	55.9364918759714\\
-0.251953125	55.8672421413837\\
-0.25146484375	55.7990784475837\\
-0.2509765625	55.7319959771977\\
-0.25048828125	55.6659899485429\\
-0.25	55.6010556203839\\
-0.24951171875	55.537188296373\\
-0.2490234375	55.4743833291921\\
-0.24853515625	55.4126361244161\\
-0.248046875	55.3519421441158\\
-0.24755859375	55.2922969102169\\
-0.2470703125	55.2336960076317\\
-0.24658203125	55.1761350871764\\
-0.24609375	55.1196098682917\\
-0.24560546875	55.0641161415759\\
-0.2451171875	55.009649771146\\
-0.24462890625	54.9562066968373\\
-0.244140625	54.9037829362522\\
-0.24365234375	54.8523745866701\\
-0.2431640625	54.8019778268264\\
-0.24267578125	54.7525889185716\\
-0.2421875	54.7042042084176\\
-0.24169921875	54.6568201289809\\
-0.2412109375	54.6104332003299\\
-0.24072265625	54.5650400312424\\
-0.240234375	54.5206373203829\\
-0.23974609375	54.477221857404\\
-0.2392578125	54.4347905239777\\
-0.23876953125	54.3933402947657\\
-0.23828125	54.3528682383297\\
-0.23779296875	54.3133715179903\\
-0.2373046875	54.2748473926381\\
-0.23681640625	54.2372932175015\\
-0.236328125	54.200706444876\\
-0.23583984375	54.1650846248193\\
-0.2353515625	54.1304254058152\\
-0.23486328125	54.0967265354115\\
-0.234375	54.0639858608342\\
-0.23388671875	54.032201329582\\
-0.2333984375	54.0013709900042\\
-0.23291015625	53.9714929918648\\
-0.232421875	53.9425655868962\\
-0.23193359375	53.9145871293449\\
-0.2314453125	53.8875560765121\\
-0.23095703125	53.8614709892913\\
-0.23046875	53.8363305327058\\
-0.22998046875	53.8121334764478\\
-0.2294921875	53.7888786954232\\
-0.22900390625	53.766565170301\\
-0.228515625	53.7451919880721\\
-0.22802734375	53.7247583426191\\
-0.2275390625	53.7052635352973\\
-0.22705078125	53.686706975531\\
-0.2265625	53.6690881814261\\
-0.22607421875	53.652406780399\\
-0.2255859375	53.6366625098275\\
-0.22509765625	53.621855217721\\
-0.224609375	53.6079848634148\\
-0.22412109375	53.5950515182886\\
-0.2236328125	53.5830553665111\\
-0.22314453125	53.571996705813\\
-0.22265625	53.5618759482884\\
-0.22216796875	53.5526936212283\\
-0.2216796875	53.5444503679848\\
-0.22119140625	53.5371469488714\\
-0.220703125	53.5307842420969\\
-0.22021484375	53.5253632447378\\
-0.2197265625	53.5208850737481\\
-0.21923828125	53.5173509670103\\
-0.21875	53.514762284427\\
-0.21826171875	53.5131205090568\\
-0.2177734375	53.5124272482945\\
-0.21728515625	53.5126842350979\\
-0.216796875	53.5138933292627\\
-0.21630859375	53.5160565187472\\
-0.2158203125	53.519175921049\\
-0.21533203125	53.5232537846339\\
-0.21484375	53.5282924904209\\
-0.21435546875	53.5342945533236\\
-0.2138671875	53.5412626238506\\
-0.21337890625	53.5491994897658\\
-0.212890625	53.5581080778122\\
-0.21240234375	53.5679914555004\\
-0.2119140625	53.5788528329629\\
-0.21142578125	53.5906955648784\\
-0.2109375	53.6035231524666\\
-0.21044921875	53.6173392455565\\
-0.2099609375	53.6321476447308\\
-0.20947265625	53.6479523035484\\
-0.208984375	53.6647573308474\\
-0.20849609375	53.6825669931324\\
-0.2080078125	53.7013857170467\\
-0.20751953125	53.721218091935\\
-0.20703125	53.7420688724966\\
-0.20654296875	53.7639429815345\\
-0.2060546875	53.7868455128025\\
-0.20556640625	53.8107817339531\\
-0.205078125	53.8357570895915\\
-0.20458984375	53.8617772044366\\
-0.2041015625	53.888847886595\\
-0.20361328125	53.9169751309502\\
-0.203125	53.946165122673\\
-0.20263671875	53.9764242408541\\
-0.2021484375	54.0077590622666\\
-0.20166015625	54.0401763652615\\
-0.201171875	54.0736831337991\\
-0.20068359375	54.1082865616249\\
-0.2001953125	54.1439940565924\\
-0.19970703125	54.1808132451382\\
-0.19921875	54.2187519769169\\
-0.19873046875	54.2578183295997\\
-0.1982421875	54.2980206138428\\
-0.19775390625	54.3393673784339\\
-0.197265625	54.3818674156204\\
-0.19677734375	54.4255297666298\\
-0.1962890625	54.4703637273854\\
-0.19580078125	54.5163788544287\\
-0.1953125	54.5635849710536\\
-0.19482421875	54.6119921736619\\
-0.1943359375	54.661610838349\\
-0.19384765625	54.7124516277278\\
-0.193359375	54.7645254979998\\
-0.19287109375	54.8178437062869\\
-0.1923828125	54.8724178182279\\
-0.19189453125	54.9282597158567\\
-0.19140625	54.9853816057699\\
-0.19091796875	55.0437960275967\\
-0.1904296875	55.1035158627831\\
-0.18994140625	55.1645543437044\\
-0.189453125	55.2269250631181\\
-0.18896484375	55.2906419839732\\
-0.1884765625	55.3557194495895\\
-0.18798828125	55.4221721942226\\
-0.1875	55.4900153540327\\
-0.18701171875	55.5592644784725\\
-0.1865234375	55.6299355421132\\
-0.18603515625	55.7020449569279\\
-0.185546875	55.7756095850522\\
-0.18505859375	55.8506467520421\\
-0.1845703125	55.9271742606529\\
-0.18408203125	56.0052104051607\\
-0.18359375	56.0847739862514\\
-0.18310546875	56.1658843265033\\
-0.1826171875	56.2485612864893\\
-0.18212890625	56.3328252815267\\
-0.181640625	56.4186972991054\\
-0.18115234375	56.5061989170255\\
-0.1806640625	56.5953523222747\\
-0.18017578125	56.6861803306847\\
-0.1796875	56.7787064073972\\
-0.17919921875	56.8729546881828\\
-0.1787109375	56.9689500016496\\
-0.17822265625	57.0667178923842\\
-0.177734375	57.1662846450719\\
-0.17724609375	57.267677309639\\
-0.1767578125	57.3709237274691\\
-0.17626953125	57.476052558745\\
-0.17578125	57.5830933109694\\
-0.17529296875	57.6920763687213\\
-0.1748046875	57.8030330247113\\
-0.17431640625	57.9159955121948\\
-0.173828125	58.0309970388126\\
-0.17333984375	58.1480718219294\\
-0.1728515625	58.2672551255396\\
-0.17236328125	58.3885832988228\\
-0.171875	58.512093816426\\
-0.17138671875	58.6378253205598\\
-0.1708984375	58.7658176649972\\
-0.17041015625	58.8961119610678\\
-0.169921875	59.0287506257457\\
-0.16943359375	59.1637774319342\\
-0.1689453125	59.3012375610525\\
-0.16845703125	59.4411776580376\\
-0.16796875	59.5836458888761\\
-0.16748046875	59.7286920007883\\
-0.1669921875	59.8763673851886\\
-0.16650390625	60.0267251435519\\
-0.166015625	60.179820156322\\
-0.16552734375	60.3357091549958\\
-0.1650390625	60.4944507975292\\
-0.16455078125	60.6561057472033\\
-0.1640625	60.8207367551019\\
-0.16357421875	60.9884087463412\\
-0.1630859375	61.1591889102036\\
-0.16259765625	61.3331467943149\\
-0.162109375	61.5103544030072\\
-0.16162109375	61.6908862999998\\
-0.1611328125	61.8748197155229\\
-0.16064453125	62.0622346579901\\
-0.16015625	62.2532140303176\\
-0.15966796875	62.4478437509504\\
-0.1591796875	62.6462128796377\\
-0.15869140625	62.8484137479532\\
-0.158203125	63.0545420945073\\
-0.15771484375	63.2646972047411\\
-0.1572265625	63.4789820551107\\
-0.15673828125	63.6975034613776\\
-0.15625	63.9203722306089\\
-0.15576171875	64.1477033163451\\
-0.1552734375	64.3796159762225\\
-0.15478515625	64.6162339311239\\
-0.154296875	64.8576855246815\\
-0.15380859375	65.1041038816346\\
-0.1533203125	65.3556270631759\\
-0.15283203125	65.6123982169538\\
-0.15234375	65.8745657188477\\
-0.15185546875	66.1422833029553\\
-0.1513671875	66.4157101754186\\
-0.15087890625	66.6950111067195\\
-0.150390625	66.9803564958858\\
-0.14990234375	67.2719223985904\\
-0.1494140625	67.5698905093674\\
-0.14892578125	67.8744480860387\\
-0.1484375	68.1857878018527\\
-0.14794921875	68.5041075077109\\
-0.1474609375	68.8296098830394\\
-0.14697265625	69.1625019492612\\
-0.146484375	69.5029944142234\\
-0.14599609375	69.8513008091546\\
-0.1455078125	70.207636371493\\
-0.14501953125	70.5722166169868\\
-0.14453125	70.9452555324143\\
-0.14404296875	71.3269633057387\\
-0.1435546875	71.7175434930023\\
-0.14306640625	72.1171895002479\\
-0.142578125	72.5260802336243\\
-0.14208984375	72.9443747409826\\
-0.1416015625	73.3722056330579\\
-0.14111328125	73.8096710312468\\
-0.140625	74.2568247417124\\
-0.14013671875	74.7136643022034\\
-0.1396484375	75.1801164894337\\
-0.13916015625	75.656019813279\\
-0.138671875	76.1411034635714\\
-0.13818359375	76.6349621230734\\
-0.1376953125	77.1370260280834\\
-0.13720703125	77.6465256643076\\
-0.13671875	78.1624505573062\\
-0.13623046875	78.6835017930683\\
-0.1357421875	79.2080382384838\\
-0.13525390625	79.7340169931676\\
-0.134765625	80.2589294768886\\
-0.13427734375	80.7797358301952\\
-0.1337890625	81.2928020546988\\
-0.13330078125	81.7938465659155\\
-0.1328125	82.2779054825286\\
-0.13232421875	82.7393287373976\\
-0.1318359375	83.1718213780123\\
-0.13134765625	83.5685452810506\\
-0.130859375	83.9222946823868\\
-0.13037109375	84.2257531155962\\
-0.1298828125	84.4718287442757\\
-0.12939453125	84.654050105863\\
-0.12890625	84.7669873433763\\
-0.12841796875	84.8066495110044\\
-0.1279296875	84.7708019416255\\
-0.12744140625	84.6591532685515\\
-0.126953125	84.4733802555628\\
-0.12646484375	84.2169860936039\\
-0.1259765625	83.8950165861283\\
-0.12548828125	83.5136804484178\\
-0.125	83.0799294746556\\
-0.12451171875	82.6010509728914\\
-0.1240234375	82.0843121277406\\
-0.12353515625	81.536679192897\\
-0.123046875	80.9646185254473\\
-0.12255859375	80.373974467261\\
-0.1220703125	79.7699119088035\\
-0.12158203125	79.1569085063378\\
-0.12109375	78.5387817348737\\
-0.12060546875	77.9187379382333\\
-0.1201171875	77.2994332351409\\
-0.11962890625	76.6830388654719\\
-0.119140625	76.0713059425094\\
-0.11865234375	75.4656264730859\\
-0.1181640625	74.8670889137786\\
-0.11767578125	74.2765275123045\\
-0.1171875	73.6945653268851\\
-0.11669921875	73.1216512093113\\
-0.1162109375	72.5580912536888\\
-0.11572265625	72.0040753105118\\
-0.115234375	71.4596991877977\\
-0.11474609375	70.9249831378474\\
-0.1142578125	70.3998871804286\\
-0.11376953125	69.8843237542021\\
-0.11328125	69.3781681264013\\
-0.11279296875	68.8812669310586\\
-0.1123046875	68.3934451510577\\
-0.11181640625	67.9145118102057\\
-0.111328125	67.4442645986316\\
-0.11083984375	66.9824936179392\\
-0.1103515625	66.5289844011812\\
-0.10986328125	66.0835203362722\\
-0.109375	65.6458845993047\\
-0.10888671875	65.215861685751\\
-0.1083984375	64.7932386121758\\
-0.10791015625	64.3778058483648\\
-0.107421875	63.9693580292371\\
-0.10693359375	63.5676944872191\\
-0.1064453125	63.1726196385801\\
-0.10595703125	62.7839432513024\\
-0.10546875	62.4014806171844\\
-0.10498046875	62.0250526468463\\
-0.1044921875	61.6544859029895\\
-0.10400390625	61.2896125845234\\
-0.103515625	60.9302704719222\\
-0.10302734375	60.5763028423011\\
-0.1025390625	60.2275583611791\\
-0.10205078125	59.8838909566169\\
-0.1015625	59.5451596803752\\
-0.10107421875	59.2112285598792\\
-0.1005859375	58.8819664440568\\
-0.10009765625	58.5572468455307\\
-0.099609375	58.2369477811627\\
-0.09912109375	57.9209516125435\\
-0.0986328125	57.6091448876953\\
-0.09814453125	57.3014181849799\\
-0.09765625	56.9976659599851\\
-0.09716796875	56.697786395975\\
-0.0966796875	56.4016812583432\\
-0.09619140625	56.1092557533833\\
-0.095703125	55.8204183915915\\
-0.09521484375	55.5350808556348\\
-0.0947265625	55.2531578730542\\
-0.09423828125	54.9745670937159\\
-0.09375	54.6992289719887\\
-0.09326171875	54.4270666535831\\
-0.0927734375	54.1580058669708\\
-0.09228515625	53.8919748192782\\
-0.091796875	53.6289040965333\\
-0.09130859375	53.368726568137\\
-0.0908203125	53.1113772954195\\
-0.09033203125	52.8567934441386\\
-0.08984375	52.6049142007746\\
-0.08935546875	52.3556806924758\\
-0.0888671875	52.1090359105071\\
-0.08837890625	51.8649246370586\\
-0.087890625	51.6232933752712\\
-0.08740234375	51.3840902823413\\
-0.0869140625	51.1472651055687\\
-0.08642578125	50.9127691212181\\
-0.0859375	50.6805550760651\\
-0.08544921875	50.4505771315072\\
-0.0849609375	50.2227908101205\\
-0.08447265625	49.9971529445497\\
-0.083984375	49.7736216286234\\
-0.08349609375	49.5521561705899\\
-0.0830078125	49.3327170483768\\
-0.08251953125	49.1152658667762\\
-0.08203125	48.8997653164678\\
-0.08154296875	48.6861791347928\\
-0.0810546875	48.4744720681961\\
-0.08056640625	48.2646098362585\\
-0.080078125	48.0565590972445\\
-0.07958984375	47.850287415095\\
-0.0791015625	47.6457632277964\\
-0.07861328125	47.4429558170633\\
-0.078125	47.2418352792716\\
-0.07763671875	47.0423724975859\\
-0.0771484375	46.8445391152258\\
-0.07666015625	46.6483075098159\\
-0.076171875	46.4536507687738\\
-0.07568359375	46.2605426656849\\
-0.0751953125	46.068957637621\\
-0.07470703125	45.8788707633593\\
-0.07421875	45.6902577424606\\
-0.07373046875	45.5030948751687\\
-0.0732421875	45.3173590430933\\
-0.07275390625	45.1330276906421\\
-0.072265625	44.9500788071687\\
-0.07177734375	44.7684909098037\\
-0.0712890625	44.5882430269403\\
-0.07080078125	44.4093146823447\\
-0.0703125	44.2316858798641\\
-0.06982421875	44.0553370887072\\
-0.0693359375	43.8802492292711\\
-0.06884765625	43.7064036594923\\
-0.068359375	43.5337821616987\\
-0.06787109375	43.362366929942\\
-0.0673828125	43.1921405577891\\
-0.06689453125	43.0230860265539\\
-0.06640625	42.8551866939514\\
-0.06591796875	42.688426283155\\
-0.0654296875	42.5227888722429\\
-0.06494140625	42.3582588840145\\
-0.064453125	42.1948210761638\\
-0.06396484375	42.0324605317945\\
-0.0634765625	41.8711626502629\\
-0.06298828125	41.7109131383359\\
-0.0625	41.5516980016503\\
-0.06201171875	41.3935035364628\\
-0.0615234375	41.2363163216781\\
-0.06103515625	41.0801232111447\\
-0.060546875	40.9249113262075\\
-0.06005859375	40.7706680485077\\
-0.0595703125	40.6173810130198\\
-0.05908203125	40.4650381013167\\
-0.05859375	40.313627435055\\
-0.05810546875	40.1631373696699\\
-0.0576171875	40.013556488275\\
-0.05712890625	39.8648735957556\\
-0.056640625	39.7170777130516\\
-0.05615234375	39.5701580716204\\
-0.0556640625	39.4241041080744\\
-0.05517578125	39.2789054589863\\
-0.0546875	39.134551955856\\
-0.05419921875	38.9910336202332\\
-0.0537109375	38.8483406589902\\
-0.05322265625	38.7064634597393\\
-0.052734375	38.5653925863892\\
-0.05224609375	38.4251187748363\\
-0.0517578125	38.2856329287857\\
-0.05126953125	38.1469261156968\\
-0.05078125	38.0089895628499\\
-0.05029296875	37.8718146535285\\
-0.0498046875	37.7353929233152\\
-0.04931640625	37.5997160564949\\
-0.048828125	37.4647758825631\\
-0.04833984375	37.3305643728351\\
-0.0478515625	37.1970736371534\\
-0.04736328125	37.0642959206888\\
-0.046875	36.9322236008325\\
-0.04638671875	36.8008491841773\\
-0.0458984375	36.6701653035827\\
-0.04541015625	36.5401647153227\\
-0.044921875	36.4108402963135\\
-0.04443359375	36.2821850414179\\
-0.0439453125	36.1541920608236\\
-0.04345703125	36.0268545774944\\
-0.04296875	35.90016592469\\
-0.04248046875	35.7741195435544\\
-0.0419921875	35.6487089807682\\
-0.04150390625	35.5239278862648\\
-0.041015625	35.3997700110079\\
-0.04052734375	35.2762292048275\\
-0.0400390625	35.1532994143134\\
-0.03955078125	35.030974680765\\
-0.0390625	34.9092491381935\\
-0.03857421875	34.7881170113774\\
-0.0380859375	34.667572613968\\
-0.03759765625	34.5476103466436\\
-0.037109375	34.4282246953118\\
-0.03662109375	34.3094102293571\\
-0.0361328125	34.1911615999338\\
-0.03564453125	34.0734735383011\\
-0.03515625	33.9563408542014\\
-0.03466796875	33.8397584342788\\
-0.0341796875	33.7237212405363\\
-0.03369140625	33.6082243088325\\
-0.033203125	33.4932627474154\\
-0.03271484375	33.3788317354907\\
-0.0322265625	33.2649265218276\\
-0.03173828125	33.1515424233964\\
-0.03125	33.0386748240405\\
-0.03076171875	32.92631917318\\
-0.0302734375	32.8144709845466\\
-0.02978515625	32.7031258349484\\
-0.029296875	32.5922793630645\\
-0.02880859375	32.481927268268\\
-0.0283203125	32.3720653094772\\
-0.02783203125	32.2626893040327\\
-0.02734375	32.1537951266023\\
-0.02685546875	32.0453787081097\\
-0.0263671875	31.9374360346894\\
-0.02587890625	31.8299631466651\\
-0.025390625	31.7229561375516\\
-0.02490234375	31.6164111530793\\
-0.0244140625	31.5103243902413\\
-0.02392578125	31.4046920963623\\
-0.0234375	31.2995105681873\\
-0.02294921875	31.1947761509924\\
-0.0224609375	31.090485237714\\
-0.02197265625	30.9866342680986\\
-0.021484375	30.8832197278705\\
-0.02099609375	30.7802381479184\\
-0.0205078125	30.6776861034997\\
-0.02001953125	30.5755602134623\\
-0.01953125	30.4738571394831\\
-0.01904296875	30.3725735853235\\
-0.0185546875	30.2717062961003\\
-0.01806640625	30.1712520575728\\
-0.017578125	30.0712076954453\\
-0.01708984375	29.9715700746836\\
-0.0166015625	29.8723360988473\\
-0.01611328125	29.7735027094349\\
-0.015625	29.6750668852436\\
-0.01513671875	29.5770256417424\\
-0.0146484375	29.4793760304577\\
-0.01416015625	29.3821151383724\\
-0.013671875	29.2852400873374\\
-0.01318359375	29.1887480334942\\
-0.0126953125	29.092636166711\\
-0.01220703125	28.996901710029\\
-0.01171875	28.9015419191205\\
-0.01123046875	28.8065540817583\\
-0.0107421875	28.7119355172951\\
-0.01025390625	28.617683576154\\
-0.009765625	28.5237956393288\\
-0.00927734375	28.4302691178947\\
-0.0087890625	28.3371014525283\\
-0.00830078125	28.2442901130376\\
-0.0078125	28.1518325979007\\
-0.00732421875	28.0597264338137\\
-0.0068359375	27.9679691752483\\
-0.00634765625	27.8765584040166\\
-0.005859375	27.7854917288455\\
-0.00537109375	27.6947667849588\\
-0.0048828125	27.6043812336675\\
-0.00439453125	27.5143327619681\\
-0.00390625	27.4246190821483\\
-0.00341796875	27.3352379314002\\
-0.0029296875	27.2461870714414\\
-0.00244140625	27.1574642881426\\
-0.001953125	27.0690673911625\\
-0.00146484375	26.9809942135897\\
-0.0009765625	26.8932426115909\\
-0.00048828125	26.8058104640661\\
0	26.7186956723099\\
0.00048828125	26.8058104640661\\
0.0009765625	26.8932426115909\\
0.00146484375	26.9809942135897\\
0.001953125	27.0690673911625\\
0.00244140625	27.1574642881426\\
0.0029296875	27.2461870714414\\
0.00341796875	27.3352379314002\\
0.00390625	27.4246190821483\\
0.00439453125	27.5143327619681\\
0.0048828125	27.6043812336675\\
0.00537109375	27.6947667849588\\
0.005859375	27.7854917288455\\
0.00634765625	27.8765584040166\\
0.0068359375	27.9679691752483\\
0.00732421875	28.0597264338137\\
0.0078125	28.1518325979007\\
0.00830078125	28.2442901130376\\
0.0087890625	28.3371014525283\\
0.00927734375	28.4302691178947\\
0.009765625	28.5237956393288\\
0.01025390625	28.617683576154\\
0.0107421875	28.7119355172951\\
0.01123046875	28.8065540817583\\
0.01171875	28.9015419191205\\
0.01220703125	28.996901710029\\
0.0126953125	29.092636166711\\
0.01318359375	29.1887480334942\\
0.013671875	29.2852400873374\\
0.01416015625	29.3821151383724\\
0.0146484375	29.4793760304577\\
0.01513671875	29.5770256417424\\
0.015625	29.6750668852436\\
0.01611328125	29.7735027094349\\
0.0166015625	29.8723360988473\\
0.01708984375	29.9715700746836\\
0.017578125	30.0712076954453\\
0.01806640625	30.1712520575728\\
0.0185546875	30.2717062961003\\
0.01904296875	30.3725735853235\\
0.01953125	30.4738571394831\\
0.02001953125	30.5755602134623\\
0.0205078125	30.6776861034997\\
0.02099609375	30.7802381479184\\
0.021484375	30.8832197278705\\
0.02197265625	30.9866342680986\\
0.0224609375	31.090485237714\\
0.02294921875	31.1947761509924\\
0.0234375	31.2995105681873\\
0.02392578125	31.4046920963623\\
0.0244140625	31.5103243902413\\
0.02490234375	31.6164111530793\\
0.025390625	31.7229561375516\\
0.02587890625	31.8299631466651\\
0.0263671875	31.9374360346894\\
0.02685546875	32.0453787081097\\
0.02734375	32.1537951266023\\
0.02783203125	32.2626893040327\\
0.0283203125	32.3720653094772\\
0.02880859375	32.481927268268\\
0.029296875	32.5922793630645\\
0.02978515625	32.7031258349484\\
0.0302734375	32.8144709845466\\
0.03076171875	32.92631917318\\
0.03125	33.0386748240405\\
0.03173828125	33.1515424233964\\
0.0322265625	33.2649265218276\\
0.03271484375	33.3788317354907\\
0.033203125	33.4932627474154\\
0.03369140625	33.6082243088325\\
0.0341796875	33.7237212405363\\
0.03466796875	33.8397584342788\\
0.03515625	33.9563408542014\\
0.03564453125	34.0734735383011\\
0.0361328125	34.1911615999338\\
0.03662109375	34.3094102293571\\
0.037109375	34.4282246953118\\
0.03759765625	34.5476103466436\\
0.0380859375	34.667572613968\\
0.03857421875	34.7881170113774\\
0.0390625	34.9092491381935\\
0.03955078125	35.030974680765\\
0.0400390625	35.1532994143134\\
0.04052734375	35.2762292048275\\
0.041015625	35.3997700110079\\
0.04150390625	35.5239278862648\\
0.0419921875	35.6487089807682\\
0.04248046875	35.7741195435544\\
0.04296875	35.90016592469\\
0.04345703125	36.0268545774944\\
0.0439453125	36.1541920608236\\
0.04443359375	36.2821850414179\\
0.044921875	36.4108402963135\\
0.04541015625	36.5401647153227\\
0.0458984375	36.6701653035827\\
0.04638671875	36.8008491841773\\
0.046875	36.9322236008325\\
0.04736328125	37.0642959206888\\
0.0478515625	37.1970736371534\\
0.04833984375	37.3305643728351\\
0.048828125	37.4647758825631\\
0.04931640625	37.5997160564949\\
0.0498046875	37.7353929233152\\
0.05029296875	37.8718146535285\\
0.05078125	38.0089895628499\\
0.05126953125	38.1469261156968\\
0.0517578125	38.2856329287857\\
0.05224609375	38.4251187748363\\
0.052734375	38.5653925863892\\
0.05322265625	38.7064634597393\\
0.0537109375	38.8483406589902\\
0.05419921875	38.9910336202332\\
0.0546875	39.134551955856\\
0.05517578125	39.2789054589863\\
0.0556640625	39.4241041080744\\
0.05615234375	39.5701580716204\\
0.056640625	39.7170777130516\\
0.05712890625	39.8648735957556\\
0.0576171875	40.013556488275\\
0.05810546875	40.1631373696699\\
0.05859375	40.313627435055\\
0.05908203125	40.4650381013167\\
0.0595703125	40.6173810130198\\
0.06005859375	40.7706680485077\\
0.060546875	40.9249113262075\\
0.06103515625	41.0801232111447\\
0.0615234375	41.2363163216781\\
0.06201171875	41.3935035364628\\
0.0625	41.5516980016503\\
0.06298828125	41.7109131383359\\
0.0634765625	41.8711626502629\\
0.06396484375	42.0324605317945\\
0.064453125	42.1948210761638\\
0.06494140625	42.3582588840145\\
0.0654296875	42.5227888722429\\
0.06591796875	42.688426283155\\
0.06640625	42.8551866939514\\
0.06689453125	43.0230860265539\\
0.0673828125	43.1921405577891\\
0.06787109375	43.362366929942\\
0.068359375	43.5337821616987\\
0.06884765625	43.7064036594923\\
0.0693359375	43.8802492292711\\
0.06982421875	44.0553370887072\\
0.0703125	44.2316858798641\\
0.07080078125	44.4093146823447\\
0.0712890625	44.5882430269403\\
0.07177734375	44.7684909098037\\
0.072265625	44.9500788071687\\
0.07275390625	45.1330276906421\\
0.0732421875	45.3173590430933\\
0.07373046875	45.5030948751687\\
0.07421875	45.6902577424606\\
0.07470703125	45.8788707633593\\
0.0751953125	46.068957637621\\
0.07568359375	46.2605426656849\\
0.076171875	46.4536507687738\\
0.07666015625	46.6483075098159\\
0.0771484375	46.8445391152258\\
0.07763671875	47.0423724975859\\
0.078125	47.2418352792716\\
0.07861328125	47.4429558170633\\
0.0791015625	47.6457632277964\\
0.07958984375	47.850287415095\\
0.080078125	48.0565590972445\\
0.08056640625	48.2646098362585\\
0.0810546875	48.4744720681961\\
0.08154296875	48.6861791347928\\
0.08203125	48.8997653164678\\
0.08251953125	49.1152658667762\\
0.0830078125	49.3327170483768\\
0.08349609375	49.5521561705899\\
0.083984375	49.7736216286234\\
0.08447265625	49.9971529445497\\
0.0849609375	50.2227908101205\\
0.08544921875	50.4505771315072\\
0.0859375	50.6805550760651\\
0.08642578125	50.9127691212181\\
0.0869140625	51.1472651055687\\
0.08740234375	51.3840902823413\\
0.087890625	51.6232933752712\\
0.08837890625	51.8649246370586\\
0.0888671875	52.1090359105071\\
0.08935546875	52.3556806924758\\
0.08984375	52.6049142007746\\
0.09033203125	52.8567934441386\\
0.0908203125	53.1113772954195\\
0.09130859375	53.368726568137\\
0.091796875	53.6289040965333\\
0.09228515625	53.8919748192782\\
0.0927734375	54.1580058669708\\
0.09326171875	54.4270666535831\\
0.09375	54.6992289719887\\
0.09423828125	54.9745670937159\\
0.0947265625	55.2531578730542\\
0.09521484375	55.5350808556348\\
0.095703125	55.8204183915915\\
0.09619140625	56.1092557533833\\
0.0966796875	56.4016812583432\\
0.09716796875	56.697786395975\\
0.09765625	56.9976659599851\\
0.09814453125	57.3014181849799\\
0.0986328125	57.6091448876953\\
0.09912109375	57.9209516125435\\
0.099609375	58.2369477811627\\
0.10009765625	58.5572468455307\\
0.1005859375	58.8819664440568\\
0.10107421875	59.2112285598792\\
0.1015625	59.5451596803752\\
0.10205078125	59.8838909566169\\
0.1025390625	60.2275583611791\\
0.10302734375	60.5763028423011\\
0.103515625	60.9302704719222\\
0.10400390625	61.2896125845234\\
0.1044921875	61.6544859029895\\
0.10498046875	62.0250526468463\\
0.10546875	62.4014806171844\\
0.10595703125	62.7839432513024\\
0.1064453125	63.1726196385801\\
0.10693359375	63.5676944872191\\
0.107421875	63.9693580292371\\
0.10791015625	64.3778058483648\\
0.1083984375	64.7932386121758\\
0.10888671875	65.215861685751\\
0.109375	65.6458845993047\\
0.10986328125	66.0835203362722\\
0.1103515625	66.5289844011812\\
0.11083984375	66.9824936179392\\
0.111328125	67.4442645986316\\
0.11181640625	67.9145118102057\\
0.1123046875	68.3934451510577\\
0.11279296875	68.8812669310586\\
0.11328125	69.3781681264013\\
0.11376953125	69.8843237542021\\
0.1142578125	70.3998871804286\\
0.11474609375	70.9249831378474\\
0.115234375	71.4596991877977\\
0.11572265625	72.0040753105118\\
0.1162109375	72.5580912536888\\
0.11669921875	73.1216512093113\\
0.1171875	73.6945653268851\\
0.11767578125	74.2765275123045\\
0.1181640625	74.8670889137786\\
0.11865234375	75.4656264730859\\
0.119140625	76.0713059425094\\
0.11962890625	76.6830388654719\\
0.1201171875	77.2994332351409\\
0.12060546875	77.9187379382333\\
0.12109375	78.5387817348737\\
0.12158203125	79.1569085063378\\
0.1220703125	79.7699119088035\\
0.12255859375	80.373974467261\\
0.123046875	80.9646185254473\\
0.12353515625	81.536679192897\\
0.1240234375	82.0843121277406\\
0.12451171875	82.6010509728914\\
0.125	83.0799294746556\\
0.12548828125	83.5136804484178\\
0.1259765625	83.8950165861283\\
0.12646484375	84.2169860936039\\
0.126953125	84.4733802555628\\
0.12744140625	84.6591532685515\\
0.1279296875	84.7708019416255\\
0.12841796875	84.8066495110044\\
0.12890625	84.7669873433763\\
0.12939453125	84.654050105863\\
0.1298828125	84.4718287442757\\
0.13037109375	84.2257531155962\\
0.130859375	83.9222946823868\\
0.13134765625	83.5685452810506\\
0.1318359375	83.1718213780123\\
0.13232421875	82.7393287373976\\
0.1328125	82.2779054825286\\
0.13330078125	81.7938465659155\\
0.1337890625	81.2928020546988\\
0.13427734375	80.7797358301952\\
0.134765625	80.2589294768886\\
0.13525390625	79.7340169931676\\
0.1357421875	79.2080382384838\\
0.13623046875	78.6835017930683\\
0.13671875	78.1624505573062\\
0.13720703125	77.6465256643076\\
0.1376953125	77.1370260280834\\
0.13818359375	76.6349621230734\\
0.138671875	76.1411034635714\\
0.13916015625	75.656019813279\\
0.1396484375	75.1801164894337\\
0.14013671875	74.7136643022034\\
0.140625	74.2568247417124\\
0.14111328125	73.8096710312468\\
0.1416015625	73.3722056330579\\
0.14208984375	72.9443747409826\\
0.142578125	72.5260802336243\\
0.14306640625	72.1171895002479\\
0.1435546875	71.7175434930023\\
0.14404296875	71.3269633057387\\
0.14453125	70.9452555324143\\
0.14501953125	70.5722166169868\\
0.1455078125	70.207636371493\\
0.14599609375	69.8513008091546\\
0.146484375	69.5029944142234\\
0.14697265625	69.1625019492612\\
0.1474609375	68.8296098830394\\
0.14794921875	68.5041075077109\\
0.1484375	68.1857878018527\\
0.14892578125	67.8744480860387\\
0.1494140625	67.5698905093674\\
0.14990234375	67.2719223985904\\
0.150390625	66.9803564958858\\
0.15087890625	66.6950111067195\\
0.1513671875	66.4157101754186\\
0.15185546875	66.1422833029553\\
0.15234375	65.8745657188477\\
0.15283203125	65.6123982169538\\
0.1533203125	65.3556270631759\\
0.15380859375	65.1041038816346\\
0.154296875	64.8576855246815\\
0.15478515625	64.6162339311239\\
0.1552734375	64.3796159762225\\
0.15576171875	64.1477033163451\\
0.15625	63.9203722306089\\
0.15673828125	63.6975034613776\\
0.1572265625	63.4789820551107\\
0.15771484375	63.2646972047411\\
0.158203125	63.0545420945073\\
0.15869140625	62.8484137479532\\
0.1591796875	62.6462128796377\\
0.15966796875	62.4478437509504\\
0.16015625	62.2532140303176\\
0.16064453125	62.0622346579901\\
0.1611328125	61.8748197155229\\
0.16162109375	61.6908862999998\\
0.162109375	61.5103544030072\\
0.16259765625	61.3331467943149\\
0.1630859375	61.1591889102036\\
0.16357421875	60.9884087463412\\
0.1640625	60.8207367551019\\
0.16455078125	60.6561057472033\\
0.1650390625	60.4944507975292\\
0.16552734375	60.3357091549958\\
0.166015625	60.179820156322\\
0.16650390625	60.0267251435519\\
0.1669921875	59.8763673851886\\
0.16748046875	59.7286920007883\\
0.16796875	59.5836458888761\\
0.16845703125	59.4411776580376\\
0.1689453125	59.3012375610525\\
0.16943359375	59.1637774319342\\
0.169921875	59.0287506257457\\
0.17041015625	58.8961119610678\\
0.1708984375	58.7658176649972\\
0.17138671875	58.6378253205598\\
0.171875	58.512093816426\\
0.17236328125	58.3885832988228\\
0.1728515625	58.2672551255396\\
0.17333984375	58.1480718219294\\
0.173828125	58.0309970388126\\
0.17431640625	57.9159955121948\\
0.1748046875	57.8030330247113\\
0.17529296875	57.6920763687213\\
0.17578125	57.5830933109694\\
0.17626953125	57.476052558745\\
0.1767578125	57.3709237274691\\
0.17724609375	57.267677309639\\
0.177734375	57.1662846450719\\
0.17822265625	57.0667178923842\\
0.1787109375	56.9689500016496\\
0.17919921875	56.8729546881828\\
0.1796875	56.7787064073972\\
0.18017578125	56.6861803306847\\
0.1806640625	56.5953523222747\\
0.18115234375	56.5061989170255\\
0.181640625	56.4186972991054\\
0.18212890625	56.3328252815267\\
0.1826171875	56.2485612864893\\
0.18310546875	56.1658843265033\\
0.18359375	56.0847739862514\\
0.18408203125	56.0052104051607\\
0.1845703125	55.9271742606529\\
0.18505859375	55.8506467520421\\
0.185546875	55.7756095850522\\
0.18603515625	55.7020449569279\\
0.1865234375	55.6299355421132\\
0.18701171875	55.5592644784725\\
0.1875	55.4900153540327\\
0.18798828125	55.4221721942226\\
0.1884765625	55.3557194495895\\
0.18896484375	55.2906419839732\\
0.189453125	55.2269250631181\\
0.18994140625	55.1645543437044\\
0.1904296875	55.1035158627831\\
0.19091796875	55.0437960275967\\
0.19140625	54.9853816057699\\
0.19189453125	54.9282597158567\\
0.1923828125	54.8724178182279\\
0.19287109375	54.8178437062869\\
0.193359375	54.7645254979998\\
0.19384765625	54.7124516277278\\
0.1943359375	54.661610838349\\
0.19482421875	54.6119921736619\\
0.1953125	54.5635849710536\\
0.19580078125	54.5163788544287\\
0.1962890625	54.4703637273854\\
0.19677734375	54.4255297666298\\
0.197265625	54.3818674156204\\
0.19775390625	54.3393673784339\\
0.1982421875	54.2980206138428\\
0.19873046875	54.2578183295997\\
0.19921875	54.2187519769169\\
0.19970703125	54.1808132451382\\
0.2001953125	54.1439940565924\\
0.20068359375	54.1082865616249\\
0.201171875	54.0736831337991\\
0.20166015625	54.0401763652615\\
0.2021484375	54.0077590622666\\
0.20263671875	53.9764242408541\\
0.203125	53.946165122673\\
0.20361328125	53.9169751309502\\
0.2041015625	53.888847886595\\
0.20458984375	53.8617772044366\\
0.205078125	53.8357570895915\\
0.20556640625	53.8107817339531\\
0.2060546875	53.7868455128025\\
0.20654296875	53.7639429815345\\
0.20703125	53.7420688724966\\
0.20751953125	53.721218091935\\
0.2080078125	53.7013857170467\\
0.20849609375	53.6825669931324\\
0.208984375	53.6647573308474\\
0.20947265625	53.6479523035484\\
0.2099609375	53.6321476447308\\
0.21044921875	53.6173392455565\\
0.2109375	53.6035231524666\\
0.21142578125	53.5906955648784\\
0.2119140625	53.5788528329629\\
0.21240234375	53.5679914555004\\
0.212890625	53.5581080778122\\
0.21337890625	53.5491994897658\\
0.2138671875	53.5412626238506\\
0.21435546875	53.5342945533236\\
0.21484375	53.5282924904209\\
0.21533203125	53.5232537846339\\
0.2158203125	53.519175921049\\
0.21630859375	53.5160565187472\\
0.216796875	53.5138933292627\\
0.21728515625	53.5126842350979\\
0.2177734375	53.5124272482945\\
0.21826171875	53.5131205090568\\
0.21875	53.514762284427\\
0.21923828125	53.5173509670103\\
0.2197265625	53.5208850737481\\
0.22021484375	53.5253632447378\\
0.220703125	53.5307842420969\\
0.22119140625	53.5371469488714\\
0.2216796875	53.5444503679848\\
0.22216796875	53.5526936212283\\
0.22265625	53.5618759482884\\
0.22314453125	53.571996705813\\
0.2236328125	53.5830553665111\\
0.22412109375	53.5950515182886\\
0.224609375	53.6079848634148\\
0.22509765625	53.621855217721\\
0.2255859375	53.6366625098275\\
0.22607421875	53.652406780399\\
0.2265625	53.6690881814261\\
0.22705078125	53.686706975531\\
0.2275390625	53.7052635352973\\
0.22802734375	53.7247583426191\\
0.228515625	53.7451919880721\\
0.22900390625	53.766565170301\\
0.2294921875	53.7888786954232\\
0.22998046875	53.8121334764478\\
0.23046875	53.8363305327058\\
0.23095703125	53.8614709892913\\
0.2314453125	53.8875560765121\\
0.23193359375	53.9145871293449\\
0.232421875	53.9425655868962\\
0.23291015625	53.9714929918648\\
0.2333984375	54.0013709900042\\
0.23388671875	54.032201329582\\
0.234375	54.0639858608342\\
0.23486328125	54.0967265354115\\
0.2353515625	54.1304254058152\\
0.23583984375	54.1650846248193\\
0.236328125	54.200706444876\\
0.23681640625	54.2372932175015\\
0.2373046875	54.2748473926381\\
0.23779296875	54.3133715179903\\
0.23828125	54.3528682383297\\
0.23876953125	54.3933402947657\\
0.2392578125	54.4347905239777\\
0.23974609375	54.477221857404\\
0.240234375	54.5206373203829\\
0.24072265625	54.5650400312424\\
0.2412109375	54.6104332003299\\
0.24169921875	54.6568201289809\\
0.2421875	54.7042042084176\\
0.24267578125	54.7525889185716\\
0.2431640625	54.8019778268264\\
0.24365234375	54.8523745866701\\
0.244140625	54.9037829362522\\
0.24462890625	54.9562066968373\\
0.2451171875	55.009649771146\\
0.24560546875	55.0641161415759\\
0.24609375	55.1196098682917\\
0.24658203125	55.1761350871764\\
0.2470703125	55.2336960076317\\
0.24755859375	55.2922969102169\\
0.248046875	55.3519421441158\\
0.24853515625	55.4126361244161\\
0.2490234375	55.4743833291921\\
0.24951171875	55.537188296373\\
0.25	55.6010556203839\\
0.25048828125	55.6659899485429\\
0.2509765625	55.7319959771977\\
0.25146484375	55.7990784475837\\
0.251953125	55.8672421413837\\
0.25244140625	55.9364918759714\\
0.2529296875	56.0068324993141\\
0.25341796875	56.0782688845136\\
0.25390625	56.1508059239611\\
0.25439453125	56.22444852308\\
0.2548828125	56.2992015936287\\
0.25537109375	56.3750700465349\\
0.255859375	56.452058784231\\
0.25634765625	56.5301726924569\\
0.2568359375	56.6094166314955\\
0.25732421875	56.689795426805\\
0.2578125	56.7713138590075\\
0.25830078125	56.8539766531926\\
0.2587890625	56.9377884674922\\
0.25927734375	57.0227538808801\\
0.259765625	57.1088773801448\\
0.26025390625	57.1961633459843\\
0.2607421875	57.2846160381669\\
0.26123046875	57.3742395796984\\
0.26171875	57.465037939933\\
0.26220703125	57.5570149165609\\
0.2626953125	57.6501741164027\\
0.26318359375	57.7445189349363\\
0.263671875	57.8400525344772\\
0.26416015625	57.9367778209284\\
0.2646484375	58.0346974190125\\
0.26513671875	58.1338136458926\\
0.265625	58.2341284830844\\
0.26611328125	58.3356435465553\\
0.2666015625	58.4383600549023\\
0.26708984375	58.5422787954932\\
0.267578125	58.6474000884512\\
0.26806640625	58.7537237483557\\
0.2685546875	58.8612490435265\\
0.26904296875	58.9699746527523\\
0.26953125	59.0798986193174\\
0.27001953125	59.1910183021754\\
0.2705078125	59.3033303241091\\
0.27099609375	59.4168305167139\\
0.271484375	59.5315138620321\\
0.27197265625	59.6473744306598\\
0.2724609375	59.7644053161446\\
0.27294921875	59.8825985654831\\
0.2734375	60.0019451055265\\
0.27392578125	60.1224346650943\\
0.2744140625	60.2440556925954\\
0.27490234375	60.3667952689522\\
0.275390625	60.4906390156222\\
0.27587890625	60.6155709975125\\
0.2763671875	60.7415736205833\\
0.27685546875	60.8686275239419\\
0.27734375	60.9967114662356\\
0.27783203125	61.1258022061567\\
0.2783203125	61.2558743768944\\
0.27880859375	61.3869003543745\\
0.279296875	61.5188501191562\\
0.27978515625	61.6516911118754\\
0.2802734375	61.7853880821611\\
0.28076171875	61.9199029309812\\
0.28125	62.0551945464247\\
0.28173828125	62.1912186329781\\
0.2822265625	62.3279275344106\\
0.28271484375	62.4652700504606\\
0.283203125	62.6031912475907\\
0.28369140625	62.7416322641751\\
0.2841796875	62.8805301105856\\
0.28466796875	63.0198174647675\\
0.28515625	63.1594224640209\\
0.28564453125	63.2992684938609\\
0.2861328125	63.4392739749896\\
0.28662109375	63.5793521495998\\
0.287109375	63.7194108684224\\
0.28759765625	63.8593523801544\\
0.2880859375	63.9990731251342\\
0.28857421875	64.1384635353792\\
0.2890625	64.277407843373\\
0.28955078125	64.4157839022623\\
0.2900390625	64.5534630204187\\
0.29052734375	64.6903098136151\\
0.291015625	64.8261820783728\\
0.29150390625	64.9609306903265\\
0.2919921875	65.0943995317529\\
0.29248046875	65.2264254526755\\
0.29296875	65.3568382702074\\
0.29345703125	65.4854608110038\\
0.2939453125	65.6121090018554\\
0.29443359375	65.7365920135551\\
0.294921875	65.8587124631887\\
0.29541015625	65.978266679935\\
0.2958984375	66.09504503928\\
0.29638671875	66.2088323702526\\
0.296875	66.3194084398526\\
0.29736328125	66.4265485182492\\
0.2978515625	66.5300240275823\\
0.29833984375	66.6296032762733\\
0.298828125	66.7250522796574\\
0.29931640625	66.8161356664793\\
0.2998046875	66.9026176693577\\
0.30029296875	66.9842631957369\\
0.30078125	67.0608389741276\\
0.30126953125	67.1321147686214\\
0.3017578125	67.1978646527928\\
0.30224609375	67.2578683322095\\
0.302734375	67.3119125029357\\
0.30322265625	67.3597922316583\\
0.3037109375	67.4013123415049\\
0.30419921875	67.4362887862704\\
0.3046875	67.4645499947296\\
0.30517578125	67.4859381660283\\
0.3056640625	67.5003104968695\\
0.30615234375	67.5075403213922\\
0.306640625	67.5075181452907\\
0.30712890625	67.5001525568754\\
0.3076171875	67.485370999395\\
0.30810546875	67.4631203910181\\
0.30859375	67.4333675813549\\
0.30908203125	67.3960996362138\\
0.3095703125	67.3513239453672\\
0.31005859375	67.2990681513369\\
0.310546875	67.2393799005117\\
0.31103515625	67.1723264211713\\
0.3115234375	67.0979939361063\\
0.31201171875	67.0164869204007\\
0.3125	66.9279272175031\\
0.31298828125	66.8324530288722\\
0.3134765625	66.7302177942051\\
0.31396484375	66.6213889805023\\
0.314453125	66.5061467989778\\
0.31494140625	66.3846828690989\\
0.3154296875	66.2571988488541\\
0.31591796875	66.1239050497474\\
0.31640625	65.9850190540328\\
0.31689453125	65.8407643504287\\
0.3173828125	65.6913690030132\\
0.31787109375	65.5370643662852\\
0.318359375	65.3780838575557\\
0.31884765625	65.2146617959421\\
0.3193359375	65.0470323153639\\
0.31982421875	64.8754283570999\\
0.3203125	64.7000807457349\\
0.32080078125	64.5212173507032\\
0.3212890625	64.3390623341661\\
0.32177734375	64.1538354846628\\
0.322265625	63.9657516348323\\
0.32275390625	63.7750201605563\\
0.3232421875	63.5818445580791\\
0.32373046875	63.3864220950436\\
0.32421875	63.1889435309181\\
0.32470703125	62.9895929019521\\
0.3251953125	62.7885473656113\\
0.32568359375	62.5859770993399\\
0.326171875	62.3820452485132\\
0.32666015625	62.1769079185166\\
0.3271484375	61.9707142060488\\
0.32763671875	61.7636062649355\\
0.328125	61.5557194019902\\
0.32861328125	61.34718219872\\
0.3291015625	61.1381166549647\\
0.32958984375	60.9286383508557\\
0.330078125	60.7188566237805\\
0.33056640625	60.5088747573381\\
0.3310546875	60.2987901795647\\
0.33154296875	60.0886946679905\\
0.33203125	59.878674559352\\
0.33251953125	59.66881096205\\
0.3330078125	59.459179969665\\
0.33349609375	59.2498528740792\\
0.333984375	59.0408963769399\\
0.33447265625	58.8323727984002\\
0.3349609375	58.6243402822307\\
0.33544921875	58.4168529965535\\
0.3359375	58.2099613295855\\
0.33642578125	58.0037120798978\\
0.3369140625	57.7981486408107\\
0.33740234375	57.5933111786334\\
0.337890625	57.3892368045466\\
0.33837890625	57.1859597399941\\
0.3388671875	56.9835114755166\\
0.33935546875	56.7819209230114\\
0.33984375	56.5812145614502\\
0.34033203125	56.3814165761231\\
0.3408203125	56.1825489915139\\
0.34130859375	55.9846317979321\\
0.341796875	55.7876830720547\\
0.34228515625	55.5917190915437\\
0.3427734375	55.3967544439207\\
0.34326171875	55.2028021298902\\
0.34375	55.0098736613074\\
0.34423828125	54.8179791539959\\
0.3447265625	54.627127415616\\
0.34521484375	54.4373260287945\\
0.345703125	54.2485814297145\\
0.34619140625	54.0608989823717\\
0.3466796875	53.8742830486946\\
0.34716796875	53.6887370547223\\
0.34765625	53.5042635530325\\
0.34814453125	53.3208642816017\\
0.3486328125	53.1385402192772\\
0.34912109375	52.9572916380346\\
0.349609375	52.7771181521862\\
0.35009765625	52.5980187646992\\
0.3505859375	52.4199919107802\\
0.35107421875	52.2430354988696\\
0.3515625	52.0671469491888\\
0.35205078125	51.8923232299733\\
0.3525390625	51.7185608915194\\
0.35302734375	51.5458560981673\\
0.353515625	51.3742046583345\\
0.35400390625	51.203602052713\\
0.3544921875	51.0340434607304\\
0.35498046875	50.8655237853773\\
0.35546875	50.6980376764939\\
0.35595703125	50.5315795526045\\
0.3564453125	50.3661436213842\\
0.35693359375	50.2017238988374\\
0.357421875	50.038314227264\\
0.35791015625	49.8759082920839\\
0.3583984375	49.7144996375862\\
0.35888671875	49.5540816816685\\
0.359375	49.3946477296248\\
0.35986328125	49.2361909870374\\
0.3603515625	49.0787045718302\\
0.36083984375	48.9221815255273\\
0.361328125	48.766614823771\\
0.36181640625	48.611997386138\\
0.3623046875	48.4583220853001\\
0.36279296875	48.3055817555661\\
0.36328125	48.1537692008438\\
0.36376953125	48.0028772020568\\
0.3642578125	47.8528985240484\\
0.36474609375	47.7038259220056\\
0.365234375	47.5556521474292\\
0.36572265625	47.4083699536812\\
0.3662109375	47.2619721011339\\
0.36669921875	47.116451361943\\
0.3671875	46.9718005244716\\
0.36767578125	46.8280123973837\\
0.3681640625	46.6850798134288\\
0.36865234375	46.5429956329357\\
0.369140625	46.4017527470354\\
0.36962890625	46.261344080627\\
0.3701171875	46.1217625951057\\
0.37060546875	45.9830012908653\\
0.37109375	45.8450532095908\\
0.37158203125	45.707911436354\\
0.3720703125	45.571569101523\\
0.37255859375	45.4360193825002\\
0.373046875	45.3012555052971\\
0.37353515625	45.1672707459576\\
0.3740234375	45.0340584318393\\
0.37451171875	44.9016119427614\\
0.375	44.7699247120284\\
0.37548828125	44.6389902273373\\
0.3759765625	44.5088020315767\\
0.37646484375	44.3793537235228\\
0.376953125	44.250638958442\\
0.37744140625	44.1226514486036\\
0.3779296875	43.9953849637104\\
0.37841796875	43.8688333312521\\
0.37890625	43.7429904367862\\
0.37939453125	43.6178502241523\\
0.3798828125	43.4934066956241\\
0.38037109375	43.3696539120023\\
0.380859375	43.246585992655\\
0.38134765625	43.1241971155062\\
0.3818359375	43.0024815169791\\
0.38232421875	42.8814334918949\\
0.3828125	42.7610473933328\\
0.38330078125	42.6413176324517\\
0.3837890625	42.5222386782782\\
0.38427734375	42.4038050574631\\
0.384765625	42.2860113540081\\
0.38525390625	42.1688522089658\\
0.3857421875	42.0523223201154\\
0.38623046875	41.936416441615\\
0.38671875	41.8211293836335\\
0.38720703125	41.7064560119633\\
0.3876953125	41.5923912476162\\
0.38818359375	41.4789300664026\\
0.388671875	41.3660674984974\\
0.38916015625	41.2537986279921\\
0.3896484375	41.1421185924365\\
0.39013671875	41.0310225823683\\
0.390625	40.9205058408353\\
0.39111328125	40.8105636629079\\
0.3916015625	40.7011913951855\\
0.39208984375	40.5923844352962\\
0.392578125	40.4841382313905\\
0.39306640625	40.3764482816317\\
0.3935546875	40.2693101336816\\
0.39404296875	40.1627193841829\\
0.39453125	40.0566716782399\\
0.39501953125	39.9511627088969\\
0.3955078125	39.8461882166155\\
0.39599609375	39.7417439887507\\
0.396484375	39.6378258590276\\
0.39697265625	39.5344297070173\\
0.3974609375	39.4315514576148\\
0.39794921875	39.3291870805162\\
0.3984375	39.2273325896994\\
0.39892578125	39.1259840429041\\
0.3994140625	39.0251375411164\\
0.39990234375	38.9247892280541\\
0.400390625	38.8249352896549\\
0.40087890625	38.7255719535679\\
0.4013671875	38.6266954886484\\
0.40185546875	38.5283022044549\\
0.40234375	38.4303884507507\\
0.40283203125	38.332950617009\\
0.4033203125	38.2359851319212\\
0.40380859375	38.1394884629102\\
0.404296875	38.0434571156469\\
0.40478515625	37.9478876335711\\
0.4052734375	37.8527765974176\\
0.40576171875	37.7581206247453\\
0.40625	37.6639163694724\\
0.40673828125	37.5701605214146\\
0.4072265625	37.4768498058294\\
0.40771484375	37.3839809829637\\
0.408203125	37.291550847608\\
0.40869140625	37.1995562286531\\
0.4091796875	37.1079939886537\\
0.40966796875	37.0168610233959\\
0.41015625	36.9261542614691\\
0.41064453125	36.8358706638443\\
0.4111328125	36.7460072234555\\
0.41162109375	36.6565609647875\\
0.412109375	36.567528943468\\
0.41259765625	36.4789082458641\\
0.4130859375	36.3906959886847\\
0.41357421875	36.3028893185872\\
0.4140625	36.2154854117891\\
0.41455078125	36.1284814736845\\
0.4150390625	36.0418747384654\\
0.41552734375	35.955662468748\\
0.416015625	35.8698419552035\\
0.41650390625	35.7844105161937\\
0.4169921875	35.6993654974117\\
0.41748046875	35.6147042715266\\
0.41796875	35.5304242378332\\
0.41845703125	35.4465228219066\\
0.4189453125	35.362997475261\\
0.41943359375	35.279845675013\\
0.419921875	35.1970649235495\\
0.42041015625	35.1146527482001\\
0.4208984375	35.0326067009139\\
0.42138671875	34.9509243579408\\
0.421875	34.8696033195164\\
0.42236328125	34.7886412095523\\
0.4228515625	34.7080356753296\\
0.42333984375	34.6277843871975\\
0.423828125	34.5478850382747\\
0.42431640625	34.4683353441569\\
0.4248046875	34.389133042626\\
0.42529296875	34.3102758933655\\
0.42578125	34.2317616776779\\
0.42626953125	34.1535881982076\\
0.4267578125	34.0757532786665\\
0.42724609375	33.9982547635641\\
0.427734375	33.921090517941\\
0.42822265625	33.8442584271061\\
0.4287109375	33.7677563963778\\
0.42919921875	33.6915823508284\\
0.4296875	33.6157342350323\\
0.43017578125	33.5402100128175\\
0.4306640625	33.4650076670209\\
0.43115234375	33.3901251992467\\
0.431640625	33.3155606296284\\
0.43212890625	33.241311996594\\
0.4326171875	33.1673773566343\\
0.43310546875	33.0937547840752\\
0.43359375	33.020442370852\\
0.43408203125	32.9474382262879\\
0.4345703125	32.874740476875\\
0.43505859375	32.8023472660587\\
0.435546875	32.7302567540248\\
0.43603515625	32.6584671174902\\
0.4365234375	32.5869765494953\\
0.43701171875	32.5157832592007\\
0.4375	32.4448854716858\\
0.43798828125	32.3742814277507\\
0.4384765625	32.3039693837206\\
0.43896484375	32.233947611253\\
0.439453125	32.1642143971476\\
0.43994140625	32.0947680431588\\
0.4404296875	32.025606865811\\
0.44091796875	31.956729196216\\
0.44140625	31.8881333798938\\
0.44189453125	31.8198177765948\\
0.4423828125	31.7517807601252\\
0.44287109375	31.6840207181745\\
0.443359375	31.6165360521461\\
0.44384765625	31.5493251769887\\
0.4443359375	31.4823865210317\\
0.44482421875	31.4157185258219\\
0.4453125	31.3493196459628\\
0.44580078125	31.2831883489556\\
0.4462890625	31.2173231150436\\
0.44677734375	31.1517224370573\\
0.447265625	31.0863848202621\\
0.44775390625	31.021308782209\\
0.4482421875	30.956492852586\\
0.44873046875	30.8919355730723\\
0.44921875	30.8276354971941\\
0.44970703125	30.7635911901831\\
0.4501953125	30.6998012288357\\
0.45068359375	30.6362642013755\\
0.451171875	30.5729787073163\\
0.45166015625	30.5099433573281\\
0.4521484375	30.4471567731045\\
0.45263671875	30.3846175872315\\
0.453125	30.3223244430584\\
0.45361328125	30.260275994571\\
0.4541015625	30.1984709062651\\
0.45458984375	30.1369078530234\\
0.455078125	30.0755855199927\\
0.45556640625	30.0145026024633\\
0.4560546875	29.9536578057501\\
0.45654296875	29.8930498450749\\
0.45703125	29.832677445451\\
0.45751953125	29.7725393415684\\
0.4580078125	29.7126342776809\\
0.45849609375	29.6529610074954\\
0.458984375	29.5935182940613\\
0.45947265625	29.5343049096627\\
0.4599609375	29.4753196357108\\
0.46044921875	29.4165612626392\\
0.4609375	29.3580285897988\\
0.46142578125	29.2997204253558\\
0.4619140625	29.2416355861896\\
0.46240234375	29.183772897793\\
0.462890625	29.1261311941735\\
0.46337890625	29.0687093177555\\
0.4638671875	29.0115061192841\\
0.46435546875	28.9545204577305\\
0.46484375	28.8977512001973\\
0.46533203125	28.8411972218272\\
0.4658203125	28.7848574057105\\
0.46630859375	28.7287306427956\\
0.466796875	28.6728158317998\\
0.46728515625	28.6171118791215\\
0.4677734375	28.5616176987531\\
0.46826171875	28.5063322121962\\
0.46875	28.4512543483762\\
0.46923828125	28.3963830435592\\
0.4697265625	28.3417172412699\\
0.47021484375	28.2872558922099\\
0.470703125	28.2329979541776\\
0.47119140625	28.1789423919888\\
0.4716796875	28.1250881773984\\
0.47216796875	28.0714342890238\\
0.47265625	28.0179797122674\\
0.47314453125	27.9647234392424\\
0.4736328125	27.9116644686978\\
0.47412109375	27.8588018059452\\
0.474609375	27.8061344627861\\
0.47509765625	27.7536614574407\\
0.4755859375	27.7013818144766\\
0.47607421875	27.6492945647396\\
0.4765625	27.5973987452842\\
0.47705078125	27.545693399306\\
0.4775390625	27.4941775760741\\
0.47802734375	27.442850330865\\
0.478515625	27.3917107248967\\
0.47900390625	27.3407578252643\\
0.4794921875	27.2899907048755\\
0.47998046875	27.2394084423884\\
0.48046875	27.1890101221479\\
0.48095703125	27.1387948341252\\
0.4814453125	27.0887616738562\\
0.48193359375	27.0389097423816\\
0.482421875	26.9892381461879\\
0.48291015625	26.939745997148\\
0.4833984375	26.8904324124641\\
0.48388671875	26.8412965146101\\
0.484375	26.7923374312749\\
0.48486328125	26.7435542953071\\
0.4853515625	26.6949462446594\\
0.48583984375	26.6465124223344\\
0.486328125	26.5982519763307\\
0.48681640625	26.5501640595897\\
0.4873046875	26.5022478299434\\
0.48779296875	26.4545024500622\\
0.48828125	26.4069270874036\\
0.48876953125	26.3595209141622\\
0.4892578125	26.312283107219\\
0.48974609375	26.2652128480923\\
0.490234375	26.2183093228891\\
0.49072265625	26.1715717222565\\
0.4912109375	26.1249992413344\\
0.49169921875	26.0785910797082\\
0.4921875	26.0323464413625\\
0.49267578125	25.9862645346352\\
0.4931640625	25.9403445721719\\
0.49365234375	25.8945857708814\\
0.494140625	25.848987351891\\
0.49462890625	25.8035485405031\\
0.4951171875	25.7582685661519\\
0.49560546875	25.7131466623605\\
0.49609375	25.6681820666988\\
0.49658203125	25.623374020742\\
0.4970703125	25.5787217700291\\
0.49755859375	25.5342245640224\\
0.498046875	25.4898816560671\\
0.49853515625	25.4456923033518\\
0.4990234375	25.4016557668689\\
0.49951171875	25.3577713113764\\
0.5	25.3140382053588\\
0.50048828125	25.2704557209899\\
0.5009765625	25.2270231340948\\
0.50146484375	25.1837397241136\\
0.501953125	25.1406047740642\\
0.50244140625	25.0976175705066\\
0.5029296875	25.0547774035073\\
0.50341796875	25.0120835666037\\
0.50390625	24.9695353567696\\
0.50439453125	24.9271320743806\\
0.5048828125	24.8848730231804\\
0.50537109375	24.8427575102469\\
0.505859375	24.8007848459591\\
0.50634765625	24.7589543439645\\
0.5068359375	24.7172653211465\\
0.50732421875	24.6757170975925\\
0.5078125	24.6343089965623\\
0.50830078125	24.5930403444568\\
0.5087890625	24.5519104707872\\
0.50927734375	24.5109187081444\\
0.509765625	24.4700643921689\\
0.51025390625	24.4293468615213\\
0.5107421875	24.3887654578524\\
0.51123046875	24.3483195257747\\
0.51171875	24.3080084128334\\
0.51220703125	24.2678314694777\\
0.5126953125	24.2277880490338\\
0.51318359375	24.187877507676\\
0.513671875	24.1480992044004\\
0.51416015625	24.1084525009973\\
0.5146484375	24.0689367620246\\
0.51513671875	24.0295513547818\\
0.515625	23.9902956492834\\
0.51611328125	23.9511690182339\\
0.5166015625	23.9121708370017\\
0.51708984375	23.8733004835944\\
0.517578125	23.8345573386342\\
0.51806640625	23.7959407853329\\
0.5185546875	23.7574502094681\\
0.51904296875	23.7190849993594\\
0.51953125	23.6808445458446\\
0.52001953125	23.6427282422565\\
0.5205078125	23.6047354844\\
0.52099609375	23.5668656705294\\
0.521484375	23.5291182013255\\
0.52197265625	23.4914924798744\\
0.5224609375	23.4539879116449\\
0.52294921875	23.4166039044671\\
0.5234375	23.379339868511\\
0.52392578125	23.342195216266\\
0.5244140625	23.3051693625192\\
0.52490234375	23.2682617243359\\
0.525390625	23.2314717210386\\
0.52587890625	23.1947987741873\\
0.5263671875	23.1582423075598\\
0.52685546875	23.1218017471323\\
0.52734375	23.0854765210599\\
0.52783203125	23.049266059658\\
0.5283203125	23.0131697953833\\
0.52880859375	22.9771871628154\\
0.529296875	22.9413175986387\\
0.52978515625	22.9055605416242\\
0.5302734375	22.8699154326121\\
0.53076171875	22.8343817144937\\
0.53125	22.7989588321949\\
0.53173828125	22.7636462326583\\
0.5322265625	22.7284433648273\\
0.53271484375	22.6933496796286\\
0.533203125	22.6583646299567\\
0.53369140625	22.6234876706571\\
0.5341796875	22.5887182585107\\
0.53466796875	22.5540558522182\\
0.53515625	22.5194999123844\\
0.53564453125	22.4850499015031\\
0.5361328125	22.450705283942\\
0.53662109375	22.4164655259277\\
0.537109375	22.3823300955318\\
0.53759765625	22.3482984626554\\
0.5380859375	22.3143700990158\\
0.53857421875	22.2805444781322\\
0.5390625	22.2468210753119\\
0.53955078125	22.2131993676366\\
0.5400390625	22.1796788339494\\
0.54052734375	22.1462589548413\\
0.541015625	22.1129392126383\\
0.54150390625	22.0797190913885\\
0.5419921875	22.0465980768499\\
0.54248046875	22.0135756564774\\
0.54296875	21.9806513194114\\
0.54345703125	21.9478245564649\\
0.5439453125	21.9150948601123\\
0.54443359375	21.8824617244776\\
0.544921875	21.8499246453227\\
0.54541015625	21.8174831200366\\
0.5458984375	21.785136647624\\
0.54638671875	21.7528847286943\\
0.546875	21.7207268654513\\
0.54736328125	21.6886625616821\\
0.5478515625	21.6566913227471\\
0.54833984375	21.6248126555701\\
0.548828125	21.5930260686273\\
0.54931640625	21.5613310719385\\
0.5498046875	21.529727177057\\
0.55029296875	21.4982138970599\\
0.55078125	21.4667907465391\\
0.55126953125	21.435457241592\\
0.5517578125	21.4042128998125\\
0.55224609375	21.3730572402822\\
0.552734375	21.3419897835616\\
0.55322265625	21.3110100516818\\
0.5537109375	21.2801175681363\\
0.55419921875	21.2493118578723\\
0.5546875	21.2185924472829\\
0.55517578125	21.1879588641997\\
0.5556640625	21.1574106378844\\
0.55615234375	21.1269472990217\\
0.556640625	21.0965683797116\\
0.55712890625	21.0662734134626\\
0.5576171875	21.0360619351841\\
0.55810546875	21.0059334811798\\
0.55859375	20.9758875891408\\
0.55908203125	20.9459237981391\\
0.5595703125	20.9160416486209\\
0.56005859375	20.8862406824003\\
0.560546875	20.8565204426534\\
0.56103515625	20.8268804739116\\
0.5615234375	20.7973203220566\\
0.56201171875	20.7678395343139\\
0.5625	20.7384376592474\\
0.56298828125	20.7091142467541\\
0.5634765625	20.6798688480585\\
0.56396484375	20.6507010157076\\
0.564453125	20.6216103035657\\
0.56494140625	20.5925962668095\\
0.5654296875	20.5636584619236\\
0.56591796875	20.5347964466952\\
0.56640625	20.5060097802102\\
0.56689453125	20.4772980228486\\
0.5673828125	20.4486607362802\\
0.56787109375	20.4200974834604\\
0.568359375	20.3916078286263\\
0.56884765625	20.3631913372931\\
0.5693359375	20.3348475762499\\
0.56982421875	20.3065761135562\\
0.5703125	20.2783765185387\\
0.57080078125	20.2502483617876\\
0.5712890625	20.2221912151536\\
0.57177734375	20.1942046517446\\
0.572265625	20.1662882459231\\
0.57275390625	20.1384415733025\\
0.5732421875	20.1106642107454\\
0.57373046875	20.0829557363602\\
0.57421875	20.0553157294989\\
0.57470703125	20.0277437707546\\
0.5751953125	20.0002394419592\\
0.57568359375	19.9728023261816\\
0.576171875	19.9454320077251\\
0.57666015625	19.9181280721258\\
0.5771484375	19.8908901061506\\
0.57763671875	19.8637176977958\\
0.578125	19.8366104362851\\
0.57861328125	19.8095679120684\\
0.5791015625	19.7825897168204\\
0.57958984375	19.7556754434389\\
0.580078125	19.7288246860445\\
0.58056640625	19.7020370399786\\
0.5810546875	19.675312101803\\
0.58154296875	19.648649469299\\
0.58203125	19.6220487414665\\
0.58251953125	19.5955095185234\\
0.5830078125	19.5690314019051\\
0.58349609375	19.5426139942641\\
0.583984375	19.5162568994695\\
0.58447265625	19.4899597226069\\
0.5849609375	19.4637220699781\\
0.58544921875	19.4375435491014\\
0.5859375	19.4114237687111\\
0.58642578125	19.3853623387581\\
0.5869140625	19.3593588704099\\
0.58740234375	19.3334129760511\\
0.587890625	19.3075242692835\\
0.58837890625	19.2816923649271\\
0.5888671875	19.2559168790201\\
0.58935546875	19.2301974288201\\
0.58984375	19.2045336328047\\
0.59033203125	19.1789251106723\\
0.5908203125	19.1533714833429\\
0.59130859375	19.1278723729595\\
0.591796875	19.1024274028892\\
0.59228515625	19.0770361977239\\
0.5927734375	19.051698383282\\
0.59326171875	19.0264135866099\\
0.59375	19.001181435983\\
0.59423828125	18.9760015609074\\
0.5947265625	18.9508735921219\\
0.59521484375	18.9257971615989\\
0.595703125	18.9007719025469\\
0.59619140625	18.8757974494117\\
0.5966796875	18.8508734378789\\
0.59716796875	18.8259995048754\\
0.59765625	18.8011752885718\\
0.59814453125	18.7764004283841\\
0.5986328125	18.7516745649764\\
0.59912109375	18.7269973402627\\
0.599609375	18.7023683974095\\
0.60009765625	18.6777873808381\\
0.6005859375	18.6532539362271\\
0.60107421875	18.628767710515\\
0.6015625	18.6043283519024\\
0.60205078125	18.5799355098551\\
0.6025390625	18.5555888351068\\
0.60302734375	18.5312879796613\\
0.603515625	18.5070325967962\\
0.60400390625	18.482822341065\\
0.6044921875	18.4586568683005\\
0.60498046875	18.4345358356178\\
0.60546875	18.4104589014171\\
0.60595703125	18.386425725387\\
0.6064453125	18.3624359685075\\
0.60693359375	18.3384892930535\\
0.607421875	18.314585362598\\
0.60791015625	18.290723842015\\
0.6083984375	18.2669043974835\\
0.60888671875	18.2431266964905\\
0.609375	18.2193904078345\\
0.60986328125	18.1956952016291\\
0.6103515625	18.1720407493066\\
0.61083984375	18.1484267236215\\
0.611328125	18.1248527986542\\
0.61181640625	18.1013186498146\\
0.6123046875	18.0778239538457\\
0.61279296875	18.0543683888277\\
0.61328125	18.0309516341817\\
0.61376953125	18.0075733706731\\
0.6142578125	17.9842332804161\\
0.61474609375	17.9609310468771\\
0.615234375	17.937666354879\\
0.61572265625	17.9144388906049\\
0.6162109375	17.8912483416022\\
0.61669921875	17.8680943967867\\
0.6171875	17.8449767464463\\
0.61767578125	17.8218950822458\\
0.6181640625	17.7988490972302\\
0.61865234375	17.7758384858292\\
0.619140625	17.7528629438617\\
0.61962890625	17.7299221685391\\
0.6201171875	17.7070158584705\\
0.62060546875	17.6841437136662\\
0.62109375	17.6613054355423\\
0.62158203125	17.6385007269247\\
0.6220703125	17.6157292920537\\
0.62255859375	17.592990836588\\
0.623046875	17.5702850676091\\
0.62353515625	17.5476116936257\\
0.6240234375	17.5249704245781\\
0.62451171875	17.5023609718421\\
0.625	17.479783048234\\
0.62548828125	17.4572363680143\\
0.6259765625	17.4347206468926\\
0.62646484375	17.4122356020318\\
0.626953125	17.3897809520524\\
0.62744140625	17.3673564170369\\
0.6279296875	17.3449617185341\\
0.62841796875	17.3225965795639\\
0.62890625	17.3002607246213\\
0.62939453125	17.2779538796807\\
0.6298828125	17.2556757722008\\
0.63037109375	17.2334261311284\\
0.630859375	17.2112046869031\\
0.63134765625	17.1890111714617\\
0.6318359375	17.1668453182425\\
0.63232421875	17.1447068621896\\
0.6328125	17.1225955397573\\
0.63330078125	17.1005110889144\\
0.6337890625	17.0784532491487\\
0.63427734375	17.0564217614711\\
0.634765625	17.0344163684201\\
0.63525390625	17.0124368140659\\
0.6357421875	16.9904828440148\\
0.63623046875	16.9685542054134\\
0.63671875	16.946650646953\\
0.63720703125	16.9247719188737\\
0.6376953125	16.9029177729683\\
0.63818359375	16.8810879625872\\
0.638671875	16.8592822426418\\
0.63916015625	16.8375003696092\\
0.6396484375	16.815742101536\\
0.64013671875	16.7940071980425\\
0.640625	16.7722954203267\\
0.64111328125	16.7506065311684\\
0.6416015625	16.728940294933\\
0.64208984375	16.7072964775759\\
0.642578125	16.6856748466459\\
0.64306640625	16.6640751712897\\
0.6435546875	16.6424972222552\\
0.64404296875	16.6209407718956\\
0.64453125	16.5994055941734\\
0.64501953125	16.5778914646637\\
0.6455078125	16.5563981605583\\
0.64599609375	16.5349254606695\\
0.646484375	16.5134731454331\\
0.64697265625	16.492040996913\\
0.6474609375	16.4706287988036\\
0.64794921875	16.4492363364345\\
0.6484375	16.4278633967731\\
0.64892578125	16.4065097684284\\
0.6494140625	16.3851752416545\\
0.64990234375	16.3638596083537\\
0.650390625	16.3425626620799\\
0.65087890625	16.3212841980418\\
0.6513671875	16.3000240131063\\
0.65185546875	16.2787819058013\\
0.65234375	16.2575576763192\\
0.65283203125	16.2363511265196\\
0.6533203125	16.2151620599323\\
0.65380859375	16.1939902817608\\
0.654296875	16.1728355988842\\
0.65478515625	16.151697819861\\
0.6552734375	16.1305767549312\\
0.65576171875	16.1094722160193\\
0.65625	16.0883840167369\\
0.65673828125	16.0673119723853\\
0.6572265625	16.0462558999581\\
0.65771484375	16.0252156181435\\
0.658203125	16.0041909473267\\
0.65869140625	15.9831817095925\\
0.6591796875	15.9621877287274\\
0.65966796875	15.9412088302218\\
0.66015625	15.9202448412722\\
0.66064453125	15.8992955907833\\
0.6611328125	15.8783609093699\\
0.66162109375	15.857440629359\\
0.662109375	15.8365345847917\\
0.66259765625	15.8156426114248\\
0.6630859375	15.7947645467327\\
0.66357421875	15.7739002299093\\
0.6640625	15.7530495018689\\
0.66455078125	15.7322122052486\\
0.6650390625	15.7113881844091\\
0.66552734375	15.6905772854364\\
0.666015625	15.669779356143\\
0.66650390625	15.6489942460691\\
0.6669921875	15.6282218064841\\
0.66748046875	15.607461890387\\
0.66796875	15.5867143525082\\
0.66845703125	15.5659790493099\\
0.6689453125	15.545255838987\\
0.66943359375	15.5245445814682\\
0.669921875	15.5038451384162\\
0.67041015625	15.4831573732289\\
0.6708984375	15.4624811510395\\
0.67138671875	15.441816338717\\
0.671875	15.4211628048669\\
0.67236328125	15.4005204198311\\
0.6728515625	15.3798890556883\\
0.67333984375	15.3592685862544\\
0.673828125	15.3386588870818\\
0.67431640625	15.3180598354604\\
0.6748046875	15.2974713104165\\
0.67529296875	15.2768931927134\\
0.67578125	15.2563253648504\\
0.67626953125	15.2357677110632\\
0.6767578125	15.2152201173229\\
0.67724609375	15.1946824713355\\
0.677734375	15.1741546625416\\
0.67822265625	15.1536365821156\\
0.6787109375	15.1331281229645\\
0.67919921875	15.1126291797277\\
0.6796875	15.0921396487756\\
0.68017578125	15.0716594282085\\
0.6806640625	15.051188417856\\
0.68115234375	15.0307265192749\\
0.681640625	15.010273635749\\
0.68212890625	14.9898296722866\\
0.6826171875	14.9693945356199\\
0.68310546875	14.948968134203\\
0.68359375	14.9285503782102\\
0.68408203125	14.9081411795346\\
0.6845703125	14.8877404517859\\
0.68505859375	14.8673481102888\\
0.685546875	14.8469640720809\\
0.68603515625	14.8265882559103\\
0.6865234375	14.8062205822341\\
0.68701171875	14.7858609732154\\
0.6875	14.7655093527215\\
0.68798828125	14.7451656463212\\
0.6884765625	14.7248297812823\\
0.68896484375	14.7045016865691\\
0.689453125	14.6841812928395\\
0.68994140625	14.6638685324425\\
0.6904296875	14.643563339415\\
0.69091796875	14.6232656494793\\
0.69140625	14.6029754000393\\
0.69189453125	14.5826925301784\\
0.6923828125	14.5624169806552\\
0.69287109375	14.5421486939011\\
0.693359375	14.5218876140163\\
0.69384765625	14.5016336867666\\
0.6943359375	14.4813868595797\\
0.69482421875	14.4611470815419\\
0.6953125	14.4409143033939\\
0.69580078125	14.4206884775271\\
0.6962890625	14.4004695579801\\
0.69677734375	14.3802575004341\\
0.697265625	14.3600522622092\\
0.69775390625	14.3398538022603\\
0.6982421875	14.3196620811727\\
0.69873046875	14.2994770611577\\
0.69921875	14.2792987060485\\
0.69970703125	14.2591269812954\\
0.7001953125	14.2389618539614\\
0.70068359375	14.2188032927177\\
0.701171875	14.1986512678384\\
0.70166015625	14.1785057511966\\
0.7021484375	14.1583667162587\\
0.70263671875	14.1382341380796\\
0.703125	14.1181079932981\\
0.70361328125	14.0979882601313\\
0.7041015625	14.0778749183697\\
0.70458984375	14.0577679493715\\
0.705078125	14.0376673360577\\
0.70556640625	14.0175730629066\\
0.7060546875	13.9974851159479\\
0.70654296875	13.9774034827577\\
0.70703125	13.9573281524521\\
0.70751953125	13.9372591156824\\
0.7080078125	13.9171963646284\\
0.70849609375	13.8971398929931\\
0.708984375	13.8770896959964\\
0.70947265625	13.8570457703693\\
0.7099609375	13.8370081143478\\
0.71044921875	13.8169767276664\\
0.7109375	13.7969516115525\\
0.71142578125	13.7769327687193\\
0.7119140625	13.7569202033601\\
0.71240234375	13.7369139211415\\
0.712890625	13.7169139291969\\
0.71337890625	13.6969202361202\\
0.7138671875	13.6769328519587\\
0.71435546875	13.6569517882068\\
0.71484375	13.636977057799\\
0.71533203125	13.6170086751033\\
0.7158203125	13.597046655914\\
0.71630859375	13.5770910174447\\
0.716796875	13.5571417783217\\
0.71728515625	13.5371989585766\\
0.7177734375	13.5172625796393\\
0.71826171875	13.4973326643305\\
0.71875	13.4774092368548\\
0.71923828125	13.4574923227932\\
0.7197265625	13.4375819490959\\
0.72021484375	13.4176781440746\\
0.720703125	13.397780937395\\
0.72119140625	13.3778903600697\\
0.7216796875	13.3580064444502\\
0.72216796875	13.3381292242192\\
0.72265625	13.3182587343834\\
0.72314453125	13.2983950112651\\
0.7236328125	13.2785380924949\\
0.72412109375	13.2586880170036\\
0.724609375	13.2388448250145\\
0.72509765625	13.2190085580351\\
0.7255859375	13.1991792588495\\
0.72607421875	13.1793569715103\\
0.7265625	13.1595417413302\\
0.72705078125	13.1397336148742\\
0.7275390625	13.1199326399514\\
0.72802734375	13.1001388656067\\
0.728515625	13.0803523421125\\
0.72900390625	13.0605731209605\\
0.7294921875	13.0408012548534\\
0.72998046875	13.0210367976965\\
0.73046875	13.0012798045893\\
0.73095703125	12.9815303318168\\
0.7314453125	12.9617884368414\\
0.73193359375	12.9420541782945\\
0.732421875	12.9223276159672\\
0.73291015625	12.9026088108026\\
0.7333984375	12.8828978248866\\
0.73388671875	12.8631947214394\\
0.734375	12.8434995648072\\
0.73486328125	12.8238124204527\\
0.7353515625	12.8041333549472\\
0.73583984375	12.7844624359614\\
0.736328125	12.7647997322565\\
0.73681640625	12.7451453136757\\
0.7373046875	12.7254992511353\\
0.73779296875	12.7058616166157\\
0.73828125	12.6862324831526\\
0.73876953125	12.6666119248281\\
0.7392578125	12.6470000167616\\
0.73974609375	12.6273968351013\\
0.740234375	12.6078024570146\\
0.74072265625	12.5882169606797\\
0.7412109375	12.5686404252762\\
0.74169921875	12.5490729309763\\
0.7421875	12.5295145589355\\
0.74267578125	12.5099653912841\\
0.7431640625	12.4904255111174\\
0.74365234375	12.4708950024872\\
0.744140625	12.4513739503923\\
0.74462890625	12.4318624407698\\
0.7451171875	12.4123605604857\\
0.74560546875	12.3928683973258\\
0.74609375	12.3733860399867\\
0.74658203125	12.3539135780666\\
0.7470703125	12.3344511020561\\
0.74755859375	12.314998703329\\
0.748046875	12.2955564741335\\
0.74853515625	12.2761245075825\\
0.7490234375	12.2567028976448\\
0.74951171875	12.2372917391359\\
0.75	12.2178911277087\\
0.75048828125	12.1985011598445\\
0.7509765625	12.1791219328436\\
0.75146484375	12.1597535448163\\
0.751953125	12.1403960946738\\
0.75244140625	12.1210496821189\\
0.7529296875	12.1017144076367\\
0.75341796875	12.0823903724859\\
0.75390625	12.063077678689\\
0.75439453125	12.043776429024\\
0.7548828125	12.0244867270143\\
0.75537109375	12.0052086769202\\
0.755859375	11.9859423837298\\
0.75634765625	11.9666879531495\\
0.7568359375	11.9474454915951\\
0.75732421875	11.9282151061828\\
0.7578125	11.9089969047199\\
0.75830078125	11.889790995696\\
0.7587890625	11.8705974882735\\
0.75927734375	11.8514164922791\\
0.759765625	11.8322481181944\\
0.76025390625	11.813092477147\\
0.7607421875	11.7939496809014\\
0.76123046875	11.7748198418502\\
0.76171875	11.7557030730049\\
0.76220703125	11.7365994879871\\
0.7626953125	11.7175092010198\\
0.76318359375	11.6984323269177\\
0.763671875	11.6793689810793\\
0.76416015625	11.6603192794774\\
0.7646484375	11.6412833386503\\
0.76513671875	11.6222612756932\\
0.765625	11.6032532082492\\
0.76611328125	11.5842592545005\\
0.7666015625	11.5652795331599\\
0.76708984375	11.5463141634618\\
0.767578125	11.5273632651534\\
0.76806640625	11.5084269584865\\
0.7685546875	11.4895053642082\\
0.76904296875	11.4705986035527\\
0.76953125	11.4517067982327\\
0.77001953125	11.4328300704304\\
0.7705078125	11.4139685427896\\
0.77099609375	11.3951223384065\\
0.771484375	11.3762915808217\\
0.77197265625	11.3574763940115\\
0.7724609375	11.3386769023797\\
0.77294921875	11.3198932307487\\
0.7734375	11.3011255043518\\
0.77392578125	11.2823738488244\\
0.7744140625	11.2636383901956\\
0.77490234375	11.2449192548806\\
0.775390625	11.2262165696717\\
0.77587890625	11.2075304617305\\
0.7763671875	11.1888610585795\\
0.77685546875	11.1702084880945\\
0.77734375	11.1515728784957\\
0.77783203125	11.1329543583402\\
0.7783203125	11.1143530565139\\
0.77880859375	11.0957691022234\\
0.779296875	11.077202624988\\
0.77978515625	11.0586537546321\\
0.7802734375	11.0401226212767\\
0.78076171875	11.0216093553324\\
0.78125	11.0031140874909\\
0.78173828125	10.9846369487176\\
0.7822265625	10.9661780702438\\
0.78271484375	10.9477375835589\\
0.783203125	10.929315620403\\
0.78369140625	10.9109123127591\\
0.7841796875	10.8925277928457\\
0.78466796875	10.874162193109\\
0.78515625	10.8558156462159\\
0.78564453125	10.8374882850461\\
0.7861328125	10.819180242685\\
0.78662109375	10.8008916524162\\
0.787109375	10.7826226477142\\
0.78759765625	10.7643733622374\\
0.7880859375	10.7461439298205\\
0.78857421875	10.7279344844676\\
0.7890625	10.7097451603448\\
0.78955078125	10.6915760917736\\
0.7900390625	10.6734274132233\\
0.79052734375	10.6552992593043\\
0.791015625	10.6371917647611\\
0.79150390625	10.6191050644655\\
0.7919921875	10.6010392934094\\
0.79248046875	10.5829945866986\\
0.79296875	10.5649710795452\\
0.79345703125	10.5469689072617\\
0.7939453125	10.5289882052538\\
0.79443359375	10.5110291090139\\
0.794921875	10.4930917541145\\
0.79541015625	10.4751762762018\\
0.7958984375	10.4572828109891\\
0.79638671875	10.4394114942502\\
0.796875	10.4215624618134\\
0.79736328125	10.4037358495548\\
0.7978515625	10.3859317933921\\
0.79833984375	10.3681504292784\\
0.798828125	10.3503918931958\\
0.79931640625	10.3326563211497\\
0.7998046875	10.3149438491621\\
0.80029296875	10.2972546132659\\
0.80078125	10.2795887494988\\
0.80126953125	10.2619463938974\\
0.8017578125	10.2443276824911\\
0.80224609375	10.2267327512964\\
0.802734375	10.209161736311\\
0.80322265625	10.1916147735081\\
0.8037109375	10.1740919988304\\
0.80419921875	10.1565935481849\\
0.8046875	10.1391195574368\\
0.80517578125	10.1216701624042\\
0.8056640625	10.1042454988525\\
0.80615234375	10.0868457024889\\
0.806640625	10.0694709089569\\
0.80712890625	10.0521212538309\\
0.8076171875	10.0347968726112\\
0.80810546875	10.0174979007182\\
0.80859375	10.0002244734875\\
0.80908203125	9.98297672616446\\
0.8095703125	9.96575479389935\\
0.81005859375	9.94855881174201\\
0.810546875	9.93138891463689\\
0.81103515625	9.91424523741807\\
0.8115234375	9.89712791480425\\
0.81201171875	9.88003708139384\\
0.8125	9.86297287166013\\
0.81298828125	9.84593541994654\\
0.8134765625	9.82892486046171\\
0.81396484375	9.81194132727493\\
0.814453125	9.79498495431137\\
0.81494140625	9.7780558753475\\
0.8154296875	9.76115422400658\\
0.81591796875	9.744280133754\\
0.81640625	9.72743373789294\\
0.81689453125	9.71061516955979\\
0.8173828125	9.69382456171994\\
0.81787109375	9.67706204716332\\
0.818359375	9.66032775850016\\
0.81884765625	9.64362182815671\\
0.8193359375	9.62694438837109\\
0.81982421875	9.61029557118911\\
0.8203125	9.5936755084602\\
0.82080078125	9.57708433183331\\
0.8212890625	9.56052217275291\\
0.82177734375	9.54398916245506\\
0.822265625	9.52748543196345\\
0.82275390625	9.51101111208557\\
0.8232421875	9.49456633340879\\
0.82373046875	9.47815122629674\\
0.82421875	9.46176592088537\\
0.82470703125	9.44541054707944\\
0.8251953125	9.42908523454878\\
0.82568359375	9.41279011272467\\
0.826171875	9.39652531079636\\
0.82666015625	9.38029095770747\\
0.8271484375	9.36408718215258\\
0.82763671875	9.34791411257379\\
0.828125	9.33177187715733\\
0.82861328125	9.3156606038302\\
0.8291015625	9.29958042025691\\
0.82958984375	9.28353145383625\\
0.830078125	9.26751383169799\\
0.83056640625	9.2515276806998\\
0.8310546875	9.23557312742408\\
0.83154296875	9.2196502981749\\
0.83203125	9.20375931897499\\
0.83251953125	9.18790031556263\\
0.8330078125	9.17207341338882\\
0.83349609375	9.15627873761431\\
0.833984375	9.1405164131067\\
0.83447265625	9.1247865644377\\
0.8349609375	9.10908931588021\\
0.83544921875	9.09342479140568\\
0.8359375	9.07779311468132\\
0.83642578125	9.0621944090675\\
0.8369140625	9.04662879761506\\
0.83740234375	9.03109640306275\\
0.837890625	9.01559734783464\\
0.83837890625	9.00013175403769\\
0.8388671875	8.98469974345921\\
0.83935546875	8.96930143756439\\
0.83984375	8.95393695749403\\
0.84033203125	8.93860642406202\\
0.8408203125	8.92330995775318\\
0.84130859375	8.90804767872084\\
0.841796875	8.89281970678468\\
0.84228515625	8.87762616142852\\
0.8427734375	8.86246716179807\\
0.84326171875	8.84734282669889\\
0.84375	8.83225327459427\\
0.84423828125	8.8171986236031\\
0.8447265625	8.80217899149792\\
0.84521484375	8.78719449570293\\
0.845703125	8.77224525329196\\
0.84619140625	8.75733138098662\\
0.8466796875	8.74245299515443\\
0.84716796875	8.7276102118069\\
0.84765625	8.71280314659774\\
0.84814453125	8.69803191482113\\
0.8486328125	8.68329663140988\\
0.84912109375	8.6685974109338\\
0.849609375	8.653934367598\\
0.85009765625	8.63930761524117\\
0.8505859375	8.62471726733409\\
0.85107421875	8.61016343697793\\
0.8515625	8.59564623690276\\
0.85205078125	8.58116577946605\\
0.8525390625	8.56672217665114\\
0.85302734375	8.55231554006579\\
0.853515625	8.53794598094078\\
0.85400390625	8.52361361012851\\
0.8544921875	8.50931853810161\\
0.85498046875	8.49506087495166\\
0.85546875	8.48084073038785\\
0.85595703125	8.46665821373571\\
0.8564453125	8.45251343393589\\
0.85693359375	8.43840649954295\\
0.857421875	8.4243375187241\\
0.85791015625	8.41030659925818\\
0.8583984375	8.39631384853442\\
0.85888671875	8.38235937355138\\
0.859375	8.36844328091594\\
0.85986328125	8.35456567684212\\
0.8603515625	8.3407266671502\\
0.86083984375	8.32692635726567\\
0.861328125	8.31316485221827\\
0.86181640625	8.29944225664109\\
0.8623046875	8.28575867476963\\
0.86279296875	8.2721142104409\\
0.86328125	8.25850896709264\\
0.86376953125	8.24494304776243\\
0.8642578125	8.23141655508689\\
0.86474609375	8.21792959130091\\
0.865234375	8.20448225823693\\
0.86572265625	8.19107465732415\\
0.8662109375	8.17770688958786\\
0.86669921875	8.16437905564879\\
0.8671875	8.15109125572237\\
0.86767578125	8.13784358961814\\
0.8681640625	8.1246361567392\\
0.86865234375	8.11146905608154\\
0.869140625	8.09834238623347\\
0.86962890625	8.08525624537517\\
0.8701171875	8.07221073127807\\
0.87060546875	8.05920594130441\\
0.87109375	8.04624197240678\\
0.87158203125	8.03331892112758\\
0.8720703125	8.02043688359869\\
0.87255859375	8.00759595554099\\
0.873046875	7.99479623226401\\
0.87353515625	7.98203780866553\\
0.8740234375	7.96932077923124\\
0.87451171875	7.95664523803441\\
0.875	7.94401127873563\\
0.87548828125	7.93141899458241\\
0.8759765625	7.918868478409\\
0.87646484375	7.90635982263611\\
0.876953125	7.89389311927067\\
0.87744140625	7.88146845990565\\
0.8779296875	7.86908593571978\\
0.87841796875	7.85674563747749\\
0.87890625	7.84444765552866\\
0.87939453125	7.83219207980852\\
0.8798828125	7.81997899983747\\
0.88037109375	7.80780850472108\\
0.880859375	7.79568068314987\\
0.88134765625	7.78359562339935\\
0.8818359375	7.77155341332987\\
0.88232421875	7.75955414038665\\
0.8828125	7.74759789159971\\
0.88330078125	7.73568475358389\\
0.8837890625	7.72381481253882\\
0.88427734375	7.71198815424896\\
0.884765625	7.70020486408369\\
0.88525390625	7.68846502699724\\
0.8857421875	7.67676872752889\\
0.88623046875	7.66511604980294\\
0.88671875	7.65350707752889\\
0.88720703125	7.64194189400149\\
0.8876953125	7.6304205821009\\
0.88818359375	7.61894322429278\\
0.888671875	7.60750990262855\\
0.88916015625	7.59612069874536\\
0.8896484375	7.58477569386648\\
0.89013671875	7.57347496880134\\
0.890625	7.56221860394579\\
0.89111328125	7.55100667928232\\
0.8916015625	7.53983927438024\\
0.89208984375	7.52871646839599\\
0.892578125	7.5176383400733\\
0.89306640625	7.50660496774358\\
0.8935546875	7.49561642932601\\
0.89404296875	7.48467280232797\\
0.89453125	7.47377416384533\\
0.89501953125	7.46292059056267\\
0.8955078125	7.45211215875364\\
0.89599609375	7.44134894428131\\
0.896484375	7.43063102259848\\
0.89697265625	7.41995846874803\\
0.8974609375	7.40933135736324\\
0.89794921875	7.39874976266828\\
0.8984375	7.38821375847841\\
0.89892578125	7.3777234182005\\
0.8994140625	7.36727881483335\\
0.89990234375	7.35688002096816\\
0.900390625	7.34652710878887\\
0.90087890625	7.33622015007259\\
0.9013671875	7.32595921619007\\
0.90185546875	7.31574437810615\\
0.90234375	7.30557570638012\\
0.90283203125	7.2954532711662\\
0.9033203125	7.28537714221409\\
0.90380859375	7.2753473888693\\
0.904296875	7.26536408007372\\
0.90478515625	7.25542728436608\\
0.9052734375	7.24553706988243\\
0.90576171875	7.23569350435658\\
0.90625	7.22589665512075\\
0.90673828125	7.21614658910589\\
0.9072265625	7.20644337284235\\
0.90771484375	7.19678707246031\\
0.908203125	7.18717775369038\\
0.90869140625	7.17761548186403\\
0.9091796875	7.1681003219142\\
0.90966796875	7.15863233837589\\
0.91015625	7.14921159538656\\
0.91064453125	7.13983815668681\\
0.9111328125	7.13051208562093\\
0.91162109375	7.1212334451374\\
0.912109375	7.11200229778951\\
0.91259765625	7.10281870573589\\
0.9130859375	7.09368273074116\\
0.91357421875	7.08459443417642\\
0.9140625	7.07555387701992\\
0.91455078125	7.06656111985759\\
0.9150390625	7.05761622288368\\
0.91552734375	7.04871924590129\\
0.916015625	7.03987024832309\\
0.91650390625	7.03106928917175\\
0.9169921875	7.02231642708075\\
0.91748046875	7.01361172029482\\
0.91796875	7.00495522667065\\
0.91845703125	6.99634700367746\\
0.9189453125	6.9877871083977\\
0.91943359375	6.97927559752755\\
0.919921875	6.97081252737762\\
0.92041015625	6.96239795387358\\
0.9208984375	6.95403193255677\\
0.92138671875	6.94571451858485\\
0.921875	6.93744576673239\\
0.92236328125	6.92922573139158\\
0.9228515625	6.92105446657276\\
0.92333984375	6.91293202590519\\
0.923828125	6.90485846263759\\
0.92431640625	6.89683382963883\\
0.9248046875	6.88885817939853\\
0.92529296875	6.88093156402777\\
0.92578125	6.87305403525966\\
0.92626953125	6.86522564445007\\
0.9267578125	6.8574464425782\\
0.92724609375	6.84971648024728\\
0.927734375	6.84203580768517\\
0.92822265625	6.83440447474506\\
0.9287109375	6.8268225309061\\
0.92919921875	6.81929002527406\\
0.9296875	6.8118070065819\\
0.93017578125	6.80437352319058\\
0.9306640625	6.79698962308957\\
0.93115234375	6.78965535389751\\
0.931640625	6.78237076286296\\
0.93212890625	6.77513589686496\\
0.9326171875	6.76795080241371\\
0.93310546875	6.7608155256512\\
0.93359375	6.75373011235186\\
0.93408203125	6.74669460792323\\
0.9345703125	6.73970905740662\\
0.93505859375	6.73277350547769\\
0.935546875	6.72588799644714\\
0.93603515625	6.71905257426132\\
0.9365234375	6.71226728250297\\
0.93701171875	6.7055321643917\\
0.9375	6.69884726278477\\
0.93798828125	6.69221262017767\\
0.9384765625	6.6856282787047\\
0.93896484375	6.67909428013976\\
0.939453125	6.67261066589679\\
0.93994140625	6.66617747703057\\
0.9404296875	6.6597947542372\\
0.94091796875	6.6534625378549\\
0.94140625	6.64718086786444\\
0.94189453125	6.64094978388994\\
0.9423828125	6.63476932519934\\
0.94287109375	6.62863953070513\\
0.943359375	6.62256043896491\\
0.94384765625	6.61653208818201\\
0.9443359375	6.61055451620614\\
0.94482421875	6.60462776053392\\
0.9453125	6.59875185830955\\
0.94580078125	6.59292684632543\\
0.9462890625	6.58715276102266\\
0.94677734375	6.58142963849174\\
0.947265625	6.57575751447311\\
0.94775390625	6.57013642435774\\
0.9482421875	6.56456640318781\\
0.94873046875	6.5590474856571\\
0.94921875	6.5535797061118\\
0.94970703125	6.54816309855089\\
0.9501953125	6.54279769662686\\
0.95068359375	6.53748353364619\\
0.951171875	6.53222064256995\\
0.95166015625	6.5270090560144\\
0.9521484375	6.5218488062515\\
0.95263671875	6.51673992520942\\
0.953125	6.51168244447329\\
};
\addplot [color=mycolor3,solid]
  table[row sep=crcr]{0.953125	6.51168244447329\\
0.95361328125	6.50667639528546\\
0.9541015625	6.50172180854633\\
0.95458984375	6.49681871481468\\
0.955078125	6.49196714430829\\
0.95556640625	6.48716712690454\\
0.9560546875	6.48241869214081\\
0.95654296875	6.47772186921508\\
0.95703125	6.47307668698645\\
0.95751953125	6.46848317397562\\
0.9580078125	6.46394135836546\\
0.95849609375	6.45945126800145\\
0.958984375	6.45501293039225\\
0.95947265625	6.45062637271016\\
0.9599609375	6.44629162179159\\
0.96044921875	6.44200870413764\\
0.9609375	6.43777764591445\\
0.96142578125	6.43359847295383\\
0.9619140625	6.42947121075364\\
0.96240234375	6.42539588447828\\
0.962890625	6.42137251895913\\
0.96337890625	6.41740113869512\\
0.9638671875	6.41348176785303\\
0.96435546875	6.40961443026809\\
0.96484375	6.40579914944431\\
0.96533203125	6.40203594855501\\
0.9658203125	6.39832485044315\\
0.96630859375	6.3946658776219\\
0.966796875	6.39105905227497\\
0.96728515625	6.38750439625705\\
0.9677734375	6.38400193109425\\
0.96826171875	6.38055167798448\\
0.96875	6.37715365779784\\
0.96923828125	6.37380789107711\\
0.9697265625	6.37051439803804\\
0.97021484375	6.36727319856981\\
0.970703125	6.36408431223538\\
0.97119140625	6.36094775827185\\
0.9716796875	6.3578635555909\\
0.97216796875	6.35483172277907\\
0.97265625	6.35185227809823\\
0.97314453125	6.34892523948581\\
0.9736328125	6.34605062455526\\
0.97412109375	6.34322845059635\\
0.974609375	6.34045873457545\\
0.97509765625	6.33774149313602\\
0.9755859375	6.33507674259875\\
0.97607421875	6.33246449896204\\
0.9765625	6.32990477790222\\
0.97705078125	6.32739759477387\\
0.9775390625	6.32494296461018\\
0.97802734375	6.32254090212321\\
0.978515625	6.32019142170419\\
0.97900390625	6.31789453742374\\
0.9794921875	6.3156502630323\\
0.97998046875	6.31345861196025\\
0.98046875	6.31131959731832\\
0.98095703125	6.3092332318977\\
0.9814453125	6.30719952817047\\
0.98193359375	6.30521849828963\\
0.982421875	6.30329015408959\\
0.98291015625	6.30141450708625\\
0.9833984375	6.29959156847725\\
0.98388671875	6.29782134914225\\
0.984375	6.29610385964309\\
0.98486328125	6.2944391102241\\
0.9853515625	6.29282711081215\\
0.98583984375	6.29126787101701\\
0.986328125	6.28976140013144\\
0.98681640625	6.28830770713145\\
0.9873046875	6.28690680067643\\
0.98779296875	6.28555868910934\\
0.98828125	6.2842633804569\\
0.98876953125	6.28302088242972\\
0.9892578125	6.28183120242251\\
0.98974609375	6.28069434751417\\
0.990234375	6.27961032446799\\
0.99072265625	6.27857913973175\\
0.9912109375	6.27760079943785\\
0.99169921875	6.27667530940347\\
0.9921875	6.27580267513066\\
0.99267578125	6.27498290180647\\
0.9931640625	6.27421599430304\\
0.99365234375	6.2735019571777\\
0.994140625	6.2728407946731\\
0.99462890625	6.27223251071726\\
0.9951171875	6.27167710892365\\
0.99560546875	6.27117459259128\\
0.99609375	6.27072496470476\\
0.99658203125	6.2703282279344\\
0.9970703125	6.26998438463616\\
0.99755859375	6.26969343685183\\
0.998046875	6.26945538630899\\
0.99853515625	6.26927023442106\\
0.9990234375	6.26913798228735\\
0.99951171875	6.26905863069305\\
};
\addlegendentry{AR(14) Model};

\end{axis}
\end{tikzpicture}%}
	\caption{\textit{PSD of Signal, and Estimated Models for varying orders}}
	\label{fig:2_2_b}
\end{figure}


\subsubsection{Increasing data length when modelling}

Figure \ref{fig:2_2_c} shows the same plot as before, but this time with 10,000 samples rather than 1,000 as before. We can see clear under modelling with an attempted AR(2) process - it shows only one peak rather than the two in the data. Once again, there appears to be little value from over modelling since the extra weights appear to do very little to improve the estimate of the system. As before, the best fit does to be from the AR(5) estimate.

\begin{figure}[h]
	\centering 
	\resizebox{\textwidth}{!}{% This file was created by matlab2tikz v0.4.7 (commit a43cd4b78840fd166f3a8d462e163c30134293e1) running on MATLAB 8.3.
% Copyright (c) 2008--2014, Nico Schlömer <nico.schloemer@gmail.com>
% All rights reserved.
% Minimal pgfplots version: 1.3
% 
% The latest updates can be retrieved from
%   http://www.mathworks.com/matlabcentral/fileexchange/22022-matlab2tikz
% where you can also make suggestions and rate matlab2tikz.
% 
%
% defining custom colors
\definecolor{mycolor1}{rgb}{0.00000,0.00000,0.17241}%
\definecolor{mycolor2}{rgb}{1.00000,0.10345,0.72414}%
\definecolor{mycolor3}{rgb}{1.00000,0.82759,0.00000}%
\definecolor{mycolor4}{rgb}{0.00000,0.34483,0.00000}%
%
\begin{tikzpicture}

\begin{axis}[%
width=15.5in,
height=8.18395833333333in,
scale only axis,
xmin=0,
xmax=0.5,
xlabel={Normalised Frequency},
xmajorgrids,
ymin=0,
ymax=180,
ylabel={Power (dB)},
ymajorgrids,
axis x line*=bottom,
axis y line*=left,
legend style={draw=black,fill=white,legend cell align=left}
]
\addplot [color=blue,solid,forget plot]
  table[row sep=crcr]{-1	49.3427109505669\\
-0.99951171875	33.4840305938025\\
-0.9990234375	42.1004924262986\\
-0.99853515625	35.5026003346468\\
-0.998046875	41.6441182362849\\
-0.99755859375	40.2606514159014\\
-0.9970703125	49.2741246995188\\
-0.99658203125	44.5163746126886\\
-0.99609375	46.4095986210217\\
-0.99560546875	43.2942216052166\\
-0.9951171875	39.5683654331709\\
-0.99462890625	31.128277680316\\
-0.994140625	46.4321189774601\\
-0.99365234375	47.8293916881961\\
-0.9931640625	51.2178736272306\\
-0.99267578125	37.1319099526317\\
-0.9921875	40.5894169185746\\
-0.99169921875	43.9368999515923\\
-0.9912109375	41.7319643800855\\
-0.99072265625	36.8458296397267\\
-0.990234375	45.1908882736036\\
-0.98974609375	39.2833111010071\\
-0.9892578125	45.5700282456091\\
-0.98876953125	41.4517049684532\\
-0.98828125	39.3230207667723\\
-0.98779296875	47.8069882123036\\
-0.9873046875	19.0755981506559\\
-0.98681640625	47.8240529092333\\
-0.986328125	44.8308108339865\\
-0.98583984375	38.6369050630594\\
-0.9853515625	48.9993635218621\\
-0.98486328125	48.0389451578801\\
-0.984375	36.1606327484333\\
-0.98388671875	45.2547770290636\\
-0.9833984375	41.8884548151595\\
-0.98291015625	40.0065254206727\\
-0.982421875	43.8991569254168\\
-0.98193359375	33.2833856890598\\
-0.9814453125	11.2288090335758\\
-0.98095703125	30.5224196159158\\
-0.98046875	37.0913948145877\\
-0.97998046875	46.1158631827025\\
-0.9794921875	43.0998945638969\\
-0.97900390625	32.7941377820616\\
-0.978515625	36.9770710000144\\
-0.97802734375	42.1153432971061\\
-0.9775390625	40.7505833066707\\
-0.97705078125	19.0614117468029\\
-0.9765625	44.7960494772879\\
-0.97607421875	42.0789715404249\\
-0.9755859375	47.7210512660441\\
-0.97509765625	37.2670479922896\\
-0.974609375	24.0949207684338\\
-0.97412109375	36.0494845783666\\
-0.9736328125	40.1466116153944\\
-0.97314453125	44.014043688205\\
-0.97265625	32.1625749100945\\
-0.97216796875	41.7715979075418\\
-0.9716796875	42.8983073415597\\
-0.97119140625	42.9918377762979\\
-0.970703125	41.1951466727692\\
-0.97021484375	44.1582784140181\\
-0.9697265625	44.7165421403155\\
-0.96923828125	40.0744287954956\\
-0.96875	45.1637004079248\\
-0.96826171875	6.3844416562604\\
-0.9677734375	52.6522312509704\\
-0.96728515625	42.1535625844876\\
-0.966796875	34.9725829463866\\
-0.96630859375	36.5226204382661\\
-0.9658203125	32.888064847605\\
-0.96533203125	40.761947026775\\
-0.96484375	48.6427963900913\\
-0.96435546875	44.2973093352907\\
-0.9638671875	49.0058196533875\\
-0.96337890625	32.5906587469286\\
-0.962890625	50.0615448289131\\
-0.96240234375	37.9374741245416\\
-0.9619140625	44.190667109304\\
-0.96142578125	37.6567108134431\\
-0.9609375	44.8737921923561\\
-0.96044921875	39.9976881187829\\
-0.9599609375	48.4063109954674\\
-0.95947265625	46.9963860300816\\
-0.958984375	45.2504167322536\\
-0.95849609375	48.9743792087344\\
-0.9580078125	35.6262884034752\\
-0.95751953125	31.0515881390306\\
-0.95703125	38.7740793762205\\
-0.95654296875	47.0910799363296\\
-0.9560546875	42.5888502961844\\
-0.95556640625	48.9231467096276\\
-0.955078125	40.5752989014799\\
-0.95458984375	29.9422970079237\\
-0.9541015625	44.4100226630723\\
-0.95361328125	43.8742005727595\\
-0.953125	44.4247771358031\\
-0.95263671875	48.6631125240989\\
-0.9521484375	39.8451106819847\\
-0.95166015625	44.1742698475801\\
-0.951171875	44.7255648305601\\
-0.95068359375	44.1768232597441\\
-0.9501953125	44.3265725447562\\
-0.94970703125	22.1931390588425\\
-0.94921875	37.6085006141245\\
-0.94873046875	33.1304075205712\\
-0.9482421875	48.1660744551045\\
-0.94775390625	43.6440108038541\\
-0.947265625	43.799831745773\\
-0.94677734375	48.4157765027086\\
-0.9462890625	44.2900122429693\\
-0.94580078125	41.1330535770987\\
-0.9453125	49.3018186507113\\
-0.94482421875	45.1957485836251\\
-0.9443359375	43.6734202735089\\
-0.94384765625	41.4256168920519\\
-0.943359375	42.2982161580112\\
-0.94287109375	41.9021046638898\\
-0.9423828125	48.5563649607792\\
-0.94189453125	44.2246101520713\\
-0.94140625	33.1622252640878\\
-0.94091796875	44.6206638394438\\
-0.9404296875	36.3967920755677\\
-0.93994140625	45.8198643726858\\
-0.939453125	39.6961112689375\\
-0.93896484375	34.3802406765181\\
-0.9384765625	48.8430161652781\\
-0.93798828125	42.0752753709047\\
-0.9375	37.9816046159824\\
-0.93701171875	44.6694918923472\\
-0.9365234375	48.0890799331403\\
-0.93603515625	49.7412747178363\\
-0.935546875	38.2977837296855\\
-0.93505859375	48.7057640475983\\
-0.9345703125	34.6033242820394\\
-0.93408203125	35.8944761189388\\
-0.93359375	43.5295219121178\\
-0.93310546875	43.4119050977967\\
-0.9326171875	27.1134314328944\\
-0.93212890625	39.9110936827676\\
-0.931640625	49.186844330073\\
-0.93115234375	44.0115217986796\\
-0.9306640625	32.2826003820904\\
-0.93017578125	48.6143675267679\\
-0.9296875	40.839618234604\\
-0.92919921875	40.1888111773121\\
-0.9287109375	48.5045284186818\\
-0.92822265625	42.9269391852168\\
-0.927734375	35.7288544015329\\
-0.92724609375	46.1802346305035\\
-0.9267578125	43.0494958299405\\
-0.92626953125	44.4985935588442\\
-0.92578125	44.0670302420818\\
-0.92529296875	51.3952995314278\\
-0.9248046875	35.8666688786712\\
-0.92431640625	36.3853545350795\\
-0.923828125	29.0733729553999\\
-0.92333984375	43.4360858402474\\
-0.9228515625	38.0902294750456\\
-0.92236328125	44.6792056826809\\
-0.921875	44.6446817627615\\
-0.92138671875	32.4853741242304\\
-0.9208984375	41.4739038925711\\
-0.92041015625	48.2994884180402\\
-0.919921875	48.3063934175263\\
-0.91943359375	46.1216138591101\\
-0.9189453125	45.8349199497581\\
-0.91845703125	33.9170924390603\\
-0.91796875	42.1649119802676\\
-0.91748046875	43.439874205645\\
-0.9169921875	37.6265266986484\\
-0.91650390625	51.5698967625014\\
-0.916015625	40.5972874911962\\
-0.91552734375	19.0651083267633\\
-0.9150390625	47.5934632733967\\
-0.91455078125	39.0551606720821\\
-0.9140625	38.985172537933\\
-0.91357421875	39.6519131954655\\
-0.9130859375	46.2214922775729\\
-0.91259765625	39.856418838281\\
-0.912109375	44.4046197478882\\
-0.91162109375	47.4475073382877\\
-0.9111328125	14.4740413428112\\
-0.91064453125	24.3999253685423\\
-0.91015625	34.9774253826526\\
-0.90966796875	38.2151269010226\\
-0.9091796875	44.418354029437\\
-0.90869140625	18.1543291494573\\
-0.908203125	45.2757677270496\\
-0.90771484375	39.6017892998246\\
-0.9072265625	50.9174721320466\\
-0.90673828125	42.6001597037451\\
-0.90625	38.6233943259061\\
-0.90576171875	45.7427027037169\\
-0.9052734375	31.9298030536337\\
-0.90478515625	37.9572524390508\\
-0.904296875	36.128036809088\\
-0.90380859375	47.5832252690538\\
-0.9033203125	46.6465084707205\\
-0.90283203125	41.0474318592352\\
-0.90234375	21.3583207164495\\
-0.90185546875	36.2529840501327\\
-0.9013671875	28.2668277236384\\
-0.90087890625	45.4371921152521\\
-0.900390625	32.6071388810485\\
-0.89990234375	45.6833490845074\\
-0.8994140625	39.7529056964156\\
-0.89892578125	41.9261514221606\\
-0.8984375	51.6679158246315\\
-0.89794921875	46.9722829418137\\
-0.8974609375	46.0448906672066\\
-0.89697265625	40.4577128612634\\
-0.896484375	50.8418483283311\\
-0.89599609375	34.1333132750245\\
-0.8955078125	42.304314838272\\
-0.89501953125	46.1041627811392\\
-0.89453125	39.4712357717702\\
-0.89404296875	46.2023759509688\\
-0.8935546875	46.2520480500095\\
-0.89306640625	43.4009893383274\\
-0.892578125	32.2807131482769\\
-0.89208984375	30.4839843776754\\
-0.8916015625	39.4579223397516\\
-0.89111328125	41.4365480729974\\
-0.890625	31.9534204691854\\
-0.89013671875	46.1382544923271\\
-0.8896484375	34.0621394477473\\
-0.88916015625	34.3320372534738\\
-0.888671875	40.64737487316\\
-0.88818359375	41.0885697113881\\
-0.8876953125	41.7096580109791\\
-0.88720703125	37.8526994350155\\
-0.88671875	45.0438897588422\\
-0.88623046875	46.3635447007061\\
-0.8857421875	40.8060570092395\\
-0.88525390625	33.3585390527398\\
-0.884765625	41.2638894958045\\
-0.88427734375	45.8632962921911\\
-0.8837890625	45.7555980575179\\
-0.88330078125	35.7518149546364\\
-0.8828125	42.7341322610453\\
-0.88232421875	40.0889142388846\\
-0.8818359375	33.8206513845963\\
-0.88134765625	34.0298057994759\\
-0.880859375	34.0980049772003\\
-0.88037109375	45.8984246777557\\
-0.8798828125	31.0262844653759\\
-0.87939453125	50.6034846541222\\
-0.87890625	46.2895752953958\\
-0.87841796875	43.2253814051419\\
-0.8779296875	35.7172136051303\\
-0.87744140625	51.7966127105027\\
-0.876953125	43.1437377729054\\
-0.87646484375	32.6314557639789\\
-0.8759765625	39.8974018855034\\
-0.87548828125	44.6192910245592\\
-0.875	49.6874234398371\\
-0.87451171875	47.9168826773152\\
-0.8740234375	43.0445726715209\\
-0.87353515625	46.7072218715588\\
-0.873046875	42.2115522459495\\
-0.87255859375	40.0221701728291\\
-0.8720703125	42.5838477540573\\
-0.87158203125	37.1675251486371\\
-0.87109375	46.8395055563567\\
-0.87060546875	41.9989122370186\\
-0.8701171875	41.8939496002963\\
-0.86962890625	47.112003691427\\
-0.869140625	47.0560073053347\\
-0.86865234375	41.5425934088991\\
-0.8681640625	46.2431791983483\\
-0.86767578125	46.7756400277902\\
-0.8671875	42.3404382755055\\
-0.86669921875	47.7341289773772\\
-0.8662109375	45.9983558789879\\
-0.86572265625	44.5850753810241\\
-0.865234375	43.1910917361763\\
-0.86474609375	43.6327988175179\\
-0.8642578125	33.6843266074023\\
-0.86376953125	55.1431154683351\\
-0.86328125	40.1750018961916\\
-0.86279296875	46.2950444746345\\
-0.8623046875	40.2639062157054\\
-0.86181640625	41.6149868290122\\
-0.861328125	35.7373319565693\\
-0.86083984375	49.5566229186666\\
-0.8603515625	49.1476279063016\\
-0.85986328125	37.0695567496114\\
-0.859375	43.2066610843906\\
-0.85888671875	37.3625836224127\\
-0.8583984375	41.414571327357\\
-0.85791015625	40.534671157303\\
-0.857421875	51.7711220039768\\
-0.85693359375	33.3126114828893\\
-0.8564453125	37.3817956948407\\
-0.85595703125	38.6575523503968\\
-0.85546875	48.6951899361264\\
-0.85498046875	47.5563315647991\\
-0.8544921875	28.4719738591245\\
-0.85400390625	26.0342765302261\\
-0.853515625	16.2838887351161\\
-0.85302734375	33.5198577154312\\
-0.8525390625	46.1581896471277\\
-0.85205078125	37.9747529039133\\
-0.8515625	13.8513846827674\\
-0.85107421875	46.647448128666\\
-0.8505859375	43.2294160794901\\
-0.85009765625	42.0578448561496\\
-0.849609375	47.2300790884342\\
-0.84912109375	43.8888665602796\\
-0.8486328125	39.3838558720086\\
-0.84814453125	12.6125892671741\\
-0.84765625	44.8421760186132\\
-0.84716796875	35.58851083777\\
-0.8466796875	29.8754851178666\\
-0.84619140625	53.1759144828367\\
-0.845703125	50.5150279971028\\
-0.84521484375	51.6716480093775\\
-0.8447265625	41.9852035468264\\
-0.84423828125	39.6933767331735\\
-0.84375	50.6604682580075\\
-0.84326171875	44.6544472583308\\
-0.8427734375	-7.70736069909414\\
-0.84228515625	43.5516205139502\\
-0.841796875	41.4131381554837\\
-0.84130859375	42.9069178257334\\
-0.8408203125	44.742205181404\\
-0.84033203125	36.9464653164577\\
-0.83984375	34.8442798380436\\
-0.83935546875	41.8120652968637\\
-0.8388671875	48.6960585392652\\
-0.83837890625	38.9535417754841\\
-0.837890625	54.0088444354714\\
-0.83740234375	46.4828752365534\\
-0.8369140625	37.242170791997\\
-0.83642578125	42.6398445749652\\
-0.8359375	40.1425488596297\\
-0.83544921875	42.5077285447197\\
-0.8349609375	43.4806263446704\\
-0.83447265625	47.8554099016566\\
-0.833984375	42.8375173715199\\
-0.83349609375	40.8429735781106\\
-0.8330078125	29.8354821682162\\
-0.83251953125	44.5173164059295\\
-0.83203125	52.5613629055855\\
-0.83154296875	38.3115758603886\\
-0.8310546875	42.3428010970264\\
-0.83056640625	50.5474399788327\\
-0.830078125	53.0803673879827\\
-0.82958984375	30.0410768343416\\
-0.8291015625	54.4823529587517\\
-0.82861328125	43.3568020482249\\
-0.828125	43.1395934094635\\
-0.82763671875	43.2166404038611\\
-0.8271484375	46.8927562202544\\
-0.82666015625	36.1566318766863\\
-0.826171875	42.8574505213743\\
-0.82568359375	47.0889422722876\\
-0.8251953125	39.7945221865064\\
-0.82470703125	35.8432395586108\\
-0.82421875	41.0195696160033\\
-0.82373046875	25.5499985658215\\
-0.8232421875	44.7913049750995\\
-0.82275390625	53.7000999910201\\
-0.822265625	43.6916458137032\\
-0.82177734375	35.1077249663378\\
-0.8212890625	49.1971265802953\\
-0.82080078125	39.2538251515482\\
-0.8203125	32.9991165770075\\
-0.81982421875	40.0096925284874\\
-0.8193359375	45.2606863359434\\
-0.81884765625	47.7004637344394\\
-0.818359375	49.3094612309956\\
-0.81787109375	44.7184683672503\\
-0.8173828125	43.5253525482506\\
-0.81689453125	45.9199738542663\\
-0.81640625	53.9706024604443\\
-0.81591796875	56.2464929710623\\
-0.8154296875	50.6908376729855\\
-0.81494140625	34.4776631129435\\
-0.814453125	40.0236079512066\\
-0.81396484375	29.3522100703595\\
-0.8134765625	42.4873549421427\\
-0.81298828125	41.8370505375378\\
-0.8125	40.0532618496296\\
-0.81201171875	43.2521764176953\\
-0.8115234375	41.7906266218909\\
-0.81103515625	43.8936986617083\\
-0.810546875	41.0873236812038\\
-0.81005859375	41.8223567541635\\
-0.8095703125	51.0313879744667\\
-0.80908203125	12.5975737980161\\
-0.80859375	47.9158317432528\\
-0.80810546875	41.1465659601062\\
-0.8076171875	46.6738707975027\\
-0.80712890625	43.2675155603728\\
-0.806640625	54.0709052751848\\
-0.80615234375	53.3127323367903\\
-0.8056640625	37.4669614381221\\
-0.80517578125	43.7680079781556\\
-0.8046875	32.5937821380725\\
-0.80419921875	49.3679339759681\\
-0.8037109375	41.628363365925\\
-0.80322265625	44.5545085941169\\
-0.802734375	38.7203598778779\\
-0.80224609375	28.0123124581705\\
-0.8017578125	47.7406358922088\\
-0.80126953125	29.1965416877666\\
-0.80078125	38.9961932409505\\
-0.80029296875	42.0757035281821\\
-0.7998046875	48.9637805316989\\
-0.79931640625	45.4108864103539\\
-0.798828125	40.2747003408485\\
-0.79833984375	46.6695470871023\\
-0.7978515625	44.0580655989013\\
-0.79736328125	56.2800383601444\\
-0.796875	37.9851780053435\\
-0.79638671875	51.6604965249402\\
-0.7958984375	44.3504411057627\\
-0.79541015625	47.5906418082669\\
-0.794921875	42.9082501041091\\
-0.79443359375	42.9426924530696\\
-0.7939453125	29.5316044110145\\
-0.79345703125	47.1527910287819\\
-0.79296875	51.7487121540236\\
-0.79248046875	45.7315083972733\\
-0.7919921875	53.9754839842326\\
-0.79150390625	41.4067199544535\\
-0.791015625	30.8886604069447\\
-0.79052734375	37.3618911729128\\
-0.7900390625	40.8355013078969\\
-0.78955078125	30.970955857904\\
-0.7890625	47.8696646108448\\
-0.78857421875	45.348470136215\\
-0.7880859375	35.0653200982495\\
-0.78759765625	47.5326202847897\\
-0.787109375	46.4382079720232\\
-0.78662109375	49.7455510144906\\
-0.7861328125	21.7670270797915\\
-0.78564453125	50.3756218208517\\
-0.78515625	45.6466379723518\\
-0.78466796875	34.6692413879144\\
-0.7841796875	47.6301427369623\\
-0.78369140625	33.3167815467614\\
-0.783203125	46.4848641226904\\
-0.78271484375	44.8310078790428\\
-0.7822265625	46.7445840235042\\
-0.78173828125	52.5483939271214\\
-0.78125	54.9615361185452\\
-0.78076171875	43.515319166761\\
-0.7802734375	45.7295660676777\\
-0.77978515625	53.0970386510777\\
-0.779296875	18.8876233865461\\
-0.77880859375	43.119612049402\\
-0.7783203125	37.619120795861\\
-0.77783203125	43.9782914869546\\
-0.77734375	41.6986165203761\\
-0.77685546875	54.0601595026151\\
-0.7763671875	32.2155229486885\\
-0.77587890625	38.3998350397347\\
-0.775390625	47.7903072793443\\
-0.77490234375	38.0361522198501\\
-0.7744140625	46.7973013884585\\
-0.77392578125	45.4968368638027\\
-0.7734375	23.2466141389757\\
-0.77294921875	49.2236229145295\\
-0.7724609375	55.375031640778\\
-0.77197265625	52.3766972325422\\
-0.771484375	47.5754310481857\\
-0.77099609375	40.6450238867491\\
-0.7705078125	37.428174571647\\
-0.77001953125	50.2283963186616\\
-0.76953125	45.3338447300032\\
-0.76904296875	50.1508199969704\\
-0.7685546875	47.1884272284191\\
-0.76806640625	53.1627986583842\\
-0.767578125	50.2485541041187\\
-0.76708984375	24.9430073087264\\
-0.7666015625	42.9836954599725\\
-0.76611328125	50.0971256671615\\
-0.765625	11.2287551386999\\
-0.76513671875	41.3786737436848\\
-0.7646484375	42.6221860942878\\
-0.76416015625	44.2074425700368\\
-0.763671875	43.8000543536494\\
-0.76318359375	28.5164502535479\\
-0.7626953125	33.5366445926464\\
-0.76220703125	33.6569553955504\\
-0.76171875	28.3446696744649\\
-0.76123046875	19.7216112656354\\
-0.7607421875	46.0114559733171\\
-0.76025390625	49.7302364735323\\
-0.759765625	47.2596176560027\\
-0.75927734375	48.7044934690014\\
-0.7587890625	35.292091826408\\
-0.75830078125	29.2898638263873\\
-0.7578125	54.0635627492326\\
-0.75732421875	43.8274454658471\\
-0.7568359375	40.150104032345\\
-0.75634765625	23.087865421113\\
-0.755859375	43.3733593018383\\
-0.75537109375	40.9036115728959\\
-0.7548828125	46.399031565173\\
-0.75439453125	33.3257516826129\\
-0.75390625	46.5472987447314\\
-0.75341796875	47.1987423090936\\
-0.7529296875	44.9950463683825\\
-0.75244140625	33.8663751367907\\
-0.751953125	47.5443549084638\\
-0.75146484375	30.6301829435467\\
-0.7509765625	52.3351729975746\\
-0.75048828125	52.6234323115871\\
-0.75	32.7456041501024\\
-0.74951171875	34.4848862528429\\
-0.7490234375	43.9058406537383\\
-0.74853515625	49.2229701179269\\
-0.748046875	35.3728773022151\\
-0.74755859375	45.6466111533303\\
-0.7470703125	47.1997311457118\\
-0.74658203125	44.7080031109137\\
-0.74609375	25.3581352659551\\
-0.74560546875	36.9478921255809\\
-0.7451171875	39.5355111249197\\
-0.74462890625	45.6259097784978\\
-0.744140625	30.3028910667034\\
-0.74365234375	53.6079620656137\\
-0.7431640625	41.9488721230532\\
-0.74267578125	59.4687688601158\\
-0.7421875	24.3054935324122\\
-0.74169921875	50.129745558551\\
-0.7412109375	43.6225209554537\\
-0.74072265625	51.5053680360952\\
-0.740234375	44.1809319654986\\
-0.73974609375	2.46596186625821\\
-0.7392578125	50.9539301186158\\
-0.73876953125	47.1872317592519\\
-0.73828125	38.914658466152\\
-0.73779296875	50.7796820202199\\
-0.7373046875	47.798755622161\\
-0.73681640625	41.7337343238276\\
-0.736328125	42.5157902018265\\
-0.73583984375	18.8516068501103\\
-0.7353515625	27.3927176762964\\
-0.73486328125	45.486072265442\\
-0.734375	55.7853193558215\\
-0.73388671875	33.1471346576457\\
-0.7333984375	43.512754335408\\
-0.73291015625	28.7087855688821\\
-0.732421875	57.1953460147511\\
-0.73193359375	41.0947243896518\\
-0.7314453125	49.7809583531296\\
-0.73095703125	49.4029845478529\\
-0.73046875	42.3087853773626\\
-0.72998046875	29.7464276133951\\
-0.7294921875	47.6126475113583\\
-0.72900390625	30.9842095401946\\
-0.728515625	40.8404128182723\\
-0.72802734375	51.6184412950253\\
-0.7275390625	28.8466932708145\\
-0.72705078125	53.0732283071195\\
-0.7265625	41.4478121445594\\
-0.72607421875	36.8104216664421\\
-0.7255859375	39.4163458658482\\
-0.72509765625	26.3574180810684\\
-0.724609375	45.3219773845359\\
-0.72412109375	51.9735696220134\\
-0.7236328125	15.8954243653898\\
-0.72314453125	37.3227246227807\\
-0.72265625	36.6059312475776\\
-0.72216796875	23.5339769427703\\
-0.7216796875	33.3658560353333\\
-0.72119140625	38.9271318000825\\
-0.720703125	33.3122565667728\\
-0.72021484375	25.9359332160916\\
-0.7197265625	53.9376442084043\\
-0.71923828125	51.2175469793842\\
-0.71875	53.1416227050374\\
-0.71826171875	48.2197358732332\\
-0.7177734375	50.9711052049141\\
-0.71728515625	29.6549528478027\\
-0.716796875	50.8242358116882\\
-0.71630859375	49.3616051867621\\
-0.7158203125	49.0072391033175\\
-0.71533203125	50.4178074167823\\
-0.71484375	26.4282187255651\\
-0.71435546875	32.5369642452176\\
-0.7138671875	56.2904658722117\\
-0.71337890625	42.8623983894626\\
-0.712890625	50.7223912693383\\
-0.71240234375	46.3103643435764\\
-0.7119140625	42.7233225973657\\
-0.71142578125	13.4897369121735\\
-0.7109375	42.7397319272285\\
-0.71044921875	48.5061257751547\\
-0.7099609375	37.3069810374299\\
-0.70947265625	35.8786941818442\\
-0.708984375	42.3703189031401\\
-0.70849609375	38.9451996754724\\
-0.7080078125	49.8257486197598\\
-0.70751953125	52.2511883582356\\
-0.70703125	46.773143842771\\
-0.70654296875	45.0285145769331\\
-0.7060546875	50.7322635106469\\
-0.70556640625	40.0556082225398\\
-0.705078125	48.1774149223896\\
-0.70458984375	15.8652955797172\\
-0.7041015625	36.4754562682157\\
-0.70361328125	43.0856516088732\\
-0.703125	35.6571806115267\\
-0.70263671875	39.7087269569436\\
-0.7021484375	53.2047208471266\\
-0.70166015625	57.1867765594311\\
-0.701171875	53.8957836482829\\
-0.70068359375	30.4270169224459\\
-0.7001953125	56.3105608616227\\
-0.69970703125	46.5424740191315\\
-0.69921875	31.8197724258283\\
-0.69873046875	52.5155534214216\\
-0.6982421875	42.0094769750635\\
-0.69775390625	43.3863781280846\\
-0.697265625	27.1367258434307\\
-0.69677734375	45.1028185465652\\
-0.6962890625	41.6107179884611\\
-0.69580078125	45.2999629740438\\
-0.6953125	52.7144046638776\\
-0.69482421875	49.1461975669638\\
-0.6943359375	47.5481148043597\\
-0.69384765625	46.6061579894786\\
-0.693359375	21.5556460954392\\
-0.69287109375	28.1345961726443\\
-0.6923828125	59.6035067494399\\
-0.69189453125	50.3241629463028\\
-0.69140625	50.27415689325\\
-0.69091796875	28.3060236531429\\
-0.6904296875	41.1318961278602\\
-0.68994140625	49.5160210121873\\
-0.689453125	44.2567884734635\\
-0.68896484375	54.8721844965755\\
-0.6884765625	51.015287910334\\
-0.68798828125	36.984393040232\\
-0.6875	33.4062317897272\\
-0.68701171875	51.176017186954\\
-0.6865234375	55.4067579681118\\
-0.68603515625	22.2162146884655\\
-0.685546875	38.4218372700105\\
-0.68505859375	39.4873620535556\\
-0.6845703125	40.9558094201884\\
-0.68408203125	54.2834670715726\\
-0.68359375	26.6208991087815\\
-0.68310546875	48.0618686540546\\
-0.6826171875	34.4251795763047\\
-0.68212890625	39.5591897146345\\
-0.681640625	40.7897079245983\\
-0.68115234375	55.9188960200147\\
-0.6806640625	30.7616533163645\\
-0.68017578125	44.9748168060549\\
-0.6796875	51.1718566995732\\
-0.67919921875	36.851534859214\\
-0.6787109375	52.4742381168915\\
-0.67822265625	40.7445939237692\\
-0.677734375	58.0188918025352\\
-0.67724609375	40.8926386662004\\
-0.6767578125	54.7420348625918\\
-0.67626953125	39.1058058504011\\
-0.67578125	49.2675498432204\\
-0.67529296875	50.8250119332707\\
-0.6748046875	32.5611871954283\\
-0.67431640625	50.3155726504501\\
-0.673828125	49.0310380261034\\
-0.67333984375	55.6867130569078\\
-0.6728515625	49.5508066154951\\
-0.67236328125	35.0894053454286\\
-0.671875	27.5314988811155\\
-0.67138671875	40.3989605843218\\
-0.6708984375	22.609911178992\\
-0.67041015625	51.288640356667\\
-0.669921875	24.4586943975186\\
-0.66943359375	33.0182197860015\\
-0.6689453125	44.8433486315044\\
-0.66845703125	-27.2321737296978\\
-0.66796875	41.7308611331984\\
-0.66748046875	54.2102255243619\\
-0.6669921875	50.2299604458079\\
-0.66650390625	47.1887536745604\\
-0.666015625	33.5319325818427\\
-0.66552734375	55.283815050965\\
-0.6650390625	44.7664692714724\\
-0.66455078125	40.0645825909009\\
-0.6640625	45.1074404985612\\
-0.66357421875	42.5772934569266\\
-0.6630859375	54.3711730314302\\
-0.66259765625	46.6136893181259\\
-0.662109375	22.4425003185645\\
-0.66162109375	48.9312583154945\\
-0.6611328125	55.1672396053581\\
-0.66064453125	38.8229806195887\\
-0.66015625	38.9880715222038\\
-0.65966796875	48.9761244004859\\
-0.6591796875	37.5276327619808\\
-0.65869140625	47.3292116215126\\
-0.658203125	49.2412992782773\\
-0.65771484375	54.2217061307459\\
-0.6572265625	50.7268630574516\\
-0.65673828125	40.8102785575504\\
-0.65625	42.9599200090824\\
-0.65576171875	53.7715662437469\\
-0.6552734375	49.0943068879421\\
-0.65478515625	49.9819976586184\\
-0.654296875	48.0459487040736\\
-0.65380859375	54.8433040210523\\
-0.6533203125	43.4793811374281\\
-0.65283203125	52.2343864389611\\
-0.65234375	51.1196156090288\\
-0.65185546875	38.1847032554012\\
-0.6513671875	50.5338878544916\\
-0.65087890625	34.5185054149235\\
-0.650390625	46.2258106558776\\
-0.64990234375	53.4920429691415\\
-0.6494140625	52.0450125720173\\
-0.64892578125	30.1514385596209\\
-0.6484375	50.4993579364324\\
-0.64794921875	58.0129186444669\\
-0.6474609375	44.251346109368\\
-0.64697265625	42.700694300533\\
-0.646484375	57.6034084272752\\
-0.64599609375	57.4221705778324\\
-0.6455078125	37.5478057192036\\
-0.64501953125	21.8641184909345\\
-0.64453125	61.3792295193683\\
-0.64404296875	39.861838146095\\
-0.6435546875	48.0230205499542\\
-0.64306640625	53.3972586094227\\
-0.642578125	49.9027195847553\\
-0.64208984375	43.8506660516209\\
-0.6416015625	37.3481408830577\\
-0.64111328125	59.2479576189292\\
-0.640625	48.3527726118811\\
-0.64013671875	54.0744973050832\\
-0.6396484375	33.6875586325525\\
-0.63916015625	51.582115977299\\
-0.638671875	42.4572029794675\\
-0.63818359375	44.9859980415438\\
-0.6376953125	61.1599245881689\\
-0.63720703125	50.0925132988705\\
-0.63671875	52.5536743842082\\
-0.63623046875	54.8198759152316\\
-0.6357421875	59.1891502813304\\
-0.63525390625	40.1130062713107\\
-0.634765625	44.6491642139643\\
-0.63427734375	27.7009158172488\\
-0.6337890625	56.7133412230539\\
-0.63330078125	51.4771636774529\\
-0.6328125	59.2192499951171\\
-0.63232421875	49.480254704262\\
-0.6318359375	49.7640317562075\\
-0.63134765625	49.9365900622371\\
-0.630859375	54.593879571645\\
-0.63037109375	57.066219289165\\
-0.6298828125	34.4299135487945\\
-0.62939453125	52.5940949624744\\
-0.62890625	37.6568616603932\\
-0.62841796875	36.7025131141159\\
-0.6279296875	42.3478833887678\\
-0.62744140625	48.9362563475711\\
-0.626953125	43.302795842241\\
-0.62646484375	58.0785708903\\
-0.6259765625	27.972456135444\\
-0.62548828125	47.0458902559742\\
-0.625	43.9723654666295\\
-0.62451171875	41.0253721430279\\
-0.6240234375	48.7416320003331\\
-0.62353515625	51.7385271529744\\
-0.623046875	35.0074864057486\\
-0.62255859375	41.3601864965032\\
-0.6220703125	42.8488199368896\\
-0.62158203125	55.2635324046175\\
-0.62109375	54.2165743000606\\
-0.62060546875	53.2851725920465\\
-0.6201171875	57.8230737143636\\
-0.61962890625	44.0014308529737\\
-0.619140625	54.7150641980399\\
-0.61865234375	54.673710250738\\
-0.6181640625	61.9584187524219\\
-0.61767578125	48.4203142658834\\
-0.6171875	52.0721473421456\\
-0.61669921875	52.5103706877076\\
-0.6162109375	37.5571456093488\\
-0.61572265625	41.3995173288067\\
-0.615234375	56.958552978351\\
-0.61474609375	40.058450520527\\
-0.6142578125	43.9892044863667\\
-0.61376953125	57.8527843168726\\
-0.61328125	35.031583199012\\
-0.61279296875	42.6813625965345\\
-0.6123046875	35.135677903221\\
-0.61181640625	44.6641624675461\\
-0.611328125	51.4517482677845\\
-0.61083984375	47.3062802540942\\
-0.6103515625	40.1542959707111\\
-0.60986328125	32.1781986505957\\
-0.609375	50.3488304657699\\
-0.60888671875	42.0780190610069\\
-0.6083984375	44.0018091303676\\
-0.60791015625	64.3753726943666\\
-0.607421875	50.8445435583375\\
-0.60693359375	52.5150956635282\\
-0.6064453125	42.3992218644352\\
-0.60595703125	51.5960930902362\\
-0.60546875	53.2177179208383\\
-0.60498046875	58.814529297063\\
-0.6044921875	40.1586034000358\\
-0.60400390625	44.5519543740375\\
-0.603515625	59.8892125225032\\
-0.60302734375	45.6166889312374\\
-0.6025390625	53.4061056905319\\
-0.60205078125	50.9812153430712\\
-0.6015625	60.0340044373205\\
-0.60107421875	52.4031983838119\\
-0.6005859375	51.1413588785292\\
-0.60009765625	38.0367984429781\\
-0.599609375	60.6560058741345\\
-0.59912109375	46.2607342440165\\
-0.5986328125	58.8912869397482\\
-0.59814453125	38.8558592291756\\
-0.59765625	42.8790210360906\\
-0.59716796875	39.0779349443769\\
-0.5966796875	36.988034674143\\
-0.59619140625	57.2207890247478\\
-0.595703125	41.5084673720554\\
-0.59521484375	45.0953513701232\\
-0.5947265625	60.5407165743691\\
-0.59423828125	32.9965127781723\\
-0.59375	58.0159366905866\\
-0.59326171875	51.6849099503477\\
-0.5927734375	47.1065351603166\\
-0.59228515625	49.2457439109725\\
-0.591796875	43.5150620473082\\
-0.59130859375	57.3941294260301\\
-0.5908203125	25.2876545651028\\
-0.59033203125	38.297355680794\\
-0.58984375	56.7531784610841\\
-0.58935546875	52.0214658392224\\
-0.5888671875	51.3416161937445\\
-0.58837890625	41.6929615575858\\
-0.587890625	34.6576224874165\\
-0.58740234375	44.1054182455765\\
-0.5869140625	24.641669801539\\
-0.58642578125	55.1562538311948\\
-0.5859375	44.2381963736329\\
-0.58544921875	62.2177591756257\\
-0.5849609375	40.9265533858646\\
-0.58447265625	59.7390820926263\\
-0.583984375	14.797303972492\\
-0.58349609375	54.7169809787939\\
-0.5830078125	31.7488828852792\\
-0.58251953125	60.9382233073804\\
-0.58203125	46.616288299957\\
-0.58154296875	39.2270424577949\\
-0.5810546875	50.9221078254738\\
-0.58056640625	35.095703737201\\
-0.580078125	34.2675448565524\\
-0.57958984375	64.673234049342\\
-0.5791015625	63.2177119845114\\
-0.57861328125	57.0978559791041\\
-0.578125	52.2627646335648\\
-0.57763671875	55.1230056994888\\
-0.5771484375	50.031103229543\\
-0.57666015625	49.0341438687114\\
-0.576171875	45.6164849603436\\
-0.57568359375	52.3394719569775\\
-0.5751953125	51.5522795340378\\
-0.57470703125	63.160695693402\\
-0.57421875	55.4144881509307\\
-0.57373046875	61.793867021419\\
-0.5732421875	47.4290629116676\\
-0.57275390625	28.7623636941248\\
-0.572265625	48.5765991518103\\
-0.57177734375	51.1144443898591\\
-0.5712890625	67.7571996252966\\
-0.57080078125	45.3159647850074\\
-0.5703125	52.9230656781361\\
-0.56982421875	59.3540564071116\\
-0.5693359375	45.2036164754151\\
-0.56884765625	39.6813897587285\\
-0.568359375	35.7355700162374\\
-0.56787109375	60.7310779483362\\
-0.5673828125	57.1579620550524\\
-0.56689453125	51.6177948994478\\
-0.56640625	37.6891209301453\\
-0.56591796875	50.5411104778419\\
-0.5654296875	36.5555404037948\\
-0.56494140625	49.3034918697316\\
-0.564453125	64.6131709939013\\
-0.56396484375	55.147722752342\\
-0.5634765625	59.5464460605248\\
-0.56298828125	37.621925649119\\
-0.5625	54.0231231526663\\
-0.56201171875	51.4011282428522\\
-0.5615234375	56.8373479127688\\
-0.56103515625	37.8681577633396\\
-0.560546875	55.3990810152612\\
-0.56005859375	50.650236546264\\
-0.5595703125	56.3618006609499\\
-0.55908203125	41.8800414978164\\
-0.55859375	57.8817107472837\\
-0.55810546875	54.5569116802864\\
-0.5576171875	47.856061762906\\
-0.55712890625	53.3063513357231\\
-0.556640625	62.1681614403948\\
-0.55615234375	60.0838283319336\\
-0.5556640625	40.6721068794377\\
-0.55517578125	51.7674769726513\\
-0.5546875	53.5336972386606\\
-0.55419921875	49.2364005212963\\
-0.5537109375	45.5108954817255\\
-0.55322265625	47.60717238738\\
-0.552734375	52.1418422014848\\
-0.55224609375	46.4434155770191\\
-0.5517578125	59.5369141926766\\
-0.55126953125	57.9300285708023\\
-0.55078125	44.4171930113491\\
-0.55029296875	45.3331890863119\\
-0.5498046875	49.9413629487257\\
-0.54931640625	57.582586503763\\
-0.548828125	36.4614836024648\\
-0.54833984375	60.409851855089\\
-0.5478515625	38.9869434242087\\
-0.54736328125	59.8312202735549\\
-0.546875	40.4192508782062\\
-0.54638671875	53.0493881758388\\
-0.5458984375	54.8423361948711\\
-0.54541015625	66.8590869458186\\
-0.544921875	37.9566168537904\\
-0.54443359375	51.9415229531067\\
-0.5439453125	53.0852757641109\\
-0.54345703125	32.8720437619321\\
-0.54296875	55.8086245773392\\
-0.54248046875	66.010083182495\\
-0.5419921875	42.568294891254\\
-0.54150390625	60.4329160552536\\
-0.541015625	56.3425906510728\\
-0.54052734375	55.1687098678079\\
-0.5400390625	61.0756910119393\\
-0.53955078125	64.6761555577853\\
-0.5390625	65.0006147135874\\
-0.53857421875	59.9094920144068\\
-0.5380859375	64.038146420111\\
-0.53759765625	55.2365833874077\\
-0.537109375	30.3127248999677\\
-0.53662109375	52.7265002300318\\
-0.5361328125	54.341322530387\\
-0.53564453125	43.7551876190416\\
-0.53515625	64.0370765009604\\
-0.53466796875	37.8674578592657\\
-0.5341796875	64.1120474608094\\
-0.53369140625	59.7247179645559\\
-0.533203125	57.1164441263737\\
-0.53271484375	62.9010159812599\\
-0.5322265625	23.9890280925568\\
-0.53173828125	55.994121427409\\
-0.53125	45.3624601705454\\
-0.53076171875	43.5679055554657\\
-0.5302734375	67.8901035674373\\
-0.52978515625	55.215275686094\\
-0.529296875	58.9155927681137\\
-0.52880859375	36.3046506374415\\
-0.5283203125	67.1152824893005\\
-0.52783203125	63.5081870330274\\
-0.52734375	42.443956120862\\
-0.52685546875	56.2049701103465\\
-0.5263671875	28.7635061345244\\
-0.52587890625	50.7647045487276\\
-0.525390625	55.8896198811675\\
-0.52490234375	58.4816749085633\\
-0.5244140625	48.1717356183155\\
-0.52392578125	58.5125644078761\\
-0.5234375	64.442929847804\\
-0.52294921875	56.8218211837663\\
-0.5224609375	62.1859402522525\\
-0.52197265625	61.2647273048481\\
-0.521484375	66.1434819266742\\
-0.52099609375	44.8629716663984\\
-0.5205078125	46.9978520681803\\
-0.52001953125	65.536944490969\\
-0.51953125	61.2983160631103\\
-0.51904296875	55.9304722416927\\
-0.5185546875	56.2321815727253\\
-0.51806640625	48.1319917918437\\
-0.517578125	38.2528316724364\\
-0.51708984375	47.223362665054\\
-0.5166015625	68.4570307153526\\
-0.51611328125	62.137976380583\\
-0.515625	59.905806905265\\
-0.51513671875	75.2759677256233\\
-0.5146484375	53.2427562659799\\
-0.51416015625	53.6607164817719\\
-0.513671875	57.8366307336544\\
-0.51318359375	61.6391689335106\\
-0.5126953125	51.6967404437566\\
-0.51220703125	39.1794408699781\\
-0.51171875	71.6671268976006\\
-0.51123046875	46.3937516303101\\
-0.5107421875	61.3818643781663\\
-0.51025390625	66.5276628660472\\
-0.509765625	45.0438389786676\\
-0.50927734375	42.2492885826134\\
-0.5087890625	73.2752613923338\\
-0.50830078125	62.0877110237799\\
-0.5078125	60.3460382061071\\
-0.50732421875	46.0157870968864\\
-0.5068359375	51.4292299481309\\
-0.50634765625	67.3266311060325\\
-0.505859375	69.6107980528551\\
-0.50537109375	59.3317992853082\\
-0.5048828125	59.8794600070126\\
-0.50439453125	53.0748590897242\\
-0.50390625	48.4233735608439\\
-0.50341796875	61.5515389087488\\
-0.5029296875	66.3731522568525\\
-0.50244140625	50.8086690814547\\
-0.501953125	53.0846330952622\\
-0.50146484375	57.8203314924683\\
-0.5009765625	67.294355771\\
-0.50048828125	70.0743958791518\\
-0.5	67.1430981507742\\
-0.49951171875	67.2630992747107\\
-0.4990234375	42.5859926072694\\
-0.49853515625	40.3134941157197\\
-0.498046875	61.6021084447279\\
-0.49755859375	69.5924688380744\\
-0.4970703125	59.9472975198783\\
-0.49658203125	10.0978307381054\\
-0.49609375	59.2495016564297\\
-0.49560546875	62.3983880153342\\
-0.4951171875	47.4677084292134\\
-0.49462890625	33.6963489554577\\
-0.494140625	66.7862045690259\\
-0.49365234375	52.4678332455548\\
-0.4931640625	69.7494680454585\\
-0.49267578125	50.172579895204\\
-0.4921875	56.2893591294219\\
-0.49169921875	56.5333013508076\\
-0.4912109375	73.6234210798236\\
-0.49072265625	64.7206280544114\\
-0.490234375	50.9491923839463\\
-0.48974609375	53.9419954888665\\
-0.4892578125	67.4651676558712\\
-0.48876953125	70.9522841960966\\
-0.48828125	54.0461433344485\\
-0.48779296875	52.1332185410164\\
-0.4873046875	68.1172891112559\\
-0.48681640625	65.611684690462\\
-0.486328125	71.1239284787181\\
-0.48583984375	37.7652269142676\\
-0.4853515625	68.6555576438268\\
-0.48486328125	64.4377343922625\\
-0.484375	65.4104975308357\\
-0.48388671875	57.8547743064364\\
-0.4833984375	61.8038566716651\\
-0.48291015625	48.2446110016668\\
-0.482421875	65.4212032009274\\
-0.48193359375	47.4773870723709\\
-0.4814453125	75.8926499696775\\
-0.48095703125	70.77904163142\\
-0.48046875	58.2577950038163\\
-0.47998046875	75.1552932160021\\
-0.4794921875	74.4337707870397\\
-0.47900390625	61.1991752830357\\
-0.478515625	72.3318094800517\\
-0.47802734375	73.2753379634603\\
-0.4775390625	69.7064158577123\\
-0.47705078125	74.8930070237578\\
-0.4765625	39.2879984331628\\
-0.47607421875	62.9265483657594\\
-0.4755859375	59.9405176192656\\
-0.47509765625	57.0724728668928\\
-0.474609375	64.6657839425725\\
-0.47412109375	72.1382511413395\\
-0.4736328125	68.0728393658292\\
-0.47314453125	69.2234426335349\\
-0.47265625	49.9694589930238\\
-0.47216796875	77.4434567181829\\
-0.4716796875	74.0572852909179\\
-0.47119140625	52.4890083826809\\
-0.470703125	70.3557308006611\\
-0.47021484375	75.3507771594166\\
-0.4697265625	66.2733232044511\\
-0.46923828125	62.449540822704\\
-0.46875	54.925840592288\\
-0.46826171875	73.2452166486226\\
-0.4677734375	21.5793595471563\\
-0.46728515625	38.0860106624307\\
-0.466796875	69.955860075827\\
-0.46630859375	50.1828823503406\\
-0.4658203125	42.6568634136514\\
-0.46533203125	65.4414726208739\\
-0.46484375	70.1053278576637\\
-0.46435546875	46.0918061154613\\
-0.4638671875	60.0377507482099\\
-0.46337890625	67.9168444232513\\
-0.462890625	75.065106257554\\
-0.46240234375	75.376650598385\\
-0.4619140625	79.2115295425376\\
-0.46142578125	76.0415183356978\\
-0.4609375	49.9470617186775\\
-0.46044921875	73.2183970486448\\
-0.4599609375	66.9957965657868\\
-0.45947265625	67.5103141613899\\
-0.458984375	49.6908500533143\\
-0.45849609375	61.4473154128899\\
-0.4580078125	67.9992297833368\\
-0.45751953125	70.8827213400039\\
-0.45703125	77.0922088372337\\
-0.45654296875	30.6042421321089\\
-0.4560546875	55.2342505572461\\
-0.45556640625	76.5809766408455\\
-0.455078125	76.0366656762903\\
-0.45458984375	67.6694660557573\\
-0.4541015625	71.4005784934188\\
-0.45361328125	47.7172253660175\\
-0.453125	60.7083101249311\\
-0.45263671875	40.2210066708127\\
-0.4521484375	77.0498522360035\\
-0.45166015625	63.9647889767036\\
-0.451171875	71.9447767968937\\
-0.45068359375	59.9190421351614\\
-0.4501953125	71.977357591261\\
-0.44970703125	76.1540434082099\\
-0.44921875	77.1636691848826\\
-0.44873046875	68.6129212606423\\
-0.4482421875	80.1630880942974\\
-0.44775390625	62.112441358633\\
-0.447265625	75.8076425315059\\
-0.44677734375	68.5247076737053\\
-0.4462890625	62.0316743112908\\
-0.44580078125	75.4546272241189\\
-0.4453125	73.4483062098259\\
-0.44482421875	52.2036364602129\\
-0.4443359375	65.6842137027052\\
-0.44384765625	75.2060244603461\\
-0.443359375	75.8237276261749\\
-0.44287109375	45.5962463135675\\
-0.4423828125	62.8186134044609\\
-0.44189453125	70.3651188664651\\
-0.44140625	67.4694297209486\\
-0.44091796875	62.5440851392866\\
-0.4404296875	63.3330441546004\\
-0.43994140625	58.6363547868862\\
-0.439453125	64.095639103231\\
-0.43896484375	77.7015252423051\\
-0.4384765625	66.0890491378844\\
-0.43798828125	83.5169821083761\\
-0.4375	86.06625754829\\
-0.43701171875	70.0586610509453\\
-0.4365234375	80.7036490345797\\
-0.43603515625	78.4260984199121\\
-0.435546875	71.2617912131415\\
-0.43505859375	80.7884681628656\\
-0.4345703125	74.2360700244928\\
-0.43408203125	78.0595295664824\\
-0.43359375	73.6430315210161\\
-0.43310546875	56.5609546225338\\
-0.4326171875	72.8435896956287\\
-0.43212890625	74.097881846561\\
-0.431640625	82.8082650179666\\
-0.43115234375	63.5251193630812\\
-0.4306640625	58.7788192007979\\
-0.43017578125	69.0803399143592\\
-0.4296875	86.0551578571734\\
-0.42919921875	86.902273990686\\
-0.4287109375	75.5585510817399\\
-0.42822265625	66.758146304333\\
-0.427734375	69.5887717909758\\
-0.42724609375	55.533517670437\\
-0.4267578125	71.1002049748954\\
-0.42626953125	69.3074794854417\\
-0.42578125	69.075551269174\\
-0.42529296875	77.1705489496598\\
-0.4248046875	73.1043652270394\\
-0.42431640625	68.6893492952118\\
-0.423828125	60.130386323553\\
-0.42333984375	43.5144780415587\\
-0.4228515625	69.9746946141252\\
-0.42236328125	85.5736161957137\\
-0.421875	75.563161313381\\
-0.42138671875	71.8694479635809\\
-0.4208984375	70.5587885269105\\
-0.42041015625	57.2483704978341\\
-0.419921875	77.879197244891\\
-0.41943359375	72.9862875973438\\
-0.4189453125	77.4319698439434\\
-0.41845703125	77.4992733626283\\
-0.41796875	59.2602304399433\\
-0.41748046875	85.4767660089344\\
-0.4169921875	64.5081225147501\\
-0.41650390625	68.099587254946\\
-0.416015625	83.5438425328191\\
-0.41552734375	42.2605411807069\\
-0.4150390625	73.3726680562014\\
-0.41455078125	82.6965764971593\\
-0.4140625	61.4407319600309\\
-0.41357421875	82.1947590945089\\
-0.4130859375	63.5763461989805\\
-0.41259765625	54.7014223415067\\
-0.412109375	82.1796885553442\\
-0.41162109375	79.1744123253081\\
-0.4111328125	61.6881129263248\\
-0.41064453125	70.9068555953925\\
-0.41015625	71.8711543125326\\
-0.40966796875	79.3408017502739\\
-0.4091796875	78.8032790506771\\
-0.40869140625	82.3595184004207\\
-0.408203125	76.2926818755215\\
-0.40771484375	78.5753172744681\\
-0.4072265625	62.9819475920066\\
-0.40673828125	93.2592751099308\\
-0.40625	80.7434493644305\\
-0.40576171875	74.6110115683929\\
-0.4052734375	74.8261071238308\\
-0.40478515625	57.831623502328\\
-0.404296875	75.4981213314299\\
-0.40380859375	81.6836379936231\\
-0.4033203125	81.8603977883454\\
-0.40283203125	78.7306336709893\\
-0.40234375	59.370411373423\\
-0.40185546875	89.0279093776633\\
-0.4013671875	67.89331190565\\
-0.40087890625	70.5414856531707\\
-0.400390625	70.1462769474692\\
-0.39990234375	77.5444548027143\\
-0.3994140625	95.8832153725556\\
-0.39892578125	69.6272580596179\\
-0.3984375	76.0291072648271\\
-0.39794921875	72.7481613728504\\
-0.3974609375	72.3691621519633\\
-0.39697265625	83.1794865049716\\
-0.396484375	72.5886606390524\\
-0.39599609375	87.039453094029\\
-0.3955078125	84.9633299657106\\
-0.39501953125	63.2382124304666\\
-0.39453125	75.9266793450577\\
-0.39404296875	78.3843411257821\\
-0.3935546875	95.1163036224532\\
-0.39306640625	79.4463479389155\\
-0.392578125	48.3897643241262\\
-0.39208984375	84.8301268148623\\
-0.3916015625	78.2091346094252\\
-0.39111328125	54.2358966894366\\
-0.390625	77.7055857142543\\
-0.39013671875	89.4084150358547\\
-0.3896484375	91.2401844793129\\
-0.38916015625	58.4020144626108\\
-0.388671875	91.3027526000526\\
-0.38818359375	93.4049799212052\\
-0.3876953125	87.6373250203799\\
-0.38720703125	84.8911537990095\\
-0.38671875	79.9990066045846\\
-0.38623046875	79.0247808306581\\
-0.3857421875	85.6716270815094\\
-0.38525390625	86.9795653909627\\
-0.384765625	75.6460063270968\\
-0.38427734375	84.242123592442\\
-0.3837890625	56.9125468382559\\
-0.38330078125	68.4854484822012\\
-0.3828125	93.9372268258104\\
-0.38232421875	94.5989144452886\\
-0.3818359375	79.4834019593306\\
-0.38134765625	73.1554406211034\\
-0.380859375	86.1136417322803\\
-0.38037109375	82.5440783729913\\
-0.3798828125	82.4842167034051\\
-0.37939453125	76.1523831590747\\
-0.37890625	75.8690239382317\\
-0.37841796875	87.0606011159003\\
-0.3779296875	82.671781782604\\
-0.37744140625	86.3885867519993\\
-0.376953125	95.3912861593055\\
-0.37646484375	102.261384331584\\
-0.3759765625	84.661661974466\\
-0.37548828125	103.232245120009\\
-0.375	73.0128636534633\\
-0.37451171875	78.5597469832137\\
-0.3740234375	89.0660529816454\\
-0.37353515625	64.5009865762132\\
-0.373046875	51.9115640226058\\
-0.37255859375	90.1278895380233\\
-0.3720703125	92.9506525617157\\
-0.37158203125	75.2438669610694\\
-0.37109375	98.5851690344121\\
-0.37060546875	92.4316742333212\\
-0.3701171875	90.4933656686418\\
-0.36962890625	86.9109569695942\\
-0.369140625	84.2181567781558\\
-0.36865234375	91.263365109729\\
-0.3681640625	89.7910086741422\\
-0.36767578125	71.4707026021664\\
-0.3671875	79.1398546614846\\
-0.36669921875	97.0046969074928\\
-0.3662109375	96.1225436659613\\
-0.36572265625	88.7732509915017\\
-0.365234375	86.8627889579645\\
-0.36474609375	86.7587266005038\\
-0.3642578125	92.8037105297737\\
-0.36376953125	97.4756913372602\\
-0.36328125	75.3808251726557\\
-0.36279296875	93.7237034319655\\
-0.3623046875	92.2309852945179\\
-0.36181640625	95.2503799309847\\
-0.361328125	74.3489941553874\\
-0.36083984375	77.2256339372433\\
-0.3603515625	87.2030169010621\\
-0.35986328125	90.7411540483203\\
-0.359375	93.5717823813409\\
-0.35888671875	72.5868211350682\\
-0.3583984375	76.0917015814455\\
-0.35791015625	102.77563529322\\
-0.357421875	87.1030266925935\\
-0.35693359375	86.9313176665584\\
-0.3564453125	100.714631349974\\
-0.35595703125	82.6797647567923\\
-0.35546875	93.643852003611\\
-0.35498046875	93.0041882903467\\
-0.3544921875	84.8077527310084\\
-0.35400390625	88.5253747205783\\
-0.353515625	51.4174168004479\\
-0.35302734375	94.5369385872934\\
-0.3525390625	102.553476893565\\
-0.35205078125	103.999159235709\\
-0.3515625	107.989762184678\\
-0.35107421875	100.812588976359\\
-0.3505859375	98.4206040134107\\
-0.35009765625	89.6469648030309\\
-0.349609375	94.5937309375112\\
-0.34912109375	94.3457875497766\\
-0.3486328125	56.4897054685164\\
-0.34814453125	75.4769336226711\\
-0.34765625	81.7989043132135\\
-0.34716796875	106.236163544003\\
-0.3466796875	108.439881898985\\
-0.34619140625	90.7215546059906\\
-0.345703125	71.5513061037076\\
-0.34521484375	93.3348555503135\\
-0.3447265625	98.5111179440828\\
-0.34423828125	102.605268055651\\
-0.34375	107.611757406066\\
-0.34326171875	92.9951663605504\\
-0.3427734375	93.6482577019767\\
-0.34228515625	108.275846210372\\
-0.341796875	106.671497283435\\
-0.34130859375	98.1464888569359\\
-0.3408203125	69.9905907172227\\
-0.34033203125	97.4391633775043\\
-0.33984375	101.730580077846\\
-0.33935546875	107.872579307329\\
-0.3388671875	92.5099341732337\\
-0.33837890625	118.780272010959\\
-0.337890625	91.1705289384241\\
-0.33740234375	108.777364027288\\
-0.3369140625	88.9220850404787\\
-0.33642578125	101.403830420952\\
-0.3359375	108.968020963957\\
-0.33544921875	106.630685893738\\
-0.3349609375	113.306566714914\\
-0.33447265625	91.9183616553575\\
-0.333984375	94.0997361908787\\
-0.33349609375	102.015755038084\\
-0.3330078125	90.9939370528613\\
-0.33251953125	114.567263105698\\
-0.33203125	104.807189051332\\
-0.33154296875	106.781054924922\\
-0.3310546875	93.5544355305899\\
-0.33056640625	106.642610547164\\
-0.330078125	114.755906157189\\
-0.32958984375	105.253270718955\\
-0.3291015625	106.17327536322\\
-0.32861328125	97.7133487642827\\
-0.328125	112.465042015636\\
-0.32763671875	97.2355334696157\\
-0.3271484375	115.292004527132\\
-0.32666015625	107.711483652166\\
-0.326171875	90.2555869629513\\
-0.32568359375	115.274996963504\\
-0.3251953125	117.908371286054\\
-0.32470703125	90.6243217452452\\
-0.32421875	101.881115712183\\
-0.32373046875	122.392293841743\\
-0.3232421875	116.181132319152\\
-0.32275390625	100.169852364583\\
-0.322265625	109.972005369061\\
-0.32177734375	104.854902195509\\
-0.3212890625	111.886895944885\\
-0.32080078125	98.9362334641309\\
-0.3203125	111.905816260831\\
-0.31982421875	125.375191653586\\
-0.3193359375	97.554441044193\\
-0.31884765625	107.552356653922\\
-0.318359375	113.940545921339\\
-0.31787109375	109.968746715455\\
-0.3173828125	118.184998769993\\
-0.31689453125	97.4150717140906\\
-0.31640625	117.893844293752\\
-0.31591796875	114.166034953688\\
-0.3154296875	124.201289456759\\
-0.31494140625	118.278744105591\\
-0.314453125	123.000995278744\\
-0.31396484375	120.518467200845\\
-0.3134765625	119.696054525674\\
-0.31298828125	113.608370112988\\
-0.3125	115.30869100998\\
-0.31201171875	116.730845237409\\
-0.3115234375	111.322611792669\\
-0.31103515625	124.39467136829\\
-0.310546875	128.943665931098\\
-0.31005859375	122.264272189709\\
-0.3095703125	119.121536295717\\
-0.30908203125	122.236937032315\\
-0.30859375	129.379948823034\\
-0.30810546875	118.995140902058\\
-0.3076171875	127.593178205852\\
-0.30712890625	102.537228610737\\
-0.306640625	124.486524396126\\
-0.30615234375	115.758774846434\\
-0.3056640625	124.641038787369\\
-0.30517578125	103.445201516651\\
-0.3046875	123.878688572913\\
-0.30419921875	129.336456145929\\
-0.3037109375	109.381459759615\\
-0.30322265625	122.854407278154\\
-0.302734375	111.540216244321\\
-0.30224609375	126.532309008934\\
-0.3017578125	126.573758735537\\
-0.30126953125	124.993803220118\\
-0.30078125	131.015088969555\\
-0.30029296875	119.070084535362\\
-0.2998046875	129.355869897226\\
-0.29931640625	131.077474974049\\
-0.298828125	136.213068428455\\
-0.29833984375	116.478984071111\\
-0.2978515625	134.808125507377\\
-0.29736328125	135.593963648629\\
-0.296875	139.096280231416\\
-0.29638671875	137.958601639517\\
-0.2958984375	134.265185578252\\
-0.29541015625	133.864391647456\\
-0.294921875	125.688646886726\\
-0.29443359375	128.143978300804\\
-0.2939453125	119.724228300905\\
-0.29345703125	142.583272641319\\
-0.29296875	136.893960741356\\
-0.29248046875	137.941238643698\\
-0.2919921875	143.924258933891\\
-0.29150390625	129.11240107645\\
-0.291015625	129.070708933175\\
-0.29052734375	138.291052089564\\
-0.2900390625	129.969991696133\\
-0.28955078125	141.524181201867\\
-0.2890625	130.524369471478\\
-0.28857421875	139.320883268092\\
-0.2880859375	139.730250606296\\
-0.28759765625	125.373727359043\\
-0.287109375	130.547036430212\\
-0.28662109375	152.033046714375\\
-0.2861328125	140.357493536656\\
-0.28564453125	140.696470304937\\
-0.28515625	143.316658451648\\
-0.28466796875	148.339491971311\\
-0.2841796875	147.577041416192\\
-0.28369140625	134.842830048713\\
-0.283203125	140.688819850921\\
-0.28271484375	137.828072165959\\
-0.2822265625	165.322705429943\\
-0.28173828125	142.558245047542\\
-0.28125	157.47685551064\\
-0.28076171875	166.352760113385\\
-0.2802734375	140.376056799253\\
-0.27978515625	169.834011174239\\
-0.279296875	140.659205968566\\
-0.27880859375	165.881667079165\\
-0.2783203125	149.937287232292\\
-0.27783203125	152.946498374531\\
-0.27734375	148.887738966326\\
-0.27685546875	160.034377381747\\
-0.2763671875	151.755337652097\\
-0.27587890625	166.064546385265\\
-0.275390625	171.954633583415\\
-0.27490234375	139.484134234981\\
-0.2744140625	165.354376141496\\
-0.27392578125	160.039257174116\\
-0.2734375	164.971127813015\\
-0.27294921875	157.151974706596\\
-0.2724609375	145.186677154179\\
-0.27197265625	133.269233582695\\
-0.271484375	158.891598019457\\
-0.27099609375	153.914480720081\\
-0.2705078125	154.922841756428\\
-0.27001953125	154.546601179504\\
-0.26953125	160.030350969513\\
-0.26904296875	155.998499038551\\
-0.2685546875	140.113553673081\\
-0.26806640625	149.517829212033\\
-0.267578125	153.888002314315\\
-0.26708984375	144.190946856818\\
-0.2666015625	165.8800795775\\
-0.26611328125	140.973120052809\\
-0.265625	145.203705539595\\
-0.26513671875	156.777748137389\\
-0.2646484375	151.047390699151\\
-0.26416015625	117.751995531093\\
-0.263671875	147.420607624999\\
-0.26318359375	122.331691440853\\
-0.2626953125	139.022464917741\\
-0.26220703125	146.954199250498\\
-0.26171875	151.171655419028\\
-0.26123046875	140.765025949008\\
-0.2607421875	155.266040756872\\
-0.26025390625	150.881662767856\\
-0.259765625	139.330432566187\\
-0.25927734375	145.712168676157\\
-0.2587890625	142.186644379446\\
-0.25830078125	148.555531806638\\
-0.2578125	151.624682600508\\
-0.25732421875	140.142060377561\\
-0.2568359375	150.231925760302\\
-0.25634765625	154.833476897976\\
-0.255859375	135.950646177308\\
-0.25537109375	136.107881498354\\
-0.2548828125	151.180993596397\\
-0.25439453125	134.116383866558\\
-0.25390625	129.954247327671\\
-0.25341796875	151.361656184216\\
-0.2529296875	135.733413862766\\
-0.25244140625	147.2325265606\\
-0.251953125	132.423298389923\\
-0.25146484375	127.315090593859\\
-0.2509765625	145.756882385362\\
-0.25048828125	136.90276266643\\
-0.25	142.34554057525\\
-0.24951171875	140.312088195178\\
-0.2490234375	150.370133919533\\
-0.24853515625	145.572838112512\\
-0.248046875	143.298272795465\\
-0.24755859375	153.632759374737\\
-0.2470703125	156.395204196351\\
-0.24658203125	139.460265332779\\
-0.24609375	124.571199117372\\
-0.24560546875	153.654672563075\\
-0.2451171875	127.3938896558\\
-0.24462890625	129.505728709593\\
-0.244140625	142.214500680853\\
-0.24365234375	118.14176021042\\
-0.2431640625	151.23692821767\\
-0.24267578125	121.471141984884\\
-0.2421875	139.375784604787\\
-0.24169921875	139.567652053127\\
-0.2412109375	139.068676917829\\
-0.24072265625	150.300341236091\\
-0.240234375	140.27226889169\\
-0.23974609375	143.910396838053\\
-0.2392578125	134.497187629144\\
-0.23876953125	139.804470423286\\
-0.23828125	140.434021969422\\
-0.23779296875	143.247144141793\\
-0.2373046875	137.45839131111\\
-0.23681640625	140.841270544839\\
-0.236328125	162.448161580924\\
-0.23583984375	152.720539988044\\
-0.2353515625	144.775935054326\\
-0.23486328125	132.673526813598\\
-0.234375	147.078184208966\\
-0.23388671875	137.661119713641\\
-0.2333984375	152.610528950232\\
-0.23291015625	160.907266419302\\
-0.232421875	146.01814947441\\
-0.23193359375	133.470873455909\\
-0.2314453125	136.392379270057\\
-0.23095703125	153.505766014317\\
-0.23046875	159.664479793338\\
-0.22998046875	138.727864090635\\
-0.2294921875	155.357628612193\\
-0.22900390625	160.208599259529\\
-0.228515625	170.29251911177\\
-0.22802734375	148.737048262236\\
-0.2275390625	153.176984481992\\
-0.22705078125	145.910990803248\\
-0.2265625	132.284189540426\\
-0.22607421875	151.233428118178\\
-0.2255859375	154.15813500919\\
-0.22509765625	142.52747153775\\
-0.224609375	164.210207330507\\
-0.22412109375	133.29352660558\\
-0.2236328125	152.3223925026\\
-0.22314453125	157.959994495015\\
-0.22265625	147.492369326192\\
-0.22216796875	139.062170774057\\
-0.2216796875	149.765947231907\\
-0.22119140625	161.886497780753\\
-0.220703125	131.133869992709\\
-0.22021484375	156.347294426317\\
-0.2197265625	163.130197764478\\
-0.21923828125	147.170669964713\\
-0.21875	155.891868012634\\
-0.21826171875	153.88943465618\\
-0.2177734375	140.373651474867\\
-0.21728515625	162.8388579692\\
-0.216796875	157.236090182001\\
-0.21630859375	139.941150212505\\
-0.2158203125	140.804335136331\\
-0.21533203125	151.42379877378\\
-0.21484375	143.196473172243\\
-0.21435546875	153.578371933805\\
-0.2138671875	149.481888962547\\
-0.21337890625	152.598469393034\\
-0.212890625	151.920790908564\\
-0.21240234375	138.918332260205\\
-0.2119140625	133.548773855663\\
-0.21142578125	143.779211190863\\
-0.2109375	134.539095899658\\
-0.21044921875	131.199244372919\\
-0.2099609375	150.571564923433\\
-0.20947265625	117.060371989753\\
-0.208984375	148.289845808174\\
-0.20849609375	131.69673797525\\
-0.2080078125	145.880776239585\\
-0.20751953125	120.213062741245\\
-0.20703125	133.731269411314\\
-0.20654296875	131.428858911701\\
-0.2060546875	151.441205646673\\
-0.20556640625	134.543701547092\\
-0.205078125	120.938871505372\\
-0.20458984375	142.180145820509\\
-0.2041015625	131.050193328777\\
-0.20361328125	134.136235827509\\
-0.203125	124.830696524727\\
-0.20263671875	124.294445536105\\
-0.2021484375	136.471780229581\\
-0.20166015625	143.559973926999\\
-0.201171875	136.802315828989\\
-0.20068359375	139.451768716548\\
-0.2001953125	128.632468702228\\
-0.19970703125	131.516544437445\\
-0.19921875	130.748520561758\\
-0.19873046875	127.341230184005\\
-0.1982421875	140.127200902406\\
-0.19775390625	140.138209041179\\
-0.197265625	136.60526379125\\
-0.19677734375	131.092079675915\\
-0.1962890625	122.515040232949\\
-0.19580078125	91.0467216878592\\
-0.1953125	125.372286705358\\
-0.19482421875	124.013034279129\\
-0.1943359375	115.789086649981\\
-0.19384765625	125.736872071975\\
-0.193359375	109.634837505058\\
-0.19287109375	125.552118626483\\
-0.1923828125	132.689303861802\\
-0.19189453125	88.4183824002187\\
-0.19140625	121.53720637571\\
-0.19091796875	130.81666928475\\
-0.1904296875	119.316725766137\\
-0.18994140625	129.238207963191\\
-0.189453125	109.309324303492\\
-0.18896484375	131.828546408191\\
-0.1884765625	126.48346959759\\
-0.18798828125	113.99044899247\\
-0.1875	118.808850243294\\
-0.18701171875	120.99666867229\\
-0.1865234375	128.742411708596\\
-0.18603515625	123.352680311535\\
-0.185546875	120.419414793993\\
-0.18505859375	100.090812056896\\
-0.1845703125	107.960713543582\\
-0.18408203125	122.780660780307\\
-0.18359375	90.9261831238771\\
-0.18310546875	130.804072855363\\
-0.1826171875	122.906842822357\\
-0.18212890625	114.006014011334\\
-0.181640625	101.520238679638\\
-0.18115234375	130.348361186726\\
-0.1806640625	98.1150574023806\\
-0.18017578125	118.572613444112\\
-0.1796875	121.948645109559\\
-0.17919921875	114.647406112545\\
-0.1787109375	110.626781988151\\
-0.17822265625	111.252429308368\\
-0.177734375	88.6331426137474\\
-0.17724609375	123.741886941376\\
-0.1767578125	119.862086417097\\
-0.17626953125	112.447646042468\\
-0.17578125	116.386790424039\\
-0.17529296875	107.073920704774\\
-0.1748046875	94.0190732495265\\
-0.17431640625	117.242211946354\\
-0.173828125	108.505353108032\\
-0.17333984375	111.058944161844\\
-0.1728515625	119.584620786857\\
-0.17236328125	87.5891867955987\\
-0.171875	100.216439211677\\
-0.17138671875	104.264150756217\\
-0.1708984375	124.765182563859\\
-0.17041015625	93.1848256023725\\
-0.169921875	125.818512482041\\
-0.16943359375	126.121505611855\\
-0.1689453125	120.128814595747\\
-0.16845703125	120.108163905407\\
-0.16796875	106.043922401174\\
-0.16748046875	108.103809233334\\
-0.1669921875	112.100545539648\\
-0.16650390625	93.5100026096232\\
-0.166015625	121.059370820577\\
-0.16552734375	112.44118927233\\
-0.1650390625	119.306328390563\\
-0.16455078125	111.51803573899\\
-0.1640625	72.277356201716\\
-0.16357421875	115.360507170922\\
-0.1630859375	103.049380711029\\
-0.16259765625	106.532812878997\\
-0.162109375	111.500107478688\\
-0.16162109375	114.514520365751\\
-0.1611328125	111.422255908837\\
-0.16064453125	123.025861923428\\
-0.16015625	122.800713684238\\
-0.15966796875	113.559958549864\\
-0.1591796875	116.83480675739\\
-0.15869140625	106.394773348255\\
-0.158203125	115.596658999659\\
-0.15771484375	71.4974825814999\\
-0.1572265625	112.142122201231\\
-0.15673828125	105.750093509138\\
-0.15625	103.104267269194\\
-0.15576171875	105.70765471594\\
-0.1552734375	109.992637861987\\
-0.15478515625	120.72359578273\\
-0.154296875	104.446203917206\\
-0.15380859375	122.532501639115\\
-0.1533203125	102.833547396388\\
-0.15283203125	106.722511482192\\
-0.15234375	113.193444445309\\
-0.15185546875	95.1074708270425\\
-0.1513671875	110.080343722652\\
-0.15087890625	117.316325162468\\
-0.150390625	106.410988107523\\
-0.14990234375	103.329212537351\\
-0.1494140625	105.127891871114\\
-0.14892578125	98.2262926297934\\
-0.1484375	86.956355248902\\
-0.14794921875	99.687337701795\\
-0.1474609375	109.692051348868\\
-0.14697265625	96.7456150075344\\
-0.146484375	100.704434706238\\
-0.14599609375	102.431513223081\\
-0.1455078125	95.7345733836681\\
-0.14501953125	121.141554579714\\
-0.14453125	110.059225497177\\
-0.14404296875	114.364268109563\\
-0.1435546875	95.0701803229271\\
-0.14306640625	86.4286543420029\\
-0.142578125	92.5962446109389\\
-0.14208984375	79.2489440977799\\
-0.1416015625	111.21967271152\\
-0.14111328125	108.84254743932\\
-0.140625	85.2619145703681\\
-0.14013671875	103.570342429512\\
-0.1396484375	88.007376885226\\
-0.13916015625	105.837544497802\\
-0.138671875	72.7542049719097\\
-0.13818359375	94.6481629400566\\
-0.1376953125	83.6385126968903\\
-0.13720703125	111.116720966169\\
-0.13671875	117.757779256478\\
-0.13623046875	104.180502484719\\
-0.1357421875	91.640208332406\\
-0.13525390625	108.443882541312\\
-0.134765625	108.644723014341\\
-0.13427734375	112.090688740824\\
-0.1337890625	82.5135928822798\\
-0.13330078125	109.129224121964\\
-0.1328125	74.3239777765718\\
-0.13232421875	102.573114700666\\
-0.1318359375	86.1219373025828\\
-0.13134765625	90.5472033409754\\
-0.130859375	94.413022131379\\
-0.13037109375	86.5277129654565\\
-0.1298828125	108.230135720471\\
-0.12939453125	88.0475886816849\\
-0.12890625	76.1278454909727\\
-0.12841796875	101.034398194784\\
-0.1279296875	93.9434580184604\\
-0.12744140625	117.398223015036\\
-0.126953125	106.327858215692\\
-0.12646484375	100.961605535873\\
-0.1259765625	97.7085693481986\\
-0.12548828125	99.5666184719527\\
-0.125	85.1513749296348\\
-0.12451171875	86.8194729715279\\
-0.1240234375	102.272770635541\\
-0.12353515625	100.327827545746\\
-0.123046875	102.613257928368\\
-0.12255859375	92.7012452299908\\
-0.1220703125	103.436250553055\\
-0.12158203125	85.4527427904136\\
-0.12109375	101.399706642258\\
-0.12060546875	89.5301862560798\\
-0.1201171875	78.1939576155642\\
-0.11962890625	97.8344060824712\\
-0.119140625	82.0011099405856\\
-0.11865234375	105.481808190265\\
-0.1181640625	100.418608242158\\
-0.11767578125	96.9198180187988\\
-0.1171875	91.9625950930603\\
-0.11669921875	87.1694908410265\\
-0.1162109375	112.697626550113\\
-0.11572265625	101.762095808085\\
-0.115234375	94.5026098945471\\
-0.11474609375	82.5659366773889\\
-0.1142578125	108.345880297914\\
-0.11376953125	70.6633966751535\\
-0.11328125	88.242824668998\\
-0.11279296875	106.140578148041\\
-0.1123046875	79.3063171952822\\
-0.11181640625	74.6303945864196\\
-0.111328125	104.980733664148\\
-0.11083984375	79.3361942748184\\
-0.1103515625	104.768354710957\\
-0.10986328125	101.104572228448\\
-0.109375	92.785420192032\\
-0.10888671875	92.1227214760489\\
-0.1083984375	94.4820086718452\\
-0.10791015625	97.8780088030359\\
-0.107421875	104.307566743984\\
-0.10693359375	71.9821787300471\\
-0.1064453125	105.657788844288\\
-0.10595703125	94.0752113166478\\
-0.10546875	108.68764236039\\
-0.10498046875	99.1500954743894\\
-0.1044921875	99.184138410094\\
-0.10400390625	93.7098497583665\\
-0.103515625	102.0304017189\\
-0.10302734375	99.7899014109875\\
-0.1025390625	91.0397976514313\\
-0.10205078125	95.8547978132554\\
-0.1015625	99.149845216686\\
-0.10107421875	101.878842166234\\
-0.1005859375	104.76975973651\\
-0.10009765625	100.525398139873\\
-0.099609375	99.175027159704\\
-0.09912109375	99.6351944914394\\
-0.0986328125	86.9516156902544\\
-0.09814453125	90.3001895730837\\
-0.09765625	82.4722501288436\\
-0.09716796875	103.856202786523\\
-0.0966796875	92.0041783926657\\
-0.09619140625	97.0554777028148\\
-0.095703125	96.945352190951\\
-0.09521484375	89.0264500448458\\
-0.0947265625	100.655712550024\\
-0.09423828125	68.6186397047829\\
-0.09375	102.514509076352\\
-0.09326171875	103.253803843304\\
-0.0927734375	94.2531295627999\\
-0.09228515625	89.4373334476174\\
-0.091796875	92.9203039334209\\
-0.09130859375	85.4267840407974\\
-0.0908203125	91.4105979654642\\
-0.09033203125	116.016542760293\\
-0.08984375	88.7599064186126\\
-0.08935546875	105.722839999933\\
-0.0888671875	92.0656693183639\\
-0.08837890625	101.11444455744\\
-0.087890625	88.4315816946032\\
-0.08740234375	99.6795649955837\\
-0.0869140625	97.6353068914719\\
-0.08642578125	108.996799185869\\
-0.0859375	93.1795712289473\\
-0.08544921875	103.619732725197\\
-0.0849609375	106.023421640609\\
-0.08447265625	97.71809605394\\
-0.083984375	96.021030396779\\
-0.08349609375	83.8993587047269\\
-0.0830078125	92.7599536694769\\
-0.08251953125	74.2942395141523\\
-0.08203125	101.410940418539\\
-0.08154296875	89.944820629465\\
-0.0810546875	89.3363146558254\\
-0.08056640625	96.3641460320911\\
-0.080078125	89.6321653267758\\
-0.07958984375	88.1082212180399\\
-0.0791015625	69.2278877063932\\
-0.07861328125	92.8901973135939\\
-0.078125	84.4547311808142\\
-0.07763671875	91.8037571640294\\
-0.0771484375	97.7941401848392\\
-0.07666015625	103.571424117681\\
-0.076171875	105.515497340014\\
-0.07568359375	94.990179598508\\
-0.0751953125	82.3478347976071\\
-0.07470703125	84.6798125905745\\
-0.07421875	74.1201877857931\\
-0.07373046875	105.734744666251\\
-0.0732421875	94.5651877139645\\
-0.07275390625	90.3179277209566\\
-0.072265625	92.7298983640881\\
-0.07177734375	90.7909400796979\\
-0.0712890625	94.2607988318917\\
-0.07080078125	88.0376141809335\\
-0.0703125	85.9679977235727\\
-0.06982421875	93.230628830292\\
-0.0693359375	95.8348190686024\\
-0.06884765625	83.9969385317656\\
-0.068359375	58.2260307771464\\
-0.06787109375	99.1499195178151\\
-0.0673828125	96.9438829653325\\
-0.06689453125	101.516927975928\\
-0.06640625	80.4576373056225\\
-0.06591796875	84.1034785379057\\
-0.0654296875	94.5845776821823\\
-0.06494140625	106.666988322002\\
-0.064453125	102.742016108334\\
-0.06396484375	85.0689677260501\\
-0.0634765625	77.5104013295309\\
-0.06298828125	30.3957738307257\\
-0.0625	100.621364879953\\
-0.06201171875	103.034376583182\\
-0.0615234375	82.3614496385345\\
-0.06103515625	82.1407081187677\\
-0.060546875	83.19316136894\\
-0.06005859375	76.8740515324256\\
-0.0595703125	95.7397421937378\\
-0.05908203125	79.0899750551668\\
-0.05859375	104.279217341496\\
-0.05810546875	69.8232738730532\\
-0.0576171875	93.4968348189448\\
-0.05712890625	104.214338999658\\
-0.056640625	89.6255165865851\\
-0.05615234375	85.1471959872587\\
-0.0556640625	75.5634756289153\\
-0.05517578125	96.0689516660235\\
-0.0546875	95.8937898785291\\
-0.05419921875	103.225160068651\\
-0.0537109375	97.8171781835629\\
-0.05322265625	81.4110348013313\\
-0.052734375	90.4888025163676\\
-0.05224609375	101.072993867814\\
-0.0517578125	86.7475996773986\\
-0.05126953125	87.3112553960586\\
-0.05078125	83.9727325382813\\
-0.05029296875	89.523757974773\\
-0.0498046875	95.2925440963949\\
-0.04931640625	79.3556559373119\\
-0.048828125	79.422638078496\\
-0.04833984375	92.4549997454572\\
-0.0478515625	81.3904518500986\\
-0.04736328125	104.35545222453\\
-0.046875	83.4453153055802\\
-0.04638671875	92.4755395335239\\
-0.0458984375	105.620678568311\\
-0.04541015625	86.718297580821\\
-0.044921875	96.9322707907296\\
-0.04443359375	98.1449983014447\\
-0.0439453125	91.3618973415299\\
-0.04345703125	71.2300701946554\\
-0.04296875	91.5387946692927\\
-0.04248046875	88.9893257715155\\
-0.0419921875	99.6096590366877\\
-0.04150390625	80.0989143747467\\
-0.041015625	83.9127053842641\\
-0.04052734375	88.6249515890499\\
-0.0400390625	84.9751530383777\\
-0.03955078125	106.144536169764\\
-0.0390625	45.591741100669\\
-0.03857421875	80.8969620186763\\
-0.0380859375	61.6139193186081\\
-0.03759765625	81.8705632799237\\
-0.037109375	71.6226313409128\\
-0.03662109375	95.3212332445603\\
-0.0361328125	102.95891616516\\
-0.03564453125	94.0155142443573\\
-0.03515625	86.1884393076885\\
-0.03466796875	89.8419594412878\\
-0.0341796875	70.3372471275242\\
-0.03369140625	72.2535768095869\\
-0.033203125	91.0642593426271\\
-0.03271484375	92.2943236504031\\
-0.0322265625	90.1444324806324\\
-0.03173828125	100.923368086206\\
-0.03125	88.7509982243168\\
-0.03076171875	83.3237022892661\\
-0.0302734375	84.0440251780153\\
-0.02978515625	99.5388160165776\\
-0.029296875	94.9897512305235\\
-0.02880859375	100.145428841162\\
-0.0283203125	90.09370593677\\
-0.02783203125	85.7504632939079\\
-0.02734375	76.4169813465533\\
-0.02685546875	89.5602953260923\\
-0.0263671875	86.2330353815862\\
-0.02587890625	107.171062314984\\
-0.025390625	94.400368195914\\
-0.02490234375	100.489353021502\\
-0.0244140625	95.0918879900551\\
-0.02392578125	90.4278413258611\\
-0.0234375	93.014854302503\\
-0.02294921875	62.2840261207413\\
-0.0224609375	87.6342005537515\\
-0.02197265625	94.4935793394751\\
-0.021484375	102.946987615731\\
-0.02099609375	70.9083670195215\\
-0.0205078125	80.8931144449568\\
-0.02001953125	91.01050983775\\
-0.01953125	101.552887904897\\
-0.01904296875	85.1078975524484\\
-0.0185546875	97.9885201966184\\
-0.01806640625	94.6063424048585\\
-0.017578125	86.8299197631559\\
-0.01708984375	86.020976165359\\
-0.0166015625	95.6746525085497\\
-0.01611328125	100.784154781635\\
-0.015625	90.7830781673497\\
-0.01513671875	82.6838690480734\\
-0.0146484375	83.3729689437893\\
-0.01416015625	73.1909958773538\\
-0.013671875	88.336459131825\\
-0.01318359375	93.7129918740626\\
-0.0126953125	88.2546368623791\\
-0.01220703125	85.8771738891602\\
-0.01171875	66.3920572771271\\
-0.01123046875	85.9364806845495\\
-0.0107421875	91.7236418366358\\
-0.01025390625	93.774537726013\\
-0.009765625	100.811137816782\\
-0.00927734375	90.0254733216966\\
-0.0087890625	82.7786786028476\\
-0.00830078125	82.214216299171\\
-0.0078125	93.3491584986917\\
-0.00732421875	80.2782681291578\\
-0.0068359375	90.3902781392214\\
-0.00634765625	86.8458057548655\\
-0.005859375	93.4832936242476\\
-0.00537109375	84.4264510996605\\
-0.0048828125	66.9475271243719\\
-0.00439453125	104.528004040024\\
-0.00390625	89.7084259843848\\
-0.00341796875	86.1894208561492\\
-0.0029296875	95.7042977814298\\
-0.00244140625	78.170689907758\\
-0.001953125	80.8310430929879\\
-0.00146484375	82.6470804196139\\
-0.0009765625	87.3437994892652\\
-0.00048828125	81.667747557014\\
0	84.2361692277948\\
0.00048828125	81.667747557014\\
0.0009765625	87.3437994892652\\
0.00146484375	82.6470804196139\\
0.001953125	80.8310430929879\\
0.00244140625	78.170689907758\\
0.0029296875	95.7042977814298\\
0.00341796875	86.1894208561492\\
0.00390625	89.7084259843848\\
0.00439453125	104.528004040024\\
0.0048828125	66.9475271243719\\
0.00537109375	84.4264510996605\\
0.005859375	93.4832936242476\\
0.00634765625	86.8458057548655\\
0.0068359375	90.3902781392214\\
0.00732421875	80.2782681291578\\
0.0078125	93.3491584986917\\
0.00830078125	82.214216299171\\
0.0087890625	82.7786786028476\\
0.00927734375	90.0254733216966\\
0.009765625	100.811137816782\\
0.01025390625	93.774537726013\\
0.0107421875	91.7236418366358\\
0.01123046875	85.9364806845495\\
0.01171875	66.3920572771271\\
0.01220703125	85.8771738891602\\
0.0126953125	88.2546368623791\\
0.01318359375	93.7129918740626\\
0.013671875	88.336459131825\\
0.01416015625	73.1909958773538\\
0.0146484375	83.3729689437893\\
0.01513671875	82.6838690480734\\
0.015625	90.7830781673497\\
0.01611328125	100.784154781635\\
0.0166015625	95.6746525085497\\
0.01708984375	86.020976165359\\
0.017578125	86.8299197631559\\
0.01806640625	94.6063424048585\\
0.0185546875	97.9885201966184\\
0.01904296875	85.1078975524484\\
0.01953125	101.552887904897\\
0.02001953125	91.01050983775\\
0.0205078125	80.8931144449568\\
0.02099609375	70.9083670195215\\
0.021484375	102.946987615731\\
0.02197265625	94.4935793394751\\
0.0224609375	87.6342005537515\\
0.02294921875	62.2840261207413\\
0.0234375	93.014854302503\\
0.02392578125	90.4278413258611\\
0.0244140625	95.0918879900551\\
0.02490234375	100.489353021502\\
0.025390625	94.400368195914\\
0.02587890625	107.171062314984\\
0.0263671875	86.2330353815862\\
0.02685546875	89.5602953260923\\
0.02734375	76.4169813465533\\
0.02783203125	85.7504632939079\\
0.0283203125	90.09370593677\\
0.02880859375	100.145428841162\\
0.029296875	94.9897512305235\\
0.02978515625	99.5388160165776\\
0.0302734375	84.0440251780153\\
0.03076171875	83.3237022892661\\
0.03125	88.7509982243168\\
0.03173828125	100.923368086206\\
0.0322265625	90.1444324806324\\
0.03271484375	92.2943236504031\\
0.033203125	91.0642593426271\\
0.03369140625	72.2535768095869\\
0.0341796875	70.3372471275242\\
0.03466796875	89.8419594412878\\
0.03515625	86.1884393076885\\
0.03564453125	94.0155142443573\\
0.0361328125	102.95891616516\\
0.03662109375	95.3212332445603\\
0.037109375	71.6226313409128\\
0.03759765625	81.8705632799237\\
0.0380859375	61.6139193186081\\
0.03857421875	80.8969620186763\\
0.0390625	45.591741100669\\
0.03955078125	106.144536169764\\
0.0400390625	84.9751530383777\\
0.04052734375	88.6249515890499\\
0.041015625	83.9127053842641\\
0.04150390625	80.0989143747467\\
0.0419921875	99.6096590366877\\
0.04248046875	88.9893257715155\\
0.04296875	91.5387946692927\\
0.04345703125	71.2300701946554\\
0.0439453125	91.3618973415299\\
0.04443359375	98.1449983014447\\
0.044921875	96.9322707907296\\
0.04541015625	86.718297580821\\
0.0458984375	105.620678568311\\
0.04638671875	92.4755395335239\\
0.046875	83.4453153055802\\
0.04736328125	104.35545222453\\
0.0478515625	81.3904518500986\\
0.04833984375	92.4549997454572\\
0.048828125	79.422638078496\\
0.04931640625	79.3556559373119\\
0.0498046875	95.2925440963949\\
0.05029296875	89.523757974773\\
0.05078125	83.9727325382813\\
0.05126953125	87.3112553960586\\
0.0517578125	86.7475996773986\\
0.05224609375	101.072993867814\\
0.052734375	90.4888025163676\\
0.05322265625	81.4110348013313\\
0.0537109375	97.8171781835629\\
0.05419921875	103.225160068651\\
0.0546875	95.8937898785291\\
0.05517578125	96.0689516660235\\
0.0556640625	75.5634756289153\\
0.05615234375	85.1471959872587\\
0.056640625	89.6255165865851\\
0.05712890625	104.214338999658\\
0.0576171875	93.4968348189448\\
0.05810546875	69.8232738730532\\
0.05859375	104.279217341496\\
0.05908203125	79.0899750551668\\
0.0595703125	95.7397421937378\\
0.06005859375	76.8740515324256\\
0.060546875	83.19316136894\\
0.06103515625	82.1407081187677\\
0.0615234375	82.3614496385345\\
0.06201171875	103.034376583182\\
0.0625	100.621364879953\\
0.06298828125	30.3957738307257\\
0.0634765625	77.5104013295309\\
0.06396484375	85.0689677260501\\
0.064453125	102.742016108334\\
0.06494140625	106.666988322002\\
0.0654296875	94.5845776821823\\
0.06591796875	84.1034785379057\\
0.06640625	80.4576373056225\\
0.06689453125	101.516927975928\\
0.0673828125	96.9438829653325\\
0.06787109375	99.1499195178151\\
0.068359375	58.2260307771464\\
0.06884765625	83.9969385317656\\
0.0693359375	95.8348190686024\\
0.06982421875	93.230628830292\\
0.0703125	85.9679977235727\\
0.07080078125	88.0376141809335\\
0.0712890625	94.2607988318917\\
0.07177734375	90.7909400796979\\
0.072265625	92.7298983640881\\
0.07275390625	90.3179277209566\\
0.0732421875	94.5651877139645\\
0.07373046875	105.734744666251\\
0.07421875	74.1201877857931\\
0.07470703125	84.6798125905745\\
0.0751953125	82.3478347976071\\
0.07568359375	94.990179598508\\
0.076171875	105.515497340014\\
0.07666015625	103.571424117681\\
0.0771484375	97.7941401848392\\
0.07763671875	91.8037571640294\\
0.078125	84.4547311808142\\
0.07861328125	92.8901973135939\\
0.0791015625	69.2278877063932\\
0.07958984375	88.1082212180399\\
0.080078125	89.6321653267758\\
0.08056640625	96.3641460320911\\
0.0810546875	89.3363146558254\\
0.08154296875	89.944820629465\\
0.08203125	101.410940418539\\
0.08251953125	74.2942395141523\\
0.0830078125	92.7599536694769\\
0.08349609375	83.8993587047269\\
0.083984375	96.021030396779\\
0.08447265625	97.71809605394\\
0.0849609375	106.023421640609\\
0.08544921875	103.619732725197\\
0.0859375	93.1795712289473\\
0.08642578125	108.996799185869\\
0.0869140625	97.6353068914719\\
0.08740234375	99.6795649955837\\
0.087890625	88.4315816946032\\
0.08837890625	101.11444455744\\
0.0888671875	92.0656693183639\\
0.08935546875	105.722839999933\\
0.08984375	88.7599064186126\\
0.09033203125	116.016542760293\\
0.0908203125	91.4105979654642\\
0.09130859375	85.4267840407974\\
0.091796875	92.9203039334209\\
0.09228515625	89.4373334476174\\
0.0927734375	94.2531295627999\\
0.09326171875	103.253803843304\\
0.09375	102.514509076352\\
0.09423828125	68.6186397047829\\
0.0947265625	100.655712550024\\
0.09521484375	89.0264500448458\\
0.095703125	96.945352190951\\
0.09619140625	97.0554777028148\\
0.0966796875	92.0041783926657\\
0.09716796875	103.856202786523\\
0.09765625	82.4722501288436\\
0.09814453125	90.3001895730837\\
0.0986328125	86.9516156902544\\
0.09912109375	99.6351944914394\\
0.099609375	99.175027159704\\
0.10009765625	100.525398139873\\
0.1005859375	104.76975973651\\
0.10107421875	101.878842166234\\
0.1015625	99.149845216686\\
0.10205078125	95.8547978132554\\
0.1025390625	91.0397976514313\\
0.10302734375	99.7899014109875\\
0.103515625	102.0304017189\\
0.10400390625	93.7098497583665\\
0.1044921875	99.184138410094\\
0.10498046875	99.1500954743894\\
0.10546875	108.68764236039\\
0.10595703125	94.0752113166478\\
0.1064453125	105.657788844288\\
0.10693359375	71.9821787300471\\
0.107421875	104.307566743984\\
0.10791015625	97.8780088030359\\
0.1083984375	94.4820086718452\\
0.10888671875	92.1227214760489\\
0.109375	92.785420192032\\
0.10986328125	101.104572228448\\
0.1103515625	104.768354710957\\
0.11083984375	79.3361942748184\\
0.111328125	104.980733664148\\
0.11181640625	74.6303945864196\\
0.1123046875	79.3063171952822\\
0.11279296875	106.140578148041\\
0.11328125	88.242824668998\\
0.11376953125	70.6633966751535\\
0.1142578125	108.345880297914\\
0.11474609375	82.5659366773889\\
0.115234375	94.5026098945471\\
0.11572265625	101.762095808085\\
0.1162109375	112.697626550113\\
0.11669921875	87.1694908410265\\
0.1171875	91.9625950930603\\
0.11767578125	96.9198180187988\\
0.1181640625	100.418608242158\\
0.11865234375	105.481808190265\\
0.119140625	82.0011099405856\\
0.11962890625	97.8344060824712\\
0.1201171875	78.1939576155642\\
0.12060546875	89.5301862560798\\
0.12109375	101.399706642258\\
0.12158203125	85.4527427904136\\
0.1220703125	103.436250553055\\
0.12255859375	92.7012452299908\\
0.123046875	102.613257928368\\
0.12353515625	100.327827545746\\
0.1240234375	102.272770635541\\
0.12451171875	86.8194729715279\\
0.125	85.1513749296348\\
0.12548828125	99.5666184719527\\
0.1259765625	97.7085693481986\\
0.12646484375	100.961605535873\\
0.126953125	106.327858215692\\
0.12744140625	117.398223015036\\
0.1279296875	93.9434580184604\\
0.12841796875	101.034398194784\\
0.12890625	76.1278454909727\\
0.12939453125	88.0475886816849\\
0.1298828125	108.230135720471\\
0.13037109375	86.5277129654565\\
0.130859375	94.413022131379\\
0.13134765625	90.5472033409754\\
0.1318359375	86.1219373025828\\
0.13232421875	102.573114700666\\
0.1328125	74.3239777765718\\
0.13330078125	109.129224121964\\
0.1337890625	82.5135928822798\\
0.13427734375	112.090688740824\\
0.134765625	108.644723014341\\
0.13525390625	108.443882541312\\
0.1357421875	91.640208332406\\
0.13623046875	104.180502484719\\
0.13671875	117.757779256478\\
0.13720703125	111.116720966169\\
0.1376953125	83.6385126968903\\
0.13818359375	94.6481629400566\\
0.138671875	72.7542049719097\\
0.13916015625	105.837544497802\\
0.1396484375	88.007376885226\\
0.14013671875	103.570342429512\\
0.140625	85.2619145703681\\
0.14111328125	108.84254743932\\
0.1416015625	111.21967271152\\
0.14208984375	79.2489440977799\\
0.142578125	92.5962446109389\\
0.14306640625	86.4286543420029\\
0.1435546875	95.0701803229271\\
0.14404296875	114.364268109563\\
0.14453125	110.059225497177\\
0.14501953125	121.141554579714\\
0.1455078125	95.7345733836681\\
0.14599609375	102.431513223081\\
0.146484375	100.704434706238\\
0.14697265625	96.7456150075344\\
0.1474609375	109.692051348868\\
0.14794921875	99.687337701795\\
0.1484375	86.956355248902\\
0.14892578125	98.2262926297934\\
0.1494140625	105.127891871114\\
0.14990234375	103.329212537351\\
0.150390625	106.410988107523\\
0.15087890625	117.316325162468\\
0.1513671875	110.080343722652\\
0.15185546875	95.1074708270425\\
0.15234375	113.193444445309\\
0.15283203125	106.722511482192\\
0.1533203125	102.833547396388\\
0.15380859375	122.532501639115\\
0.154296875	104.446203917206\\
0.15478515625	120.72359578273\\
0.1552734375	109.992637861987\\
0.15576171875	105.70765471594\\
0.15625	103.104267269194\\
0.15673828125	105.750093509138\\
0.1572265625	112.142122201231\\
0.15771484375	71.4974825814999\\
0.158203125	115.596658999659\\
0.15869140625	106.394773348255\\
0.1591796875	116.83480675739\\
0.15966796875	113.559958549864\\
0.16015625	122.800713684238\\
0.16064453125	123.025861923428\\
0.1611328125	111.422255908837\\
0.16162109375	114.514520365751\\
0.162109375	111.500107478688\\
0.16259765625	106.532812878997\\
0.1630859375	103.049380711029\\
0.16357421875	115.360507170922\\
0.1640625	72.277356201716\\
0.16455078125	111.51803573899\\
0.1650390625	119.306328390563\\
0.16552734375	112.44118927233\\
0.166015625	121.059370820577\\
0.16650390625	93.5100026096232\\
0.1669921875	112.100545539648\\
0.16748046875	108.103809233334\\
0.16796875	106.043922401174\\
0.16845703125	120.108163905407\\
0.1689453125	120.128814595747\\
0.16943359375	126.121505611855\\
0.169921875	125.818512482041\\
0.17041015625	93.1848256023725\\
0.1708984375	124.765182563859\\
0.17138671875	104.264150756217\\
0.171875	100.216439211677\\
0.17236328125	87.5891867955987\\
0.1728515625	119.584620786857\\
0.17333984375	111.058944161844\\
0.173828125	108.505353108032\\
0.17431640625	117.242211946354\\
0.1748046875	94.0190732495265\\
0.17529296875	107.073920704774\\
0.17578125	116.386790424039\\
0.17626953125	112.447646042468\\
0.1767578125	119.862086417097\\
0.17724609375	123.741886941376\\
0.177734375	88.6331426137474\\
0.17822265625	111.252429308368\\
0.1787109375	110.626781988151\\
0.17919921875	114.647406112545\\
0.1796875	121.948645109559\\
0.18017578125	118.572613444112\\
0.1806640625	98.1150574023806\\
0.18115234375	130.348361186726\\
0.181640625	101.520238679638\\
0.18212890625	114.006014011334\\
0.1826171875	122.906842822357\\
0.18310546875	130.804072855363\\
0.18359375	90.9261831238771\\
0.18408203125	122.780660780307\\
0.1845703125	107.960713543582\\
0.18505859375	100.090812056896\\
0.185546875	120.419414793993\\
0.18603515625	123.352680311535\\
0.1865234375	128.742411708596\\
0.18701171875	120.99666867229\\
0.1875	118.808850243294\\
0.18798828125	113.99044899247\\
0.1884765625	126.48346959759\\
0.18896484375	131.828546408191\\
0.189453125	109.309324303492\\
0.18994140625	129.238207963191\\
0.1904296875	119.316725766137\\
0.19091796875	130.81666928475\\
0.19140625	121.53720637571\\
0.19189453125	88.4183824002187\\
0.1923828125	132.689303861802\\
0.19287109375	125.552118626483\\
0.193359375	109.634837505058\\
0.19384765625	125.736872071975\\
0.1943359375	115.789086649981\\
0.19482421875	124.013034279129\\
0.1953125	125.372286705358\\
0.19580078125	91.0467216878592\\
0.1962890625	122.515040232949\\
0.19677734375	131.092079675915\\
0.197265625	136.60526379125\\
0.19775390625	140.138209041179\\
0.1982421875	140.127200902406\\
0.19873046875	127.341230184005\\
0.19921875	130.748520561758\\
0.19970703125	131.516544437445\\
0.2001953125	128.632468702228\\
0.20068359375	139.451768716548\\
0.201171875	136.802315828989\\
0.20166015625	143.559973926999\\
0.2021484375	136.471780229581\\
0.20263671875	124.294445536105\\
0.203125	124.830696524727\\
0.20361328125	134.136235827509\\
0.2041015625	131.050193328777\\
0.20458984375	142.180145820509\\
0.205078125	120.938871505372\\
0.20556640625	134.543701547092\\
0.2060546875	151.441205646673\\
0.20654296875	131.428858911701\\
0.20703125	133.731269411314\\
0.20751953125	120.213062741245\\
0.2080078125	145.880776239585\\
0.20849609375	131.69673797525\\
0.208984375	148.289845808174\\
0.20947265625	117.060371989753\\
0.2099609375	150.571564923433\\
0.21044921875	131.199244372919\\
0.2109375	134.539095899658\\
0.21142578125	143.779211190863\\
0.2119140625	133.548773855663\\
0.21240234375	138.918332260205\\
0.212890625	151.920790908564\\
0.21337890625	152.598469393034\\
0.2138671875	149.481888962547\\
0.21435546875	153.578371933805\\
0.21484375	143.196473172243\\
0.21533203125	151.42379877378\\
0.2158203125	140.804335136331\\
0.21630859375	139.941150212505\\
0.216796875	157.236090182001\\
0.21728515625	162.8388579692\\
0.2177734375	140.373651474867\\
0.21826171875	153.88943465618\\
0.21875	155.891868012634\\
0.21923828125	147.170669964713\\
0.2197265625	163.130197764478\\
0.22021484375	156.347294426317\\
0.220703125	131.133869992709\\
0.22119140625	161.886497780753\\
0.2216796875	149.765947231907\\
0.22216796875	139.062170774057\\
0.22265625	147.492369326192\\
0.22314453125	157.959994495015\\
0.2236328125	152.3223925026\\
0.22412109375	133.29352660558\\
0.224609375	164.210207330507\\
0.22509765625	142.52747153775\\
0.2255859375	154.15813500919\\
0.22607421875	151.233428118178\\
0.2265625	132.284189540426\\
0.22705078125	145.910990803248\\
0.2275390625	153.176984481992\\
0.22802734375	148.737048262236\\
0.228515625	170.29251911177\\
0.22900390625	160.208599259529\\
0.2294921875	155.357628612193\\
0.22998046875	138.727864090635\\
0.23046875	159.664479793338\\
0.23095703125	153.505766014317\\
0.2314453125	136.392379270057\\
0.23193359375	133.470873455909\\
0.232421875	146.01814947441\\
0.23291015625	160.907266419302\\
0.2333984375	152.610528950232\\
0.23388671875	137.661119713641\\
0.234375	147.078184208966\\
0.23486328125	132.673526813598\\
0.2353515625	144.775935054326\\
0.23583984375	152.720539988044\\
0.236328125	162.448161580924\\
0.23681640625	140.841270544839\\
0.2373046875	137.45839131111\\
0.23779296875	143.247144141793\\
0.23828125	140.434021969422\\
0.23876953125	139.804470423286\\
0.2392578125	134.497187629144\\
0.23974609375	143.910396838053\\
0.240234375	140.27226889169\\
0.24072265625	150.300341236091\\
0.2412109375	139.068676917829\\
0.24169921875	139.567652053127\\
0.2421875	139.375784604787\\
0.24267578125	121.471141984884\\
0.2431640625	151.23692821767\\
0.24365234375	118.14176021042\\
0.244140625	142.214500680853\\
0.24462890625	129.505728709593\\
0.2451171875	127.3938896558\\
0.24560546875	153.654672563075\\
0.24609375	124.571199117372\\
0.24658203125	139.460265332779\\
0.2470703125	156.395204196351\\
0.24755859375	153.632759374737\\
0.248046875	143.298272795465\\
0.24853515625	145.572838112512\\
0.2490234375	150.370133919533\\
0.24951171875	140.312088195178\\
0.25	142.34554057525\\
0.25048828125	136.90276266643\\
0.2509765625	145.756882385362\\
0.25146484375	127.315090593859\\
0.251953125	132.423298389923\\
0.25244140625	147.2325265606\\
0.2529296875	135.733413862766\\
0.25341796875	151.361656184216\\
0.25390625	129.954247327671\\
0.25439453125	134.116383866558\\
0.2548828125	151.180993596397\\
0.25537109375	136.107881498354\\
0.255859375	135.950646177308\\
0.25634765625	154.833476897976\\
0.2568359375	150.231925760302\\
0.25732421875	140.142060377561\\
0.2578125	151.624682600508\\
0.25830078125	148.555531806638\\
0.2587890625	142.186644379446\\
0.25927734375	145.712168676157\\
0.259765625	139.330432566187\\
0.26025390625	150.881662767856\\
0.2607421875	155.266040756872\\
0.26123046875	140.765025949008\\
0.26171875	151.171655419028\\
0.26220703125	146.954199250498\\
0.2626953125	139.022464917741\\
0.26318359375	122.331691440853\\
0.263671875	147.420607624999\\
0.26416015625	117.751995531093\\
0.2646484375	151.047390699151\\
0.26513671875	156.777748137389\\
0.265625	145.203705539595\\
0.26611328125	140.973120052809\\
0.2666015625	165.8800795775\\
0.26708984375	144.190946856818\\
0.267578125	153.888002314315\\
0.26806640625	149.517829212033\\
0.2685546875	140.113553673081\\
0.26904296875	155.998499038551\\
0.26953125	160.030350969513\\
0.27001953125	154.546601179504\\
0.2705078125	154.922841756428\\
0.27099609375	153.914480720081\\
0.271484375	158.891598019457\\
0.27197265625	133.269233582695\\
0.2724609375	145.186677154179\\
0.27294921875	157.151974706596\\
0.2734375	164.971127813015\\
0.27392578125	160.039257174116\\
0.2744140625	165.354376141496\\
0.27490234375	139.484134234981\\
0.275390625	171.954633583415\\
0.27587890625	166.064546385265\\
0.2763671875	151.755337652097\\
0.27685546875	160.034377381747\\
0.27734375	148.887738966326\\
0.27783203125	152.946498374531\\
0.2783203125	149.937287232292\\
0.27880859375	165.881667079165\\
0.279296875	140.659205968566\\
0.27978515625	169.834011174239\\
0.2802734375	140.376056799253\\
0.28076171875	166.352760113385\\
0.28125	157.47685551064\\
0.28173828125	142.558245047542\\
0.2822265625	165.322705429943\\
0.28271484375	137.828072165959\\
0.283203125	140.688819850921\\
0.28369140625	134.842830048713\\
0.2841796875	147.577041416192\\
0.28466796875	148.339491971311\\
0.28515625	143.316658451648\\
0.28564453125	140.696470304937\\
0.2861328125	140.357493536656\\
0.28662109375	152.033046714375\\
0.287109375	130.547036430212\\
0.28759765625	125.373727359043\\
0.2880859375	139.730250606296\\
0.28857421875	139.320883268092\\
0.2890625	130.524369471478\\
0.28955078125	141.524181201867\\
0.2900390625	129.969991696133\\
0.29052734375	138.291052089564\\
0.291015625	129.070708933175\\
0.29150390625	129.11240107645\\
0.2919921875	143.924258933891\\
0.29248046875	137.941238643698\\
0.29296875	136.893960741356\\
0.29345703125	142.583272641319\\
0.2939453125	119.724228300905\\
0.29443359375	128.143978300804\\
0.294921875	125.688646886726\\
0.29541015625	133.864391647456\\
0.2958984375	134.265185578252\\
0.29638671875	137.958601639517\\
0.296875	139.096280231416\\
0.29736328125	135.593963648629\\
0.2978515625	134.808125507377\\
0.29833984375	116.478984071111\\
0.298828125	136.213068428455\\
0.29931640625	131.077474974049\\
0.2998046875	129.355869897226\\
0.30029296875	119.070084535362\\
0.30078125	131.015088969555\\
0.30126953125	124.993803220118\\
0.3017578125	126.573758735537\\
0.30224609375	126.532309008934\\
0.302734375	111.540216244321\\
0.30322265625	122.854407278154\\
0.3037109375	109.381459759615\\
0.30419921875	129.336456145929\\
0.3046875	123.878688572913\\
0.30517578125	103.445201516651\\
0.3056640625	124.641038787369\\
0.30615234375	115.758774846434\\
0.306640625	124.486524396126\\
0.30712890625	102.537228610737\\
0.3076171875	127.593178205852\\
0.30810546875	118.995140902058\\
0.30859375	129.379948823034\\
0.30908203125	122.236937032315\\
0.3095703125	119.121536295717\\
0.31005859375	122.264272189709\\
0.310546875	128.943665931098\\
0.31103515625	124.39467136829\\
0.3115234375	111.322611792669\\
0.31201171875	116.730845237409\\
0.3125	115.30869100998\\
0.31298828125	113.608370112988\\
0.3134765625	119.696054525674\\
0.31396484375	120.518467200845\\
0.314453125	123.000995278744\\
0.31494140625	118.278744105591\\
0.3154296875	124.201289456759\\
0.31591796875	114.166034953688\\
0.31640625	117.893844293752\\
0.31689453125	97.4150717140906\\
0.3173828125	118.184998769993\\
0.31787109375	109.968746715455\\
0.318359375	113.940545921339\\
0.31884765625	107.552356653922\\
0.3193359375	97.554441044193\\
0.31982421875	125.375191653586\\
0.3203125	111.905816260831\\
0.32080078125	98.9362334641309\\
0.3212890625	111.886895944885\\
0.32177734375	104.854902195509\\
0.322265625	109.972005369061\\
0.32275390625	100.169852364583\\
0.3232421875	116.181132319152\\
0.32373046875	122.392293841743\\
0.32421875	101.881115712183\\
0.32470703125	90.6243217452452\\
0.3251953125	117.908371286054\\
0.32568359375	115.274996963504\\
0.326171875	90.2555869629513\\
0.32666015625	107.711483652166\\
0.3271484375	115.292004527132\\
0.32763671875	97.2355334696157\\
0.328125	112.465042015636\\
0.32861328125	97.7133487642827\\
0.3291015625	106.17327536322\\
0.32958984375	105.253270718955\\
0.330078125	114.755906157189\\
0.33056640625	106.642610547164\\
0.3310546875	93.5544355305899\\
0.33154296875	106.781054924922\\
0.33203125	104.807189051332\\
0.33251953125	114.567263105698\\
0.3330078125	90.9939370528613\\
0.33349609375	102.015755038084\\
0.333984375	94.0997361908787\\
0.33447265625	91.9183616553575\\
0.3349609375	113.306566714914\\
0.33544921875	106.630685893738\\
0.3359375	108.968020963957\\
0.33642578125	101.403830420952\\
0.3369140625	88.9220850404787\\
0.33740234375	108.777364027288\\
0.337890625	91.1705289384241\\
0.33837890625	118.780272010959\\
0.3388671875	92.5099341732337\\
0.33935546875	107.872579307329\\
0.33984375	101.730580077846\\
0.34033203125	97.4391633775043\\
0.3408203125	69.9905907172227\\
0.34130859375	98.1464888569359\\
0.341796875	106.671497283435\\
0.34228515625	108.275846210372\\
0.3427734375	93.6482577019767\\
0.34326171875	92.9951663605504\\
0.34375	107.611757406066\\
0.34423828125	102.605268055651\\
0.3447265625	98.5111179440828\\
0.34521484375	93.3348555503135\\
0.345703125	71.5513061037076\\
0.34619140625	90.7215546059906\\
0.3466796875	108.439881898985\\
0.34716796875	106.236163544003\\
0.34765625	81.7989043132135\\
0.34814453125	75.4769336226711\\
0.3486328125	56.4897054685164\\
0.34912109375	94.3457875497766\\
0.349609375	94.5937309375112\\
0.35009765625	89.6469648030309\\
0.3505859375	98.4206040134107\\
0.35107421875	100.812588976359\\
0.3515625	107.989762184678\\
0.35205078125	103.999159235709\\
0.3525390625	102.553476893565\\
0.35302734375	94.5369385872934\\
0.353515625	51.4174168004479\\
0.35400390625	88.5253747205783\\
0.3544921875	84.8077527310084\\
0.35498046875	93.0041882903467\\
0.35546875	93.643852003611\\
0.35595703125	82.6797647567923\\
0.3564453125	100.714631349974\\
0.35693359375	86.9313176665584\\
0.357421875	87.1030266925935\\
0.35791015625	102.77563529322\\
0.3583984375	76.0917015814455\\
0.35888671875	72.5868211350682\\
0.359375	93.5717823813409\\
0.35986328125	90.7411540483203\\
0.3603515625	87.2030169010621\\
0.36083984375	77.2256339372433\\
0.361328125	74.3489941553874\\
0.36181640625	95.2503799309847\\
0.3623046875	92.2309852945179\\
0.36279296875	93.7237034319655\\
0.36328125	75.3808251726557\\
0.36376953125	97.4756913372602\\
0.3642578125	92.8037105297737\\
0.36474609375	86.7587266005038\\
0.365234375	86.8627889579645\\
0.36572265625	88.7732509915017\\
0.3662109375	96.1225436659613\\
0.36669921875	97.0046969074928\\
0.3671875	79.1398546614846\\
0.36767578125	71.4707026021664\\
0.3681640625	89.7910086741422\\
0.36865234375	91.263365109729\\
0.369140625	84.2181567781558\\
0.36962890625	86.9109569695942\\
0.3701171875	90.4933656686418\\
0.37060546875	92.4316742333212\\
0.37109375	98.5851690344121\\
0.37158203125	75.2438669610694\\
0.3720703125	92.9506525617157\\
0.37255859375	90.1278895380233\\
0.373046875	51.9115640226058\\
0.37353515625	64.5009865762132\\
0.3740234375	89.0660529816454\\
0.37451171875	78.5597469832137\\
0.375	73.0128636534633\\
0.37548828125	103.232245120009\\
0.3759765625	84.661661974466\\
0.37646484375	102.261384331584\\
0.376953125	95.3912861593055\\
0.37744140625	86.3885867519993\\
0.3779296875	82.671781782604\\
0.37841796875	87.0606011159003\\
0.37890625	75.8690239382317\\
0.37939453125	76.1523831590747\\
0.3798828125	82.4842167034051\\
0.38037109375	82.5440783729913\\
0.380859375	86.1136417322803\\
0.38134765625	73.1554406211034\\
0.3818359375	79.4834019593306\\
0.38232421875	94.5989144452886\\
0.3828125	93.9372268258104\\
0.38330078125	68.4854484822012\\
0.3837890625	56.9125468382559\\
0.38427734375	84.242123592442\\
0.384765625	75.6460063270968\\
0.38525390625	86.9795653909627\\
0.3857421875	85.6716270815094\\
0.38623046875	79.0247808306581\\
0.38671875	79.9990066045846\\
0.38720703125	84.8911537990095\\
0.3876953125	87.6373250203799\\
0.38818359375	93.4049799212052\\
0.388671875	91.3027526000526\\
0.38916015625	58.4020144626108\\
0.3896484375	91.2401844793129\\
0.39013671875	89.4084150358547\\
0.390625	77.7055857142543\\
0.39111328125	54.2358966894366\\
0.3916015625	78.2091346094252\\
0.39208984375	84.8301268148623\\
0.392578125	48.3897643241262\\
0.39306640625	79.4463479389155\\
0.3935546875	95.1163036224532\\
0.39404296875	78.3843411257821\\
0.39453125	75.9266793450577\\
0.39501953125	63.2382124304666\\
0.3955078125	84.9633299657106\\
0.39599609375	87.039453094029\\
0.396484375	72.5886606390524\\
0.39697265625	83.1794865049716\\
0.3974609375	72.3691621519633\\
0.39794921875	72.7481613728504\\
0.3984375	76.0291072648271\\
0.39892578125	69.6272580596179\\
0.3994140625	95.8832153725556\\
0.39990234375	77.5444548027143\\
0.400390625	70.1462769474692\\
0.40087890625	70.5414856531707\\
0.4013671875	67.89331190565\\
0.40185546875	89.0279093776633\\
0.40234375	59.370411373423\\
0.40283203125	78.7306336709893\\
0.4033203125	81.8603977883454\\
0.40380859375	81.6836379936231\\
0.404296875	75.4981213314299\\
0.40478515625	57.831623502328\\
0.4052734375	74.8261071238308\\
0.40576171875	74.6110115683929\\
0.40625	80.7434493644305\\
0.40673828125	93.2592751099308\\
0.4072265625	62.9819475920066\\
0.40771484375	78.5753172744681\\
0.408203125	76.2926818755215\\
0.40869140625	82.3595184004207\\
0.4091796875	78.8032790506771\\
0.40966796875	79.3408017502739\\
0.41015625	71.8711543125326\\
0.41064453125	70.9068555953925\\
0.4111328125	61.6881129263248\\
0.41162109375	79.1744123253081\\
0.412109375	82.1796885553442\\
0.41259765625	54.7014223415067\\
0.4130859375	63.5763461989805\\
0.41357421875	82.1947590945089\\
0.4140625	61.4407319600309\\
0.41455078125	82.6965764971593\\
0.4150390625	73.3726680562014\\
0.41552734375	42.2605411807069\\
0.416015625	83.5438425328191\\
0.41650390625	68.099587254946\\
0.4169921875	64.5081225147501\\
0.41748046875	85.4767660089344\\
0.41796875	59.2602304399433\\
0.41845703125	77.4992733626283\\
0.4189453125	77.4319698439434\\
0.41943359375	72.9862875973438\\
0.419921875	77.879197244891\\
0.42041015625	57.2483704978341\\
0.4208984375	70.5587885269105\\
0.42138671875	71.8694479635809\\
0.421875	75.563161313381\\
0.42236328125	85.5736161957137\\
0.4228515625	69.9746946141252\\
0.42333984375	43.5144780415587\\
0.423828125	60.130386323553\\
0.42431640625	68.6893492952118\\
0.4248046875	73.1043652270394\\
0.42529296875	77.1705489496598\\
0.42578125	69.075551269174\\
0.42626953125	69.3074794854417\\
0.4267578125	71.1002049748954\\
0.42724609375	55.533517670437\\
0.427734375	69.5887717909758\\
0.42822265625	66.758146304333\\
0.4287109375	75.5585510817399\\
0.42919921875	86.902273990686\\
0.4296875	86.0551578571734\\
0.43017578125	69.0803399143592\\
0.4306640625	58.7788192007979\\
0.43115234375	63.5251193630812\\
0.431640625	82.8082650179666\\
0.43212890625	74.097881846561\\
0.4326171875	72.8435896956287\\
0.43310546875	56.5609546225338\\
0.43359375	73.6430315210161\\
0.43408203125	78.0595295664824\\
0.4345703125	74.2360700244928\\
0.43505859375	80.7884681628656\\
0.435546875	71.2617912131415\\
0.43603515625	78.4260984199121\\
0.4365234375	80.7036490345797\\
0.43701171875	70.0586610509453\\
0.4375	86.06625754829\\
0.43798828125	83.5169821083761\\
0.4384765625	66.0890491378844\\
0.43896484375	77.7015252423051\\
0.439453125	64.095639103231\\
0.43994140625	58.6363547868862\\
0.4404296875	63.3330441546004\\
0.44091796875	62.5440851392866\\
0.44140625	67.4694297209486\\
0.44189453125	70.3651188664651\\
0.4423828125	62.8186134044609\\
0.44287109375	45.5962463135675\\
0.443359375	75.8237276261749\\
0.44384765625	75.2060244603461\\
0.4443359375	65.6842137027052\\
0.44482421875	52.2036364602129\\
0.4453125	73.4483062098259\\
0.44580078125	75.4546272241189\\
0.4462890625	62.0316743112908\\
0.44677734375	68.5247076737053\\
0.447265625	75.8076425315059\\
0.44775390625	62.112441358633\\
0.4482421875	80.1630880942974\\
0.44873046875	68.6129212606423\\
0.44921875	77.1636691848826\\
0.44970703125	76.1540434082099\\
0.4501953125	71.977357591261\\
0.45068359375	59.9190421351614\\
0.451171875	71.9447767968937\\
0.45166015625	63.9647889767036\\
0.4521484375	77.0498522360035\\
0.45263671875	40.2210066708127\\
0.453125	60.7083101249311\\
0.45361328125	47.7172253660175\\
0.4541015625	71.4005784934188\\
0.45458984375	67.6694660557573\\
0.455078125	76.0366656762903\\
0.45556640625	76.5809766408455\\
0.4560546875	55.2342505572461\\
0.45654296875	30.6042421321089\\
0.45703125	77.0922088372337\\
0.45751953125	70.8827213400039\\
0.4580078125	67.9992297833368\\
0.45849609375	61.4473154128899\\
0.458984375	49.6908500533143\\
0.45947265625	67.5103141613899\\
0.4599609375	66.9957965657868\\
0.46044921875	73.2183970486448\\
0.4609375	49.9470617186775\\
0.46142578125	76.0415183356978\\
0.4619140625	79.2115295425376\\
0.46240234375	75.376650598385\\
0.462890625	75.065106257554\\
0.46337890625	67.9168444232513\\
0.4638671875	60.0377507482099\\
0.46435546875	46.0918061154613\\
0.46484375	70.1053278576637\\
0.46533203125	65.4414726208739\\
0.4658203125	42.6568634136514\\
0.46630859375	50.1828823503406\\
0.466796875	69.955860075827\\
0.46728515625	38.0860106624307\\
0.4677734375	21.5793595471563\\
0.46826171875	73.2452166486226\\
0.46875	54.925840592288\\
0.46923828125	62.449540822704\\
0.4697265625	66.2733232044511\\
0.47021484375	75.3507771594166\\
0.470703125	70.3557308006611\\
0.47119140625	52.4890083826809\\
0.4716796875	74.0572852909179\\
0.47216796875	77.4434567181829\\
0.47265625	49.9694589930238\\
0.47314453125	69.2234426335349\\
0.4736328125	68.0728393658292\\
0.47412109375	72.1382511413395\\
0.474609375	64.6657839425725\\
0.47509765625	57.0724728668928\\
0.4755859375	59.9405176192656\\
0.47607421875	62.9265483657594\\
0.4765625	39.2879984331628\\
0.47705078125	74.8930070237578\\
0.4775390625	69.7064158577123\\
0.47802734375	73.2753379634603\\
0.478515625	72.3318094800517\\
0.47900390625	61.1991752830357\\
0.4794921875	74.4337707870397\\
0.47998046875	75.1552932160021\\
0.48046875	58.2577950038163\\
0.48095703125	70.77904163142\\
0.4814453125	75.8926499696775\\
0.48193359375	47.4773870723709\\
0.482421875	65.4212032009274\\
0.48291015625	48.2446110016668\\
0.4833984375	61.8038566716651\\
0.48388671875	57.8547743064364\\
0.484375	65.4104975308357\\
0.48486328125	64.4377343922625\\
0.4853515625	68.6555576438268\\
0.48583984375	37.7652269142676\\
0.486328125	71.1239284787181\\
0.48681640625	65.611684690462\\
0.4873046875	68.1172891112559\\
0.48779296875	52.1332185410164\\
0.48828125	54.0461433344485\\
0.48876953125	70.9522841960966\\
0.4892578125	67.4651676558712\\
0.48974609375	53.9419954888665\\
0.490234375	50.9491923839463\\
0.49072265625	64.7206280544114\\
0.4912109375	73.6234210798236\\
0.49169921875	56.5333013508076\\
0.4921875	56.2893591294219\\
0.49267578125	50.172579895204\\
0.4931640625	69.7494680454585\\
0.49365234375	52.4678332455548\\
0.494140625	66.7862045690259\\
0.49462890625	33.6963489554577\\
0.4951171875	47.4677084292134\\
0.49560546875	62.3983880153342\\
0.49609375	59.2495016564297\\
0.49658203125	10.0978307381054\\
0.4970703125	59.9472975198783\\
0.49755859375	69.5924688380744\\
0.498046875	61.6021084447279\\
0.49853515625	40.3134941157197\\
0.4990234375	42.5859926072694\\
0.49951171875	67.2630992747107\\
0.5	67.1430981507742\\
0.50048828125	70.0743958791518\\
0.5009765625	67.294355771\\
0.50146484375	57.8203314924683\\
0.501953125	53.0846330952622\\
0.50244140625	50.8086690814547\\
0.5029296875	66.3731522568525\\
0.50341796875	61.5515389087488\\
0.50390625	48.4233735608439\\
0.50439453125	53.0748590897242\\
0.5048828125	59.8794600070126\\
0.50537109375	59.3317992853082\\
0.505859375	69.6107980528551\\
0.50634765625	67.3266311060325\\
0.5068359375	51.4292299481309\\
0.50732421875	46.0157870968864\\
0.5078125	60.3460382061071\\
0.50830078125	62.0877110237799\\
0.5087890625	73.2752613923338\\
0.50927734375	42.2492885826134\\
0.509765625	45.0438389786676\\
0.51025390625	66.5276628660472\\
0.5107421875	61.3818643781663\\
0.51123046875	46.3937516303101\\
0.51171875	71.6671268976006\\
0.51220703125	39.1794408699781\\
0.5126953125	51.6967404437566\\
0.51318359375	61.6391689335106\\
0.513671875	57.8366307336544\\
0.51416015625	53.6607164817719\\
0.5146484375	53.2427562659799\\
0.51513671875	75.2759677256233\\
0.515625	59.905806905265\\
0.51611328125	62.137976380583\\
0.5166015625	68.4570307153526\\
0.51708984375	47.223362665054\\
0.517578125	38.2528316724364\\
0.51806640625	48.1319917918437\\
0.5185546875	56.2321815727253\\
0.51904296875	55.9304722416927\\
0.51953125	61.2983160631103\\
0.52001953125	65.536944490969\\
0.5205078125	46.9978520681803\\
0.52099609375	44.8629716663984\\
0.521484375	66.1434819266742\\
0.52197265625	61.2647273048481\\
0.5224609375	62.1859402522525\\
0.52294921875	56.8218211837663\\
0.5234375	64.442929847804\\
0.52392578125	58.5125644078761\\
0.5244140625	48.1717356183155\\
0.52490234375	58.4816749085633\\
0.525390625	55.8896198811675\\
0.52587890625	50.7647045487276\\
0.5263671875	28.7635061345244\\
0.52685546875	56.2049701103465\\
0.52734375	42.443956120862\\
0.52783203125	63.5081870330274\\
0.5283203125	67.1152824893005\\
0.52880859375	36.3046506374415\\
0.529296875	58.9155927681137\\
0.52978515625	55.215275686094\\
0.5302734375	67.8901035674373\\
0.53076171875	43.5679055554657\\
0.53125	45.3624601705454\\
0.53173828125	55.994121427409\\
0.5322265625	23.9890280925568\\
0.53271484375	62.9010159812599\\
0.533203125	57.1164441263737\\
0.53369140625	59.7247179645559\\
0.5341796875	64.1120474608094\\
0.53466796875	37.8674578592657\\
0.53515625	64.0370765009604\\
0.53564453125	43.7551876190416\\
0.5361328125	54.341322530387\\
0.53662109375	52.7265002300318\\
0.537109375	30.3127248999677\\
0.53759765625	55.2365833874077\\
0.5380859375	64.038146420111\\
0.53857421875	59.9094920144068\\
0.5390625	65.0006147135874\\
0.53955078125	64.6761555577853\\
0.5400390625	61.0756910119393\\
0.54052734375	55.1687098678079\\
0.541015625	56.3425906510728\\
0.54150390625	60.4329160552536\\
0.5419921875	42.568294891254\\
0.54248046875	66.010083182495\\
0.54296875	55.8086245773392\\
0.54345703125	32.8720437619321\\
0.5439453125	53.0852757641109\\
0.54443359375	51.9415229531067\\
0.544921875	37.9566168537904\\
0.54541015625	66.8590869458186\\
0.5458984375	54.8423361948711\\
0.54638671875	53.0493881758388\\
0.546875	40.4192508782062\\
0.54736328125	59.8312202735549\\
0.5478515625	38.9869434242087\\
0.54833984375	60.409851855089\\
0.548828125	36.4614836024648\\
0.54931640625	57.582586503763\\
0.5498046875	49.9413629487257\\
0.55029296875	45.3331890863119\\
0.55078125	44.4171930113491\\
0.55126953125	57.9300285708023\\
0.5517578125	59.5369141926766\\
0.55224609375	46.4434155770191\\
0.552734375	52.1418422014848\\
0.55322265625	47.60717238738\\
0.5537109375	45.5108954817255\\
0.55419921875	49.2364005212963\\
0.5546875	53.5336972386606\\
0.55517578125	51.7674769726513\\
0.5556640625	40.6721068794377\\
0.55615234375	60.0838283319336\\
0.556640625	62.1681614403948\\
0.55712890625	53.3063513357231\\
0.5576171875	47.856061762906\\
0.55810546875	54.5569116802864\\
0.55859375	57.8817107472837\\
0.55908203125	41.8800414978164\\
0.5595703125	56.3618006609499\\
0.56005859375	50.650236546264\\
0.560546875	55.3990810152612\\
0.56103515625	37.8681577633396\\
0.5615234375	56.8373479127688\\
0.56201171875	51.4011282428522\\
0.5625	54.0231231526663\\
0.56298828125	37.621925649119\\
0.5634765625	59.5464460605248\\
0.56396484375	55.147722752342\\
0.564453125	64.6131709939013\\
0.56494140625	49.3034918697316\\
0.5654296875	36.5555404037948\\
0.56591796875	50.5411104778419\\
0.56640625	37.6891209301453\\
0.56689453125	51.6177948994478\\
0.5673828125	57.1579620550524\\
0.56787109375	60.7310779483362\\
0.568359375	35.7355700162374\\
0.56884765625	39.6813897587285\\
0.5693359375	45.2036164754151\\
0.56982421875	59.3540564071116\\
0.5703125	52.9230656781361\\
0.57080078125	45.3159647850074\\
0.5712890625	67.7571996252966\\
0.57177734375	51.1144443898591\\
0.572265625	48.5765991518103\\
0.57275390625	28.7623636941248\\
0.5732421875	47.4290629116676\\
0.57373046875	61.793867021419\\
0.57421875	55.4144881509307\\
0.57470703125	63.160695693402\\
0.5751953125	51.5522795340378\\
0.57568359375	52.3394719569775\\
0.576171875	45.6164849603436\\
0.57666015625	49.0341438687114\\
0.5771484375	50.031103229543\\
0.57763671875	55.1230056994888\\
0.578125	52.2627646335648\\
0.57861328125	57.0978559791041\\
0.5791015625	63.2177119845114\\
0.57958984375	64.673234049342\\
0.580078125	34.2675448565524\\
0.58056640625	35.095703737201\\
0.5810546875	50.9221078254738\\
0.58154296875	39.2270424577949\\
0.58203125	46.616288299957\\
0.58251953125	60.9382233073804\\
0.5830078125	31.7488828852792\\
0.58349609375	54.7169809787939\\
0.583984375	14.797303972492\\
0.58447265625	59.7390820926263\\
0.5849609375	40.9265533858646\\
0.58544921875	62.2177591756257\\
0.5859375	44.2381963736329\\
0.58642578125	55.1562538311948\\
0.5869140625	24.641669801539\\
0.58740234375	44.1054182455765\\
0.587890625	34.6576224874165\\
0.58837890625	41.6929615575858\\
0.5888671875	51.3416161937445\\
0.58935546875	52.0214658392224\\
0.58984375	56.7531784610841\\
0.59033203125	38.297355680794\\
0.5908203125	25.2876545651028\\
0.59130859375	57.3941294260301\\
0.591796875	43.5150620473082\\
0.59228515625	49.2457439109725\\
0.5927734375	47.1065351603166\\
0.59326171875	51.6849099503477\\
0.59375	58.0159366905866\\
0.59423828125	32.9965127781723\\
0.5947265625	60.5407165743691\\
0.59521484375	45.0953513701232\\
0.595703125	41.5084673720554\\
0.59619140625	57.2207890247478\\
0.5966796875	36.988034674143\\
0.59716796875	39.0779349443769\\
0.59765625	42.8790210360906\\
0.59814453125	38.8558592291756\\
0.5986328125	58.8912869397482\\
0.59912109375	46.2607342440165\\
0.599609375	60.6560058741345\\
0.60009765625	38.0367984429781\\
0.6005859375	51.1413588785292\\
0.60107421875	52.4031983838119\\
0.6015625	60.0340044373205\\
0.60205078125	50.9812153430712\\
0.6025390625	53.4061056905319\\
0.60302734375	45.6166889312374\\
0.603515625	59.8892125225032\\
0.60400390625	44.5519543740375\\
0.6044921875	40.1586034000358\\
0.60498046875	58.814529297063\\
0.60546875	53.2177179208383\\
0.60595703125	51.5960930902362\\
0.6064453125	42.3992218644352\\
0.60693359375	52.5150956635282\\
0.607421875	50.8445435583375\\
0.60791015625	64.3753726943666\\
0.6083984375	44.0018091303676\\
0.60888671875	42.0780190610069\\
0.609375	50.3488304657699\\
0.60986328125	32.1781986505957\\
0.6103515625	40.1542959707111\\
0.61083984375	47.3062802540942\\
0.611328125	51.4517482677845\\
0.61181640625	44.6641624675461\\
0.6123046875	35.135677903221\\
0.61279296875	42.6813625965345\\
0.61328125	35.031583199012\\
0.61376953125	57.8527843168726\\
0.6142578125	43.9892044863667\\
0.61474609375	40.058450520527\\
0.615234375	56.958552978351\\
0.61572265625	41.3995173288067\\
0.6162109375	37.5571456093488\\
0.61669921875	52.5103706877076\\
0.6171875	52.0721473421456\\
0.61767578125	48.4203142658834\\
0.6181640625	61.9584187524219\\
0.61865234375	54.673710250738\\
0.619140625	54.7150641980399\\
0.61962890625	44.0014308529737\\
0.6201171875	57.8230737143636\\
0.62060546875	53.2851725920465\\
0.62109375	54.2165743000606\\
0.62158203125	55.2635324046175\\
0.6220703125	42.8488199368896\\
0.62255859375	41.3601864965032\\
0.623046875	35.0074864057486\\
0.62353515625	51.7385271529744\\
0.6240234375	48.7416320003331\\
0.62451171875	41.0253721430279\\
0.625	43.9723654666295\\
0.62548828125	47.0458902559742\\
0.6259765625	27.972456135444\\
0.62646484375	58.0785708903\\
0.626953125	43.302795842241\\
0.62744140625	48.9362563475711\\
0.6279296875	42.3478833887678\\
0.62841796875	36.7025131141159\\
0.62890625	37.6568616603932\\
0.62939453125	52.5940949624744\\
0.6298828125	34.4299135487945\\
0.63037109375	57.066219289165\\
0.630859375	54.593879571645\\
0.63134765625	49.9365900622371\\
0.6318359375	49.7640317562075\\
0.63232421875	49.480254704262\\
0.6328125	59.2192499951171\\
0.63330078125	51.4771636774529\\
0.6337890625	56.7133412230539\\
0.63427734375	27.7009158172488\\
0.634765625	44.6491642139643\\
0.63525390625	40.1130062713107\\
0.6357421875	59.1891502813304\\
0.63623046875	54.8198759152316\\
0.63671875	52.5536743842082\\
0.63720703125	50.0925132988705\\
0.6376953125	61.1599245881689\\
0.63818359375	44.9859980415438\\
0.638671875	42.4572029794675\\
0.63916015625	51.582115977299\\
0.6396484375	33.6875586325525\\
0.64013671875	54.0744973050832\\
0.640625	48.3527726118811\\
0.64111328125	59.2479576189292\\
0.6416015625	37.3481408830577\\
0.64208984375	43.8506660516209\\
0.642578125	49.9027195847553\\
0.64306640625	53.3972586094227\\
0.6435546875	48.0230205499542\\
0.64404296875	39.861838146095\\
0.64453125	61.3792295193683\\
0.64501953125	21.8641184909345\\
0.6455078125	37.5478057192036\\
0.64599609375	57.4221705778324\\
0.646484375	57.6034084272752\\
0.64697265625	42.700694300533\\
0.6474609375	44.251346109368\\
0.64794921875	58.0129186444669\\
0.6484375	50.4993579364324\\
0.64892578125	30.1514385596209\\
0.6494140625	52.0450125720173\\
0.64990234375	53.4920429691415\\
0.650390625	46.2258106558776\\
0.65087890625	34.5185054149235\\
0.6513671875	50.5338878544916\\
0.65185546875	38.1847032554012\\
0.65234375	51.1196156090288\\
0.65283203125	52.2343864389611\\
0.6533203125	43.4793811374281\\
0.65380859375	54.8433040210523\\
0.654296875	48.0459487040736\\
0.65478515625	49.9819976586184\\
0.6552734375	49.0943068879421\\
0.65576171875	53.7715662437469\\
0.65625	42.9599200090824\\
0.65673828125	40.8102785575504\\
0.6572265625	50.7268630574516\\
0.65771484375	54.2217061307459\\
0.658203125	49.2412992782773\\
0.65869140625	47.3292116215126\\
0.6591796875	37.5276327619808\\
0.65966796875	48.9761244004859\\
0.66015625	38.9880715222038\\
0.66064453125	38.8229806195887\\
0.6611328125	55.1672396053581\\
0.66162109375	48.9312583154945\\
0.662109375	22.4425003185645\\
0.66259765625	46.6136893181259\\
0.6630859375	54.3711730314302\\
0.66357421875	42.5772934569266\\
0.6640625	45.1074404985612\\
0.66455078125	40.0645825909009\\
0.6650390625	44.7664692714724\\
0.66552734375	55.283815050965\\
0.666015625	33.5319325818427\\
0.66650390625	47.1887536745604\\
0.6669921875	50.2299604458079\\
0.66748046875	54.2102255243619\\
0.66796875	41.7308611331984\\
0.66845703125	-27.2321737296978\\
0.6689453125	44.8433486315044\\
0.66943359375	33.0182197860015\\
0.669921875	24.4586943975186\\
0.67041015625	51.288640356667\\
0.6708984375	22.609911178992\\
0.67138671875	40.3989605843218\\
0.671875	27.5314988811155\\
0.67236328125	35.0894053454286\\
0.6728515625	49.5508066154951\\
0.67333984375	55.6867130569078\\
0.673828125	49.0310380261034\\
0.67431640625	50.3155726504501\\
0.6748046875	32.5611871954283\\
0.67529296875	50.8250119332707\\
0.67578125	49.2675498432204\\
0.67626953125	39.1058058504011\\
0.6767578125	54.7420348625918\\
0.67724609375	40.8926386662004\\
0.677734375	58.0188918025352\\
0.67822265625	40.7445939237692\\
0.6787109375	52.4742381168915\\
0.67919921875	36.851534859214\\
0.6796875	51.1718566995732\\
0.68017578125	44.9748168060549\\
0.6806640625	30.7616533163645\\
0.68115234375	55.9188960200147\\
0.681640625	40.7897079245983\\
0.68212890625	39.5591897146345\\
0.6826171875	34.4251795763047\\
0.68310546875	48.0618686540546\\
0.68359375	26.6208991087815\\
0.68408203125	54.2834670715726\\
0.6845703125	40.9558094201884\\
0.68505859375	39.4873620535556\\
0.685546875	38.4218372700105\\
0.68603515625	22.2162146884655\\
0.6865234375	55.4067579681118\\
0.68701171875	51.176017186954\\
0.6875	33.4062317897272\\
0.68798828125	36.984393040232\\
0.6884765625	51.015287910334\\
0.68896484375	54.8721844965755\\
0.689453125	44.2567884734635\\
0.68994140625	49.5160210121873\\
0.6904296875	41.1318961278602\\
0.69091796875	28.3060236531429\\
0.69140625	50.27415689325\\
0.69189453125	50.3241629463028\\
0.6923828125	59.6035067494399\\
0.69287109375	28.1345961726443\\
0.693359375	21.5556460954392\\
0.69384765625	46.6061579894786\\
0.6943359375	47.5481148043597\\
0.69482421875	49.1461975669638\\
0.6953125	52.7144046638776\\
0.69580078125	45.2999629740438\\
0.6962890625	41.6107179884611\\
0.69677734375	45.1028185465652\\
0.697265625	27.1367258434307\\
0.69775390625	43.3863781280846\\
0.6982421875	42.0094769750635\\
0.69873046875	52.5155534214216\\
0.69921875	31.8197724258283\\
0.69970703125	46.5424740191315\\
0.7001953125	56.3105608616227\\
0.70068359375	30.4270169224459\\
0.701171875	53.8957836482829\\
0.70166015625	57.1867765594311\\
0.7021484375	53.2047208471266\\
0.70263671875	39.7087269569436\\
0.703125	35.6571806115267\\
0.70361328125	43.0856516088732\\
0.7041015625	36.4754562682157\\
0.70458984375	15.8652955797172\\
0.705078125	48.1774149223896\\
0.70556640625	40.0556082225398\\
0.7060546875	50.7322635106469\\
0.70654296875	45.0285145769331\\
0.70703125	46.773143842771\\
0.70751953125	52.2511883582356\\
0.7080078125	49.8257486197598\\
0.70849609375	38.9451996754724\\
0.708984375	42.3703189031401\\
0.70947265625	35.8786941818442\\
0.7099609375	37.3069810374299\\
0.71044921875	48.5061257751547\\
0.7109375	42.7397319272285\\
0.71142578125	13.4897369121735\\
0.7119140625	42.7233225973657\\
0.71240234375	46.3103643435764\\
0.712890625	50.7223912693383\\
0.71337890625	42.8623983894626\\
0.7138671875	56.2904658722117\\
0.71435546875	32.5369642452176\\
0.71484375	26.4282187255651\\
0.71533203125	50.4178074167823\\
0.7158203125	49.0072391033175\\
0.71630859375	49.3616051867621\\
0.716796875	50.8242358116882\\
0.71728515625	29.6549528478027\\
0.7177734375	50.9711052049141\\
0.71826171875	48.2197358732332\\
0.71875	53.1416227050374\\
0.71923828125	51.2175469793842\\
0.7197265625	53.9376442084043\\
0.72021484375	25.9359332160916\\
0.720703125	33.3122565667728\\
0.72119140625	38.9271318000825\\
0.7216796875	33.3658560353333\\
0.72216796875	23.5339769427703\\
0.72265625	36.6059312475776\\
0.72314453125	37.3227246227807\\
0.7236328125	15.8954243653898\\
0.72412109375	51.9735696220134\\
0.724609375	45.3219773845359\\
0.72509765625	26.3574180810684\\
0.7255859375	39.4163458658482\\
0.72607421875	36.8104216664421\\
0.7265625	41.4478121445594\\
0.72705078125	53.0732283071195\\
0.7275390625	28.8466932708145\\
0.72802734375	51.6184412950253\\
0.728515625	40.8404128182723\\
0.72900390625	30.9842095401946\\
0.7294921875	47.6126475113583\\
0.72998046875	29.7464276133951\\
0.73046875	42.3087853773626\\
0.73095703125	49.4029845478529\\
0.7314453125	49.7809583531296\\
0.73193359375	41.0947243896518\\
0.732421875	57.1953460147511\\
0.73291015625	28.7087855688821\\
0.7333984375	43.512754335408\\
0.73388671875	33.1471346576457\\
0.734375	55.7853193558215\\
0.73486328125	45.486072265442\\
0.7353515625	27.3927176762964\\
0.73583984375	18.8516068501103\\
0.736328125	42.5157902018265\\
0.73681640625	41.7337343238276\\
0.7373046875	47.798755622161\\
0.73779296875	50.7796820202199\\
0.73828125	38.914658466152\\
0.73876953125	47.1872317592519\\
0.7392578125	50.9539301186158\\
0.73974609375	2.46596186625821\\
0.740234375	44.1809319654986\\
0.74072265625	51.5053680360952\\
0.7412109375	43.6225209554537\\
0.74169921875	50.129745558551\\
0.7421875	24.3054935324122\\
0.74267578125	59.4687688601158\\
0.7431640625	41.9488721230532\\
0.74365234375	53.6079620656137\\
0.744140625	30.3028910667034\\
0.74462890625	45.6259097784978\\
0.7451171875	39.5355111249197\\
0.74560546875	36.9478921255809\\
0.74609375	25.3581352659551\\
0.74658203125	44.7080031109137\\
0.7470703125	47.1997311457118\\
0.74755859375	45.6466111533303\\
0.748046875	35.3728773022151\\
0.74853515625	49.2229701179269\\
0.7490234375	43.9058406537383\\
0.74951171875	34.4848862528429\\
0.75	32.7456041501024\\
0.75048828125	52.6234323115871\\
0.7509765625	52.3351729975746\\
0.75146484375	30.6301829435467\\
0.751953125	47.5443549084638\\
0.75244140625	33.8663751367907\\
0.7529296875	44.9950463683825\\
0.75341796875	47.1987423090936\\
0.75390625	46.5472987447314\\
0.75439453125	33.3257516826129\\
0.7548828125	46.399031565173\\
0.75537109375	40.9036115728959\\
0.755859375	43.3733593018383\\
0.75634765625	23.087865421113\\
0.7568359375	40.150104032345\\
0.75732421875	43.8274454658471\\
0.7578125	54.0635627492326\\
0.75830078125	29.2898638263873\\
0.7587890625	35.292091826408\\
0.75927734375	48.7044934690014\\
0.759765625	47.2596176560027\\
0.76025390625	49.7302364735323\\
0.7607421875	46.0114559733171\\
0.76123046875	19.7216112656354\\
0.76171875	28.3446696744649\\
0.76220703125	33.6569553955504\\
0.7626953125	33.5366445926464\\
0.76318359375	28.5164502535479\\
0.763671875	43.8000543536494\\
0.76416015625	44.2074425700368\\
0.7646484375	42.6221860942878\\
0.76513671875	41.3786737436848\\
0.765625	11.2287551386999\\
0.76611328125	50.0971256671615\\
0.7666015625	42.9836954599725\\
0.76708984375	24.9430073087264\\
0.767578125	50.2485541041187\\
0.76806640625	53.1627986583842\\
0.7685546875	47.1884272284191\\
0.76904296875	50.1508199969704\\
0.76953125	45.3338447300032\\
0.77001953125	50.2283963186616\\
0.7705078125	37.428174571647\\
0.77099609375	40.6450238867491\\
0.771484375	47.5754310481857\\
0.77197265625	52.3766972325422\\
0.7724609375	55.375031640778\\
0.77294921875	49.2236229145295\\
0.7734375	23.2466141389757\\
0.77392578125	45.4968368638027\\
0.7744140625	46.7973013884585\\
0.77490234375	38.0361522198501\\
0.775390625	47.7903072793443\\
0.77587890625	38.3998350397347\\
0.7763671875	32.2155229486885\\
0.77685546875	54.0601595026151\\
0.77734375	41.6986165203761\\
0.77783203125	43.9782914869546\\
0.7783203125	37.619120795861\\
0.77880859375	43.119612049402\\
0.779296875	18.8876233865461\\
0.77978515625	53.0970386510777\\
0.7802734375	45.7295660676777\\
0.78076171875	43.515319166761\\
0.78125	54.9615361185452\\
0.78173828125	52.5483939271214\\
0.7822265625	46.7445840235042\\
0.78271484375	44.8310078790428\\
0.783203125	46.4848641226904\\
0.78369140625	33.3167815467614\\
0.7841796875	47.6301427369623\\
0.78466796875	34.6692413879144\\
0.78515625	45.6466379723518\\
0.78564453125	50.3756218208517\\
0.7861328125	21.7670270797915\\
0.78662109375	49.7455510144906\\
0.787109375	46.4382079720232\\
0.78759765625	47.5326202847897\\
0.7880859375	35.0653200982495\\
0.78857421875	45.348470136215\\
0.7890625	47.8696646108448\\
0.78955078125	30.970955857904\\
0.7900390625	40.8355013078969\\
0.79052734375	37.3618911729128\\
0.791015625	30.8886604069447\\
0.79150390625	41.4067199544535\\
0.7919921875	53.9754839842326\\
0.79248046875	45.7315083972733\\
0.79296875	51.7487121540236\\
0.79345703125	47.1527910287819\\
0.7939453125	29.5316044110145\\
0.79443359375	42.9426924530696\\
0.794921875	42.9082501041091\\
0.79541015625	47.5906418082669\\
0.7958984375	44.3504411057627\\
0.79638671875	51.6604965249402\\
0.796875	37.9851780053435\\
0.79736328125	56.2800383601444\\
0.7978515625	44.0580655989013\\
0.79833984375	46.6695470871023\\
0.798828125	40.2747003408485\\
0.79931640625	45.4108864103539\\
0.7998046875	48.9637805316989\\
0.80029296875	42.0757035281821\\
0.80078125	38.9961932409505\\
0.80126953125	29.1965416877666\\
0.8017578125	47.7406358922088\\
0.80224609375	28.0123124581705\\
0.802734375	38.7203598778779\\
0.80322265625	44.5545085941169\\
0.8037109375	41.628363365925\\
0.80419921875	49.3679339759681\\
0.8046875	32.5937821380725\\
0.80517578125	43.7680079781556\\
0.8056640625	37.4669614381221\\
0.80615234375	53.3127323367903\\
0.806640625	54.0709052751848\\
0.80712890625	43.2675155603728\\
0.8076171875	46.6738707975027\\
0.80810546875	41.1465659601062\\
0.80859375	47.9158317432528\\
0.80908203125	12.5975737980161\\
0.8095703125	51.0313879744667\\
0.81005859375	41.8223567541635\\
0.810546875	41.0873236812038\\
0.81103515625	43.8936986617083\\
0.8115234375	41.7906266218909\\
0.81201171875	43.2521764176953\\
0.8125	40.0532618496296\\
0.81298828125	41.8370505375378\\
0.8134765625	42.4873549421427\\
0.81396484375	29.3522100703595\\
0.814453125	40.0236079512066\\
0.81494140625	34.4776631129435\\
0.8154296875	50.6908376729855\\
0.81591796875	56.2464929710623\\
0.81640625	53.9706024604443\\
0.81689453125	45.9199738542663\\
0.8173828125	43.5253525482506\\
0.81787109375	44.7184683672503\\
0.818359375	49.3094612309956\\
0.81884765625	47.7004637344394\\
0.8193359375	45.2606863359434\\
0.81982421875	40.0096925284874\\
0.8203125	32.9991165770075\\
0.82080078125	39.2538251515482\\
0.8212890625	49.1971265802953\\
0.82177734375	35.1077249663378\\
0.822265625	43.6916458137032\\
0.82275390625	53.7000999910201\\
0.8232421875	44.7913049750995\\
0.82373046875	25.5499985658215\\
0.82421875	41.0195696160033\\
0.82470703125	35.8432395586108\\
0.8251953125	39.7945221865064\\
0.82568359375	47.0889422722876\\
0.826171875	42.8574505213743\\
0.82666015625	36.1566318766863\\
0.8271484375	46.8927562202544\\
0.82763671875	43.2166404038611\\
0.828125	43.1395934094635\\
0.82861328125	43.3568020482249\\
0.8291015625	54.4823529587517\\
0.82958984375	30.0410768343416\\
0.830078125	53.0803673879827\\
0.83056640625	50.5474399788327\\
0.8310546875	42.3428010970264\\
0.83154296875	38.3115758603886\\
0.83203125	52.5613629055855\\
0.83251953125	44.5173164059295\\
0.8330078125	29.8354821682162\\
0.83349609375	40.8429735781106\\
0.833984375	42.8375173715199\\
0.83447265625	47.8554099016566\\
0.8349609375	43.4806263446704\\
0.83544921875	42.5077285447197\\
0.8359375	40.1425488596297\\
0.83642578125	42.6398445749652\\
0.8369140625	37.242170791997\\
0.83740234375	46.4828752365534\\
0.837890625	54.0088444354714\\
0.83837890625	38.9535417754841\\
0.8388671875	48.6960585392652\\
0.83935546875	41.8120652968637\\
0.83984375	34.8442798380436\\
0.84033203125	36.9464653164577\\
0.8408203125	44.742205181404\\
0.84130859375	42.9069178257334\\
0.841796875	41.4131381554837\\
0.84228515625	43.5516205139502\\
0.8427734375	-7.70736069909414\\
0.84326171875	44.6544472583308\\
0.84375	50.6604682580075\\
0.84423828125	39.6933767331735\\
0.8447265625	41.9852035468264\\
0.84521484375	51.6716480093775\\
0.845703125	50.5150279971028\\
0.84619140625	53.1759144828367\\
0.8466796875	29.8754851178666\\
0.84716796875	35.58851083777\\
0.84765625	44.8421760186132\\
0.84814453125	12.6125892671741\\
0.8486328125	39.3838558720086\\
0.84912109375	43.8888665602796\\
0.849609375	47.2300790884342\\
0.85009765625	42.0578448561496\\
0.8505859375	43.2294160794901\\
0.85107421875	46.647448128666\\
0.8515625	13.8513846827674\\
0.85205078125	37.9747529039133\\
0.8525390625	46.1581896471277\\
0.85302734375	33.5198577154312\\
0.853515625	16.2838887351161\\
0.85400390625	26.0342765302261\\
0.8544921875	28.4719738591245\\
0.85498046875	47.5563315647991\\
0.85546875	48.6951899361264\\
0.85595703125	38.6575523503968\\
0.8564453125	37.3817956948407\\
0.85693359375	33.3126114828893\\
0.857421875	51.7711220039768\\
0.85791015625	40.534671157303\\
0.8583984375	41.414571327357\\
0.85888671875	37.3625836224127\\
0.859375	43.2066610843906\\
0.85986328125	37.0695567496114\\
0.8603515625	49.1476279063016\\
0.86083984375	49.5566229186666\\
0.861328125	35.7373319565693\\
0.86181640625	41.6149868290122\\
0.8623046875	40.2639062157054\\
0.86279296875	46.2950444746345\\
0.86328125	40.1750018961916\\
0.86376953125	55.1431154683351\\
0.8642578125	33.6843266074023\\
0.86474609375	43.6327988175179\\
0.865234375	43.1910917361763\\
0.86572265625	44.5850753810241\\
0.8662109375	45.9983558789879\\
0.86669921875	47.7341289773772\\
0.8671875	42.3404382755055\\
0.86767578125	46.7756400277902\\
0.8681640625	46.2431791983483\\
0.86865234375	41.5425934088991\\
0.869140625	47.0560073053347\\
0.86962890625	47.112003691427\\
0.8701171875	41.8939496002963\\
0.87060546875	41.9989122370186\\
0.87109375	46.8395055563567\\
0.87158203125	37.1675251486371\\
0.8720703125	42.5838477540573\\
0.87255859375	40.0221701728291\\
0.873046875	42.2115522459495\\
0.87353515625	46.7072218715588\\
0.8740234375	43.0445726715209\\
0.87451171875	47.9168826773152\\
0.875	49.6874234398371\\
0.87548828125	44.6192910245592\\
0.8759765625	39.8974018855034\\
0.87646484375	32.6314557639789\\
0.876953125	43.1437377729054\\
0.87744140625	51.7966127105027\\
0.8779296875	35.7172136051303\\
0.87841796875	43.2253814051419\\
0.87890625	46.2895752953958\\
0.87939453125	50.6034846541222\\
0.8798828125	31.0262844653759\\
0.88037109375	45.8984246777557\\
0.880859375	34.0980049772003\\
0.88134765625	34.0298057994759\\
0.8818359375	33.8206513845963\\
0.88232421875	40.0889142388846\\
0.8828125	42.7341322610453\\
0.88330078125	35.7518149546364\\
0.8837890625	45.7555980575179\\
0.88427734375	45.8632962921911\\
0.884765625	41.2638894958045\\
0.88525390625	33.3585390527398\\
0.8857421875	40.8060570092395\\
0.88623046875	46.3635447007061\\
0.88671875	45.0438897588422\\
0.88720703125	37.8526994350155\\
0.8876953125	41.7096580109791\\
0.88818359375	41.0885697113881\\
0.888671875	40.64737487316\\
0.88916015625	34.3320372534738\\
0.8896484375	34.0621394477473\\
0.89013671875	46.1382544923271\\
0.890625	31.9534204691854\\
0.89111328125	41.4365480729974\\
0.8916015625	39.4579223397516\\
0.89208984375	30.4839843776754\\
0.892578125	32.2807131482769\\
0.89306640625	43.4009893383274\\
0.8935546875	46.2520480500095\\
0.89404296875	46.2023759509688\\
0.89453125	39.4712357717702\\
0.89501953125	46.1041627811392\\
0.8955078125	42.304314838272\\
0.89599609375	34.1333132750245\\
0.896484375	50.8418483283311\\
0.89697265625	40.4577128612634\\
0.8974609375	46.0448906672066\\
0.89794921875	46.9722829418137\\
0.8984375	51.6679158246315\\
0.89892578125	41.9261514221606\\
0.8994140625	39.7529056964156\\
0.89990234375	45.6833490845074\\
0.900390625	32.6071388810485\\
0.90087890625	45.4371921152521\\
0.9013671875	28.2668277236384\\
0.90185546875	36.2529840501327\\
0.90234375	21.3583207164495\\
0.90283203125	41.0474318592352\\
0.9033203125	46.6465084707205\\
0.90380859375	47.5832252690538\\
0.904296875	36.128036809088\\
0.90478515625	37.9572524390508\\
0.9052734375	31.9298030536337\\
0.90576171875	45.7427027037169\\
0.90625	38.6233943259061\\
0.90673828125	42.6001597037451\\
0.9072265625	50.9174721320466\\
0.90771484375	39.6017892998246\\
0.908203125	45.2757677270496\\
0.90869140625	18.1543291494573\\
0.9091796875	44.418354029437\\
0.90966796875	38.2151269010226\\
0.91015625	34.9774253826526\\
0.91064453125	24.3999253685423\\
0.9111328125	14.4740413428112\\
0.91162109375	47.4475073382877\\
0.912109375	44.4046197478882\\
0.91259765625	39.856418838281\\
0.9130859375	46.2214922775729\\
0.91357421875	39.6519131954655\\
0.9140625	38.985172537933\\
0.91455078125	39.0551606720821\\
0.9150390625	47.5934632733967\\
0.91552734375	19.0651083267633\\
0.916015625	40.5972874911962\\
0.91650390625	51.5698967625014\\
0.9169921875	37.6265266986484\\
0.91748046875	43.439874205645\\
0.91796875	42.1649119802676\\
0.91845703125	33.9170924390603\\
0.9189453125	45.8349199497581\\
0.91943359375	46.1216138591101\\
0.919921875	48.3063934175263\\
0.92041015625	48.2994884180402\\
0.9208984375	41.4739038925711\\
0.92138671875	32.4853741242304\\
0.921875	44.6446817627615\\
0.92236328125	44.6792056826809\\
0.9228515625	38.0902294750456\\
0.92333984375	43.4360858402474\\
0.923828125	29.0733729553999\\
0.92431640625	36.3853545350795\\
0.9248046875	35.8666688786712\\
0.92529296875	51.3952995314278\\
0.92578125	44.0670302420818\\
0.92626953125	44.4985935588442\\
0.9267578125	43.0494958299405\\
0.92724609375	46.1802346305035\\
0.927734375	35.7288544015329\\
0.92822265625	42.9269391852168\\
0.9287109375	48.5045284186818\\
0.92919921875	40.1888111773121\\
0.9296875	40.839618234604\\
0.93017578125	48.6143675267679\\
0.9306640625	32.2826003820904\\
0.93115234375	44.0115217986796\\
0.931640625	49.186844330073\\
0.93212890625	39.9110936827676\\
0.9326171875	27.1134314328944\\
0.93310546875	43.4119050977967\\
0.93359375	43.5295219121178\\
0.93408203125	35.8944761189388\\
0.9345703125	34.6033242820394\\
0.93505859375	48.7057640475983\\
0.935546875	38.2977837296855\\
0.93603515625	49.7412747178363\\
0.9365234375	48.0890799331403\\
0.93701171875	44.6694918923472\\
0.9375	37.9816046159824\\
0.93798828125	42.0752753709047\\
0.9384765625	48.8430161652781\\
0.93896484375	34.3802406765181\\
0.939453125	39.6961112689375\\
0.93994140625	45.8198643726858\\
0.9404296875	36.3967920755677\\
0.94091796875	44.6206638394438\\
0.94140625	33.1622252640878\\
0.94189453125	44.2246101520713\\
0.9423828125	48.5563649607792\\
0.94287109375	41.9021046638898\\
0.943359375	42.2982161580112\\
0.94384765625	41.4256168920519\\
0.9443359375	43.6734202735089\\
0.94482421875	45.1957485836251\\
0.9453125	49.3018186507113\\
0.94580078125	41.1330535770987\\
0.9462890625	44.2900122429693\\
0.94677734375	48.4157765027086\\
0.947265625	43.799831745773\\
0.94775390625	43.6440108038541\\
0.9482421875	48.1660744551045\\
0.94873046875	33.1304075205712\\
0.94921875	37.6085006141245\\
0.94970703125	22.1931390588425\\
0.9501953125	44.3265725447562\\
0.95068359375	44.1768232597441\\
0.951171875	44.7255648305601\\
0.95166015625	44.1742698475801\\
0.9521484375	39.8451106819847\\
0.95263671875	48.6631125240989\\
0.953125	44.4247771358031\\
};
\addplot [color=blue,solid]
  table[row sep=crcr]{0.953125	44.4247771358031\\
0.95361328125	43.8742005727595\\
0.9541015625	44.4100226630723\\
0.95458984375	29.9422970079237\\
0.955078125	40.5752989014799\\
0.95556640625	48.9231467096276\\
0.9560546875	42.5888502961844\\
0.95654296875	47.0910799363296\\
0.95703125	38.7740793762205\\
0.95751953125	31.0515881390306\\
0.9580078125	35.6262884034752\\
0.95849609375	48.9743792087344\\
0.958984375	45.2504167322536\\
0.95947265625	46.9963860300816\\
0.9599609375	48.4063109954674\\
0.96044921875	39.9976881187829\\
0.9609375	44.8737921923561\\
0.96142578125	37.6567108134431\\
0.9619140625	44.190667109304\\
0.96240234375	37.9374741245416\\
0.962890625	50.0615448289131\\
0.96337890625	32.5906587469286\\
0.9638671875	49.0058196533875\\
0.96435546875	44.2973093352907\\
0.96484375	48.6427963900913\\
0.96533203125	40.761947026775\\
0.9658203125	32.888064847605\\
0.96630859375	36.5226204382661\\
0.966796875	34.9725829463866\\
0.96728515625	42.1535625844876\\
0.9677734375	52.6522312509704\\
0.96826171875	6.3844416562604\\
0.96875	45.1637004079248\\
0.96923828125	40.0744287954956\\
0.9697265625	44.7165421403155\\
0.97021484375	44.1582784140181\\
0.970703125	41.1951466727692\\
0.97119140625	42.9918377762979\\
0.9716796875	42.8983073415597\\
0.97216796875	41.7715979075418\\
0.97265625	32.1625749100945\\
0.97314453125	44.014043688205\\
0.9736328125	40.1466116153944\\
0.97412109375	36.0494845783666\\
0.974609375	24.0949207684338\\
0.97509765625	37.2670479922896\\
0.9755859375	47.7210512660441\\
0.97607421875	42.0789715404249\\
0.9765625	44.7960494772879\\
0.97705078125	19.0614117468029\\
0.9775390625	40.7505833066707\\
0.97802734375	42.1153432971061\\
0.978515625	36.9770710000144\\
0.97900390625	32.7941377820616\\
0.9794921875	43.0998945638969\\
0.97998046875	46.1158631827025\\
0.98046875	37.0913948145877\\
0.98095703125	30.5224196159158\\
0.9814453125	11.2288090335758\\
0.98193359375	33.2833856890598\\
0.982421875	43.8991569254168\\
0.98291015625	40.0065254206727\\
0.9833984375	41.8884548151595\\
0.98388671875	45.2547770290636\\
0.984375	36.1606327484333\\
0.98486328125	48.0389451578801\\
0.9853515625	48.9993635218621\\
0.98583984375	38.6369050630594\\
0.986328125	44.8308108339865\\
0.98681640625	47.8240529092333\\
0.9873046875	19.0755981506559\\
0.98779296875	47.8069882123036\\
0.98828125	39.3230207667723\\
0.98876953125	41.4517049684532\\
0.9892578125	45.5700282456091\\
0.98974609375	39.2833111010071\\
0.990234375	45.1908882736036\\
0.99072265625	36.8458296397267\\
0.9912109375	41.7319643800855\\
0.99169921875	43.9368999515923\\
0.9921875	40.5894169185746\\
0.99267578125	37.1319099526317\\
0.9931640625	51.2178736272306\\
0.99365234375	47.8293916881961\\
0.994140625	46.4321189774601\\
0.99462890625	31.128277680316\\
0.9951171875	39.5683654331709\\
0.99560546875	43.2942216052166\\
0.99609375	46.4095986210217\\
0.99658203125	44.5163746126886\\
0.9970703125	49.2741246995188\\
0.99755859375	40.2606514159014\\
0.998046875	41.6441182362849\\
0.99853515625	35.5026003346468\\
0.9990234375	42.1004924262986\\
0.99951171875	33.4840305938025\\
};
\addlegendentry{Measured PSD};

\addplot [color=red,solid,forget plot]
  table[row sep=crcr]{-1	15.9904379211962\\
-0.99951171875	15.9904448674156\\
-0.9990234375	15.9904657060886\\
-0.99853515625	15.9905004372604\\
-0.998046875	15.9905490610059\\
-0.99755859375	15.9906115774302\\
-0.9970703125	15.9906879866682\\
-0.99658203125	15.9907782888851\\
-0.99609375	15.9908824842758\\
-0.99560546875	15.9910005730654\\
-0.9951171875	15.9911325555091\\
-0.99462890625	15.9912784318919\\
-0.994140625	15.9914382025288\\
-0.99365234375	15.991611867765\\
-0.9931640625	15.9917994279758\\
-0.99267578125	15.9920008835661\\
-0.9921875	15.9922162349713\\
-0.99169921875	15.9924454826566\\
-0.9912109375	15.9926886271172\\
-0.99072265625	15.9929456688784\\
-0.990234375	15.9932166084957\\
-0.98974609375	15.9935014465543\\
-0.9892578125	15.9938001836697\\
-0.98876953125	15.9941128204875\\
-0.98828125	15.9944393576832\\
-0.98779296875	15.9947797959623\\
-0.9873046875	15.9951341360608\\
-0.98681640625	15.9955023787442\\
-0.986328125	15.9958845248084\\
-0.98583984375	15.9962805750795\\
-0.9853515625	15.9966905304134\\
-0.98486328125	15.9971143916964\\
-0.984375	15.9975521598445\\
-0.98388671875	15.9980038358043\\
-0.9833984375	15.9984694205522\\
-0.98291015625	15.9989489150949\\
-0.982421875	15.999442320469\\
-0.98193359375	15.9999496377415\\
-0.9814453125	16.0004708680095\\
-0.98095703125	16.0010060124001\\
-0.98046875	16.0015550720708\\
-0.97998046875	16.0021180482091\\
-0.9794921875	16.0026949420328\\
-0.97900390625	16.0032857547898\\
-0.978515625	16.0038904877584\\
-0.97802734375	16.0045091422468\\
-0.9775390625	16.0051417195936\\
-0.97705078125	16.0057882211678\\
-0.9765625	16.0064486483684\\
-0.97607421875	16.0071230026248\\
-0.9755859375	16.0078112853965\\
-0.97509765625	16.0085134981734\\
-0.974609375	16.0092296424758\\
-0.97412109375	16.0099597198541\\
-0.9736328125	16.0107037318892\\
-0.97314453125	16.011461680192\\
-0.97265625	16.0122335664043\\
-0.97216796875	16.0130193921976\\
-0.9716796875	16.0138191592742\\
-0.97119140625	16.0146328693667\\
-0.970703125	16.0154605242381\\
-0.97021484375	16.0163021256816\\
-0.9697265625	16.017157675521\\
-0.96923828125	16.0180271756105\\
-0.96875	16.0189106278349\\
-0.96826171875	16.0198080341091\\
-0.9677734375	16.0207193963787\\
-0.96728515625	16.0216447166199\\
-0.966796875	16.0225839968392\\
-0.96630859375	16.0235372390736\\
-0.9658203125	16.0245044453909\\
-0.96533203125	16.0254856178891\\
-0.96484375	16.026480758697\\
-0.96435546875	16.0274898699739\\
-0.9638671875	16.0285129539098\\
-0.96337890625	16.0295500127252\\
-0.962890625	16.0306010486713\\
-0.96240234375	16.0316660640298\\
-0.9619140625	16.0327450611133\\
-0.96142578125	16.033838042265\\
-0.9609375	16.0349450098586\\
-0.96044921875	16.0360659662989\\
-0.9599609375	16.0372009140211\\
-0.95947265625	16.0383498554914\\
-0.958984375	16.0395127932066\\
-0.95849609375	16.0406897296943\\
-0.9580078125	16.0418806675132\\
-0.95751953125	16.0430856092524\\
-0.95703125	16.0443045575321\\
-0.95654296875	16.0455375150035\\
-0.9560546875	16.0467844843484\\
-0.95556640625	16.0480454682798\\
-0.955078125	16.0493204695413\\
-0.95458984375	16.0506094909078\\
-0.9541015625	16.0519125351849\\
-0.95361328125	16.0532296052095\\
-0.953125	16.0545607038492\\
-0.95263671875	16.0559058340029\\
-0.9521484375	16.0572649986004\\
-0.95166015625	16.0586382006027\\
-0.951171875	16.0600254430019\\
-0.95068359375	16.0614267288211\\
-0.9501953125	16.0628420611147\\
-0.94970703125	16.0642714429682\\
-0.94921875	16.0657148774984\\
-0.94873046875	16.0671723678532\\
-0.9482421875	16.068643917212\\
-0.94775390625	16.0701295287851\\
-0.947265625	16.0716292058144\\
-0.94677734375	16.073142951573\\
-0.9462890625	16.0746707693654\\
-0.94580078125	16.0762126625274\\
-0.9453125	16.0777686344264\\
-0.94482421875	16.0793386884609\\
-0.9443359375	16.0809228280613\\
-0.94384765625	16.082521056689\\
-0.943359375	16.0841333778373\\
-0.94287109375	16.0857597950308\\
-0.9423828125	16.0874003118257\\
-0.94189453125	16.0890549318099\\
-0.94140625	16.0907236586029\\
-0.94091796875	16.0924064958556\\
-0.9404296875	16.094103447251\\
-0.93994140625	16.0958145165034\\
-0.939453125	16.0975397073591\\
-0.93896484375	16.099279023596\\
-0.9384765625	16.1010324690239\\
-0.93798828125	16.1028000474843\\
-0.9375	16.1045817628507\\
-0.93701171875	16.1063776190284\\
-0.9365234375	16.1081876199546\\
-0.93603515625	16.1100117695984\\
-0.935546875	16.1118500719609\\
-0.93505859375	16.1137025310753\\
-0.9345703125	16.1155691510068\\
-0.93408203125	16.1174499358525\\
-0.93359375	16.1193448897417\\
-0.93310546875	16.1212540168359\\
-0.9326171875	16.1231773213287\\
-0.93212890625	16.1251148074459\\
-0.931640625	16.1270664794456\\
-0.93115234375	16.1290323416179\\
-0.9306640625	16.1310123982855\\
-0.93017578125	16.1330066538033\\
-0.9296875	16.1350151125585\\
-0.92919921875	16.1370377789708\\
-0.9287109375	16.1390746574923\\
-0.92822265625	16.1411257526075\\
-0.927734375	16.1431910688335\\
-0.92724609375	16.1452706107198\\
-0.9267578125	16.1473643828486\\
-0.92626953125	16.1494723898347\\
-0.92578125	16.1515946363254\\
-0.92529296875	16.1537311270008\\
-0.9248046875	16.1558818665738\\
-0.92431640625	16.1580468597898\\
-0.923828125	16.1602261114271\\
-0.92333984375	16.162419626297\\
-0.9228515625	16.1646274092434\\
-0.92236328125	16.1668494651433\\
-0.921875	16.1690857989065\\
-0.92138671875	16.1713364154758\\
-0.9208984375	16.1736013198271\\
-0.92041015625	16.1758805169693\\
-0.919921875	16.1781740119445\\
-0.91943359375	16.1804818098276\\
-0.9189453125	16.1828039157272\\
-0.91845703125	16.1851403347846\\
-0.91796875	16.1874910721746\\
-0.91748046875	16.1898561331054\\
-0.9169921875	16.1922355228184\\
-0.91650390625	16.1946292465883\\
-0.916015625	16.1970373097233\\
-0.91552734375	16.1994597175652\\
-0.9150390625	16.2018964754892\\
-0.91455078125	16.2043475889038\\
-0.9140625	16.2068130632515\\
-0.91357421875	16.2092929040082\\
-0.9130859375	16.2117871166836\\
-0.91259765625	16.2142957068209\\
-0.912109375	16.2168186799973\\
-0.91162109375	16.2193560418237\\
-0.9111328125	16.2219077979448\\
-0.91064453125	16.2244739540393\\
-0.91015625	16.2270545158199\\
-0.90966796875	16.229649489033\\
-0.9091796875	16.2322588794594\\
-0.90869140625	16.2348826929137\\
-0.908203125	16.2375209352447\\
-0.90771484375	16.2401736123355\\
-0.9072265625	16.2428407301032\\
-0.90673828125	16.2455222944994\\
-0.90625	16.2482183115097\\
-0.90576171875	16.2509287871545\\
-0.9052734375	16.2536537274881\\
-0.90478515625	16.2563931385997\\
-0.904296875	16.2591470266127\\
-0.90380859375	16.2619153976853\\
-0.9033203125	16.26469825801\\
-0.90283203125	16.2674956138143\\
-0.90234375	16.2703074713602\\
-0.90185546875	16.2731338369443\\
-0.9013671875	16.2759747168984\\
-0.90087890625	16.2788301175888\\
-0.900390625	16.281700045417\\
-0.89990234375	16.2845845068191\\
-0.8994140625	16.2874835082665\\
-0.89892578125	16.2903970562656\\
-0.8984375	16.2933251573578\\
-0.89794921875	16.2962678181198\\
-0.8974609375	16.2992250451634\\
-0.89697265625	16.3021968451358\\
-0.896484375	16.3051832247194\\
-0.89599609375	16.3081841906321\\
-0.8955078125	16.311199749627\\
-0.89501953125	16.314229908493\\
-0.89453125	16.3172746740543\\
-0.89404296875	16.3203340531708\\
-0.8935546875	16.3234080527381\\
-0.89306640625	16.3264966796874\\
-0.892578125	16.3295999409857\\
-0.89208984375	16.3327178436358\\
-0.8916015625	16.3358503946764\\
-0.89111328125	16.3389976011822\\
-0.890625	16.3421594702638\\
-0.89013671875	16.3453360090678\\
-0.8896484375	16.3485272247771\\
-0.88916015625	16.3517331246105\\
-0.888671875	16.3549537158234\\
-0.88818359375	16.358189005707\\
-0.8876953125	16.3614390015893\\
-0.88720703125	16.3647037108343\\
-0.88671875	16.3679831408428\\
-0.88623046875	16.371277299052\\
-0.8857421875	16.3745861929355\\
-0.88525390625	16.3779098300039\\
-0.884765625	16.3812482178041\\
-0.88427734375	16.3846013639202\\
-0.8837890625	16.3879692759728\\
-0.88330078125	16.3913519616195\\
-0.8828125	16.3947494285548\\
-0.88232421875	16.3981616845104\\
-0.8818359375	16.4015887372548\\
-0.88134765625	16.4050305945939\\
-0.880859375	16.4084872643707\\
-0.88037109375	16.4119587544654\\
-0.8798828125	16.4154450727956\\
-0.87939453125	16.4189462273165\\
-0.87890625	16.4224622260203\\
-0.87841796875	16.4259930769373\\
-0.8779296875	16.4295387881349\\
-0.87744140625	16.4330993677185\\
-0.876953125	16.4366748238311\\
-0.87646484375	16.4402651646535\\
-0.8759765625	16.4438703984044\\
-0.87548828125	16.4474905333405\\
-0.875	16.4511255777562\\
-0.87451171875	16.4547755399844\\
-0.8740234375	16.458440428396\\
-0.87353515625	16.4621202513998\\
-0.873046875	16.4658150174434\\
-0.87255859375	16.4695247350124\\
-0.8720703125	16.4732494126309\\
-0.87158203125	16.4769890588615\\
-0.87109375	16.4807436823054\\
-0.87060546875	16.4845132916024\\
-0.8701171875	16.488297895431\\
-0.86962890625	16.4920975025086\\
-0.869140625	16.4959121215912\\
-0.86865234375	16.499741761474\\
-0.8681640625	16.5035864309911\\
-0.86767578125	16.5074461390156\\
-0.8671875	16.5113208944599\\
-0.86669921875	16.5152107062754\\
-0.8662109375	16.519115583453\\
-0.86572265625	16.5230355350229\\
-0.865234375	16.5269705700547\\
-0.86474609375	16.5309206976577\\
-0.8642578125	16.5348859269806\\
-0.86376953125	16.538866267212\\
-0.86328125	16.5428617275799\\
-0.86279296875	16.5468723173525\\
-0.8623046875	16.5508980458378\\
-0.86181640625	16.5549389223837\\
-0.861328125	16.5589949563783\\
-0.86083984375	16.5630661572497\\
-0.8603515625	16.5671525344664\\
-0.85986328125	16.5712540975372\\
-0.859375	16.575370856011\\
-0.85888671875	16.5795028194775\\
-0.8583984375	16.5836499975669\\
-0.85791015625	16.58781239995\\
-0.857421875	16.5919900363382\\
-0.85693359375	16.5961829164839\\
-0.8564453125	16.6003910501803\\
-0.85595703125	16.6046144472614\\
-0.85546875	16.6088531176025\\
-0.85498046875	16.61310707112\\
-0.8544921875	16.6173763177714\\
-0.85400390625	16.6216608675556\\
-0.853515625	16.6259607305128\\
-0.85302734375	16.6302759167248\\
-0.8525390625	16.6346064363149\\
-0.85205078125	16.638952299448\\
-0.8515625	16.6433135163307\\
-0.85107421875	16.6476900972118\\
-0.8505859375	16.6520820523814\\
-0.85009765625	16.6564893921722\\
-0.849609375	16.6609121269586\\
-0.84912109375	16.6653502671572\\
-0.8486328125	16.6698038232271\\
-0.84814453125	16.6742728056697\\
-0.84765625	16.6787572250286\\
-0.84716796875	16.6832570918903\\
-0.8466796875	16.6877724168837\\
-0.84619140625	16.6923032106804\\
-0.845703125	16.6968494839951\\
-0.84521484375	16.701411247585\\
-0.8447265625	16.7059885122507\\
-0.84423828125	16.7105812888357\\
-0.84375	16.7151895882267\\
-0.84326171875	16.7198134213537\\
-0.8427734375	16.7244527991901\\
-0.84228515625	16.7291077327529\\
-0.841796875	16.7337782331023\\
-0.84130859375	16.7384643113427\\
-0.8408203125	16.7431659786219\\
-0.84033203125	16.7478832461316\\
-0.83984375	16.7526161251076\\
-0.83935546875	16.7573646268297\\
-0.8388671875	16.7621287626218\\
-0.83837890625	16.7669085438523\\
-0.837890625	16.7717039819336\\
-0.83740234375	16.7765150883229\\
-0.8369140625	16.7813418745217\\
-0.83642578125	16.7861843520764\\
-0.8359375	16.7910425325781\\
-0.83544921875	16.7959164276626\\
-0.8349609375	16.8008060490108\\
-0.83447265625	16.8057114083488\\
-0.833984375	16.8106325174477\\
-0.83349609375	16.8155693881239\\
-0.8330078125	16.8205220322394\\
-0.83251953125	16.8254904617014\\
-0.83203125	16.8304746884628\\
-0.83154296875	16.8354747245225\\
-0.8310546875	16.8404905819247\\
-0.83056640625	16.8455222727599\\
-0.830078125	16.8505698091645\\
-0.82958984375	16.8556332033211\\
-0.8291015625	16.8607124674585\\
-0.82861328125	16.8658076138519\\
-0.828125	16.8709186548228\\
-0.82763671875	16.8760456027396\\
-0.8271484375	16.8811884700171\\
-0.82666015625	16.886347269117\\
-0.826171875	16.891522012548\\
-0.82568359375	16.8967127128657\\
-0.8251953125	16.901919382673\\
-0.82470703125	16.9071420346197\\
-0.82421875	16.9123806814034\\
-0.82373046875	16.917635335769\\
-0.8232421875	16.9229060105089\\
-0.82275390625	16.9281927184634\\
-0.822265625	16.9334954725204\\
-0.82177734375	16.9388142856161\\
-0.8212890625	16.9441491707343\\
-0.82080078125	16.9495001409075\\
-0.8203125	16.954867209216\\
-0.81982421875	16.960250388789\\
-0.8193359375	16.9656496928037\\
-0.81884765625	16.9710651344866\\
-0.818359375	16.9764967271123\\
-0.81787109375	16.9819444840049\\
-0.8173828125	16.9874084185372\\
-0.81689453125	16.9928885441312\\
-0.81640625	16.9983848742582\\
-0.81591796875	17.0038974224389\\
-0.8154296875	17.0094262022435\\
-0.81494140625	17.0149712272918\\
-0.814453125	17.0205325112536\\
-0.81396484375	17.0261100678482\\
-0.8134765625	17.0317039108453\\
-0.81298828125	17.0373140540646\\
-0.8125	17.042940511376\\
-0.81201171875	17.0485832966999\\
-0.8115234375	17.0542424240073\\
-0.81103515625	17.0599179073196\\
-0.810546875	17.0656097607094\\
-0.81005859375	17.0713179982999\\
-0.8095703125	17.0770426342655\\
-0.80908203125	17.0827836828317\\
-0.80859375	17.0885411582754\\
-0.80810546875	17.094315074925\\
-0.8076171875	17.1001054471605\\
-0.80712890625	17.1059122894134\\
-0.806640625	17.1117356161672\\
-0.80615234375	17.1175754419576\\
-0.8056640625	17.1234317813722\\
-0.80517578125	17.1293046490508\\
-0.8046875	17.1351940596859\\
-0.80419921875	17.1411000280222\\
-0.8037109375	17.1470225688574\\
-0.80322265625	17.1529616970419\\
-0.802734375	17.1589174274791\\
-0.80224609375	17.1648897751253\\
-0.8017578125	17.1708787549903\\
-0.80126953125	17.1768843821373\\
-0.80078125	17.1829066716829\\
-0.80029296875	17.1889456387974\\
-0.7998046875	17.1950012987049\\
-0.79931640625	17.2010736666836\\
-0.798828125	17.2071627580658\\
-0.79833984375	17.2132685882378\\
-0.7978515625	17.2193911726407\\
-0.79736328125	17.2255305267698\\
-0.796875	17.2316866661754\\
-0.79638671875	17.2378596064625\\
-0.7958984375	17.244049363291\\
-0.79541015625	17.2502559523762\\
-0.794921875	17.2564793894885\\
-0.79443359375	17.2627196904539\\
-0.7939453125	17.2689768711539\\
-0.79345703125	17.2752509475259\\
-0.79296875	17.281541935563\\
-0.79248046875	17.2878498513146\\
-0.7919921875	17.2941747108862\\
-0.79150390625	17.3005165304398\\
-0.791015625	17.3068753261937\\
-0.79052734375	17.3132511144232\\
-0.7900390625	17.3196439114602\\
-0.78955078125	17.3260537336938\\
-0.7890625	17.3324805975702\\
-0.78857421875	17.338924519593\\
-0.7880859375	17.3453855163231\\
-0.78759765625	17.3518636043794\\
-0.787109375	17.3583588004383\\
-0.78662109375	17.3648711212344\\
-0.7861328125	17.3714005835603\\
-0.78564453125	17.3779472042671\\
-0.78515625	17.3845110002642\\
-0.78466796875	17.3910919885198\\
-0.7841796875	17.3976901860607\\
-0.78369140625	17.404305609973\\
-0.783203125	17.4109382774017\\
-0.78271484375	17.4175882055513\\
-0.7822265625	17.4242554116857\\
-0.78173828125	17.4309399131284\\
-0.78125	17.4376417272629\\
-0.78076171875	17.4443608715327\\
-0.7802734375	17.4510973634413\\
-0.77978515625	17.4578512205528\\
-0.779296875	17.4646224604917\\
-0.77880859375	17.4714111009433\\
-0.7783203125	17.4782171596536\\
-0.77783203125	17.48504065443\\
-0.77734375	17.4918816031408\\
-0.77685546875	17.498740023716\\
-0.7763671875	17.505615934147\\
-0.77587890625	17.5125093524873\\
-0.775390625	17.5194202968519\\
-0.77490234375	17.5263487854185\\
-0.7744140625	17.5332948364267\\
-0.77392578125	17.5402584681788\\
-0.7734375	17.5472396990398\\
-0.77294921875	17.5542385474377\\
-0.7724609375	17.5612550318634\\
-0.77197265625	17.5682891708713\\
-0.771484375	17.5753409830789\\
-0.77099609375	17.5824104871678\\
-0.7705078125	17.5894977018831\\
-0.77001953125	17.5966026460341\\
-0.76953125	17.6037253384944\\
-0.76904296875	17.6108657982017\\
-0.7685546875	17.6180240441588\\
-0.76806640625	17.6252000954328\\
-0.767578125	17.6323939711563\\
-0.76708984375	17.6396056905267\\
-0.7666015625	17.646835272807\\
-0.76611328125	17.6540827373258\\
-0.765625	17.6613481034776\\
-0.76513671875	17.6686313907226\\
-0.7646484375	17.6759326185875\\
-0.76416015625	17.6832518066654\\
-0.763671875	17.6905889746158\\
-0.76318359375	17.6979441421652\\
-0.7626953125	17.7053173291072\\
-0.76220703125	17.7127085553023\\
-0.76171875	17.7201178406787\\
-0.76123046875	17.7275452052324\\
-0.7607421875	17.7349906690269\\
-0.76025390625	17.742454252194\\
-0.759765625	17.7499359749336\\
-0.75927734375	17.7574358575145\\
-0.7587890625	17.7649539202737\\
-0.75830078125	17.7724901836174\\
-0.7578125	17.7800446680209\\
-0.75732421875	17.7876173940289\\
-0.7568359375	17.7952083822557\\
-0.75634765625	17.8028176533852\\
-0.755859375	17.8104452281714\\
-0.75537109375	17.8180911274388\\
-0.7548828125	17.825755372082\\
-0.75439453125	17.8334379830664\\
-0.75390625	17.8411389814284\\
-0.75341796875	17.8488583882754\\
-0.7529296875	17.8565962247863\\
-0.75244140625	17.8643525122114\\
-0.751953125	17.8721272718729\\
-0.75146484375	17.8799205251652\\
-0.7509765625	17.8877322935548\\
-0.75048828125	17.8955625985807\\
-0.75	17.9034114618546\\
-0.74951171875	17.9112789050614\\
-0.7490234375	17.919164949959\\
-0.74853515625	17.9270696183788\\
-0.748046875	17.9349929322259\\
-0.74755859375	17.9429349134793\\
-0.7470703125	17.9508955841922\\
-0.74658203125	17.9588749664921\\
-0.74609375	17.9668730825813\\
-0.74560546875	17.9748899547368\\
-0.7451171875	17.9829256053109\\
-0.74462890625	17.9909800567313\\
-0.744140625	17.9990533315011\\
-0.74365234375	18.0071454521994\\
-0.7431640625	18.0152564414815\\
-0.74267578125	18.0233863220789\\
-0.7421875	18.0315351167999\\
-0.74169921875	18.0397028485295\\
-0.7412109375	18.04788954023\\
-0.74072265625	18.056095214941\\
-0.740234375	18.0643198957797\\
-0.73974609375	18.0725636059413\\
-0.7392578125	18.080826368699\\
-0.73876953125	18.0891082074045\\
-0.73828125	18.0974091454883\\
-0.73779296875	18.1057292064597\\
-0.7373046875	18.1140684139071\\
-0.73681640625	18.1224267914986\\
-0.736328125	18.130804362982\\
-0.73583984375	18.1392011521849\\
-0.7353515625	18.1476171830155\\
-0.73486328125	18.1560524794621\\
-0.734375	18.1645070655943\\
-0.73388671875	18.1729809655624\\
-0.7333984375	18.1814742035983\\
-0.73291015625	18.1899868040154\\
-0.732421875	18.198518791209\\
-0.73193359375	18.2070701896569\\
-0.7314453125	18.2156410239189\\
-0.73095703125	18.2242313186379\\
-0.73046875	18.2328410985396\\
-0.72998046875	18.2414703884333\\
-0.7294921875	18.2501192132116\\
-0.72900390625	18.2587875978512\\
-0.728515625	18.2674755674129\\
-0.72802734375	18.276183147042\\
-0.7275390625	18.2849103619684\\
-0.72705078125	18.2936572375074\\
-0.7265625	18.3024237990593\\
-0.72607421875	18.3112100721101\\
-0.7255859375	18.320016082232\\
-0.72509765625	18.3288418550831\\
-0.724609375	18.3376874164082\\
-0.72412109375	18.3465527920389\\
-0.7236328125	18.3554380078939\\
-0.72314453125	18.3643430899795\\
-0.72265625	18.3732680643895\\
-0.72216796875	18.382212957306\\
-0.7216796875	18.3911777949992\\
-0.72119140625	18.4001626038282\\
-0.720703125	18.4091674102409\\
-0.72021484375	18.4181922407747\\
-0.7197265625	18.4272371220565\\
-0.71923828125	18.436302080803\\
-0.71875	18.4453871438213\\
-0.71826171875	18.4544923380091\\
-0.7177734375	18.4636176903548\\
-0.71728515625	18.4727632279382\\
-0.716796875	18.4819289779305\\
-0.71630859375	18.4911149675948\\
-0.7158203125	18.5003212242863\\
-0.71533203125	18.5095477754529\\
-0.71484375	18.5187946486351\\
-0.71435546875	18.5280618714668\\
-0.7138671875	18.5373494716753\\
-0.71337890625	18.5466574770817\\
-0.712890625	18.5559859156014\\
-0.71240234375	18.5653348152441\\
-0.7119140625	18.5747042041146\\
-0.71142578125	18.5840941104128\\
-0.7109375	18.5935045624341\\
-0.71044921875	18.6029355885699\\
-0.7099609375	18.6123872173077\\
-0.70947265625	18.6218594772317\\
-0.708984375	18.6313523970229\\
-0.70849609375	18.6408660054598\\
-0.7080078125	18.6504003314184\\
-0.70751953125	18.6599554038726\\
-0.70703125	18.669531251895\\
-0.70654296875	18.6791279046565\\
-0.7060546875	18.6887453914273\\
-0.70556640625	18.6983837415771\\
-0.705078125	18.7080429845753\\
-0.70458984375	18.7177231499914\\
-0.7041015625	18.7274242674957\\
-0.70361328125	18.7371463668592\\
-0.703125	18.7468894779541\\
-0.70263671875	18.7566536307546\\
-0.7021484375	18.7664388553366\\
-0.70166015625	18.7762451818786\\
-0.701171875	18.7860726406619\\
-0.70068359375	18.7959212620708\\
-0.7001953125	18.8057910765934\\
-0.69970703125	18.8156821148215\\
-0.69921875	18.8255944074515\\
-0.69873046875	18.8355279852843\\
-0.6982421875	18.8454828792261\\
-0.69775390625	18.8554591202885\\
-0.697265625	18.865456739589\\
-0.69677734375	18.8754757683514\\
-0.6962890625	18.8855162379063\\
-0.69580078125	18.8955781796913\\
-0.6953125	18.9056616252515\\
-0.69482421875	18.9157666062402\\
-0.6943359375	18.9258931544186\\
-0.69384765625	18.9360413016569\\
-0.693359375	18.9462110799344\\
-0.69287109375	18.95640252134\\
-0.6923828125	18.9666156580724\\
-0.69189453125	18.9768505224409\\
-0.69140625	18.9871071468655\\
-0.69091796875	18.9973855638777\\
-0.6904296875	19.0076858061202\\
-0.68994140625	19.0180079063482\\
-0.689453125	19.0283518974292\\
-0.68896484375	19.0387178123439\\
-0.6884765625	19.0491056841861\\
-0.68798828125	19.0595155461636\\
-0.6875	19.0699474315985\\
-0.68701171875	19.0804013739274\\
-0.6865234375	19.0908774067023\\
-0.68603515625	19.1013755635908\\
-0.685546875	19.1118958783762\\
-0.68505859375	19.1224383849587\\
-0.6845703125	19.1330031173553\\
-0.68408203125	19.1435901097003\\
-0.68359375	19.1541993962461\\
-0.68310546875	19.1648310113631\\
-0.6826171875	19.1754849895408\\
-0.68212890625	19.1861613653877\\
-0.681640625	19.1968601736322\\
-0.68115234375	19.2075814491228\\
-0.6806640625	19.2183252268287\\
-0.68017578125	19.2290915418401\\
-0.6796875	19.2398804293691\\
-0.67919921875	19.2506919247495\\
-0.6787109375	19.2615260634379\\
-0.67822265625	19.2723828810141\\
-0.677734375	19.2832624131811\\
-0.67724609375	19.2941646957661\\
-0.6767578125	19.3050897647209\\
-0.67626953125	19.3160376561222\\
-0.67578125	19.3270084061723\\
-0.67529296875	19.3380020511995\\
-0.6748046875	19.3490186276585\\
-0.67431640625	19.3600581721313\\
-0.673828125	19.371120721327\\
-0.67333984375	19.3822063120832\\
-0.6728515625	19.3933149813658\\
-0.67236328125	19.4044467662699\\
-0.671875	19.41560170402\\
-0.67138671875	19.426779831971\\
-0.6708984375	19.4379811876082\\
-0.67041015625	19.4492058085481\\
-0.669921875	19.4604537325391\\
-0.66943359375	19.4717249974615\\
-0.6689453125	19.4830196413288\\
-0.66845703125	19.4943377022873\\
-0.66796875	19.5056792186176\\
-0.66748046875	19.5170442287344\\
-0.6669921875	19.5284327711875\\
-0.66650390625	19.5398448846621\\
-0.666015625	19.5512806079795\\
-0.66552734375	19.5627399800975\\
-0.6650390625	19.5742230401112\\
-0.66455078125	19.5857298272535\\
-0.6640625	19.5972603808952\\
-0.66357421875	19.6088147405464\\
-0.6630859375	19.6203929458563\\
-0.66259765625	19.6319950366145\\
-0.662109375	19.6436210527506\\
-0.66162109375	19.6552710343359\\
-0.6611328125	19.6669450215833\\
-0.66064453125	19.6786430548478\\
-0.66015625	19.6903651746276\\
-0.65966796875	19.7021114215644\\
-0.6591796875	19.7138818364438\\
-0.65869140625	19.7256764601963\\
-0.658203125	19.7374953338978\\
-0.65771484375	19.74933849877\\
-0.6572265625	19.761205996181\\
-0.65673828125	19.7730978676464\\
-0.65625	19.7850141548292\\
-0.65576171875	19.7969548995412\\
-0.6552734375	19.8089201437428\\
-0.65478515625	19.8209099295444\\
-0.654296875	19.8329242992064\\
-0.65380859375	19.8449632951402\\
-0.6533203125	19.8570269599088\\
-0.65283203125	19.8691153362274\\
-0.65234375	19.881228466964\\
-0.65185546875	19.8933663951399\\
-0.6513671875	19.9055291639309\\
-0.65087890625	19.9177168166673\\
-0.650390625	19.929929396835\\
-0.64990234375	19.9421669480758\\
-0.6494140625	19.9544295141886\\
-0.64892578125	19.9667171391295\\
-0.6484375	19.9790298670129\\
-0.64794921875	19.9913677421118\\
-0.6474609375	20.0037308088588\\
-0.64697265625	20.0161191118467\\
-0.646484375	20.0285326958291\\
-0.64599609375	20.0409716057213\\
-0.6455078125	20.0534358866005\\
-0.64501953125	20.0659255837072\\
-0.64453125	20.0784407424454\\
-0.64404296875	20.0909814083836\\
-0.6435546875	20.1035476272551\\
-0.64306640625	20.1161394449593\\
-0.642578125	20.128756907562\\
-0.64208984375	20.1414000612961\\
-0.6416015625	20.1540689525628\\
-0.64111328125	20.1667636279316\\
-0.640625	20.1794841341418\\
-0.64013671875	20.1922305181027\\
-0.6396484375	20.2050028268945\\
-0.63916015625	20.2178011077693\\
-0.638671875	20.2306254081514\\
-0.63818359375	20.2434757756383\\
-0.6376953125	20.2563522580017\\
-0.63720703125	20.2692549031878\\
-0.63671875	20.2821837593184\\
-0.63623046875	20.2951388746915\\
-0.6357421875	20.3081202977822\\
-0.63525390625	20.3211280772436\\
-0.634765625	20.334162261907\\
-0.63427734375	20.3472229007836\\
-0.6337890625	20.3603100430647\\
-0.63330078125	20.3734237381223\\
-0.6328125	20.3865640355108\\
-0.63232421875	20.3997309849667\\
-0.6318359375	20.4129246364103\\
-0.63134765625	20.4261450399462\\
-0.630859375	20.4393922458639\\
-0.63037109375	20.4526663046391\\
-0.6298828125	20.4659672669341\\
-0.62939453125	20.479295183599\\
-0.62890625	20.4926501056721\\
-0.62841796875	20.5060320843813\\
-0.6279296875	20.5194411711446\\
-0.62744140625	20.532877417571\\
-0.626953125	20.5463408754615\\
-0.62646484375	20.5598315968098\\
-0.6259765625	20.5733496338033\\
-0.62548828125	20.586895038824\\
-0.625	20.6004678644492\\
-0.62451171875	20.6140681634526\\
-0.6240234375	20.6276959888052\\
-0.62353515625	20.6413513936759\\
-0.623046875	20.6550344314329\\
-0.62255859375	20.6687451556442\\
-0.6220703125	20.6824836200786\\
-0.62158203125	20.6962498787069\\
-0.62109375	20.7100439857024\\
-0.62060546875	20.7238659954422\\
-0.6201171875	20.737715962508\\
-0.61962890625	20.7515939416869\\
-0.619140625	20.7654999879726\\
-0.61865234375	20.7794341565663\\
-0.6181640625	20.7933965028776\\
-0.61767578125	20.8073870825253\\
-0.6171875	20.8214059513388\\
-0.61669921875	20.8354531653588\\
-0.6162109375	20.8495287808382\\
-0.61572265625	20.8636328542433\\
-0.615234375	20.8777654422548\\
-0.61474609375	20.8919266017687\\
-0.6142578125	20.9061163898971\\
-0.61376953125	20.9203348639699\\
-0.61328125	20.934582081535\\
-0.61279296875	20.9488581003598\\
-0.6123046875	20.9631629784322\\
-0.61181640625	20.9774967739617\\
-0.611328125	20.99185954538\\
-0.61083984375	21.0062513513428\\
-0.6103515625	21.0206722507301\\
-0.60986328125	21.0351223026478\\
-0.609375	21.0496015664285\\
-0.60888671875	21.0641101016328\\
-0.6083984375	21.0786479680501\\
-0.60791015625	21.0932152257001\\
-0.607421875	21.1078119348332\\
-0.60693359375	21.1224381559326\\
-0.6064453125	21.1370939497145\\
-0.60595703125	21.1517793771297\\
-0.60546875	21.1664944993647\\
-0.60498046875	21.1812393778426\\
-0.6044921875	21.1960140742246\\
-0.60400390625	21.2108186504108\\
-0.603515625	21.2256531685415\\
-0.60302734375	21.2405176909984\\
-0.6025390625	21.2554122804059\\
-0.60205078125	21.2703369996318\\
-0.6015625	21.2852919117889\\
-0.60107421875	21.3002770802364\\
-0.6005859375	21.3152925685803\\
-0.60009765625	21.3303384406755\\
-0.599609375	21.3454147606263\\
-0.59912109375	21.3605215927882\\
-0.5986328125	21.3756590017687\\
-0.59814453125	21.3908270524288\\
-0.59765625	21.406025809884\\
-0.59716796875	21.4212553395058\\
-0.5966796875	21.436515706923\\
-0.59619140625	21.4518069780225\\
-0.595703125	21.4671292189512\\
-0.59521484375	21.4824824961168\\
-0.5947265625	21.4978668761895\\
-0.59423828125	21.5132824261028\\
-0.59375	21.5287292130553\\
-0.59326171875	21.5442073045118\\
-0.5927734375	21.5597167682045\\
-0.59228515625	21.5752576721346\\
-0.591796875	21.5908300845736\\
-0.59130859375	21.6064340740643\\
-0.5908203125	21.6220697094225\\
-0.59033203125	21.6377370597386\\
-0.58984375	21.6534361943783\\
-0.58935546875	21.6691671829846\\
-0.5888671875	21.6849300954787\\
-0.58837890625	21.700725002062\\
-0.587890625	21.7165519732168\\
-0.58740234375	21.7324110797085\\
-0.5869140625	21.7483023925862\\
-0.58642578125	21.7642259831849\\
-0.5859375	21.7801819231265\\
-0.58544921875	21.7961702843213\\
-0.5849609375	21.8121911389696\\
-0.58447265625	21.8282445595633\\
-0.583984375	21.8443306188868\\
-0.58349609375	21.8604493900194\\
-0.5830078125	21.8766009463359\\
-0.58251953125	21.8927853615088\\
-0.58203125	21.9090027095095\\
-0.58154296875	21.9252530646097\\
-0.5810546875	21.9415365013835\\
-0.58056640625	21.9578530947083\\
-0.580078125	21.9742029197668\\
-0.57958984375	21.9905860520484\\
-0.5791015625	22.0070025673508\\
-0.57861328125	22.0234525417819\\
-0.578125	22.0399360517607\\
-0.57763671875	22.0564531740197\\
-0.5771484375	22.0730039856061\\
-0.57666015625	22.0895885638835\\
-0.576171875	22.1062069865336\\
-0.57568359375	22.1228593315578\\
-0.5751953125	22.139545677279\\
-0.57470703125	22.1562661023431\\
-0.57421875	22.1730206857211\\
-0.57373046875	22.18980950671\\
-0.5732421875	22.2066326449356\\
-0.57275390625	22.2234901803531\\
-0.572265625	22.2403821932498\\
-0.57177734375	22.2573087642464\\
-0.5712890625	22.2742699742986\\
-0.57080078125	22.2912659046994\\
-0.5703125	22.3082966370803\\
-0.56982421875	22.3253622534136\\
-0.5693359375	22.3424628360139\\
-0.56884765625	22.3595984675401\\
-0.568359375	22.376769230997\\
-0.56787109375	22.3939752097375\\
-0.5673828125	22.4112164874642\\
-0.56689453125	22.4284931482313\\
-0.56640625	22.4458052764467\\
-0.56591796875	22.4631529568736\\
-0.5654296875	22.4805362746326\\
-0.56494140625	22.4979553152037\\
-0.564453125	22.5154101644278\\
-0.56396484375	22.5329009085093\\
-0.5634765625	22.5504276340176\\
-0.56298828125	22.5679904278893\\
-0.5625	22.5855893774301\\
-0.56201171875	22.6032245703168\\
-0.5615234375	22.6208960945993\\
-0.56103515625	22.6386040387028\\
-0.560546875	22.6563484914297\\
-0.56005859375	22.6741295419616\\
-0.5595703125	22.6919472798616\\
-0.55908203125	22.7098017950762\\
-0.55859375	22.7276931779375\\
-0.55810546875	22.7456215191654\\
-0.5576171875	22.7635869098694\\
-0.55712890625	22.7815894415512\\
-0.556640625	22.7996292061065\\
-0.55615234375	22.8177062958275\\
-0.5556640625	22.8358208034047\\
-0.55517578125	22.8539728219295\\
-0.5546875	22.8721624448964\\
-0.55419921875	22.8903897662048\\
-0.5537109375	22.9086548801618\\
-0.55322265625	22.9269578814841\\
-0.552734375	22.9452988653006\\
-0.55224609375	22.9636779271545\\
-0.5517578125	22.9820951630055\\
-0.55126953125	23.0005506692325\\
-0.55078125	23.0190445426358\\
-0.55029296875	23.0375768804392\\
-0.5498046875	23.0561477802929\\
-0.54931640625	23.0747573402754\\
-0.548828125	23.0934056588963\\
-0.54833984375	23.1120928350987\\
-0.5478515625	23.1308189682613\\
-0.54736328125	23.1495841582014\\
-0.546875	23.168388505177\\
-0.54638671875	23.1872321098894\\
-0.5458984375	23.2061150734858\\
-0.54541015625	23.225037497562\\
-0.544921875	23.2439994841647\\
-0.54443359375	23.263001135794\\
-0.5439453125	23.2820425554064\\
-0.54345703125	23.3011238464171\\
-0.54296875	23.3202451127029\\
-0.54248046875	23.3394064586045\\
-0.5419921875	23.3586079889295\\
-0.54150390625	23.3778498089551\\
-0.541015625	23.3971320244305\\
-0.54052734375	23.41645474158\\
-0.5400390625	23.4358180671057\\
-0.53955078125	23.4552221081899\\
-0.5390625	23.4746669724984\\
-0.53857421875	23.4941527681832\\
-0.5380859375	23.5136796038852\\
-0.53759765625	23.5332475887369\\
-0.537109375	23.5528568323658\\
-0.53662109375	23.5725074448968\\
-0.5361328125	23.5921995369557\\
-0.53564453125	23.6119332196713\\
-0.53515625	23.6317086046792\\
-0.53466796875	23.6515258041243\\
-0.5341796875	23.6713849306642\\
-0.53369140625	23.6912860974717\\
-0.533203125	23.7112294182383\\
-0.53271484375	23.7312150071772\\
-0.5322265625	23.7512429790263\\
-0.53173828125	23.7713134490515\\
-0.53125	23.7914265330496\\
-0.53076171875	23.8115823473517\\
-0.5302734375	23.8317810088262\\
-0.52978515625	23.8520226348824\\
-0.529296875	23.8723073434732\\
-0.52880859375	23.8926352530989\\
-0.5283203125	23.9130064828101\\
-0.52783203125	23.9334211522114\\
-0.52734375	23.9538793814643\\
-0.52685546875	23.974381291291\\
-0.5263671875	23.9949270029775\\
-0.52587890625	24.0155166383772\\
-0.525390625	24.0361503199144\\
-0.52490234375	24.0568281705874\\
-0.5244140625	24.0775503139726\\
-0.52392578125	24.0983168742275\\
-0.5234375	24.1191279760946\\
-0.52294921875	24.1399837449048\\
-0.5224609375	24.1608843065811\\
-0.52197265625	24.1818297876422\\
-0.521484375	24.2028203152062\\
-0.52099609375	24.2238560169943\\
-0.5205078125	24.2449370213345\\
-0.52001953125	24.2660634571655\\
-0.51953125	24.2872354540402\\
-0.51904296875	24.3084531421299\\
-0.5185546875	24.3297166522276\\
-0.51806640625	24.3510261157525\\
-0.517578125	24.3723816647535\\
-0.51708984375	24.3937834319132\\
-0.5166015625	24.4152315505522\\
-0.51611328125	24.4367261546324\\
-0.515625	24.4582673787617\\
-0.51513671875	24.4798553581978\\
-0.5146484375	24.5014902288524\\
-0.51416015625	24.5231721272949\\
-0.513671875	24.5449011907573\\
-0.51318359375	24.5666775571378\\
-0.5126953125	24.5885013650051\\
-0.51220703125	24.6103727536031\\
-0.51171875	24.6322918628546\\
-0.51123046875	24.654258833366\\
-0.5107421875	24.6762738064317\\
-0.51025390625	24.698336924038\\
-0.509765625	24.7204483288684\\
-0.50927734375	24.7426081643071\\
-0.5087890625	24.7648165744444\\
-0.50830078125	24.7870737040804\\
-0.5078125	24.8093796987304\\
-0.50732421875	24.8317347046289\\
-0.5068359375	24.8541388687345\\
-0.50634765625	24.8765923387345\\
-0.505859375	24.8990952630499\\
-0.50537109375	24.9216477908398\\
-0.5048828125	24.9442500720063\\
-0.50439453125	24.9669022571996\\
-0.50390625	24.9896044978226\\
-0.50341796875	25.012356946036\\
-0.5029296875	25.0351597547631\\
-0.50244140625	25.058013077695\\
-0.501953125	25.0809170692955\\
-0.50146484375	25.1038718848062\\
-0.5009765625	25.1268776802518\\
-0.50048828125	25.1499346124451\\
-0.5	25.1730428389923\\
-0.49951171875	25.1962025182981\\
-0.4990234375	25.2194138095713\\
-0.49853515625	25.24267687283\\
-0.498046875	25.265991868907\\
-0.49755859375	25.2893589594549\\
-0.4970703125	25.3127783069525\\
-0.49658203125	25.3362500747092\\
-0.49609375	25.3597744268715\\
-0.49560546875	25.383351528428\\
-0.4951171875	25.4069815452155\\
-0.49462890625	25.4306646439245\\
-0.494140625	25.454400992105\\
-0.49365234375	25.4781907581723\\
-0.4931640625	25.5020341114129\\
-0.49267578125	25.5259312219906\\
-0.4921875	25.5498822609521\\
-0.49169921875	25.5738874002331\\
-0.4912109375	25.5979468126649\\
-0.49072265625	25.6220606719797\\
-0.490234375	25.6462291528174\\
-0.48974609375	25.6704524307315\\
-0.4892578125	25.6947306821955\\
-0.48876953125	25.7190640846093\\
-0.48828125	25.7434528163055\\
-0.48779296875	25.7678970565558\\
-0.4873046875	25.7923969855776\\
-0.48681640625	25.8169527845406\\
-0.486328125	25.8415646355733\\
-0.48583984375	25.8662327217696\\
-0.4853515625	25.8909572271959\\
-0.48486328125	25.9157383368975\\
-0.484375	25.9405762369054\\
-0.48388671875	25.9654711142438\\
-0.4833984375	25.9904231569365\\
-0.48291015625	26.015432554014\\
-0.482421875	26.0404994955208\\
-0.48193359375	26.0656241725225\\
-0.4814453125	26.0908067771129\\
-0.48095703125	26.1160475024215\\
-0.48046875	26.1413465426207\\
-0.47998046875	26.166704092933\\
-0.4794921875	26.192120349639\\
-0.47900390625	26.2175955100847\\
-0.478515625	26.2431297726887\\
-0.47802734375	26.2687233369508\\
-0.4775390625	26.2943764034588\\
-0.47705078125	26.3200891738969\\
-0.4765625	26.3458618510535\\
-0.47607421875	26.3716946388291\\
-0.4755859375	26.3975877422442\\
-0.47509765625	26.4235413674477\\
-0.474609375	26.449555721725\\
-0.47412109375	26.475631013506\\
-0.4736328125	26.5017674523737\\
-0.47314453125	26.5279652490727\\
-0.47265625	26.5542246155173\\
-0.47216796875	26.5805457648005\\
-0.4716796875	26.6069289112023\\
-0.47119140625	26.6333742701987\\
-0.470703125	26.6598820584701\\
-0.47021484375	26.6864524939107\\
-0.4697265625	26.7130857956372\\
-0.46923828125	26.7397821839978\\
-0.46875	26.7665418805813\\
-0.46826171875	26.7933651082267\\
-0.4677734375	26.8202520910319\\
-0.46728515625	26.8472030543636\\
-0.466796875	26.8742182248667\\
-0.46630859375	26.9012978304734\\
-0.4658203125	26.9284421004137\\
-0.46533203125	26.9556512652243\\
-0.46484375	26.9829255567588\\
-0.46435546875	27.010265208198\\
-0.4638671875	27.0376704540593\\
-0.46337890625	27.0651415302071\\
-0.462890625	27.092678673863\\
-0.46240234375	27.1202821236163\\
-0.4619140625	27.1479521194341\\
-0.46142578125	27.1756889026719\\
-0.4609375	27.2034927160845\\
-0.46044921875	27.2313638038363\\
-0.4599609375	27.2593024115124\\
-0.45947265625	27.2873087861293\\
-0.458984375	27.3153831761462\\
-0.45849609375	27.3435258314759\\
-0.4580078125	27.3717370034962\\
-0.45751953125	27.4000169450611\\
-0.45703125	27.4283659105125\\
-0.45654296875	27.4567841556915\\
-0.4560546875	27.4852719379501\\
-0.45556640625	27.5138295161634\\
-0.455078125	27.5424571507411\\
-0.45458984375	27.5711551036394\\
-0.4541015625	27.5999236383737\\
-0.45361328125	27.6287630200306\\
-0.453125	27.65767351528\\
-0.45263671875	27.6866553923882\\
-0.4521484375	27.71570892123\\
-0.45166015625	27.7448343733016\\
-0.451171875	27.774032021734\\
-0.45068359375	27.803302141305\\
-0.4501953125	27.8326450084535\\
-0.44970703125	27.8620609012917\\
-0.44921875	27.8915500996196\\
-0.44873046875	27.9211128849374\\
-0.4482421875	27.9507495404602\\
-0.44775390625	27.9804603511311\\
-0.447265625	28.0102456036357\\
-0.44677734375	28.0401055864156\\
-0.4462890625	28.0700405896835\\
-0.44580078125	28.1000509054366\\
-0.4453125	28.1301368274719\\
-0.44482421875	28.1602986514004\\
-0.4443359375	28.1905366746625\\
-0.44384765625	28.2208511965422\\
-0.443359375	28.251242518183\\
-0.44287109375	28.2817109426027\\
-0.4423828125	28.3122567747091\\
-0.44189453125	28.3428803213155\\
-0.44140625	28.3735818911566\\
-0.44091796875	28.4043617949042\\
-0.4404296875	28.4352203451837\\
-0.43994140625	28.4661578565898\\
-0.439453125	28.4971746457036\\
-0.43896484375	28.5282710311088\\
-0.4384765625	28.5594473334085\\
-0.43798828125	28.5907038752422\\
-0.4375	28.6220409813033\\
-0.43701171875	28.6534589783558\\
-0.4365234375	28.6849581952523\\
-0.43603515625	28.7165389629514\\
-0.435546875	28.7482016145357\\
-0.43505859375	28.7799464852301\\
-0.4345703125	28.8117739124196\\
-0.43408203125	28.8436842356681\\
-0.43359375	28.8756777967368\\
-0.43310546875	28.9077549396035\\
-0.4326171875	28.939916010481\\
-0.43212890625	28.9721613578371\\
-0.431640625	29.0044913324134\\
-0.43115234375	29.0369062872456\\
-0.4306640625	29.0694065776827\\
-0.43017578125	29.1019925614078\\
-0.4296875	29.134664598458\\
-0.42919921875	29.1674230512451\\
-0.4287109375	29.2002682845765\\
-0.42822265625	29.2332006656757\\
-0.427734375	29.2662205642043\\
-0.42724609375	29.2993283522828\\
-0.4267578125	29.3325244045128\\
-0.42626953125	29.3658090979986\\
-0.42578125	29.3991828123694\\
-0.42529296875	29.4326459298021\\
-0.4248046875	29.4661988350435\\
-0.42431640625	29.4998419154337\\
-0.423828125	29.533575560929\\
-0.42333984375	29.5674001641253\\
-0.4228515625	29.601316120282\\
-0.42236328125	29.635323827346\\
-0.421875	29.669423685976\\
-0.42138671875	29.7036160995666\\
-0.4208984375	29.7379014742737\\
-0.42041015625	29.7722802190392\\
-0.419921875	29.8067527456164\\
-0.41943359375	29.841319468596\\
-0.4189453125	29.8759808054315\\
-0.41845703125	29.9107371764662\\
-0.41796875	29.9455890049588\\
-0.41748046875	29.9805367171112\\
-0.4169921875	30.0155807420949\\
-0.41650390625	30.0507215120793\\
-0.416015625	30.0859594622586\\
-0.41552734375	30.1212950308807\\
-0.4150390625	30.156728659275\\
-0.41455078125	30.1922607918819\\
-0.4140625	30.2278918762812\\
-0.41357421875	30.263622363222\\
-0.4130859375	30.2994527066527\\
-0.41259765625	30.3353833637504\\
-0.412109375	30.3714147949522\\
-0.41162109375	30.4075474639857\\
-0.4111328125	30.4437818379003\\
-0.41064453125	30.4801183870989\\
-0.41015625	30.5165575853697\\
-0.40966796875	30.5530999099188\\
-0.4091796875	30.5897458414029\\
-0.40869140625	30.6264958639623\\
-0.408203125	30.6633504652546\\
-0.40771484375	30.7003101364888\\
-0.4072265625	30.7373753724592\\
-0.40673828125	30.7745466715809\\
-0.40625	30.8118245359241\\
-0.40576171875	30.8492094712508\\
-0.4052734375	30.8867019870498\\
-0.40478515625	30.924302596574\\
-0.404296875	30.962011816877\\
-0.40380859375	30.9998301688507\\
-0.4033203125	31.0377581772629\\
-0.40283203125	31.075796370796\\
-0.40234375	31.1139452820855\\
-0.40185546875	31.1522054477595\\
-0.4013671875	31.1905774084785\\
-0.40087890625	31.2290617089754\\
-0.400390625	31.2676588980967\\
-0.39990234375	31.3063695288437\\
-0.3994140625	31.3451941584138\\
-0.39892578125	31.3841333482438\\
-0.3984375	31.4231876640519\\
-0.39794921875	31.4623576758816\\
-0.3974609375	31.5016439581455\\
-0.39697265625	31.5410470896701\\
-0.396484375	31.5805676537403\\
-0.39599609375	31.620206238146\\
-0.3955078125	31.6599634352273\\
-0.39501953125	31.6998398419222\\
-0.39453125	31.7398360598139\\
-0.39404296875	31.7799526951784\\
-0.3935546875	31.8201903590339\\
-0.39306640625	31.8605496671895\\
-0.392578125	31.9010312402955\\
-0.39208984375	31.9416357038942\\
-0.3916015625	31.9823636884708\\
-0.39111328125	32.0232158295055\\
-0.390625	32.0641927675263\\
-0.39013671875	32.1052951481622\\
-0.3896484375	32.1465236221972\\
-0.38916015625	32.1878788456253\\
-0.388671875	32.2293614797057\\
-0.38818359375	32.2709721910192\\
-0.3876953125	32.3127116515251\\
-0.38720703125	32.354580538619\\
-0.38671875	32.3965795351914\\
-0.38623046875	32.4387093296869\\
-0.3857421875	32.4809706161642\\
-0.38525390625	32.5233640943573\\
-0.384765625	32.565890469737\\
-0.38427734375	32.6085504535738\\
-0.3837890625	32.6513447630009\\
-0.38330078125	32.6942741210789\\
-0.3828125	32.7373392568609\\
-0.38232421875	32.7805409054585\\
-0.3818359375	32.8238798081089\\
-0.38134765625	32.8673567122427\\
-0.380859375	32.9109723715531\\
-0.38037109375	32.9547275460652\\
-0.3798828125	32.9986230022076\\
-0.37939453125	33.0426595128832\\
-0.37890625	33.086837857543\\
-0.37841796875	33.1311588222592\\
-0.3779296875	33.1756231998003\\
-0.37744140625	33.2202317897075\\
-0.376953125	33.2649853983708\\
-0.37646484375	33.3098848391078\\
-0.3759765625	33.3549309322429\\
-0.37548828125	33.4001245051873\\
-0.375	33.4454663925205\\
-0.37451171875	33.4909574360736\\
-0.3740234375	33.5365984850125\\
-0.37353515625	33.5823903959231\\
-0.373046875	33.628334032898\\
-0.37255859375	33.6744302676238\\
-0.3720703125	33.7206799794698\\
-0.37158203125	33.7670840555784\\
-0.37109375	33.8136433909567\\
-0.37060546875	33.8603588885689\\
-0.3701171875	33.907231459431\\
-0.36962890625	33.9542620227058\\
-0.369140625	34.0014515058007\\
-0.36865234375	34.0488008444652\\
-0.3681640625	34.0963109828917\\
-0.36767578125	34.1439828738165\\
-0.3671875	34.1918174786226\\
-0.36669921875	34.2398157674447\\
-0.3662109375	34.2879787192747\\
-0.36572265625	34.3363073220698\\
-0.365234375	34.3848025728618\\
-0.36474609375	34.4334654778679\\
-0.3642578125	34.4822970526034\\
-0.36376953125	34.5312983219963\\
-0.36328125	34.5804703205035\\
-0.36279296875	34.6298140922283\\
-0.3623046875	34.6793306910409\\
-0.36181640625	34.7290211806997\\
-0.361328125	34.7788866349746\\
-0.36083984375	34.8289281377731\\
-0.3603515625	34.8791467832673\\
-0.35986328125	34.9295436760233\\
-0.359375	34.9801199311332\\
-0.35888671875	35.0308766743481\\
-0.3583984375	35.0818150422143\\
-0.35791015625	35.1329361822107\\
-0.357421875	35.1842412528893\\
-0.35693359375	35.2357314240172\\
-0.3564453125	35.2874078767212\\
-0.35595703125	35.3392718036351\\
-0.35546875	35.3913244090485\\
-0.35498046875	35.4435669090587\\
-0.3544921875	35.4960005317252\\
-0.35400390625	35.5486265172259\\
-0.353515625	35.6014461180168\\
-0.35302734375	35.6544605989937\\
-0.3525390625	35.7076712376568\\
-0.35205078125	35.7610793242779\\
-0.3515625	35.8146861620706\\
-0.35107421875	35.8684930673633\\
-0.3505859375	35.9225013697746\\
-0.35009765625	35.9767124123923\\
-0.349609375	36.0311275519553\\
-0.34912109375	36.0857481590379\\
-0.3486328125	36.1405756182383\\
-0.34814453125	36.1956113283691\\
-0.34765625	36.2508567026524\\
-0.34716796875	36.3063131689169\\
-0.3466796875	36.3619821697991\\
-0.34619140625	36.4178651629481\\
-0.345703125	36.4739636212336\\
-0.34521484375	36.5302790329572\\
-0.3447265625	36.5868129020685\\
-0.34423828125	36.6435667483835\\
-0.34375	36.700542107808\\
-0.34326171875	36.7577405325642\\
-0.3427734375	36.8151635914218\\
-0.34228515625	36.872812869933\\
-0.341796875	36.9306899706713\\
-0.34130859375	36.9887965134751\\
-0.3408203125	37.0471341356956\\
-0.34033203125	37.1057044924486\\
-0.33984375	37.1645092568714\\
-0.33935546875	37.2235501203844\\
-0.3388671875	37.2828287929568\\
-0.33837890625	37.3423470033776\\
-0.337890625	37.4021064995318\\
-0.33740234375	37.4621090486808\\
-0.3369140625	37.5223564377489\\
-0.33642578125	37.5828504736148\\
-0.3359375	37.6435929834078\\
-0.33544921875	37.7045858148106\\
-0.3349609375	37.7658308363668\\
-0.33447265625	37.8273299377951\\
-0.333984375	37.8890850303084\\
-0.33349609375	37.9510980469397\\
-0.3330078125	38.0133709428736\\
-0.33251953125	38.0759056957849\\
-0.33203125	38.1387043061823\\
-0.33154296875	38.2017687977602\\
-0.3310546875	38.2651012177564\\
-0.33056640625	38.3287036373165\\
-0.330078125	38.3925781518663\\
-0.32958984375	38.4567268814902\\
-0.3291015625	38.5211519713183\\
-0.32861328125	38.5858555919197\\
-0.328125	38.6508399397049\\
-0.32763671875	38.7161072373348\\
-0.3271484375	38.7816597341391\\
-0.32666015625	38.8474997065419\\
-0.326171875	38.9136294584967\\
-0.32568359375	38.9800513219292\\
-0.3251953125	39.0467676571898\\
-0.32470703125	39.113780853515\\
-0.32421875	39.1810933294978\\
-0.32373046875	39.2487075335682\\
-0.3232421875	39.3166259444831\\
-0.32275390625	39.3848510718266\\
-0.322265625	39.4533854565201\\
-0.32177734375	39.5222316713436\\
-0.3212890625	39.5913923214674\\
-0.32080078125	39.6608700449948\\
-0.3203125	39.7306675135164\\
-0.31982421875	39.800787432676\\
-0.3193359375	39.8712325427485\\
-0.31884765625	39.9420056192297\\
-0.318359375	40.0131094734391\\
-0.31787109375	40.0845469531352\\
-0.3173828125	40.1563209431442\\
-0.31689453125	40.2284343660024\\
-0.31640625	40.3008901826119\\
-0.31591796875	40.3736913929112\\
-0.3154296875	40.44684103656\\
-0.31494140625	40.5203421936392\\
-0.314453125	40.5941979853664\\
-0.31396484375	40.6684115748265\\
-0.3134765625	40.7429861677196\\
-0.31298828125	40.8179250131248\\
-0.3125	40.893231404281\\
-0.31201171875	40.9689086793859\\
-0.3115234375	41.0449602224125\\
-0.31103515625	41.121389463944\\
-0.310546875	41.1981998820283\\
-0.31005859375	41.2753950030514\\
-0.3095703125	41.3529784026313\\
-0.30908203125	41.4309537065321\\
-0.30859375	41.5093245915998\\
-0.30810546875	41.5880947867193\\
-0.3076171875	41.6672680737942\\
-0.30712890625	41.7468482887492\\
-0.306640625	41.8268393225567\\
-0.30615234375	41.9072451222871\\
-0.3056640625	41.9880696921847\\
-0.30517578125	42.0693170947684\\
-0.3046875	42.15099145196\\
-0.30419921875	42.2330969462388\\
-0.3037109375	42.3156378218248\\
-0.30322265625	42.3986183858898\\
-0.302734375	42.4820430097993\\
-0.30224609375	42.5659161303843\\
-0.3017578125	42.6502422512441\\
-0.30126953125	42.7350259440822\\
-0.30078125	42.8202718500752\\
-0.30029296875	42.9059846812755\\
-0.2998046875	42.99216922205\\
-0.29931640625	43.0788303305548\\
-0.298828125	43.1659729402473\\
-0.29833984375	43.253602061437\\
-0.2978515625	43.3417227828763\\
-0.29736328125	43.4303402733916\\
-0.296875	43.5194597835574\\
-0.29638671875	43.6090866474137\\
-0.2958984375	43.6992262842278\\
-0.29541015625	43.7898842003033\\
-0.294921875	43.8810659908361\\
-0.29443359375	43.9727773418201\\
-0.2939453125	44.0650240320036\\
-0.29345703125	44.1578119348977\\
-0.29296875	44.25114702084\\
-0.29248046875	44.3450353591125\\
-0.2919921875	44.4394831201187\\
-0.29150390625	44.5344965776191\\
-0.291015625	44.6300821110289\\
-0.29052734375	44.7262462077787\\
-0.2900390625	44.8229954657411\\
-0.28955078125	44.9203365957248\\
-0.2890625	45.0182764240389\\
-0.28857421875	45.1168218951291\\
-0.2880859375	45.2159800742895\\
-0.28759765625	45.3157581504506\\
-0.287109375	45.4161634390482\\
-0.28662109375	45.5172033849743\\
-0.2861328125	45.6188855656136\\
-0.28564453125	45.7212176939687\\
-0.28515625	45.824207621876\\
-0.28466796875	45.9278633433175\\
-0.2841796875	46.0321929978293\\
-0.28369140625	46.1372048740122\\
-0.283203125	46.2429074131472\\
-0.28271484375	46.3493092129194\\
-0.2822265625	46.4564190312553\\
-0.28173828125	46.5642457902754\\
-0.28125	46.672798580369\\
-0.28076171875	46.7820866643937\\
-0.2802734375	46.8921194820042\\
-0.27978515625	47.0029066541168\\
-0.279296875	47.114457987512\\
-0.27880859375	47.2267834795831\\
-0.2783203125	47.3398933232342\\
-0.27783203125	47.4537979119337\\
-0.27734375	47.5685078449299\\
-0.27685546875	47.6840339326338\\
-0.2763671875	47.8003872021763\\
-0.27587890625	47.9175789031456\\
-0.275390625	48.0356205135127\\
-0.27490234375	48.1545237457517\\
-0.2744140625	48.2743005531619\\
-0.27392578125	48.3949631364012\\
-0.2734375	48.5165239502374\\
-0.27294921875	48.6389957105269\\
-0.2724609375	48.7623914014297\\
-0.27197265625	48.8867242828703\\
-0.271484375	49.012007898253\\
-0.27099609375	49.1382560824444\\
-0.2705078125	49.2654829700314\\
-0.27001953125	49.3937030038676\\
-0.26953125	49.5229309439186\\
-0.26904296875	49.6531818764199\\
-0.2685546875	49.784471223359\\
-0.26806640625	49.916814752296\\
-0.267578125	50.0502285865365\\
-0.26708984375	50.1847292156717\\
-0.2666015625	50.3203335065008\\
-0.26611328125	50.457058714353\\
-0.265625	50.5949224948245\\
-0.26513671875	50.7339429159493\\
-0.2646484375	50.8741384708226\\
-0.26416015625	51.0155280906956\\
-0.263671875	51.1581311585626\\
-0.26318359375	51.3019675232618\\
-0.2626953125	51.4470575141123\\
-0.26220703125	51.5934219561107\\
-0.26171875	51.7410821857122\\
-0.26123046875	51.8900600672219\\
-0.2607421875	52.0403780098237\\
-0.26025390625	52.1920589852753\\
-0.259765625	52.3451265462981\\
-0.25927734375	52.4996048456959\\
-0.2587890625	52.6555186562312\\
-0.25830078125	52.8128933912982\\
-0.2578125	52.9717551264242\\
-0.25732421875	53.1321306216401\\
-0.2568359375	53.2940473447587\\
-0.25634765625	53.4575334956019\\
-0.255859375	53.6226180312203\\
-0.25537109375	53.7893306921522\\
-0.2548828125	53.9577020297673\\
-0.25439453125	54.1277634347465\\
-0.25390625	54.2995471667504\\
-0.25341796875	54.4730863853286\\
-0.2529296875	54.6484151821306\\
-0.25244140625	54.8255686144737\\
-0.251953125	55.0045827403327\\
-0.25146484375	55.1854946548158\\
-0.2509765625	55.368342528192\\
-0.25048828125	55.5531656455429\\
-0.25	55.7400044481097\\
-0.24951171875	55.9289005764102\\
-0.2490234375	56.1198969152064\\
-0.24853515625	56.3130376404001\\
-0.248046875	56.5083682679409\\
-0.24755859375	56.7059357048322\\
-0.2470703125	56.90578830232\\
-0.24658203125	57.1079759113556\\
-0.24609375	57.31254994042\\
-0.24560546875	57.5195634158011\\
-0.2451171875	57.729071044415\\
-0.24462890625	57.941129279259\\
-0.244140625	58.1557963875862\\
-0.24365234375	58.3731325218846\\
-0.2431640625	58.593199793742\\
-0.24267578125	58.8160623506698\\
-0.2421875	59.0417864559512\\
-0.24169921875	59.2704405715669\\
-0.2412109375	59.5020954442397\\
-0.24072265625	59.7368241946169\\
-0.240234375	59.9747024095917\\
-0.23974609375	60.2158082377318\\
-0.2392578125	60.4602224877536\\
-0.23876953125	60.7080287299334\\
-0.23828125	60.9593134002991\\
-0.23779296875	61.2141659073819\\
-0.2373046875	61.4726787412307\\
-0.23681640625	61.7349475843031\\
-0.236328125	62.0010714237358\\
-0.23583984375	62.2711526643635\\
-0.2353515625	62.5452972417008\\
-0.23486328125	62.8236147339071\\
-0.234375	63.1062184715307\\
-0.23388671875	63.3932256435519\\
-0.2333984375	63.6847573979232\\
-0.23291015625	63.9809389344096\\
-0.232421875	64.2818995870692\\
-0.23193359375	64.5877728931518\\
-0.2314453125	64.8986966445307\\
-0.23095703125	65.2148129169776\\
-0.23046875	65.5362680716404\\
-0.22998046875	65.863212721934\\
-0.2294921875	66.1958016576917\\
-0.22900390625	66.5341937167794\\
-0.228515625	66.8785515924156\\
-0.22802734375	67.2290415620933\\
-0.2275390625	67.5858331211787\\
-0.22705078125	67.9490985009023\\
-0.2265625	68.319012046419\\
-0.22607421875	68.6957494257946\\
-0.2255859375	69.0794866350054\\
-0.22509765625	69.4703987571541\\
-0.224609375	69.8686584258912\\
-0.22412109375	70.2744339332562\\
-0.2236328125	70.6878869105558\\
-0.22314453125	71.1091694971731\\
-0.22265625	71.5384208960552\\
-0.22216796875	71.9757631957199\\
-0.2216796875	72.4212963166863\\
-0.22119140625	72.8750919150162\\
-0.220703125	73.3371860470972\\
-0.22021484375	73.8075703680456\\
-0.2197265625	74.2861816017686\\
-0.21923828125	74.7728889850407\\
-0.21875	75.2674793531856\\
-0.21826171875	75.7696395049077\\
-0.2177734375	76.2789354643989\\
-0.21728515625	76.7947882590734\\
-0.216796875	77.3164458642238\\
-0.21630859375	77.8429510499902\\
-0.2158203125	78.3731050263187\\
-0.21533203125	78.9054270505096\\
-0.21484375	79.4381105793439\\
-0.21435546875	79.9689771584422\\
-0.2138671875	80.4954300891987\\
-0.21337890625	81.0144110286321\\
-0.212890625	81.5223640557134\\
-0.21240234375	82.015213309238\\
-0.2119140625	82.4883618910578\\
-0.21142578125	82.936721010757\\
-0.2109375	83.3547788286069\\
-0.21044921875	83.7367174920625\\
-0.2099609375	84.0765837856761\\
-0.20947265625	84.3685131564999\\
-0.208984375	84.6069987068226\\
-0.20849609375	84.7871870083615\\
-0.2080078125	84.9051732468722\\
-0.20751953125	84.9582619489747\\
-0.20703125	84.9451590068327\\
-0.20654296875	84.8660673546322\\
-0.2060546875	84.7226718158563\\
-0.20556640625	84.5180154450862\\
-0.205078125	84.2562858992942\\
-0.20458984375	83.9425419817497\\
-0.2041015625	83.5824151758797\\
-0.20361328125	83.1818186990115\\
-0.203125	82.7466892378028\\
-0.20263671875	82.2827768317729\\
-0.2021484375	81.7954889038683\\
-0.20166015625	81.2897868910168\\
-0.201171875	80.7701289865453\\
-0.20068359375	80.2404500593991\\
-0.2001953125	79.7041693093909\\
-0.19970703125	79.1642169721521\\
-0.19921875	78.623072793636\\
-0.19873046875	78.0828106031321\\
-0.1982421875	77.5451448453974\\
-0.19775390625	77.0114762433836\\
-0.197265625	76.4829348046548\\
-0.19677734375	75.9604191646778\\
-0.1962890625	75.4446318143537\\
-0.19580078125	74.9361101323357\\
-0.1953125	74.4352533787614\\
-0.19482421875	73.9423459439497\\
-0.1943359375	73.4575772137776\\
-0.19384765625	72.9810584360397\\
-0.193359375	72.5128369659222\\
-0.19287109375	72.0529082457317\\
-0.1923828125	71.6012258421985\\
-0.19189453125	71.157709829383\\
-0.19140625	70.7222537697074\\
-0.19091796875	70.2947305119058\\
-0.1904296875	69.8749969937562\\
-0.18994140625	69.4628982097929\\
-0.189453125	69.058270479884\\
-0.18896484375	68.6609441334585\\
-0.1884765625	68.2707457060343\\
-0.18798828125	67.8874997292401\\
-0.1875	67.5110301823951\\
-0.18701171875	67.1411616626371\\
-0.1865234375	66.7777203212567\\
-0.18603515625	66.4205346060627\\
-0.185546875	66.0694358430369\\
-0.18505859375	65.7242586850381\\
-0.1845703125	65.3848414507212\\
-0.18408203125	65.0510263729921\\
-0.18359375	64.7226597731139\\
-0.18310546875	64.3995921738925\\
-0.1826171875	64.0816783631426\\
-0.18212890625	63.7687774167522\\
-0.181640625	63.4607526891134\\
-0.18115234375	63.157471777375\\
-0.1806640625	62.8588064648898\\
-0.18017578125	62.5646326483083\\
-0.1796875	62.2748302520221\\
-0.17919921875	61.9892831330106\\
-0.1787109375	61.7078789786234\\
-0.17822265625	61.430509199378\\
-0.177734375	61.1570688184863\\
-0.17724609375	60.8874563595072\\
-0.1767578125	60.6215737332671\\
-0.17626953125	60.3593261249725\\
-0.17578125	60.1006218822542\\
-0.17529296875	59.8453724047374\\
-0.1748046875	59.5934920356016\\
-0.17431640625	59.3448979554919\\
-0.173828125	59.0995100790565\\
-0.17333984375	58.8572509543114\\
-0.1728515625	58.6180456649785\\
-0.17236328125	58.381821735886\\
-0.171875	58.148509041491\\
-0.17138671875	57.9180397175417\\
-0.1708984375	57.6903480758739\\
-0.17041015625	57.4653705223189\\
-0.169921875	57.2430454776732\\
-0.16943359375	57.0233133016783\\
-0.1689453125	56.806116219939\\
-0.16845703125	56.5913982537071\\
-0.16796875	56.3791051524467\\
-0.16748046875	56.1691843290984\\
-0.1669921875	55.9615847979508\\
-0.16650390625	55.756257115032\\
-0.166015625	55.5531533209293\\
-0.16552734375	55.3522268859465\\
-0.1650390625	55.1534326575114\\
-0.16455078125	54.9567268097423\\
-0.1640625	54.7620667950896\\
-0.16357421875	54.5694112979665\\
-0.1630859375	54.3787201902877\\
-0.16259765625	54.1899544888349\\
-0.162109375	54.0030763143739\\
-0.16162109375	53.8180488524469\\
-0.1611328125	53.6348363157682\\
-0.16064453125	53.4534039081552\\
-0.16015625	53.2737177899276\\
-0.15966796875	53.0957450447098\\
-0.1591796875	52.9194536475762\\
-0.15869140625	52.7448124344813\\
-0.158203125	52.5717910729157\\
-0.15771484375	52.4003600337374\\
-0.1572265625	52.230490564124\\
-0.15673828125	52.0621546615987\\
-0.15625	51.8953250490814\\
-0.15576171875	51.7299751509211\\
-0.1552734375	51.5660790698657\\
-0.15478515625	51.4036115649292\\
-0.154296875	51.2425480301165\\
-0.15380859375	51.0828644739685\\
-0.1533203125	50.9245374998925\\
-0.15283203125	50.7675442872431\\
-0.15234375	50.6118625731219\\
-0.15185546875	50.4574706348644\\
-0.1513671875	50.304347273186\\
-0.15087890625	50.1524717959567\\
-0.150390625	50.0018240025799\\
-0.14990234375	49.8523841689489\\
-0.1494140625	49.7041330329565\\
-0.14892578125	49.5570517805339\\
-0.1484375	49.4111220321987\\
-0.14794921875	49.2663258300886\\
-0.1474609375	49.1226456254617\\
-0.14697265625	48.9800642666441\\
-0.146484375	48.8385649874061\\
-0.14599609375	48.6981313957495\\
-0.1455078125	48.5587474630898\\
-0.14501953125	48.4203975138157\\
-0.14453125	48.2830662152132\\
-0.14404296875	48.1467385677365\\
-0.1435546875	48.0113998956155\\
-0.14306640625	47.877035837782\\
-0.142578125	47.7436323391068\\
-0.14208984375	47.611175641932\\
-0.1416015625	47.4796522778874\\
-0.14111328125	47.3490490599821\\
-0.140625	47.2193530749577\\
-0.14013671875	47.0905516758947\\
-0.1396484375	46.9626324750619\\
-0.13916015625	46.8355833370002\\
-0.138671875	46.7093923718301\\
-0.13818359375	46.5840479287764\\
-0.1376953125	46.4595385899009\\
-0.13720703125	46.3358531640353\\
-0.13671875	46.2129806809068\\
-0.13623046875	46.0909103854502\\
-0.1357421875	45.969631732298\\
-0.13525390625	45.849134380443\\
-0.134765625	45.7294081880677\\
-0.13427734375	45.6104432075319\\
-0.1337890625	45.4922296805164\\
-0.13330078125	45.3747580333142\\
-0.1328125	45.2580188722649\\
-0.13232421875	45.1420029793276\\
-0.1318359375	45.0267013077874\\
-0.13134765625	44.9121049780896\\
-0.130859375	44.7982052737984\\
-0.13037109375	44.6849936376757\\
-0.1298828125	44.5724616678744\\
-0.12939453125	44.4606011142445\\
-0.12890625	44.3494038747459\\
-0.12841796875	44.2388619919656\\
-0.1279296875	44.1289676497355\\
-0.12744140625	44.0197131698469\\
-0.126953125	43.9110910088593\\
-0.12646484375	43.8030937549989\\
-0.1259765625	43.6957141251462\\
-0.12548828125	43.5889449619066\\
-0.125	43.482779230764\\
-0.12451171875	43.3772100173135\\
-0.1240234375	43.2722305245701\\
-0.12353515625	43.1678340703521\\
-0.123046875	43.0640140847359\\
-0.12255859375	42.9607641075802\\
-0.1220703125	42.8580777861174\\
-0.12158203125	42.7559488726101\\
-0.12109375	42.6543712220703\\
-0.12060546875	42.5533387900395\\
-0.1201171875	42.4528456304284\\
-0.11962890625	42.3528858934129\\
-0.119140625	42.253453823386\\
-0.11865234375	42.1545437569635\\
-0.1181640625	42.0561501210411\\
-0.11767578125	41.9582674309025\\
-0.1171875	41.8608902883759\\
-0.11669921875	41.7640133800385\\
-0.1162109375	41.6676314754658\\
-0.11572265625	41.5717394255265\\
-0.115234375	41.4763321607205\\
-0.11474609375	41.3814046895578\\
-0.1142578125	41.2869520969793\\
-0.11376953125	41.1929695428164\\
-0.11328125	41.0994522602888\\
-0.11279296875	41.0063955545399\\
-0.1123046875	40.9137948012077\\
-0.11181640625	40.8216454450314\\
-0.111328125	40.7299429984911\\
-0.11083984375	40.6386830404817\\
-0.1103515625	40.5478612150177\\
-0.10986328125	40.4574732299697\\
-0.109375	40.3675148558314\\
-0.10888671875	40.2779819245156\\
-0.1083984375	40.1888703281786\\
-0.10791015625	40.1001760180732\\
-0.107421875	40.0118950034278\\
-0.10693359375	39.9240233503522\\
-0.1064453125	39.8365571807689\\
-0.10595703125	39.749492671369\\
-0.10546875	39.6628260525924\\
-0.10498046875	39.5765536076315\\
-0.1044921875	39.4906716714574\\
-0.10400390625	39.4051766298689\\
-0.103515625	39.3200649185625\\
-0.10302734375	39.2353330222237\\
-0.1025390625	39.1509774736389\\
-0.10205078125	39.0669948528266\\
-0.1015625	38.983381786189\\
-0.10107421875	38.9001349456814\\
-0.1005859375	38.817251048001\\
-0.10009765625	38.7347268537923\\
-0.099609375	38.6525591668715\\
-0.09912109375	38.5707448334663\\
-0.0986328125	38.4892807414734\\
-0.09814453125	38.4081638197317\\
-0.09765625	38.3273910373109\\
-0.09716796875	38.2469594028157\\
-0.0966796875	38.1668659637049\\
-0.09619140625	38.0871078056249\\
-0.095703125	38.0076820517577\\
-0.09521484375	37.9285858621815\\
-0.0947265625	37.8498164332468\\
-0.09423828125	37.7713709969634\\
-0.09375	37.6932468204019\\
-0.09326171875	37.6154412051068\\
-0.0927734375	37.5379514865216\\
-0.09228515625	37.4607750334268\\
-0.091796875	37.3839092473878\\
-0.09130859375	37.307351562216\\
-0.0908203125	37.2310994434394\\
-0.09033203125	37.1551503877846\\
-0.08984375	37.0795019226691\\
-0.08935546875	37.0041516057044\\
-0.0888671875	36.9290970242077\\
-0.08837890625	36.854335794725\\
-0.087890625	36.7798655625628\\
-0.08740234375	36.7056840013291\\
-0.0869140625	36.6317888124835\\
-0.08642578125	36.5581777248966\\
-0.0859375	36.4848484944177\\
-0.08544921875	36.4117989034505\\
-0.0849609375	36.3390267605382\\
-0.08447265625	36.2665298999556\\
-0.083984375	36.1943061813094\\
-0.08349609375	36.1223534891464\\
-0.0830078125	36.0506697325693\\
-0.08251953125	35.979252844859\\
-0.08203125	35.9081007831054\\
-0.08154296875	35.837211527844\\
-0.0810546875	35.7665830827002\\
-0.08056640625	35.6962134740396\\
-0.080078125	35.6261007506257\\
-0.07958984375	35.5562429832832\\
-0.0791015625	35.486638264568\\
-0.07861328125	35.4172847084433\\
-0.078125	35.3481804499614\\
-0.07763671875	35.2793236449519\\
-0.0771484375	35.210712469715\\
-0.07666015625	35.1423451207208\\
-0.076171875	35.0742198143143\\
-0.07568359375	35.0063347864247\\
-0.0751953125	34.9386882922817\\
-0.07470703125	34.8712786061351\\
-0.07421875	34.8041040209807\\
-0.07373046875	34.7371628482906\\
-0.0732421875	34.6704534177482\\
-0.07275390625	34.6039740769886\\
-0.072265625	34.5377231913423\\
-0.07177734375	34.4716991435852\\
-0.0712890625	34.4059003336912\\
-0.07080078125	34.3403251785903\\
-0.0703125	34.2749721119307\\
-0.06982421875	34.209839583845\\
-0.0693359375	34.14492606072\\
-0.06884765625	34.0802300249713\\
-0.068359375	34.0157499748211\\
-0.06787109375	33.9514844240803\\
-0.0673828125	33.887431901934\\
-0.06689453125	33.8235909527308\\
-0.06640625	33.7599601357759\\
-0.06591796875	33.6965380251269\\
-0.0654296875	33.6333232093944\\
-0.06494140625	33.5703142915447\\
-0.064453125	33.5075098887062\\
-0.06396484375	33.4449086319797\\
-0.0634765625	33.3825091662507\\
-0.06298828125	33.3203101500059\\
-0.0625	33.258310255152\\
-0.06201171875	33.1965081668379\\
-0.0615234375	33.1349025832795\\
-0.06103515625	33.0734922155881\\
-0.060546875	33.0122757876002\\
-0.06005859375	32.9512520357118\\
-0.0595703125	32.8904197087141\\
-0.05908203125	32.8297775676325\\
-0.05859375	32.7693243855677\\
-0.05810546875	32.7090589475402\\
-0.0576171875	32.6489800503362\\
-0.05712890625	32.5890865023569\\
-0.056640625	32.5293771234699\\
-0.05615234375	32.4698507448626\\
-0.0556640625	32.4105062088981\\
-0.05517578125	32.3513423689743\\
-0.0546875	32.2923580893835\\
-0.05419921875	32.2335522451755\\
-0.0537109375	32.1749237220227\\
-0.05322265625	32.1164714160867\\
-0.052734375	32.0581942338876\\
-0.05224609375	32.0000910921748\\
-0.0517578125	31.9421609178005\\
-0.05126953125	31.8844026475944\\
-0.05078125	31.8268152282407\\
-0.05029296875	31.769397616157\\
-0.0498046875	31.712148777375\\
-0.04931640625	31.6550676874229\\
-0.048828125	31.59815333121\\
-0.04833984375	31.5414047029122\\
-0.0478515625	31.4848208058604\\
-0.04736328125	31.4284006524293\\
-0.046875	31.3721432639293\\
-0.04638671875	31.3160476704984\\
-0.0458984375	31.2601129109972\\
-0.04541015625	31.2043380329044\\
-0.044921875	31.1487220922146\\
-0.04443359375	31.0932641533369\\
-0.0439453125	31.0379632889957\\
-0.04345703125	30.9828185801325\\
-0.04296875	30.9278291158092\\
-0.04248046875	30.8729939931132\\
-0.0419921875	30.818312317063\\
-0.04150390625	30.7637832005161\\
-0.041015625	30.7094057640782\\
-0.04052734375	30.6551791360127\\
-0.0400390625	30.6011024521524\\
-0.03955078125	30.5471748558128\\
-0.0390625	30.4933954977054\\
-0.03857421875	30.4397635358533\\
-0.0380859375	30.3862781355076\\
-0.03759765625	30.3329384690649\\
-0.037109375	30.2797437159863\\
-0.03662109375	30.2266930627173\\
-0.0361328125	30.1737857026087\\
-0.03564453125	30.1210208358391\\
-0.03515625	30.0683976693378\\
-0.03466796875	30.0159154167096\\
-0.0341796875	29.9635732981597\\
-0.03369140625	29.9113705404205\\
-0.033203125	29.859306376679\\
-0.03271484375	29.8073800465053\\
-0.0322265625	29.7555907957817\\
-0.03173828125	29.7039378766336\\
-0.03125	29.6524205473606\\
-0.03076171875	29.6010380723689\\
-0.0302734375	29.5497897221045\\
-0.02978515625	29.4986747729874\\
-0.029296875	29.4476925073465\\
-0.02880859375	29.3968422133557\\
-0.0283203125	29.3461231849705\\
-0.02783203125	29.2955347218659\\
-0.02734375	29.2450761293743\\
-0.02685546875	29.1947467184257\\
-0.0263671875	29.144545805487\\
-0.02587890625	29.0944727125032\\
-0.025390625	29.044526766839\\
-0.02490234375	28.9947073012215\\
-0.0244140625	28.9450136536832\\
-0.02392578125	28.8954451675059\\
-0.0234375	28.8460011911657\\
-0.02294921875	28.7966810782781\\
-0.0224609375	28.7474841875443\\
-0.02197265625	28.6984098826981\\
-0.021484375	28.6494575324534\\
-0.02099609375	28.6006265104521\\
-0.0205078125	28.5519161952137\\
-0.02001953125	28.5033259700841\\
-0.01953125	28.4548552231861\\
-0.01904296875	28.4065033473704\\
-0.0185546875	28.3582697401668\\
-0.01806640625	28.3101538037365\\
-0.017578125	28.2621549448246\\
-0.01708984375	28.2142725747134\\
-0.0166015625	28.1665061091766\\
-0.01611328125	28.1188549684333\\
-0.015625	28.0713185771034\\
-0.01513671875	28.0238963641632\\
-0.0146484375	27.9765877629014\\
-0.01416015625	27.9293922108758\\
-0.013671875	27.8823091498707\\
-0.01318359375	27.8353380258547\\
-0.0126953125	27.7884782889391\\
-0.01220703125	27.7417293933362\\
-0.01171875	27.6950907973195\\
-0.01123046875	27.6485619631827\\
-0.0107421875	27.6021423572006\\
-0.01025390625	27.5558314495899\\
-0.009765625	27.50962871447\\
-0.00927734375	27.4635336298255\\
-0.0087890625	27.4175456774679\\
-0.00830078125	27.3716643429985\\
-0.0078125	27.3258891157716\\
-0.00732421875	27.2802194888579\\
-0.0068359375	27.2346549590088\\
-0.00634765625	27.1891950266209\\
-0.005859375	27.1438391957006\\
-0.00537109375	27.0985869738296\\
-0.0048828125	27.0534378721306\\
-0.00439453125	27.0083914052336\\
-0.00390625	26.9634470912423\\
-0.00341796875	26.9186044517009\\
-0.0029296875	26.873863011562\\
-0.00244140625	26.8292222991535\\
-0.001953125	26.7846818461475\\
-0.00146484375	26.7402411875281\\
-0.0009765625	26.6958998615608\\
-0.00048828125	26.6516574097616\\
0	26.6075133768661\\
0.00048828125	26.6516574097616\\
0.0009765625	26.6958998615608\\
0.00146484375	26.7402411875281\\
0.001953125	26.7846818461475\\
0.00244140625	26.8292222991535\\
0.0029296875	26.873863011562\\
0.00341796875	26.9186044517009\\
0.00390625	26.9634470912423\\
0.00439453125	27.0083914052336\\
0.0048828125	27.0534378721306\\
0.00537109375	27.0985869738296\\
0.005859375	27.1438391957006\\
0.00634765625	27.1891950266209\\
0.0068359375	27.2346549590088\\
0.00732421875	27.2802194888579\\
0.0078125	27.3258891157716\\
0.00830078125	27.3716643429985\\
0.0087890625	27.4175456774679\\
0.00927734375	27.4635336298255\\
0.009765625	27.50962871447\\
0.01025390625	27.5558314495899\\
0.0107421875	27.6021423572006\\
0.01123046875	27.6485619631827\\
0.01171875	27.6950907973195\\
0.01220703125	27.7417293933362\\
0.0126953125	27.7884782889391\\
0.01318359375	27.8353380258547\\
0.013671875	27.8823091498707\\
0.01416015625	27.9293922108758\\
0.0146484375	27.9765877629014\\
0.01513671875	28.0238963641632\\
0.015625	28.0713185771034\\
0.01611328125	28.1188549684333\\
0.0166015625	28.1665061091766\\
0.01708984375	28.2142725747134\\
0.017578125	28.2621549448246\\
0.01806640625	28.3101538037365\\
0.0185546875	28.3582697401668\\
0.01904296875	28.4065033473704\\
0.01953125	28.4548552231861\\
0.02001953125	28.5033259700841\\
0.0205078125	28.5519161952137\\
0.02099609375	28.6006265104521\\
0.021484375	28.6494575324534\\
0.02197265625	28.6984098826981\\
0.0224609375	28.7474841875443\\
0.02294921875	28.7966810782781\\
0.0234375	28.8460011911657\\
0.02392578125	28.8954451675059\\
0.0244140625	28.9450136536832\\
0.02490234375	28.9947073012215\\
0.025390625	29.044526766839\\
0.02587890625	29.0944727125032\\
0.0263671875	29.144545805487\\
0.02685546875	29.1947467184257\\
0.02734375	29.2450761293743\\
0.02783203125	29.2955347218659\\
0.0283203125	29.3461231849705\\
0.02880859375	29.3968422133557\\
0.029296875	29.4476925073465\\
0.02978515625	29.4986747729874\\
0.0302734375	29.5497897221045\\
0.03076171875	29.6010380723689\\
0.03125	29.6524205473606\\
0.03173828125	29.7039378766336\\
0.0322265625	29.7555907957817\\
0.03271484375	29.8073800465053\\
0.033203125	29.859306376679\\
0.03369140625	29.9113705404205\\
0.0341796875	29.9635732981597\\
0.03466796875	30.0159154167096\\
0.03515625	30.0683976693378\\
0.03564453125	30.1210208358391\\
0.0361328125	30.1737857026087\\
0.03662109375	30.2266930627173\\
0.037109375	30.2797437159863\\
0.03759765625	30.3329384690649\\
0.0380859375	30.3862781355076\\
0.03857421875	30.4397635358533\\
0.0390625	30.4933954977054\\
0.03955078125	30.5471748558128\\
0.0400390625	30.6011024521524\\
0.04052734375	30.6551791360127\\
0.041015625	30.7094057640782\\
0.04150390625	30.7637832005161\\
0.0419921875	30.818312317063\\
0.04248046875	30.8729939931132\\
0.04296875	30.9278291158092\\
0.04345703125	30.9828185801325\\
0.0439453125	31.0379632889957\\
0.04443359375	31.0932641533369\\
0.044921875	31.1487220922146\\
0.04541015625	31.2043380329044\\
0.0458984375	31.2601129109972\\
0.04638671875	31.3160476704984\\
0.046875	31.3721432639293\\
0.04736328125	31.4284006524293\\
0.0478515625	31.4848208058604\\
0.04833984375	31.5414047029122\\
0.048828125	31.59815333121\\
0.04931640625	31.6550676874229\\
0.0498046875	31.712148777375\\
0.05029296875	31.769397616157\\
0.05078125	31.8268152282407\\
0.05126953125	31.8844026475944\\
0.0517578125	31.9421609178005\\
0.05224609375	32.0000910921748\\
0.052734375	32.0581942338876\\
0.05322265625	32.1164714160867\\
0.0537109375	32.1749237220227\\
0.05419921875	32.2335522451755\\
0.0546875	32.2923580893835\\
0.05517578125	32.3513423689743\\
0.0556640625	32.4105062088981\\
0.05615234375	32.4698507448626\\
0.056640625	32.5293771234699\\
0.05712890625	32.5890865023569\\
0.0576171875	32.6489800503362\\
0.05810546875	32.7090589475402\\
0.05859375	32.7693243855677\\
0.05908203125	32.8297775676325\\
0.0595703125	32.8904197087141\\
0.06005859375	32.9512520357118\\
0.060546875	33.0122757876002\\
0.06103515625	33.0734922155881\\
0.0615234375	33.1349025832795\\
0.06201171875	33.1965081668379\\
0.0625	33.258310255152\\
0.06298828125	33.3203101500059\\
0.0634765625	33.3825091662507\\
0.06396484375	33.4449086319797\\
0.064453125	33.5075098887062\\
0.06494140625	33.5703142915447\\
0.0654296875	33.6333232093944\\
0.06591796875	33.6965380251269\\
0.06640625	33.7599601357759\\
0.06689453125	33.8235909527308\\
0.0673828125	33.887431901934\\
0.06787109375	33.9514844240803\\
0.068359375	34.0157499748211\\
0.06884765625	34.0802300249713\\
0.0693359375	34.14492606072\\
0.06982421875	34.209839583845\\
0.0703125	34.2749721119307\\
0.07080078125	34.3403251785903\\
0.0712890625	34.4059003336912\\
0.07177734375	34.4716991435852\\
0.072265625	34.5377231913423\\
0.07275390625	34.6039740769886\\
0.0732421875	34.6704534177482\\
0.07373046875	34.7371628482906\\
0.07421875	34.8041040209807\\
0.07470703125	34.8712786061351\\
0.0751953125	34.9386882922817\\
0.07568359375	35.0063347864247\\
0.076171875	35.0742198143143\\
0.07666015625	35.1423451207208\\
0.0771484375	35.210712469715\\
0.07763671875	35.2793236449519\\
0.078125	35.3481804499614\\
0.07861328125	35.4172847084433\\
0.0791015625	35.486638264568\\
0.07958984375	35.5562429832832\\
0.080078125	35.6261007506257\\
0.08056640625	35.6962134740396\\
0.0810546875	35.7665830827002\\
0.08154296875	35.837211527844\\
0.08203125	35.9081007831054\\
0.08251953125	35.979252844859\\
0.0830078125	36.0506697325693\\
0.08349609375	36.1223534891464\\
0.083984375	36.1943061813094\\
0.08447265625	36.2665298999556\\
0.0849609375	36.3390267605382\\
0.08544921875	36.4117989034505\\
0.0859375	36.4848484944177\\
0.08642578125	36.5581777248966\\
0.0869140625	36.6317888124835\\
0.08740234375	36.7056840013291\\
0.087890625	36.7798655625628\\
0.08837890625	36.854335794725\\
0.0888671875	36.9290970242077\\
0.08935546875	37.0041516057044\\
0.08984375	37.0795019226691\\
0.09033203125	37.1551503877846\\
0.0908203125	37.2310994434394\\
0.09130859375	37.307351562216\\
0.091796875	37.3839092473878\\
0.09228515625	37.4607750334268\\
0.0927734375	37.5379514865216\\
0.09326171875	37.6154412051068\\
0.09375	37.6932468204019\\
0.09423828125	37.7713709969634\\
0.0947265625	37.8498164332468\\
0.09521484375	37.9285858621815\\
0.095703125	38.0076820517577\\
0.09619140625	38.0871078056249\\
0.0966796875	38.1668659637049\\
0.09716796875	38.2469594028157\\
0.09765625	38.3273910373109\\
0.09814453125	38.4081638197317\\
0.0986328125	38.4892807414734\\
0.09912109375	38.5707448334663\\
0.099609375	38.6525591668715\\
0.10009765625	38.7347268537923\\
0.1005859375	38.817251048001\\
0.10107421875	38.9001349456814\\
0.1015625	38.983381786189\\
0.10205078125	39.0669948528266\\
0.1025390625	39.1509774736389\\
0.10302734375	39.2353330222237\\
0.103515625	39.3200649185625\\
0.10400390625	39.4051766298689\\
0.1044921875	39.4906716714574\\
0.10498046875	39.5765536076315\\
0.10546875	39.6628260525924\\
0.10595703125	39.749492671369\\
0.1064453125	39.8365571807689\\
0.10693359375	39.9240233503522\\
0.107421875	40.0118950034278\\
0.10791015625	40.1001760180732\\
0.1083984375	40.1888703281786\\
0.10888671875	40.2779819245156\\
0.109375	40.3675148558314\\
0.10986328125	40.4574732299697\\
0.1103515625	40.5478612150177\\
0.11083984375	40.6386830404817\\
0.111328125	40.7299429984911\\
0.11181640625	40.8216454450314\\
0.1123046875	40.9137948012077\\
0.11279296875	41.0063955545399\\
0.11328125	41.0994522602888\\
0.11376953125	41.1929695428164\\
0.1142578125	41.2869520969793\\
0.11474609375	41.3814046895578\\
0.115234375	41.4763321607205\\
0.11572265625	41.5717394255265\\
0.1162109375	41.6676314754658\\
0.11669921875	41.7640133800385\\
0.1171875	41.8608902883759\\
0.11767578125	41.9582674309025\\
0.1181640625	42.0561501210411\\
0.11865234375	42.1545437569635\\
0.119140625	42.253453823386\\
0.11962890625	42.3528858934129\\
0.1201171875	42.4528456304284\\
0.12060546875	42.5533387900395\\
0.12109375	42.6543712220703\\
0.12158203125	42.7559488726101\\
0.1220703125	42.8580777861174\\
0.12255859375	42.9607641075802\\
0.123046875	43.0640140847359\\
0.12353515625	43.1678340703521\\
0.1240234375	43.2722305245701\\
0.12451171875	43.3772100173135\\
0.125	43.482779230764\\
0.12548828125	43.5889449619066\\
0.1259765625	43.6957141251462\\
0.12646484375	43.8030937549989\\
0.126953125	43.9110910088593\\
0.12744140625	44.0197131698469\\
0.1279296875	44.1289676497355\\
0.12841796875	44.2388619919656\\
0.12890625	44.3494038747459\\
0.12939453125	44.4606011142445\\
0.1298828125	44.5724616678744\\
0.13037109375	44.6849936376757\\
0.130859375	44.7982052737984\\
0.13134765625	44.9121049780896\\
0.1318359375	45.0267013077874\\
0.13232421875	45.1420029793276\\
0.1328125	45.2580188722649\\
0.13330078125	45.3747580333142\\
0.1337890625	45.4922296805164\\
0.13427734375	45.6104432075319\\
0.134765625	45.7294081880677\\
0.13525390625	45.849134380443\\
0.1357421875	45.969631732298\\
0.13623046875	46.0909103854502\\
0.13671875	46.2129806809068\\
0.13720703125	46.3358531640353\\
0.1376953125	46.4595385899009\\
0.13818359375	46.5840479287764\\
0.138671875	46.7093923718301\\
0.13916015625	46.8355833370002\\
0.1396484375	46.9626324750619\\
0.14013671875	47.0905516758947\\
0.140625	47.2193530749577\\
0.14111328125	47.3490490599821\\
0.1416015625	47.4796522778874\\
0.14208984375	47.611175641932\\
0.142578125	47.7436323391068\\
0.14306640625	47.877035837782\\
0.1435546875	48.0113998956155\\
0.14404296875	48.1467385677365\\
0.14453125	48.2830662152132\\
0.14501953125	48.4203975138157\\
0.1455078125	48.5587474630898\\
0.14599609375	48.6981313957495\\
0.146484375	48.8385649874061\\
0.14697265625	48.9800642666441\\
0.1474609375	49.1226456254617\\
0.14794921875	49.2663258300886\\
0.1484375	49.4111220321987\\
0.14892578125	49.5570517805339\\
0.1494140625	49.7041330329565\\
0.14990234375	49.8523841689489\\
0.150390625	50.0018240025799\\
0.15087890625	50.1524717959567\\
0.1513671875	50.304347273186\\
0.15185546875	50.4574706348644\\
0.15234375	50.6118625731219\\
0.15283203125	50.7675442872431\\
0.1533203125	50.9245374998925\\
0.15380859375	51.0828644739685\\
0.154296875	51.2425480301165\\
0.15478515625	51.4036115649292\\
0.1552734375	51.5660790698657\\
0.15576171875	51.7299751509211\\
0.15625	51.8953250490814\\
0.15673828125	52.0621546615987\\
0.1572265625	52.230490564124\\
0.15771484375	52.4003600337374\\
0.158203125	52.5717910729157\\
0.15869140625	52.7448124344813\\
0.1591796875	52.9194536475762\\
0.15966796875	53.0957450447098\\
0.16015625	53.2737177899276\\
0.16064453125	53.4534039081552\\
0.1611328125	53.6348363157682\\
0.16162109375	53.8180488524469\\
0.162109375	54.0030763143739\\
0.16259765625	54.1899544888349\\
0.1630859375	54.3787201902877\\
0.16357421875	54.5694112979665\\
0.1640625	54.7620667950896\\
0.16455078125	54.9567268097423\\
0.1650390625	55.1534326575114\\
0.16552734375	55.3522268859465\\
0.166015625	55.5531533209293\\
0.16650390625	55.756257115032\\
0.1669921875	55.9615847979508\\
0.16748046875	56.1691843290984\\
0.16796875	56.3791051524467\\
0.16845703125	56.5913982537071\\
0.1689453125	56.806116219939\\
0.16943359375	57.0233133016783\\
0.169921875	57.2430454776732\\
0.17041015625	57.4653705223189\\
0.1708984375	57.6903480758739\\
0.17138671875	57.9180397175417\\
0.171875	58.148509041491\\
0.17236328125	58.381821735886\\
0.1728515625	58.6180456649785\\
0.17333984375	58.8572509543114\\
0.173828125	59.0995100790565\\
0.17431640625	59.3448979554919\\
0.1748046875	59.5934920356016\\
0.17529296875	59.8453724047374\\
0.17578125	60.1006218822542\\
0.17626953125	60.3593261249725\\
0.1767578125	60.6215737332671\\
0.17724609375	60.8874563595072\\
0.177734375	61.1570688184863\\
0.17822265625	61.430509199378\\
0.1787109375	61.7078789786234\\
0.17919921875	61.9892831330106\\
0.1796875	62.2748302520221\\
0.18017578125	62.5646326483083\\
0.1806640625	62.8588064648898\\
0.18115234375	63.157471777375\\
0.181640625	63.4607526891134\\
0.18212890625	63.7687774167522\\
0.1826171875	64.0816783631426\\
0.18310546875	64.3995921738925\\
0.18359375	64.7226597731139\\
0.18408203125	65.0510263729921\\
0.1845703125	65.3848414507212\\
0.18505859375	65.7242586850381\\
0.185546875	66.0694358430369\\
0.18603515625	66.4205346060627\\
0.1865234375	66.7777203212567\\
0.18701171875	67.1411616626371\\
0.1875	67.5110301823951\\
0.18798828125	67.8874997292401\\
0.1884765625	68.2707457060343\\
0.18896484375	68.6609441334585\\
0.189453125	69.058270479884\\
0.18994140625	69.4628982097929\\
0.1904296875	69.8749969937562\\
0.19091796875	70.2947305119058\\
0.19140625	70.7222537697074\\
0.19189453125	71.157709829383\\
0.1923828125	71.6012258421985\\
0.19287109375	72.0529082457317\\
0.193359375	72.5128369659222\\
0.19384765625	72.9810584360397\\
0.1943359375	73.4575772137776\\
0.19482421875	73.9423459439497\\
0.1953125	74.4352533787614\\
0.19580078125	74.9361101323357\\
0.1962890625	75.4446318143537\\
0.19677734375	75.9604191646778\\
0.197265625	76.4829348046548\\
0.19775390625	77.0114762433836\\
0.1982421875	77.5451448453974\\
0.19873046875	78.0828106031321\\
0.19921875	78.623072793636\\
0.19970703125	79.1642169721521\\
0.2001953125	79.7041693093909\\
0.20068359375	80.2404500593991\\
0.201171875	80.7701289865453\\
0.20166015625	81.2897868910168\\
0.2021484375	81.7954889038683\\
0.20263671875	82.2827768317729\\
0.203125	82.7466892378028\\
0.20361328125	83.1818186990115\\
0.2041015625	83.5824151758797\\
0.20458984375	83.9425419817497\\
0.205078125	84.2562858992942\\
0.20556640625	84.5180154450862\\
0.2060546875	84.7226718158563\\
0.20654296875	84.8660673546322\\
0.20703125	84.9451590068327\\
0.20751953125	84.9582619489747\\
0.2080078125	84.9051732468722\\
0.20849609375	84.7871870083615\\
0.208984375	84.6069987068226\\
0.20947265625	84.3685131564999\\
0.2099609375	84.0765837856761\\
0.21044921875	83.7367174920625\\
0.2109375	83.3547788286069\\
0.21142578125	82.936721010757\\
0.2119140625	82.4883618910578\\
0.21240234375	82.015213309238\\
0.212890625	81.5223640557134\\
0.21337890625	81.0144110286321\\
0.2138671875	80.4954300891987\\
0.21435546875	79.9689771584422\\
0.21484375	79.4381105793439\\
0.21533203125	78.9054270505096\\
0.2158203125	78.3731050263187\\
0.21630859375	77.8429510499902\\
0.216796875	77.3164458642238\\
0.21728515625	76.7947882590734\\
0.2177734375	76.2789354643989\\
0.21826171875	75.7696395049077\\
0.21875	75.2674793531856\\
0.21923828125	74.7728889850407\\
0.2197265625	74.2861816017686\\
0.22021484375	73.8075703680456\\
0.220703125	73.3371860470972\\
0.22119140625	72.8750919150162\\
0.2216796875	72.4212963166863\\
0.22216796875	71.9757631957199\\
0.22265625	71.5384208960552\\
0.22314453125	71.1091694971731\\
0.2236328125	70.6878869105558\\
0.22412109375	70.2744339332562\\
0.224609375	69.8686584258912\\
0.22509765625	69.4703987571541\\
0.2255859375	69.0794866350054\\
0.22607421875	68.6957494257946\\
0.2265625	68.319012046419\\
0.22705078125	67.9490985009023\\
0.2275390625	67.5858331211787\\
0.22802734375	67.2290415620933\\
0.228515625	66.8785515924156\\
0.22900390625	66.5341937167794\\
0.2294921875	66.1958016576917\\
0.22998046875	65.863212721934\\
0.23046875	65.5362680716404\\
0.23095703125	65.2148129169776\\
0.2314453125	64.8986966445307\\
0.23193359375	64.5877728931518\\
0.232421875	64.2818995870692\\
0.23291015625	63.9809389344096\\
0.2333984375	63.6847573979232\\
0.23388671875	63.3932256435519\\
0.234375	63.1062184715307\\
0.23486328125	62.8236147339071\\
0.2353515625	62.5452972417008\\
0.23583984375	62.2711526643635\\
0.236328125	62.0010714237358\\
0.23681640625	61.7349475843031\\
0.2373046875	61.4726787412307\\
0.23779296875	61.2141659073819\\
0.23828125	60.9593134002991\\
0.23876953125	60.7080287299334\\
0.2392578125	60.4602224877536\\
0.23974609375	60.2158082377318\\
0.240234375	59.9747024095917\\
0.24072265625	59.7368241946169\\
0.2412109375	59.5020954442397\\
0.24169921875	59.2704405715669\\
0.2421875	59.0417864559512\\
0.24267578125	58.8160623506698\\
0.2431640625	58.593199793742\\
0.24365234375	58.3731325218846\\
0.244140625	58.1557963875862\\
0.24462890625	57.941129279259\\
0.2451171875	57.729071044415\\
0.24560546875	57.5195634158011\\
0.24609375	57.31254994042\\
0.24658203125	57.1079759113556\\
0.2470703125	56.90578830232\\
0.24755859375	56.7059357048322\\
0.248046875	56.5083682679409\\
0.24853515625	56.3130376404001\\
0.2490234375	56.1198969152064\\
0.24951171875	55.9289005764102\\
0.25	55.7400044481097\\
0.25048828125	55.5531656455429\\
0.2509765625	55.368342528192\\
0.25146484375	55.1854946548158\\
0.251953125	55.0045827403327\\
0.25244140625	54.8255686144737\\
0.2529296875	54.6484151821306\\
0.25341796875	54.4730863853286\\
0.25390625	54.2995471667504\\
0.25439453125	54.1277634347465\\
0.2548828125	53.9577020297673\\
0.25537109375	53.7893306921522\\
0.255859375	53.6226180312203\\
0.25634765625	53.4575334956019\\
0.2568359375	53.2940473447587\\
0.25732421875	53.1321306216401\\
0.2578125	52.9717551264242\\
0.25830078125	52.8128933912982\\
0.2587890625	52.6555186562312\\
0.25927734375	52.4996048456959\\
0.259765625	52.3451265462981\\
0.26025390625	52.1920589852753\\
0.2607421875	52.0403780098237\\
0.26123046875	51.8900600672219\\
0.26171875	51.7410821857122\\
0.26220703125	51.5934219561107\\
0.2626953125	51.4470575141123\\
0.26318359375	51.3019675232618\\
0.263671875	51.1581311585626\\
0.26416015625	51.0155280906956\\
0.2646484375	50.8741384708226\\
0.26513671875	50.7339429159493\\
0.265625	50.5949224948245\\
0.26611328125	50.457058714353\\
0.2666015625	50.3203335065008\\
0.26708984375	50.1847292156717\\
0.267578125	50.0502285865365\\
0.26806640625	49.916814752296\\
0.2685546875	49.784471223359\\
0.26904296875	49.6531818764199\\
0.26953125	49.5229309439186\\
0.27001953125	49.3937030038676\\
0.2705078125	49.2654829700314\\
0.27099609375	49.1382560824444\\
0.271484375	49.012007898253\\
0.27197265625	48.8867242828703\\
0.2724609375	48.7623914014297\\
0.27294921875	48.6389957105269\\
0.2734375	48.5165239502374\\
0.27392578125	48.3949631364012\\
0.2744140625	48.2743005531619\\
0.27490234375	48.1545237457517\\
0.275390625	48.0356205135127\\
0.27587890625	47.9175789031456\\
0.2763671875	47.8003872021763\\
0.27685546875	47.6840339326338\\
0.27734375	47.5685078449299\\
0.27783203125	47.4537979119337\\
0.2783203125	47.3398933232342\\
0.27880859375	47.2267834795831\\
0.279296875	47.114457987512\\
0.27978515625	47.0029066541168\\
0.2802734375	46.8921194820042\\
0.28076171875	46.7820866643937\\
0.28125	46.672798580369\\
0.28173828125	46.5642457902754\\
0.2822265625	46.4564190312553\\
0.28271484375	46.3493092129194\\
0.283203125	46.2429074131472\\
0.28369140625	46.1372048740122\\
0.2841796875	46.0321929978293\\
0.28466796875	45.9278633433175\\
0.28515625	45.824207621876\\
0.28564453125	45.7212176939687\\
0.2861328125	45.6188855656136\\
0.28662109375	45.5172033849743\\
0.287109375	45.4161634390482\\
0.28759765625	45.3157581504506\\
0.2880859375	45.2159800742895\\
0.28857421875	45.1168218951291\\
0.2890625	45.0182764240389\\
0.28955078125	44.9203365957248\\
0.2900390625	44.8229954657411\\
0.29052734375	44.7262462077787\\
0.291015625	44.6300821110289\\
0.29150390625	44.5344965776191\\
0.2919921875	44.4394831201187\\
0.29248046875	44.3450353591125\\
0.29296875	44.25114702084\\
0.29345703125	44.1578119348977\\
0.2939453125	44.0650240320036\\
0.29443359375	43.9727773418201\\
0.294921875	43.8810659908361\\
0.29541015625	43.7898842003033\\
0.2958984375	43.6992262842278\\
0.29638671875	43.6090866474137\\
0.296875	43.5194597835574\\
0.29736328125	43.4303402733916\\
0.2978515625	43.3417227828763\\
0.29833984375	43.253602061437\\
0.298828125	43.1659729402473\\
0.29931640625	43.0788303305548\\
0.2998046875	42.99216922205\\
0.30029296875	42.9059846812755\\
0.30078125	42.8202718500752\\
0.30126953125	42.7350259440822\\
0.3017578125	42.6502422512441\\
0.30224609375	42.5659161303843\\
0.302734375	42.4820430097993\\
0.30322265625	42.3986183858898\\
0.3037109375	42.3156378218248\\
0.30419921875	42.2330969462388\\
0.3046875	42.15099145196\\
0.30517578125	42.0693170947684\\
0.3056640625	41.9880696921847\\
0.30615234375	41.9072451222871\\
0.306640625	41.8268393225567\\
0.30712890625	41.7468482887492\\
0.3076171875	41.6672680737942\\
0.30810546875	41.5880947867193\\
0.30859375	41.5093245915998\\
0.30908203125	41.4309537065321\\
0.3095703125	41.3529784026313\\
0.31005859375	41.2753950030514\\
0.310546875	41.1981998820283\\
0.31103515625	41.121389463944\\
0.3115234375	41.0449602224125\\
0.31201171875	40.9689086793859\\
0.3125	40.893231404281\\
0.31298828125	40.8179250131248\\
0.3134765625	40.7429861677196\\
0.31396484375	40.6684115748265\\
0.314453125	40.5941979853664\\
0.31494140625	40.5203421936392\\
0.3154296875	40.44684103656\\
0.31591796875	40.3736913929112\\
0.31640625	40.3008901826119\\
0.31689453125	40.2284343660024\\
0.3173828125	40.1563209431442\\
0.31787109375	40.0845469531352\\
0.318359375	40.0131094734391\\
0.31884765625	39.9420056192297\\
0.3193359375	39.8712325427485\\
0.31982421875	39.800787432676\\
0.3203125	39.7306675135164\\
0.32080078125	39.6608700449948\\
0.3212890625	39.5913923214674\\
0.32177734375	39.5222316713436\\
0.322265625	39.4533854565201\\
0.32275390625	39.3848510718266\\
0.3232421875	39.3166259444831\\
0.32373046875	39.2487075335682\\
0.32421875	39.1810933294978\\
0.32470703125	39.113780853515\\
0.3251953125	39.0467676571898\\
0.32568359375	38.9800513219292\\
0.326171875	38.9136294584967\\
0.32666015625	38.8474997065419\\
0.3271484375	38.7816597341391\\
0.32763671875	38.7161072373348\\
0.328125	38.6508399397049\\
0.32861328125	38.5858555919197\\
0.3291015625	38.5211519713183\\
0.32958984375	38.4567268814902\\
0.330078125	38.3925781518663\\
0.33056640625	38.3287036373165\\
0.3310546875	38.2651012177564\\
0.33154296875	38.2017687977602\\
0.33203125	38.1387043061823\\
0.33251953125	38.0759056957849\\
0.3330078125	38.0133709428736\\
0.33349609375	37.9510980469397\\
0.333984375	37.8890850303084\\
0.33447265625	37.8273299377951\\
0.3349609375	37.7658308363668\\
0.33544921875	37.7045858148106\\
0.3359375	37.6435929834078\\
0.33642578125	37.5828504736148\\
0.3369140625	37.5223564377489\\
0.33740234375	37.4621090486808\\
0.337890625	37.4021064995318\\
0.33837890625	37.3423470033776\\
0.3388671875	37.2828287929568\\
0.33935546875	37.2235501203844\\
0.33984375	37.1645092568714\\
0.34033203125	37.1057044924486\\
0.3408203125	37.0471341356956\\
0.34130859375	36.9887965134751\\
0.341796875	36.9306899706713\\
0.34228515625	36.872812869933\\
0.3427734375	36.8151635914218\\
0.34326171875	36.7577405325642\\
0.34375	36.700542107808\\
0.34423828125	36.6435667483835\\
0.3447265625	36.5868129020685\\
0.34521484375	36.5302790329572\\
0.345703125	36.4739636212336\\
0.34619140625	36.4178651629481\\
0.3466796875	36.3619821697991\\
0.34716796875	36.3063131689169\\
0.34765625	36.2508567026524\\
0.34814453125	36.1956113283691\\
0.3486328125	36.1405756182383\\
0.34912109375	36.0857481590379\\
0.349609375	36.0311275519553\\
0.35009765625	35.9767124123923\\
0.3505859375	35.9225013697746\\
0.35107421875	35.8684930673633\\
0.3515625	35.8146861620706\\
0.35205078125	35.7610793242779\\
0.3525390625	35.7076712376568\\
0.35302734375	35.6544605989937\\
0.353515625	35.6014461180168\\
0.35400390625	35.5486265172259\\
0.3544921875	35.4960005317252\\
0.35498046875	35.4435669090587\\
0.35546875	35.3913244090485\\
0.35595703125	35.3392718036351\\
0.3564453125	35.2874078767212\\
0.35693359375	35.2357314240172\\
0.357421875	35.1842412528893\\
0.35791015625	35.1329361822107\\
0.3583984375	35.0818150422143\\
0.35888671875	35.0308766743481\\
0.359375	34.9801199311332\\
0.35986328125	34.9295436760233\\
0.3603515625	34.8791467832673\\
0.36083984375	34.8289281377731\\
0.361328125	34.7788866349746\\
0.36181640625	34.7290211806997\\
0.3623046875	34.6793306910409\\
0.36279296875	34.6298140922283\\
0.36328125	34.5804703205035\\
0.36376953125	34.5312983219963\\
0.3642578125	34.4822970526034\\
0.36474609375	34.4334654778679\\
0.365234375	34.3848025728618\\
0.36572265625	34.3363073220698\\
0.3662109375	34.2879787192747\\
0.36669921875	34.2398157674447\\
0.3671875	34.1918174786226\\
0.36767578125	34.1439828738165\\
0.3681640625	34.0963109828917\\
0.36865234375	34.0488008444652\\
0.369140625	34.0014515058007\\
0.36962890625	33.9542620227058\\
0.3701171875	33.907231459431\\
0.37060546875	33.8603588885689\\
0.37109375	33.8136433909567\\
0.37158203125	33.7670840555784\\
0.3720703125	33.7206799794698\\
0.37255859375	33.6744302676238\\
0.373046875	33.628334032898\\
0.37353515625	33.5823903959231\\
0.3740234375	33.5365984850125\\
0.37451171875	33.4909574360736\\
0.375	33.4454663925205\\
0.37548828125	33.4001245051873\\
0.3759765625	33.3549309322429\\
0.37646484375	33.3098848391078\\
0.376953125	33.2649853983708\\
0.37744140625	33.2202317897075\\
0.3779296875	33.1756231998003\\
0.37841796875	33.1311588222592\\
0.37890625	33.086837857543\\
0.37939453125	33.0426595128832\\
0.3798828125	32.9986230022076\\
0.38037109375	32.9547275460652\\
0.380859375	32.9109723715531\\
0.38134765625	32.8673567122427\\
0.3818359375	32.8238798081089\\
0.38232421875	32.7805409054585\\
0.3828125	32.7373392568609\\
0.38330078125	32.6942741210789\\
0.3837890625	32.6513447630009\\
0.38427734375	32.6085504535738\\
0.384765625	32.565890469737\\
0.38525390625	32.5233640943573\\
0.3857421875	32.4809706161642\\
0.38623046875	32.4387093296869\\
0.38671875	32.3965795351914\\
0.38720703125	32.354580538619\\
0.3876953125	32.3127116515251\\
0.38818359375	32.2709721910192\\
0.388671875	32.2293614797057\\
0.38916015625	32.1878788456253\\
0.3896484375	32.1465236221972\\
0.39013671875	32.1052951481622\\
0.390625	32.0641927675263\\
0.39111328125	32.0232158295055\\
0.3916015625	31.9823636884708\\
0.39208984375	31.9416357038942\\
0.392578125	31.9010312402955\\
0.39306640625	31.8605496671895\\
0.3935546875	31.8201903590339\\
0.39404296875	31.7799526951784\\
0.39453125	31.7398360598139\\
0.39501953125	31.6998398419222\\
0.3955078125	31.6599634352273\\
0.39599609375	31.620206238146\\
0.396484375	31.5805676537403\\
0.39697265625	31.5410470896701\\
0.3974609375	31.5016439581455\\
0.39794921875	31.4623576758816\\
0.3984375	31.4231876640519\\
0.39892578125	31.3841333482438\\
0.3994140625	31.3451941584138\\
0.39990234375	31.3063695288437\\
0.400390625	31.2676588980967\\
0.40087890625	31.2290617089754\\
0.4013671875	31.1905774084785\\
0.40185546875	31.1522054477595\\
0.40234375	31.1139452820855\\
0.40283203125	31.075796370796\\
0.4033203125	31.0377581772629\\
0.40380859375	30.9998301688507\\
0.404296875	30.962011816877\\
0.40478515625	30.924302596574\\
0.4052734375	30.8867019870498\\
0.40576171875	30.8492094712508\\
0.40625	30.8118245359241\\
0.40673828125	30.7745466715809\\
0.4072265625	30.7373753724592\\
0.40771484375	30.7003101364888\\
0.408203125	30.6633504652546\\
0.40869140625	30.6264958639623\\
0.4091796875	30.5897458414029\\
0.40966796875	30.5530999099188\\
0.41015625	30.5165575853697\\
0.41064453125	30.4801183870989\\
0.4111328125	30.4437818379003\\
0.41162109375	30.4075474639857\\
0.412109375	30.3714147949522\\
0.41259765625	30.3353833637504\\
0.4130859375	30.2994527066527\\
0.41357421875	30.263622363222\\
0.4140625	30.2278918762812\\
0.41455078125	30.1922607918819\\
0.4150390625	30.156728659275\\
0.41552734375	30.1212950308807\\
0.416015625	30.0859594622586\\
0.41650390625	30.0507215120793\\
0.4169921875	30.0155807420949\\
0.41748046875	29.9805367171112\\
0.41796875	29.9455890049588\\
0.41845703125	29.9107371764662\\
0.4189453125	29.8759808054315\\
0.41943359375	29.841319468596\\
0.419921875	29.8067527456164\\
0.42041015625	29.7722802190392\\
0.4208984375	29.7379014742737\\
0.42138671875	29.7036160995666\\
0.421875	29.669423685976\\
0.42236328125	29.635323827346\\
0.4228515625	29.601316120282\\
0.42333984375	29.5674001641253\\
0.423828125	29.533575560929\\
0.42431640625	29.4998419154337\\
0.4248046875	29.4661988350435\\
0.42529296875	29.4326459298021\\
0.42578125	29.3991828123694\\
0.42626953125	29.3658090979986\\
0.4267578125	29.3325244045128\\
0.42724609375	29.2993283522828\\
0.427734375	29.2662205642043\\
0.42822265625	29.2332006656757\\
0.4287109375	29.2002682845765\\
0.42919921875	29.1674230512451\\
0.4296875	29.134664598458\\
0.43017578125	29.1019925614078\\
0.4306640625	29.0694065776827\\
0.43115234375	29.0369062872456\\
0.431640625	29.0044913324134\\
0.43212890625	28.9721613578371\\
0.4326171875	28.939916010481\\
0.43310546875	28.9077549396035\\
0.43359375	28.8756777967368\\
0.43408203125	28.8436842356681\\
0.4345703125	28.8117739124196\\
0.43505859375	28.7799464852301\\
0.435546875	28.7482016145357\\
0.43603515625	28.7165389629514\\
0.4365234375	28.6849581952523\\
0.43701171875	28.6534589783558\\
0.4375	28.6220409813033\\
0.43798828125	28.5907038752422\\
0.4384765625	28.5594473334085\\
0.43896484375	28.5282710311088\\
0.439453125	28.4971746457036\\
0.43994140625	28.4661578565898\\
0.4404296875	28.4352203451837\\
0.44091796875	28.4043617949042\\
0.44140625	28.3735818911566\\
0.44189453125	28.3428803213155\\
0.4423828125	28.3122567747091\\
0.44287109375	28.2817109426027\\
0.443359375	28.251242518183\\
0.44384765625	28.2208511965422\\
0.4443359375	28.1905366746625\\
0.44482421875	28.1602986514004\\
0.4453125	28.1301368274719\\
0.44580078125	28.1000509054366\\
0.4462890625	28.0700405896835\\
0.44677734375	28.0401055864156\\
0.447265625	28.0102456036357\\
0.44775390625	27.9804603511311\\
0.4482421875	27.9507495404602\\
0.44873046875	27.9211128849374\\
0.44921875	27.8915500996196\\
0.44970703125	27.8620609012917\\
0.4501953125	27.8326450084535\\
0.45068359375	27.803302141305\\
0.451171875	27.774032021734\\
0.45166015625	27.7448343733016\\
0.4521484375	27.71570892123\\
0.45263671875	27.6866553923882\\
0.453125	27.65767351528\\
0.45361328125	27.6287630200306\\
0.4541015625	27.5999236383737\\
0.45458984375	27.5711551036394\\
0.455078125	27.5424571507411\\
0.45556640625	27.5138295161634\\
0.4560546875	27.4852719379501\\
0.45654296875	27.4567841556915\\
0.45703125	27.4283659105125\\
0.45751953125	27.4000169450611\\
0.4580078125	27.3717370034962\\
0.45849609375	27.3435258314759\\
0.458984375	27.3153831761462\\
0.45947265625	27.2873087861293\\
0.4599609375	27.2593024115124\\
0.46044921875	27.2313638038363\\
0.4609375	27.2034927160845\\
0.46142578125	27.1756889026719\\
0.4619140625	27.1479521194341\\
0.46240234375	27.1202821236163\\
0.462890625	27.092678673863\\
0.46337890625	27.0651415302071\\
0.4638671875	27.0376704540593\\
0.46435546875	27.010265208198\\
0.46484375	26.9829255567588\\
0.46533203125	26.9556512652243\\
0.4658203125	26.9284421004137\\
0.46630859375	26.9012978304734\\
0.466796875	26.8742182248667\\
0.46728515625	26.8472030543636\\
0.4677734375	26.8202520910319\\
0.46826171875	26.7933651082267\\
0.46875	26.7665418805813\\
0.46923828125	26.7397821839978\\
0.4697265625	26.7130857956372\\
0.47021484375	26.6864524939107\\
0.470703125	26.6598820584701\\
0.47119140625	26.6333742701987\\
0.4716796875	26.6069289112023\\
0.47216796875	26.5805457648005\\
0.47265625	26.5542246155173\\
0.47314453125	26.5279652490727\\
0.4736328125	26.5017674523737\\
0.47412109375	26.475631013506\\
0.474609375	26.449555721725\\
0.47509765625	26.4235413674477\\
0.4755859375	26.3975877422442\\
0.47607421875	26.3716946388291\\
0.4765625	26.3458618510535\\
0.47705078125	26.3200891738969\\
0.4775390625	26.2943764034588\\
0.47802734375	26.2687233369508\\
0.478515625	26.2431297726887\\
0.47900390625	26.2175955100847\\
0.4794921875	26.192120349639\\
0.47998046875	26.166704092933\\
0.48046875	26.1413465426207\\
0.48095703125	26.1160475024215\\
0.4814453125	26.0908067771129\\
0.48193359375	26.0656241725225\\
0.482421875	26.0404994955208\\
0.48291015625	26.015432554014\\
0.4833984375	25.9904231569365\\
0.48388671875	25.9654711142438\\
0.484375	25.9405762369054\\
0.48486328125	25.9157383368975\\
0.4853515625	25.8909572271959\\
0.48583984375	25.8662327217696\\
0.486328125	25.8415646355733\\
0.48681640625	25.8169527845406\\
0.4873046875	25.7923969855776\\
0.48779296875	25.7678970565558\\
0.48828125	25.7434528163055\\
0.48876953125	25.7190640846093\\
0.4892578125	25.6947306821955\\
0.48974609375	25.6704524307315\\
0.490234375	25.6462291528174\\
0.49072265625	25.6220606719797\\
0.4912109375	25.5979468126649\\
0.49169921875	25.5738874002331\\
0.4921875	25.5498822609521\\
0.49267578125	25.5259312219906\\
0.4931640625	25.5020341114129\\
0.49365234375	25.4781907581723\\
0.494140625	25.454400992105\\
0.49462890625	25.4306646439245\\
0.4951171875	25.4069815452155\\
0.49560546875	25.383351528428\\
0.49609375	25.3597744268715\\
0.49658203125	25.3362500747092\\
0.4970703125	25.3127783069525\\
0.49755859375	25.2893589594549\\
0.498046875	25.265991868907\\
0.49853515625	25.24267687283\\
0.4990234375	25.2194138095713\\
0.49951171875	25.1962025182981\\
0.5	25.1730428389923\\
0.50048828125	25.1499346124451\\
0.5009765625	25.1268776802518\\
0.50146484375	25.1038718848062\\
0.501953125	25.0809170692955\\
0.50244140625	25.058013077695\\
0.5029296875	25.0351597547631\\
0.50341796875	25.012356946036\\
0.50390625	24.9896044978226\\
0.50439453125	24.9669022571996\\
0.5048828125	24.9442500720063\\
0.50537109375	24.9216477908398\\
0.505859375	24.8990952630499\\
0.50634765625	24.8765923387345\\
0.5068359375	24.8541388687345\\
0.50732421875	24.8317347046289\\
0.5078125	24.8093796987304\\
0.50830078125	24.7870737040804\\
0.5087890625	24.7648165744444\\
0.50927734375	24.7426081643071\\
0.509765625	24.7204483288684\\
0.51025390625	24.698336924038\\
0.5107421875	24.6762738064317\\
0.51123046875	24.654258833366\\
0.51171875	24.6322918628546\\
0.51220703125	24.6103727536031\\
0.5126953125	24.5885013650051\\
0.51318359375	24.5666775571378\\
0.513671875	24.5449011907573\\
0.51416015625	24.5231721272949\\
0.5146484375	24.5014902288524\\
0.51513671875	24.4798553581978\\
0.515625	24.4582673787617\\
0.51611328125	24.4367261546324\\
0.5166015625	24.4152315505522\\
0.51708984375	24.3937834319132\\
0.517578125	24.3723816647535\\
0.51806640625	24.3510261157525\\
0.5185546875	24.3297166522276\\
0.51904296875	24.3084531421299\\
0.51953125	24.2872354540402\\
0.52001953125	24.2660634571655\\
0.5205078125	24.2449370213345\\
0.52099609375	24.2238560169943\\
0.521484375	24.2028203152062\\
0.52197265625	24.1818297876422\\
0.5224609375	24.1608843065811\\
0.52294921875	24.1399837449048\\
0.5234375	24.1191279760946\\
0.52392578125	24.0983168742275\\
0.5244140625	24.0775503139726\\
0.52490234375	24.0568281705874\\
0.525390625	24.0361503199144\\
0.52587890625	24.0155166383772\\
0.5263671875	23.9949270029775\\
0.52685546875	23.974381291291\\
0.52734375	23.9538793814643\\
0.52783203125	23.9334211522114\\
0.5283203125	23.9130064828101\\
0.52880859375	23.8926352530989\\
0.529296875	23.8723073434732\\
0.52978515625	23.8520226348824\\
0.5302734375	23.8317810088262\\
0.53076171875	23.8115823473517\\
0.53125	23.7914265330496\\
0.53173828125	23.7713134490515\\
0.5322265625	23.7512429790263\\
0.53271484375	23.7312150071772\\
0.533203125	23.7112294182383\\
0.53369140625	23.6912860974717\\
0.5341796875	23.6713849306642\\
0.53466796875	23.6515258041243\\
0.53515625	23.6317086046792\\
0.53564453125	23.6119332196713\\
0.5361328125	23.5921995369557\\
0.53662109375	23.5725074448968\\
0.537109375	23.5528568323658\\
0.53759765625	23.5332475887369\\
0.5380859375	23.5136796038852\\
0.53857421875	23.4941527681832\\
0.5390625	23.4746669724984\\
0.53955078125	23.4552221081899\\
0.5400390625	23.4358180671057\\
0.54052734375	23.41645474158\\
0.541015625	23.3971320244305\\
0.54150390625	23.3778498089551\\
0.5419921875	23.3586079889295\\
0.54248046875	23.3394064586045\\
0.54296875	23.3202451127029\\
0.54345703125	23.3011238464171\\
0.5439453125	23.2820425554064\\
0.54443359375	23.263001135794\\
0.544921875	23.2439994841647\\
0.54541015625	23.225037497562\\
0.5458984375	23.2061150734858\\
0.54638671875	23.1872321098894\\
0.546875	23.168388505177\\
0.54736328125	23.1495841582014\\
0.5478515625	23.1308189682613\\
0.54833984375	23.1120928350987\\
0.548828125	23.0934056588963\\
0.54931640625	23.0747573402754\\
0.5498046875	23.0561477802929\\
0.55029296875	23.0375768804392\\
0.55078125	23.0190445426358\\
0.55126953125	23.0005506692325\\
0.5517578125	22.9820951630055\\
0.55224609375	22.9636779271545\\
0.552734375	22.9452988653006\\
0.55322265625	22.9269578814841\\
0.5537109375	22.9086548801618\\
0.55419921875	22.8903897662048\\
0.5546875	22.8721624448964\\
0.55517578125	22.8539728219295\\
0.5556640625	22.8358208034047\\
0.55615234375	22.8177062958275\\
0.556640625	22.7996292061065\\
0.55712890625	22.7815894415512\\
0.5576171875	22.7635869098694\\
0.55810546875	22.7456215191654\\
0.55859375	22.7276931779375\\
0.55908203125	22.7098017950762\\
0.5595703125	22.6919472798616\\
0.56005859375	22.6741295419616\\
0.560546875	22.6563484914297\\
0.56103515625	22.6386040387028\\
0.5615234375	22.6208960945993\\
0.56201171875	22.6032245703168\\
0.5625	22.5855893774301\\
0.56298828125	22.5679904278893\\
0.5634765625	22.5504276340176\\
0.56396484375	22.5329009085093\\
0.564453125	22.5154101644278\\
0.56494140625	22.4979553152037\\
0.5654296875	22.4805362746326\\
0.56591796875	22.4631529568736\\
0.56640625	22.4458052764467\\
0.56689453125	22.4284931482313\\
0.5673828125	22.4112164874642\\
0.56787109375	22.3939752097375\\
0.568359375	22.376769230997\\
0.56884765625	22.3595984675401\\
0.5693359375	22.3424628360139\\
0.56982421875	22.3253622534136\\
0.5703125	22.3082966370803\\
0.57080078125	22.2912659046994\\
0.5712890625	22.2742699742986\\
0.57177734375	22.2573087642464\\
0.572265625	22.2403821932498\\
0.57275390625	22.2234901803531\\
0.5732421875	22.2066326449356\\
0.57373046875	22.18980950671\\
0.57421875	22.1730206857211\\
0.57470703125	22.1562661023431\\
0.5751953125	22.139545677279\\
0.57568359375	22.1228593315578\\
0.576171875	22.1062069865336\\
0.57666015625	22.0895885638835\\
0.5771484375	22.0730039856061\\
0.57763671875	22.0564531740197\\
0.578125	22.0399360517607\\
0.57861328125	22.0234525417819\\
0.5791015625	22.0070025673508\\
0.57958984375	21.9905860520484\\
0.580078125	21.9742029197668\\
0.58056640625	21.9578530947083\\
0.5810546875	21.9415365013835\\
0.58154296875	21.9252530646097\\
0.58203125	21.9090027095095\\
0.58251953125	21.8927853615088\\
0.5830078125	21.8766009463359\\
0.58349609375	21.8604493900194\\
0.583984375	21.8443306188868\\
0.58447265625	21.8282445595633\\
0.5849609375	21.8121911389696\\
0.58544921875	21.7961702843213\\
0.5859375	21.7801819231265\\
0.58642578125	21.7642259831849\\
0.5869140625	21.7483023925862\\
0.58740234375	21.7324110797085\\
0.587890625	21.7165519732168\\
0.58837890625	21.700725002062\\
0.5888671875	21.6849300954787\\
0.58935546875	21.6691671829846\\
0.58984375	21.6534361943783\\
0.59033203125	21.6377370597386\\
0.5908203125	21.6220697094225\\
0.59130859375	21.6064340740643\\
0.591796875	21.5908300845736\\
0.59228515625	21.5752576721346\\
0.5927734375	21.5597167682045\\
0.59326171875	21.5442073045118\\
0.59375	21.5287292130553\\
0.59423828125	21.5132824261028\\
0.5947265625	21.4978668761895\\
0.59521484375	21.4824824961168\\
0.595703125	21.4671292189512\\
0.59619140625	21.4518069780225\\
0.5966796875	21.436515706923\\
0.59716796875	21.4212553395058\\
0.59765625	21.406025809884\\
0.59814453125	21.3908270524288\\
0.5986328125	21.3756590017687\\
0.59912109375	21.3605215927882\\
0.599609375	21.3454147606263\\
0.60009765625	21.3303384406755\\
0.6005859375	21.3152925685803\\
0.60107421875	21.3002770802364\\
0.6015625	21.2852919117889\\
0.60205078125	21.2703369996318\\
0.6025390625	21.2554122804059\\
0.60302734375	21.2405176909984\\
0.603515625	21.2256531685415\\
0.60400390625	21.2108186504108\\
0.6044921875	21.1960140742246\\
0.60498046875	21.1812393778426\\
0.60546875	21.1664944993647\\
0.60595703125	21.1517793771297\\
0.6064453125	21.1370939497145\\
0.60693359375	21.1224381559326\\
0.607421875	21.1078119348332\\
0.60791015625	21.0932152257001\\
0.6083984375	21.0786479680501\\
0.60888671875	21.0641101016328\\
0.609375	21.0496015664285\\
0.60986328125	21.0351223026478\\
0.6103515625	21.0206722507301\\
0.61083984375	21.0062513513428\\
0.611328125	20.99185954538\\
0.61181640625	20.9774967739617\\
0.6123046875	20.9631629784322\\
0.61279296875	20.9488581003598\\
0.61328125	20.934582081535\\
0.61376953125	20.9203348639699\\
0.6142578125	20.9061163898971\\
0.61474609375	20.8919266017687\\
0.615234375	20.8777654422548\\
0.61572265625	20.8636328542433\\
0.6162109375	20.8495287808382\\
0.61669921875	20.8354531653588\\
0.6171875	20.8214059513388\\
0.61767578125	20.8073870825253\\
0.6181640625	20.7933965028776\\
0.61865234375	20.7794341565663\\
0.619140625	20.7654999879726\\
0.61962890625	20.7515939416869\\
0.6201171875	20.737715962508\\
0.62060546875	20.7238659954422\\
0.62109375	20.7100439857024\\
0.62158203125	20.6962498787069\\
0.6220703125	20.6824836200786\\
0.62255859375	20.6687451556442\\
0.623046875	20.6550344314329\\
0.62353515625	20.6413513936759\\
0.6240234375	20.6276959888052\\
0.62451171875	20.6140681634526\\
0.625	20.6004678644492\\
0.62548828125	20.586895038824\\
0.6259765625	20.5733496338033\\
0.62646484375	20.5598315968098\\
0.626953125	20.5463408754615\\
0.62744140625	20.532877417571\\
0.6279296875	20.5194411711446\\
0.62841796875	20.5060320843813\\
0.62890625	20.4926501056721\\
0.62939453125	20.479295183599\\
0.6298828125	20.4659672669341\\
0.63037109375	20.4526663046391\\
0.630859375	20.4393922458639\\
0.63134765625	20.4261450399462\\
0.6318359375	20.4129246364103\\
0.63232421875	20.3997309849667\\
0.6328125	20.3865640355108\\
0.63330078125	20.3734237381223\\
0.6337890625	20.3603100430647\\
0.63427734375	20.3472229007836\\
0.634765625	20.334162261907\\
0.63525390625	20.3211280772436\\
0.6357421875	20.3081202977822\\
0.63623046875	20.2951388746915\\
0.63671875	20.2821837593184\\
0.63720703125	20.2692549031878\\
0.6376953125	20.2563522580017\\
0.63818359375	20.2434757756383\\
0.638671875	20.2306254081514\\
0.63916015625	20.2178011077693\\
0.6396484375	20.2050028268945\\
0.64013671875	20.1922305181027\\
0.640625	20.1794841341418\\
0.64111328125	20.1667636279316\\
0.6416015625	20.1540689525628\\
0.64208984375	20.1414000612961\\
0.642578125	20.128756907562\\
0.64306640625	20.1161394449593\\
0.6435546875	20.1035476272551\\
0.64404296875	20.0909814083836\\
0.64453125	20.0784407424454\\
0.64501953125	20.0659255837072\\
0.6455078125	20.0534358866005\\
0.64599609375	20.0409716057213\\
0.646484375	20.0285326958291\\
0.64697265625	20.0161191118467\\
0.6474609375	20.0037308088588\\
0.64794921875	19.9913677421118\\
0.6484375	19.9790298670129\\
0.64892578125	19.9667171391295\\
0.6494140625	19.9544295141886\\
0.64990234375	19.9421669480758\\
0.650390625	19.929929396835\\
0.65087890625	19.9177168166673\\
0.6513671875	19.9055291639309\\
0.65185546875	19.8933663951399\\
0.65234375	19.881228466964\\
0.65283203125	19.8691153362274\\
0.6533203125	19.8570269599088\\
0.65380859375	19.8449632951402\\
0.654296875	19.8329242992064\\
0.65478515625	19.8209099295444\\
0.6552734375	19.8089201437428\\
0.65576171875	19.7969548995412\\
0.65625	19.7850141548292\\
0.65673828125	19.7730978676464\\
0.6572265625	19.761205996181\\
0.65771484375	19.74933849877\\
0.658203125	19.7374953338978\\
0.65869140625	19.7256764601963\\
0.6591796875	19.7138818364438\\
0.65966796875	19.7021114215644\\
0.66015625	19.6903651746276\\
0.66064453125	19.6786430548478\\
0.6611328125	19.6669450215833\\
0.66162109375	19.6552710343359\\
0.662109375	19.6436210527506\\
0.66259765625	19.6319950366145\\
0.6630859375	19.6203929458563\\
0.66357421875	19.6088147405464\\
0.6640625	19.5972603808952\\
0.66455078125	19.5857298272535\\
0.6650390625	19.5742230401112\\
0.66552734375	19.5627399800975\\
0.666015625	19.5512806079795\\
0.66650390625	19.5398448846621\\
0.6669921875	19.5284327711875\\
0.66748046875	19.5170442287344\\
0.66796875	19.5056792186176\\
0.66845703125	19.4943377022873\\
0.6689453125	19.4830196413288\\
0.66943359375	19.4717249974615\\
0.669921875	19.4604537325391\\
0.67041015625	19.4492058085481\\
0.6708984375	19.4379811876082\\
0.67138671875	19.426779831971\\
0.671875	19.41560170402\\
0.67236328125	19.4044467662699\\
0.6728515625	19.3933149813658\\
0.67333984375	19.3822063120832\\
0.673828125	19.371120721327\\
0.67431640625	19.3600581721313\\
0.6748046875	19.3490186276585\\
0.67529296875	19.3380020511995\\
0.67578125	19.3270084061723\\
0.67626953125	19.3160376561222\\
0.6767578125	19.3050897647209\\
0.67724609375	19.2941646957661\\
0.677734375	19.2832624131811\\
0.67822265625	19.2723828810141\\
0.6787109375	19.2615260634379\\
0.67919921875	19.2506919247495\\
0.6796875	19.2398804293691\\
0.68017578125	19.2290915418401\\
0.6806640625	19.2183252268287\\
0.68115234375	19.2075814491228\\
0.681640625	19.1968601736322\\
0.68212890625	19.1861613653877\\
0.6826171875	19.1754849895408\\
0.68310546875	19.1648310113631\\
0.68359375	19.1541993962461\\
0.68408203125	19.1435901097003\\
0.6845703125	19.1330031173553\\
0.68505859375	19.1224383849587\\
0.685546875	19.1118958783762\\
0.68603515625	19.1013755635908\\
0.6865234375	19.0908774067023\\
0.68701171875	19.0804013739274\\
0.6875	19.0699474315985\\
0.68798828125	19.0595155461636\\
0.6884765625	19.0491056841861\\
0.68896484375	19.0387178123439\\
0.689453125	19.0283518974292\\
0.68994140625	19.0180079063482\\
0.6904296875	19.0076858061202\\
0.69091796875	18.9973855638777\\
0.69140625	18.9871071468655\\
0.69189453125	18.9768505224409\\
0.6923828125	18.9666156580724\\
0.69287109375	18.95640252134\\
0.693359375	18.9462110799344\\
0.69384765625	18.9360413016569\\
0.6943359375	18.9258931544186\\
0.69482421875	18.9157666062402\\
0.6953125	18.9056616252515\\
0.69580078125	18.8955781796913\\
0.6962890625	18.8855162379063\\
0.69677734375	18.8754757683514\\
0.697265625	18.865456739589\\
0.69775390625	18.8554591202885\\
0.6982421875	18.8454828792261\\
0.69873046875	18.8355279852843\\
0.69921875	18.8255944074515\\
0.69970703125	18.8156821148215\\
0.7001953125	18.8057910765934\\
0.70068359375	18.7959212620708\\
0.701171875	18.7860726406619\\
0.70166015625	18.7762451818786\\
0.7021484375	18.7664388553366\\
0.70263671875	18.7566536307546\\
0.703125	18.7468894779541\\
0.70361328125	18.7371463668592\\
0.7041015625	18.7274242674957\\
0.70458984375	18.7177231499914\\
0.705078125	18.7080429845753\\
0.70556640625	18.6983837415771\\
0.7060546875	18.6887453914273\\
0.70654296875	18.6791279046565\\
0.70703125	18.669531251895\\
0.70751953125	18.6599554038726\\
0.7080078125	18.6504003314184\\
0.70849609375	18.6408660054598\\
0.708984375	18.6313523970229\\
0.70947265625	18.6218594772317\\
0.7099609375	18.6123872173077\\
0.71044921875	18.6029355885699\\
0.7109375	18.5935045624341\\
0.71142578125	18.5840941104128\\
0.7119140625	18.5747042041146\\
0.71240234375	18.5653348152441\\
0.712890625	18.5559859156014\\
0.71337890625	18.5466574770817\\
0.7138671875	18.5373494716753\\
0.71435546875	18.5280618714668\\
0.71484375	18.5187946486351\\
0.71533203125	18.5095477754529\\
0.7158203125	18.5003212242863\\
0.71630859375	18.4911149675948\\
0.716796875	18.4819289779305\\
0.71728515625	18.4727632279382\\
0.7177734375	18.4636176903548\\
0.71826171875	18.4544923380091\\
0.71875	18.4453871438213\\
0.71923828125	18.436302080803\\
0.7197265625	18.4272371220565\\
0.72021484375	18.4181922407747\\
0.720703125	18.4091674102409\\
0.72119140625	18.4001626038282\\
0.7216796875	18.3911777949992\\
0.72216796875	18.382212957306\\
0.72265625	18.3732680643895\\
0.72314453125	18.3643430899795\\
0.7236328125	18.3554380078939\\
0.72412109375	18.3465527920389\\
0.724609375	18.3376874164082\\
0.72509765625	18.3288418550831\\
0.7255859375	18.320016082232\\
0.72607421875	18.3112100721101\\
0.7265625	18.3024237990593\\
0.72705078125	18.2936572375074\\
0.7275390625	18.2849103619684\\
0.72802734375	18.276183147042\\
0.728515625	18.2674755674129\\
0.72900390625	18.2587875978512\\
0.7294921875	18.2501192132116\\
0.72998046875	18.2414703884333\\
0.73046875	18.2328410985396\\
0.73095703125	18.2242313186379\\
0.7314453125	18.2156410239189\\
0.73193359375	18.2070701896569\\
0.732421875	18.198518791209\\
0.73291015625	18.1899868040154\\
0.7333984375	18.1814742035983\\
0.73388671875	18.1729809655624\\
0.734375	18.1645070655943\\
0.73486328125	18.1560524794621\\
0.7353515625	18.1476171830155\\
0.73583984375	18.1392011521849\\
0.736328125	18.130804362982\\
0.73681640625	18.1224267914986\\
0.7373046875	18.1140684139071\\
0.73779296875	18.1057292064597\\
0.73828125	18.0974091454883\\
0.73876953125	18.0891082074045\\
0.7392578125	18.080826368699\\
0.73974609375	18.0725636059413\\
0.740234375	18.0643198957797\\
0.74072265625	18.056095214941\\
0.7412109375	18.04788954023\\
0.74169921875	18.0397028485295\\
0.7421875	18.0315351167999\\
0.74267578125	18.0233863220789\\
0.7431640625	18.0152564414815\\
0.74365234375	18.0071454521994\\
0.744140625	17.9990533315011\\
0.74462890625	17.9909800567313\\
0.7451171875	17.9829256053109\\
0.74560546875	17.9748899547368\\
0.74609375	17.9668730825813\\
0.74658203125	17.9588749664921\\
0.7470703125	17.9508955841922\\
0.74755859375	17.9429349134793\\
0.748046875	17.9349929322259\\
0.74853515625	17.9270696183788\\
0.7490234375	17.919164949959\\
0.74951171875	17.9112789050614\\
0.75	17.9034114618546\\
0.75048828125	17.8955625985807\\
0.7509765625	17.8877322935548\\
0.75146484375	17.8799205251652\\
0.751953125	17.8721272718729\\
0.75244140625	17.8643525122114\\
0.7529296875	17.8565962247863\\
0.75341796875	17.8488583882754\\
0.75390625	17.8411389814284\\
0.75439453125	17.8334379830664\\
0.7548828125	17.825755372082\\
0.75537109375	17.8180911274388\\
0.755859375	17.8104452281714\\
0.75634765625	17.8028176533852\\
0.7568359375	17.7952083822557\\
0.75732421875	17.7876173940289\\
0.7578125	17.7800446680209\\
0.75830078125	17.7724901836174\\
0.7587890625	17.7649539202737\\
0.75927734375	17.7574358575145\\
0.759765625	17.7499359749336\\
0.76025390625	17.742454252194\\
0.7607421875	17.7349906690269\\
0.76123046875	17.7275452052324\\
0.76171875	17.7201178406787\\
0.76220703125	17.7127085553023\\
0.7626953125	17.7053173291072\\
0.76318359375	17.6979441421652\\
0.763671875	17.6905889746158\\
0.76416015625	17.6832518066654\\
0.7646484375	17.6759326185875\\
0.76513671875	17.6686313907226\\
0.765625	17.6613481034776\\
0.76611328125	17.6540827373258\\
0.7666015625	17.646835272807\\
0.76708984375	17.6396056905267\\
0.767578125	17.6323939711563\\
0.76806640625	17.6252000954328\\
0.7685546875	17.6180240441588\\
0.76904296875	17.6108657982017\\
0.76953125	17.6037253384944\\
0.77001953125	17.5966026460341\\
0.7705078125	17.5894977018831\\
0.77099609375	17.5824104871678\\
0.771484375	17.5753409830789\\
0.77197265625	17.5682891708713\\
0.7724609375	17.5612550318634\\
0.77294921875	17.5542385474377\\
0.7734375	17.5472396990398\\
0.77392578125	17.5402584681788\\
0.7744140625	17.5332948364267\\
0.77490234375	17.5263487854185\\
0.775390625	17.5194202968519\\
0.77587890625	17.5125093524873\\
0.7763671875	17.505615934147\\
0.77685546875	17.498740023716\\
0.77734375	17.4918816031408\\
0.77783203125	17.48504065443\\
0.7783203125	17.4782171596536\\
0.77880859375	17.4714111009433\\
0.779296875	17.4646224604917\\
0.77978515625	17.4578512205528\\
0.7802734375	17.4510973634413\\
0.78076171875	17.4443608715327\\
0.78125	17.4376417272629\\
0.78173828125	17.4309399131284\\
0.7822265625	17.4242554116857\\
0.78271484375	17.4175882055513\\
0.783203125	17.4109382774017\\
0.78369140625	17.404305609973\\
0.7841796875	17.3976901860607\\
0.78466796875	17.3910919885198\\
0.78515625	17.3845110002642\\
0.78564453125	17.3779472042671\\
0.7861328125	17.3714005835603\\
0.78662109375	17.3648711212344\\
0.787109375	17.3583588004383\\
0.78759765625	17.3518636043794\\
0.7880859375	17.3453855163231\\
0.78857421875	17.338924519593\\
0.7890625	17.3324805975702\\
0.78955078125	17.3260537336938\\
0.7900390625	17.3196439114602\\
0.79052734375	17.3132511144232\\
0.791015625	17.3068753261937\\
0.79150390625	17.3005165304398\\
0.7919921875	17.2941747108862\\
0.79248046875	17.2878498513146\\
0.79296875	17.281541935563\\
0.79345703125	17.2752509475259\\
0.7939453125	17.2689768711539\\
0.79443359375	17.2627196904539\\
0.794921875	17.2564793894885\\
0.79541015625	17.2502559523762\\
0.7958984375	17.244049363291\\
0.79638671875	17.2378596064625\\
0.796875	17.2316866661754\\
0.79736328125	17.2255305267698\\
0.7978515625	17.2193911726407\\
0.79833984375	17.2132685882378\\
0.798828125	17.2071627580658\\
0.79931640625	17.2010736666836\\
0.7998046875	17.1950012987049\\
0.80029296875	17.1889456387974\\
0.80078125	17.1829066716829\\
0.80126953125	17.1768843821373\\
0.8017578125	17.1708787549903\\
0.80224609375	17.1648897751253\\
0.802734375	17.1589174274791\\
0.80322265625	17.1529616970419\\
0.8037109375	17.1470225688574\\
0.80419921875	17.1411000280222\\
0.8046875	17.1351940596859\\
0.80517578125	17.1293046490508\\
0.8056640625	17.1234317813722\\
0.80615234375	17.1175754419576\\
0.806640625	17.1117356161672\\
0.80712890625	17.1059122894134\\
0.8076171875	17.1001054471605\\
0.80810546875	17.094315074925\\
0.80859375	17.0885411582754\\
0.80908203125	17.0827836828317\\
0.8095703125	17.0770426342655\\
0.81005859375	17.0713179982999\\
0.810546875	17.0656097607094\\
0.81103515625	17.0599179073196\\
0.8115234375	17.0542424240073\\
0.81201171875	17.0485832966999\\
0.8125	17.042940511376\\
0.81298828125	17.0373140540646\\
0.8134765625	17.0317039108453\\
0.81396484375	17.0261100678482\\
0.814453125	17.0205325112536\\
0.81494140625	17.0149712272918\\
0.8154296875	17.0094262022435\\
0.81591796875	17.0038974224389\\
0.81640625	16.9983848742582\\
0.81689453125	16.9928885441312\\
0.8173828125	16.9874084185372\\
0.81787109375	16.9819444840049\\
0.818359375	16.9764967271123\\
0.81884765625	16.9710651344866\\
0.8193359375	16.9656496928037\\
0.81982421875	16.960250388789\\
0.8203125	16.954867209216\\
0.82080078125	16.9495001409075\\
0.8212890625	16.9441491707343\\
0.82177734375	16.9388142856161\\
0.822265625	16.9334954725204\\
0.82275390625	16.9281927184634\\
0.8232421875	16.9229060105089\\
0.82373046875	16.917635335769\\
0.82421875	16.9123806814034\\
0.82470703125	16.9071420346197\\
0.8251953125	16.901919382673\\
0.82568359375	16.8967127128657\\
0.826171875	16.891522012548\\
0.82666015625	16.886347269117\\
0.8271484375	16.8811884700171\\
0.82763671875	16.8760456027396\\
0.828125	16.8709186548228\\
0.82861328125	16.8658076138519\\
0.8291015625	16.8607124674585\\
0.82958984375	16.8556332033211\\
0.830078125	16.8505698091645\\
0.83056640625	16.8455222727599\\
0.8310546875	16.8404905819247\\
0.83154296875	16.8354747245225\\
0.83203125	16.8304746884628\\
0.83251953125	16.8254904617014\\
0.8330078125	16.8205220322394\\
0.83349609375	16.8155693881239\\
0.833984375	16.8106325174477\\
0.83447265625	16.8057114083488\\
0.8349609375	16.8008060490108\\
0.83544921875	16.7959164276626\\
0.8359375	16.7910425325781\\
0.83642578125	16.7861843520764\\
0.8369140625	16.7813418745217\\
0.83740234375	16.7765150883229\\
0.837890625	16.7717039819336\\
0.83837890625	16.7669085438523\\
0.8388671875	16.7621287626218\\
0.83935546875	16.7573646268297\\
0.83984375	16.7526161251076\\
0.84033203125	16.7478832461316\\
0.8408203125	16.7431659786219\\
0.84130859375	16.7384643113427\\
0.841796875	16.7337782331023\\
0.84228515625	16.7291077327529\\
0.8427734375	16.7244527991901\\
0.84326171875	16.7198134213537\\
0.84375	16.7151895882267\\
0.84423828125	16.7105812888357\\
0.8447265625	16.7059885122507\\
0.84521484375	16.701411247585\\
0.845703125	16.6968494839951\\
0.84619140625	16.6923032106804\\
0.8466796875	16.6877724168837\\
0.84716796875	16.6832570918903\\
0.84765625	16.6787572250286\\
0.84814453125	16.6742728056697\\
0.8486328125	16.6698038232271\\
0.84912109375	16.6653502671572\\
0.849609375	16.6609121269586\\
0.85009765625	16.6564893921722\\
0.8505859375	16.6520820523814\\
0.85107421875	16.6476900972118\\
0.8515625	16.6433135163307\\
0.85205078125	16.638952299448\\
0.8525390625	16.6346064363149\\
0.85302734375	16.6302759167248\\
0.853515625	16.6259607305128\\
0.85400390625	16.6216608675556\\
0.8544921875	16.6173763177714\\
0.85498046875	16.61310707112\\
0.85546875	16.6088531176025\\
0.85595703125	16.6046144472614\\
0.8564453125	16.6003910501803\\
0.85693359375	16.5961829164839\\
0.857421875	16.5919900363382\\
0.85791015625	16.58781239995\\
0.8583984375	16.5836499975669\\
0.85888671875	16.5795028194775\\
0.859375	16.575370856011\\
0.85986328125	16.5712540975372\\
0.8603515625	16.5671525344664\\
0.86083984375	16.5630661572497\\
0.861328125	16.5589949563783\\
0.86181640625	16.5549389223837\\
0.8623046875	16.5508980458378\\
0.86279296875	16.5468723173525\\
0.86328125	16.5428617275799\\
0.86376953125	16.538866267212\\
0.8642578125	16.5348859269806\\
0.86474609375	16.5309206976577\\
0.865234375	16.5269705700547\\
0.86572265625	16.5230355350229\\
0.8662109375	16.519115583453\\
0.86669921875	16.5152107062754\\
0.8671875	16.5113208944599\\
0.86767578125	16.5074461390156\\
0.8681640625	16.5035864309911\\
0.86865234375	16.499741761474\\
0.869140625	16.4959121215912\\
0.86962890625	16.4920975025086\\
0.8701171875	16.488297895431\\
0.87060546875	16.4845132916024\\
0.87109375	16.4807436823054\\
0.87158203125	16.4769890588615\\
0.8720703125	16.4732494126309\\
0.87255859375	16.4695247350124\\
0.873046875	16.4658150174434\\
0.87353515625	16.4621202513998\\
0.8740234375	16.458440428396\\
0.87451171875	16.4547755399844\\
0.875	16.4511255777562\\
0.87548828125	16.4474905333405\\
0.8759765625	16.4438703984044\\
0.87646484375	16.4402651646535\\
0.876953125	16.4366748238311\\
0.87744140625	16.4330993677185\\
0.8779296875	16.4295387881349\\
0.87841796875	16.4259930769373\\
0.87890625	16.4224622260203\\
0.87939453125	16.4189462273165\\
0.8798828125	16.4154450727956\\
0.88037109375	16.4119587544654\\
0.880859375	16.4084872643707\\
0.88134765625	16.4050305945939\\
0.8818359375	16.4015887372548\\
0.88232421875	16.3981616845104\\
0.8828125	16.3947494285548\\
0.88330078125	16.3913519616195\\
0.8837890625	16.3879692759728\\
0.88427734375	16.3846013639202\\
0.884765625	16.3812482178041\\
0.88525390625	16.3779098300039\\
0.8857421875	16.3745861929355\\
0.88623046875	16.371277299052\\
0.88671875	16.3679831408428\\
0.88720703125	16.3647037108343\\
0.8876953125	16.3614390015893\\
0.88818359375	16.358189005707\\
0.888671875	16.3549537158234\\
0.88916015625	16.3517331246105\\
0.8896484375	16.3485272247771\\
0.89013671875	16.3453360090678\\
0.890625	16.3421594702638\\
0.89111328125	16.3389976011822\\
0.8916015625	16.3358503946764\\
0.89208984375	16.3327178436358\\
0.892578125	16.3295999409857\\
0.89306640625	16.3264966796874\\
0.8935546875	16.3234080527381\\
0.89404296875	16.3203340531708\\
0.89453125	16.3172746740543\\
0.89501953125	16.314229908493\\
0.8955078125	16.311199749627\\
0.89599609375	16.3081841906321\\
0.896484375	16.3051832247194\\
0.89697265625	16.3021968451358\\
0.8974609375	16.2992250451634\\
0.89794921875	16.2962678181198\\
0.8984375	16.2933251573578\\
0.89892578125	16.2903970562656\\
0.8994140625	16.2874835082665\\
0.89990234375	16.2845845068191\\
0.900390625	16.281700045417\\
0.90087890625	16.2788301175888\\
0.9013671875	16.2759747168984\\
0.90185546875	16.2731338369443\\
0.90234375	16.2703074713602\\
0.90283203125	16.2674956138143\\
0.9033203125	16.26469825801\\
0.90380859375	16.2619153976853\\
0.904296875	16.2591470266127\\
0.90478515625	16.2563931385997\\
0.9052734375	16.2536537274881\\
0.90576171875	16.2509287871545\\
0.90625	16.2482183115097\\
0.90673828125	16.2455222944994\\
0.9072265625	16.2428407301032\\
0.90771484375	16.2401736123355\\
0.908203125	16.2375209352447\\
0.90869140625	16.2348826929137\\
0.9091796875	16.2322588794594\\
0.90966796875	16.229649489033\\
0.91015625	16.2270545158199\\
0.91064453125	16.2244739540393\\
0.9111328125	16.2219077979448\\
0.91162109375	16.2193560418237\\
0.912109375	16.2168186799973\\
0.91259765625	16.2142957068209\\
0.9130859375	16.2117871166836\\
0.91357421875	16.2092929040082\\
0.9140625	16.2068130632515\\
0.91455078125	16.2043475889038\\
0.9150390625	16.2018964754892\\
0.91552734375	16.1994597175652\\
0.916015625	16.1970373097233\\
0.91650390625	16.1946292465883\\
0.9169921875	16.1922355228184\\
0.91748046875	16.1898561331054\\
0.91796875	16.1874910721746\\
0.91845703125	16.1851403347846\\
0.9189453125	16.1828039157272\\
0.91943359375	16.1804818098276\\
0.919921875	16.1781740119445\\
0.92041015625	16.1758805169693\\
0.9208984375	16.1736013198271\\
0.92138671875	16.1713364154758\\
0.921875	16.1690857989065\\
0.92236328125	16.1668494651433\\
0.9228515625	16.1646274092434\\
0.92333984375	16.162419626297\\
0.923828125	16.1602261114271\\
0.92431640625	16.1580468597898\\
0.9248046875	16.1558818665738\\
0.92529296875	16.1537311270008\\
0.92578125	16.1515946363254\\
0.92626953125	16.1494723898347\\
0.9267578125	16.1473643828486\\
0.92724609375	16.1452706107198\\
0.927734375	16.1431910688335\\
0.92822265625	16.1411257526075\\
0.9287109375	16.1390746574923\\
0.92919921875	16.1370377789708\\
0.9296875	16.1350151125585\\
0.93017578125	16.1330066538033\\
0.9306640625	16.1310123982855\\
0.93115234375	16.1290323416179\\
0.931640625	16.1270664794456\\
0.93212890625	16.1251148074459\\
0.9326171875	16.1231773213287\\
0.93310546875	16.1212540168359\\
0.93359375	16.1193448897417\\
0.93408203125	16.1174499358525\\
0.9345703125	16.1155691510068\\
0.93505859375	16.1137025310753\\
0.935546875	16.1118500719609\\
0.93603515625	16.1100117695984\\
0.9365234375	16.1081876199546\\
0.93701171875	16.1063776190284\\
0.9375	16.1045817628507\\
0.93798828125	16.1028000474843\\
0.9384765625	16.1010324690239\\
0.93896484375	16.099279023596\\
0.939453125	16.0975397073591\\
0.93994140625	16.0958145165034\\
0.9404296875	16.094103447251\\
0.94091796875	16.0924064958556\\
0.94140625	16.0907236586029\\
0.94189453125	16.0890549318099\\
0.9423828125	16.0874003118257\\
0.94287109375	16.0857597950308\\
0.943359375	16.0841333778373\\
0.94384765625	16.082521056689\\
0.9443359375	16.0809228280613\\
0.94482421875	16.0793386884609\\
0.9453125	16.0777686344264\\
0.94580078125	16.0762126625274\\
0.9462890625	16.0746707693654\\
0.94677734375	16.073142951573\\
0.947265625	16.0716292058144\\
0.94775390625	16.0701295287851\\
0.9482421875	16.068643917212\\
0.94873046875	16.0671723678532\\
0.94921875	16.0657148774984\\
0.94970703125	16.0642714429682\\
0.9501953125	16.0628420611147\\
0.95068359375	16.0614267288211\\
0.951171875	16.0600254430019\\
0.95166015625	16.0586382006027\\
0.9521484375	16.0572649986004\\
0.95263671875	16.0559058340029\\
0.953125	16.0545607038492\\
};
\addplot [color=red,solid]
  table[row sep=crcr]{0.953125	16.0545607038492\\
0.95361328125	16.0532296052095\\
0.9541015625	16.0519125351849\\
0.95458984375	16.0506094909078\\
0.955078125	16.0493204695413\\
0.95556640625	16.0480454682798\\
0.9560546875	16.0467844843484\\
0.95654296875	16.0455375150035\\
0.95703125	16.0443045575321\\
0.95751953125	16.0430856092524\\
0.9580078125	16.0418806675132\\
0.95849609375	16.0406897296943\\
0.958984375	16.0395127932066\\
0.95947265625	16.0383498554914\\
0.9599609375	16.0372009140211\\
0.96044921875	16.0360659662989\\
0.9609375	16.0349450098586\\
0.96142578125	16.033838042265\\
0.9619140625	16.0327450611133\\
0.96240234375	16.0316660640298\\
0.962890625	16.0306010486713\\
0.96337890625	16.0295500127252\\
0.9638671875	16.0285129539098\\
0.96435546875	16.0274898699739\\
0.96484375	16.026480758697\\
0.96533203125	16.0254856178891\\
0.9658203125	16.0245044453909\\
0.96630859375	16.0235372390736\\
0.966796875	16.0225839968392\\
0.96728515625	16.0216447166199\\
0.9677734375	16.0207193963787\\
0.96826171875	16.0198080341091\\
0.96875	16.0189106278349\\
0.96923828125	16.0180271756105\\
0.9697265625	16.017157675521\\
0.97021484375	16.0163021256816\\
0.970703125	16.0154605242381\\
0.97119140625	16.0146328693667\\
0.9716796875	16.0138191592742\\
0.97216796875	16.0130193921976\\
0.97265625	16.0122335664043\\
0.97314453125	16.011461680192\\
0.9736328125	16.0107037318892\\
0.97412109375	16.0099597198541\\
0.974609375	16.0092296424758\\
0.97509765625	16.0085134981734\\
0.9755859375	16.0078112853965\\
0.97607421875	16.0071230026248\\
0.9765625	16.0064486483684\\
0.97705078125	16.0057882211678\\
0.9775390625	16.0051417195936\\
0.97802734375	16.0045091422468\\
0.978515625	16.0038904877584\\
0.97900390625	16.0032857547898\\
0.9794921875	16.0026949420328\\
0.97998046875	16.0021180482091\\
0.98046875	16.0015550720708\\
0.98095703125	16.0010060124001\\
0.9814453125	16.0004708680095\\
0.98193359375	15.9999496377415\\
0.982421875	15.999442320469\\
0.98291015625	15.9989489150949\\
0.9833984375	15.9984694205522\\
0.98388671875	15.9980038358043\\
0.984375	15.9975521598445\\
0.98486328125	15.9971143916964\\
0.9853515625	15.9966905304134\\
0.98583984375	15.9962805750795\\
0.986328125	15.9958845248084\\
0.98681640625	15.9955023787442\\
0.9873046875	15.9951341360608\\
0.98779296875	15.9947797959623\\
0.98828125	15.9944393576832\\
0.98876953125	15.9941128204875\\
0.9892578125	15.9938001836697\\
0.98974609375	15.9935014465543\\
0.990234375	15.9932166084957\\
0.99072265625	15.9929456688784\\
0.9912109375	15.9926886271172\\
0.99169921875	15.9924454826566\\
0.9921875	15.9922162349713\\
0.99267578125	15.9920008835661\\
0.9931640625	15.9917994279758\\
0.99365234375	15.991611867765\\
0.994140625	15.9914382025288\\
0.99462890625	15.9912784318919\\
0.9951171875	15.9911325555091\\
0.99560546875	15.9910005730654\\
0.99609375	15.9908824842758\\
0.99658203125	15.9907782888851\\
0.9970703125	15.9906879866682\\
0.99755859375	15.9906115774302\\
0.998046875	15.9905490610059\\
0.99853515625	15.9905004372604\\
0.9990234375	15.9904657060886\\
0.99951171875	15.9904448674156\\
};
\addlegendentry{AR(2) Model};

\addplot [color=green,solid,forget plot]
  table[row sep=crcr]{-1	8.7698680233221\\
-0.99951171875	8.76988231160608\\
-0.9990234375	8.7699251764907\\
-0.99853515625	8.76999661807412\\
-0.998046875	8.77009663651996\\
-0.99755859375	8.77022523205717\\
-0.9970703125	8.7703824049803\\
-0.99658203125	8.77056815564921\\
-0.99609375	8.77078248448924\\
-0.99560546875	8.7710253919912\\
-0.9951171875	8.77129687871132\\
-0.99462890625	8.77159694527133\\
-0.994140625	8.77192559235837\\
-0.99365234375	8.77228282072511\\
-0.9931640625	8.77266863118958\\
-0.99267578125	8.77308302463543\\
-0.9921875	8.77352600201171\\
-0.99169921875	8.77399756433296\\
-0.9912109375	8.77449771267927\\
-0.99072265625	8.77502644819624\\
-0.990234375	8.7755837720949\\
-0.98974609375	8.77616968565198\\
-0.9892578125	8.77678419020958\\
-0.98876953125	8.77742728717544\\
-0.98828125	8.77809897802284\\
-0.98779296875	8.77879926429069\\
-0.9873046875	8.7795281475834\\
-0.98681640625	8.78028562957103\\
-0.986328125	8.78107171198925\\
-0.98583984375	8.7818863966394\\
-0.9853515625	8.7827296853884\\
-0.98486328125	8.78360158016881\\
-0.984375	8.78450208297898\\
-0.98388671875	8.78543119588288\\
-0.9833984375	8.78638892101014\\
-0.98291015625	8.78737526055621\\
-0.982421875	8.78839021678223\\
-0.98193359375	8.78943379201509\\
-0.9814453125	8.79050598864753\\
-0.98095703125	8.79160680913802\\
-0.98046875	8.79273625601091\\
-0.97998046875	8.79389433185635\\
-0.9794921875	8.79508103933036\\
-0.97900390625	8.79629638115484\\
-0.978515625	8.79754036011763\\
-0.97802734375	8.79881297907248\\
-0.9775390625	8.80011424093909\\
-0.97705078125	8.80144414870317\\
-0.9765625	8.80280270541636\\
-0.97607421875	8.80418991419638\\
-0.9755859375	8.80560577822705\\
-0.97509765625	8.80705030075817\\
-0.974609375	8.80852348510569\\
-0.97412109375	8.81002533465171\\
-0.9736328125	8.81155585284448\\
-0.97314453125	8.81311504319843\\
-0.97265625	8.81470290929425\\
-0.97216796875	8.81631945477876\\
-0.9716796875	8.81796468336522\\
-0.97119140625	8.81963859883309\\
-0.970703125	8.82134120502819\\
-0.97021484375	8.82307250586273\\
-0.9697265625	8.82483250531532\\
-0.96923828125	8.82662120743101\\
-0.96875	8.82843861632125\\
-0.96826171875	8.83028473616412\\
-0.9677734375	8.83215957120413\\
-0.96728515625	8.83406312575242\\
-0.966796875	8.83599540418669\\
-0.96630859375	8.83795641095135\\
-0.9658203125	8.83994615055743\\
-0.96533203125	8.84196462758273\\
-0.96484375	8.84401184667177\\
-0.96435546875	8.84608781253583\\
-0.9638671875	8.84819252995312\\
-0.96337890625	8.85032600376862\\
-0.962890625	8.85248823889429\\
-0.96240234375	8.85467924030906\\
-0.9619140625	8.85689901305877\\
-0.96142578125	8.85914756225636\\
-0.9609375	8.8614248930818\\
-0.96044921875	8.86373101078224\\
-0.9599609375	8.86606592067196\\
-0.95947265625	8.86842962813243\\
-0.958984375	8.8708221386124\\
-0.95849609375	8.87324345762795\\
-0.9580078125	8.87569359076242\\
-0.95751953125	8.87817254366662\\
-0.95703125	8.88068032205872\\
-0.95654296875	8.88321693172444\\
-0.9560546875	8.88578237851698\\
-0.95556640625	8.88837666835716\\
-0.955078125	8.89099980723342\\
-0.95458984375	8.89365180120186\\
-0.9541015625	8.89633265638633\\
-0.95361328125	8.89904237897842\\
-0.953125	8.9017809752376\\
-0.95263671875	8.90454845149122\\
-0.9521484375	8.90734481413451\\
-0.95166015625	8.91017006963074\\
-0.951171875	8.91302422451121\\
-0.95068359375	8.91590728537533\\
-0.9501953125	8.91881925889063\\
-0.94970703125	8.92176015179284\\
-0.94921875	8.92472997088599\\
-0.94873046875	8.92772872304244\\
-0.9482421875	8.93075641520285\\
-0.94775390625	8.93381305437638\\
-0.947265625	8.93689864764069\\
-0.94677734375	8.94001320214194\\
-0.9462890625	8.9431567250949\\
-0.94580078125	8.94632922378312\\
-0.9453125	8.94953070555872\\
-0.94482421875	8.95276117784272\\
-0.9443359375	8.95602064812498\\
-0.94384765625	8.95930912396424\\
-0.943359375	8.96262661298825\\
-0.94287109375	8.96597312289381\\
-0.9423828125	8.96934866144681\\
-0.94189453125	8.97275323648232\\
-0.94140625	8.97618685590465\\
-0.94091796875	8.97964952768739\\
-0.9404296875	8.98314125987356\\
-0.93994140625	8.98666206057557\\
-0.939453125	8.99021193797536\\
-0.93896484375	8.99379090032442\\
-0.9384765625	8.99739895594389\\
-0.93798828125	9.00103611322468\\
-0.9375	9.00470238062735\\
-0.93701171875	9.00839776668251\\
-0.9365234375	9.01212227999055\\
-0.93603515625	9.01587592922187\\
-0.935546875	9.01965872311699\\
-0.93505859375	9.02347067048657\\
-0.9345703125	9.02731178021146\\
-0.93408203125	9.03118206124286\\
-0.93359375	9.03508152260224\\
-0.93310546875	9.03901017338159\\
-0.9326171875	9.04296802274344\\
-0.93212890625	9.04695507992087\\
-0.931640625	9.05097135421764\\
-0.93115234375	9.05501685500831\\
-0.9306640625	9.0590915917382\\
-0.93017578125	9.06319557392362\\
-0.9296875	9.06732881115181\\
-0.92919921875	9.07149131308115\\
-0.9287109375	9.07568308944114\\
-0.92822265625	9.07990415003247\\
-0.927734375	9.08415450472723\\
-0.92724609375	9.08843416346892\\
-0.9267578125	9.09274313627246\\
-0.92626953125	9.09708143322439\\
-0.92578125	9.10144906448289\\
-0.92529296875	9.10584604027795\\
-0.9248046875	9.11027237091127\\
-0.92431640625	9.11472806675656\\
-0.923828125	9.11921313825954\\
-0.92333984375	9.12372759593799\\
-0.9228515625	9.12827145038184\\
-0.92236328125	9.13284471225341\\
-0.921875	9.13744739228728\\
-0.92138671875	9.14207950129055\\
-0.9208984375	9.14674105014285\\
-0.92041015625	9.15143204979642\\
-0.919921875	9.15615251127629\\
-0.91943359375	9.16090244568028\\
-0.9189453125	9.16568186417915\\
-0.91845703125	9.1704907780167\\
-0.91796875	9.17532919850978\\
-0.91748046875	9.18019713704853\\
-0.9169921875	9.18509460509633\\
-0.91650390625	9.19002161419004\\
-0.916015625	9.19497817593998\\
-0.91552734375	9.19996430203003\\
-0.9150390625	9.20498000421789\\
-0.91455078125	9.21002529433494\\
-0.9140625	9.2151001842866\\
-0.91357421875	9.22020468605219\\
-0.9130859375	9.22533881168518\\
-0.91259765625	9.23050257331324\\
-0.912109375	9.23569598313843\\
-0.91162109375	9.24091905343712\\
-0.9111328125	9.24617179656032\\
-0.91064453125	9.25145422493361\\
-0.91015625	9.25676635105732\\
-0.90966796875	9.26210818750664\\
-0.9091796875	9.26747974693174\\
-0.90869140625	9.27288104205782\\
-0.908203125	9.27831208568527\\
-0.90771484375	9.28377289068979\\
-0.9072265625	9.28926347002249\\
-0.90673828125	9.29478383670992\\
-0.90625	9.30033400385434\\
-0.90576171875	9.30591398463369\\
-0.9052734375	9.31152379230178\\
-0.90478515625	9.31716344018838\\
-0.904296875	9.32283294169938\\
-0.90380859375	9.32853231031681\\
-0.9033203125	9.33426155959905\\
-0.90283203125	9.3400207031809\\
-0.90234375	9.34580975477373\\
-0.90185546875	9.35162872816555\\
-0.9013671875	9.35747763722119\\
-0.90087890625	9.3633564958824\\
-0.900390625	9.3692653181679\\
-0.89990234375	9.37520411817362\\
-0.8994140625	9.38117291007278\\
-0.89892578125	9.38717170811594\\
-0.8984375	9.39320052663123\\
-0.89794921875	9.39925938002444\\
-0.8974609375	9.40534828277911\\
-0.89697265625	9.4114672494567\\
-0.896484375	9.41761629469666\\
-0.89599609375	9.42379543321665\\
-0.8955078125	9.43000467981263\\
-0.89501953125	9.4362440493589\\
-0.89453125	9.44251355680839\\
-0.89404296875	9.44881321719265\\
-0.8935546875	9.4551430456221\\
-0.89306640625	9.46150305728603\\
-0.892578125	9.46789326745288\\
-0.89208984375	9.47431369147027\\
-0.8916015625	9.48076434476515\\
-0.89111328125	9.48724524284398\\
-0.890625	9.49375640129287\\
-0.89013671875	9.50029783577757\\
-0.8896484375	9.5068695620439\\
-0.88916015625	9.51347159591755\\
-0.888671875	9.52010395330451\\
-0.88818359375	9.52676665019102\\
-0.8876953125	9.53345970264379\\
-0.88720703125	9.54018312681015\\
-0.88671875	9.54693693891813\\
-0.88623046875	9.55372115527668\\
-0.8857421875	9.56053579227577\\
-0.88525390625	9.56738086638658\\
-0.884765625	9.5742563941616\\
-0.88427734375	9.58116239223472\\
-0.8837890625	9.58809887732161\\
-0.88330078125	9.59506586621951\\
-0.8828125	9.60206337580775\\
-0.88232421875	9.60909142304762\\
-0.8818359375	9.61615002498269\\
-0.88134765625	9.6232391987389\\
-0.880859375	9.63035896152467\\
-0.88037109375	9.63750933063113\\
-0.8798828125	9.64469032343226\\
-0.87939453125	9.65190195738505\\
-0.87890625	9.65914425002954\\
-0.87841796875	9.6664172189892\\
-0.8779296875	9.67372088197088\\
-0.87744140625	9.68105525676511\\
-0.876953125	9.68842036124611\\
-0.87646484375	9.69581621337217\\
-0.8759765625	9.70324283118559\\
-0.87548828125	9.710700232813\\
-0.875	9.71818843646541\\
-0.87451171875	9.72570746043845\\
-0.8740234375	9.73325732311251\\
-0.87353515625	9.74083804295295\\
-0.873046875	9.74844963851018\\
-0.87255859375	9.75609212841987\\
-0.8720703125	9.76376553140309\\
-0.87158203125	9.77146986626664\\
-0.87109375	9.77920515190295\\
-0.87060546875	9.7869714072905\\
-0.8701171875	9.79476865149386\\
-0.86962890625	9.80259690366382\\
-0.869140625	9.81045618303771\\
-0.86865234375	9.81834650893953\\
-0.8681640625	9.82626790078004\\
-0.86767578125	9.83422037805701\\
-0.8671875	9.84220396035535\\
-0.86669921875	9.8502186673474\\
-0.8662109375	9.85826451879295\\
-0.86572265625	9.86634153453959\\
-0.865234375	9.87444973452278\\
-0.86474609375	9.88258913876595\\
-0.8642578125	9.89075976738096\\
-0.86376953125	9.89896164056802\\
-0.86328125	9.90719477861599\\
-0.86279296875	9.91545920190257\\
-0.8623046875	9.92375493089447\\
-0.86181640625	9.93208198614755\\
-0.861328125	9.94044038830711\\
-0.86083984375	9.94883015810804\\
-0.8603515625	9.95725131637491\\
-0.85986328125	9.96570388402237\\
-0.859375	9.97418788205513\\
-0.85888671875	9.98270333156834\\
-0.8583984375	9.99125025374762\\
-0.85791015625	9.99982866986937\\
-0.857421875	10.0084386013009\\
-0.85693359375	10.0170800695007\\
-0.8564453125	10.0257530960187\\
-0.85595703125	10.0344577024961\\
-0.85546875	10.043193910666\\
-0.85498046875	10.0519617423537\\
-0.8544921875	10.0607612194761\\
-0.85400390625	10.0695923640429\\
-0.853515625	10.0784551981562\\
-0.85302734375	10.0873497440109\\
-0.8525390625	10.0962760238949\\
-0.85205078125	10.1052340601892\\
-0.8515625	10.1142238753683\\
-0.85107421875	10.1232454920002\\
-0.8505859375	10.1322989327468\\
-0.85009765625	10.1413842203641\\
-0.849609375	10.1505013777021\\
-0.84912109375	10.1596504277056\\
-0.8486328125	10.1688313934138\\
-0.84814453125	10.178044297961\\
-0.84765625	10.1872891645765\\
-0.84716796875	10.1965660165849\\
-0.8466796875	10.2058748774066\\
-0.84619140625	10.2152157705574\\
-0.845703125	10.2245887196495\\
-0.84521484375	10.2339937483911\\
-0.8447265625	10.2434308805869\\
-0.84423828125	10.2529001401383\\
-0.84375	10.2624015510435\\
-0.84326171875	10.2719351373981\\
-0.8427734375	10.2815009233949\\
-0.84228515625	10.2910989333243\\
-0.841796875	10.3007291915745\\
-0.84130859375	10.310391722632\\
-0.8408203125	10.3200865510812\\
-0.84033203125	10.3298137016054\\
-0.83984375	10.3395731989865\\
-0.83935546875	10.3493650681056\\
-0.8388671875	10.3591893339428\\
-0.83837890625	10.3690460215778\\
-0.837890625	10.3789351561901\\
-0.83740234375	10.3888567630592\\
-0.8369140625	10.3988108675647\\
-0.83642578125	10.4087974951868\\
-0.8359375	10.4188166715062\\
-0.83544921875	10.428868422205\\
-0.8349609375	10.4389527730661\\
-0.83447265625	10.449069749974\\
-0.833984375	10.459219378915\\
-0.83349609375	10.4694016859772\\
-0.8330078125	10.4796166973512\\
-0.83251953125	10.4898644393298\\
-0.83203125	10.5001449383088\\
-0.83154296875	10.5104582207867\\
-0.8310546875	10.5208043133655\\
-0.83056640625	10.5311832427506\\
-0.830078125	10.5415950357513\\
-0.82958984375	10.5520397192808\\
-0.8291015625	10.5625173203568\\
-0.82861328125	10.5730278661012\\
-0.828125	10.5835713837412\\
-0.82763671875	10.5941479006088\\
-0.8271484375	10.6047574441416\\
-0.82666015625	10.6154000418826\\
-0.826171875	10.6260757214809\\
-0.82568359375	10.6367845106919\\
-0.8251953125	10.6475264373771\\
-0.82470703125	10.6583015295051\\
-0.82421875	10.6691098151515\\
-0.82373046875	10.6799513224989\\
-0.8232421875	10.690826079838\\
-0.82275390625	10.7017341155668\\
-0.822265625	10.7126754581919\\
-0.82177734375	10.7236501363281\\
-0.8212890625	10.734658178699\\
-0.82080078125	10.7456996141372\\
-0.8203125	10.7567744715848\\
-0.81982421875	10.767882780093\\
-0.8193359375	10.7790245688234\\
-0.81884765625	10.7901998670475\\
-0.818359375	10.8014087041473\\
-0.81787109375	10.8126511096158\\
-0.8173828125	10.8239271130566\\
-0.81689453125	10.8352367441851\\
-0.81640625	10.8465800328281\\
-0.81591796875	10.8579570089247\\
-0.8154296875	10.8693677025258\\
-0.81494140625	10.8808121437952\\
-0.814453125	10.8922903630096\\
-0.81396484375	10.9038023905587\\
-0.8134765625	10.9153482569459\\
-0.81298828125	10.9269279927882\\
-0.8125	10.9385416288168\\
-0.81201171875	10.9501891958775\\
-0.8115234375	10.9618707249307\\
-0.81103515625	10.9735862470518\\
-0.810546875	10.9853357934319\\
-0.81005859375	10.9971193953774\\
-0.8095703125	11.0089370843109\\
-0.80908203125	11.0207888917716\\
-0.80859375	11.0326748494151\\
-0.80810546875	11.0445949890139\\
-0.8076171875	11.0565493424582\\
-0.80712890625	11.0685379417557\\
-0.806640625	11.0805608190319\\
-0.80615234375	11.0926180065309\\
-0.8056640625	11.1047095366153\\
-0.80517578125	11.1168354417669\\
-0.8046875	11.1289957545866\\
-0.80419921875	11.1411905077951\\
-0.8037109375	11.1534197342332\\
-0.80322265625	11.1656834668617\\
-0.802734375	11.1779817387627\\
-0.80224609375	11.1903145831387\\
-0.8017578125	11.2026820333141\\
-0.80126953125	11.2150841227349\\
-0.80078125	11.2275208849691\\
-0.80029296875	11.2399923537072\\
-0.7998046875	11.2524985627626\\
-0.79931640625	11.2650395460717\\
-0.798828125	11.2776153376946\\
-0.79833984375	11.2902259718152\\
-0.7978515625	11.3028714827416\\
-0.79736328125	11.3155519049067\\
-0.796875	11.328267272868\\
-0.79638671875	11.3410176213087\\
-0.7958984375	11.3538029850377\\
-0.79541015625	11.3666233989896\\
-0.794921875	11.3794788982259\\
-0.79443359375	11.3923695179347\\
-0.7939453125	11.4052952934313\\
-0.79345703125	11.4182562601587\\
-0.79296875	11.4312524536877\\
-0.79248046875	11.4442839097176\\
-0.7919921875	11.4573506640763\\
-0.79150390625	11.4704527527209\\
-0.791015625	11.4835902117379\\
-0.79052734375	11.4967630773438\\
-0.7900390625	11.5099713858853\\
-0.78955078125	11.5232151738399\\
-0.7890625	11.5364944778158\\
-0.78857421875	11.5498093345532\\
-0.7880859375	11.5631597809238\\
-0.78759765625	11.5765458539316\\
-0.787109375	11.5899675907133\\
-0.78662109375	11.6034250285388\\
-0.7861328125	11.6169182048112\\
-0.78564453125	11.6304471570677\\
-0.78515625	11.6440119229798\\
-0.78466796875	11.6576125403536\\
-0.7841796875	11.6712490471303\\
-0.78369140625	11.6849214813868\\
-0.783203125	11.6986298813359\\
-0.78271484375	11.7123742853266\\
-0.7822265625	11.7261547318449\\
-0.78173828125	11.739971259514\\
-0.78125	11.7538239070947\\
-0.78076171875	11.7677127134859\\
-0.7802734375	11.781637717725\\
-0.77978515625	11.7955989589884\\
-0.779296875	11.8095964765918\\
-0.77880859375	11.8236303099908\\
-0.7783203125	11.8377004987811\\
-0.77783203125	11.8518070826994\\
-0.77734375	11.8659501016232\\
-0.77685546875	11.8801295955719\\
-0.7763671875	11.8943456047067\\
-0.77587890625	11.9085981693314\\
-0.775390625	11.9228873298928\\
-0.77490234375	11.937213126981\\
-0.7744140625	11.9515756013298\\
-0.77392578125	11.9659747938177\\
-0.7734375	11.9804107454678\\
-0.77294921875	11.9948834974482\\
-0.7724609375	12.0093930910731\\
-0.77197265625	12.0239395678026\\
-0.771484375	12.0385229692437\\
-0.77099609375	12.0531433371501\\
-0.7705078125	12.0678007134236\\
-0.77001953125	12.0824951401137\\
-0.76953125	12.0972266594186\\
-0.76904296875	12.1119953136855\\
-0.7685546875	12.1268011454112\\
-0.76806640625	12.1416441972424\\
-0.767578125	12.1565245119764\\
-0.76708984375	12.1714421325615\\
-0.7666015625	12.1863971020973\\
-0.76611328125	12.2013894638356\\
-0.765625	12.2164192611805\\
-0.76513671875	12.2314865376893\\
-0.7646484375	12.2465913370727\\
-0.76416015625	12.2617337031952\\
-0.763671875	12.2769136800761\\
-0.76318359375	12.2921313118897\\
-0.7626953125	12.3073866429656\\
-0.76220703125	12.3226797177896\\
-0.76171875	12.338010581004\\
-0.76123046875	12.3533792774084\\
-0.7607421875	12.3687858519598\\
-0.76025390625	12.3842303497735\\
-0.759765625	12.3997128161233\\
-0.75927734375	12.4152332964424\\
-0.7587890625	12.4307918363236\\
-0.75830078125	12.4463884815201\\
-0.7578125	12.4620232779458\\
-0.75732421875	12.4776962716762\\
-0.7568359375	12.4934075089484\\
-0.75634765625	12.5091570361621\\
-0.755859375	12.5249448998803\\
-0.75537109375	12.540771146829\\
-0.7548828125	12.5566358238989\\
-0.75439453125	12.5725389781451\\
-0.75390625	12.5884806567879\\
-0.75341796875	12.6044609072135\\
-0.7529296875	12.6204797769747\\
-0.75244140625	12.6365373137909\\
-0.751953125	12.6526335655492\\
-0.75146484375	12.668768580305\\
-0.7509765625	12.684942406282\\
-0.75048828125	12.7011550918735\\
-0.75	12.7174066856425\\
-0.74951171875	12.7336972363226\\
-0.7490234375	12.7500267928182\\
-0.74853515625	12.7663954042056\\
-0.748046875	12.7828031197332\\
-0.74755859375	12.7992499888222\\
-0.7470703125	12.8157360610674\\
-0.74658203125	12.8322613862375\\
-0.74609375	12.848826014276\\
-0.74560546875	12.8654299953015\\
-0.7451171875	12.8820733796086\\
-0.74462890625	12.8987562176685\\
-0.744140625	12.9154785601293\\
-0.74365234375	12.932240457817\\
-0.7431640625	12.949041961736\\
-0.74267578125	12.9658831230698\\
-0.7421875	12.9827639931813\\
-0.74169921875	12.9996846236139\\
-0.7412109375	13.016645066092\\
-0.74072265625	13.0336453725213\\
-0.740234375	13.05068559499\\
-0.73974609375	13.0677657857691\\
-0.7392578125	13.084885997313\\
-0.73876953125	13.1020462822606\\
-0.73828125	13.1192466934354\\
-0.73779296875	13.1364872838465\\
-0.7373046875	13.1537681066893\\
-0.73681640625	13.1710892153458\\
-0.736328125	13.1884506633858\\
-0.73583984375	13.2058525045671\\
-0.7353515625	13.2232947928368\\
-0.73486328125	13.2407775823312\\
-0.734375	13.2583009273769\\
-0.73388671875	13.2758648824917\\
-0.7333984375	13.2934695023849\\
-0.73291015625	13.3111148419582\\
-0.732421875	13.3288009563064\\
-0.73193359375	13.346527900718\\
-0.7314453125	13.3642957306762\\
-0.73095703125	13.3821045018592\\
-0.73046875	13.3999542701413\\
-0.72998046875	13.417845091593\\
-0.7294921875	13.435777022483\\
-0.72900390625	13.4537501192772\\
-0.728515625	13.471764438641\\
-0.72802734375	13.489820037439\\
-0.7275390625	13.5079169727364\\
-0.72705078125	13.5260553017992\\
-0.7265625	13.5442350820953\\
-0.72607421875	13.5624563712953\\
-0.7255859375	13.5807192272728\\
-0.72509765625	13.5990237081059\\
-0.724609375	13.6173698720773\\
-0.72412109375	13.6357577776752\\
-0.7236328125	13.6541874835944\\
-0.72314453125	13.6726590487369\\
-0.72265625	13.6911725322124\\
-0.72216796875	13.7097279933394\\
-0.7216796875	13.7283254916462\\
-0.72119140625	13.7469650868711\\
-0.720703125	13.7656468389635\\
-0.72021484375	13.7843708080849\\
-0.7197265625	13.8031370546094\\
-0.71923828125	13.8219456391247\\
-0.71875	13.8407966224327\\
-0.71826171875	13.8596900655507\\
-0.7177734375	13.8786260297116\\
-0.71728515625	13.8976045763655\\
-0.716796875	13.9166257671799\\
-0.71630859375	13.9356896640408\\
-0.7158203125	13.9547963290537\\
-0.71533203125	13.973945824544\\
-0.71484375	13.9931382130583\\
-0.71435546875	14.0123735573648\\
-0.7138671875	14.0316519204548\\
-0.71337890625	14.0509733655429\\
-0.712890625	14.0703379560683\\
-0.71240234375	14.0897457556954\\
-0.7119140625	14.1091968283148\\
-0.71142578125	14.1286912380443\\
-0.7109375	14.1482290492296\\
-0.71044921875	14.1678103264451\\
-0.7099609375	14.1874351344954\\
-0.70947265625	14.2071035384151\\
-0.708984375	14.2268156034709\\
-0.70849609375	14.2465713951617\\
-0.7080078125	14.26637097922\\
-0.70751953125	14.2862144216122\\
-0.70703125	14.3061017885402\\
-0.70654296875	14.3260331464422\\
-0.7060546875	14.3460085619931\\
-0.70556640625	14.3660281021062\\
-0.705078125	14.3860918339334\\
-0.70458984375	14.4061998248669\\
-0.7041015625	14.4263521425395\\
-0.70361328125	14.4465488548258\\
-0.703125	14.4667900298435\\
-0.70263671875	14.4870757359539\\
-0.7021484375	14.5074060417628\\
-0.70166015625	14.5277810161222\\
-0.701171875	14.5482007281304\\
-0.70068359375	14.5686652471336\\
-0.7001953125	14.5891746427266\\
-0.69970703125	14.609728984754\\
-0.69921875	14.6303283433111\\
-0.69873046875	14.6509727887449\\
-0.6982421875	14.671662391655\\
-0.69775390625	14.6923972228949\\
-0.697265625	14.7131773535729\\
-0.69677734375	14.7340028550532\\
-0.6962890625	14.7548737989565\\
-0.69580078125	14.7757902571618\\
-0.6953125	14.796752301807\\
-0.69482421875	14.8177600052898\\
-0.6943359375	14.8388134402693\\
-0.69384765625	14.8599126796665\\
-0.693359375	14.8810577966657\\
-0.69287109375	14.9022488647155\\
-0.6923828125	14.9234859575301\\
-0.69189453125	14.9447691490899\\
-0.69140625	14.966098513643\\
-0.69091796875	14.9874741257063\\
-0.6904296875	15.0088960600663\\
-0.68994140625	15.0303643917806\\
-0.689453125	15.0518791961788\\
-0.68896484375	15.0734405488636\\
-0.6884765625	15.0950485257122\\
-0.68798828125	15.116703202877\\
-0.6875	15.1384046567872\\
-0.68701171875	15.1601529641497\\
-0.6865234375	15.1819482019504\\
-0.68603515625	15.203790447455\\
-0.685546875	15.2256797782108\\
-0.68505859375	15.2476162720474\\
-0.6845703125	15.269600007078\\
-0.68408203125	15.2916310617007\\
-0.68359375	15.3137095145996\\
-0.68310546875	15.3358354447459\\
-0.6826171875	15.3580089313995\\
-0.68212890625	15.3802300541099\\
-0.681640625	15.4024988927172\\
-0.68115234375	15.424815527354\\
-0.6806640625	15.4471800384462\\
-0.68017578125	15.469592506714\\
-0.6796875	15.492053013174\\
-0.67919921875	15.5145616391396\\
-0.6787109375	15.5371184662225\\
-0.67822265625	15.5597235763344\\
-0.677734375	15.5823770516878\\
-0.67724609375	15.6050789747975\\
-0.6767578125	15.6278294284818\\
-0.67626953125	15.6506284958638\\
-0.67578125	15.6734762603731\\
-0.67529296875	15.6963728057464\\
-0.6748046875	15.7193182160295\\
-0.67431640625	15.7423125755781\\
-0.673828125	15.7653559690598\\
-0.67333984375	15.7884484814547\\
-0.6728515625	15.8115901980572\\
-0.67236328125	15.8347812044775\\
-0.671875	15.8580215866423\\
-0.67138671875	15.881311430797\\
-0.6708984375	15.9046508235067\\
-0.67041015625	15.9280398516576\\
-0.669921875	15.9514786024583\\
-0.66943359375	15.9749671634414\\
-0.6689453125	15.998505622465\\
-0.66845703125	16.0220940677139\\
-0.66796875	16.045732587701\\
-0.66748046875	16.0694212712692\\
-0.6669921875	16.0931602075921\\
-0.66650390625	16.1169494861764\\
-0.666015625	16.1407891968625\\
-0.66552734375	16.1646794298264\\
-0.6650390625	16.1886202755814\\
-0.66455078125	16.212611824979\\
-0.6640625	16.2366541692109\\
-0.66357421875	16.2607473998104\\
-0.6630859375	16.2848916086537\\
-0.66259765625	16.309086887962\\
-0.662109375	16.3333333303023\\
-0.66162109375	16.3576310285893\\
-0.6611328125	16.3819800760871\\
-0.66064453125	16.4063805664108\\
-0.66015625	16.4308325935275\\
-0.65966796875	16.4553362517586\\
-0.6591796875	16.4798916357811\\
-0.65869140625	16.5044988406289\\
-0.658203125	16.5291579616952\\
-0.65771484375	16.5538690947331\\
-0.6572265625	16.5786323358583\\
-0.65673828125	16.6034477815497\\
-0.65625	16.6283155286521\\
-0.65576171875	16.6532356743768\\
-0.6552734375	16.6782083163042\\
-0.65478515625	16.7032335523849\\
-0.654296875	16.7283114809417\\
-0.65380859375	16.753442200671\\
-0.6533203125	16.7786258106449\\
-0.65283203125	16.8038624103126\\
-0.65234375	16.8291520995021\\
-0.65185546875	16.8544949784226\\
-0.6513671875	16.8798911476651\\
-0.65087890625	16.9053407082053\\
-0.650390625	16.9308437614045\\
-0.64990234375	16.9564004090122\\
-0.6494140625	16.982010753167\\
-0.64892578125	17.0076748963993\\
-0.6484375	17.0333929416324\\
-0.64794921875	17.0591649921847\\
-0.6474609375	17.0849911517714\\
-0.64697265625	17.1108715245066\\
-0.646484375	17.1368062149046\\
-0.64599609375	17.1627953278826\\
-0.6455078125	17.1888389687617\\
-0.64501953125	17.2149372432695\\
-0.64453125	17.2410902575416\\
-0.64404296875	17.2672981181235\\
-0.6435546875	17.293560931973\\
-0.64306640625	17.3198788064616\\
-0.642578125	17.3462518493765\\
-0.64208984375	17.3726801689231\\
-0.6416015625	17.3991638737263\\
-0.64111328125	17.4257030728329\\
-0.640625	17.4522978757136\\
-0.64013671875	17.4789483922646\\
-0.6396484375	17.5056547328103\\
-0.63916015625	17.5324170081045\\
-0.638671875	17.5592353293334\\
-0.63818359375	17.5861098081169\\
-0.6376953125	17.613040556511\\
-0.63720703125	17.6400276870097\\
-0.63671875	17.6670713125476\\
-0.63623046875	17.6941715465013\\
-0.6357421875	17.7213285026919\\
-0.63525390625	17.7485422953873\\
-0.634765625	17.7758130393041\\
-0.63427734375	17.8031408496096\\
-0.6337890625	17.8305258419246\\
-0.63330078125	17.8579681323249\\
-0.6328125	17.8854678373439\\
-0.63232421875	17.9130250739748\\
-0.6318359375	17.9406399596726\\
-0.63134765625	17.9683126123565\\
-0.630859375	17.9960431504123\\
-0.63037109375	18.0238316926946\\
-0.6298828125	18.0516783585287\\
-0.62939453125	18.0795832677135\\
-0.62890625	18.1075465405236\\
-0.62841796875	18.1355682977112\\
-0.6279296875	18.1636486605092\\
-0.62744140625	18.1917877506331\\
-0.626953125	18.2199856902834\\
-0.62646484375	18.2482426021479\\
-0.6259765625	18.2765586094046\\
-0.62548828125	18.3049338357234\\
-0.625	18.3333684052693\\
-0.62451171875	18.3618624427041\\
-0.6240234375	18.3904160731894\\
-0.62353515625	18.4190294223889\\
-0.623046875	18.4477026164709\\
-0.62255859375	18.4764357821108\\
-0.6220703125	18.5052290464936\\
-0.62158203125	18.5340825373166\\
-0.62109375	18.5629963827918\\
-0.62060546875	18.5919707116485\\
-0.6201171875	18.6210056531359\\
-0.61962890625	18.6501013370258\\
-0.619140625	18.6792578936152\\
-0.61865234375	18.7084754537287\\
-0.6181640625	18.7377541487216\\
-0.61767578125	18.7670941104823\\
-0.6171875	18.7964954714347\\
-0.61669921875	18.8259583645418\\
-0.6162109375	18.8554829233074\\
-0.61572265625	18.8850692817796\\
-0.615234375	18.9147175745532\\
-0.61474609375	18.9444279367724\\
-0.6142578125	18.9742005041339\\
-0.61376953125	19.0040354128898\\
-0.61328125	19.0339327998497\\
-0.61279296875	19.0638928023846\\
-0.6123046875	19.0939155584289\\
-0.61181640625	19.1240012064836\\
-0.611328125	19.1541498856196\\
-0.61083984375	19.1843617354798\\
-0.6103515625	19.2146368962829\\
-0.60986328125	19.2449755088258\\
-0.609375	19.2753777144866\\
-0.60888671875	19.3058436552281\\
-0.6083984375	19.3363734736003\\
-0.60791015625	19.3669673127436\\
-0.607421875	19.3976253163921\\
-0.60693359375	19.4283476288761\\
-0.6064453125	19.4591343951261\\
-0.60595703125	19.4899857606749\\
-0.60546875	19.5209018716615\\
-0.60498046875	19.5518828748341\\
-0.6044921875	19.582928917553\\
-0.60400390625	19.6140401477941\\
-0.603515625	19.6452167141521\\
-0.60302734375	19.6764587658436\\
-0.6025390625	19.7077664527103\\
-0.60205078125	19.7391399252228\\
-0.6015625	19.7705793344832\\
-0.60107421875	19.802084832229\\
-0.6005859375	19.8336565708363\\
-0.60009765625	19.8652947033229\\
-0.599609375	19.896999383352\\
-0.59912109375	19.9287707652356\\
-0.5986328125	19.9606090039381\\
-0.59814453125	19.9925142550791\\
-0.59765625	20.0244866749378\\
-0.59716796875	20.0565264204558\\
-0.5966796875	20.088633649241\\
-0.59619140625	20.1208085195711\\
-0.595703125	20.1530511903972\\
-0.59521484375	20.1853618213472\\
-0.5947265625	20.2177405727298\\
-0.59423828125	20.250187605538\\
-0.59375	20.2827030814524\\
-0.59326171875	20.3152871628457\\
-0.5927734375	20.3479400127858\\
-0.59228515625	20.3806617950396\\
-0.591796875	20.4134526740771\\
-0.59130859375	20.4463128150752\\
-0.5908203125	20.4792423839209\\
-0.59033203125	20.5122415472162\\
-0.58984375	20.545310472281\\
-0.58935546875	20.5784493271576\\
-0.5888671875	20.6116582806145\\
-0.58837890625	20.6449375021503\\
-0.587890625	20.6782871619976\\
-0.58740234375	20.7117074311275\\
-0.5869140625	20.745198481253\\
-0.58642578125	20.7787604848334\\
-0.5859375	20.8123936150785\\
-0.58544921875	20.8460980459525\\
-0.5849609375	20.8798739521783\\
-0.58447265625	20.9137215092417\\
-0.583984375	20.9476408933955\\
-0.58349609375	20.9816322816639\\
-0.5830078125	21.0156958518465\\
-0.58251953125	21.049831782523\\
-0.58203125	21.0840402530573\\
-0.58154296875	21.1183214436017\\
-0.5810546875	21.1526755351016\\
-0.58056640625	21.1871027092999\\
-0.580078125	21.2216031487411\\
-0.57958984375	21.2561770367762\\
-0.5791015625	21.2908245575671\\
-0.57861328125	21.3255458960909\\
-0.578125	21.3603412381449\\
-0.57763671875	21.3952107703507\\
-0.5771484375	21.4301546801594\\
-0.57666015625	21.4651731558558\\
-0.576171875	21.5002663865633\\
-0.57568359375	21.535434562249\\
-0.5751953125	21.5706778737274\\
-0.57470703125	21.6059965126666\\
-0.57421875	21.6413906715921\\
-0.57373046875	21.676860543892\\
-0.5732421875	21.7124063238219\\
-0.57275390625	21.7480282065103\\
-0.572265625	21.7837263879625\\
-0.57177734375	21.8195010650667\\
-0.5712890625	21.8553524355984\\
-0.57080078125	21.8912806982257\\
-0.5703125	21.9272860525144\\
-0.56982421875	21.9633686989331\\
-0.5693359375	21.9995288388585\\
-0.56884765625	22.0357666745805\\
-0.568359375	22.0720824093075\\
-0.56787109375	22.1084762471716\\
-0.5673828125	22.1449483932344\\
-0.56689453125	22.1814990534916\\
-0.56640625	22.218128434879\\
-0.56591796875	22.2548367452778\\
-0.5654296875	22.29162419352\\
-0.56494140625	22.3284909893942\\
-0.564453125	22.3654373436505\\
-0.56396484375	22.402463468007\\
-0.5634765625	22.4395695751549\\
-0.56298828125	22.476755878764\\
-0.5625	22.5140225934892\\
-0.56201171875	22.5513699349755\\
-0.5615234375	22.588798119864\\
-0.56103515625	22.6263073657981\\
-0.560546875	22.6638978914289\\
-0.56005859375	22.7015699164216\\
-0.5595703125	22.7393236614611\\
-0.55908203125	22.777159348258\\
-0.55859375	22.8150771995553\\
-0.55810546875	22.8530774391335\\
-0.5576171875	22.8911602918177\\
-0.55712890625	22.9293259834831\\
-0.556640625	22.9675747410616\\
-0.55615234375	23.0059067925481\\
-0.5556640625	23.0443223670064\\
-0.55517578125	23.0828216945764\\
-0.5546875	23.1214050064797\\
-0.55419921875	23.1600725350262\\
-0.5537109375	23.1988245136213\\
-0.55322265625	23.2376611767716\\
-0.552734375	23.2765827600919\\
-0.55224609375	23.315589500312\\
-0.5517578125	23.354681635283\\
-0.55126953125	23.3938594039845\\
-0.55078125	23.4331230465309\\
-0.55029296875	23.472472804179\\
-0.5498046875	23.5119089193339\\
-0.54931640625	23.5514316355568\\
-0.548828125	23.5910411975717\\
-0.54833984375	23.6307378512723\\
-0.5478515625	23.6705218437295\\
-0.54736328125	23.7103934231979\\
-0.546875	23.7503528391237\\
-0.54638671875	23.7904003421516\\
-0.5458984375	23.8305361841324\\
-0.54541015625	23.8707606181297\\
-0.544921875	23.9110738984284\\
-0.54443359375	23.9514762805409\\
-0.5439453125	23.9919680212157\\
-0.54345703125	24.0325493784446\\
-0.54296875	24.0732206114701\\
-0.54248046875	24.1139819807934\\
-0.5419921875	24.1548337481823\\
-0.54150390625	24.1957761766784\\
-0.541015625	24.2368095306057\\
-0.54052734375	24.2779340755779\\
-0.5400390625	24.319150078507\\
-0.53955078125	24.3604578076107\\
-0.5390625	24.4018575324212\\
-0.53857421875	24.4433495237925\\
-0.5380859375	24.4849340539095\\
-0.53759765625	24.5266113962959\\
-0.537109375	24.5683818258223\\
-0.53662109375	24.610245618715\\
-0.5361328125	24.6522030525644\\
-0.53564453125	24.6942544063334\\
-0.53515625	24.7363999603661\\
-0.53466796875	24.7786399963964\\
-0.5341796875	24.8209747975566\\
-0.53369140625	24.8634046483867\\
-0.533203125	24.9059298348426\\
-0.53271484375	24.9485506443055\\
-0.5322265625	24.9912673655905\\
-0.53173828125	25.0340802889561\\
-0.53125	25.076989706113\\
-0.53076171875	25.1199959102332\\
-0.5302734375	25.1630991959598\\
-0.52978515625	25.2062998594156\\
-0.529296875	25.2495981982131\\
-0.52880859375	25.2929945114637\\
-0.5283203125	25.3364890997871\\
-0.52783203125	25.3800822653212\\
-0.52734375	25.4237743117318\\
-0.52685546875	25.4675655442219\\
-0.5263671875	25.511456269542\\
-0.52587890625	25.5554467959999\\
-0.525390625	25.5995374334706\\
-0.52490234375	25.643728493406\\
-0.5244140625	25.6880202888459\\
-0.52392578125	25.7324131344274\\
-0.5234375	25.7769073463954\\
-0.52294921875	25.8215032426131\\
-0.5224609375	25.8662011425723\\
-0.52197265625	25.911001367404\\
-0.521484375	25.9559042398888\\
-0.52099609375	26.0009100844677\\
-0.5205078125	26.046019227253\\
-0.52001953125	26.0912319960386\\
-0.51953125	26.1365487203117\\
-0.51904296875	26.1819697312628\\
-0.5185546875	26.227495361798\\
-0.51806640625	26.2731259465487\\
-0.517578125	26.3188618218842\\
-0.51708984375	26.3647033259222\\
-0.5166015625	26.4106507985402\\
-0.51611328125	26.4567045813875\\
-0.515625	26.5028650178964\\
-0.51513671875	26.549132453294\\
-0.5146484375	26.5955072346136\\
-0.51416015625	26.6419897107072\\
-0.513671875	26.6885802322567\\
-0.51318359375	26.7352791517865\\
-0.5126953125	26.7820868236752\\
-0.51220703125	26.8290036041682\\
-0.51171875	26.8760298513893\\
-0.51123046875	26.923165925354\\
-0.5107421875	26.9704121879811\\
-0.51025390625	27.0177690031058\\
-0.509765625	27.0652367364922\\
-0.50927734375	27.1128157558457\\
-0.5087890625	27.1605064308267\\
-0.50830078125	27.2083091330628\\
-0.5078125	27.2562242361617\\
-0.50732421875	27.3042521157253\\
-0.5068359375	27.3523931493615\\
-0.50634765625	27.4006477166991\\
-0.505859375	27.4490161994\\
-0.50537109375	27.4974989811733\\
-0.5048828125	27.5460964477886\\
-0.50439453125	27.5948089870901\\
-0.50390625	27.6436369890101\\
-0.50341796875	27.6925808455832\\
-0.5029296875	27.7416409509599\\
-0.50244140625	27.7908177014213\\
-0.501953125	27.8401114953928\\
-0.50146484375	27.8895227334589\\
-0.5009765625	27.9390518183771\\
-0.50048828125	27.9886991550931\\
-0.5	28.0384651507549\\
-0.49951171875	28.0883502147278\\
-0.4990234375	28.1383547586093\\
-0.49853515625	28.1884791962438\\
-0.498046875	28.2387239437383\\
-0.49755859375	28.2890894194769\\
-0.4970703125	28.3395760441364\\
-0.49658203125	28.3901842407018\\
-0.49609375	28.4409144344816\\
-0.49560546875	28.4917670531237\\
-0.4951171875	28.542742526631\\
-0.49462890625	28.5938412873772\\
-0.494140625	28.6450637701228\\
-0.49365234375	28.6964104120312\\
-0.4931640625	28.7478816526853\\
-0.49267578125	28.7994779341029\\
-0.4921875	28.8511997007541\\
-0.49169921875	28.9030473995772\\
-0.4912109375	28.9550214799956\\
-0.49072265625	29.0071223939346\\
-0.490234375	29.0593505958384\\
-0.48974609375	29.1117065426866\\
-0.4892578125	29.1641906940122\\
-0.48876953125	29.2168035119179\\
-0.48828125	29.2695454610942\\
-0.48779296875	29.3224170088367\\
-0.4873046875	29.3754186250633\\
-0.48681640625	29.4285507823325\\
-0.486328125	29.481813955861\\
-0.48583984375	29.5352086235417\\
-0.4853515625	29.5887352659616\\
-0.48486328125	29.6423943664205\\
-0.484375	29.6961864109488\\
-0.48388671875	29.7501118883262\\
-0.4833984375	29.8041712901004\\
-0.48291015625	29.8583651106057\\
-0.482421875	29.9126938469815\\
-0.48193359375	29.9671579991922\\
-0.4814453125	30.0217580700449\\
-0.48095703125	30.0764945652099\\
-0.48046875	30.1313679932392\\
-0.47998046875	30.1863788655862\\
-0.4794921875	30.2415276966254\\
-0.47900390625	30.2968150036719\\
-0.478515625	30.3522413070014\\
-0.47802734375	30.4078071298699\\
-0.4775390625	30.4635129985338\\
-0.47705078125	30.5193594422705\\
-0.4765625	30.5753469933984\\
-0.47607421875	30.6314761872969\\
-0.4755859375	30.6877475624278\\
-0.47509765625	30.7441616603555\\
-0.474609375	30.8007190257678\\
-0.47412109375	30.857420206497\\
-0.4736328125	30.9142657535409\\
-0.47314453125	30.9712562210838\\
-0.47265625	31.028392166518\\
-0.47216796875	31.0856741504651\\
-0.4716796875	31.1431027367976\\
-0.47119140625	31.2006784926607\\
-0.470703125	31.2584019884934\\
-0.47021484375	31.3162737980514\\
-0.4697265625	31.3742944984282\\
-0.46923828125	31.4324646700778\\
-0.46875	31.4907848968368\\
-0.46826171875	31.5492557659462\\
-0.4677734375	31.6078778680752\\
-0.46728515625	31.6666517973419\\
-0.466796875	31.7255781513379\\
-0.46630859375	31.7846575311495\\
-0.4658203125	31.8438905413816\\
-0.46533203125	31.9032777901805\\
-0.46484375	31.9628198892566\\
-0.46435546875	32.0225174539081\\
-0.4638671875	32.082371103044\\
-0.46337890625	32.142381459208\\
-0.462890625	32.2025491486012\\
-0.46240234375	32.2628748011062\\
-0.4619140625	32.3233590503111\\
-0.46142578125	32.3840025335328\\
-0.4609375	32.4448058918408\\
-0.46044921875	32.5057697700817\\
-0.4599609375	32.5668948169027\\
-0.45947265625	32.6281816847762\\
-0.458984375	32.6896310300235\\
-0.45849609375	32.7512435128397\\
-0.4580078125	32.8130197973173\\
-0.45751953125	32.8749605514713\\
-0.45703125	32.9370664472635\\
-0.45654296875	32.9993381606267\\
-0.4560546875	33.0617763714898\\
-0.45556640625	33.1243817638019\\
-0.455078125	33.1871550255577\\
-0.45458984375	33.2500968488216\\
-0.4541015625	33.3132079297528\\
-0.45361328125	33.3764889686304\\
-0.453125	33.4399406698775\\
-0.45263671875	33.5035637420869\\
-0.4521484375	33.5673588980453\\
-0.45166015625	33.6313268547591\\
-0.451171875	33.6954683334781\\
-0.45068359375	33.7597840597217\\
-0.4501953125	33.8242747633028\\
-0.44970703125	33.8889411783535\\
-0.44921875	33.9537840433496\\
-0.44873046875	34.0188041011357\\
-0.4482421875	34.0840020989492\\
-0.44775390625	34.1493787884468\\
-0.447265625	34.2149349257278\\
-0.44677734375	34.2806712713592\\
-0.4462890625	34.3465885904006\\
-0.44580078125	34.4126876524288\\
-0.4453125	34.4789692315618\\
-0.44482421875	34.545434106484\\
-0.4443359375	34.6120830604697\\
-0.44384765625	34.6789168814076\\
-0.443359375	34.7459363618254\\
-0.44287109375	34.8131422989131\\
-0.4423828125	34.880535494547\\
-0.44189453125	34.9481167553133\\
-0.44140625	35.015886892532\\
-0.44091796875	35.0838467222797\\
-0.4404296875	35.151997065413\\
-0.43994140625	35.2203387475913\\
-0.439453125	35.2888725992996\\
-0.43896484375	35.357599455871\\
-0.4384765625	35.4265201575089\\
-0.43798828125	35.4956355493087\\
-0.4375	35.5649464812801\\
-0.43701171875	35.6344538083675\\
-0.4365234375	35.7041583904724\\
-0.43603515625	35.7740610924727\\
-0.435546875	35.8441627842443\\
-0.43505859375	35.9144643406803\\
-0.4345703125	35.9849666417109\\
-0.43408203125	36.0556705723231\\
-0.43359375	36.1265770225784\\
-0.43310546875	36.1976868876324\\
-0.4326171875	36.2690010677513\\
-0.43212890625	36.3405204683309\\
-0.431640625	36.4122459999118\\
-0.43115234375	36.4841785781968\\
-0.4306640625	36.5563191240661\\
-0.43017578125	36.6286685635926\\
-0.4296875	36.7012278280561\\
-0.42919921875	36.7739978539576\\
-0.4287109375	36.8469795830321\\
-0.42822265625	36.920173962261\\
-0.427734375	36.9935819438839\\
-0.42724609375	37.0672044854093\\
-0.4267578125	37.1410425496247\\
-0.42626953125	37.215097104606\\
-0.42578125	37.2893691237256\\
-0.42529296875	37.3638595856598\\
-0.4248046875	37.438569474395\\
-0.42431640625	37.5134997792335\\
-0.423828125	37.5886514947977\\
-0.42333984375	37.6640256210331\\
-0.4228515625	37.7396231632105\\
-0.42236328125	37.8154451319273\\
-0.421875	37.8914925431066\\
-0.42138671875	37.9677664179958\\
-0.4208984375	38.0442677831639\\
-0.42041015625	38.1209976704972\\
-0.419921875	38.1979571171925\\
-0.41943359375	38.2751471657515\\
-0.4189453125	38.3525688639704\\
-0.41845703125	38.4302232649307\\
-0.41796875	38.508111426986\\
-0.41748046875	38.5862344137489\\
-0.4169921875	38.6645932940744\\
-0.41650390625	38.743189142044\\
-0.416015625	38.8220230369438\\
-0.41552734375	38.901096063245\\
-0.4150390625	38.980409310579\\
-0.41455078125	39.059963873712\\
-0.4140625	39.1397608525168\\
-0.41357421875	39.219801351942\\
-0.4130859375	39.3000864819798\\
-0.41259765625	39.3806173576297\\
-0.412109375	39.4613950988609\\
-0.41162109375	39.5424208305717\\
-0.4111328125	39.623695682545\\
-0.41064453125	39.7052207894026\\
-0.41015625	39.7869972905557\\
-0.40966796875	39.8690263301511\\
-0.4091796875	39.9513090570165\\
-0.40869140625	40.0338466246006\\
-0.408203125	40.11664019091\\
-0.40771484375	40.1996909184432\\
-0.4072265625	40.2829999741204\\
-0.40673828125	40.3665685292092\\
-0.40625	40.4503977592461\\
-0.40576171875	40.5344888439555\\
-0.4052734375	40.6188429671613\\
-0.40478515625	40.703461316698\\
-0.404296875	40.7883450843131\\
-0.40380859375	40.8734954655676\\
-0.4033203125	40.9589136597308\\
-0.40283203125	41.0446008696688\\
-0.40234375	41.1305583017297\\
-0.40185546875	41.2167871656214\\
-0.4013671875	41.3032886742855\\
-0.40087890625	41.390064043763\\
-0.400390625	41.4771144930569\\
-0.39990234375	41.5644412439853\\
-0.3994140625	41.6520455210307\\
-0.39892578125	41.739928551181\\
-0.3984375	41.8280915637635\\
-0.39794921875	41.9165357902727\\
-0.3974609375	42.0052624641892\\
-0.39697265625	42.0942728207927\\
-0.396484375	42.1835680969649\\
-0.39599609375	42.2731495309855\\
-0.3955078125	42.3630183623199\\
-0.39501953125	42.4531758313964\\
-0.39453125	42.5436231793755\\
-0.39404296875	42.63436164791\\
-0.3935546875	42.7253924788943\\
-0.39306640625	42.816716914204\\
-0.392578125	42.908336195425\\
-0.39208984375	43.0002515635722\\
-0.3916015625	43.0924642587959\\
-0.39111328125	43.1849755200774\\
-0.390625	43.2777865849128\\
-0.39013671875	43.3708986889841\\
-0.3896484375	43.4643130658164\\
-0.38916015625	43.5580309464241\\
-0.388671875	43.6520535589418\\
-0.38818359375	43.7463821282406\\
-0.3876953125	43.8410178755317\\
-0.38720703125	43.9359620179531\\
-0.38671875	44.0312157681407\\
-0.38623046875	44.1267803337849\\
-0.3857421875	44.2226569171687\\
-0.38525390625	44.3188467146886\\
-0.384765625	44.4153509163596\\
-0.38427734375	44.5121707052986\\
-0.3837890625	44.6093072571918\\
-0.38330078125	44.7067617397403\\
-0.3828125	44.8045353120864\\
-0.38232421875	44.9026291242179\\
-0.3818359375	45.0010443163522\\
-0.38134765625	45.0997820182961\\
-0.380859375	45.1988433487835\\
-0.38037109375	45.298229414789\\
-0.3798828125	45.3979413108167\\
-0.37939453125	45.4979801181629\\
-0.37890625	45.5983469041526\\
-0.37841796875	45.6990427213499\\
-0.3779296875	45.8000686067381\\
-0.37744140625	45.9014255808724\\
-0.376953125	46.0031146470012\\
-0.37646484375	46.1051367901579\\
-0.3759765625	46.2074929762189\\
-0.37548828125	46.3101841509302\\
-0.375	46.4132112388985\\
-0.37451171875	46.5165751425477\\
-0.3740234375	46.6202767410396\\
-0.37353515625	46.7243168891547\\
-0.373046875	46.8286964161373\\
-0.37255859375	46.9334161244988\\
-0.3720703125	47.03847678878\\
-0.37158203125	47.1438791542706\\
-0.37109375	47.2496239356861\\
-0.37060546875	47.3557118157973\\
-0.3701171875	47.4621434440153\\
-0.36962890625	47.5689194349264\\
-0.369140625	47.676040366778\\
-0.36865234375	47.7835067799153\\
-0.3681640625	47.8913191751613\\
-0.36767578125	47.9994780121456\\
-0.3671875	48.1079837075755\\
-0.36669921875	48.2168366334516\\
-0.3662109375	48.32603711522\\
-0.36572265625	48.4355854298671\\
-0.365234375	48.5454818039489\\
-0.36474609375	48.655726411557\\
-0.3642578125	48.766319372216\\
-0.36376953125	48.8772607487135\\
-0.36328125	48.9885505448586\\
-0.36279296875	49.1001887031679\\
-0.3623046875	49.2121751024763\\
-0.36181640625	49.3245095554704\\
-0.361328125	49.4371918061444\\
-0.36083984375	49.5502215271711\\
-0.3603515625	49.6635983171944\\
-0.35986328125	49.7773216980302\\
-0.359375	49.891391111784\\
-0.35888671875	50.0058059178747\\
-0.3583984375	50.1205653899669\\
-0.35791015625	50.2356687128071\\
-0.357421875	50.3511149789633\\
-0.35693359375	50.4669031854634\\
-0.3564453125	50.5830322303313\\
-0.35595703125	50.6995009090175\\
-0.35546875	50.816307910722\\
-0.35498046875	50.9334518146083\\
-0.3544921875	51.0509310859025\\
-0.35400390625	51.168744071879\\
-0.353515625	51.2868889977279\\
-0.35302734375	51.4053639623021\\
-0.3525390625	51.5241669337407\\
-0.35205078125	51.6432957449702\\
-0.3515625	51.762748089076\\
-0.35107421875	51.8825215145424\\
-0.3505859375	52.0026134203654\\
-0.35009765625	52.1230210510254\\
-0.349609375	52.243741491328\\
-0.34912109375	52.3647716611035\\
-0.3486328125	52.4861083097663\\
-0.34814453125	52.6077480107328\\
-0.34765625	52.7296871556939\\
-0.34716796875	52.8519219487433\\
-0.3466796875	52.9744484003566\\
-0.34619140625	53.0972623212247\\
-0.345703125	53.2203593159358\\
-0.34521484375	53.3437347765084\\
-0.3447265625	53.4673838757739\\
-0.34423828125	53.5913015606067\\
-0.34375	53.7154825450069\\
-0.34326171875	53.8399213030287\\
-0.3427734375	53.9646120615619\\
-0.34228515625	54.0895487929641\\
-0.341796875	54.2147252075465\\
-0.34130859375	54.3401347459125\\
-0.3408203125	54.4657705711571\\
-0.34033203125	54.591625560925\\
-0.33984375	54.7176922993331\\
-0.33935546875	54.8439630687616\\
-0.3388671875	54.9704298415177\\
-0.33837890625	55.0970842713796\\
-0.337890625	55.223917685024\\
-0.33740234375	55.3509210733456\\
-0.3369140625	55.4780850826788\\
-0.33642578125	55.6054000059252\\
-0.3359375	55.7328557736034\\
-0.33544921875	55.8604419448235\\
-0.3349609375	55.9881476982082\\
-0.33447265625	56.1159618227632\\
-0.333984375	56.2438727087193\\
-0.33349609375	56.3718683383577\\
-0.3330078125	56.4999362768329\\
-0.33251953125	56.6280636630173\\
-0.33203125	56.756237200383\\
-0.33154296875	56.8844431479406\\
-0.3310546875	57.0126673112599\\
-0.33056640625	57.1408950335968\\
-0.330078125	57.2691111871506\\
-0.32958984375	57.3973001644807\\
-0.3291015625	57.5254458701059\\
-0.32861328125	57.6535317123304\\
-0.328125	57.7815405953143\\
-0.32763671875	57.9094549114336\\
-0.3271484375	58.0372565339621\\
-0.32666015625	58.1649268101129\\
-0.326171875	58.2924465544856\\
-0.32568359375	58.4197960429551\\
-0.3251953125	58.5469550070475\\
-0.32470703125	58.6739026288586\\
-0.32421875	58.8006175365463\\
-0.32373046875	58.9270778004637\\
-0.3232421875	59.0532609299785\\
-0.32275390625	59.1791438710272\\
-0.322265625	59.3047030044703\\
-0.32177734375	59.4299141452967\\
-0.3212890625	59.5547525427414\\
-0.32080078125	59.6791928813735\\
-0.3203125	59.8032092832153\\
-0.31982421875	59.926775310957\\
-0.3193359375	60.0498639723292\\
-0.31884765625	60.17244772569\\
-0.318359375	60.2944984869015\\
-0.31787109375	60.4159876375476\\
-0.3173828125	60.5368860345645\\
-0.31689453125	60.6571640213358\\
-0.31640625	60.7767914403265\\
-0.31591796875	60.8957376473028\\
-0.3154296875	61.0139715271973\\
-0.31494140625	61.1314615116809\\
-0.314453125	61.2481755984836\\
-0.31396484375	61.3640813725167\\
-0.3134765625	61.4791460288398\\
-0.31298828125	61.5933363975092\\
-0.3125	61.7066189703459\\
-0.31201171875	61.8189599296428\\
-0.3115234375	61.9303251788428\\
-0.31103515625	62.0406803751904\\
-0.310546875	62.1499909643744\\
-0.31005859375	62.2582222171491\\
-0.3095703125	62.3653392679329\\
-0.30908203125	62.4713071553555\\
-0.30859375	62.5760908647271\\
-0.30810546875	62.6796553723902\\
-0.3076171875	62.7819656918977\\
-0.30712890625	62.8829869219595\\
-0.306640625	62.9826842960716\\
-0.30615234375	63.0810232337599\\
-0.3056640625	63.177969393314\\
-0.30517578125	63.2734887259216\\
-0.3046875	63.3675475310672\\
-0.30419921875	63.4601125130585\\
-0.3037109375	63.5511508385385\\
-0.30322265625	63.6406301948195\\
-0.302734375	63.7285188488598\\
-0.30224609375	63.8147857067157\\
-0.3017578125	63.8994003732601\\
-0.30126953125	63.9823332119756\\
-0.30078125	64.0635554046066\\
-0.30029296875	64.1430390104533\\
-0.2998046875	64.2207570250781\\
-0.29931640625	64.2966834382073\\
-0.298828125	64.3707932905735\\
-0.29833984375	64.4430627294851\\
-0.2978515625	64.513469062868\\
-0.29736328125	64.5819908115531\\
-0.296875	64.6486077595741\\
-0.29638671875	64.7133010022453\\
-0.2958984375	64.7760529917929\\
-0.29541015625	64.8368475803313\\
-0.294921875	64.895670059969\\
-0.29443359375	64.9525071998549\\
-0.2939453125	65.0073472799794\\
-0.29345703125	65.0601801215563\\
-0.29296875	65.1109971138428\\
-0.29248046875	65.1597912372508\\
-0.2919921875	65.2065570826316\\
-0.29150390625	65.2512908666424\\
-0.291015625	65.2939904431037\\
-0.29052734375	65.334655310297\\
-0.2900390625	65.3732866141582\\
-0.28955078125	65.4098871473572\\
-0.2890625	65.4444613442611\\
-0.28857421875	65.4770152718139\\
-0.2880859375	65.5075566163789\\
-0.28759765625	65.5360946666108\\
-0.287109375	65.5626402924504\\
-0.28662109375	65.5872059203492\\
-0.2861328125	65.6098055048512\\
-0.28564453125	65.6304544966762\\
-0.28515625	65.6491698074676\\
-0.28466796875	65.6659697713852\\
-0.2841796875	65.6808741037191\\
-0.28369140625	65.6939038567378\\
-0.283203125	65.7050813729819\\
-0.28271484375	65.7144302362153\\
-0.2822265625	65.7219752202653\\
-0.28173828125	65.7277422359868\\
-0.28125	65.7317582765768\\
-0.28076171875	65.7340513614861\\
-0.2802734375	65.7346504791551\\
-0.27978515625	65.7335855288169\\
-0.279296875	65.7308872615974\\
-0.27880859375	65.7265872211332\\
-0.2783203125	65.7207176839468\\
-0.27783203125	65.713311599773\\
-0.27734375	65.7044025320597\\
-0.27685546875	65.6940245988302\\
-0.2763671875	65.682212414108\\
-0.27587890625	65.6690010300671\\
-0.275390625	65.6544258800869\\
-0.27490234375	65.638522722856\\
-0.2744140625	65.621327587679\\
-0.27392578125	65.6028767211069\\
-0.2734375	65.5832065350169\\
-0.27294921875	65.5623535562409\\
-0.2724609375	65.5403543778447\\
-0.27197265625	65.5172456121308\\
-0.271484375	65.4930638454413\\
-0.27099609375	65.4678455948117\\
-0.2705078125	65.4416272665328\\
-0.27001953125	65.4144451166446\\
-0.26953125	65.3863352133966\\
-0.26904296875	65.3573334016888\\
-0.2685546875	65.3274752694932\\
-0.26806640625	65.296796116268\\
-0.267578125	65.2653309233368\\
-0.26708984375	65.2331143262316\\
-0.2666015625	65.2001805889639\\
-0.26611328125	65.1665635801981\\
-0.265625	65.1322967512939\\
-0.26513671875	65.0974131161666\\
-0.2646484375	65.061945232934\\
-0.26416015625	65.0259251872855\\
-0.263671875	64.9893845775344\\
-0.26318359375	64.9523545012855\\
-0.2626953125	64.9148655436697\\
-0.26220703125	64.8769477670789\\
-0.26171875	64.8386307023475\\
-0.26123046875	64.7999433413055\\
-0.2607421875	64.7609141306569\\
-0.26025390625	64.7215709671072\\
-0.259765625	64.6819411936817\\
-0.25927734375	64.6420515971703\\
-0.2587890625	64.6019284066407\\
-0.25830078125	64.5615972929523\\
-0.2578125	64.5210833692139\\
-0.25732421875	64.4804111921245\\
-0.2568359375	64.4396047641425\\
-0.25634765625	64.3986875364226\\
-0.255859375	64.3576824124679\\
-0.25537109375	64.3166117524416\\
-0.2548828125	64.2754973780937\\
-0.25439453125	64.2343605782455\\
-0.25390625	64.1932221147871\\
-0.25341796875	64.1521022291463\\
-0.2529296875	64.1110206491774\\
-0.25244140625	64.0699965964391\\
-0.251953125	64.0290487938071\\
-0.25146484375	63.9881954734015\\
-0.2509765625	63.9474543847778\\
-0.25048828125	63.9068428033551\\
-0.25	63.8663775390504\\
-0.24951171875	63.8260749450857\\
-0.2490234375	63.785950926937\\
-0.24853515625	63.7460209514101\\
-0.248046875	63.7063000558021\\
-0.24755859375	63.6668028571395\\
-0.2470703125	63.6275435614624\\
-0.24658203125	63.5885359731352\\
-0.24609375	63.5497935041744\\
-0.24560546875	63.5113291835604\\
-0.2451171875	63.4731556665319\\
-0.24462890625	63.4352852438367\\
-0.244140625	63.3977298509367\\
-0.24365234375	63.3605010771425\\
-0.2431640625	63.3236101746757\\
-0.24267578125	63.2870680676423\\
-0.2421875	63.2508853609146\\
-0.24169921875	63.2150723489037\\
-0.2412109375	63.1796390242248\\
-0.24072265625	63.1445950862436\\
-0.240234375	63.1099499494939\\
-0.23974609375	63.0757127519735\\
-0.2392578125	63.0418923632963\\
-0.23876953125	63.0084973927149\\
-0.23828125	62.975536196997\\
-0.23779296875	62.9430168881553\\
-0.2373046875	62.9109473410374\\
-0.23681640625	62.8793352007589\\
-0.236328125	62.8481878899926\\
-0.23583984375	62.8175126161029\\
-0.2353515625	62.7873163781298\\
-0.23486328125	62.7576059736186\\
-0.234375	62.7283880053003\\
-0.23388671875	62.6996688876167\\
-0.2333984375	62.6714548530958\\
-0.23291015625	62.643751958577\\
-0.232421875	62.6165660912822\\
-0.23193359375	62.5899029747423\\
-0.2314453125	62.563768174576\\
-0.23095703125	62.5381671041189\\
-0.23046875	62.513105029909\\
-0.22998046875	62.4885870770352\\
-0.2294921875	62.4646182343346\\
-0.22900390625	62.4412033594579\\
-0.228515625	62.4183471837958\\
-0.22802734375	62.39605431727\\
-0.2275390625	62.3743292529885\\
-0.22705078125	62.3531763717734\\
-0.2265625	62.3325999465577\\
-0.22607421875	62.3126041466553\\
-0.2255859375	62.2931930419068\\
-0.22509765625	62.2743706067021\\
-0.224609375	62.2561407238828\\
-0.22412109375	62.2385071885269\\
-0.2236328125	62.221473711615\\
-0.22314453125	62.2050439235853\\
-0.22265625	62.1892213777744\\
-0.22216796875	62.1740095537494\\
-0.2216796875	62.1594118605303\\
-0.22119140625	62.1454316397086\\
-0.220703125	62.132072168458\\
-0.22021484375	62.1193366624472\\
-0.2197265625	62.1072282786468\\
-0.21923828125	62.0957501180427\\
-0.21875	62.0849052282489\\
-0.21826171875	62.0746966060223\\
-0.2177734375	62.0651271996897\\
-0.21728515625	62.0561999114769\\
-0.216796875	62.0479175997457\\
-0.21630859375	62.0402830811461\\
-0.2158203125	62.033299132673\\
-0.21533203125	62.0269684936409\\
-0.21484375	62.0212938675658\\
-0.21435546875	62.0162779239678\\
-0.2138671875	62.0119233000826\\
-0.21337890625	62.008232602492\\
-0.212890625	62.0052084086716\\
-0.21240234375	62.0028532684513\\
-0.2119140625	62.0011697053987\\
-0.21142578125	62.0001602181154\\
-0.2109375	61.9998272814547\\
-0.21044921875	62.0001733476567\\
-0.2099609375	62.0012008473988\\
-0.20947265625	62.0029121907714\\
-0.208984375	62.0053097681628\\
-0.20849609375	62.0083959510694\\
-0.2080078125	62.0121730928189\\
-0.20751953125	62.0166435292119\\
-0.20703125	62.0218095790778\\
-0.20654296875	62.0276735447525\\
-0.2060546875	62.03423771246\\
-0.20556640625	62.0415043526159\\
-0.205078125	62.0494757200406\\
-0.20458984375	62.0581540540822\\
-0.2041015625	62.0675415786503\\
-0.20361328125	62.0776405021558\\
-0.203125	62.0884530173579\\
-0.20263671875	62.0999813011155\\
-0.2021484375	62.1122275140364\\
-0.20166015625	62.1251938000335\\
-0.201171875	62.1388822857679\\
-0.20068359375	62.1532950799955\\
-0.2001953125	62.1684342727972\\
-0.19970703125	62.1843019347035\\
-0.19921875	62.2009001156996\\
-0.19873046875	62.2182308441128\\
-0.1982421875	62.236296125377\\
-0.19775390625	62.2550979406739\\
-0.197265625	62.2746382454376\\
-0.19677734375	62.2949189677278\\
-0.1962890625	62.3159420064648\\
-0.19580078125	62.3377092295109\\
-0.1953125	62.3602224716126\\
-0.19482421875	62.3834835321764\\
-0.1943359375	62.4074941728919\\
-0.19384765625	62.4322561151766\\
-0.193359375	62.4577710374555\\
-0.19287109375	62.4840405722522\\
-0.1923828125	62.5110663030912\\
-0.19189453125	62.5388497612104\\
-0.19140625	62.5673924220595\\
-0.19091796875	62.5966957015902\\
-0.1904296875	62.6267609523252\\
-0.18994140625	62.6575894591928\\
-0.189453125	62.6891824351213\\
-0.18896484375	62.7215410163833\\
-0.1884765625	62.7546662576806\\
-0.18798828125	62.7885591269488\\
-0.1875	62.8232204998817\\
-0.18701171875	62.8586511541637\\
-0.1865234375	62.8948517633849\\
-0.18603515625	62.9318228906344\\
-0.185546875	62.9695649817642\\
-0.18505859375	63.0080783582955\\
-0.1845703125	63.047363209958\\
-0.18408203125	63.0874195868572\\
-0.18359375	63.1282473912324\\
-0.18310546875	63.1698463688066\\
-0.1826171875	63.2122160997064\\
-0.18212890625	63.2553559889231\\
-0.181640625	63.299265256314\\
-0.18115234375	63.3439429261107\\
-0.1806640625	63.3893878159154\\
-0.18017578125	63.4355985251746\\
-0.1796875	63.4825734230952\\
-0.17919921875	63.5303106359894\\
-0.1787109375	63.5788080340221\\
-0.17822265625	63.6280632173374\\
-0.177734375	63.6780735015432\\
-0.17724609375	63.7288359025212\\
-0.1767578125	63.7803471205445\\
-0.17626953125	63.8326035236743\\
-0.17578125	63.8856011304059\\
-0.17529296875	63.9393355915415\\
-0.1748046875	63.9938021712587\\
-0.17431640625	64.0489957273481\\
-0.173828125	64.1049106905909\\
-0.17333984375	64.1615410432521\\
-0.1728515625	64.2188802966542\\
-0.17236328125	64.2769214678041\\
-0.171875	64.3356570550551\\
-0.17138671875	64.395079012756\\
-0.1708984375	64.4551787248847\\
-0.17041015625	64.5159469776142\\
-0.169921875	64.5773739308102\\
-0.16943359375	64.6394490884088\\
-0.1689453125	64.7021612676794\\
-0.16845703125	64.765498567328\\
-0.16796875	64.8294483344365\\
-0.16748046875	64.8939971302117\\
-0.1669921875	64.9591306945394\\
-0.16650390625	65.0248339093263\\
-0.166015625	65.0910907606221\\
-0.16552734375	65.1578842995313\\
-0.1650390625	65.2251966018996\\
-0.16455078125	65.2930087268026\\
-0.1640625	65.3613006738333\\
-0.16357421875	65.4300513392315\\
-0.1630859375	65.4992384708582\\
-0.16259765625	65.5688386220889\\
-0.162109375	65.6388271046365\\
-0.16162109375	65.7091779403923\\
-0.1611328125	65.779863812345\\
-0.16064453125	65.8508560146587\\
-0.16015625	65.922124402015\\
-0.15966796875	65.9936373383301\\
-0.1591796875	66.0653616449764\\
-0.15869140625	66.1372625486528\\
-0.158203125	66.2093036290815\\
-0.15771484375	66.2814467667086\\
-0.1572265625	66.3536520906209\\
-0.15673828125	66.4258779269178\\
-0.15625	66.4980807477853\\
-0.15576171875	66.5702151215695\\
-0.1552734375	66.6422336641444\\
-0.15478515625	66.7140869919337\\
-0.154296875	66.7857236769358\\
-0.15380859375	66.8570902041634\\
-0.1533203125	66.928130931931\\
-0.15283203125	66.9987880554412\\
-0.15234375	67.0690015741769\\
-0.15185546875	67.1387092636084\\
-0.1513671875	67.2078466517999\\
-0.15087890625	67.2763470014723\\
-0.150390625	67.3441412981548\\
-0.14990234375	67.4111582450701\\
-0.1494140625	67.4773242654065\\
-0.14892578125	67.542563512678\\
-0.1484375	67.6067978898587\\
-0.14794921875	67.6699470780346\\
-0.1474609375	67.7319285752479\\
-0.14697265625	67.7926577463039\\
-0.146484375	67.8520478842186\\
-0.14599609375	67.9100102839985\\
-0.1455078125	67.9664543294486\\
-0.14501953125	68.0212875935985\\
-0.14453125	68.0744159533665\\
-0.14404296875	68.1257437189494\\
-0.1435546875	68.1751737784133\\
-0.14306640625	68.2226077578204\\
-0.142578125	68.2679461971756\\
-0.14208984375	68.3110887423177\\
-0.1416015625	68.3519343527999\\
-0.14111328125	68.3903815256019\\
-0.140625	68.4263285344224\\
-0.14013671875	68.4596736840965\\
-0.1396484375	68.490315579509\\
-0.13916015625	68.5181534082104\\
-0.138671875	68.5430872357058\\
-0.13818359375	68.5650183122505\\
-0.1376953125	68.5838493897143\\
-0.13720703125	68.5994850469464\\
-0.13671875	68.6118320218278\\
-0.13623046875	68.6207995480509\\
-0.1357421875	68.6262996944571\\
-0.13525390625	68.6282477046552\\
-0.134765625	68.6265623344556\\
-0.13427734375	68.6211661845943\\
-0.1337890625	68.6119860260928\\
-0.13330078125	68.5989531156062\\
-0.1328125	68.5820034980124\\
-0.13232421875	68.5610782936153\\
-0.1318359375	68.5361239673076\\
-0.13134765625	68.507092577182\\
-0.130859375	68.473942000207\\
-0.13037109375	68.4366361327461\\
-0.1298828125	68.3951450639242\\
-0.12939453125	68.3494452200852\\
-0.12890625	68.2995194788447\\
-0.12841796875	68.2453572515712\\
-0.1279296875	68.1869545334193\\
-0.12744140625	68.1243139203952\\
-0.126953125	68.0574445932747\\
-0.12646484375	67.9863622685426\\
-0.1259765625	67.9110891168613\\
-0.12548828125	67.8316536499372\\
-0.125	67.7480905769443\\
-0.12451171875	67.6604406319817\\
-0.1240234375	67.5687503743274\\
-0.12353515625	67.4730719634631\\
-0.123046875	67.373462911098\\
-0.12255859375	67.2699858125607\\
-0.1220703125	67.1627080600788\\
-0.12158203125	67.0517015405732\\
-0.12109375	66.9370423206299\\
-0.12060546875	66.818810321365\\
-0.1201171875	66.6970889858486\\
-0.11962890625	66.5719649417255\\
-0.119140625	66.4435276615992\\
-0.11865234375	66.3118691235947\\
-0.1181640625	66.1770834744327\\
-0.11767578125	66.0392666971524\\
-0.1171875	65.8985162854743\\
-0.11669921875	65.7549309265994\\
-0.1162109375	65.608610194045\\
-0.11572265625	65.4596542519345\\
-0.115234375	65.3081635719318\\
-0.11474609375	65.154238663849\\
-0.1142578125	64.9979798207304\\
-0.11376953125	64.8394868790474\\
-0.11328125	64.6788589944467\\
-0.11279296875	64.5161944333502\\
-0.1123046875	64.3515903805218\\
-0.11181640625	64.1851427625946\\
-0.111328125	64.0169460874106\\
-0.11083984375	63.8470932989259\\
-0.1103515625	63.6756756473098\\
-0.10986328125	63.5027825738038\\
-0.109375	63.3285016098191\\
-0.10888671875	63.1529182896863\\
-0.1083984375	62.9761160764432\\
-0.10791015625	62.7981762999779\\
-0.107421875	62.6191781068513\\
-0.10693359375	62.439198421071\\
-0.1064453125	62.2583119151209\\
-0.10595703125	62.0765909905007\\
-0.10546875	61.8941057670751\\
-0.10498046875	61.7109240805326\\
-0.1044921875	61.5271114872523\\
-0.10400390625	61.3427312759266\\
-0.103515625	61.1578444852854\\
-0.10302734375	60.9725099273116\\
-0.1025390625	60.7867842153597\\
-0.10205078125	60.6007217966128\\
-0.1015625	60.4143749883635\\
-0.10107421875	60.2277940176097\\
-0.1005859375	60.0410270635107\\
-0.10009765625	59.8541203022707\\
-0.099609375	59.6671179540473\\
-0.09912109375	59.4800623315112\\
-0.0986328125	59.2929938897293\\
-0.09814453125	59.1059512770399\\
-0.09765625	58.9189713866558\\
-0.09716796875	58.7320894087199\\
-0.0966796875	58.5453388825891\\
-0.09619140625	58.3587517491343\\
-0.095703125	58.1723584028661\\
-0.09521484375	57.986187743719\\
-0.0947265625	57.800267228348\\
-0.09423828125	57.6146229208039\\
-0.09375	57.4292795424759\\
-0.09326171875	57.2442605212002\\
-0.0927734375	57.059588039455\\
-0.09228515625	56.8752830815626\\
-0.091796875	56.691365479847\\
-0.09130859375	56.5078539596919\\
-0.0908203125	56.3247661834639\\
-0.09033203125	56.1421187932695\\
-0.08984375	55.9599274525198\\
-0.08935546875	55.7782068862964\\
-0.0888671875	55.5969709205\\
-0.08837890625	55.4162325197844\\
-0.087890625	55.2360038242758\\
-0.08740234375	55.0562961850889\\
-0.0869140625	54.8771201986367\\
-0.08642578125	54.6984857397678\\
-0.0859375	54.5204019937276\\
-0.08544921875	54.3428774869779\\
-0.0849609375	54.1659201168878\\
-0.08447265625	53.9895371803219\\
-0.083984375	53.8137354011501\\
-0.08349609375	53.6385209567066\\
-0.0830078125	53.4638995032198\\
-0.08251953125	53.2898762002489\\
-0.08203125	53.1164557341457\\
-0.08154296875	52.9436423405767\\
-0.0810546875	52.771439826135\\
-0.08056640625	52.5998515890638\\
-0.080078125	52.4288806391258\\
-0.07958984375	52.2585296166463\\
-0.0791015625	52.088800810757\\
-0.07861328125	51.919696176866\\
-0.078125	51.751217353389\\
-0.07763671875	51.583365677757\\
-0.0771484375	51.4161422017404\\
-0.07666015625	51.249547706102\\
-0.076171875	51.0835827146138\\
-0.07568359375	50.9182475074582\\
-0.0751953125	50.7535421340383\\
-0.07470703125	50.5894664252178\\
-0.07421875	50.4260200050194\\
-0.07373046875	50.2632023017968\\
-0.0732421875	50.1010125589044\\
-0.07275390625	49.9394498448875\\
-0.072265625	49.7785130632076\\
-0.07177734375	49.6182009615274\\
-0.0712890625	49.4585121405698\\
-0.07080078125	49.2994450625709\\
-0.0703125	49.1409980593434\\
-0.06982421875	48.9831693399679\\
-0.0693359375	48.8259569981265\\
-0.06884765625	48.6693590190966\\
-0.068359375	48.5133732864142\\
-0.06787109375	48.3579975882292\\
-0.0673828125	48.2032296233585\\
-0.06689453125	48.0490670070539\\
-0.06640625	47.895507276498\\
-0.06591796875	47.7425478960363\\
-0.0654296875	47.5901862621607\\
-0.06494140625	47.4384197082561\\
-0.064453125	47.2872455091144\\
-0.06396484375	47.136660885235\\
-0.0634765625	46.986663006914\\
-0.06298828125	46.8372489981368\\
-0.0625	46.6884159402784\\
-0.06201171875	46.5401608756232\\
-0.0615234375	46.3924808107128\\
-0.06103515625	46.2453727195264\\
-0.060546875	46.0988335465038\\
-0.06005859375	45.9528602094186\\
-0.0595703125	45.8074496021053\\
-0.05908203125	45.6625985970523\\
-0.05859375	45.5183040478603\\
-0.05810546875	45.3745627915771\\
-0.0576171875	45.2313716509146\\
-0.05712890625	45.0887274363474\\
-0.056640625	44.9466269481081\\
-0.05615234375	44.8050669780754\\
-0.0556640625	44.6640443115632\\
-0.05517578125	44.5235557290174\\
-0.0546875	44.3835980076211\\
-0.05419921875	44.2441679228141\\
-0.0537109375	44.1052622497311\\
-0.05322265625	43.9668777645613\\
-0.052734375	43.8290112458333\\
-0.05224609375	43.6916594756305\\
-0.0517578125	43.5548192407359\\
-0.05126953125	43.4184873337171\\
-0.05078125	43.2826605539427\\
-0.05029296875	43.1473357085485\\
-0.0498046875	43.0125096133406\\
-0.04931640625	42.87817909365\\
-0.048828125	42.7443409851332\\
-0.04833984375	42.6109921345276\\
-0.0478515625	42.4781294003572\\
-0.04736328125	42.3457496535982\\
-0.046875	42.2138497783004\\
-0.04638671875	42.0824266721687\\
-0.0458984375	41.9514772471082\\
-0.04541015625	41.8209984297321\\
-0.044921875	41.6909871618354\\
-0.04443359375	41.5614404008368\\
-0.0439453125	41.4323551201874\\
-0.04345703125	41.3037283097518\\
-0.04296875	41.1755569761605\\
-0.04248046875	41.0478381431348\\
-0.0419921875	40.9205688517866\\
-0.04150390625	40.7937461608941\\
-0.041015625	40.6673671471544\\
-0.04052734375	40.541428905414\\
-0.0400390625	40.415928548878\\
-0.03955078125	40.2908632093012\\
-0.0390625	40.1662300371571\\
-0.03857421875	40.0420262017939\\
-0.0380859375	39.9182488915677\\
-0.03759765625	39.794895313965\\
-0.037109375	39.6719626957063\\
-0.03662109375	39.5494482828362\\
-0.0361328125	39.4273493408006\\
-0.03564453125	39.3056631545099\\
-0.03515625	39.1843870283896\\
-0.03466796875	39.0635182864203\\
-0.0341796875	38.9430542721655\\
-0.03369140625	38.822992348789\\
-0.033203125	38.7033298990639\\
-0.03271484375	38.5840643253693\\
-0.0322265625	38.465193049681\\
-0.03173828125	38.3467135135514\\
-0.03125	38.2286231780836\\
-0.03076171875	38.1109195238963\\
-0.0302734375	37.9936000510823\\
-0.02978515625	37.8766622791608\\
-0.029296875	37.7601037470216\\
-0.02880859375	37.6439220128667\\
-0.0283203125	37.5281146541418\\
-0.02783203125	37.4126792674678\\
-0.02734375	37.2976134685618\\
-0.02685546875	37.1829148921588\\
-0.0263671875	37.0685811919246\\
-0.02587890625	36.9546100403679\\
-0.025390625	36.840999128747\\
-0.02490234375	36.7277461669728\\
-0.0244140625	36.6148488835097\\
-0.02392578125	36.5023050252722\\
-0.0234375	36.3901123575196\\
-0.02294921875	36.2782686637471\\
-0.0224609375	36.1667717455755\\
-0.02197265625	36.0556194226373\\
-0.021484375	35.9448095324626\\
-0.02099609375	35.8343399303607\\
-0.0205078125	35.7242084893022\\
-0.02001953125	35.6144130997982\\
-0.01953125	35.5049516697778\\
-0.01904296875	35.3958221244659\\
-0.0185546875	35.2870224062575\\
-0.01806640625	35.1785504745929\\
-0.017578125	35.070404305831\\
-0.01708984375	34.9625818931223\\
-0.0166015625	34.8550812462798\\
-0.01611328125	34.7479003916513\\
-0.015625	34.64103737199\\
-0.01513671875	34.5344902463244\\
-0.0146484375	34.4282570898289\\
-0.01416015625	34.3223359936924\\
-0.013671875	34.2167250649893\\
-0.01318359375	34.1114224265478\\
-0.0126953125	34.0064262168195\\
-0.01220703125	33.9017345897489\\
-0.01171875	33.7973457146425\\
-0.01123046875	33.6932577760392\\
-0.0107421875	33.5894689735786\\
-0.01025390625	33.4859775218729\\
-0.009765625	33.3827816503755\\
-0.00927734375	33.2798796032526\\
-0.0087890625	33.1772696392538\\
-0.00830078125	33.0749500315842\\
-0.0078125	32.9729190677756\\
-0.00732421875	32.8711750495592\\
-0.0068359375	32.7697162927391\\
-0.00634765625	32.668541127065\\
-0.005859375	32.567647896107\\
-0.00537109375	32.4670349571301\\
-0.0048828125	32.36670068097\\
-0.00439453125	32.2666434519089\\
-0.00390625	32.1668616675523\\
-0.00341796875	32.067353738707\\
-0.0029296875	31.9681180892592\\
-0.00244140625	31.8691531560533\\
-0.001953125	31.7704573887725\\
-0.00146484375	31.6720292498186\\
-0.0009765625	31.5738672141946\\
-0.00048828125	31.4759697693859\\
0	31.3783354152442\\
0.00048828125	31.4759697693859\\
0.0009765625	31.5738672141946\\
0.00146484375	31.6720292498186\\
0.001953125	31.7704573887725\\
0.00244140625	31.8691531560533\\
0.0029296875	31.9681180892592\\
0.00341796875	32.067353738707\\
0.00390625	32.1668616675523\\
0.00439453125	32.2666434519089\\
0.0048828125	32.36670068097\\
0.00537109375	32.4670349571301\\
0.005859375	32.567647896107\\
0.00634765625	32.668541127065\\
0.0068359375	32.7697162927391\\
0.00732421875	32.8711750495592\\
0.0078125	32.9729190677756\\
0.00830078125	33.0749500315842\\
0.0087890625	33.1772696392538\\
0.00927734375	33.2798796032526\\
0.009765625	33.3827816503755\\
0.01025390625	33.4859775218729\\
0.0107421875	33.5894689735786\\
0.01123046875	33.6932577760392\\
0.01171875	33.7973457146425\\
0.01220703125	33.9017345897489\\
0.0126953125	34.0064262168195\\
0.01318359375	34.1114224265478\\
0.013671875	34.2167250649893\\
0.01416015625	34.3223359936924\\
0.0146484375	34.4282570898289\\
0.01513671875	34.5344902463244\\
0.015625	34.64103737199\\
0.01611328125	34.7479003916513\\
0.0166015625	34.8550812462798\\
0.01708984375	34.9625818931223\\
0.017578125	35.070404305831\\
0.01806640625	35.1785504745929\\
0.0185546875	35.2870224062575\\
0.01904296875	35.3958221244659\\
0.01953125	35.5049516697778\\
0.02001953125	35.6144130997982\\
0.0205078125	35.7242084893022\\
0.02099609375	35.8343399303607\\
0.021484375	35.9448095324626\\
0.02197265625	36.0556194226373\\
0.0224609375	36.1667717455755\\
0.02294921875	36.2782686637471\\
0.0234375	36.3901123575196\\
0.02392578125	36.5023050252722\\
0.0244140625	36.6148488835097\\
0.02490234375	36.7277461669728\\
0.025390625	36.840999128747\\
0.02587890625	36.9546100403679\\
0.0263671875	37.0685811919246\\
0.02685546875	37.1829148921588\\
0.02734375	37.2976134685618\\
0.02783203125	37.4126792674678\\
0.0283203125	37.5281146541418\\
0.02880859375	37.6439220128667\\
0.029296875	37.7601037470216\\
0.02978515625	37.8766622791608\\
0.0302734375	37.9936000510823\\
0.03076171875	38.1109195238963\\
0.03125	38.2286231780836\\
0.03173828125	38.3467135135514\\
0.0322265625	38.465193049681\\
0.03271484375	38.5840643253693\\
0.033203125	38.7033298990639\\
0.03369140625	38.822992348789\\
0.0341796875	38.9430542721655\\
0.03466796875	39.0635182864203\\
0.03515625	39.1843870283896\\
0.03564453125	39.3056631545099\\
0.0361328125	39.4273493408006\\
0.03662109375	39.5494482828362\\
0.037109375	39.6719626957063\\
0.03759765625	39.794895313965\\
0.0380859375	39.9182488915677\\
0.03857421875	40.0420262017939\\
0.0390625	40.1662300371571\\
0.03955078125	40.2908632093012\\
0.0400390625	40.415928548878\\
0.04052734375	40.541428905414\\
0.041015625	40.6673671471544\\
0.04150390625	40.7937461608941\\
0.0419921875	40.9205688517866\\
0.04248046875	41.0478381431348\\
0.04296875	41.1755569761605\\
0.04345703125	41.3037283097518\\
0.0439453125	41.4323551201874\\
0.04443359375	41.5614404008368\\
0.044921875	41.6909871618354\\
0.04541015625	41.8209984297321\\
0.0458984375	41.9514772471082\\
0.04638671875	42.0824266721687\\
0.046875	42.2138497783004\\
0.04736328125	42.3457496535982\\
0.0478515625	42.4781294003572\\
0.04833984375	42.6109921345276\\
0.048828125	42.7443409851332\\
0.04931640625	42.87817909365\\
0.0498046875	43.0125096133406\\
0.05029296875	43.1473357085485\\
0.05078125	43.2826605539427\\
0.05126953125	43.4184873337171\\
0.0517578125	43.5548192407359\\
0.05224609375	43.6916594756305\\
0.052734375	43.8290112458333\\
0.05322265625	43.9668777645613\\
0.0537109375	44.1052622497311\\
0.05419921875	44.2441679228141\\
0.0546875	44.3835980076211\\
0.05517578125	44.5235557290174\\
0.0556640625	44.6640443115632\\
0.05615234375	44.8050669780754\\
0.056640625	44.9466269481081\\
0.05712890625	45.0887274363474\\
0.0576171875	45.2313716509146\\
0.05810546875	45.3745627915771\\
0.05859375	45.5183040478603\\
0.05908203125	45.6625985970523\\
0.0595703125	45.8074496021053\\
0.06005859375	45.9528602094186\\
0.060546875	46.0988335465038\\
0.06103515625	46.2453727195264\\
0.0615234375	46.3924808107128\\
0.06201171875	46.5401608756232\\
0.0625	46.6884159402784\\
0.06298828125	46.8372489981368\\
0.0634765625	46.986663006914\\
0.06396484375	47.136660885235\\
0.064453125	47.2872455091144\\
0.06494140625	47.4384197082561\\
0.0654296875	47.5901862621607\\
0.06591796875	47.7425478960363\\
0.06640625	47.895507276498\\
0.06689453125	48.0490670070539\\
0.0673828125	48.2032296233585\\
0.06787109375	48.3579975882292\\
0.068359375	48.5133732864142\\
0.06884765625	48.6693590190966\\
0.0693359375	48.8259569981265\\
0.06982421875	48.9831693399679\\
0.0703125	49.1409980593434\\
0.07080078125	49.2994450625709\\
0.0712890625	49.4585121405698\\
0.07177734375	49.6182009615274\\
0.072265625	49.7785130632076\\
0.07275390625	49.9394498448875\\
0.0732421875	50.1010125589044\\
0.07373046875	50.2632023017968\\
0.07421875	50.4260200050194\\
0.07470703125	50.5894664252178\\
0.0751953125	50.7535421340383\\
0.07568359375	50.9182475074582\\
0.076171875	51.0835827146138\\
0.07666015625	51.249547706102\\
0.0771484375	51.4161422017404\\
0.07763671875	51.583365677757\\
0.078125	51.751217353389\\
0.07861328125	51.919696176866\\
0.0791015625	52.088800810757\\
0.07958984375	52.2585296166463\\
0.080078125	52.4288806391258\\
0.08056640625	52.5998515890638\\
0.0810546875	52.771439826135\\
0.08154296875	52.9436423405767\\
0.08203125	53.1164557341457\\
0.08251953125	53.2898762002489\\
0.0830078125	53.4638995032198\\
0.08349609375	53.6385209567066\\
0.083984375	53.8137354011501\\
0.08447265625	53.9895371803219\\
0.0849609375	54.1659201168878\\
0.08544921875	54.3428774869779\\
0.0859375	54.5204019937276\\
0.08642578125	54.6984857397678\\
0.0869140625	54.8771201986367\\
0.08740234375	55.0562961850889\\
0.087890625	55.2360038242758\\
0.08837890625	55.4162325197844\\
0.0888671875	55.5969709205\\
0.08935546875	55.7782068862964\\
0.08984375	55.9599274525198\\
0.09033203125	56.1421187932695\\
0.0908203125	56.3247661834639\\
0.09130859375	56.5078539596919\\
0.091796875	56.691365479847\\
0.09228515625	56.8752830815626\\
0.0927734375	57.059588039455\\
0.09326171875	57.2442605212002\\
0.09375	57.4292795424759\\
0.09423828125	57.6146229208039\\
0.0947265625	57.800267228348\\
0.09521484375	57.986187743719\\
0.095703125	58.1723584028661\\
0.09619140625	58.3587517491343\\
0.0966796875	58.5453388825891\\
0.09716796875	58.7320894087199\\
0.09765625	58.9189713866558\\
0.09814453125	59.1059512770399\\
0.0986328125	59.2929938897293\\
0.09912109375	59.4800623315112\\
0.099609375	59.6671179540473\\
0.10009765625	59.8541203022707\\
0.1005859375	60.0410270635107\\
0.10107421875	60.2277940176097\\
0.1015625	60.4143749883635\\
0.10205078125	60.6007217966128\\
0.1025390625	60.7867842153597\\
0.10302734375	60.9725099273116\\
0.103515625	61.1578444852854\\
0.10400390625	61.3427312759266\\
0.1044921875	61.5271114872523\\
0.10498046875	61.7109240805326\\
0.10546875	61.8941057670751\\
0.10595703125	62.0765909905007\\
0.1064453125	62.2583119151209\\
0.10693359375	62.439198421071\\
0.107421875	62.6191781068513\\
0.10791015625	62.7981762999779\\
0.1083984375	62.9761160764432\\
0.10888671875	63.1529182896863\\
0.109375	63.3285016098191\\
0.10986328125	63.5027825738038\\
0.1103515625	63.6756756473098\\
0.11083984375	63.8470932989259\\
0.111328125	64.0169460874106\\
0.11181640625	64.1851427625946\\
0.1123046875	64.3515903805218\\
0.11279296875	64.5161944333502\\
0.11328125	64.6788589944467\\
0.11376953125	64.8394868790474\\
0.1142578125	64.9979798207304\\
0.11474609375	65.154238663849\\
0.115234375	65.3081635719318\\
0.11572265625	65.4596542519345\\
0.1162109375	65.608610194045\\
0.11669921875	65.7549309265994\\
0.1171875	65.8985162854743\\
0.11767578125	66.0392666971524\\
0.1181640625	66.1770834744327\\
0.11865234375	66.3118691235947\\
0.119140625	66.4435276615992\\
0.11962890625	66.5719649417255\\
0.1201171875	66.6970889858486\\
0.12060546875	66.818810321365\\
0.12109375	66.9370423206299\\
0.12158203125	67.0517015405732\\
0.1220703125	67.1627080600788\\
0.12255859375	67.2699858125607\\
0.123046875	67.373462911098\\
0.12353515625	67.4730719634631\\
0.1240234375	67.5687503743274\\
0.12451171875	67.6604406319817\\
0.125	67.7480905769443\\
0.12548828125	67.8316536499372\\
0.1259765625	67.9110891168613\\
0.12646484375	67.9863622685426\\
0.126953125	68.0574445932747\\
0.12744140625	68.1243139203952\\
0.1279296875	68.1869545334193\\
0.12841796875	68.2453572515712\\
0.12890625	68.2995194788447\\
0.12939453125	68.3494452200852\\
0.1298828125	68.3951450639242\\
0.13037109375	68.4366361327461\\
0.130859375	68.473942000207\\
0.13134765625	68.507092577182\\
0.1318359375	68.5361239673076\\
0.13232421875	68.5610782936153\\
0.1328125	68.5820034980124\\
0.13330078125	68.5989531156062\\
0.1337890625	68.6119860260928\\
0.13427734375	68.6211661845943\\
0.134765625	68.6265623344556\\
0.13525390625	68.6282477046552\\
0.1357421875	68.6262996944571\\
0.13623046875	68.6207995480509\\
0.13671875	68.6118320218278\\
0.13720703125	68.5994850469464\\
0.1376953125	68.5838493897143\\
0.13818359375	68.5650183122505\\
0.138671875	68.5430872357058\\
0.13916015625	68.5181534082104\\
0.1396484375	68.490315579509\\
0.14013671875	68.4596736840965\\
0.140625	68.4263285344224\\
0.14111328125	68.3903815256019\\
0.1416015625	68.3519343527999\\
0.14208984375	68.3110887423177\\
0.142578125	68.2679461971756\\
0.14306640625	68.2226077578204\\
0.1435546875	68.1751737784133\\
0.14404296875	68.1257437189494\\
0.14453125	68.0744159533665\\
0.14501953125	68.0212875935985\\
0.1455078125	67.9664543294486\\
0.14599609375	67.9100102839985\\
0.146484375	67.8520478842186\\
0.14697265625	67.7926577463039\\
0.1474609375	67.7319285752479\\
0.14794921875	67.6699470780346\\
0.1484375	67.6067978898587\\
0.14892578125	67.542563512678\\
0.1494140625	67.4773242654065\\
0.14990234375	67.4111582450701\\
0.150390625	67.3441412981548\\
0.15087890625	67.2763470014723\\
0.1513671875	67.2078466517999\\
0.15185546875	67.1387092636084\\
0.15234375	67.0690015741769\\
0.15283203125	66.9987880554412\\
0.1533203125	66.928130931931\\
0.15380859375	66.8570902041634\\
0.154296875	66.7857236769358\\
0.15478515625	66.7140869919337\\
0.1552734375	66.6422336641444\\
0.15576171875	66.5702151215695\\
0.15625	66.4980807477853\\
0.15673828125	66.4258779269178\\
0.1572265625	66.3536520906209\\
0.15771484375	66.2814467667086\\
0.158203125	66.2093036290815\\
0.15869140625	66.1372625486528\\
0.1591796875	66.0653616449764\\
0.15966796875	65.9936373383301\\
0.16015625	65.922124402015\\
0.16064453125	65.8508560146587\\
0.1611328125	65.779863812345\\
0.16162109375	65.7091779403923\\
0.162109375	65.6388271046365\\
0.16259765625	65.5688386220889\\
0.1630859375	65.4992384708582\\
0.16357421875	65.4300513392315\\
0.1640625	65.3613006738333\\
0.16455078125	65.2930087268026\\
0.1650390625	65.2251966018996\\
0.16552734375	65.1578842995313\\
0.166015625	65.0910907606221\\
0.16650390625	65.0248339093263\\
0.1669921875	64.9591306945394\\
0.16748046875	64.8939971302117\\
0.16796875	64.8294483344365\\
0.16845703125	64.765498567328\\
0.1689453125	64.7021612676794\\
0.16943359375	64.6394490884088\\
0.169921875	64.5773739308102\\
0.17041015625	64.5159469776142\\
0.1708984375	64.4551787248847\\
0.17138671875	64.395079012756\\
0.171875	64.3356570550551\\
0.17236328125	64.2769214678041\\
0.1728515625	64.2188802966542\\
0.17333984375	64.1615410432521\\
0.173828125	64.1049106905909\\
0.17431640625	64.0489957273481\\
0.1748046875	63.9938021712587\\
0.17529296875	63.9393355915415\\
0.17578125	63.8856011304059\\
0.17626953125	63.8326035236743\\
0.1767578125	63.7803471205445\\
0.17724609375	63.7288359025212\\
0.177734375	63.6780735015432\\
0.17822265625	63.6280632173374\\
0.1787109375	63.5788080340221\\
0.17919921875	63.5303106359894\\
0.1796875	63.4825734230952\\
0.18017578125	63.4355985251746\\
0.1806640625	63.3893878159154\\
0.18115234375	63.3439429261107\\
0.181640625	63.299265256314\\
0.18212890625	63.2553559889231\\
0.1826171875	63.2122160997064\\
0.18310546875	63.1698463688066\\
0.18359375	63.1282473912324\\
0.18408203125	63.0874195868572\\
0.1845703125	63.047363209958\\
0.18505859375	63.0080783582955\\
0.185546875	62.9695649817642\\
0.18603515625	62.9318228906344\\
0.1865234375	62.8948517633849\\
0.18701171875	62.8586511541637\\
0.1875	62.8232204998817\\
0.18798828125	62.7885591269488\\
0.1884765625	62.7546662576806\\
0.18896484375	62.7215410163833\\
0.189453125	62.6891824351213\\
0.18994140625	62.6575894591928\\
0.1904296875	62.6267609523252\\
0.19091796875	62.5966957015902\\
0.19140625	62.5673924220595\\
0.19189453125	62.5388497612104\\
0.1923828125	62.5110663030912\\
0.19287109375	62.4840405722522\\
0.193359375	62.4577710374555\\
0.19384765625	62.4322561151766\\
0.1943359375	62.4074941728919\\
0.19482421875	62.3834835321764\\
0.1953125	62.3602224716126\\
0.19580078125	62.3377092295109\\
0.1962890625	62.3159420064648\\
0.19677734375	62.2949189677278\\
0.197265625	62.2746382454376\\
0.19775390625	62.2550979406739\\
0.1982421875	62.236296125377\\
0.19873046875	62.2182308441128\\
0.19921875	62.2009001156996\\
0.19970703125	62.1843019347035\\
0.2001953125	62.1684342727972\\
0.20068359375	62.1532950799955\\
0.201171875	62.1388822857679\\
0.20166015625	62.1251938000335\\
0.2021484375	62.1122275140364\\
0.20263671875	62.0999813011155\\
0.203125	62.0884530173579\\
0.20361328125	62.0776405021558\\
0.2041015625	62.0675415786503\\
0.20458984375	62.0581540540822\\
0.205078125	62.0494757200406\\
0.20556640625	62.0415043526159\\
0.2060546875	62.03423771246\\
0.20654296875	62.0276735447525\\
0.20703125	62.0218095790778\\
0.20751953125	62.0166435292119\\
0.2080078125	62.0121730928189\\
0.20849609375	62.0083959510694\\
0.208984375	62.0053097681628\\
0.20947265625	62.0029121907714\\
0.2099609375	62.0012008473988\\
0.21044921875	62.0001733476567\\
0.2109375	61.9998272814547\\
0.21142578125	62.0001602181154\\
0.2119140625	62.0011697053987\\
0.21240234375	62.0028532684513\\
0.212890625	62.0052084086716\\
0.21337890625	62.008232602492\\
0.2138671875	62.0119233000826\\
0.21435546875	62.0162779239678\\
0.21484375	62.0212938675658\\
0.21533203125	62.0269684936409\\
0.2158203125	62.033299132673\\
0.21630859375	62.0402830811461\\
0.216796875	62.0479175997457\\
0.21728515625	62.0561999114769\\
0.2177734375	62.0651271996897\\
0.21826171875	62.0746966060223\\
0.21875	62.0849052282489\\
0.21923828125	62.0957501180427\\
0.2197265625	62.1072282786468\\
0.22021484375	62.1193366624472\\
0.220703125	62.132072168458\\
0.22119140625	62.1454316397086\\
0.2216796875	62.1594118605303\\
0.22216796875	62.1740095537494\\
0.22265625	62.1892213777744\\
0.22314453125	62.2050439235853\\
0.2236328125	62.221473711615\\
0.22412109375	62.2385071885269\\
0.224609375	62.2561407238828\\
0.22509765625	62.2743706067021\\
0.2255859375	62.2931930419068\\
0.22607421875	62.3126041466553\\
0.2265625	62.3325999465577\\
0.22705078125	62.3531763717734\\
0.2275390625	62.3743292529885\\
0.22802734375	62.39605431727\\
0.228515625	62.4183471837958\\
0.22900390625	62.4412033594579\\
0.2294921875	62.4646182343346\\
0.22998046875	62.4885870770352\\
0.23046875	62.513105029909\\
0.23095703125	62.5381671041189\\
0.2314453125	62.563768174576\\
0.23193359375	62.5899029747423\\
0.232421875	62.6165660912822\\
0.23291015625	62.643751958577\\
0.2333984375	62.6714548530958\\
0.23388671875	62.6996688876167\\
0.234375	62.7283880053003\\
0.23486328125	62.7576059736186\\
0.2353515625	62.7873163781298\\
0.23583984375	62.8175126161029\\
0.236328125	62.8481878899926\\
0.23681640625	62.8793352007589\\
0.2373046875	62.9109473410374\\
0.23779296875	62.9430168881553\\
0.23828125	62.975536196997\\
0.23876953125	63.0084973927149\\
0.2392578125	63.0418923632963\\
0.23974609375	63.0757127519735\\
0.240234375	63.1099499494939\\
0.24072265625	63.1445950862436\\
0.2412109375	63.1796390242248\\
0.24169921875	63.2150723489037\\
0.2421875	63.2508853609146\\
0.24267578125	63.2870680676423\\
0.2431640625	63.3236101746757\\
0.24365234375	63.3605010771425\\
0.244140625	63.3977298509367\\
0.24462890625	63.4352852438367\\
0.2451171875	63.4731556665319\\
0.24560546875	63.5113291835604\\
0.24609375	63.5497935041744\\
0.24658203125	63.5885359731352\\
0.2470703125	63.6275435614624\\
0.24755859375	63.6668028571395\\
0.248046875	63.7063000558021\\
0.24853515625	63.7460209514101\\
0.2490234375	63.785950926937\\
0.24951171875	63.8260749450857\\
0.25	63.8663775390504\\
0.25048828125	63.9068428033551\\
0.2509765625	63.9474543847778\\
0.25146484375	63.9881954734015\\
0.251953125	64.0290487938071\\
0.25244140625	64.0699965964391\\
0.2529296875	64.1110206491774\\
0.25341796875	64.1521022291463\\
0.25390625	64.1932221147871\\
0.25439453125	64.2343605782455\\
0.2548828125	64.2754973780937\\
0.25537109375	64.3166117524416\\
0.255859375	64.3576824124679\\
0.25634765625	64.3986875364226\\
0.2568359375	64.4396047641425\\
0.25732421875	64.4804111921245\\
0.2578125	64.5210833692139\\
0.25830078125	64.5615972929523\\
0.2587890625	64.6019284066407\\
0.25927734375	64.6420515971703\\
0.259765625	64.6819411936817\\
0.26025390625	64.7215709671072\\
0.2607421875	64.7609141306569\\
0.26123046875	64.7999433413055\\
0.26171875	64.8386307023475\\
0.26220703125	64.8769477670789\\
0.2626953125	64.9148655436697\\
0.26318359375	64.9523545012855\\
0.263671875	64.9893845775344\\
0.26416015625	65.0259251872855\\
0.2646484375	65.061945232934\\
0.26513671875	65.0974131161666\\
0.265625	65.1322967512939\\
0.26611328125	65.1665635801981\\
0.2666015625	65.2001805889639\\
0.26708984375	65.2331143262316\\
0.267578125	65.2653309233368\\
0.26806640625	65.296796116268\\
0.2685546875	65.3274752694932\\
0.26904296875	65.3573334016888\\
0.26953125	65.3863352133966\\
0.27001953125	65.4144451166446\\
0.2705078125	65.4416272665328\\
0.27099609375	65.4678455948117\\
0.271484375	65.4930638454413\\
0.27197265625	65.5172456121308\\
0.2724609375	65.5403543778447\\
0.27294921875	65.5623535562409\\
0.2734375	65.5832065350169\\
0.27392578125	65.6028767211069\\
0.2744140625	65.621327587679\\
0.27490234375	65.638522722856\\
0.275390625	65.6544258800869\\
0.27587890625	65.6690010300671\\
0.2763671875	65.682212414108\\
0.27685546875	65.6940245988302\\
0.27734375	65.7044025320597\\
0.27783203125	65.713311599773\\
0.2783203125	65.7207176839468\\
0.27880859375	65.7265872211332\\
0.279296875	65.7308872615974\\
0.27978515625	65.7335855288169\\
0.2802734375	65.7346504791551\\
0.28076171875	65.7340513614861\\
0.28125	65.7317582765768\\
0.28173828125	65.7277422359868\\
0.2822265625	65.7219752202653\\
0.28271484375	65.7144302362153\\
0.283203125	65.7050813729819\\
0.28369140625	65.6939038567378\\
0.2841796875	65.6808741037191\\
0.28466796875	65.6659697713852\\
0.28515625	65.6491698074676\\
0.28564453125	65.6304544966762\\
0.2861328125	65.6098055048512\\
0.28662109375	65.5872059203492\\
0.287109375	65.5626402924504\\
0.28759765625	65.5360946666108\\
0.2880859375	65.5075566163789\\
0.28857421875	65.4770152718139\\
0.2890625	65.4444613442611\\
0.28955078125	65.4098871473572\\
0.2900390625	65.3732866141582\\
0.29052734375	65.334655310297\\
0.291015625	65.2939904431037\\
0.29150390625	65.2512908666424\\
0.2919921875	65.2065570826316\\
0.29248046875	65.1597912372508\\
0.29296875	65.1109971138428\\
0.29345703125	65.0601801215563\\
0.2939453125	65.0073472799794\\
0.29443359375	64.9525071998549\\
0.294921875	64.895670059969\\
0.29541015625	64.8368475803313\\
0.2958984375	64.7760529917929\\
0.29638671875	64.7133010022453\\
0.296875	64.6486077595741\\
0.29736328125	64.5819908115531\\
0.2978515625	64.513469062868\\
0.29833984375	64.4430627294851\\
0.298828125	64.3707932905735\\
0.29931640625	64.2966834382073\\
0.2998046875	64.2207570250781\\
0.30029296875	64.1430390104533\\
0.30078125	64.0635554046066\\
0.30126953125	63.9823332119756\\
0.3017578125	63.8994003732601\\
0.30224609375	63.8147857067157\\
0.302734375	63.7285188488598\\
0.30322265625	63.6406301948195\\
0.3037109375	63.5511508385385\\
0.30419921875	63.4601125130585\\
0.3046875	63.3675475310672\\
0.30517578125	63.2734887259216\\
0.3056640625	63.177969393314\\
0.30615234375	63.0810232337599\\
0.306640625	62.9826842960716\\
0.30712890625	62.8829869219595\\
0.3076171875	62.7819656918977\\
0.30810546875	62.6796553723902\\
0.30859375	62.5760908647271\\
0.30908203125	62.4713071553555\\
0.3095703125	62.3653392679329\\
0.31005859375	62.2582222171491\\
0.310546875	62.1499909643744\\
0.31103515625	62.0406803751904\\
0.3115234375	61.9303251788428\\
0.31201171875	61.8189599296428\\
0.3125	61.7066189703459\\
0.31298828125	61.5933363975092\\
0.3134765625	61.4791460288398\\
0.31396484375	61.3640813725167\\
0.314453125	61.2481755984836\\
0.31494140625	61.1314615116809\\
0.3154296875	61.0139715271973\\
0.31591796875	60.8957376473028\\
0.31640625	60.7767914403265\\
0.31689453125	60.6571640213358\\
0.3173828125	60.5368860345645\\
0.31787109375	60.4159876375476\\
0.318359375	60.2944984869015\\
0.31884765625	60.17244772569\\
0.3193359375	60.0498639723292\\
0.31982421875	59.926775310957\\
0.3203125	59.8032092832153\\
0.32080078125	59.6791928813735\\
0.3212890625	59.5547525427414\\
0.32177734375	59.4299141452967\\
0.322265625	59.3047030044703\\
0.32275390625	59.1791438710272\\
0.3232421875	59.0532609299785\\
0.32373046875	58.9270778004637\\
0.32421875	58.8006175365463\\
0.32470703125	58.6739026288586\\
0.3251953125	58.5469550070475\\
0.32568359375	58.4197960429551\\
0.326171875	58.2924465544856\\
0.32666015625	58.1649268101129\\
0.3271484375	58.0372565339621\\
0.32763671875	57.9094549114336\\
0.328125	57.7815405953143\\
0.32861328125	57.6535317123304\\
0.3291015625	57.5254458701059\\
0.32958984375	57.3973001644807\\
0.330078125	57.2691111871506\\
0.33056640625	57.1408950335968\\
0.3310546875	57.0126673112599\\
0.33154296875	56.8844431479406\\
0.33203125	56.756237200383\\
0.33251953125	56.6280636630173\\
0.3330078125	56.4999362768329\\
0.33349609375	56.3718683383577\\
0.333984375	56.2438727087193\\
0.33447265625	56.1159618227632\\
0.3349609375	55.9881476982082\\
0.33544921875	55.8604419448235\\
0.3359375	55.7328557736034\\
0.33642578125	55.6054000059252\\
0.3369140625	55.4780850826788\\
0.33740234375	55.3509210733456\\
0.337890625	55.223917685024\\
0.33837890625	55.0970842713796\\
0.3388671875	54.9704298415177\\
0.33935546875	54.8439630687616\\
0.33984375	54.7176922993331\\
0.34033203125	54.591625560925\\
0.3408203125	54.4657705711571\\
0.34130859375	54.3401347459125\\
0.341796875	54.2147252075465\\
0.34228515625	54.0895487929641\\
0.3427734375	53.9646120615619\\
0.34326171875	53.8399213030287\\
0.34375	53.7154825450069\\
0.34423828125	53.5913015606067\\
0.3447265625	53.4673838757739\\
0.34521484375	53.3437347765084\\
0.345703125	53.2203593159358\\
0.34619140625	53.0972623212247\\
0.3466796875	52.9744484003566\\
0.34716796875	52.8519219487433\\
0.34765625	52.7296871556939\\
0.34814453125	52.6077480107328\\
0.3486328125	52.4861083097663\\
0.34912109375	52.3647716611035\\
0.349609375	52.243741491328\\
0.35009765625	52.1230210510254\\
0.3505859375	52.0026134203654\\
0.35107421875	51.8825215145424\\
0.3515625	51.762748089076\\
0.35205078125	51.6432957449702\\
0.3525390625	51.5241669337407\\
0.35302734375	51.4053639623021\\
0.353515625	51.2868889977279\\
0.35400390625	51.168744071879\\
0.3544921875	51.0509310859025\\
0.35498046875	50.9334518146083\\
0.35546875	50.816307910722\\
0.35595703125	50.6995009090175\\
0.3564453125	50.5830322303313\\
0.35693359375	50.4669031854634\\
0.357421875	50.3511149789633\\
0.35791015625	50.2356687128071\\
0.3583984375	50.1205653899669\\
0.35888671875	50.0058059178747\\
0.359375	49.891391111784\\
0.35986328125	49.7773216980302\\
0.3603515625	49.6635983171944\\
0.36083984375	49.5502215271711\\
0.361328125	49.4371918061444\\
0.36181640625	49.3245095554704\\
0.3623046875	49.2121751024763\\
0.36279296875	49.1001887031679\\
0.36328125	48.9885505448586\\
0.36376953125	48.8772607487135\\
0.3642578125	48.766319372216\\
0.36474609375	48.655726411557\\
0.365234375	48.5454818039489\\
0.36572265625	48.4355854298671\\
0.3662109375	48.32603711522\\
0.36669921875	48.2168366334516\\
0.3671875	48.1079837075755\\
0.36767578125	47.9994780121456\\
0.3681640625	47.8913191751613\\
0.36865234375	47.7835067799153\\
0.369140625	47.676040366778\\
0.36962890625	47.5689194349264\\
0.3701171875	47.4621434440153\\
0.37060546875	47.3557118157973\\
0.37109375	47.2496239356861\\
0.37158203125	47.1438791542706\\
0.3720703125	47.03847678878\\
0.37255859375	46.9334161244988\\
0.373046875	46.8286964161373\\
0.37353515625	46.7243168891547\\
0.3740234375	46.6202767410396\\
0.37451171875	46.5165751425477\\
0.375	46.4132112388985\\
0.37548828125	46.3101841509302\\
0.3759765625	46.2074929762189\\
0.37646484375	46.1051367901579\\
0.376953125	46.0031146470012\\
0.37744140625	45.9014255808724\\
0.3779296875	45.8000686067381\\
0.37841796875	45.6990427213499\\
0.37890625	45.5983469041526\\
0.37939453125	45.4979801181629\\
0.3798828125	45.3979413108167\\
0.38037109375	45.298229414789\\
0.380859375	45.1988433487835\\
0.38134765625	45.0997820182961\\
0.3818359375	45.0010443163522\\
0.38232421875	44.9026291242179\\
0.3828125	44.8045353120864\\
0.38330078125	44.7067617397403\\
0.3837890625	44.6093072571918\\
0.38427734375	44.5121707052986\\
0.384765625	44.4153509163596\\
0.38525390625	44.3188467146886\\
0.3857421875	44.2226569171687\\
0.38623046875	44.1267803337849\\
0.38671875	44.0312157681407\\
0.38720703125	43.9359620179531\\
0.3876953125	43.8410178755317\\
0.38818359375	43.7463821282406\\
0.388671875	43.6520535589418\\
0.38916015625	43.5580309464241\\
0.3896484375	43.4643130658164\\
0.39013671875	43.3708986889841\\
0.390625	43.2777865849128\\
0.39111328125	43.1849755200774\\
0.3916015625	43.0924642587959\\
0.39208984375	43.0002515635722\\
0.392578125	42.908336195425\\
0.39306640625	42.816716914204\\
0.3935546875	42.7253924788943\\
0.39404296875	42.63436164791\\
0.39453125	42.5436231793755\\
0.39501953125	42.4531758313964\\
0.3955078125	42.3630183623199\\
0.39599609375	42.2731495309855\\
0.396484375	42.1835680969649\\
0.39697265625	42.0942728207927\\
0.3974609375	42.0052624641892\\
0.39794921875	41.9165357902727\\
0.3984375	41.8280915637635\\
0.39892578125	41.739928551181\\
0.3994140625	41.6520455210307\\
0.39990234375	41.5644412439853\\
0.400390625	41.4771144930569\\
0.40087890625	41.390064043763\\
0.4013671875	41.3032886742855\\
0.40185546875	41.2167871656214\\
0.40234375	41.1305583017297\\
0.40283203125	41.0446008696688\\
0.4033203125	40.9589136597308\\
0.40380859375	40.8734954655676\\
0.404296875	40.7883450843131\\
0.40478515625	40.703461316698\\
0.4052734375	40.6188429671613\\
0.40576171875	40.5344888439555\\
0.40625	40.4503977592461\\
0.40673828125	40.3665685292092\\
0.4072265625	40.2829999741204\\
0.40771484375	40.1996909184432\\
0.408203125	40.11664019091\\
0.40869140625	40.0338466246006\\
0.4091796875	39.9513090570165\\
0.40966796875	39.8690263301511\\
0.41015625	39.7869972905557\\
0.41064453125	39.7052207894026\\
0.4111328125	39.623695682545\\
0.41162109375	39.5424208305717\\
0.412109375	39.4613950988609\\
0.41259765625	39.3806173576297\\
0.4130859375	39.3000864819798\\
0.41357421875	39.219801351942\\
0.4140625	39.1397608525168\\
0.41455078125	39.059963873712\\
0.4150390625	38.980409310579\\
0.41552734375	38.901096063245\\
0.416015625	38.8220230369438\\
0.41650390625	38.743189142044\\
0.4169921875	38.6645932940744\\
0.41748046875	38.5862344137489\\
0.41796875	38.508111426986\\
0.41845703125	38.4302232649307\\
0.4189453125	38.3525688639704\\
0.41943359375	38.2751471657515\\
0.419921875	38.1979571171925\\
0.42041015625	38.1209976704972\\
0.4208984375	38.0442677831639\\
0.42138671875	37.9677664179958\\
0.421875	37.8914925431066\\
0.42236328125	37.8154451319273\\
0.4228515625	37.7396231632105\\
0.42333984375	37.6640256210331\\
0.423828125	37.5886514947977\\
0.42431640625	37.5134997792335\\
0.4248046875	37.438569474395\\
0.42529296875	37.3638595856598\\
0.42578125	37.2893691237256\\
0.42626953125	37.215097104606\\
0.4267578125	37.1410425496247\\
0.42724609375	37.0672044854093\\
0.427734375	36.9935819438839\\
0.42822265625	36.920173962261\\
0.4287109375	36.8469795830321\\
0.42919921875	36.7739978539576\\
0.4296875	36.7012278280561\\
0.43017578125	36.6286685635926\\
0.4306640625	36.5563191240661\\
0.43115234375	36.4841785781968\\
0.431640625	36.4122459999118\\
0.43212890625	36.3405204683309\\
0.4326171875	36.2690010677513\\
0.43310546875	36.1976868876324\\
0.43359375	36.1265770225784\\
0.43408203125	36.0556705723231\\
0.4345703125	35.9849666417109\\
0.43505859375	35.9144643406803\\
0.435546875	35.8441627842443\\
0.43603515625	35.7740610924727\\
0.4365234375	35.7041583904724\\
0.43701171875	35.6344538083675\\
0.4375	35.5649464812801\\
0.43798828125	35.4956355493087\\
0.4384765625	35.4265201575089\\
0.43896484375	35.357599455871\\
0.439453125	35.2888725992996\\
0.43994140625	35.2203387475913\\
0.4404296875	35.151997065413\\
0.44091796875	35.0838467222797\\
0.44140625	35.015886892532\\
0.44189453125	34.9481167553133\\
0.4423828125	34.880535494547\\
0.44287109375	34.8131422989131\\
0.443359375	34.7459363618254\\
0.44384765625	34.6789168814076\\
0.4443359375	34.6120830604697\\
0.44482421875	34.545434106484\\
0.4453125	34.4789692315618\\
0.44580078125	34.4126876524288\\
0.4462890625	34.3465885904006\\
0.44677734375	34.2806712713592\\
0.447265625	34.2149349257278\\
0.44775390625	34.1493787884468\\
0.4482421875	34.0840020989492\\
0.44873046875	34.0188041011357\\
0.44921875	33.9537840433496\\
0.44970703125	33.8889411783535\\
0.4501953125	33.8242747633028\\
0.45068359375	33.7597840597217\\
0.451171875	33.6954683334781\\
0.45166015625	33.6313268547591\\
0.4521484375	33.5673588980453\\
0.45263671875	33.5035637420869\\
0.453125	33.4399406698775\\
0.45361328125	33.3764889686304\\
0.4541015625	33.3132079297528\\
0.45458984375	33.2500968488216\\
0.455078125	33.1871550255577\\
0.45556640625	33.1243817638019\\
0.4560546875	33.0617763714898\\
0.45654296875	32.9993381606267\\
0.45703125	32.9370664472635\\
0.45751953125	32.8749605514713\\
0.4580078125	32.8130197973173\\
0.45849609375	32.7512435128397\\
0.458984375	32.6896310300235\\
0.45947265625	32.6281816847762\\
0.4599609375	32.5668948169027\\
0.46044921875	32.5057697700817\\
0.4609375	32.4448058918408\\
0.46142578125	32.3840025335328\\
0.4619140625	32.3233590503111\\
0.46240234375	32.2628748011062\\
0.462890625	32.2025491486012\\
0.46337890625	32.142381459208\\
0.4638671875	32.082371103044\\
0.46435546875	32.0225174539081\\
0.46484375	31.9628198892566\\
0.46533203125	31.9032777901805\\
0.4658203125	31.8438905413816\\
0.46630859375	31.7846575311495\\
0.466796875	31.7255781513379\\
0.46728515625	31.6666517973419\\
0.4677734375	31.6078778680752\\
0.46826171875	31.5492557659462\\
0.46875	31.4907848968368\\
0.46923828125	31.4324646700778\\
0.4697265625	31.3742944984282\\
0.47021484375	31.3162737980514\\
0.470703125	31.2584019884934\\
0.47119140625	31.2006784926607\\
0.4716796875	31.1431027367976\\
0.47216796875	31.0856741504651\\
0.47265625	31.028392166518\\
0.47314453125	30.9712562210838\\
0.4736328125	30.9142657535409\\
0.47412109375	30.857420206497\\
0.474609375	30.8007190257678\\
0.47509765625	30.7441616603555\\
0.4755859375	30.6877475624278\\
0.47607421875	30.6314761872969\\
0.4765625	30.5753469933984\\
0.47705078125	30.5193594422705\\
0.4775390625	30.4635129985338\\
0.47802734375	30.4078071298699\\
0.478515625	30.3522413070014\\
0.47900390625	30.2968150036719\\
0.4794921875	30.2415276966254\\
0.47998046875	30.1863788655862\\
0.48046875	30.1313679932392\\
0.48095703125	30.0764945652099\\
0.4814453125	30.0217580700449\\
0.48193359375	29.9671579991922\\
0.482421875	29.9126938469815\\
0.48291015625	29.8583651106057\\
0.4833984375	29.8041712901004\\
0.48388671875	29.7501118883262\\
0.484375	29.6961864109488\\
0.48486328125	29.6423943664205\\
0.4853515625	29.5887352659616\\
0.48583984375	29.5352086235417\\
0.486328125	29.481813955861\\
0.48681640625	29.4285507823325\\
0.4873046875	29.3754186250633\\
0.48779296875	29.3224170088367\\
0.48828125	29.2695454610942\\
0.48876953125	29.2168035119179\\
0.4892578125	29.1641906940122\\
0.48974609375	29.1117065426866\\
0.490234375	29.0593505958384\\
0.49072265625	29.0071223939346\\
0.4912109375	28.9550214799956\\
0.49169921875	28.9030473995772\\
0.4921875	28.8511997007541\\
0.49267578125	28.7994779341029\\
0.4931640625	28.7478816526853\\
0.49365234375	28.6964104120312\\
0.494140625	28.6450637701228\\
0.49462890625	28.5938412873772\\
0.4951171875	28.542742526631\\
0.49560546875	28.4917670531237\\
0.49609375	28.4409144344816\\
0.49658203125	28.3901842407018\\
0.4970703125	28.3395760441364\\
0.49755859375	28.2890894194769\\
0.498046875	28.2387239437383\\
0.49853515625	28.1884791962438\\
0.4990234375	28.1383547586093\\
0.49951171875	28.0883502147278\\
0.5	28.0384651507549\\
0.50048828125	27.9886991550931\\
0.5009765625	27.9390518183771\\
0.50146484375	27.8895227334589\\
0.501953125	27.8401114953928\\
0.50244140625	27.7908177014213\\
0.5029296875	27.7416409509599\\
0.50341796875	27.6925808455832\\
0.50390625	27.6436369890101\\
0.50439453125	27.5948089870901\\
0.5048828125	27.5460964477886\\
0.50537109375	27.4974989811733\\
0.505859375	27.4490161994\\
0.50634765625	27.4006477166991\\
0.5068359375	27.3523931493615\\
0.50732421875	27.3042521157253\\
0.5078125	27.2562242361617\\
0.50830078125	27.2083091330628\\
0.5087890625	27.1605064308267\\
0.50927734375	27.1128157558457\\
0.509765625	27.0652367364922\\
0.51025390625	27.0177690031058\\
0.5107421875	26.9704121879811\\
0.51123046875	26.923165925354\\
0.51171875	26.8760298513893\\
0.51220703125	26.8290036041682\\
0.5126953125	26.7820868236752\\
0.51318359375	26.7352791517865\\
0.513671875	26.6885802322567\\
0.51416015625	26.6419897107072\\
0.5146484375	26.5955072346136\\
0.51513671875	26.549132453294\\
0.515625	26.5028650178964\\
0.51611328125	26.4567045813875\\
0.5166015625	26.4106507985402\\
0.51708984375	26.3647033259222\\
0.517578125	26.3188618218842\\
0.51806640625	26.2731259465487\\
0.5185546875	26.227495361798\\
0.51904296875	26.1819697312628\\
0.51953125	26.1365487203117\\
0.52001953125	26.0912319960386\\
0.5205078125	26.046019227253\\
0.52099609375	26.0009100844677\\
0.521484375	25.9559042398888\\
0.52197265625	25.911001367404\\
0.5224609375	25.8662011425723\\
0.52294921875	25.8215032426131\\
0.5234375	25.7769073463954\\
0.52392578125	25.7324131344274\\
0.5244140625	25.6880202888459\\
0.52490234375	25.643728493406\\
0.525390625	25.5995374334706\\
0.52587890625	25.5554467959999\\
0.5263671875	25.511456269542\\
0.52685546875	25.4675655442219\\
0.52734375	25.4237743117318\\
0.52783203125	25.3800822653212\\
0.5283203125	25.3364890997871\\
0.52880859375	25.2929945114637\\
0.529296875	25.2495981982131\\
0.52978515625	25.2062998594156\\
0.5302734375	25.1630991959598\\
0.53076171875	25.1199959102332\\
0.53125	25.076989706113\\
0.53173828125	25.0340802889561\\
0.5322265625	24.9912673655905\\
0.53271484375	24.9485506443055\\
0.533203125	24.9059298348426\\
0.53369140625	24.8634046483867\\
0.5341796875	24.8209747975566\\
0.53466796875	24.7786399963964\\
0.53515625	24.7363999603661\\
0.53564453125	24.6942544063334\\
0.5361328125	24.6522030525644\\
0.53662109375	24.610245618715\\
0.537109375	24.5683818258223\\
0.53759765625	24.5266113962959\\
0.5380859375	24.4849340539095\\
0.53857421875	24.4433495237925\\
0.5390625	24.4018575324212\\
0.53955078125	24.3604578076107\\
0.5400390625	24.319150078507\\
0.54052734375	24.2779340755779\\
0.541015625	24.2368095306057\\
0.54150390625	24.1957761766784\\
0.5419921875	24.1548337481823\\
0.54248046875	24.1139819807934\\
0.54296875	24.0732206114701\\
0.54345703125	24.0325493784446\\
0.5439453125	23.9919680212157\\
0.54443359375	23.9514762805409\\
0.544921875	23.9110738984284\\
0.54541015625	23.8707606181297\\
0.5458984375	23.8305361841324\\
0.54638671875	23.7904003421516\\
0.546875	23.7503528391237\\
0.54736328125	23.7103934231979\\
0.5478515625	23.6705218437295\\
0.54833984375	23.6307378512723\\
0.548828125	23.5910411975717\\
0.54931640625	23.5514316355568\\
0.5498046875	23.5119089193339\\
0.55029296875	23.472472804179\\
0.55078125	23.4331230465309\\
0.55126953125	23.3938594039845\\
0.5517578125	23.354681635283\\
0.55224609375	23.315589500312\\
0.552734375	23.2765827600919\\
0.55322265625	23.2376611767716\\
0.5537109375	23.1988245136213\\
0.55419921875	23.1600725350262\\
0.5546875	23.1214050064797\\
0.55517578125	23.0828216945764\\
0.5556640625	23.0443223670064\\
0.55615234375	23.0059067925481\\
0.556640625	22.9675747410616\\
0.55712890625	22.9293259834831\\
0.5576171875	22.8911602918177\\
0.55810546875	22.8530774391335\\
0.55859375	22.8150771995553\\
0.55908203125	22.777159348258\\
0.5595703125	22.7393236614611\\
0.56005859375	22.7015699164216\\
0.560546875	22.6638978914289\\
0.56103515625	22.6263073657981\\
0.5615234375	22.588798119864\\
0.56201171875	22.5513699349755\\
0.5625	22.5140225934892\\
0.56298828125	22.476755878764\\
0.5634765625	22.4395695751549\\
0.56396484375	22.402463468007\\
0.564453125	22.3654373436505\\
0.56494140625	22.3284909893942\\
0.5654296875	22.29162419352\\
0.56591796875	22.2548367452778\\
0.56640625	22.218128434879\\
0.56689453125	22.1814990534916\\
0.5673828125	22.1449483932344\\
0.56787109375	22.1084762471716\\
0.568359375	22.0720824093075\\
0.56884765625	22.0357666745805\\
0.5693359375	21.9995288388585\\
0.56982421875	21.9633686989331\\
0.5703125	21.9272860525144\\
0.57080078125	21.8912806982257\\
0.5712890625	21.8553524355984\\
0.57177734375	21.8195010650667\\
0.572265625	21.7837263879625\\
0.57275390625	21.7480282065103\\
0.5732421875	21.7124063238219\\
0.57373046875	21.676860543892\\
0.57421875	21.6413906715921\\
0.57470703125	21.6059965126666\\
0.5751953125	21.5706778737274\\
0.57568359375	21.535434562249\\
0.576171875	21.5002663865633\\
0.57666015625	21.4651731558558\\
0.5771484375	21.4301546801594\\
0.57763671875	21.3952107703507\\
0.578125	21.3603412381449\\
0.57861328125	21.3255458960909\\
0.5791015625	21.2908245575671\\
0.57958984375	21.2561770367762\\
0.580078125	21.2216031487411\\
0.58056640625	21.1871027092999\\
0.5810546875	21.1526755351016\\
0.58154296875	21.1183214436017\\
0.58203125	21.0840402530573\\
0.58251953125	21.049831782523\\
0.5830078125	21.0156958518465\\
0.58349609375	20.9816322816639\\
0.583984375	20.9476408933955\\
0.58447265625	20.9137215092417\\
0.5849609375	20.8798739521783\\
0.58544921875	20.8460980459525\\
0.5859375	20.8123936150785\\
0.58642578125	20.7787604848334\\
0.5869140625	20.745198481253\\
0.58740234375	20.7117074311275\\
0.587890625	20.6782871619976\\
0.58837890625	20.6449375021503\\
0.5888671875	20.6116582806145\\
0.58935546875	20.5784493271576\\
0.58984375	20.545310472281\\
0.59033203125	20.5122415472162\\
0.5908203125	20.4792423839209\\
0.59130859375	20.4463128150752\\
0.591796875	20.4134526740771\\
0.59228515625	20.3806617950396\\
0.5927734375	20.3479400127858\\
0.59326171875	20.3152871628457\\
0.59375	20.2827030814524\\
0.59423828125	20.250187605538\\
0.5947265625	20.2177405727298\\
0.59521484375	20.1853618213472\\
0.595703125	20.1530511903972\\
0.59619140625	20.1208085195711\\
0.5966796875	20.088633649241\\
0.59716796875	20.0565264204558\\
0.59765625	20.0244866749378\\
0.59814453125	19.9925142550791\\
0.5986328125	19.9606090039381\\
0.59912109375	19.9287707652356\\
0.599609375	19.896999383352\\
0.60009765625	19.8652947033229\\
0.6005859375	19.8336565708363\\
0.60107421875	19.802084832229\\
0.6015625	19.7705793344832\\
0.60205078125	19.7391399252228\\
0.6025390625	19.7077664527103\\
0.60302734375	19.6764587658436\\
0.603515625	19.6452167141521\\
0.60400390625	19.6140401477941\\
0.6044921875	19.582928917553\\
0.60498046875	19.5518828748341\\
0.60546875	19.5209018716615\\
0.60595703125	19.4899857606749\\
0.6064453125	19.4591343951261\\
0.60693359375	19.4283476288761\\
0.607421875	19.3976253163921\\
0.60791015625	19.3669673127436\\
0.6083984375	19.3363734736003\\
0.60888671875	19.3058436552281\\
0.609375	19.2753777144866\\
0.60986328125	19.2449755088258\\
0.6103515625	19.2146368962829\\
0.61083984375	19.1843617354798\\
0.611328125	19.1541498856196\\
0.61181640625	19.1240012064836\\
0.6123046875	19.0939155584289\\
0.61279296875	19.0638928023846\\
0.61328125	19.0339327998497\\
0.61376953125	19.0040354128898\\
0.6142578125	18.9742005041339\\
0.61474609375	18.9444279367724\\
0.615234375	18.9147175745532\\
0.61572265625	18.8850692817796\\
0.6162109375	18.8554829233074\\
0.61669921875	18.8259583645418\\
0.6171875	18.7964954714347\\
0.61767578125	18.7670941104823\\
0.6181640625	18.7377541487216\\
0.61865234375	18.7084754537287\\
0.619140625	18.6792578936152\\
0.61962890625	18.6501013370258\\
0.6201171875	18.6210056531359\\
0.62060546875	18.5919707116485\\
0.62109375	18.5629963827918\\
0.62158203125	18.5340825373166\\
0.6220703125	18.5052290464936\\
0.62255859375	18.4764357821108\\
0.623046875	18.4477026164709\\
0.62353515625	18.4190294223889\\
0.6240234375	18.3904160731894\\
0.62451171875	18.3618624427041\\
0.625	18.3333684052693\\
0.62548828125	18.3049338357234\\
0.6259765625	18.2765586094046\\
0.62646484375	18.2482426021479\\
0.626953125	18.2199856902834\\
0.62744140625	18.1917877506331\\
0.6279296875	18.1636486605092\\
0.62841796875	18.1355682977112\\
0.62890625	18.1075465405236\\
0.62939453125	18.0795832677135\\
0.6298828125	18.0516783585287\\
0.63037109375	18.0238316926946\\
0.630859375	17.9960431504123\\
0.63134765625	17.9683126123565\\
0.6318359375	17.9406399596726\\
0.63232421875	17.9130250739748\\
0.6328125	17.8854678373439\\
0.63330078125	17.8579681323249\\
0.6337890625	17.8305258419246\\
0.63427734375	17.8031408496096\\
0.634765625	17.7758130393041\\
0.63525390625	17.7485422953873\\
0.6357421875	17.7213285026919\\
0.63623046875	17.6941715465013\\
0.63671875	17.6670713125476\\
0.63720703125	17.6400276870097\\
0.6376953125	17.613040556511\\
0.63818359375	17.5861098081169\\
0.638671875	17.5592353293334\\
0.63916015625	17.5324170081045\\
0.6396484375	17.5056547328103\\
0.64013671875	17.4789483922646\\
0.640625	17.4522978757136\\
0.64111328125	17.4257030728329\\
0.6416015625	17.3991638737263\\
0.64208984375	17.3726801689231\\
0.642578125	17.3462518493765\\
0.64306640625	17.3198788064616\\
0.6435546875	17.293560931973\\
0.64404296875	17.2672981181235\\
0.64453125	17.2410902575416\\
0.64501953125	17.2149372432695\\
0.6455078125	17.1888389687617\\
0.64599609375	17.1627953278826\\
0.646484375	17.1368062149046\\
0.64697265625	17.1108715245066\\
0.6474609375	17.0849911517714\\
0.64794921875	17.0591649921847\\
0.6484375	17.0333929416324\\
0.64892578125	17.0076748963993\\
0.6494140625	16.982010753167\\
0.64990234375	16.9564004090122\\
0.650390625	16.9308437614045\\
0.65087890625	16.9053407082053\\
0.6513671875	16.8798911476651\\
0.65185546875	16.8544949784226\\
0.65234375	16.8291520995021\\
0.65283203125	16.8038624103126\\
0.6533203125	16.7786258106449\\
0.65380859375	16.753442200671\\
0.654296875	16.7283114809417\\
0.65478515625	16.7032335523849\\
0.6552734375	16.6782083163042\\
0.65576171875	16.6532356743768\\
0.65625	16.6283155286521\\
0.65673828125	16.6034477815497\\
0.6572265625	16.5786323358583\\
0.65771484375	16.5538690947331\\
0.658203125	16.5291579616952\\
0.65869140625	16.5044988406289\\
0.6591796875	16.4798916357811\\
0.65966796875	16.4553362517586\\
0.66015625	16.4308325935275\\
0.66064453125	16.4063805664108\\
0.6611328125	16.3819800760871\\
0.66162109375	16.3576310285893\\
0.662109375	16.3333333303023\\
0.66259765625	16.309086887962\\
0.6630859375	16.2848916086537\\
0.66357421875	16.2607473998104\\
0.6640625	16.2366541692109\\
0.66455078125	16.212611824979\\
0.6650390625	16.1886202755814\\
0.66552734375	16.1646794298264\\
0.666015625	16.1407891968625\\
0.66650390625	16.1169494861764\\
0.6669921875	16.0931602075921\\
0.66748046875	16.0694212712692\\
0.66796875	16.045732587701\\
0.66845703125	16.0220940677139\\
0.6689453125	15.998505622465\\
0.66943359375	15.9749671634414\\
0.669921875	15.9514786024583\\
0.67041015625	15.9280398516576\\
0.6708984375	15.9046508235067\\
0.67138671875	15.881311430797\\
0.671875	15.8580215866423\\
0.67236328125	15.8347812044775\\
0.6728515625	15.8115901980572\\
0.67333984375	15.7884484814547\\
0.673828125	15.7653559690598\\
0.67431640625	15.7423125755781\\
0.6748046875	15.7193182160295\\
0.67529296875	15.6963728057464\\
0.67578125	15.6734762603731\\
0.67626953125	15.6506284958638\\
0.6767578125	15.6278294284818\\
0.67724609375	15.6050789747975\\
0.677734375	15.5823770516878\\
0.67822265625	15.5597235763344\\
0.6787109375	15.5371184662225\\
0.67919921875	15.5145616391396\\
0.6796875	15.492053013174\\
0.68017578125	15.469592506714\\
0.6806640625	15.4471800384462\\
0.68115234375	15.424815527354\\
0.681640625	15.4024988927172\\
0.68212890625	15.3802300541099\\
0.6826171875	15.3580089313995\\
0.68310546875	15.3358354447459\\
0.68359375	15.3137095145996\\
0.68408203125	15.2916310617007\\
0.6845703125	15.269600007078\\
0.68505859375	15.2476162720474\\
0.685546875	15.2256797782108\\
0.68603515625	15.203790447455\\
0.6865234375	15.1819482019504\\
0.68701171875	15.1601529641497\\
0.6875	15.1384046567872\\
0.68798828125	15.116703202877\\
0.6884765625	15.0950485257122\\
0.68896484375	15.0734405488636\\
0.689453125	15.0518791961788\\
0.68994140625	15.0303643917806\\
0.6904296875	15.0088960600663\\
0.69091796875	14.9874741257063\\
0.69140625	14.966098513643\\
0.69189453125	14.9447691490899\\
0.6923828125	14.9234859575301\\
0.69287109375	14.9022488647155\\
0.693359375	14.8810577966657\\
0.69384765625	14.8599126796665\\
0.6943359375	14.8388134402693\\
0.69482421875	14.8177600052898\\
0.6953125	14.796752301807\\
0.69580078125	14.7757902571618\\
0.6962890625	14.7548737989565\\
0.69677734375	14.7340028550532\\
0.697265625	14.7131773535729\\
0.69775390625	14.6923972228949\\
0.6982421875	14.671662391655\\
0.69873046875	14.6509727887449\\
0.69921875	14.6303283433111\\
0.69970703125	14.609728984754\\
0.7001953125	14.5891746427266\\
0.70068359375	14.5686652471336\\
0.701171875	14.5482007281304\\
0.70166015625	14.5277810161222\\
0.7021484375	14.5074060417628\\
0.70263671875	14.4870757359539\\
0.703125	14.4667900298435\\
0.70361328125	14.4465488548258\\
0.7041015625	14.4263521425395\\
0.70458984375	14.4061998248669\\
0.705078125	14.3860918339334\\
0.70556640625	14.3660281021062\\
0.7060546875	14.3460085619931\\
0.70654296875	14.3260331464422\\
0.70703125	14.3061017885402\\
0.70751953125	14.2862144216122\\
0.7080078125	14.26637097922\\
0.70849609375	14.2465713951617\\
0.708984375	14.2268156034709\\
0.70947265625	14.2071035384151\\
0.7099609375	14.1874351344954\\
0.71044921875	14.1678103264451\\
0.7109375	14.1482290492296\\
0.71142578125	14.1286912380443\\
0.7119140625	14.1091968283148\\
0.71240234375	14.0897457556954\\
0.712890625	14.0703379560683\\
0.71337890625	14.0509733655429\\
0.7138671875	14.0316519204548\\
0.71435546875	14.0123735573648\\
0.71484375	13.9931382130583\\
0.71533203125	13.973945824544\\
0.7158203125	13.9547963290537\\
0.71630859375	13.9356896640408\\
0.716796875	13.9166257671799\\
0.71728515625	13.8976045763655\\
0.7177734375	13.8786260297116\\
0.71826171875	13.8596900655507\\
0.71875	13.8407966224327\\
0.71923828125	13.8219456391247\\
0.7197265625	13.8031370546094\\
0.72021484375	13.7843708080849\\
0.720703125	13.7656468389635\\
0.72119140625	13.7469650868711\\
0.7216796875	13.7283254916462\\
0.72216796875	13.7097279933394\\
0.72265625	13.6911725322124\\
0.72314453125	13.6726590487369\\
0.7236328125	13.6541874835944\\
0.72412109375	13.6357577776752\\
0.724609375	13.6173698720773\\
0.72509765625	13.5990237081059\\
0.7255859375	13.5807192272728\\
0.72607421875	13.5624563712953\\
0.7265625	13.5442350820953\\
0.72705078125	13.5260553017992\\
0.7275390625	13.5079169727364\\
0.72802734375	13.489820037439\\
0.728515625	13.471764438641\\
0.72900390625	13.4537501192772\\
0.7294921875	13.435777022483\\
0.72998046875	13.417845091593\\
0.73046875	13.3999542701413\\
0.73095703125	13.3821045018592\\
0.7314453125	13.3642957306762\\
0.73193359375	13.346527900718\\
0.732421875	13.3288009563064\\
0.73291015625	13.3111148419582\\
0.7333984375	13.2934695023849\\
0.73388671875	13.2758648824917\\
0.734375	13.2583009273769\\
0.73486328125	13.2407775823312\\
0.7353515625	13.2232947928368\\
0.73583984375	13.2058525045671\\
0.736328125	13.1884506633858\\
0.73681640625	13.1710892153458\\
0.7373046875	13.1537681066893\\
0.73779296875	13.1364872838465\\
0.73828125	13.1192466934354\\
0.73876953125	13.1020462822606\\
0.7392578125	13.084885997313\\
0.73974609375	13.0677657857691\\
0.740234375	13.05068559499\\
0.74072265625	13.0336453725213\\
0.7412109375	13.016645066092\\
0.74169921875	12.9996846236139\\
0.7421875	12.9827639931813\\
0.74267578125	12.9658831230698\\
0.7431640625	12.949041961736\\
0.74365234375	12.932240457817\\
0.744140625	12.9154785601293\\
0.74462890625	12.8987562176685\\
0.7451171875	12.8820733796086\\
0.74560546875	12.8654299953015\\
0.74609375	12.848826014276\\
0.74658203125	12.8322613862375\\
0.7470703125	12.8157360610674\\
0.74755859375	12.7992499888222\\
0.748046875	12.7828031197332\\
0.74853515625	12.7663954042056\\
0.7490234375	12.7500267928182\\
0.74951171875	12.7336972363226\\
0.75	12.7174066856425\\
0.75048828125	12.7011550918735\\
0.7509765625	12.684942406282\\
0.75146484375	12.668768580305\\
0.751953125	12.6526335655492\\
0.75244140625	12.6365373137909\\
0.7529296875	12.6204797769747\\
0.75341796875	12.6044609072135\\
0.75390625	12.5884806567879\\
0.75439453125	12.5725389781451\\
0.7548828125	12.5566358238989\\
0.75537109375	12.540771146829\\
0.755859375	12.5249448998803\\
0.75634765625	12.5091570361621\\
0.7568359375	12.4934075089484\\
0.75732421875	12.4776962716762\\
0.7578125	12.4620232779458\\
0.75830078125	12.4463884815201\\
0.7587890625	12.4307918363236\\
0.75927734375	12.4152332964424\\
0.759765625	12.3997128161233\\
0.76025390625	12.3842303497735\\
0.7607421875	12.3687858519598\\
0.76123046875	12.3533792774084\\
0.76171875	12.338010581004\\
0.76220703125	12.3226797177896\\
0.7626953125	12.3073866429656\\
0.76318359375	12.2921313118897\\
0.763671875	12.2769136800761\\
0.76416015625	12.2617337031952\\
0.7646484375	12.2465913370727\\
0.76513671875	12.2314865376893\\
0.765625	12.2164192611805\\
0.76611328125	12.2013894638356\\
0.7666015625	12.1863971020973\\
0.76708984375	12.1714421325615\\
0.767578125	12.1565245119764\\
0.76806640625	12.1416441972424\\
0.7685546875	12.1268011454112\\
0.76904296875	12.1119953136855\\
0.76953125	12.0972266594186\\
0.77001953125	12.0824951401137\\
0.7705078125	12.0678007134236\\
0.77099609375	12.0531433371501\\
0.771484375	12.0385229692437\\
0.77197265625	12.0239395678026\\
0.7724609375	12.0093930910731\\
0.77294921875	11.9948834974482\\
0.7734375	11.9804107454678\\
0.77392578125	11.9659747938177\\
0.7744140625	11.9515756013298\\
0.77490234375	11.937213126981\\
0.775390625	11.9228873298928\\
0.77587890625	11.9085981693314\\
0.7763671875	11.8943456047067\\
0.77685546875	11.8801295955719\\
0.77734375	11.8659501016232\\
0.77783203125	11.8518070826994\\
0.7783203125	11.8377004987811\\
0.77880859375	11.8236303099908\\
0.779296875	11.8095964765918\\
0.77978515625	11.7955989589884\\
0.7802734375	11.781637717725\\
0.78076171875	11.7677127134859\\
0.78125	11.7538239070947\\
0.78173828125	11.739971259514\\
0.7822265625	11.7261547318449\\
0.78271484375	11.7123742853266\\
0.783203125	11.6986298813359\\
0.78369140625	11.6849214813868\\
0.7841796875	11.6712490471303\\
0.78466796875	11.6576125403536\\
0.78515625	11.6440119229798\\
0.78564453125	11.6304471570677\\
0.7861328125	11.6169182048112\\
0.78662109375	11.6034250285388\\
0.787109375	11.5899675907133\\
0.78759765625	11.5765458539316\\
0.7880859375	11.5631597809238\\
0.78857421875	11.5498093345532\\
0.7890625	11.5364944778158\\
0.78955078125	11.5232151738399\\
0.7900390625	11.5099713858853\\
0.79052734375	11.4967630773438\\
0.791015625	11.4835902117379\\
0.79150390625	11.4704527527209\\
0.7919921875	11.4573506640763\\
0.79248046875	11.4442839097176\\
0.79296875	11.4312524536877\\
0.79345703125	11.4182562601587\\
0.7939453125	11.4052952934313\\
0.79443359375	11.3923695179347\\
0.794921875	11.3794788982259\\
0.79541015625	11.3666233989896\\
0.7958984375	11.3538029850377\\
0.79638671875	11.3410176213087\\
0.796875	11.328267272868\\
0.79736328125	11.3155519049067\\
0.7978515625	11.3028714827416\\
0.79833984375	11.2902259718152\\
0.798828125	11.2776153376946\\
0.79931640625	11.2650395460717\\
0.7998046875	11.2524985627626\\
0.80029296875	11.2399923537072\\
0.80078125	11.2275208849691\\
0.80126953125	11.2150841227349\\
0.8017578125	11.2026820333141\\
0.80224609375	11.1903145831387\\
0.802734375	11.1779817387627\\
0.80322265625	11.1656834668617\\
0.8037109375	11.1534197342332\\
0.80419921875	11.1411905077951\\
0.8046875	11.1289957545866\\
0.80517578125	11.1168354417669\\
0.8056640625	11.1047095366153\\
0.80615234375	11.0926180065309\\
0.806640625	11.0805608190319\\
0.80712890625	11.0685379417557\\
0.8076171875	11.0565493424582\\
0.80810546875	11.0445949890139\\
0.80859375	11.0326748494151\\
0.80908203125	11.0207888917716\\
0.8095703125	11.0089370843109\\
0.81005859375	10.9971193953774\\
0.810546875	10.9853357934319\\
0.81103515625	10.9735862470518\\
0.8115234375	10.9618707249307\\
0.81201171875	10.9501891958775\\
0.8125	10.9385416288168\\
0.81298828125	10.9269279927882\\
0.8134765625	10.9153482569459\\
0.81396484375	10.9038023905587\\
0.814453125	10.8922903630096\\
0.81494140625	10.8808121437952\\
0.8154296875	10.8693677025258\\
0.81591796875	10.8579570089247\\
0.81640625	10.8465800328281\\
0.81689453125	10.8352367441851\\
0.8173828125	10.8239271130566\\
0.81787109375	10.8126511096158\\
0.818359375	10.8014087041473\\
0.81884765625	10.7901998670475\\
0.8193359375	10.7790245688234\\
0.81982421875	10.767882780093\\
0.8203125	10.7567744715848\\
0.82080078125	10.7456996141372\\
0.8212890625	10.734658178699\\
0.82177734375	10.7236501363281\\
0.822265625	10.7126754581919\\
0.82275390625	10.7017341155668\\
0.8232421875	10.690826079838\\
0.82373046875	10.6799513224989\\
0.82421875	10.6691098151515\\
0.82470703125	10.6583015295051\\
0.8251953125	10.6475264373771\\
0.82568359375	10.6367845106919\\
0.826171875	10.6260757214809\\
0.82666015625	10.6154000418826\\
0.8271484375	10.6047574441416\\
0.82763671875	10.5941479006088\\
0.828125	10.5835713837412\\
0.82861328125	10.5730278661012\\
0.8291015625	10.5625173203568\\
0.82958984375	10.5520397192808\\
0.830078125	10.5415950357513\\
0.83056640625	10.5311832427506\\
0.8310546875	10.5208043133655\\
0.83154296875	10.5104582207867\\
0.83203125	10.5001449383088\\
0.83251953125	10.4898644393298\\
0.8330078125	10.4796166973512\\
0.83349609375	10.4694016859772\\
0.833984375	10.459219378915\\
0.83447265625	10.449069749974\\
0.8349609375	10.4389527730661\\
0.83544921875	10.428868422205\\
0.8359375	10.4188166715062\\
0.83642578125	10.4087974951868\\
0.8369140625	10.3988108675647\\
0.83740234375	10.3888567630592\\
0.837890625	10.3789351561901\\
0.83837890625	10.3690460215778\\
0.8388671875	10.3591893339428\\
0.83935546875	10.3493650681056\\
0.83984375	10.3395731989865\\
0.84033203125	10.3298137016054\\
0.8408203125	10.3200865510812\\
0.84130859375	10.310391722632\\
0.841796875	10.3007291915745\\
0.84228515625	10.2910989333243\\
0.8427734375	10.2815009233949\\
0.84326171875	10.2719351373981\\
0.84375	10.2624015510435\\
0.84423828125	10.2529001401383\\
0.8447265625	10.2434308805869\\
0.84521484375	10.2339937483911\\
0.845703125	10.2245887196495\\
0.84619140625	10.2152157705574\\
0.8466796875	10.2058748774066\\
0.84716796875	10.1965660165849\\
0.84765625	10.1872891645765\\
0.84814453125	10.178044297961\\
0.8486328125	10.1688313934138\\
0.84912109375	10.1596504277056\\
0.849609375	10.1505013777021\\
0.85009765625	10.1413842203641\\
0.8505859375	10.1322989327468\\
0.85107421875	10.1232454920002\\
0.8515625	10.1142238753683\\
0.85205078125	10.1052340601892\\
0.8525390625	10.0962760238949\\
0.85302734375	10.0873497440109\\
0.853515625	10.0784551981562\\
0.85400390625	10.0695923640429\\
0.8544921875	10.0607612194761\\
0.85498046875	10.0519617423537\\
0.85546875	10.043193910666\\
0.85595703125	10.0344577024961\\
0.8564453125	10.0257530960187\\
0.85693359375	10.0170800695007\\
0.857421875	10.0084386013009\\
0.85791015625	9.99982866986937\\
0.8583984375	9.99125025374762\\
0.85888671875	9.98270333156834\\
0.859375	9.97418788205513\\
0.85986328125	9.96570388402237\\
0.8603515625	9.95725131637491\\
0.86083984375	9.94883015810804\\
0.861328125	9.94044038830711\\
0.86181640625	9.93208198614755\\
0.8623046875	9.92375493089447\\
0.86279296875	9.91545920190257\\
0.86328125	9.90719477861599\\
0.86376953125	9.89896164056802\\
0.8642578125	9.89075976738096\\
0.86474609375	9.88258913876595\\
0.865234375	9.87444973452278\\
0.86572265625	9.86634153453959\\
0.8662109375	9.85826451879295\\
0.86669921875	9.8502186673474\\
0.8671875	9.84220396035535\\
0.86767578125	9.83422037805701\\
0.8681640625	9.82626790078004\\
0.86865234375	9.81834650893953\\
0.869140625	9.81045618303771\\
0.86962890625	9.80259690366382\\
0.8701171875	9.79476865149386\\
0.87060546875	9.7869714072905\\
0.87109375	9.77920515190295\\
0.87158203125	9.77146986626664\\
0.8720703125	9.76376553140309\\
0.87255859375	9.75609212841987\\
0.873046875	9.74844963851018\\
0.87353515625	9.74083804295295\\
0.8740234375	9.73325732311251\\
0.87451171875	9.72570746043845\\
0.875	9.71818843646541\\
0.87548828125	9.710700232813\\
0.8759765625	9.70324283118559\\
0.87646484375	9.69581621337217\\
0.876953125	9.68842036124611\\
0.87744140625	9.68105525676511\\
0.8779296875	9.67372088197088\\
0.87841796875	9.6664172189892\\
0.87890625	9.65914425002954\\
0.87939453125	9.65190195738505\\
0.8798828125	9.64469032343226\\
0.88037109375	9.63750933063113\\
0.880859375	9.63035896152467\\
0.88134765625	9.6232391987389\\
0.8818359375	9.61615002498269\\
0.88232421875	9.60909142304762\\
0.8828125	9.60206337580775\\
0.88330078125	9.59506586621951\\
0.8837890625	9.58809887732161\\
0.88427734375	9.58116239223472\\
0.884765625	9.5742563941616\\
0.88525390625	9.56738086638658\\
0.8857421875	9.56053579227577\\
0.88623046875	9.55372115527668\\
0.88671875	9.54693693891813\\
0.88720703125	9.54018312681015\\
0.8876953125	9.53345970264379\\
0.88818359375	9.52676665019102\\
0.888671875	9.52010395330451\\
0.88916015625	9.51347159591755\\
0.8896484375	9.5068695620439\\
0.89013671875	9.50029783577757\\
0.890625	9.49375640129287\\
0.89111328125	9.48724524284398\\
0.8916015625	9.48076434476515\\
0.89208984375	9.47431369147027\\
0.892578125	9.46789326745288\\
0.89306640625	9.46150305728603\\
0.8935546875	9.4551430456221\\
0.89404296875	9.44881321719265\\
0.89453125	9.44251355680839\\
0.89501953125	9.4362440493589\\
0.8955078125	9.43000467981263\\
0.89599609375	9.42379543321665\\
0.896484375	9.41761629469666\\
0.89697265625	9.4114672494567\\
0.8974609375	9.40534828277911\\
0.89794921875	9.39925938002444\\
0.8984375	9.39320052663123\\
0.89892578125	9.38717170811594\\
0.8994140625	9.38117291007278\\
0.89990234375	9.37520411817362\\
0.900390625	9.3692653181679\\
0.90087890625	9.3633564958824\\
0.9013671875	9.35747763722119\\
0.90185546875	9.35162872816555\\
0.90234375	9.34580975477373\\
0.90283203125	9.3400207031809\\
0.9033203125	9.33426155959905\\
0.90380859375	9.32853231031681\\
0.904296875	9.32283294169938\\
0.90478515625	9.31716344018838\\
0.9052734375	9.31152379230178\\
0.90576171875	9.30591398463369\\
0.90625	9.30033400385434\\
0.90673828125	9.29478383670992\\
0.9072265625	9.28926347002249\\
0.90771484375	9.28377289068979\\
0.908203125	9.27831208568527\\
0.90869140625	9.27288104205782\\
0.9091796875	9.26747974693174\\
0.90966796875	9.26210818750664\\
0.91015625	9.25676635105732\\
0.91064453125	9.25145422493361\\
0.9111328125	9.24617179656032\\
0.91162109375	9.24091905343712\\
0.912109375	9.23569598313843\\
0.91259765625	9.23050257331324\\
0.9130859375	9.22533881168518\\
0.91357421875	9.22020468605219\\
0.9140625	9.2151001842866\\
0.91455078125	9.21002529433494\\
0.9150390625	9.20498000421789\\
0.91552734375	9.19996430203003\\
0.916015625	9.19497817593998\\
0.91650390625	9.19002161419004\\
0.9169921875	9.18509460509633\\
0.91748046875	9.18019713704853\\
0.91796875	9.17532919850978\\
0.91845703125	9.1704907780167\\
0.9189453125	9.16568186417915\\
0.91943359375	9.16090244568028\\
0.919921875	9.15615251127629\\
0.92041015625	9.15143204979642\\
0.9208984375	9.14674105014285\\
0.92138671875	9.14207950129055\\
0.921875	9.13744739228728\\
0.92236328125	9.13284471225341\\
0.9228515625	9.12827145038184\\
0.92333984375	9.12372759593799\\
0.923828125	9.11921313825954\\
0.92431640625	9.11472806675656\\
0.9248046875	9.11027237091127\\
0.92529296875	9.10584604027795\\
0.92578125	9.10144906448289\\
0.92626953125	9.09708143322439\\
0.9267578125	9.09274313627246\\
0.92724609375	9.08843416346892\\
0.927734375	9.08415450472723\\
0.92822265625	9.07990415003247\\
0.9287109375	9.07568308944114\\
0.92919921875	9.07149131308115\\
0.9296875	9.06732881115181\\
0.93017578125	9.06319557392362\\
0.9306640625	9.0590915917382\\
0.93115234375	9.05501685500831\\
0.931640625	9.05097135421764\\
0.93212890625	9.04695507992087\\
0.9326171875	9.04296802274344\\
0.93310546875	9.03901017338159\\
0.93359375	9.03508152260224\\
0.93408203125	9.03118206124286\\
0.9345703125	9.02731178021146\\
0.93505859375	9.02347067048657\\
0.935546875	9.01965872311699\\
0.93603515625	9.01587592922187\\
0.9365234375	9.01212227999055\\
0.93701171875	9.00839776668251\\
0.9375	9.00470238062735\\
0.93798828125	9.00103611322468\\
0.9384765625	8.99739895594389\\
0.93896484375	8.99379090032442\\
0.939453125	8.99021193797536\\
0.93994140625	8.98666206057557\\
0.9404296875	8.98314125987356\\
0.94091796875	8.97964952768739\\
0.94140625	8.97618685590465\\
0.94189453125	8.97275323648232\\
0.9423828125	8.96934866144681\\
0.94287109375	8.96597312289381\\
0.943359375	8.96262661298825\\
0.94384765625	8.95930912396424\\
0.9443359375	8.95602064812498\\
0.94482421875	8.95276117784272\\
0.9453125	8.94953070555872\\
0.94580078125	8.94632922378312\\
0.9462890625	8.9431567250949\\
0.94677734375	8.94001320214194\\
0.947265625	8.93689864764069\\
0.94775390625	8.93381305437638\\
0.9482421875	8.93075641520285\\
0.94873046875	8.92772872304244\\
0.94921875	8.92472997088599\\
0.94970703125	8.92176015179284\\
0.9501953125	8.91881925889063\\
0.95068359375	8.91590728537533\\
0.951171875	8.91302422451121\\
0.95166015625	8.91017006963074\\
0.9521484375	8.90734481413451\\
0.95263671875	8.90454845149122\\
0.953125	8.9017809752376\\
};
\addplot [color=green,solid]
  table[row sep=crcr]{0.953125	8.9017809752376\\
0.95361328125	8.89904237897842\\
0.9541015625	8.89633265638633\\
0.95458984375	8.89365180120186\\
0.955078125	8.89099980723342\\
0.95556640625	8.88837666835716\\
0.9560546875	8.88578237851698\\
0.95654296875	8.88321693172444\\
0.95703125	8.88068032205872\\
0.95751953125	8.87817254366662\\
0.9580078125	8.87569359076242\\
0.95849609375	8.87324345762795\\
0.958984375	8.8708221386124\\
0.95947265625	8.86842962813243\\
0.9599609375	8.86606592067196\\
0.96044921875	8.86373101078224\\
0.9609375	8.8614248930818\\
0.96142578125	8.85914756225636\\
0.9619140625	8.85689901305877\\
0.96240234375	8.85467924030906\\
0.962890625	8.85248823889429\\
0.96337890625	8.85032600376862\\
0.9638671875	8.84819252995312\\
0.96435546875	8.84608781253583\\
0.96484375	8.84401184667177\\
0.96533203125	8.84196462758273\\
0.9658203125	8.83994615055743\\
0.96630859375	8.83795641095135\\
0.966796875	8.83599540418669\\
0.96728515625	8.83406312575242\\
0.9677734375	8.83215957120413\\
0.96826171875	8.83028473616412\\
0.96875	8.82843861632125\\
0.96923828125	8.82662120743101\\
0.9697265625	8.82483250531532\\
0.97021484375	8.82307250586273\\
0.970703125	8.82134120502819\\
0.97119140625	8.81963859883309\\
0.9716796875	8.81796468336522\\
0.97216796875	8.81631945477876\\
0.97265625	8.81470290929425\\
0.97314453125	8.81311504319843\\
0.9736328125	8.81155585284448\\
0.97412109375	8.81002533465171\\
0.974609375	8.80852348510569\\
0.97509765625	8.80705030075817\\
0.9755859375	8.80560577822705\\
0.97607421875	8.80418991419638\\
0.9765625	8.80280270541636\\
0.97705078125	8.80144414870317\\
0.9775390625	8.80011424093909\\
0.97802734375	8.79881297907248\\
0.978515625	8.79754036011763\\
0.97900390625	8.79629638115484\\
0.9794921875	8.79508103933036\\
0.97998046875	8.79389433185635\\
0.98046875	8.79273625601091\\
0.98095703125	8.79160680913802\\
0.9814453125	8.79050598864753\\
0.98193359375	8.78943379201509\\
0.982421875	8.78839021678223\\
0.98291015625	8.78737526055621\\
0.9833984375	8.78638892101014\\
0.98388671875	8.78543119588288\\
0.984375	8.78450208297898\\
0.98486328125	8.78360158016881\\
0.9853515625	8.7827296853884\\
0.98583984375	8.7818863966394\\
0.986328125	8.78107171198925\\
0.98681640625	8.78028562957103\\
0.9873046875	8.7795281475834\\
0.98779296875	8.77879926429069\\
0.98828125	8.77809897802284\\
0.98876953125	8.77742728717544\\
0.9892578125	8.77678419020958\\
0.98974609375	8.77616968565198\\
0.990234375	8.7755837720949\\
0.99072265625	8.77502644819624\\
0.9912109375	8.77449771267927\\
0.99169921875	8.77399756433296\\
0.9921875	8.77352600201171\\
0.99267578125	8.77308302463543\\
0.9931640625	8.77266863118958\\
0.99365234375	8.77228282072511\\
0.994140625	8.77192559235837\\
0.99462890625	8.77159694527133\\
0.9951171875	8.77129687871132\\
0.99560546875	8.7710253919912\\
0.99609375	8.77078248448924\\
0.99658203125	8.77056815564921\\
0.9970703125	8.7703824049803\\
0.99755859375	8.77022523205717\\
0.998046875	8.77009663651996\\
0.99853515625	8.76999661807412\\
0.9990234375	8.7699251764907\\
0.99951171875	8.76988231160608\\
};
\addlegendentry{AR(4) Model};

\addplot [color=mycolor1,solid,forget plot]
  table[row sep=crcr]{-1	2.81710079948377\\
-0.99951171875	2.817115995231\\
-0.9990234375	2.81716158250844\\
-0.99853515625	2.81723756142353\\
-0.998046875	2.81734393215531\\
-0.99755859375	2.8174806949544\\
-0.9970703125	2.81764785014307\\
-0.99658203125	2.81784539811519\\
-0.99609375	2.81807333933627\\
-0.99560546875	2.8183316743434\\
-0.9951171875	2.8186204037454\\
-0.99462890625	2.8189395282226\\
-0.994140625	2.81928904852706\\
-0.99365234375	2.81966896548247\\
-0.9931640625	2.82007927998412\\
-0.99267578125	2.82051999299908\\
-0.9921875	2.82099110556597\\
-0.99169921875	2.82149261879518\\
-0.9912109375	2.82202453386874\\
-0.99072265625	2.8225868520404\\
-0.990234375	2.82317957463562\\
-0.98974609375	2.8238027030516\\
-0.9892578125	2.82445623875724\\
-0.98876953125	2.82514018329322\\
-0.98828125	2.82585453827198\\
-0.98779296875	2.82659930537771\\
-0.9873046875	2.82737448636644\\
-0.98681640625	2.82818008306593\\
-0.986328125	2.82901609737588\\
-0.98583984375	2.8298825312677\\
-0.9853515625	2.83077938678478\\
-0.98486328125	2.83170666604226\\
-0.984375	2.83266437122729\\
-0.98388671875	2.83365250459887\\
-0.9833984375	2.83467106848792\\
-0.98291015625	2.83572006529738\\
-0.982421875	2.83679949750208\\
-0.98193359375	2.8379093676489\\
-0.9814453125	2.83904967835674\\
-0.98095703125	2.84022043231652\\
-0.98046875	2.84142163229123\\
-0.97998046875	2.84265328111597\\
-0.9794921875	2.84391538169791\\
-0.97900390625	2.84520793701641\\
-0.978515625	2.84653095012295\\
-0.97802734375	2.84788442414126\\
-0.9775390625	2.84926836226723\\
-0.97705078125	2.85068276776905\\
-0.9765625	2.85212764398714\\
-0.97607421875	2.85360299433425\\
-0.9755859375	2.8551088222955\\
-0.97509765625	2.85664513142832\\
-0.974609375	2.85821192536256\\
-0.97412109375	2.85980920780052\\
-0.9736328125	2.86143698251692\\
-0.97314453125	2.86309525335904\\
-0.97265625	2.86478402424666\\
-0.97216796875	2.86650329917208\\
-0.9716796875	2.86825308220026\\
-0.97119140625	2.87003337746879\\
-0.970703125	2.87184418918793\\
-0.97021484375	2.87368552164059\\
-0.9697265625	2.87555737918252\\
-0.96923828125	2.87745976624223\\
-0.96875	2.879392687321\\
-0.96826171875	2.88135614699304\\
-0.9677734375	2.88335014990542\\
-0.96728515625	2.88537470077821\\
-0.966796875	2.88742980440439\\
-0.96630859375	2.88951546565006\\
-0.9658203125	2.89163168945431\\
-0.96533203125	2.8937784808294\\
-0.96484375	2.89595584486077\\
-0.96435546875	2.898163786707\\
-0.9638671875	2.90040231159996\\
-0.96337890625	2.90267142484481\\
-0.962890625	2.90497113182006\\
-0.96240234375	2.90730143797764\\
-0.9619140625	2.90966234884287\\
-0.96142578125	2.91205387001461\\
-0.9609375	2.91447600716517\\
-0.96044921875	2.91692876604061\\
-0.9599609375	2.91941215246047\\
-0.95947265625	2.92192617231807\\
-0.958984375	2.92447083158042\\
-0.95849609375	2.92704613628841\\
-0.9580078125	2.92965209255672\\
-0.95751953125	2.93228870657394\\
-0.95703125	2.93495598460261\\
-0.95654296875	2.93765393297936\\
-0.9560546875	2.94038255811479\\
-0.95556640625	2.94314186649369\\
-0.955078125	2.94593186467504\\
-0.95458984375	2.94875255929205\\
-0.9541015625	2.95160395705224\\
-0.95361328125	2.95448606473745\\
-0.953125	2.95739888920401\\
-0.95263671875	2.96034243738273\\
-0.9521484375	2.96331671627894\\
-0.95166015625	2.96632173297258\\
-0.951171875	2.96935749461827\\
-0.95068359375	2.97242400844537\\
-0.9501953125	2.97552128175806\\
-0.94970703125	2.97864932193532\\
-0.94921875	2.98180813643115\\
-0.94873046875	2.98499773277452\\
-0.9482421875	2.98821811856939\\
-0.94775390625	2.991469301495\\
-0.947265625	2.99475128930562\\
-0.94677734375	2.99806408983096\\
-0.9462890625	3.00140771097597\\
-0.94580078125	3.00478216072103\\
-0.9453125	3.00818744712201\\
-0.94482421875	3.01162357831034\\
-0.9443359375	3.0150905624931\\
-0.94384765625	3.01858840795303\\
-0.943359375	3.02211712304866\\
-0.94287109375	3.02567671621439\\
-0.9423828125	3.02926719596056\\
-0.94189453125	3.03288857087345\\
-0.94140625	3.03654084961548\\
-0.94091796875	3.04022404092523\\
-0.9404296875	3.04393815361748\\
-0.93994140625	3.04768319658335\\
-0.939453125	3.05145917879035\\
-0.93896484375	3.05526610928248\\
-0.9384765625	3.05910399718026\\
-0.93798828125	3.06297285168094\\
-0.9375	3.06687268205836\\
-0.93701171875	3.07080349766329\\
-0.9365234375	3.07476530792333\\
-0.93603515625	3.07875812234308\\
-0.935546875	3.08278195050417\\
-0.93505859375	3.0868368020654\\
-0.9345703125	3.09092268676283\\
-0.93408203125	3.09503961440981\\
-0.93359375	3.09918759489712\\
-0.93310546875	3.103366638193\\
-0.9326171875	3.10757675434338\\
-0.93212890625	3.11181795347177\\
-0.931640625	3.11609024577952\\
-0.93115234375	3.12039364154581\\
-0.9306640625	3.12472815112778\\
-0.93017578125	3.12909378496067\\
-0.9296875	3.13349055355783\\
-0.92919921875	3.13791846751087\\
-0.9287109375	3.14237753748975\\
-0.92822265625	3.14686777424283\\
-0.927734375	3.15138918859702\\
-0.92724609375	3.1559417914579\\
-0.9267578125	3.16052559380975\\
-0.92626953125	3.16514060671568\\
-0.92578125	3.16978684131775\\
-0.92529296875	3.17446430883706\\
-0.9248046875	3.17917302057383\\
-0.92431640625	3.18391298790754\\
-0.923828125	3.18868422229701\\
-0.92333984375	3.19348673528047\\
-0.9228515625	3.19832053847578\\
-0.92236328125	3.20318564358039\\
-0.921875	3.20808206237158\\
-0.92138671875	3.21300980670645\\
-0.9208984375	3.21796888852214\\
-0.92041015625	3.22295931983582\\
-0.919921875	3.22798111274493\\
-0.91943359375	3.2330342794272\\
-0.9189453125	3.23811883214073\\
-0.91845703125	3.24323478322425\\
-0.91796875	3.24838214509712\\
-0.91748046875	3.25356093025941\\
-0.9169921875	3.25877115129215\\
-0.91650390625	3.26401282085734\\
-0.916015625	3.26928595169803\\
-0.91552734375	3.27459055663859\\
-0.9150390625	3.27992664858473\\
-0.91455078125	3.28529424052356\\
-0.9140625	3.29069334552387\\
-0.91357421875	3.29612397673609\\
-0.9130859375	3.30158614739254\\
-0.91259765625	3.3070798708074\\
-0.912109375	3.31260516037704\\
-0.91162109375	3.31816202957996\\
-0.9111328125	3.323750491977\\
-0.91064453125	3.32937056121148\\
-0.91015625	3.33502225100926\\
-0.90966796875	3.34070557517888\\
-0.9091796875	3.34642054761182\\
-0.90869140625	3.3521671822824\\
-0.908203125	3.35794549324813\\
-0.90771484375	3.36375549464965\\
-0.9072265625	3.36959720071106\\
-0.90673828125	3.37547062573986\\
-0.90625	3.3813757841272\\
-0.90576171875	3.38731269034802\\
-0.9052734375	3.3932813589611\\
-0.90478515625	3.39928180460926\\
-0.904296875	3.40531404201948\\
-0.90380859375	3.41137808600306\\
-0.9033203125	3.41747395145573\\
-0.90283203125	3.4236016533578\\
-0.90234375	3.42976120677428\\
-0.90185546875	3.43595262685507\\
-0.9013671875	3.44217592883504\\
-0.90087890625	3.44843112803427\\
-0.900390625	3.45471823985803\\
-0.89990234375	3.46103727979708\\
-0.8994140625	3.4673882634278\\
-0.89892578125	3.47377120641222\\
-0.8984375	3.48018612449829\\
-0.89794921875	3.48663303351998\\
-0.8974609375	3.49311194939738\\
-0.89697265625	3.49962288813699\\
-0.896484375	3.50616586583175\\
-0.89599609375	3.51274089866115\\
-0.8955078125	3.51934800289161\\
-0.89501953125	3.52598719487632\\
-0.89453125	3.53265849105566\\
-0.89404296875	3.53936190795727\\
-0.8935546875	3.54609746219607\\
-0.89306640625	3.5528651704747\\
-0.892578125	3.55966504958335\\
-0.89208984375	3.56649711640026\\
-0.8916015625	3.57336138789154\\
-0.89111328125	3.5802578811116\\
-0.890625	3.58718661320323\\
-0.89013671875	3.59414760139764\\
-0.8896484375	3.60114086301483\\
-0.88916015625	3.60816641546361\\
-0.888671875	3.61522427624182\\
-0.88818359375	3.62231446293646\\
-0.8876953125	3.62943699322396\\
-0.88720703125	3.63659188487019\\
-0.88671875	3.6437791557308\\
-0.88623046875	3.65099882375123\\
-0.8857421875	3.65825090696698\\
-0.88525390625	3.66553542350383\\
-0.884765625	3.67285239157787\\
-0.88427734375	3.68020182949576\\
-0.8837890625	3.68758375565499\\
-0.88330078125	3.69499818854382\\
-0.8828125	3.7024451467417\\
-0.88232421875	3.70992464891938\\
-0.8818359375	3.71743671383898\\
-0.88134765625	3.72498136035433\\
-0.880859375	3.73255860741105\\
-0.88037109375	3.74016847404674\\
-0.8798828125	3.74781097939123\\
-0.87939453125	3.75548614266671\\
-0.87890625	3.76319398318792\\
-0.87841796875	3.77093452036234\\
-0.8779296875	3.77870777369041\\
-0.87744140625	3.78651376276568\\
-0.876953125	3.79435250727505\\
-0.87646484375	3.80222402699889\\
-0.8759765625	3.81012834181127\\
-0.87548828125	3.81806547168019\\
-0.875	3.82603543666769\\
-0.87451171875	3.83403825693019\\
-0.8740234375	3.84207395271851\\
-0.87353515625	3.85014254437819\\
-0.873046875	3.85824405234962\\
-0.87255859375	3.86637849716837\\
-0.8720703125	3.87454589946515\\
-0.87158203125	3.88274627996627\\
-0.87109375	3.89097965949372\\
-0.87060546875	3.89924605896532\\
-0.8701171875	3.90754549939513\\
-0.86962890625	3.91587800189338\\
-0.869140625	3.92424358766688\\
-0.86865234375	3.93264227801919\\
-0.8681640625	3.94107409435079\\
-0.86767578125	3.94953905815933\\
-0.8671875	3.95803719103978\\
-0.86669921875	3.96656851468476\\
-0.8662109375	3.97513305088463\\
-0.86572265625	3.98373082152781\\
-0.865234375	3.99236184860091\\
-0.86474609375	4.00102615418902\\
-0.8642578125	4.00972376047591\\
-0.86376953125	4.01845468974419\\
-0.86328125	4.02721896437564\\
-0.86279296875	4.03601660685132\\
-0.8623046875	4.04484763975196\\
-0.86181640625	4.05371208575798\\
-0.861328125	4.06260996764983\\
-0.86083984375	4.07154130830824\\
-0.8603515625	4.0805061307144\\
-0.85986328125	4.08950445795023\\
-0.859375	4.09853631319856\\
-0.85888671875	4.10760171974339\\
-0.8583984375	4.11670070097015\\
-0.85791015625	4.12583328036592\\
-0.857421875	4.13499948151964\\
-0.85693359375	4.14419932812237\\
-0.8564453125	4.15343284396754\\
-0.85595703125	4.16270005295119\\
-0.85546875	4.17200097907221\\
-0.85498046875	4.18133564643255\\
-0.8544921875	4.19070407923747\\
-0.85400390625	4.2001063017959\\
-0.853515625	4.20954233852052\\
-0.85302734375	4.21901221392811\\
-0.8525390625	4.22851595263978\\
-0.85205078125	4.23805357938118\\
-0.8515625	4.24762511898286\\
-0.85107421875	4.25723059638039\\
-0.8505859375	4.26687003661475\\
-0.85009765625	4.27654346483243\\
-0.849609375	4.28625090628582\\
-0.84912109375	4.29599238633347\\
-0.8486328125	4.30576793044019\\
-0.84814453125	4.31557756417759\\
-0.84765625	4.32542131322405\\
-0.84716796875	4.33529920336515\\
-0.8466796875	4.3452112604939\\
-0.84619140625	4.35515751061104\\
-0.845703125	4.36513797982531\\
-0.84521484375	4.37515269435359\\
-0.8447265625	4.38520168052136\\
-0.84423828125	4.39528496476286\\
-0.84375	4.40540257362139\\
-0.84326171875	4.41555453374957\\
-0.8427734375	4.4257408719097\\
-0.84228515625	4.4359616149739\\
-0.841796875	4.44621678992449\\
-0.84130859375	4.45650642385427\\
-0.8408203125	4.4668305439668\\
-0.84033203125	4.47718917757662\\
-0.83984375	4.48758235210959\\
-0.83935546875	4.4980100951032\\
-0.8388671875	4.50847243420679\\
-0.83837890625	4.51896939718193\\
-0.837890625	4.52950101190264\\
-0.83740234375	4.54006730635571\\
-0.8369140625	4.55066830864098\\
-0.83642578125	4.56130404697171\\
-0.8359375	4.57197454967475\\
-0.83544921875	4.58267984519092\\
-0.8349609375	4.5934199620754\\
-0.83447265625	4.60419492899775\\
-0.833984375	4.6150047747426\\
-0.83349609375	4.62584952820965\\
-0.8330078125	4.63672921841407\\
-0.83251953125	4.64764387448685\\
-0.83203125	4.65859352567515\\
-0.83154296875	4.66957820134246\\
-0.8310546875	4.68059793096901\\
-0.83056640625	4.69165274415213\\
-0.830078125	4.7027426706065\\
-0.82958984375	4.71386774016446\\
-0.8291015625	4.72502798277636\\
-0.82861328125	4.73622342851091\\
-0.828125	4.74745410755546\\
-0.82763671875	4.75872005021633\\
-0.8271484375	4.77002128691917\\
-0.82666015625	4.7813578482092\\
-0.826171875	4.79272976475173\\
-0.82568359375	4.80413706733224\\
-0.8251953125	4.81557978685692\\
-0.82470703125	4.82705795435294\\
-0.82421875	4.83857160096868\\
-0.82373046875	4.8501207579743\\
-0.8232421875	4.86170545676187\\
-0.82275390625	4.8733257288458\\
-0.822265625	4.88498160586319\\
-0.82177734375	4.89667311957417\\
-0.8212890625	4.9084003018622\\
-0.82080078125	4.92016318473451\\
-0.8203125	4.93196180032238\\
-0.81982421875	4.94379618088152\\
-0.8193359375	4.95566635879243\\
-0.81884765625	4.96757236656073\\
-0.818359375	4.97951423681757\\
-0.81787109375	4.9914920023199\\
-0.8173828125	5.00350569595098\\
-0.81689453125	5.01555535072062\\
-0.81640625	5.02764099976551\\
-0.81591796875	5.03976267634977\\
-0.8154296875	5.05192041386516\\
-0.81494140625	5.06411424583152\\
-0.814453125	5.07634420589712\\
-0.81396484375	5.08861032783903\\
-0.8134765625	5.10091264556356\\
-0.81298828125	5.11325119310652\\
-0.8125	5.12562600463374\\
-0.81201171875	5.13803711444136\\
-0.8115234375	5.15048455695623\\
-0.81103515625	5.16296836673634\\
-0.810546875	5.17548857847117\\
-0.81005859375	5.18804522698211\\
-0.8095703125	5.20063834722278\\
-0.80908203125	5.21326797427952\\
-0.80859375	5.22593414337176\\
-0.80810546875	5.2386368898524\\
-0.8076171875	5.25137624920821\\
-0.80712890625	5.26415225706024\\
-0.806640625	5.27696494916427\\
-0.80615234375	5.28981436141113\\
-0.8056640625	5.30270052982716\\
-0.80517578125	5.3156234905747\\
-0.8046875	5.32858327995232\\
-0.80419921875	5.34157993439539\\
-0.8037109375	5.35461349047645\\
-0.80322265625	5.36768398490562\\
-0.802734375	5.38079145453108\\
-0.80224609375	5.39393593633934\\
-0.8017578125	5.40711746745588\\
-0.80126953125	5.42033608514547\\
-0.80078125	5.43359182681251\\
-0.80029296875	5.44688473000169\\
-0.7998046875	5.46021483239819\\
-0.79931640625	5.47358217182826\\
-0.798828125	5.48698678625967\\
-0.79833984375	5.50042871380206\\
-0.7978515625	5.51390799270738\\
-0.79736328125	5.52742466137051\\
-0.796875	5.54097875832951\\
-0.79638671875	5.55457032226619\\
-0.7958984375	5.56819939200651\\
-0.79541015625	5.58186600652099\\
-0.794921875	5.59557020492535\\
-0.79443359375	5.60931202648077\\
-0.7939453125	5.62309151059448\\
-0.79345703125	5.63690869682013\\
-0.79296875	5.65076362485833\\
-0.79248046875	5.66465633455715\\
-0.7919921875	5.67858686591248\\
-0.79150390625	5.69255525906861\\
-0.791015625	5.70656155431863\\
-0.79052734375	5.72060579210499\\
-0.7900390625	5.73468801301993\\
-0.78955078125	5.74880825780598\\
-0.7890625	5.76296656735639\\
-0.78857421875	5.77716298271575\\
-0.7880859375	5.79139754508041\\
-0.78759765625	5.80567029579894\\
-0.787109375	5.81998127637263\\
-0.78662109375	5.83433052845611\\
-0.7861328125	5.84871809385767\\
-0.78564453125	5.86314401453998\\
-0.78515625	5.87760833262036\\
-0.78466796875	5.89211109037153\\
-0.7841796875	5.90665233022189\\
-0.78369140625	5.92123209475625\\
-0.783203125	5.9358504267162\\
-0.78271484375	5.95050736900069\\
-0.7822265625	5.9652029646666\\
-0.78173828125	5.97993725692913\\
-0.78125	5.9947102891625\\
-0.78076171875	6.00952210490035\\
-0.7802734375	6.02437274783633\\
-0.77978515625	6.03926226182465\\
-0.779296875	6.05419069088059\\
-0.77880859375	6.0691580791811\\
-0.7783203125	6.08416447106524\\
-0.77783203125	6.09920991103486\\
-0.77734375	6.11429444375503\\
-0.77685546875	6.1294181140547\\
-0.7763671875	6.14458096692718\\
-0.77587890625	6.15978304753079\\
-0.775390625	6.17502440118926\\
-0.77490234375	6.19030507339255\\
-0.7744140625	6.20562510979711\\
-0.77392578125	6.22098455622677\\
-0.7734375	6.23638345867311\\
-0.77294921875	6.25182186329602\\
-0.7724609375	6.26729981642446\\
-0.77197265625	6.28281736455687\\
-0.771484375	6.2983745543619\\
-0.77099609375	6.31397143267885\\
-0.7705078125	6.3296080465184\\
-0.77001953125	6.34528444306314\\
-0.76953125	6.36100066966815\\
-0.76904296875	6.37675677386173\\
-0.7685546875	6.39255280334576\\
-0.76806640625	6.40838880599662\\
-0.767578125	6.42426482986557\\
-0.76708984375	6.44018092317941\\
-0.7666015625	6.45613713434125\\
-0.76611328125	6.47213351193091\\
-0.765625	6.48817010470569\\
-0.76513671875	6.50424696160098\\
-0.7646484375	6.52036413173086\\
-0.76416015625	6.53652166438872\\
-0.763671875	6.55271960904802\\
-0.76318359375	6.56895801536278\\
-0.7626953125	6.58523693316831\\
-0.76220703125	6.60155641248182\\
-0.76171875	6.61791650350316\\
-0.76123046875	6.63431725661532\\
-0.7607421875	6.65075872238529\\
-0.76025390625	6.66724095156455\\
-0.759765625	6.68376399508985\\
-0.75927734375	6.70032790408375\\
-0.7587890625	6.71693272985547\\
-0.75830078125	6.73357852390145\\
-0.7578125	6.75026533790605\\
-0.75732421875	6.76699322374231\\
-0.7568359375	6.78376223347249\\
-0.75634765625	6.80057241934888\\
-0.755859375	6.81742383381445\\
-0.75537109375	6.83431652950364\\
-0.7548828125	6.85125055924292\\
-0.75439453125	6.86822597605158\\
-0.75390625	6.88524283314241\\
-0.75341796875	6.90230118392251\\
-0.7529296875	6.91940108199387\\
-0.75244140625	6.93654258115418\\
-0.751953125	6.95372573539747\\
-0.75146484375	6.97095059891503\\
-0.7509765625	6.98821722609593\\
-0.75048828125	7.00552567152783\\
-0.75	7.02287598999787\\
-0.74951171875	7.04026823649311\\
-0.7490234375	7.05770246620159\\
-0.74853515625	7.07517873451296\\
-0.748046875	7.0926970970191\\
-0.74755859375	7.11025760951515\\
-0.7470703125	7.12786032800012\\
-0.74658203125	7.14550530867762\\
-0.74609375	7.16319260795678\\
-0.74560546875	7.18092228245286\\
-0.7451171875	7.19869438898823\\
-0.74462890625	7.216508984593\\
-0.744140625	7.23436612650586\\
-0.74365234375	7.25226587217484\\
-0.7431640625	7.27020827925826\\
-0.74267578125	7.28819340562534\\
-0.7421875	7.30622130935712\\
-0.74169921875	7.32429204874727\\
-0.7412109375	7.34240568230291\\
-0.74072265625	7.36056226874535\\
-0.740234375	7.37876186701106\\
-0.73974609375	7.39700453625233\\
-0.7392578125	7.41529033583829\\
-0.73876953125	7.43361932535565\\
-0.73828125	7.45199156460954\\
-0.73779296875	7.47040711362436\\
-0.7373046875	7.48886603264467\\
-0.73681640625	7.5073683821361\\
-0.736328125	7.52591422278607\\
-0.73583984375	7.54450361550473\\
-0.7353515625	7.56313662142595\\
-0.73486328125	7.581813301908\\
-0.734375	7.60053371853456\\
-0.73388671875	7.61929793311556\\
-0.7333984375	7.63810600768815\\
-0.73291015625	7.65695800451748\\
-0.732421875	7.67585398609772\\
-0.73193359375	7.69479401515279\\
-0.7314453125	7.71377815463761\\
-0.73095703125	7.7328064677386\\
-0.73046875	7.75187901787494\\
-0.72998046875	7.77099586869929\\
-0.7294921875	7.7901570840989\\
-0.72900390625	7.8093627281964\\
-0.728515625	7.82861286535075\\
-0.72802734375	7.84790756015834\\
-0.7275390625	7.86724687745375\\
-0.72705078125	7.88663088231091\\
-0.7265625	7.90605964004389\\
-0.72607421875	7.92553321620793\\
-0.7255859375	7.94505167660055\\
-0.72509765625	7.96461508726225\\
-0.724609375	7.9842235144778\\
-0.72412109375	8.00387702477704\\
-0.7236328125	8.02357568493593\\
-0.72314453125	8.04331956197757\\
-0.72265625	8.06310872317323\\
-0.72216796875	8.08294323604331\\
-0.7216796875	8.10282316835841\\
-0.72119140625	8.12274858814028\\
-0.720703125	8.14271956366302\\
-0.72021484375	8.16273616345395\\
-0.7197265625	8.18279845629471\\
-0.71923828125	8.20290651122234\\
-0.71875	8.22306039753032\\
-0.71826171875	8.24326018476962\\
-0.7177734375	8.26350594274981\\
-0.71728515625	8.28379774154005\\
-0.716796875	8.3041356514703\\
-0.71630859375	8.32451974313226\\
-0.7158203125	8.3449500873806\\
-0.71533203125	8.365426755334\\
-0.71484375	8.38594981837623\\
-0.71435546875	8.4065193481573\\
-0.7138671875	8.42713541659461\\
-0.71337890625	8.44779809587395\\
-0.712890625	8.46850745845081\\
-0.71240234375	8.4892635770514\\
-0.7119140625	8.51006652467379\\
-0.71142578125	8.53091637458913\\
-0.7109375	8.55181320034279\\
-0.71044921875	8.57275707575547\\
-0.7099609375	8.59374807492447\\
-0.70947265625	8.61478627222475\\
-0.708984375	8.63587174231024\\
-0.70849609375	8.6570045601149\\
-0.7080078125	8.67818480085408\\
-0.70751953125	8.69941254002557\\
-0.70703125	8.72068785341089\\
-0.70654296875	8.74201081707653\\
-0.7060546875	8.76338150737514\\
-0.70556640625	8.7848000009468\\
-0.705078125	8.80626637472017\\
-0.70458984375	8.82778070591391\\
-0.7041015625	8.84934307203772\\
-0.70361328125	8.87095355089381\\
-0.703125	8.89261222057807\\
-0.70263671875	8.9143191594813\\
-0.7021484375	8.93607444629065\\
-0.70166015625	8.95787815999074\\
-0.701171875	8.97973037986508\\
-0.70068359375	9.00163118549736\\
-0.7001953125	9.02358065677273\\
-0.69970703125	9.04557887387914\\
-0.69921875	9.0676259173087\\
-0.69873046875	9.08972186785903\\
-0.6982421875	9.11186680663448\\
-0.69775390625	9.1340608150477\\
-0.697265625	9.15630397482084\\
-0.69677734375	9.17859636798697\\
-0.6962890625	9.20093807689146\\
-0.69580078125	9.22332918419341\\
-0.6953125	9.24576977286704\\
-0.69482421875	9.26825992620294\\
-0.6943359375	9.29079972780978\\
-0.69384765625	9.31338926161548\\
-0.693359375	9.3360286118687\\
-0.69287109375	9.3587178631404\\
-0.6923828125	9.38145710032513\\
-0.69189453125	9.40424640864264\\
-0.69140625	9.42708587363913\\
-0.69091796875	9.44997558118898\\
-0.6904296875	9.47291561749604\\
-0.68994140625	9.49590606909527\\
-0.689453125	9.5189470228541\\
-0.68896484375	9.54203856597402\\
-0.6884765625	9.56518078599213\\
-0.68798828125	9.5883737707826\\
-0.6875	9.61161760855826\\
-0.68701171875	9.6349123878721\\
-0.6865234375	9.65825819761886\\
-0.68603515625	9.68165512703665\\
-0.685546875	9.7051032657084\\
-0.68505859375	9.72860270356352\\
-0.6845703125	9.75215353087956\\
-0.68408203125	9.77575583828367\\
-0.68359375	9.79940971675436\\
-0.68310546875	9.82311525762302\\
-0.6826171875	9.84687255257558\\
-0.68212890625	9.87068169365424\\
-0.681640625	9.89454277325902\\
-0.68115234375	9.91845588414945\\
-0.6806640625	9.94242111944629\\
-0.68017578125	9.96643857263316\\
-0.6796875	9.99050833755828\\
-0.67919921875	10.0146305084362\\
-0.6787109375	10.0388051798494\\
-0.67822265625	10.0630324467502\\
-0.677734375	10.0873124044623\\
-0.67724609375	10.1116451486828\\
-0.6767578125	10.1360307754835\\
-0.67626953125	10.1604693813133\\
-0.67578125	10.1849610629994\\
-0.67529296875	10.2095059177495\\
-0.6748046875	10.2341040431533\\
-0.67431640625	10.2587555371847\\
-0.673828125	10.2834604982032\\
-0.67333984375	10.308219024956\\
-0.6728515625	10.3330312165797\\
-0.67236328125	10.3578971726025\\
-0.671875	10.3828169929455\\
-0.67138671875	10.4077907779252\\
-0.6708984375	10.432818628255\\
-0.67041015625	10.4579006450473\\
-0.669921875	10.4830369298151\\
-0.66943359375	10.5082275844744\\
-0.6689453125	10.5334727113461\\
-0.66845703125	10.5587724131574\\
-0.66796875	10.5841267930444\\
-0.66748046875	10.6095359545538\\
-0.6669921875	10.635000001645\\
-0.66650390625	10.6605190386921\\
-0.666015625	10.6860931704856\\
-0.66552734375	10.7117225022351\\
-0.6650390625	10.7374071395708\\
-0.66455078125	10.7631471885456\\
-0.6640625	10.7889427556377\\
-0.66357421875	10.8147939477517\\
-0.6630859375	10.8407008722218\\
-0.66259765625	10.8666636368133\\
-0.662109375	10.8926823497249\\
-0.66162109375	10.9187571195905\\
-0.6611328125	10.9448880554819\\
-0.66064453125	10.9710752669107\\
-0.66015625	10.9973188638305\\
-0.65966796875	11.0236189566391\\
-0.6591796875	11.0499756561807\\
-0.65869140625	11.0763890737482\\
-0.658203125	11.1028593210853\\
-0.65771484375	11.1293865103889\\
-0.6572265625	11.1559707543115\\
-0.65673828125	11.1826121659629\\
-0.65625	11.2093108589135\\
-0.65576171875	11.2360669471954\\
-0.6552734375	11.2628805453059\\
-0.65478515625	11.2897517682089\\
-0.654296875	11.3166807313381\\
-0.65380859375	11.3436675505986\\
-0.6533203125	11.3707123423698\\
-0.65283203125	11.3978152235078\\
-0.65234375	11.4249763113474\\
-0.65185546875	11.4521957237051\\
-0.6513671875	11.4794735788812\\
-0.65087890625	11.5068099956623\\
-0.650390625	11.5342050933241\\
-0.64990234375	11.5616589916334\\
-0.6494140625	11.589171810851\\
-0.64892578125	11.6167436717343\\
-0.6484375	11.6443746955394\\
-0.64794921875	11.6720650040244\\
-0.6474609375	11.6998147194512\\
-0.64697265625	11.7276239645886\\
-0.646484375	11.755492862715\\
-0.64599609375	11.7834215376205\\
-0.6455078125	11.8114101136103\\
-0.64501953125	11.8394587155069\\
-0.64453125	11.8675674686528\\
-0.64404296875	11.8957364989134\\
-0.6435546875	11.9239659326797\\
-0.64306640625	11.9522558968711\\
-0.642578125	11.9806065189379\\
-0.64208984375	12.0090179268646\\
-0.6416015625	12.037490249172\\
-0.64111328125	12.0660236149209\\
-0.640625	12.0946181537142\\
-0.64013671875	12.1232739957002\\
-0.6396484375	12.1519912715752\\
-0.63916015625	12.1807701125868\\
-0.638671875	12.2096106505365\\
-0.63818359375	12.2385130177827\\
-0.6376953125	12.2674773472438\\
-0.63720703125	12.2965037724011\\
-0.63671875	12.3255924273019\\
-0.63623046875	12.3547434465622\\
-0.6357421875	12.3839569653704\\
-0.63525390625	12.4132331194897\\
-0.634765625	12.4425720452617\\
-0.63427734375	12.4719738796092\\
-0.6337890625	12.5014387600394\\
-0.63330078125	12.5309668246473\\
-0.6328125	12.5605582121186\\
-0.63232421875	12.5902130617329\\
-0.6318359375	12.6199315133673\\
-0.63134765625	12.6497137074991\\
-0.630859375	12.6795597852096\\
-0.63037109375	12.7094698881871\\
-0.6298828125	12.7394441587302\\
-0.62939453125	12.7694827397514\\
-0.62890625	12.79958577478\\
-0.62841796875	12.8297534079662\\
-0.6279296875	12.8599857840838\\
-0.62744140625	12.8902830485341\\
-0.626953125	12.9206453473492\\
-0.62646484375	12.9510728271954\\
-0.6259765625	12.981565635377\\
-0.62548828125	13.0121239198396\\
-0.625	13.0427478291735\\
-0.62451171875	13.073437512618\\
-0.6240234375	13.104193120064\\
-0.62353515625	13.1350148020585\\
-0.623046875	13.165902709808\\
-0.62255859375	13.196856995182\\
-0.6220703125	13.2278778107168\\
-0.62158203125	13.2589653096195\\
-0.62109375	13.2901196457714\\
-0.62060546875	13.3213409737322\\
-0.6201171875	13.3526294487434\\
-0.61962890625	13.3839852267326\\
-0.619140625	13.4154084643171\\
-0.61865234375	13.4468993188077\\
-0.6181640625	13.4784579482132\\
-0.61767578125	13.5100845112436\\
-0.6171875	13.5417791673148\\
-0.61669921875	13.5735420765522\\
-0.6162109375	13.605373399795\\
-0.61572265625	13.6372732986\\
-0.615234375	13.6692419352461\\
-0.61474609375	13.7012794727382\\
-0.6142578125	13.7333860748113\\
-0.61376953125	13.7655619059352\\
-0.61328125	13.7978071313179\\
-0.61279296875	13.8301219169108\\
-0.6123046875	13.8625064294125\\
-0.61181640625	13.894960836273\\
-0.611328125	13.9274853056985\\
-0.61083984375	13.9600800066555\\
-0.6103515625	13.9927451088754\\
-0.60986328125	14.0254807828588\\
-0.609375	14.0582871998802\\
-0.60888671875	14.0911645319923\\
-0.6083984375	14.124112952031\\
-0.60791015625	14.1571326336194\\
-0.607421875	14.190223751173\\
-0.60693359375	14.2233864799039\\
-0.6064453125	14.256620995826\\
-0.60595703125	14.2899274757594\\
-0.60546875	14.3233060973351\\
-0.60498046875	14.3567570390002\\
-0.6044921875	14.3902804800226\\
-0.60400390625	14.4238766004955\\
-0.603515625	14.4575455813432\\
-0.60302734375	14.4912876043251\\
-0.6025390625	14.5251028520413\\
-0.60205078125	14.5589915079375\\
-0.6015625	14.5929537563101\\
-0.60107421875	14.6269897823112\\
-0.6005859375	14.661099771954\\
-0.60009765625	14.6952839121178\\
-0.599609375	14.7295423905531\\
-0.59912109375	14.7638753958874\\
-0.5986328125	14.7982831176302\\
-0.59814453125	14.8327657461779\\
-0.59765625	14.8673234728203\\
-0.59716796875	14.901956489745\\
-0.5966796875	14.9366649900434\\
-0.59619140625	14.9714491677163\\
-0.595703125	15.0063092176791\\
-0.59521484375	15.041245335768\\
-0.5947265625	15.0762577187447\\
-0.59423828125	15.1113465643033\\
-0.59375	15.1465120710753\\
-0.59326171875	15.1817544386353\\
-0.5927734375	15.2170738675075\\
-0.59228515625	15.252470559171\\
-0.591796875	15.2879447160659\\
-0.59130859375	15.3234965415995\\
-0.5908203125	15.359126240152\\
-0.59033203125	15.3948340170828\\
-0.58984375	15.4306200787366\\
-0.58935546875	15.4664846324494\\
-0.5888671875	15.5024278865552\\
-0.58837890625	15.5384500503915\\
-0.587890625	15.5745513343064\\
-0.58740234375	15.6107319496647\\
-0.5869140625	15.6469921088539\\
-0.58642578125	15.6833320252915\\
-0.5859375	15.7197519134308\\
-0.58544921875	15.756251988768\\
-0.5849609375	15.7928324678484\\
-0.58447265625	15.8294935682735\\
-0.583984375	15.8662355087075\\
-0.58349609375	15.9030585088842\\
-0.5830078125	15.9399627896137\\
-0.58251953125	15.9769485727896\\
-0.58203125	16.0140160813958\\
-0.58154296875	16.0511655395134\\
-0.5810546875	16.088397172328\\
-0.58056640625	16.1257112061369\\
-0.580078125	16.1631078683559\\
-0.57958984375	16.200587387527\\
-0.5791015625	16.2381499933256\\
-0.57861328125	16.2757959165678\\
-0.578125	16.3135253892177\\
-0.57763671875	16.3513386443954\\
-0.5771484375	16.3892359163839\\
-0.57666015625	16.4272174406374\\
-0.576171875	16.4652834537884\\
-0.57568359375	16.5034341936561\\
-0.5751953125	16.5416698992534\\
-0.57470703125	16.5799908107955\\
-0.57421875	16.6183971697076\\
-0.57373046875	16.6568892186329\\
-0.5732421875	16.6954672014408\\
-0.57275390625	16.7341313632347\\
-0.572265625	16.7728819503608\\
-0.57177734375	16.8117192104159\\
-0.5712890625	16.8506433922559\\
-0.57080078125	16.8896547460047\\
-0.5703125	16.9287535230617\\
-0.56982421875	16.9679399761115\\
-0.5693359375	17.0072143591317\\
-0.56884765625	17.0465769274022\\
-0.568359375	17.0860279375137\\
-0.56787109375	17.1255676473766\\
-0.5673828125	17.1651963162299\\
-0.56689453125	17.2049142046507\\
-0.56640625	17.2447215745626\\
-0.56591796875	17.2846186892453\\
-0.5654296875	17.324605813344\\
-0.56494140625	17.3646832128784\\
-0.564453125	17.4048511552523\\
-0.56396484375	17.445109909263\\
-0.5634765625	17.4854597451112\\
-0.56298828125	17.5259009344104\\
-0.5625	17.5664337501966\\
-0.56201171875	17.6070584669383\\
-0.5615234375	17.6477753605468\\
-0.56103515625	17.6885847083852\\
-0.560546875	17.7294867892796\\
-0.56005859375	17.7704818835287\\
-0.5595703125	17.8115702729143\\
-0.55908203125	17.8527522407115\\
-0.55859375	17.8940280716994\\
-0.55810546875	17.9353980521715\\
-0.5576171875	17.9768624699464\\
-0.55712890625	18.0184216143785\\
-0.556640625	18.0600757763691\\
-0.55615234375	18.1018252483771\\
-0.5556640625	18.1436703244298\\
-0.55517578125	18.1856113001349\\
-0.5546875	18.2276484726908\\
-0.55419921875	18.2697821408984\\
-0.5537109375	18.3120126051725\\
-0.55322265625	18.3543401675534\\
-0.552734375	18.3967651317184\\
-0.55224609375	18.4392878029937\\
-0.5517578125	18.4819084883663\\
-0.55126953125	18.5246274964957\\
-0.55078125	18.5674451377262\\
-0.55029296875	18.6103617240994\\
-0.5498046875	18.6533775693658\\
-0.54931640625	18.6964929889977\\
-0.548828125	18.7397083002017\\
-0.54833984375	18.7830238219311\\
-0.5478515625	18.8264398748992\\
-0.54736328125	18.8699567815915\\
-0.546875	18.913574866279\\
-0.54638671875	18.9572944550317\\
-0.5458984375	19.001115875731\\
-0.54541015625	19.045039458084\\
-0.544921875	19.0890655336363\\
-0.54443359375	19.1331944357859\\
-0.5439453125	19.177426499797\\
-0.54345703125	19.2217620628136\\
-0.54296875	19.2662014638738\\
-0.54248046875	19.3107450439238\\
-0.5419921875	19.3553931458321\\
-0.54150390625	19.4001461144042\\
-0.541015625	19.4450042963966\\
-0.54052734375	19.4899680405321\\
-0.5400390625	19.5350376975139\\
-0.53955078125	19.5802136200415\\
-0.5390625	19.6254961628247\\
-0.53857421875	19.6708856825998\\
-0.5380859375	19.7163825381444\\
-0.53759765625	19.7619870902932\\
-0.537109375	19.8076997019533\\
-0.53662109375	19.8535207381208\\
-0.5361328125	19.8994505658958\\
-0.53564453125	19.9454895544995\\
-0.53515625	19.9916380752897\\
-0.53466796875	20.0378965017777\\
-0.5341796875	20.0842652096448\\
-0.53369140625	20.1307445767592\\
-0.533203125	20.1773349831925\\
-0.53271484375	20.2240368112375\\
-0.5322265625	20.2708504454245\\
-0.53173828125	20.3177762725399\\
-0.53125	20.3648146816428\\
-0.53076171875	20.4119660640834\\
-0.5302734375	20.4592308135206\\
-0.52978515625	20.5066093259404\\
-0.529296875	20.5541019996738\\
-0.52880859375	20.601709235416\\
-0.5283203125	20.6494314362443\\
-0.52783203125	20.6972690076373\\
-0.52734375	20.7452223574942\\
-0.52685546875	20.7932918961533\\
-0.5263671875	20.8414780364124\\
-0.52587890625	20.8897811935476\\
-0.525390625	20.9382017853336\\
-0.52490234375	20.9867402320634\\
-0.5244140625	21.0353969565688\\
-0.52392578125	21.0841723842405\\
-0.5234375	21.1330669430492\\
-0.52294921875	21.1820810635659\\
-0.5224609375	21.2312151789833\\
-0.52197265625	21.2804697251368\\
-0.521484375	21.3298451405263\\
-0.52099609375	21.3793418663377\\
-0.5205078125	21.4289603464648\\
-0.52001953125	21.4787010275313\\
-0.51953125	21.5285643589137\\
-0.51904296875	21.5785507927631\\
-0.5185546875	21.6286607840288\\
-0.51806640625	21.6788947904809\\
-0.517578125	21.7292532727337\\
-0.51708984375	21.7797366942692\\
-0.5166015625	21.8303455214611\\
-0.51611328125	21.8810802235985\\
-0.515625	21.9319412729106\\
-0.51513671875	21.9829291445906\\
-0.5146484375	22.0340443168213\\
-0.51416015625	22.0852872707995\\
-0.513671875	22.1366584907615\\
-0.51318359375	22.1881584640088\\
-0.5126953125	22.2397876809341\\
-0.51220703125	22.2915466350466\\
-0.51171875	22.3434358229994\\
-0.51123046875	22.3954557446159\\
-0.5107421875	22.4476069029162\\
-0.51025390625	22.4998898041448\\
-0.509765625	22.5523049577984\\
-0.50927734375	22.604852876653\\
-0.5087890625	22.6575340767928\\
-0.50830078125	22.710349077638\\
-0.5078125	22.7632984019738\\
-0.50732421875	22.8163825759794\\
-0.5068359375	22.8696021292575\\
-0.50634765625	22.9229575948636\\
-0.505859375	22.9764495093365\\
-0.50537109375	23.030078412728\\
-0.5048828125	23.083844848634\\
-0.50439453125	23.1377493642254\\
-0.50390625	23.1917925102791\\
-0.50341796875	23.2459748412102\\
-0.5029296875	23.3002969151034\\
-0.50244140625	23.3547592937457\\
-0.501953125	23.4093625426589\\
-0.50146484375	23.4641072311332\\
-0.5009765625	23.5189939322597\\
-0.50048828125	23.5740232229649\\
-0.5	23.6291956840446\\
-0.49951171875	23.6845119001987\\
-0.4990234375	23.7399724600656\\
-0.49853515625	23.7955779562581\\
-0.498046875	23.8513289853986\\
-0.49755859375	23.9072261481557\\
-0.4970703125	23.9632700492802\\
-0.49658203125	24.0194612976425\\
-0.49609375	24.0758005062697\\
-0.49560546875	24.1322882923831\\
-0.4951171875	24.1889252774369\\
-0.49462890625	24.2457120871565\\
-0.494140625	24.3026493515773\\
-0.49365234375	24.359737705085\\
-0.4931640625	24.4169777864547\\
-0.49267578125	24.4743702388914\\
-0.4921875	24.5319157100715\\
-0.49169921875	24.5896148521834\\
-0.4912109375	24.6474683219695\\
-0.49072265625	24.7054767807685\\
-0.490234375	24.7636408945586\\
-0.48974609375	24.8219613339999\\
-0.4892578125	24.8804387744785\\
-0.48876953125	24.939073896151\\
-0.48828125	24.9978673839891\\
-0.48779296875	25.0568199278248\\
-0.4873046875	25.1159322223962\\
-0.48681640625	25.1752049673938\\
-0.486328125	25.2346388675077\\
-0.48583984375	25.2942346324746\\
-0.4853515625	25.3539929771261\\
-0.48486328125	25.413914621437\\
-0.484375	25.4740002905745\\
-0.48388671875	25.5342507149479\\
-0.4833984375	25.5946666302592\\
-0.48291015625	25.6552487775533\\
-0.482421875	25.7159979032699\\
-0.48193359375	25.7769147592956\\
-0.4814453125	25.8380001030165\\
-0.48095703125	25.8992546973715\\
-0.48046875	25.9606793109062\\
-0.47998046875	26.0222747178279\\
-0.4794921875	26.0840416980603\\
-0.47900390625	26.1459810372999\\
-0.478515625	26.2080935270722\\
-0.47802734375	26.2703799647893\\
-0.4775390625	26.3328411538076\\
-0.47705078125	26.3954779034869\\
-0.4765625	26.4582910292492\\
-0.47607421875	26.5212813526393\\
-0.4755859375	26.5844497013851\\
-0.47509765625	26.6477969094596\\
-0.474609375	26.7113238171434\\
-0.47412109375	26.7750312710871\\
-0.4736328125	26.8389201243755\\
-0.47314453125	26.9029912365925\\
-0.47265625	26.9672454738862\\
-0.47216796875	27.0316837090351\\
-0.4716796875	27.0963068215153\\
-0.47119140625	27.1611156975685\\
-0.470703125	27.22611123027\\
-0.47021484375	27.2912943195994\\
-0.4697265625	27.3566658725101\\
-0.46923828125	27.4222268030012\\
-0.46875	27.487978032189\\
-0.46826171875	27.5539204883807\\
-0.4677734375	27.6200551071483\\
-0.46728515625	27.6863828314034\\
-0.466796875	27.7529046114728\\
-0.46630859375	27.819621405176\\
-0.4658203125	27.8865341779025\\
-0.46533203125	27.9536439026903\\
-0.46484375	28.0209515603066\\
-0.46435546875	28.0884581393275\\
-0.4638671875	28.1561646362205\\
-0.46337890625	28.2240720554271\\
-0.462890625	28.2921814094465\\
-0.46240234375	28.3604937189208\\
-0.4619140625	28.4290100127211\\
-0.46142578125	28.4977313280342\\
-0.4609375	28.5666587104513\\
-0.46044921875	28.635793214057\\
-0.4599609375	28.70513590152\\
-0.45947265625	28.7746878441849\\
-0.458984375	28.8444501221646\\
-0.45849609375	28.914423824435\\
-0.4580078125	28.9846100489296\\
-0.45751953125	29.0550099026363\\
-0.45703125	29.1256245016953\\
-0.45654296875	29.1964549714979\\
-0.4560546875	29.2675024467867\\
-0.45556640625	29.3387680717572\\
-0.455078125	29.4102530001609\\
-0.45458984375	29.4819583954098\\
-0.4541015625	29.553885430681\\
-0.45361328125	29.6260352890248\\
-0.453125	29.6984091634725\\
-0.45263671875	29.7710082571463\\
-0.4521484375	29.8438337833706\\
-0.45166015625	29.9168869657846\\
-0.451171875	29.9901690384566\\
-0.45068359375	30.0636812459997\\
-0.4501953125	30.1374248436889\\
-0.44970703125	30.2114010975797\\
-0.44921875	30.2856112846287\\
-0.44873046875	30.3600566928154\\
-0.4482421875	30.4347386212657\\
-0.44775390625	30.5096583803771\\
-0.447265625	30.5848172919454\\
-0.44677734375	30.6602166892933\\
-0.4462890625	30.7358579174003\\
-0.44580078125	30.8117423330353\\
-0.4453125	30.8878713048894\\
-0.44482421875	30.964246213712\\
-0.4443359375	31.0408684524476\\
-0.44384765625	31.1177394263754\\
-0.443359375	31.1948605532498\\
-0.44287109375	31.2722332634433\\
-0.4423828125	31.3498590000913\\
-0.44189453125	31.4277392192388\\
-0.44140625	31.5058753899888\\
-0.44091796875	31.5842689946531\\
-0.4404296875	31.6629215289047\\
-0.43994140625	31.7418345019328\\
-0.439453125	31.8210094365989\\
-0.43896484375	31.9004478695966\\
-0.4384765625	31.980151351611\\
-0.43798828125	32.0601214474836\\
-0.4375	32.1403597363764\\
-0.43701171875	32.2208678119392\\
-0.4365234375	32.3016472824804\\
-0.43603515625	32.3826997711381\\
-0.435546875	32.4640269160545\\
-0.43505859375	32.5456303705528\\
-0.4345703125	32.6275118033161\\
-0.43408203125	32.7096728985684\\
-0.43359375	32.7921153562593\\
-0.43310546875	32.8748408922489\\
-0.4326171875	32.9578512384976\\
-0.43212890625	33.0411481432571\\
-0.431640625	33.1247333712635\\
-0.43115234375	33.2086087039349\\
-0.4306640625	33.292775939569\\
-0.43017578125	33.3772368935458\\
-0.4296875	33.4619933985311\\
-0.42919921875	33.5470473046838\\
-0.4287109375	33.6324004798656\\
-0.42822265625	33.7180548098534\\
-0.427734375	33.8040121985542\\
-0.42724609375	33.8902745682242\\
-0.4267578125	33.9768438596881\\
-0.42626953125	34.0637220325645\\
-0.42578125	34.1509110654916\\
-0.42529296875	34.2384129563573\\
-0.4248046875	34.3262297225317\\
-0.42431640625	34.4143634011029\\
-0.423828125	34.5028160491153\\
-0.42333984375	34.5915897438118\\
-0.4228515625	34.6806865828782\\
-0.42236328125	34.7701086846916\\
-0.421875	34.8598581885711\\
-0.42138671875	34.9499372550322\\
-0.4208984375	35.0403480660442\\
-0.42041015625	35.131092825291\\
-0.419921875	35.2221737584347\\
-0.41943359375	35.3135931133831\\
-0.4189453125	35.4053531605602\\
-0.41845703125	35.4974561931793\\
-0.41796875	35.589904527521\\
-0.41748046875	35.6827005032128\\
-0.4169921875	35.7758464835134\\
-0.41650390625	35.8693448555994\\
-0.416015625	35.9631980308566\\
-0.41552734375	36.0574084451727\\
-0.4150390625	36.1519785592356\\
-0.41455078125	36.2469108588334\\
-0.4140625	36.3422078551592\\
-0.41357421875	36.437872085118\\
-0.4130859375	36.5339061116376\\
-0.41259765625	36.6303125239834\\
-0.412109375	36.7270939380754\\
-0.41162109375	36.8242529968097\\
-0.4111328125	36.9217923703829\\
-0.41064453125	37.0197147566194\\
-0.41015625	37.118022881303\\
-0.40966796875	37.2167194985106\\
-0.4091796875	37.3158073909501\\
-0.40869140625	37.4152893703015\\
-0.408203125	37.51516827756\\
-0.40771484375	37.6154469833835\\
-0.4072265625	37.7161283884431\\
-0.40673828125	37.8172154237753\\
-0.40625	37.9187110511383\\
-0.40576171875	38.0206182633714\\
-0.4052734375	38.1229400847559\\
-0.40478515625	38.2256795713795\\
-0.404296875	38.3288398115035\\
-0.40380859375	38.4324239259311\\
-0.4033203125	38.5364350683806\\
-0.40283203125	38.6408764258574\\
-0.40234375	38.7457512190309\\
-0.40185546875	38.8510627026116\\
-0.4013671875	38.9568141657313\\
-0.40087890625	39.0630089323232\\
-0.400390625	39.1696503615046\\
-0.39990234375	39.2767418479604\\
-0.3994140625	39.3842868223271\\
-0.39892578125	39.4922887515782\\
-0.3984375	39.6007511394087\\
-0.39794921875	39.7096775266216\\
-0.3974609375	39.8190714915127\\
-0.39697265625	39.9289366502551\\
-0.396484375	40.0392766572832\\
-0.39599609375	40.1500952056753\\
-0.3955078125	40.261396027535\\
-0.39501953125	40.3731828943691\\
-0.39453125	40.4854596174653\\
-0.39404296875	40.5982300482641\\
-0.3935546875	40.7114980787297\\
-0.39306640625	40.8252676417153\\
-0.392578125	40.9395427113247\\
-0.39208984375	41.0543273032674\\
-0.3916015625	41.1696254752089\\
-0.39111328125	41.2854413271138\\
-0.390625	41.4017790015817\\
-0.39013671875	41.5186426841749\\
-0.3896484375	41.6360366037363\\
-0.38916015625	41.7539650326987\\
-0.388671875	41.8724322873818\\
-0.38818359375	41.9914427282786\\
-0.3876953125	42.1110007603266\\
-0.38720703125	42.2311108331677\\
-0.38671875	42.35177744139\\
-0.38623046875	42.4730051247546\\
-0.3857421875	42.5947984684034\\
-0.38525390625	42.7171621030471\\
-0.384765625	42.8401007051336\\
-0.38427734375	42.9636189969911\\
-0.3837890625	43.0877217469489\\
-0.38330078125	43.2124137694301\\
-0.3828125	43.3376999250174\\
-0.38232421875	43.4635851204866\\
-0.3818359375	43.5900743088089\\
-0.38134765625	43.7171724891175\\
-0.380859375	43.844884706636\\
-0.38037109375	43.9732160525672\\
-0.3798828125	44.1021716639392\\
-0.37939453125	44.2317567234041\\
-0.37890625	44.3619764589885\\
-0.37841796875	44.4928361437911\\
-0.3779296875	44.6243410956238\\
-0.37744140625	44.7564966765934\\
-0.376953125	44.889308292619\\
-0.37646484375	45.0227813928812\\
-0.3759765625	45.156921469198\\
-0.37548828125	45.2917340553252\\
-0.375	45.4272247261718\\
-0.37451171875	45.563399096929\\
-0.3740234375	45.7002628221058\\
-0.37353515625	45.8378215944638\\
-0.373046875	45.9760811438471\\
-0.37255859375	46.1150472358996\\
-0.3720703125	46.2547256706618\\
-0.37158203125	46.3951222810402\\
-0.37109375	46.5362429311418\\
-0.37060546875	46.6780935144647\\
-0.3701171875	46.8206799519347\\
-0.36962890625	46.9640081897797\\
-0.369140625	47.1080841972327\\
-0.36865234375	47.2529139640496\\
-0.3681640625	47.3985034978329\\
-0.36767578125	47.5448588211491\\
-0.3671875	47.6919859684259\\
-0.36669921875	47.8398909826159\\
-0.3662109375	47.9885799116142\\
-0.36572265625	48.1380588044125\\
-0.365234375	48.2883337069742\\
-0.36474609375	48.439410657815\\
-0.3642578125	48.5912956832694\\
-0.36376953125	48.7439947924236\\
-0.36328125	48.8975139716981\\
-0.36279296875	49.0518591790523\\
-0.3623046875	49.2070363377961\\
-0.36181640625	49.363051329978\\
-0.361328125	49.5199099893295\\
-0.36083984375	49.6776180937333\\
-0.3603515625	49.8361813571951\\
-0.35986328125	49.9956054212797\\
-0.359375	50.1558958459884\\
-0.35888671875	50.3170581000357\\
-0.3583984375	50.4790975504995\\
-0.35791015625	50.6420194517981\\
-0.357421875	50.805828933961\\
-0.35693359375	50.9705309901492\\
-0.3564453125	51.1361304633809\\
-0.35595703125	51.3026320324176\\
-0.35546875	51.4700401967593\\
-0.35498046875	51.6383592607005\\
-0.3544921875	51.807593316389\\
-0.35400390625	51.9777462258328\\
-0.353515625	52.1488216017943\\
-0.35302734375	52.3208227875074\\
-0.3525390625	52.4937528351515\\
-0.35205078125	52.6676144830136\\
-0.3515625	52.8424101312628\\
-0.35107421875	53.0181418162599\\
-0.3505859375	53.1948111833272\\
-0.35009765625	53.3724194578826\\
-0.349609375	53.5509674148634\\
-0.34912109375	53.7304553463358\\
-0.3486328125	53.9108830272011\\
-0.34814453125	54.0922496788955\\
-0.34765625	54.2745539309796\\
-0.34716796875	54.4577937805116\\
-0.3466796875	54.6419665490877\\
-0.34619140625	54.8270688374412\\
-0.345703125	55.0130964774734\\
-0.34521484375	55.2000444815991\\
-0.3447265625	55.3879069892808\\
-0.34423828125	55.5766772106188\\
-0.34375	55.7663473668747\\
-0.34326171875	55.9569086277874\\
-0.3427734375	56.148351045556\\
-0.34228515625	56.3406634853517\\
-0.341796875	56.5338335522292\\
-0.34130859375	56.7278475143034\\
-0.3408203125	56.9226902220694\\
-0.34033203125	57.118345023738\\
-0.33984375	57.3147936764722\\
-0.33935546875	57.5120162534181\\
-0.3388671875	57.7099910464286\\
-0.33837890625	57.9086944644019\\
-0.337890625	58.1081009271591\\
-0.33740234375	58.3081827548168\\
-0.3369140625	58.5089100526258\\
-0.33642578125	58.7102505912769\\
-0.3359375	58.9121696827058\\
-0.33544921875	59.1146300514618\\
-0.3349609375	59.317591701754\\
-0.33447265625	59.5210117803294\\
-0.333984375	59.7248444353913\\
-0.33349609375	59.9290406718362\\
-0.3330078125	60.1335482031415\\
-0.33251953125	60.338311300327\\
-0.33203125	60.5432706384904\\
-0.33154296875	60.7483631415101\\
-0.3310546875	60.9535218256191\\
-0.33056640625	61.1586756426623\\
-0.330078125	61.3637493239718\\
-0.32958984375	61.5686632259301\\
-0.3291015625	61.7733331784348\\
-0.32861328125	61.9776703376331\\
-0.328125	62.1815810444435\\
-0.32763671875	62.3849666905742\\
-0.3271484375	62.5877235938946\\
-0.32666015625	62.7897428852318\\
-0.326171875	62.9909104088085\\
-0.32568359375	63.1911066387575\\
-0.3251953125	63.3902066143039\\
-0.32470703125	63.5880798963942\\
-0.32421875	63.7845905486932\\
-0.32373046875	63.9795971460395\\
-0.3232421875	64.1729528135415\\
-0.32275390625	64.3645052996036\\
-0.322265625	64.5540970862163\\
-0.32177734375	64.741565539856\\
-0.3212890625	64.9267431062845\\
-0.32080078125	65.1094575524352\\
-0.3203125	65.2895322583944\\
-0.31982421875	65.4667865622243\\
-0.3193359375	65.6410361600352\\
-0.31884765625	65.8120935632689\\
-0.318359375	65.9797686146413\\
-0.31787109375	66.1438690635351\\
-0.3173828125	66.3042012009269\\
-0.31689453125	66.4605705530758\\
-0.31640625	66.6127826323045\\
-0.31591796875	66.7606437421807\\
-0.3154296875	66.9039618333446\\
-0.31494140625	67.0425474051035\\
-0.314453125	67.1762144467702\\
-0.31396484375	67.3047814115606\\
-0.3134765625	67.4280722147629\\
-0.31298828125	67.5459172468167\\
-0.3125	67.6581543909877\\
-0.31201171875	67.7646300344919\\
-0.3115234375	67.8652000612668\\
-0.31103515625	67.9597308141134\\
-0.310546875	68.0481000137309\\
-0.31005859375	68.1301976221519\\
-0.3095703125	68.2059266384205\\
-0.30908203125	68.2752038148953\\
-0.30859375	68.3379602834321\\
-0.30810546875	68.3941420818244\\
-0.3076171875	68.4437105722417\\
-0.30712890625	68.4866427449917\\
-0.306640625	68.5229314027219\\
-0.30615234375	68.5525852220612\\
-0.3056640625	68.5756286916938\\
-0.30517578125	68.5921019278894\\
-0.3046875	68.6020603704727\\
-0.30419921875	68.6055743641465\\
-0.3037109375	68.6027286318321\\
-0.30322265625	68.5936216482992\\
-0.302734375	68.578364923706\\
-0.30224609375	68.5570822078078\\
-0.3017578125	68.5299086264401\\
-0.30126953125	68.4969897624499\\
-0.30078125	68.4584806935576\\
-0.30029296875	68.4145449996313\\
-0.2998046875	68.3653537516466\\
-0.29931640625	68.3110844941389\\
-0.298828125	68.2519202322821\\
-0.29833984375	68.1880484339134\\
-0.2978515625	68.1196600558563\\
-0.29736328125	68.0469486028285\\
-0.296875	67.9701092261212\\
-0.29638671875	67.8893378680585\\
-0.2958984375	67.804830457117\\
-0.29541015625	67.7167821574634\\
-0.294921875	67.6253866755832\\
-0.29443359375	67.5308356256841\\
-0.2939453125	67.4333179546286\\
-0.29345703125	67.3330194263214\\
-0.29296875	67.230122164749\\
-0.29248046875	67.1248042542276\\
-0.2919921875	67.0172393948935\\
-0.29150390625	66.9075966110261\\
-0.291015625	66.7960400094598\\
-0.29052734375	66.6827285850684\\
-0.2900390625	66.5678160701493\\
-0.28955078125	66.4514508243945\\
-0.2890625	66.3337757621249\\
-0.28857421875	66.2149283134411\\
-0.2880859375	66.0950404160067\\
-0.28759765625	65.9742385342771\\
-0.287109375	65.852643703094\\
-0.28662109375	65.7303715927115\\
-0.2861328125	65.6075325924884\\
-0.28564453125	65.4842319106432\\
-0.28515625	65.3605696876519\\
-0.28466796875	65.2366411210579\\
-0.2841796875	65.1125365996361\\
-0.28369140625	64.988341845041\\
-0.283203125	64.8641380592485\\
-0.28271484375	64.7400020762552\\
-0.2822265625	64.6160065166827\\
-0.28173828125	64.4922199440608\\
-0.28125	64.3687070217343\\
-0.28076171875	64.245528669448\\
-0.2802734375	64.1227422188021\\
-0.27978515625	64.0004015668736\\
-0.279296875	63.8785573274136\\
-0.27880859375	63.7572569791178\\
-0.2783203125	63.6365450105484\\
-0.27783203125	63.5164630613751\\
-0.27734375	63.3970500596567\\
-0.27685546875	63.2783423549587\\
-0.2763671875	63.1603738471453\\
-0.27587890625	63.0431761107416\\
-0.275390625	62.9267785147945\\
-0.27490234375	62.8112083382016\\
-0.2744140625	62.6964908805126\\
-0.27392578125	62.5826495682212\\
-0.2734375	62.4697060566075\\
-0.27294921875	62.357680327188\\
-0.2724609375	62.2465907808691\\
-0.27197265625	62.1364543268941\\
-0.271484375	62.0272864676922\\
-0.27099609375	61.919101379751\\
-0.2705078125	61.8119119906321\\
-0.27001953125	61.7057300522532\\
-0.26953125	61.6005662105775\\
-0.26904296875	61.4964300718319\\
-0.2685546875	61.3933302653959\\
-0.26806640625	61.2912745034885\\
-0.267578125	61.1902696377869\\
-0.26708984375	61.0903217131112\\
-0.2666015625	60.9914360182973\\
-0.26611328125	60.8936171343856\\
-0.265625	60.7968689802521\\
-0.26513671875	60.7011948557949\\
-0.2646484375	60.6065974827966\\
-0.26416015625	60.5130790435707\\
-0.263671875	60.4206412175033\\
-0.26318359375	60.3292852155942\\
-0.2626953125	60.2390118130931\\
-0.26220703125	60.1498213803333\\
-0.26171875	60.0617139118494\\
-0.26123046875	59.9746890538756\\
-0.2607421875	59.8887461302984\\
-0.26025390625	59.8038841671515\\
-0.259765625	59.7201019157313\\
-0.25927734375	59.6373978744011\\
-0.2587890625	59.5557703091566\\
-0.25830078125	59.4752172730223\\
-0.2578125	59.395736624337\\
-0.25732421875	59.3173260439932\\
-0.2568359375	59.2399830516847\\
-0.25634765625	59.1637050212188\\
-0.255859375	59.0884891949459\\
-0.25537109375	59.0143326973502\\
-0.2548828125	58.9412325478533\\
-0.25439453125	58.8691856728771\\
-0.25390625	58.7981889171959\\
-0.25341796875	58.7282390546325\\
-0.2529296875	58.659332798125\\
-0.25244140625	58.5914668092032\\
-0.251953125	58.5246377069106\\
-0.25146484375	58.4588420761995\\
-0.2509765625	58.3940764758315\\
-0.25048828125	58.3303374458131\\
-0.25	58.2676215143901\\
-0.24951171875	58.2059252046281\\
-0.2490234375	58.1452450406068\\
-0.24853515625	58.0855775532398\\
-0.248046875	58.0269192857564\\
-0.24755859375	57.9692667988536\\
-0.2470703125	57.9126166755428\\
-0.24658203125	57.8569655257095\\
-0.24609375	57.8023099904018\\
-0.24560546875	57.7486467458625\\
-0.2451171875	57.6959725073262\\
-0.24462890625	57.6442840325873\\
-0.244140625	57.5935781253595\\
-0.24365234375	57.5438516384361\\
-0.2431640625	57.4951014766656\\
-0.24267578125	57.4473245997515\\
-0.2421875	57.400518024891\\
-0.24169921875	57.3546788292575\\
-0.2412109375	57.3098041523426\\
-0.24072265625	57.2658911981637\\
-0.240234375	57.2229372373454\\
-0.23974609375	57.1809396090841\\
-0.2392578125	57.1398957230052\\
-0.23876953125	57.0998030609156\\
-0.23828125	57.0606591784638\\
-0.23779296875	57.022461706712\\
-0.2373046875	56.9852083536239\\
-0.23681640625	56.9488969054802\\
-0.236328125	56.9135252282206\\
-0.23583984375	56.879091268723\\
-0.2353515625	56.8455930560225\\
-0.23486328125	56.8130287024744\\
-0.234375	56.781396404868\\
-0.23388671875	56.7506944454929\\
-0.2333984375	56.7209211931652\\
-0.23291015625	56.6920751042134\\
-0.232421875	56.6641547234309\\
-0.23193359375	56.6371586849972\\
-0.2314453125	56.6110857133729\\
-0.23095703125	56.5859346241677\\
-0.23046875	56.5617043249896\\
-0.22998046875	56.5383938162735\\
-0.2294921875	56.5160021920956\\
-0.22900390625	56.4945286409748\\
-0.228515625	56.4739724466604\\
-0.22802734375	56.4543329889184\\
-0.2275390625	56.4356097443025\\
-0.22705078125	56.4178022869316\\
-0.2265625	56.4009102892576\\
-0.22607421875	56.3849335228371\\
-0.2255859375	56.3698718591076\\
-0.22509765625	56.3557252701629\\
-0.224609375	56.3424938295391\\
-0.22412109375	56.3301777130075\\
-0.2236328125	56.3187771993749\\
-0.22314453125	56.3082926712984\\
-0.22265625	56.2987246161094\\
-0.22216796875	56.2900736266562\\
-0.2216796875	56.2823404021577\\
-0.22119140625	56.2755257490793\\
-0.220703125	56.2696305820227\\
-0.22021484375	56.2646559246399\\
-0.2197265625	56.2606029105657\\
-0.21923828125	56.2574727843745\\
-0.21875	56.2552669025609\\
-0.21826171875	56.2539867345448\\
-0.2177734375	56.2536338637048\\
-0.21728515625	56.2542099884381\\
-0.216796875	56.2557169232485\\
-0.21630859375	56.2581565998679\\
-0.2158203125	56.2615310684057\\
-0.21533203125	56.265842498533\\
-0.21484375	56.2710931806996\\
-0.21435546875	56.2772855273868\\
-0.2138671875	56.2844220743949\\
-0.21337890625	56.2925054821691\\
-0.212890625	56.3015385371651\\
-0.21240234375	56.3115241532499\\
-0.2119140625	56.3224653731477\\
-0.21142578125	56.334365369927\\
-0.2109375	56.3472274485272\\
-0.21044921875	56.3610550473328\\
-0.2099609375	56.375851739793\\
-0.20947265625	56.3916212360834\\
-0.208984375	56.4083673848218\\
-0.20849609375	56.4260941748266\\
-0.2080078125	56.4448057369273\\
-0.20751953125	56.4645063458274\\
-0.20703125	56.485200422017\\
-0.20654296875	56.5068925337388\\
-0.2060546875	56.5295873990096\\
-0.20556640625	56.5532898876969\\
-0.205078125	56.5780050236504\\
-0.20458984375	56.6037379868931\\
-0.2041015625	56.6304941158691\\
-0.20361328125	56.658278909754\\
-0.203125	56.6870980308224\\
-0.20263671875	56.7169573068798\\
-0.2021484375	56.7478627337565\\
-0.20166015625	56.7798204778657\\
-0.201171875	56.8128368788253\\
-0.20068359375	56.8469184521466\\
-0.2001953125	56.8820718919882\\
-0.19970703125	56.9183040739795\\
-0.19921875	56.9556220581094\\
-0.19873046875	56.9940330916854\\
-0.1982421875	57.0335446123621\\
-0.19775390625	57.0741642512403\\
-0.197265625	57.1158998360326\\
-0.19677734375	57.1587593943063\\
-0.1962890625	57.2027511567906\\
-0.19580078125	57.2478835607601\\
-0.1953125	57.294165253484\\
-0.19482421875	57.3416050957508\\
-0.1943359375	57.3902121654612\\
-0.19384765625	57.4399957612882\\
-0.193359375	57.4909654064108\\
-0.19287109375	57.5431308523099\\
-0.1923828125	57.5965020826359\\
-0.19189453125	57.6510893171347\\
-0.19140625	57.7069030156415\\
-0.19091796875	57.7639538821301\\
-0.1904296875	57.8222528688222\\
-0.18994140625	57.8818111803483\\
-0.189453125	57.9426402779594\\
-0.18896484375	58.0047518837823\\
-0.1884765625	58.0681579851178\\
-0.18798828125	58.1328708387692\\
-0.1875	58.1989029754021\\
-0.18701171875	58.2662672039222\\
-0.1865234375	58.3349766158665\\
-0.18603515625	58.405044589794\\
-0.185546875	58.4764847956739\\
-0.18505859375	58.5493111992466\\
-0.1845703125	58.6235380663577\\
-0.18408203125	58.6991799672438\\
-0.18359375	58.7762517807493\\
-0.18310546875	58.8547686984677\\
-0.1826171875	58.9347462287763\\
-0.18212890625	59.0162002007487\\
-0.181640625	59.0991467679173\\
-0.18115234375	59.1836024118586\\
-0.1806640625	59.2695839455735\\
-0.18017578125	59.357108516625\\
-0.1796875	59.4461936100012\\
-0.17919921875	59.5368570506552\\
-0.1787109375	59.6291170056879\\
-0.17822265625	59.7229919861129\\
-0.177734375	59.8185008481551\\
-0.17724609375	59.9156627940188\\
-0.1767578125	60.0144973720615\\
-0.17626953125	60.1150244762985\\
-0.17578125	60.2172643451551\\
-0.17529296875	60.3212375593806\\
-0.1748046875	60.4269650390247\\
-0.17431640625	60.5344680393612\\
-0.173828125	60.6437681456526\\
-0.17333984375	60.754887266613\\
-0.1728515625	60.8678476264278\\
-0.17236328125	60.9826717551724\\
-0.171875	61.0993824774534\\
-0.17138671875	61.2180028990774\\
-0.1708984375	61.3385563915411\\
-0.17041015625	61.4610665740968\\
-0.169921875	61.5855572931527\\
-0.16943359375	61.7120525987088\\
-0.1689453125	61.8405767175304\\
-0.16845703125	61.971154022702\\
-0.16796875	62.1038089992038\\
-0.16748046875	62.2385662050773\\
-0.1669921875	62.3754502277399\\
-0.16650390625	62.5144856349337\\
-0.166015625	62.6556969197722\\
-0.16552734375	62.7991084392678\\
-0.1650390625	62.9447443456823\\
-0.16455078125	63.0926285099681\\
-0.1640625	63.2427844364968\\
-0.16357421875	63.3952351681987\\
-0.1630859375	63.5500031811459\\
-0.16259765625	63.7071102675201\\
-0.162109375	63.8665774058077\\
-0.16162109375	64.0284246169515\\
-0.1611328125	64.19267080507\\
-0.16064453125	64.3593335812262\\
-0.16015625	64.528429068589\\
-0.15966796875	64.699971687183\\
-0.1591796875	64.8739739162506\\
-0.15869140625	65.0504460321003\\
-0.158203125	65.2293958191072\\
-0.15771484375	65.4108282513624\\
-0.1572265625	65.5947451422552\\
-0.15673828125	65.7811447590625\\
-0.15625	65.9700213994141\\
-0.15576171875	66.1613649262786\\
-0.1552734375	66.355160257896\\
-0.15478515625	66.5513868088889\\
-0.154296875	66.7500178785657\\
-0.15380859375	66.9510199822781\\
-0.1533203125	67.1543521215282\\
-0.15283203125	67.3599649884504\\
-0.15234375	67.5678001002292\\
-0.15185546875	67.7777888590939\\
-0.1513671875	67.9898515336549\\
-0.15087890625	68.2038961576555\\
-0.150390625	68.4198173426736\\
-0.14990234375	68.6374950019702\\
-0.1494140625	68.8567929836228\\
-0.14892578125	69.0775576123254\\
-0.1484375	69.2996161408509\\
-0.14794921875	69.5227751142524\\
-0.1474609375	69.7468186524336\\
-0.14697265625	69.9715066599238\\
-0.146484375	70.1965729755298\\
-0.14599609375	70.4217234791307\\
-0.1455078125	70.6466341783355\\
-0.14501953125	70.870949303986\\
-0.14453125	71.0942794506937\\
-0.14404296875	71.3161998066641\\
-0.1435546875	71.5362485259418\\
-0.14306640625	71.7539253057034\\
-0.142578125	71.9686902412142\\
-0.14208984375	72.1799630409711\\
-0.1416015625	72.3871226941105\\
-0.14111328125	72.5895076904998\\
-0.140625	72.78641690036\\
-0.14013671875	72.9771112237098\\
-0.1396484375	73.1608161192848\\
-0.13916015625	73.3367251166628\\
-0.138671875	73.5040044028341\\
-0.13818359375	73.6617985543642\\
-0.1376953125	73.8092374576307\\
-0.13720703125	73.9454444221007\\
-0.13671875	74.0695454453176\\
-0.13623046875	74.180679534316\\
-0.1357421875	74.2780099284983\\
-0.13525390625	74.3607360066391\\
-0.134765625	74.4281055995648\\
-0.13427734375	74.4794273749504\\
-0.1337890625	74.5140829167512\\
-0.13330078125	74.531538094124\\
-0.1328125	74.5313533076189\\
-0.13232421875	74.5131922171094\\
-0.1318359375	74.476828597456\\
-0.13134765625	74.4221510332672\\
-0.130859375	74.349165249909\\
-0.13037109375	74.2579939784779\\
-0.1298828125	74.1488743607105\\
-0.12939453125	74.0221530075277\\
-0.12890625	73.8782789242479\\
-0.12841796875	73.717794599279\\
-0.1279296875	73.5413256160056\\
-0.12744140625	73.3495691864326\\
-0.126953125	73.1432820191238\\
-0.12646484375	72.9232679245147\\
-0.1259765625	72.6903655310081\\
-0.12548828125	72.4454364398545\\
-0.125	72.1893540908208\\
-0.12451171875	71.9229935491868\\
-0.1240234375	71.6472223623548\\
-0.12353515625	71.3628925751408\\
-0.123046875	71.0708339396689\\
-0.12255859375	70.7718483103756\\
-0.1220703125	70.4667051780956\\
-0.12158203125	70.1561382694983\\
-0.12109375	69.8408431189161\\
-0.12060546875	69.5214755079143\\
-0.1201171875	69.1986506626214\\
-0.11962890625	68.8729430987401\\
-0.119140625	68.5448870079636\\
-0.11865234375	68.2149770862103\\
-0.1181640625	67.8836697125914\\
-0.11767578125	67.5513843976501\\
-0.1171875	67.2185054293696\\
-0.11669921875	66.8853836553618\\
-0.1162109375	66.5523383491123\\
-0.11572265625	66.2196591169515\\
-0.115234375	65.8876078103827\\
-0.11474609375	65.5564204155055\\
-0.1142578125	65.2263088974683\\
-0.11376953125	64.8974629831988\\
-0.11328125	64.5700518702046\\
-0.11279296875	64.2442258529966\\
-0.1123046875	63.9201178617854\\
-0.11181640625	63.597844910655\\
-0.111328125	63.2775094543872\\
-0.11083984375	62.9592006547234\\
-0.1103515625	62.642995558039\\
-0.10986328125	62.3289601873092\\
-0.109375	62.0171505519032\\
-0.10888671875	61.7076135791679\\
-0.1083984375	61.4003879720546\\
-0.10791015625	61.0955049971643\\
-0.107421875	60.792989207644\\
-0.10693359375	60.4928591053087\\
-0.1064453125	60.1951277462722\\
-0.10595703125	59.8998032942131\\
-0.10546875	59.6068895252368\\
-0.10498046875	59.3163862880812\\
-0.1044921875	59.0282899232231\\
-0.10400390625	58.7425936442028\\
-0.103515625	58.4592878842904\\
-0.10302734375	58.1783606113851\\
-0.1025390625	57.8997976138445\\
-0.10205078125	57.6235827597316\\
-0.1015625	57.3496982317811\\
-0.10107421875	57.0781247402052\\
-0.1005859375	56.8088417152889\\
-0.10009765625	56.5418274815603\\
-0.099609375	56.2770594151879\\
-0.09912109375	56.0145140860939\\
-0.0986328125	55.7541673861748\\
-0.09814453125	55.4959946448693\\
-0.09765625	55.2399707332405\\
-0.09716796875	54.9860701576\\
-0.0966796875	54.7342671436477\\
-0.09619140625	54.484535711985\\
-0.095703125	54.2368497458019\\
-0.09521484375	53.9911830514606\\
-0.0947265625	53.7475094126276\\
-0.09423828125	53.5058026385612\\
-0.09375	53.2660366070946\\
-0.09326171875	53.0281853028114\\
-0.0927734375	52.7922228508633\\
-0.09228515625	52.5581235468431\\
-0.091796875	52.3258618830818\\
-0.09130859375	52.0954125717127\\
-0.0908203125	51.866750564807\\
-0.09033203125	51.6398510718635\\
-0.08984375	51.4146895749069\\
-0.08935546875	51.191241841424\\
-0.0888671875	50.9694839353501\\
-0.08837890625	50.7493922262951\\
-0.087890625	50.5309433971859\\
-0.08740234375	50.3141144504769\\
-0.0869140625	50.0988827130791\\
-0.08642578125	49.8852258401321\\
-0.0859375	49.6731218177405\\
-0.08544921875	49.4625489647793\\
-0.0849609375	49.2534859338707\\
-0.08447265625	49.0459117116132\\
-0.083984375	48.8398056181511\\
-0.08349609375	48.6351473061516\\
-0.0830078125	48.4319167592581\\
-0.08251953125	48.2300942900806\\
-0.08203125	48.0296605377735\\
-0.08154296875	47.8305964652562\\
-0.0810546875	47.632883356114\\
-0.08056640625	47.4365028112272\\
-0.080078125	47.241436745158\\
-0.07958984375	47.0476673823344\\
-0.0791015625	46.8551772530547\\
-0.07861328125	46.6639491893481\\
-0.078125	46.4739663207054\\
-0.07763671875	46.2852120697116\\
-0.0771484375	46.0976701475934\\
-0.07666015625	45.9113245497038\\
-0.076171875	45.7261595509563\\
-0.07568359375	45.5421597012263\\
-0.0751953125	45.3593098207314\\
-0.07470703125	45.1775949954009\\
-0.07421875	44.9970005722472\\
-0.07373046875	44.8175121547469\\
-0.0732421875	44.6391155982398\\
-0.07275390625	44.4617970053532\\
-0.072265625	44.2855427214602\\
-0.07177734375	44.1103393301728\\
-0.0712890625	43.9361736488797\\
-0.07080078125	43.7630327243309\\
-0.0703125	43.5909038282725\\
-0.06982421875	43.4197744531367\\
-0.0693359375	43.2496323077875\\
-0.06884765625	43.0804653133283\\
-0.068359375	42.9122615989679\\
-0.06787109375	42.7450094979516\\
-0.0673828125	42.5786975435574\\
-0.06689453125	42.4133144651551\\
-0.06640625	42.2488491843354\\
-0.06591796875	42.0852908111027\\
-0.0654296875	41.9226286401383\\
-0.06494140625	41.7608521471288\\
-0.064453125	41.5999509851644\\
-0.06396484375	41.4399149812028\\
-0.0634765625	41.2807341326024\\
-0.06298828125	41.122398603721\\
-0.0625	40.9648987225823\\
-0.06201171875	40.8082249776076\\
-0.0615234375	40.6523680144138\\
-0.06103515625	40.4973186326755\\
-0.060546875	40.3430677830513\\
-0.06005859375	40.189606564174\\
-0.0595703125	40.0369262197016\\
-0.05908203125	39.885018135432\\
-0.05859375	39.7338738364759\\
-0.05810546875	39.5834849844916\\
-0.0576171875	39.4338433749764\\
-0.05712890625	39.2849409346172\\
-0.056640625	39.1367697186968\\
-0.05615234375	38.9893219085567\\
-0.0556640625	38.8425898091131\\
-0.05517578125	38.6965658464282\\
-0.0546875	38.551242565333\\
-0.05419921875	38.4066126271021\\
-0.0537109375	38.262668807178\\
-0.05322265625	38.1194039929465\\
-0.052734375	37.9768111815596\\
-0.05224609375	37.8348834778057\\
-0.0517578125	37.6936140920269\\
-0.05126953125	37.5529963380812\\
-0.05078125	37.4130236313498\\
-0.05029296875	37.2736894867874\\
-0.0498046875	37.1349875170152\\
-0.04931640625	36.9969114304563\\
-0.048828125	36.8594550295103\\
-0.04833984375	36.7226122087696\\
-0.0478515625	36.5863769532732\\
-0.04736328125	36.4507433367992\\
-0.046875	36.3157055201941\\
-0.04638671875	36.1812577497396\\
-0.0458984375	36.0473943555531\\
-0.04541015625	35.9141097500243\\
-0.044921875	35.7813984262856\\
-0.04443359375	35.649254956715\\
-0.0439453125	35.5176739914719\\
-0.04345703125	35.3866502570644\\
-0.04296875	35.2561785549473\\
-0.04248046875	35.1262537601509\\
-0.0419921875	34.9968708199383\\
-0.04150390625	34.8680247524922\\
-0.041015625	34.7397106456302\\
-0.04052734375	34.611923655546\\
-0.0400390625	34.484659005579\\
-0.03955078125	34.3579119850101\\
-0.0390625	34.2316779478813\\
-0.03857421875	34.1059523118423\\
-0.0380859375	33.9807305570208\\
-0.03759765625	33.8560082249162\\
-0.037109375	33.7317809173167\\
-0.03662109375	33.6080442952402\\
-0.0361328125	33.4847940778957\\
-0.03564453125	33.3620260416675\\
-0.03515625	33.2397360191207\\
-0.03466796875	33.1179198980267\\
-0.0341796875	32.9965736204087\\
-0.03369140625	32.8756931816088\\
-0.033203125	32.7552746293713\\
-0.03271484375	32.6353140629481\\
-0.0322265625	32.5158076322198\\
-0.03173828125	32.3967515368366\\
-0.03125	32.278142025376\\
-0.03076171875	32.1599753945168\\
-0.0302734375	32.0422479882324\\
-0.02978515625	31.9249561969972\\
-0.029296875	31.8080964570115\\
-0.02880859375	31.6916652494411\\
-0.0283203125	31.5756590996716\\
-0.02783203125	31.4600745765791\\
-0.02734375	31.3449082918136\\
-0.02685546875	31.2301568990985\\
-0.0263671875	31.1158170935424\\
-0.02587890625	31.001885610966\\
-0.025390625	30.8883592272409\\
-0.02490234375	30.7752347576421\\
-0.0244140625	30.6625090562138\\
-0.02392578125	30.550179015146\\
-0.0234375	30.4382415641652\\
-0.02294921875	30.3266936699347\\
-0.0224609375	30.2155323354692\\
-0.02197265625	30.104754599558\\
-0.021484375	29.9943575362017\\
-0.02099609375	29.8843382540577\\
-0.0205078125	29.7746938958977\\
-0.02001953125	29.6654216380752\\
-0.01953125	29.5565186900028\\
-0.01904296875	29.4479822936396\\
-0.0185546875	29.3398097229887\\
-0.01806640625	29.2319982836034\\
-0.017578125	29.1245453121035\\
-0.01708984375	29.0174481757002\\
-0.0166015625	28.9107042717297\\
-0.01611328125	28.8043110271961\\
-0.015625	28.6982658983224\\
-0.01513671875	28.5925663701098\\
-0.0146484375	28.487209955905\\
-0.01416015625	28.3821941969764\\
-0.013671875	28.2775166620964\\
-0.01318359375	28.1731749471332\\
-0.0126953125	28.0691666746483\\
-0.01220703125	27.9654894935028\\
-0.01171875	27.8621410784703\\
-0.01123046875	27.7591191298558\\
-0.0107421875	27.6564213731234\\
-0.01025390625	27.5540455585293\\
-0.009765625	27.4519894607613\\
-0.00927734375	27.350250878586\\
-0.0087890625	27.2488276345005\\
-0.00830078125	27.1477175743921\\
-0.0078125	27.0469185672027\\
-0.00732421875	26.9464285045995\\
-0.0068359375	26.8462453006516\\
-0.00634765625	26.7463668915129\\
-0.005859375	26.6467912351084\\
-0.00537109375	26.5475163108292\\
-0.0048828125	26.4485401192295\\
-0.00439453125	26.3498606817316\\
-0.00390625	26.2514760403336\\
-0.00341796875	26.1533842573245\\
-0.0029296875	26.055583415002\\
-0.00244140625	25.9580716153968\\
-0.001953125	25.8608469800005\\
-0.00146484375	25.7639076494989\\
-0.0009765625	25.6672517835089\\
-0.00048828125	25.5708775603212\\
0	25.4747831766459\\
0.00048828125	25.5708775603212\\
0.0009765625	25.6672517835089\\
0.00146484375	25.7639076494989\\
0.001953125	25.8608469800005\\
0.00244140625	25.9580716153968\\
0.0029296875	26.055583415002\\
0.00341796875	26.1533842573245\\
0.00390625	26.2514760403336\\
0.00439453125	26.3498606817316\\
0.0048828125	26.4485401192295\\
0.00537109375	26.5475163108292\\
0.005859375	26.6467912351084\\
0.00634765625	26.7463668915129\\
0.0068359375	26.8462453006516\\
0.00732421875	26.9464285045995\\
0.0078125	27.0469185672027\\
0.00830078125	27.1477175743921\\
0.0087890625	27.2488276345005\\
0.00927734375	27.350250878586\\
0.009765625	27.4519894607613\\
0.01025390625	27.5540455585293\\
0.0107421875	27.6564213731234\\
0.01123046875	27.7591191298558\\
0.01171875	27.8621410784703\\
0.01220703125	27.9654894935028\\
0.0126953125	28.0691666746483\\
0.01318359375	28.1731749471332\\
0.013671875	28.2775166620964\\
0.01416015625	28.3821941969764\\
0.0146484375	28.487209955905\\
0.01513671875	28.5925663701098\\
0.015625	28.6982658983224\\
0.01611328125	28.8043110271961\\
0.0166015625	28.9107042717297\\
0.01708984375	29.0174481757002\\
0.017578125	29.1245453121035\\
0.01806640625	29.2319982836034\\
0.0185546875	29.3398097229887\\
0.01904296875	29.4479822936396\\
0.01953125	29.5565186900028\\
0.02001953125	29.6654216380752\\
0.0205078125	29.7746938958977\\
0.02099609375	29.8843382540577\\
0.021484375	29.9943575362017\\
0.02197265625	30.104754599558\\
0.0224609375	30.2155323354692\\
0.02294921875	30.3266936699347\\
0.0234375	30.4382415641652\\
0.02392578125	30.550179015146\\
0.0244140625	30.6625090562138\\
0.02490234375	30.7752347576421\\
0.025390625	30.8883592272409\\
0.02587890625	31.001885610966\\
0.0263671875	31.1158170935424\\
0.02685546875	31.2301568990985\\
0.02734375	31.3449082918136\\
0.02783203125	31.4600745765791\\
0.0283203125	31.5756590996716\\
0.02880859375	31.6916652494411\\
0.029296875	31.8080964570115\\
0.02978515625	31.9249561969972\\
0.0302734375	32.0422479882324\\
0.03076171875	32.1599753945168\\
0.03125	32.278142025376\\
0.03173828125	32.3967515368366\\
0.0322265625	32.5158076322198\\
0.03271484375	32.6353140629481\\
0.033203125	32.7552746293713\\
0.03369140625	32.8756931816088\\
0.0341796875	32.9965736204087\\
0.03466796875	33.1179198980267\\
0.03515625	33.2397360191207\\
0.03564453125	33.3620260416675\\
0.0361328125	33.4847940778957\\
0.03662109375	33.6080442952402\\
0.037109375	33.7317809173167\\
0.03759765625	33.8560082249162\\
0.0380859375	33.9807305570208\\
0.03857421875	34.1059523118423\\
0.0390625	34.2316779478813\\
0.03955078125	34.3579119850101\\
0.0400390625	34.484659005579\\
0.04052734375	34.611923655546\\
0.041015625	34.7397106456302\\
0.04150390625	34.8680247524922\\
0.0419921875	34.9968708199383\\
0.04248046875	35.1262537601509\\
0.04296875	35.2561785549473\\
0.04345703125	35.3866502570644\\
0.0439453125	35.5176739914719\\
0.04443359375	35.649254956715\\
0.044921875	35.7813984262856\\
0.04541015625	35.9141097500243\\
0.0458984375	36.0473943555531\\
0.04638671875	36.1812577497396\\
0.046875	36.3157055201941\\
0.04736328125	36.4507433367992\\
0.0478515625	36.5863769532732\\
0.04833984375	36.7226122087696\\
0.048828125	36.8594550295103\\
0.04931640625	36.9969114304563\\
0.0498046875	37.1349875170152\\
0.05029296875	37.2736894867874\\
0.05078125	37.4130236313498\\
0.05126953125	37.5529963380812\\
0.0517578125	37.6936140920269\\
0.05224609375	37.8348834778057\\
0.052734375	37.9768111815596\\
0.05322265625	38.1194039929465\\
0.0537109375	38.262668807178\\
0.05419921875	38.4066126271021\\
0.0546875	38.551242565333\\
0.05517578125	38.6965658464282\\
0.0556640625	38.8425898091131\\
0.05615234375	38.9893219085567\\
0.056640625	39.1367697186968\\
0.05712890625	39.2849409346172\\
0.0576171875	39.4338433749764\\
0.05810546875	39.5834849844916\\
0.05859375	39.7338738364759\\
0.05908203125	39.885018135432\\
0.0595703125	40.0369262197016\\
0.06005859375	40.189606564174\\
0.060546875	40.3430677830513\\
0.06103515625	40.4973186326755\\
0.0615234375	40.6523680144138\\
0.06201171875	40.8082249776076\\
0.0625	40.9648987225823\\
0.06298828125	41.122398603721\\
0.0634765625	41.2807341326024\\
0.06396484375	41.4399149812028\\
0.064453125	41.5999509851644\\
0.06494140625	41.7608521471288\\
0.0654296875	41.9226286401383\\
0.06591796875	42.0852908111027\\
0.06640625	42.2488491843354\\
0.06689453125	42.4133144651551\\
0.0673828125	42.5786975435574\\
0.06787109375	42.7450094979516\\
0.068359375	42.9122615989679\\
0.06884765625	43.0804653133283\\
0.0693359375	43.2496323077875\\
0.06982421875	43.4197744531367\\
0.0703125	43.5909038282725\\
0.07080078125	43.7630327243309\\
0.0712890625	43.9361736488797\\
0.07177734375	44.1103393301728\\
0.072265625	44.2855427214602\\
0.07275390625	44.4617970053532\\
0.0732421875	44.6391155982398\\
0.07373046875	44.8175121547469\\
0.07421875	44.9970005722472\\
0.07470703125	45.1775949954009\\
0.0751953125	45.3593098207314\\
0.07568359375	45.5421597012263\\
0.076171875	45.7261595509563\\
0.07666015625	45.9113245497038\\
0.0771484375	46.0976701475934\\
0.07763671875	46.2852120697116\\
0.078125	46.4739663207054\\
0.07861328125	46.6639491893481\\
0.0791015625	46.8551772530547\\
0.07958984375	47.0476673823344\\
0.080078125	47.241436745158\\
0.08056640625	47.4365028112272\\
0.0810546875	47.632883356114\\
0.08154296875	47.8305964652562\\
0.08203125	48.0296605377735\\
0.08251953125	48.2300942900806\\
0.0830078125	48.4319167592581\\
0.08349609375	48.6351473061516\\
0.083984375	48.8398056181511\\
0.08447265625	49.0459117116132\\
0.0849609375	49.2534859338707\\
0.08544921875	49.4625489647793\\
0.0859375	49.6731218177405\\
0.08642578125	49.8852258401321\\
0.0869140625	50.0988827130791\\
0.08740234375	50.3141144504769\\
0.087890625	50.5309433971859\\
0.08837890625	50.7493922262951\\
0.0888671875	50.9694839353501\\
0.08935546875	51.191241841424\\
0.08984375	51.4146895749069\\
0.09033203125	51.6398510718635\\
0.0908203125	51.866750564807\\
0.09130859375	52.0954125717127\\
0.091796875	52.3258618830818\\
0.09228515625	52.5581235468431\\
0.0927734375	52.7922228508633\\
0.09326171875	53.0281853028114\\
0.09375	53.2660366070946\\
0.09423828125	53.5058026385612\\
0.0947265625	53.7475094126276\\
0.09521484375	53.9911830514606\\
0.095703125	54.2368497458019\\
0.09619140625	54.484535711985\\
0.0966796875	54.7342671436477\\
0.09716796875	54.9860701576\\
0.09765625	55.2399707332405\\
0.09814453125	55.4959946448693\\
0.0986328125	55.7541673861748\\
0.09912109375	56.0145140860939\\
0.099609375	56.2770594151879\\
0.10009765625	56.5418274815603\\
0.1005859375	56.8088417152889\\
0.10107421875	57.0781247402052\\
0.1015625	57.3496982317811\\
0.10205078125	57.6235827597316\\
0.1025390625	57.8997976138445\\
0.10302734375	58.1783606113851\\
0.103515625	58.4592878842904\\
0.10400390625	58.7425936442028\\
0.1044921875	59.0282899232231\\
0.10498046875	59.3163862880812\\
0.10546875	59.6068895252368\\
0.10595703125	59.8998032942131\\
0.1064453125	60.1951277462722\\
0.10693359375	60.4928591053087\\
0.107421875	60.792989207644\\
0.10791015625	61.0955049971643\\
0.1083984375	61.4003879720546\\
0.10888671875	61.7076135791679\\
0.109375	62.0171505519032\\
0.10986328125	62.3289601873092\\
0.1103515625	62.642995558039\\
0.11083984375	62.9592006547234\\
0.111328125	63.2775094543872\\
0.11181640625	63.597844910655\\
0.1123046875	63.9201178617854\\
0.11279296875	64.2442258529966\\
0.11328125	64.5700518702046\\
0.11376953125	64.8974629831988\\
0.1142578125	65.2263088974683\\
0.11474609375	65.5564204155055\\
0.115234375	65.8876078103827\\
0.11572265625	66.2196591169515\\
0.1162109375	66.5523383491123\\
0.11669921875	66.8853836553618\\
0.1171875	67.2185054293696\\
0.11767578125	67.5513843976501\\
0.1181640625	67.8836697125914\\
0.11865234375	68.2149770862103\\
0.119140625	68.5448870079636\\
0.11962890625	68.8729430987401\\
0.1201171875	69.1986506626214\\
0.12060546875	69.5214755079143\\
0.12109375	69.8408431189161\\
0.12158203125	70.1561382694983\\
0.1220703125	70.4667051780956\\
0.12255859375	70.7718483103756\\
0.123046875	71.0708339396689\\
0.12353515625	71.3628925751408\\
0.1240234375	71.6472223623548\\
0.12451171875	71.9229935491868\\
0.125	72.1893540908208\\
0.12548828125	72.4454364398545\\
0.1259765625	72.6903655310081\\
0.12646484375	72.9232679245147\\
0.126953125	73.1432820191238\\
0.12744140625	73.3495691864326\\
0.1279296875	73.5413256160056\\
0.12841796875	73.717794599279\\
0.12890625	73.8782789242479\\
0.12939453125	74.0221530075277\\
0.1298828125	74.1488743607105\\
0.13037109375	74.2579939784779\\
0.130859375	74.349165249909\\
0.13134765625	74.4221510332672\\
0.1318359375	74.476828597456\\
0.13232421875	74.5131922171094\\
0.1328125	74.5313533076189\\
0.13330078125	74.531538094124\\
0.1337890625	74.5140829167512\\
0.13427734375	74.4794273749504\\
0.134765625	74.4281055995648\\
0.13525390625	74.3607360066391\\
0.1357421875	74.2780099284983\\
0.13623046875	74.180679534316\\
0.13671875	74.0695454453176\\
0.13720703125	73.9454444221007\\
0.1376953125	73.8092374576307\\
0.13818359375	73.6617985543642\\
0.138671875	73.5040044028341\\
0.13916015625	73.3367251166628\\
0.1396484375	73.1608161192848\\
0.14013671875	72.9771112237098\\
0.140625	72.78641690036\\
0.14111328125	72.5895076904998\\
0.1416015625	72.3871226941105\\
0.14208984375	72.1799630409711\\
0.142578125	71.9686902412142\\
0.14306640625	71.7539253057034\\
0.1435546875	71.5362485259418\\
0.14404296875	71.3161998066641\\
0.14453125	71.0942794506937\\
0.14501953125	70.870949303986\\
0.1455078125	70.6466341783355\\
0.14599609375	70.4217234791307\\
0.146484375	70.1965729755298\\
0.14697265625	69.9715066599238\\
0.1474609375	69.7468186524336\\
0.14794921875	69.5227751142524\\
0.1484375	69.2996161408509\\
0.14892578125	69.0775576123254\\
0.1494140625	68.8567929836228\\
0.14990234375	68.6374950019702\\
0.150390625	68.4198173426736\\
0.15087890625	68.2038961576555\\
0.1513671875	67.9898515336549\\
0.15185546875	67.7777888590939\\
0.15234375	67.5678001002292\\
0.15283203125	67.3599649884504\\
0.1533203125	67.1543521215282\\
0.15380859375	66.9510199822781\\
0.154296875	66.7500178785657\\
0.15478515625	66.5513868088889\\
0.1552734375	66.355160257896\\
0.15576171875	66.1613649262786\\
0.15625	65.9700213994141\\
0.15673828125	65.7811447590625\\
0.1572265625	65.5947451422552\\
0.15771484375	65.4108282513624\\
0.158203125	65.2293958191072\\
0.15869140625	65.0504460321003\\
0.1591796875	64.8739739162506\\
0.15966796875	64.699971687183\\
0.16015625	64.528429068589\\
0.16064453125	64.3593335812262\\
0.1611328125	64.19267080507\\
0.16162109375	64.0284246169515\\
0.162109375	63.8665774058077\\
0.16259765625	63.7071102675201\\
0.1630859375	63.5500031811459\\
0.16357421875	63.3952351681987\\
0.1640625	63.2427844364968\\
0.16455078125	63.0926285099681\\
0.1650390625	62.9447443456823\\
0.16552734375	62.7991084392678\\
0.166015625	62.6556969197722\\
0.16650390625	62.5144856349337\\
0.1669921875	62.3754502277399\\
0.16748046875	62.2385662050773\\
0.16796875	62.1038089992038\\
0.16845703125	61.971154022702\\
0.1689453125	61.8405767175304\\
0.16943359375	61.7120525987088\\
0.169921875	61.5855572931527\\
0.17041015625	61.4610665740968\\
0.1708984375	61.3385563915411\\
0.17138671875	61.2180028990774\\
0.171875	61.0993824774534\\
0.17236328125	60.9826717551724\\
0.1728515625	60.8678476264278\\
0.17333984375	60.754887266613\\
0.173828125	60.6437681456526\\
0.17431640625	60.5344680393612\\
0.1748046875	60.4269650390247\\
0.17529296875	60.3212375593806\\
0.17578125	60.2172643451551\\
0.17626953125	60.1150244762985\\
0.1767578125	60.0144973720615\\
0.17724609375	59.9156627940188\\
0.177734375	59.8185008481551\\
0.17822265625	59.7229919861129\\
0.1787109375	59.6291170056879\\
0.17919921875	59.5368570506552\\
0.1796875	59.4461936100012\\
0.18017578125	59.357108516625\\
0.1806640625	59.2695839455735\\
0.18115234375	59.1836024118586\\
0.181640625	59.0991467679173\\
0.18212890625	59.0162002007487\\
0.1826171875	58.9347462287763\\
0.18310546875	58.8547686984677\\
0.18359375	58.7762517807493\\
0.18408203125	58.6991799672438\\
0.1845703125	58.6235380663577\\
0.18505859375	58.5493111992466\\
0.185546875	58.4764847956739\\
0.18603515625	58.405044589794\\
0.1865234375	58.3349766158665\\
0.18701171875	58.2662672039222\\
0.1875	58.1989029754021\\
0.18798828125	58.1328708387692\\
0.1884765625	58.0681579851178\\
0.18896484375	58.0047518837823\\
0.189453125	57.9426402779594\\
0.18994140625	57.8818111803483\\
0.1904296875	57.8222528688222\\
0.19091796875	57.7639538821301\\
0.19140625	57.7069030156415\\
0.19189453125	57.6510893171347\\
0.1923828125	57.5965020826359\\
0.19287109375	57.5431308523099\\
0.193359375	57.4909654064108\\
0.19384765625	57.4399957612882\\
0.1943359375	57.3902121654612\\
0.19482421875	57.3416050957508\\
0.1953125	57.294165253484\\
0.19580078125	57.2478835607601\\
0.1962890625	57.2027511567906\\
0.19677734375	57.1587593943063\\
0.197265625	57.1158998360326\\
0.19775390625	57.0741642512403\\
0.1982421875	57.0335446123621\\
0.19873046875	56.9940330916854\\
0.19921875	56.9556220581094\\
0.19970703125	56.9183040739795\\
0.2001953125	56.8820718919882\\
0.20068359375	56.8469184521466\\
0.201171875	56.8128368788253\\
0.20166015625	56.7798204778657\\
0.2021484375	56.7478627337565\\
0.20263671875	56.7169573068798\\
0.203125	56.6870980308224\\
0.20361328125	56.658278909754\\
0.2041015625	56.6304941158691\\
0.20458984375	56.6037379868931\\
0.205078125	56.5780050236504\\
0.20556640625	56.5532898876969\\
0.2060546875	56.5295873990096\\
0.20654296875	56.5068925337388\\
0.20703125	56.485200422017\\
0.20751953125	56.4645063458274\\
0.2080078125	56.4448057369273\\
0.20849609375	56.4260941748266\\
0.208984375	56.4083673848218\\
0.20947265625	56.3916212360834\\
0.2099609375	56.375851739793\\
0.21044921875	56.3610550473328\\
0.2109375	56.3472274485272\\
0.21142578125	56.334365369927\\
0.2119140625	56.3224653731477\\
0.21240234375	56.3115241532499\\
0.212890625	56.3015385371651\\
0.21337890625	56.2925054821691\\
0.2138671875	56.2844220743949\\
0.21435546875	56.2772855273868\\
0.21484375	56.2710931806996\\
0.21533203125	56.265842498533\\
0.2158203125	56.2615310684057\\
0.21630859375	56.2581565998679\\
0.216796875	56.2557169232485\\
0.21728515625	56.2542099884381\\
0.2177734375	56.2536338637048\\
0.21826171875	56.2539867345448\\
0.21875	56.2552669025609\\
0.21923828125	56.2574727843745\\
0.2197265625	56.2606029105657\\
0.22021484375	56.2646559246399\\
0.220703125	56.2696305820227\\
0.22119140625	56.2755257490793\\
0.2216796875	56.2823404021577\\
0.22216796875	56.2900736266562\\
0.22265625	56.2987246161094\\
0.22314453125	56.3082926712984\\
0.2236328125	56.3187771993749\\
0.22412109375	56.3301777130075\\
0.224609375	56.3424938295391\\
0.22509765625	56.3557252701629\\
0.2255859375	56.3698718591076\\
0.22607421875	56.3849335228371\\
0.2265625	56.4009102892576\\
0.22705078125	56.4178022869316\\
0.2275390625	56.4356097443025\\
0.22802734375	56.4543329889184\\
0.228515625	56.4739724466604\\
0.22900390625	56.4945286409748\\
0.2294921875	56.5160021920956\\
0.22998046875	56.5383938162735\\
0.23046875	56.5617043249896\\
0.23095703125	56.5859346241677\\
0.2314453125	56.6110857133729\\
0.23193359375	56.6371586849972\\
0.232421875	56.6641547234309\\
0.23291015625	56.6920751042134\\
0.2333984375	56.7209211931652\\
0.23388671875	56.7506944454929\\
0.234375	56.781396404868\\
0.23486328125	56.8130287024744\\
0.2353515625	56.8455930560225\\
0.23583984375	56.879091268723\\
0.236328125	56.9135252282206\\
0.23681640625	56.9488969054802\\
0.2373046875	56.9852083536239\\
0.23779296875	57.022461706712\\
0.23828125	57.0606591784638\\
0.23876953125	57.0998030609156\\
0.2392578125	57.1398957230052\\
0.23974609375	57.1809396090841\\
0.240234375	57.2229372373454\\
0.24072265625	57.2658911981637\\
0.2412109375	57.3098041523426\\
0.24169921875	57.3546788292575\\
0.2421875	57.400518024891\\
0.24267578125	57.4473245997515\\
0.2431640625	57.4951014766656\\
0.24365234375	57.5438516384361\\
0.244140625	57.5935781253595\\
0.24462890625	57.6442840325873\\
0.2451171875	57.6959725073262\\
0.24560546875	57.7486467458625\\
0.24609375	57.8023099904018\\
0.24658203125	57.8569655257095\\
0.2470703125	57.9126166755428\\
0.24755859375	57.9692667988536\\
0.248046875	58.0269192857564\\
0.24853515625	58.0855775532398\\
0.2490234375	58.1452450406068\\
0.24951171875	58.2059252046281\\
0.25	58.2676215143901\\
0.25048828125	58.3303374458131\\
0.2509765625	58.3940764758315\\
0.25146484375	58.4588420761995\\
0.251953125	58.5246377069106\\
0.25244140625	58.5914668092032\\
0.2529296875	58.659332798125\\
0.25341796875	58.7282390546325\\
0.25390625	58.7981889171959\\
0.25439453125	58.8691856728771\\
0.2548828125	58.9412325478533\\
0.25537109375	59.0143326973502\\
0.255859375	59.0884891949459\\
0.25634765625	59.1637050212188\\
0.2568359375	59.2399830516847\\
0.25732421875	59.3173260439932\\
0.2578125	59.395736624337\\
0.25830078125	59.4752172730223\\
0.2587890625	59.5557703091566\\
0.25927734375	59.6373978744011\\
0.259765625	59.7201019157313\\
0.26025390625	59.8038841671515\\
0.2607421875	59.8887461302984\\
0.26123046875	59.9746890538756\\
0.26171875	60.0617139118494\\
0.26220703125	60.1498213803333\\
0.2626953125	60.2390118130931\\
0.26318359375	60.3292852155942\\
0.263671875	60.4206412175033\\
0.26416015625	60.5130790435707\\
0.2646484375	60.6065974827966\\
0.26513671875	60.7011948557949\\
0.265625	60.7968689802521\\
0.26611328125	60.8936171343856\\
0.2666015625	60.9914360182973\\
0.26708984375	61.0903217131112\\
0.267578125	61.1902696377869\\
0.26806640625	61.2912745034885\\
0.2685546875	61.3933302653959\\
0.26904296875	61.4964300718319\\
0.26953125	61.6005662105775\\
0.27001953125	61.7057300522532\\
0.2705078125	61.8119119906321\\
0.27099609375	61.919101379751\\
0.271484375	62.0272864676922\\
0.27197265625	62.1364543268941\\
0.2724609375	62.2465907808691\\
0.27294921875	62.357680327188\\
0.2734375	62.4697060566075\\
0.27392578125	62.5826495682212\\
0.2744140625	62.6964908805126\\
0.27490234375	62.8112083382016\\
0.275390625	62.9267785147945\\
0.27587890625	63.0431761107416\\
0.2763671875	63.1603738471453\\
0.27685546875	63.2783423549587\\
0.27734375	63.3970500596567\\
0.27783203125	63.5164630613751\\
0.2783203125	63.6365450105484\\
0.27880859375	63.7572569791178\\
0.279296875	63.8785573274136\\
0.27978515625	64.0004015668736\\
0.2802734375	64.1227422188021\\
0.28076171875	64.245528669448\\
0.28125	64.3687070217343\\
0.28173828125	64.4922199440608\\
0.2822265625	64.6160065166827\\
0.28271484375	64.7400020762552\\
0.283203125	64.8641380592485\\
0.28369140625	64.988341845041\\
0.2841796875	65.1125365996361\\
0.28466796875	65.2366411210579\\
0.28515625	65.3605696876519\\
0.28564453125	65.4842319106432\\
0.2861328125	65.6075325924884\\
0.28662109375	65.7303715927115\\
0.287109375	65.852643703094\\
0.28759765625	65.9742385342771\\
0.2880859375	66.0950404160067\\
0.28857421875	66.2149283134411\\
0.2890625	66.3337757621249\\
0.28955078125	66.4514508243945\\
0.2900390625	66.5678160701493\\
0.29052734375	66.6827285850684\\
0.291015625	66.7960400094598\\
0.29150390625	66.9075966110261\\
0.2919921875	67.0172393948935\\
0.29248046875	67.1248042542276\\
0.29296875	67.230122164749\\
0.29345703125	67.3330194263214\\
0.2939453125	67.4333179546286\\
0.29443359375	67.5308356256841\\
0.294921875	67.6253866755832\\
0.29541015625	67.7167821574634\\
0.2958984375	67.804830457117\\
0.29638671875	67.8893378680585\\
0.296875	67.9701092261212\\
0.29736328125	68.0469486028285\\
0.2978515625	68.1196600558563\\
0.29833984375	68.1880484339134\\
0.298828125	68.2519202322821\\
0.29931640625	68.3110844941389\\
0.2998046875	68.3653537516466\\
0.30029296875	68.4145449996313\\
0.30078125	68.4584806935576\\
0.30126953125	68.4969897624499\\
0.3017578125	68.5299086264401\\
0.30224609375	68.5570822078078\\
0.302734375	68.578364923706\\
0.30322265625	68.5936216482992\\
0.3037109375	68.6027286318321\\
0.30419921875	68.6055743641465\\
0.3046875	68.6020603704727\\
0.30517578125	68.5921019278894\\
0.3056640625	68.5756286916938\\
0.30615234375	68.5525852220612\\
0.306640625	68.5229314027219\\
0.30712890625	68.4866427449917\\
0.3076171875	68.4437105722417\\
0.30810546875	68.3941420818244\\
0.30859375	68.3379602834321\\
0.30908203125	68.2752038148953\\
0.3095703125	68.2059266384205\\
0.31005859375	68.1301976221519\\
0.310546875	68.0481000137309\\
0.31103515625	67.9597308141134\\
0.3115234375	67.8652000612668\\
0.31201171875	67.7646300344919\\
0.3125	67.6581543909877\\
0.31298828125	67.5459172468167\\
0.3134765625	67.4280722147629\\
0.31396484375	67.3047814115606\\
0.314453125	67.1762144467702\\
0.31494140625	67.0425474051035\\
0.3154296875	66.9039618333446\\
0.31591796875	66.7606437421807\\
0.31640625	66.6127826323045\\
0.31689453125	66.4605705530758\\
0.3173828125	66.3042012009269\\
0.31787109375	66.1438690635351\\
0.318359375	65.9797686146413\\
0.31884765625	65.8120935632689\\
0.3193359375	65.6410361600352\\
0.31982421875	65.4667865622243\\
0.3203125	65.2895322583944\\
0.32080078125	65.1094575524352\\
0.3212890625	64.9267431062845\\
0.32177734375	64.741565539856\\
0.322265625	64.5540970862163\\
0.32275390625	64.3645052996036\\
0.3232421875	64.1729528135415\\
0.32373046875	63.9795971460395\\
0.32421875	63.7845905486932\\
0.32470703125	63.5880798963942\\
0.3251953125	63.3902066143039\\
0.32568359375	63.1911066387575\\
0.326171875	62.9909104088085\\
0.32666015625	62.7897428852318\\
0.3271484375	62.5877235938946\\
0.32763671875	62.3849666905742\\
0.328125	62.1815810444435\\
0.32861328125	61.9776703376331\\
0.3291015625	61.7733331784348\\
0.32958984375	61.5686632259301\\
0.330078125	61.3637493239718\\
0.33056640625	61.1586756426623\\
0.3310546875	60.9535218256191\\
0.33154296875	60.7483631415101\\
0.33203125	60.5432706384904\\
0.33251953125	60.338311300327\\
0.3330078125	60.1335482031415\\
0.33349609375	59.9290406718362\\
0.333984375	59.7248444353913\\
0.33447265625	59.5210117803294\\
0.3349609375	59.317591701754\\
0.33544921875	59.1146300514618\\
0.3359375	58.9121696827058\\
0.33642578125	58.7102505912769\\
0.3369140625	58.5089100526258\\
0.33740234375	58.3081827548168\\
0.337890625	58.1081009271591\\
0.33837890625	57.9086944644019\\
0.3388671875	57.7099910464286\\
0.33935546875	57.5120162534181\\
0.33984375	57.3147936764722\\
0.34033203125	57.118345023738\\
0.3408203125	56.9226902220694\\
0.34130859375	56.7278475143034\\
0.341796875	56.5338335522292\\
0.34228515625	56.3406634853517\\
0.3427734375	56.148351045556\\
0.34326171875	55.9569086277874\\
0.34375	55.7663473668747\\
0.34423828125	55.5766772106188\\
0.3447265625	55.3879069892808\\
0.34521484375	55.2000444815991\\
0.345703125	55.0130964774734\\
0.34619140625	54.8270688374412\\
0.3466796875	54.6419665490877\\
0.34716796875	54.4577937805116\\
0.34765625	54.2745539309796\\
0.34814453125	54.0922496788955\\
0.3486328125	53.9108830272011\\
0.34912109375	53.7304553463358\\
0.349609375	53.5509674148634\\
0.35009765625	53.3724194578826\\
0.3505859375	53.1948111833272\\
0.35107421875	53.0181418162599\\
0.3515625	52.8424101312628\\
0.35205078125	52.6676144830136\\
0.3525390625	52.4937528351515\\
0.35302734375	52.3208227875074\\
0.353515625	52.1488216017943\\
0.35400390625	51.9777462258328\\
0.3544921875	51.807593316389\\
0.35498046875	51.6383592607005\\
0.35546875	51.4700401967593\\
0.35595703125	51.3026320324176\\
0.3564453125	51.1361304633809\\
0.35693359375	50.9705309901492\\
0.357421875	50.805828933961\\
0.35791015625	50.6420194517981\\
0.3583984375	50.4790975504995\\
0.35888671875	50.3170581000357\\
0.359375	50.1558958459884\\
0.35986328125	49.9956054212797\\
0.3603515625	49.8361813571951\\
0.36083984375	49.6776180937333\\
0.361328125	49.5199099893295\\
0.36181640625	49.363051329978\\
0.3623046875	49.2070363377961\\
0.36279296875	49.0518591790523\\
0.36328125	48.8975139716981\\
0.36376953125	48.7439947924236\\
0.3642578125	48.5912956832694\\
0.36474609375	48.439410657815\\
0.365234375	48.2883337069742\\
0.36572265625	48.1380588044125\\
0.3662109375	47.9885799116142\\
0.36669921875	47.8398909826159\\
0.3671875	47.6919859684259\\
0.36767578125	47.5448588211491\\
0.3681640625	47.3985034978329\\
0.36865234375	47.2529139640496\\
0.369140625	47.1080841972327\\
0.36962890625	46.9640081897797\\
0.3701171875	46.8206799519347\\
0.37060546875	46.6780935144647\\
0.37109375	46.5362429311418\\
0.37158203125	46.3951222810402\\
0.3720703125	46.2547256706618\\
0.37255859375	46.1150472358996\\
0.373046875	45.9760811438471\\
0.37353515625	45.8378215944638\\
0.3740234375	45.7002628221058\\
0.37451171875	45.563399096929\\
0.375	45.4272247261718\\
0.37548828125	45.2917340553252\\
0.3759765625	45.156921469198\\
0.37646484375	45.0227813928812\\
0.376953125	44.889308292619\\
0.37744140625	44.7564966765934\\
0.3779296875	44.6243410956238\\
0.37841796875	44.4928361437911\\
0.37890625	44.3619764589885\\
0.37939453125	44.2317567234041\\
0.3798828125	44.1021716639392\\
0.38037109375	43.9732160525672\\
0.380859375	43.844884706636\\
0.38134765625	43.7171724891175\\
0.3818359375	43.5900743088089\\
0.38232421875	43.4635851204866\\
0.3828125	43.3376999250174\\
0.38330078125	43.2124137694301\\
0.3837890625	43.0877217469489\\
0.38427734375	42.9636189969911\\
0.384765625	42.8401007051336\\
0.38525390625	42.7171621030471\\
0.3857421875	42.5947984684034\\
0.38623046875	42.4730051247546\\
0.38671875	42.35177744139\\
0.38720703125	42.2311108331677\\
0.3876953125	42.1110007603266\\
0.38818359375	41.9914427282786\\
0.388671875	41.8724322873818\\
0.38916015625	41.7539650326987\\
0.3896484375	41.6360366037363\\
0.39013671875	41.5186426841749\\
0.390625	41.4017790015817\\
0.39111328125	41.2854413271138\\
0.3916015625	41.1696254752089\\
0.39208984375	41.0543273032674\\
0.392578125	40.9395427113247\\
0.39306640625	40.8252676417153\\
0.3935546875	40.7114980787297\\
0.39404296875	40.5982300482641\\
0.39453125	40.4854596174653\\
0.39501953125	40.3731828943691\\
0.3955078125	40.261396027535\\
0.39599609375	40.1500952056753\\
0.396484375	40.0392766572832\\
0.39697265625	39.9289366502551\\
0.3974609375	39.8190714915127\\
0.39794921875	39.7096775266216\\
0.3984375	39.6007511394087\\
0.39892578125	39.4922887515782\\
0.3994140625	39.3842868223271\\
0.39990234375	39.2767418479604\\
0.400390625	39.1696503615046\\
0.40087890625	39.0630089323232\\
0.4013671875	38.9568141657313\\
0.40185546875	38.8510627026116\\
0.40234375	38.7457512190309\\
0.40283203125	38.6408764258574\\
0.4033203125	38.5364350683806\\
0.40380859375	38.4324239259311\\
0.404296875	38.3288398115035\\
0.40478515625	38.2256795713795\\
0.4052734375	38.1229400847559\\
0.40576171875	38.0206182633714\\
0.40625	37.9187110511383\\
0.40673828125	37.8172154237753\\
0.4072265625	37.7161283884431\\
0.40771484375	37.6154469833835\\
0.408203125	37.51516827756\\
0.40869140625	37.4152893703015\\
0.4091796875	37.3158073909501\\
0.40966796875	37.2167194985106\\
0.41015625	37.118022881303\\
0.41064453125	37.0197147566194\\
0.4111328125	36.9217923703829\\
0.41162109375	36.8242529968097\\
0.412109375	36.7270939380754\\
0.41259765625	36.6303125239834\\
0.4130859375	36.5339061116376\\
0.41357421875	36.437872085118\\
0.4140625	36.3422078551592\\
0.41455078125	36.2469108588334\\
0.4150390625	36.1519785592356\\
0.41552734375	36.0574084451727\\
0.416015625	35.9631980308566\\
0.41650390625	35.8693448555994\\
0.4169921875	35.7758464835134\\
0.41748046875	35.6827005032128\\
0.41796875	35.589904527521\\
0.41845703125	35.4974561931793\\
0.4189453125	35.4053531605602\\
0.41943359375	35.3135931133831\\
0.419921875	35.2221737584347\\
0.42041015625	35.131092825291\\
0.4208984375	35.0403480660442\\
0.42138671875	34.9499372550322\\
0.421875	34.8598581885711\\
0.42236328125	34.7701086846916\\
0.4228515625	34.6806865828782\\
0.42333984375	34.5915897438118\\
0.423828125	34.5028160491153\\
0.42431640625	34.4143634011029\\
0.4248046875	34.3262297225317\\
0.42529296875	34.2384129563573\\
0.42578125	34.1509110654916\\
0.42626953125	34.0637220325645\\
0.4267578125	33.9768438596881\\
0.42724609375	33.8902745682242\\
0.427734375	33.8040121985542\\
0.42822265625	33.7180548098534\\
0.4287109375	33.6324004798656\\
0.42919921875	33.5470473046838\\
0.4296875	33.4619933985311\\
0.43017578125	33.3772368935458\\
0.4306640625	33.292775939569\\
0.43115234375	33.2086087039349\\
0.431640625	33.1247333712635\\
0.43212890625	33.0411481432571\\
0.4326171875	32.9578512384976\\
0.43310546875	32.8748408922489\\
0.43359375	32.7921153562593\\
0.43408203125	32.7096728985684\\
0.4345703125	32.6275118033161\\
0.43505859375	32.5456303705528\\
0.435546875	32.4640269160545\\
0.43603515625	32.3826997711381\\
0.4365234375	32.3016472824804\\
0.43701171875	32.2208678119392\\
0.4375	32.1403597363764\\
0.43798828125	32.0601214474836\\
0.4384765625	31.980151351611\\
0.43896484375	31.9004478695966\\
0.439453125	31.8210094365989\\
0.43994140625	31.7418345019328\\
0.4404296875	31.6629215289047\\
0.44091796875	31.5842689946531\\
0.44140625	31.5058753899888\\
0.44189453125	31.4277392192388\\
0.4423828125	31.3498590000913\\
0.44287109375	31.2722332634433\\
0.443359375	31.1948605532498\\
0.44384765625	31.1177394263754\\
0.4443359375	31.0408684524476\\
0.44482421875	30.964246213712\\
0.4453125	30.8878713048894\\
0.44580078125	30.8117423330353\\
0.4462890625	30.7358579174003\\
0.44677734375	30.6602166892933\\
0.447265625	30.5848172919454\\
0.44775390625	30.5096583803771\\
0.4482421875	30.4347386212657\\
0.44873046875	30.3600566928154\\
0.44921875	30.2856112846287\\
0.44970703125	30.2114010975797\\
0.4501953125	30.1374248436889\\
0.45068359375	30.0636812459997\\
0.451171875	29.9901690384566\\
0.45166015625	29.9168869657846\\
0.4521484375	29.8438337833706\\
0.45263671875	29.7710082571463\\
0.453125	29.6984091634725\\
0.45361328125	29.6260352890248\\
0.4541015625	29.553885430681\\
0.45458984375	29.4819583954098\\
0.455078125	29.4102530001609\\
0.45556640625	29.3387680717572\\
0.4560546875	29.2675024467867\\
0.45654296875	29.1964549714979\\
0.45703125	29.1256245016953\\
0.45751953125	29.0550099026363\\
0.4580078125	28.9846100489296\\
0.45849609375	28.914423824435\\
0.458984375	28.8444501221646\\
0.45947265625	28.7746878441849\\
0.4599609375	28.70513590152\\
0.46044921875	28.635793214057\\
0.4609375	28.5666587104513\\
0.46142578125	28.4977313280342\\
0.4619140625	28.4290100127211\\
0.46240234375	28.3604937189208\\
0.462890625	28.2921814094465\\
0.46337890625	28.2240720554271\\
0.4638671875	28.1561646362205\\
0.46435546875	28.0884581393275\\
0.46484375	28.0209515603066\\
0.46533203125	27.9536439026903\\
0.4658203125	27.8865341779025\\
0.46630859375	27.819621405176\\
0.466796875	27.7529046114728\\
0.46728515625	27.6863828314034\\
0.4677734375	27.6200551071483\\
0.46826171875	27.5539204883807\\
0.46875	27.487978032189\\
0.46923828125	27.4222268030012\\
0.4697265625	27.3566658725101\\
0.47021484375	27.2912943195994\\
0.470703125	27.22611123027\\
0.47119140625	27.1611156975685\\
0.4716796875	27.0963068215153\\
0.47216796875	27.0316837090351\\
0.47265625	26.9672454738862\\
0.47314453125	26.9029912365925\\
0.4736328125	26.8389201243755\\
0.47412109375	26.7750312710871\\
0.474609375	26.7113238171434\\
0.47509765625	26.6477969094596\\
0.4755859375	26.5844497013851\\
0.47607421875	26.5212813526393\\
0.4765625	26.4582910292492\\
0.47705078125	26.3954779034869\\
0.4775390625	26.3328411538076\\
0.47802734375	26.2703799647893\\
0.478515625	26.2080935270722\\
0.47900390625	26.1459810372999\\
0.4794921875	26.0840416980603\\
0.47998046875	26.0222747178279\\
0.48046875	25.9606793109062\\
0.48095703125	25.8992546973715\\
0.4814453125	25.8380001030165\\
0.48193359375	25.7769147592956\\
0.482421875	25.7159979032699\\
0.48291015625	25.6552487775533\\
0.4833984375	25.5946666302592\\
0.48388671875	25.5342507149479\\
0.484375	25.4740002905745\\
0.48486328125	25.413914621437\\
0.4853515625	25.3539929771261\\
0.48583984375	25.2942346324746\\
0.486328125	25.2346388675077\\
0.48681640625	25.1752049673938\\
0.4873046875	25.1159322223962\\
0.48779296875	25.0568199278248\\
0.48828125	24.9978673839891\\
0.48876953125	24.939073896151\\
0.4892578125	24.8804387744785\\
0.48974609375	24.8219613339999\\
0.490234375	24.7636408945586\\
0.49072265625	24.7054767807685\\
0.4912109375	24.6474683219695\\
0.49169921875	24.5896148521834\\
0.4921875	24.5319157100715\\
0.49267578125	24.4743702388914\\
0.4931640625	24.4169777864547\\
0.49365234375	24.359737705085\\
0.494140625	24.3026493515773\\
0.49462890625	24.2457120871565\\
0.4951171875	24.1889252774369\\
0.49560546875	24.1322882923831\\
0.49609375	24.0758005062697\\
0.49658203125	24.0194612976425\\
0.4970703125	23.9632700492802\\
0.49755859375	23.9072261481557\\
0.498046875	23.8513289853986\\
0.49853515625	23.7955779562581\\
0.4990234375	23.7399724600656\\
0.49951171875	23.6845119001987\\
0.5	23.6291956840446\\
0.50048828125	23.5740232229649\\
0.5009765625	23.5189939322597\\
0.50146484375	23.4641072311332\\
0.501953125	23.4093625426589\\
0.50244140625	23.3547592937457\\
0.5029296875	23.3002969151034\\
0.50341796875	23.2459748412102\\
0.50390625	23.1917925102791\\
0.50439453125	23.1377493642254\\
0.5048828125	23.083844848634\\
0.50537109375	23.030078412728\\
0.505859375	22.9764495093365\\
0.50634765625	22.9229575948636\\
0.5068359375	22.8696021292575\\
0.50732421875	22.8163825759794\\
0.5078125	22.7632984019738\\
0.50830078125	22.710349077638\\
0.5087890625	22.6575340767928\\
0.50927734375	22.604852876653\\
0.509765625	22.5523049577984\\
0.51025390625	22.4998898041448\\
0.5107421875	22.4476069029162\\
0.51123046875	22.3954557446159\\
0.51171875	22.3434358229994\\
0.51220703125	22.2915466350466\\
0.5126953125	22.2397876809341\\
0.51318359375	22.1881584640088\\
0.513671875	22.1366584907615\\
0.51416015625	22.0852872707995\\
0.5146484375	22.0340443168213\\
0.51513671875	21.9829291445906\\
0.515625	21.9319412729106\\
0.51611328125	21.8810802235985\\
0.5166015625	21.8303455214611\\
0.51708984375	21.7797366942692\\
0.517578125	21.7292532727337\\
0.51806640625	21.6788947904809\\
0.5185546875	21.6286607840288\\
0.51904296875	21.5785507927631\\
0.51953125	21.5285643589137\\
0.52001953125	21.4787010275313\\
0.5205078125	21.4289603464648\\
0.52099609375	21.3793418663377\\
0.521484375	21.3298451405263\\
0.52197265625	21.2804697251368\\
0.5224609375	21.2312151789833\\
0.52294921875	21.1820810635659\\
0.5234375	21.1330669430492\\
0.52392578125	21.0841723842405\\
0.5244140625	21.0353969565688\\
0.52490234375	20.9867402320634\\
0.525390625	20.9382017853336\\
0.52587890625	20.8897811935476\\
0.5263671875	20.8414780364124\\
0.52685546875	20.7932918961533\\
0.52734375	20.7452223574942\\
0.52783203125	20.6972690076373\\
0.5283203125	20.6494314362443\\
0.52880859375	20.601709235416\\
0.529296875	20.5541019996738\\
0.52978515625	20.5066093259404\\
0.5302734375	20.4592308135206\\
0.53076171875	20.4119660640834\\
0.53125	20.3648146816428\\
0.53173828125	20.3177762725399\\
0.5322265625	20.2708504454245\\
0.53271484375	20.2240368112375\\
0.533203125	20.1773349831925\\
0.53369140625	20.1307445767592\\
0.5341796875	20.0842652096448\\
0.53466796875	20.0378965017777\\
0.53515625	19.9916380752897\\
0.53564453125	19.9454895544995\\
0.5361328125	19.8994505658958\\
0.53662109375	19.8535207381208\\
0.537109375	19.8076997019533\\
0.53759765625	19.7619870902932\\
0.5380859375	19.7163825381444\\
0.53857421875	19.6708856825998\\
0.5390625	19.6254961628247\\
0.53955078125	19.5802136200415\\
0.5400390625	19.5350376975139\\
0.54052734375	19.4899680405321\\
0.541015625	19.4450042963966\\
0.54150390625	19.4001461144042\\
0.5419921875	19.3553931458321\\
0.54248046875	19.3107450439238\\
0.54296875	19.2662014638738\\
0.54345703125	19.2217620628136\\
0.5439453125	19.177426499797\\
0.54443359375	19.1331944357859\\
0.544921875	19.0890655336363\\
0.54541015625	19.045039458084\\
0.5458984375	19.001115875731\\
0.54638671875	18.9572944550317\\
0.546875	18.913574866279\\
0.54736328125	18.8699567815915\\
0.5478515625	18.8264398748992\\
0.54833984375	18.7830238219311\\
0.548828125	18.7397083002017\\
0.54931640625	18.6964929889977\\
0.5498046875	18.6533775693658\\
0.55029296875	18.6103617240994\\
0.55078125	18.5674451377262\\
0.55126953125	18.5246274964957\\
0.5517578125	18.4819084883663\\
0.55224609375	18.4392878029937\\
0.552734375	18.3967651317184\\
0.55322265625	18.3543401675534\\
0.5537109375	18.3120126051725\\
0.55419921875	18.2697821408984\\
0.5546875	18.2276484726908\\
0.55517578125	18.1856113001349\\
0.5556640625	18.1436703244298\\
0.55615234375	18.1018252483771\\
0.556640625	18.0600757763691\\
0.55712890625	18.0184216143785\\
0.5576171875	17.9768624699464\\
0.55810546875	17.9353980521715\\
0.55859375	17.8940280716994\\
0.55908203125	17.8527522407115\\
0.5595703125	17.8115702729143\\
0.56005859375	17.7704818835287\\
0.560546875	17.7294867892796\\
0.56103515625	17.6885847083852\\
0.5615234375	17.6477753605468\\
0.56201171875	17.6070584669383\\
0.5625	17.5664337501966\\
0.56298828125	17.5259009344104\\
0.5634765625	17.4854597451112\\
0.56396484375	17.445109909263\\
0.564453125	17.4048511552523\\
0.56494140625	17.3646832128784\\
0.5654296875	17.324605813344\\
0.56591796875	17.2846186892453\\
0.56640625	17.2447215745626\\
0.56689453125	17.2049142046507\\
0.5673828125	17.1651963162299\\
0.56787109375	17.1255676473766\\
0.568359375	17.0860279375137\\
0.56884765625	17.0465769274022\\
0.5693359375	17.0072143591317\\
0.56982421875	16.9679399761115\\
0.5703125	16.9287535230617\\
0.57080078125	16.8896547460047\\
0.5712890625	16.8506433922559\\
0.57177734375	16.8117192104159\\
0.572265625	16.7728819503608\\
0.57275390625	16.7341313632347\\
0.5732421875	16.6954672014408\\
0.57373046875	16.6568892186329\\
0.57421875	16.6183971697076\\
0.57470703125	16.5799908107955\\
0.5751953125	16.5416698992534\\
0.57568359375	16.5034341936561\\
0.576171875	16.4652834537884\\
0.57666015625	16.4272174406374\\
0.5771484375	16.3892359163839\\
0.57763671875	16.3513386443954\\
0.578125	16.3135253892177\\
0.57861328125	16.2757959165678\\
0.5791015625	16.2381499933256\\
0.57958984375	16.200587387527\\
0.580078125	16.1631078683559\\
0.58056640625	16.1257112061369\\
0.5810546875	16.088397172328\\
0.58154296875	16.0511655395134\\
0.58203125	16.0140160813958\\
0.58251953125	15.9769485727896\\
0.5830078125	15.9399627896137\\
0.58349609375	15.9030585088842\\
0.583984375	15.8662355087075\\
0.58447265625	15.8294935682735\\
0.5849609375	15.7928324678484\\
0.58544921875	15.756251988768\\
0.5859375	15.7197519134308\\
0.58642578125	15.6833320252915\\
0.5869140625	15.6469921088539\\
0.58740234375	15.6107319496647\\
0.587890625	15.5745513343064\\
0.58837890625	15.5384500503915\\
0.5888671875	15.5024278865552\\
0.58935546875	15.4664846324494\\
0.58984375	15.4306200787366\\
0.59033203125	15.3948340170828\\
0.5908203125	15.359126240152\\
0.59130859375	15.3234965415995\\
0.591796875	15.2879447160659\\
0.59228515625	15.252470559171\\
0.5927734375	15.2170738675075\\
0.59326171875	15.1817544386353\\
0.59375	15.1465120710753\\
0.59423828125	15.1113465643033\\
0.5947265625	15.0762577187447\\
0.59521484375	15.041245335768\\
0.595703125	15.0063092176791\\
0.59619140625	14.9714491677163\\
0.5966796875	14.9366649900434\\
0.59716796875	14.901956489745\\
0.59765625	14.8673234728203\\
0.59814453125	14.8327657461779\\
0.5986328125	14.7982831176302\\
0.59912109375	14.7638753958874\\
0.599609375	14.7295423905531\\
0.60009765625	14.6952839121178\\
0.6005859375	14.661099771954\\
0.60107421875	14.6269897823112\\
0.6015625	14.5929537563101\\
0.60205078125	14.5589915079375\\
0.6025390625	14.5251028520413\\
0.60302734375	14.4912876043251\\
0.603515625	14.4575455813432\\
0.60400390625	14.4238766004955\\
0.6044921875	14.3902804800226\\
0.60498046875	14.3567570390002\\
0.60546875	14.3233060973351\\
0.60595703125	14.2899274757594\\
0.6064453125	14.256620995826\\
0.60693359375	14.2233864799039\\
0.607421875	14.190223751173\\
0.60791015625	14.1571326336194\\
0.6083984375	14.124112952031\\
0.60888671875	14.0911645319923\\
0.609375	14.0582871998802\\
0.60986328125	14.0254807828588\\
0.6103515625	13.9927451088754\\
0.61083984375	13.9600800066555\\
0.611328125	13.9274853056985\\
0.61181640625	13.894960836273\\
0.6123046875	13.8625064294125\\
0.61279296875	13.8301219169108\\
0.61328125	13.7978071313179\\
0.61376953125	13.7655619059352\\
0.6142578125	13.7333860748113\\
0.61474609375	13.7012794727382\\
0.615234375	13.6692419352461\\
0.61572265625	13.6372732986\\
0.6162109375	13.605373399795\\
0.61669921875	13.5735420765522\\
0.6171875	13.5417791673148\\
0.61767578125	13.5100845112436\\
0.6181640625	13.4784579482132\\
0.61865234375	13.4468993188077\\
0.619140625	13.4154084643171\\
0.61962890625	13.3839852267326\\
0.6201171875	13.3526294487434\\
0.62060546875	13.3213409737322\\
0.62109375	13.2901196457714\\
0.62158203125	13.2589653096195\\
0.6220703125	13.2278778107168\\
0.62255859375	13.196856995182\\
0.623046875	13.165902709808\\
0.62353515625	13.1350148020585\\
0.6240234375	13.104193120064\\
0.62451171875	13.073437512618\\
0.625	13.0427478291735\\
0.62548828125	13.0121239198396\\
0.6259765625	12.981565635377\\
0.62646484375	12.9510728271954\\
0.626953125	12.9206453473492\\
0.62744140625	12.8902830485341\\
0.6279296875	12.8599857840838\\
0.62841796875	12.8297534079662\\
0.62890625	12.79958577478\\
0.62939453125	12.7694827397514\\
0.6298828125	12.7394441587302\\
0.63037109375	12.7094698881871\\
0.630859375	12.6795597852096\\
0.63134765625	12.6497137074991\\
0.6318359375	12.6199315133673\\
0.63232421875	12.5902130617329\\
0.6328125	12.5605582121186\\
0.63330078125	12.5309668246473\\
0.6337890625	12.5014387600394\\
0.63427734375	12.4719738796092\\
0.634765625	12.4425720452617\\
0.63525390625	12.4132331194897\\
0.6357421875	12.3839569653704\\
0.63623046875	12.3547434465622\\
0.63671875	12.3255924273019\\
0.63720703125	12.2965037724011\\
0.6376953125	12.2674773472438\\
0.63818359375	12.2385130177827\\
0.638671875	12.2096106505365\\
0.63916015625	12.1807701125868\\
0.6396484375	12.1519912715752\\
0.64013671875	12.1232739957002\\
0.640625	12.0946181537142\\
0.64111328125	12.0660236149209\\
0.6416015625	12.037490249172\\
0.64208984375	12.0090179268646\\
0.642578125	11.9806065189379\\
0.64306640625	11.9522558968711\\
0.6435546875	11.9239659326797\\
0.64404296875	11.8957364989134\\
0.64453125	11.8675674686528\\
0.64501953125	11.8394587155069\\
0.6455078125	11.8114101136103\\
0.64599609375	11.7834215376205\\
0.646484375	11.755492862715\\
0.64697265625	11.7276239645886\\
0.6474609375	11.6998147194512\\
0.64794921875	11.6720650040244\\
0.6484375	11.6443746955394\\
0.64892578125	11.6167436717343\\
0.6494140625	11.589171810851\\
0.64990234375	11.5616589916334\\
0.650390625	11.5342050933241\\
0.65087890625	11.5068099956623\\
0.6513671875	11.4794735788812\\
0.65185546875	11.4521957237051\\
0.65234375	11.4249763113474\\
0.65283203125	11.3978152235078\\
0.6533203125	11.3707123423698\\
0.65380859375	11.3436675505986\\
0.654296875	11.3166807313381\\
0.65478515625	11.2897517682089\\
0.6552734375	11.2628805453059\\
0.65576171875	11.2360669471954\\
0.65625	11.2093108589135\\
0.65673828125	11.1826121659629\\
0.6572265625	11.1559707543115\\
0.65771484375	11.1293865103889\\
0.658203125	11.1028593210853\\
0.65869140625	11.0763890737482\\
0.6591796875	11.0499756561807\\
0.65966796875	11.0236189566391\\
0.66015625	10.9973188638305\\
0.66064453125	10.9710752669107\\
0.6611328125	10.9448880554819\\
0.66162109375	10.9187571195905\\
0.662109375	10.8926823497249\\
0.66259765625	10.8666636368133\\
0.6630859375	10.8407008722218\\
0.66357421875	10.8147939477517\\
0.6640625	10.7889427556377\\
0.66455078125	10.7631471885456\\
0.6650390625	10.7374071395708\\
0.66552734375	10.7117225022351\\
0.666015625	10.6860931704856\\
0.66650390625	10.6605190386921\\
0.6669921875	10.635000001645\\
0.66748046875	10.6095359545538\\
0.66796875	10.5841267930444\\
0.66845703125	10.5587724131574\\
0.6689453125	10.5334727113461\\
0.66943359375	10.5082275844744\\
0.669921875	10.4830369298151\\
0.67041015625	10.4579006450473\\
0.6708984375	10.432818628255\\
0.67138671875	10.4077907779252\\
0.671875	10.3828169929455\\
0.67236328125	10.3578971726025\\
0.6728515625	10.3330312165797\\
0.67333984375	10.308219024956\\
0.673828125	10.2834604982032\\
0.67431640625	10.2587555371847\\
0.6748046875	10.2341040431533\\
0.67529296875	10.2095059177495\\
0.67578125	10.1849610629994\\
0.67626953125	10.1604693813133\\
0.6767578125	10.1360307754835\\
0.67724609375	10.1116451486828\\
0.677734375	10.0873124044623\\
0.67822265625	10.0630324467502\\
0.6787109375	10.0388051798494\\
0.67919921875	10.0146305084362\\
0.6796875	9.99050833755828\\
0.68017578125	9.96643857263316\\
0.6806640625	9.94242111944629\\
0.68115234375	9.91845588414945\\
0.681640625	9.89454277325902\\
0.68212890625	9.87068169365424\\
0.6826171875	9.84687255257558\\
0.68310546875	9.82311525762302\\
0.68359375	9.79940971675436\\
0.68408203125	9.77575583828367\\
0.6845703125	9.75215353087956\\
0.68505859375	9.72860270356352\\
0.685546875	9.7051032657084\\
0.68603515625	9.68165512703665\\
0.6865234375	9.65825819761886\\
0.68701171875	9.6349123878721\\
0.6875	9.61161760855826\\
0.68798828125	9.5883737707826\\
0.6884765625	9.56518078599213\\
0.68896484375	9.54203856597402\\
0.689453125	9.5189470228541\\
0.68994140625	9.49590606909527\\
0.6904296875	9.47291561749604\\
0.69091796875	9.44997558118898\\
0.69140625	9.42708587363913\\
0.69189453125	9.40424640864264\\
0.6923828125	9.38145710032513\\
0.69287109375	9.3587178631404\\
0.693359375	9.3360286118687\\
0.69384765625	9.31338926161548\\
0.6943359375	9.29079972780978\\
0.69482421875	9.26825992620294\\
0.6953125	9.24576977286704\\
0.69580078125	9.22332918419341\\
0.6962890625	9.20093807689146\\
0.69677734375	9.17859636798697\\
0.697265625	9.15630397482084\\
0.69775390625	9.1340608150477\\
0.6982421875	9.11186680663448\\
0.69873046875	9.08972186785903\\
0.69921875	9.0676259173087\\
0.69970703125	9.04557887387914\\
0.7001953125	9.02358065677273\\
0.70068359375	9.00163118549736\\
0.701171875	8.97973037986508\\
0.70166015625	8.95787815999074\\
0.7021484375	8.93607444629065\\
0.70263671875	8.9143191594813\\
0.703125	8.89261222057807\\
0.70361328125	8.87095355089381\\
0.7041015625	8.84934307203772\\
0.70458984375	8.82778070591391\\
0.705078125	8.80626637472017\\
0.70556640625	8.7848000009468\\
0.7060546875	8.76338150737514\\
0.70654296875	8.74201081707653\\
0.70703125	8.72068785341089\\
0.70751953125	8.69941254002557\\
0.7080078125	8.67818480085408\\
0.70849609375	8.6570045601149\\
0.708984375	8.63587174231024\\
0.70947265625	8.61478627222475\\
0.7099609375	8.59374807492447\\
0.71044921875	8.57275707575547\\
0.7109375	8.55181320034279\\
0.71142578125	8.53091637458913\\
0.7119140625	8.51006652467379\\
0.71240234375	8.4892635770514\\
0.712890625	8.46850745845081\\
0.71337890625	8.44779809587395\\
0.7138671875	8.42713541659461\\
0.71435546875	8.4065193481573\\
0.71484375	8.38594981837623\\
0.71533203125	8.365426755334\\
0.7158203125	8.3449500873806\\
0.71630859375	8.32451974313226\\
0.716796875	8.3041356514703\\
0.71728515625	8.28379774154005\\
0.7177734375	8.26350594274981\\
0.71826171875	8.24326018476962\\
0.71875	8.22306039753032\\
0.71923828125	8.20290651122234\\
0.7197265625	8.18279845629471\\
0.72021484375	8.16273616345395\\
0.720703125	8.14271956366302\\
0.72119140625	8.12274858814028\\
0.7216796875	8.10282316835841\\
0.72216796875	8.08294323604331\\
0.72265625	8.06310872317323\\
0.72314453125	8.04331956197757\\
0.7236328125	8.02357568493593\\
0.72412109375	8.00387702477704\\
0.724609375	7.9842235144778\\
0.72509765625	7.96461508726225\\
0.7255859375	7.94505167660055\\
0.72607421875	7.92553321620793\\
0.7265625	7.90605964004389\\
0.72705078125	7.88663088231091\\
0.7275390625	7.86724687745375\\
0.72802734375	7.84790756015834\\
0.728515625	7.82861286535075\\
0.72900390625	7.8093627281964\\
0.7294921875	7.7901570840989\\
0.72998046875	7.77099586869929\\
0.73046875	7.75187901787494\\
0.73095703125	7.7328064677386\\
0.7314453125	7.71377815463761\\
0.73193359375	7.69479401515279\\
0.732421875	7.67585398609772\\
0.73291015625	7.65695800451748\\
0.7333984375	7.63810600768815\\
0.73388671875	7.61929793311556\\
0.734375	7.60053371853456\\
0.73486328125	7.581813301908\\
0.7353515625	7.56313662142595\\
0.73583984375	7.54450361550473\\
0.736328125	7.52591422278607\\
0.73681640625	7.5073683821361\\
0.7373046875	7.48886603264467\\
0.73779296875	7.47040711362436\\
0.73828125	7.45199156460954\\
0.73876953125	7.43361932535565\\
0.7392578125	7.41529033583829\\
0.73974609375	7.39700453625233\\
0.740234375	7.37876186701106\\
0.74072265625	7.36056226874535\\
0.7412109375	7.34240568230291\\
0.74169921875	7.32429204874727\\
0.7421875	7.30622130935712\\
0.74267578125	7.28819340562534\\
0.7431640625	7.27020827925826\\
0.74365234375	7.25226587217484\\
0.744140625	7.23436612650586\\
0.74462890625	7.216508984593\\
0.7451171875	7.19869438898823\\
0.74560546875	7.18092228245286\\
0.74609375	7.16319260795678\\
0.74658203125	7.14550530867762\\
0.7470703125	7.12786032800012\\
0.74755859375	7.11025760951515\\
0.748046875	7.0926970970191\\
0.74853515625	7.07517873451296\\
0.7490234375	7.05770246620159\\
0.74951171875	7.04026823649311\\
0.75	7.02287598999787\\
0.75048828125	7.00552567152783\\
0.7509765625	6.98821722609593\\
0.75146484375	6.97095059891503\\
0.751953125	6.95372573539747\\
0.75244140625	6.93654258115418\\
0.7529296875	6.91940108199387\\
0.75341796875	6.90230118392251\\
0.75390625	6.88524283314241\\
0.75439453125	6.86822597605158\\
0.7548828125	6.85125055924292\\
0.75537109375	6.83431652950364\\
0.755859375	6.81742383381445\\
0.75634765625	6.80057241934888\\
0.7568359375	6.78376223347249\\
0.75732421875	6.76699322374231\\
0.7578125	6.75026533790605\\
0.75830078125	6.73357852390145\\
0.7587890625	6.71693272985547\\
0.75927734375	6.70032790408375\\
0.759765625	6.68376399508985\\
0.76025390625	6.66724095156455\\
0.7607421875	6.65075872238529\\
0.76123046875	6.63431725661532\\
0.76171875	6.61791650350316\\
0.76220703125	6.60155641248182\\
0.7626953125	6.58523693316831\\
0.76318359375	6.56895801536278\\
0.763671875	6.55271960904802\\
0.76416015625	6.53652166438872\\
0.7646484375	6.52036413173086\\
0.76513671875	6.50424696160098\\
0.765625	6.48817010470569\\
0.76611328125	6.47213351193091\\
0.7666015625	6.45613713434125\\
0.76708984375	6.44018092317941\\
0.767578125	6.42426482986557\\
0.76806640625	6.40838880599662\\
0.7685546875	6.39255280334576\\
0.76904296875	6.37675677386173\\
0.76953125	6.36100066966815\\
0.77001953125	6.34528444306314\\
0.7705078125	6.3296080465184\\
0.77099609375	6.31397143267885\\
0.771484375	6.2983745543619\\
0.77197265625	6.28281736455687\\
0.7724609375	6.26729981642446\\
0.77294921875	6.25182186329602\\
0.7734375	6.23638345867311\\
0.77392578125	6.22098455622677\\
0.7744140625	6.20562510979711\\
0.77490234375	6.19030507339255\\
0.775390625	6.17502440118926\\
0.77587890625	6.15978304753079\\
0.7763671875	6.14458096692718\\
0.77685546875	6.1294181140547\\
0.77734375	6.11429444375503\\
0.77783203125	6.09920991103486\\
0.7783203125	6.08416447106524\\
0.77880859375	6.0691580791811\\
0.779296875	6.05419069088059\\
0.77978515625	6.03926226182465\\
0.7802734375	6.02437274783633\\
0.78076171875	6.00952210490035\\
0.78125	5.9947102891625\\
0.78173828125	5.97993725692913\\
0.7822265625	5.9652029646666\\
0.78271484375	5.95050736900069\\
0.783203125	5.9358504267162\\
0.78369140625	5.92123209475625\\
0.7841796875	5.90665233022189\\
0.78466796875	5.89211109037153\\
0.78515625	5.87760833262036\\
0.78564453125	5.86314401453998\\
0.7861328125	5.84871809385767\\
0.78662109375	5.83433052845611\\
0.787109375	5.81998127637263\\
0.78759765625	5.80567029579894\\
0.7880859375	5.79139754508041\\
0.78857421875	5.77716298271575\\
0.7890625	5.76296656735639\\
0.78955078125	5.74880825780598\\
0.7900390625	5.73468801301993\\
0.79052734375	5.72060579210499\\
0.791015625	5.70656155431863\\
0.79150390625	5.69255525906861\\
0.7919921875	5.67858686591248\\
0.79248046875	5.66465633455715\\
0.79296875	5.65076362485833\\
0.79345703125	5.63690869682013\\
0.7939453125	5.62309151059448\\
0.79443359375	5.60931202648077\\
0.794921875	5.59557020492535\\
0.79541015625	5.58186600652099\\
0.7958984375	5.56819939200651\\
0.79638671875	5.55457032226619\\
0.796875	5.54097875832951\\
0.79736328125	5.52742466137051\\
0.7978515625	5.51390799270738\\
0.79833984375	5.50042871380206\\
0.798828125	5.48698678625967\\
0.79931640625	5.47358217182826\\
0.7998046875	5.46021483239819\\
0.80029296875	5.44688473000169\\
0.80078125	5.43359182681251\\
0.80126953125	5.42033608514547\\
0.8017578125	5.40711746745588\\
0.80224609375	5.39393593633934\\
0.802734375	5.38079145453108\\
0.80322265625	5.36768398490562\\
0.8037109375	5.35461349047645\\
0.80419921875	5.34157993439539\\
0.8046875	5.32858327995232\\
0.80517578125	5.3156234905747\\
0.8056640625	5.30270052982716\\
0.80615234375	5.28981436141113\\
0.806640625	5.27696494916427\\
0.80712890625	5.26415225706024\\
0.8076171875	5.25137624920821\\
0.80810546875	5.2386368898524\\
0.80859375	5.22593414337176\\
0.80908203125	5.21326797427952\\
0.8095703125	5.20063834722278\\
0.81005859375	5.18804522698211\\
0.810546875	5.17548857847117\\
0.81103515625	5.16296836673634\\
0.8115234375	5.15048455695623\\
0.81201171875	5.13803711444136\\
0.8125	5.12562600463374\\
0.81298828125	5.11325119310652\\
0.8134765625	5.10091264556356\\
0.81396484375	5.08861032783903\\
0.814453125	5.07634420589712\\
0.81494140625	5.06411424583152\\
0.8154296875	5.05192041386516\\
0.81591796875	5.03976267634977\\
0.81640625	5.02764099976551\\
0.81689453125	5.01555535072062\\
0.8173828125	5.00350569595098\\
0.81787109375	4.9914920023199\\
0.818359375	4.97951423681757\\
0.81884765625	4.96757236656073\\
0.8193359375	4.95566635879243\\
0.81982421875	4.94379618088152\\
0.8203125	4.93196180032238\\
0.82080078125	4.92016318473451\\
0.8212890625	4.9084003018622\\
0.82177734375	4.89667311957417\\
0.822265625	4.88498160586319\\
0.82275390625	4.8733257288458\\
0.8232421875	4.86170545676187\\
0.82373046875	4.8501207579743\\
0.82421875	4.83857160096868\\
0.82470703125	4.82705795435294\\
0.8251953125	4.81557978685692\\
0.82568359375	4.80413706733224\\
0.826171875	4.79272976475173\\
0.82666015625	4.7813578482092\\
0.8271484375	4.77002128691917\\
0.82763671875	4.75872005021633\\
0.828125	4.74745410755546\\
0.82861328125	4.73622342851091\\
0.8291015625	4.72502798277636\\
0.82958984375	4.71386774016446\\
0.830078125	4.7027426706065\\
0.83056640625	4.69165274415213\\
0.8310546875	4.68059793096901\\
0.83154296875	4.66957820134246\\
0.83203125	4.65859352567515\\
0.83251953125	4.64764387448685\\
0.8330078125	4.63672921841407\\
0.83349609375	4.62584952820965\\
0.833984375	4.6150047747426\\
0.83447265625	4.60419492899775\\
0.8349609375	4.5934199620754\\
0.83544921875	4.58267984519092\\
0.8359375	4.57197454967475\\
0.83642578125	4.56130404697171\\
0.8369140625	4.55066830864098\\
0.83740234375	4.54006730635571\\
0.837890625	4.52950101190264\\
0.83837890625	4.51896939718193\\
0.8388671875	4.50847243420679\\
0.83935546875	4.4980100951032\\
0.83984375	4.48758235210959\\
0.84033203125	4.47718917757662\\
0.8408203125	4.4668305439668\\
0.84130859375	4.45650642385427\\
0.841796875	4.44621678992449\\
0.84228515625	4.4359616149739\\
0.8427734375	4.4257408719097\\
0.84326171875	4.41555453374957\\
0.84375	4.40540257362139\\
0.84423828125	4.39528496476286\\
0.8447265625	4.38520168052136\\
0.84521484375	4.37515269435359\\
0.845703125	4.36513797982531\\
0.84619140625	4.35515751061104\\
0.8466796875	4.3452112604939\\
0.84716796875	4.33529920336515\\
0.84765625	4.32542131322405\\
0.84814453125	4.31557756417759\\
0.8486328125	4.30576793044019\\
0.84912109375	4.29599238633347\\
0.849609375	4.28625090628582\\
0.85009765625	4.27654346483243\\
0.8505859375	4.26687003661475\\
0.85107421875	4.25723059638039\\
0.8515625	4.24762511898286\\
0.85205078125	4.23805357938118\\
0.8525390625	4.22851595263978\\
0.85302734375	4.21901221392811\\
0.853515625	4.20954233852052\\
0.85400390625	4.2001063017959\\
0.8544921875	4.19070407923747\\
0.85498046875	4.18133564643255\\
0.85546875	4.17200097907221\\
0.85595703125	4.16270005295119\\
0.8564453125	4.15343284396754\\
0.85693359375	4.14419932812237\\
0.857421875	4.13499948151964\\
0.85791015625	4.12583328036592\\
0.8583984375	4.11670070097015\\
0.85888671875	4.10760171974339\\
0.859375	4.09853631319856\\
0.85986328125	4.08950445795023\\
0.8603515625	4.0805061307144\\
0.86083984375	4.07154130830824\\
0.861328125	4.06260996764983\\
0.86181640625	4.05371208575798\\
0.8623046875	4.04484763975196\\
0.86279296875	4.03601660685132\\
0.86328125	4.02721896437564\\
0.86376953125	4.01845468974419\\
0.8642578125	4.00972376047591\\
0.86474609375	4.00102615418902\\
0.865234375	3.99236184860091\\
0.86572265625	3.98373082152781\\
0.8662109375	3.97513305088463\\
0.86669921875	3.96656851468476\\
0.8671875	3.95803719103978\\
0.86767578125	3.94953905815933\\
0.8681640625	3.94107409435079\\
0.86865234375	3.93264227801919\\
0.869140625	3.92424358766688\\
0.86962890625	3.91587800189338\\
0.8701171875	3.90754549939513\\
0.87060546875	3.89924605896532\\
0.87109375	3.89097965949372\\
0.87158203125	3.88274627996627\\
0.8720703125	3.87454589946515\\
0.87255859375	3.86637849716837\\
0.873046875	3.85824405234962\\
0.87353515625	3.85014254437819\\
0.8740234375	3.84207395271851\\
0.87451171875	3.83403825693019\\
0.875	3.82603543666769\\
0.87548828125	3.81806547168019\\
0.8759765625	3.81012834181127\\
0.87646484375	3.80222402699889\\
0.876953125	3.79435250727505\\
0.87744140625	3.78651376276568\\
0.8779296875	3.77870777369041\\
0.87841796875	3.77093452036234\\
0.87890625	3.76319398318792\\
0.87939453125	3.75548614266671\\
0.8798828125	3.74781097939123\\
0.88037109375	3.74016847404674\\
0.880859375	3.73255860741105\\
0.88134765625	3.72498136035433\\
0.8818359375	3.71743671383898\\
0.88232421875	3.70992464891938\\
0.8828125	3.7024451467417\\
0.88330078125	3.69499818854382\\
0.8837890625	3.68758375565499\\
0.88427734375	3.68020182949576\\
0.884765625	3.67285239157787\\
0.88525390625	3.66553542350383\\
0.8857421875	3.65825090696698\\
0.88623046875	3.65099882375123\\
0.88671875	3.6437791557308\\
0.88720703125	3.63659188487019\\
0.8876953125	3.62943699322396\\
0.88818359375	3.62231446293646\\
0.888671875	3.61522427624182\\
0.88916015625	3.60816641546361\\
0.8896484375	3.60114086301483\\
0.89013671875	3.59414760139764\\
0.890625	3.58718661320323\\
0.89111328125	3.5802578811116\\
0.8916015625	3.57336138789154\\
0.89208984375	3.56649711640026\\
0.892578125	3.55966504958335\\
0.89306640625	3.5528651704747\\
0.8935546875	3.54609746219607\\
0.89404296875	3.53936190795727\\
0.89453125	3.53265849105566\\
0.89501953125	3.52598719487632\\
0.8955078125	3.51934800289161\\
0.89599609375	3.51274089866115\\
0.896484375	3.50616586583175\\
0.89697265625	3.49962288813699\\
0.8974609375	3.49311194939738\\
0.89794921875	3.48663303351998\\
0.8984375	3.48018612449829\\
0.89892578125	3.47377120641222\\
0.8994140625	3.4673882634278\\
0.89990234375	3.46103727979708\\
0.900390625	3.45471823985803\\
0.90087890625	3.44843112803427\\
0.9013671875	3.44217592883504\\
0.90185546875	3.43595262685507\\
0.90234375	3.42976120677428\\
0.90283203125	3.4236016533578\\
0.9033203125	3.41747395145573\\
0.90380859375	3.41137808600306\\
0.904296875	3.40531404201948\\
0.90478515625	3.39928180460926\\
0.9052734375	3.3932813589611\\
0.90576171875	3.38731269034802\\
0.90625	3.3813757841272\\
0.90673828125	3.37547062573986\\
0.9072265625	3.36959720071106\\
0.90771484375	3.36375549464965\\
0.908203125	3.35794549324813\\
0.90869140625	3.3521671822824\\
0.9091796875	3.34642054761182\\
0.90966796875	3.34070557517888\\
0.91015625	3.33502225100926\\
0.91064453125	3.32937056121148\\
0.9111328125	3.323750491977\\
0.91162109375	3.31816202957996\\
0.912109375	3.31260516037704\\
0.91259765625	3.3070798708074\\
0.9130859375	3.30158614739254\\
0.91357421875	3.29612397673609\\
0.9140625	3.29069334552387\\
0.91455078125	3.28529424052356\\
0.9150390625	3.27992664858473\\
0.91552734375	3.27459055663859\\
0.916015625	3.26928595169803\\
0.91650390625	3.26401282085734\\
0.9169921875	3.25877115129215\\
0.91748046875	3.25356093025941\\
0.91796875	3.24838214509712\\
0.91845703125	3.24323478322425\\
0.9189453125	3.23811883214073\\
0.91943359375	3.2330342794272\\
0.919921875	3.22798111274493\\
0.92041015625	3.22295931983582\\
0.9208984375	3.21796888852214\\
0.92138671875	3.21300980670645\\
0.921875	3.20808206237158\\
0.92236328125	3.20318564358039\\
0.9228515625	3.19832053847578\\
0.92333984375	3.19348673528047\\
0.923828125	3.18868422229701\\
0.92431640625	3.18391298790754\\
0.9248046875	3.17917302057383\\
0.92529296875	3.17446430883706\\
0.92578125	3.16978684131775\\
0.92626953125	3.16514060671568\\
0.9267578125	3.16052559380975\\
0.92724609375	3.1559417914579\\
0.927734375	3.15138918859702\\
0.92822265625	3.14686777424283\\
0.9287109375	3.14237753748975\\
0.92919921875	3.13791846751087\\
0.9296875	3.13349055355783\\
0.93017578125	3.12909378496067\\
0.9306640625	3.12472815112778\\
0.93115234375	3.12039364154581\\
0.931640625	3.11609024577952\\
0.93212890625	3.11181795347177\\
0.9326171875	3.10757675434338\\
0.93310546875	3.103366638193\\
0.93359375	3.09918759489712\\
0.93408203125	3.09503961440981\\
0.9345703125	3.09092268676283\\
0.93505859375	3.0868368020654\\
0.935546875	3.08278195050417\\
0.93603515625	3.07875812234308\\
0.9365234375	3.07476530792333\\
0.93701171875	3.07080349766329\\
0.9375	3.06687268205836\\
0.93798828125	3.06297285168094\\
0.9384765625	3.05910399718026\\
0.93896484375	3.05526610928248\\
0.939453125	3.05145917879035\\
0.93994140625	3.04768319658335\\
0.9404296875	3.04393815361748\\
0.94091796875	3.04022404092523\\
0.94140625	3.03654084961548\\
0.94189453125	3.03288857087345\\
0.9423828125	3.02926719596056\\
0.94287109375	3.02567671621439\\
0.943359375	3.02211712304866\\
0.94384765625	3.01858840795303\\
0.9443359375	3.0150905624931\\
0.94482421875	3.01162357831034\\
0.9453125	3.00818744712201\\
0.94580078125	3.00478216072103\\
0.9462890625	3.00140771097597\\
0.94677734375	2.99806408983096\\
0.947265625	2.99475128930562\\
0.94775390625	2.991469301495\\
0.9482421875	2.98821811856939\\
0.94873046875	2.98499773277452\\
0.94921875	2.98180813643115\\
0.94970703125	2.97864932193532\\
0.9501953125	2.97552128175806\\
0.95068359375	2.97242400844537\\
0.951171875	2.96935749461827\\
0.95166015625	2.96632173297258\\
0.9521484375	2.96331671627894\\
0.95263671875	2.96034243738273\\
0.953125	2.95739888920401\\
};
\addplot [color=mycolor1,solid]
  table[row sep=crcr]{0.953125	2.95739888920401\\
0.95361328125	2.95448606473745\\
0.9541015625	2.95160395705224\\
0.95458984375	2.94875255929205\\
0.955078125	2.94593186467504\\
0.95556640625	2.94314186649369\\
0.9560546875	2.94038255811479\\
0.95654296875	2.93765393297936\\
0.95703125	2.93495598460261\\
0.95751953125	2.93228870657394\\
0.9580078125	2.92965209255672\\
0.95849609375	2.92704613628841\\
0.958984375	2.92447083158042\\
0.95947265625	2.92192617231807\\
0.9599609375	2.91941215246047\\
0.96044921875	2.91692876604061\\
0.9609375	2.91447600716517\\
0.96142578125	2.91205387001461\\
0.9619140625	2.90966234884287\\
0.96240234375	2.90730143797764\\
0.962890625	2.90497113182006\\
0.96337890625	2.90267142484481\\
0.9638671875	2.90040231159996\\
0.96435546875	2.898163786707\\
0.96484375	2.89595584486077\\
0.96533203125	2.8937784808294\\
0.9658203125	2.89163168945431\\
0.96630859375	2.88951546565006\\
0.966796875	2.88742980440439\\
0.96728515625	2.88537470077821\\
0.9677734375	2.88335014990542\\
0.96826171875	2.88135614699304\\
0.96875	2.879392687321\\
0.96923828125	2.87745976624223\\
0.9697265625	2.87555737918252\\
0.97021484375	2.87368552164059\\
0.970703125	2.87184418918793\\
0.97119140625	2.87003337746879\\
0.9716796875	2.86825308220026\\
0.97216796875	2.86650329917208\\
0.97265625	2.86478402424666\\
0.97314453125	2.86309525335904\\
0.9736328125	2.86143698251692\\
0.97412109375	2.85980920780052\\
0.974609375	2.85821192536256\\
0.97509765625	2.85664513142832\\
0.9755859375	2.8551088222955\\
0.97607421875	2.85360299433425\\
0.9765625	2.85212764398714\\
0.97705078125	2.85068276776905\\
0.9775390625	2.84926836226723\\
0.97802734375	2.84788442414126\\
0.978515625	2.84653095012295\\
0.97900390625	2.84520793701641\\
0.9794921875	2.84391538169791\\
0.97998046875	2.84265328111597\\
0.98046875	2.84142163229123\\
0.98095703125	2.84022043231652\\
0.9814453125	2.83904967835674\\
0.98193359375	2.8379093676489\\
0.982421875	2.83679949750208\\
0.98291015625	2.83572006529738\\
0.9833984375	2.83467106848792\\
0.98388671875	2.83365250459887\\
0.984375	2.83266437122729\\
0.98486328125	2.83170666604226\\
0.9853515625	2.83077938678478\\
0.98583984375	2.8298825312677\\
0.986328125	2.82901609737588\\
0.98681640625	2.82818008306593\\
0.9873046875	2.82737448636644\\
0.98779296875	2.82659930537771\\
0.98828125	2.82585453827198\\
0.98876953125	2.82514018329322\\
0.9892578125	2.82445623875724\\
0.98974609375	2.8238027030516\\
0.990234375	2.82317957463562\\
0.99072265625	2.8225868520404\\
0.9912109375	2.82202453386874\\
0.99169921875	2.82149261879518\\
0.9921875	2.82099110556597\\
0.99267578125	2.82051999299908\\
0.9931640625	2.82007927998412\\
0.99365234375	2.81966896548247\\
0.994140625	2.81928904852706\\
0.99462890625	2.8189395282226\\
0.9951171875	2.8186204037454\\
0.99560546875	2.8183316743434\\
0.99609375	2.81807333933627\\
0.99658203125	2.81784539811519\\
0.9970703125	2.81764785014307\\
0.99755859375	2.8174806949544\\
0.998046875	2.81734393215531\\
0.99853515625	2.81723756142353\\
0.9990234375	2.81716158250844\\
0.99951171875	2.817115995231\\
};
\addlegendentry{AR(5) Model};

\addplot [color=mycolor2,solid,forget plot]
  table[row sep=crcr]{-1	3.36484424286993\\
-0.99951171875	3.36485921999991\\
-0.9990234375	3.36490415142544\\
-0.99853515625	3.36497903725335\\
-0.998046875	3.36508387766164\\
-0.99755859375	3.36521867289957\\
-0.9970703125	3.36538342328758\\
-0.99658203125	3.36557812921734\\
-0.99609375	3.36580279115175\\
-0.99560546875	3.36605740962491\\
-0.9951171875	3.36634198524216\\
-0.99462890625	3.36665651868008\\
-0.994140625	3.36700101068648\\
-0.99365234375	3.36737546208045\\
-0.9931640625	3.36777987375226\\
-0.99267578125	3.36821424666352\\
-0.9921875	3.36867858184707\\
-0.99169921875	3.36917288040705\\
-0.9912109375	3.36969714351885\\
-0.99072265625	3.37025137242921\\
-0.990234375	3.37083556845615\\
-0.98974609375	3.37144973298901\\
-0.9892578125	3.37209386748846\\
-0.98876953125	3.37276797348654\\
-0.98828125	3.37347205258664\\
-0.98779296875	3.3742061064635\\
-0.9873046875	3.37497013686328\\
-0.98681640625	3.3757641456035\\
-0.986328125	3.37658813457317\\
-0.98583984375	3.37744210573266\\
-0.9853515625	3.37832606111384\\
-0.98486328125	3.37924000282003\\
-0.984375	3.38018393302606\\
-0.98388671875	3.38115785397824\\
-0.9833984375	3.38216176799442\\
-0.98291015625	3.38319567746399\\
-0.982421875	3.38425958484792\\
-0.98193359375	3.38535349267878\\
-0.9814453125	3.38647740356071\\
-0.98095703125	3.3876313201695\\
-0.98046875	3.38881524525266\\
-0.97998046875	3.3900291816293\\
-0.9794921875	3.39127313219027\\
-0.97900390625	3.39254709989812\\
-0.978515625	3.39385108778723\\
-0.97802734375	3.39518509896369\\
-0.9775390625	3.39654913660544\\
-0.97705078125	3.39794320396227\\
-0.9765625	3.39936730435576\\
-0.97607421875	3.40082144117951\\
-0.9755859375	3.40230561789894\\
-0.97509765625	3.40381983805146\\
-0.974609375	3.40536410524649\\
-0.97412109375	3.40693842316546\\
-0.9736328125	3.40854279556182\\
-0.97314453125	3.41017722626113\\
-0.97265625	3.41184171916108\\
-0.97216796875	3.41353627823147\\
-0.9716796875	3.4152609075143\\
-0.97119140625	3.41701561112384\\
-0.970703125	3.41880039324657\\
-0.97021484375	3.42061525814125\\
-0.9697265625	3.422460210139\\
-0.96923828125	3.42433525364332\\
-0.96875	3.42624039313008\\
-0.96826171875	3.42817563314765\\
-0.9677734375	3.43014097831683\\
-0.96728515625	3.43213643333097\\
-0.966796875	3.43416200295601\\
-0.96630859375	3.43621769203046\\
-0.9658203125	3.43830350546551\\
-0.96533203125	3.44041944824506\\
-0.96484375	3.4425655254257\\
-0.96435546875	3.44474174213686\\
-0.9638671875	3.44694810358076\\
-0.96337890625	3.44918461503252\\
-0.962890625	3.45145128184013\\
-0.96240234375	3.45374810942467\\
-0.9619140625	3.45607510328006\\
-0.96142578125	3.45843226897345\\
-0.9609375	3.46081961214495\\
-0.96044921875	3.46323713850802\\
-0.9599609375	3.46568485384914\\
-0.95947265625	3.46816276402818\\
-0.958984375	3.47067087497826\\
-0.95849609375	3.47320919270592\\
-0.9580078125	3.47577772329106\\
-0.95751953125	3.47837647288713\\
-0.95703125	3.48100544772103\\
-0.95654296875	3.4836646540933\\
-0.9560546875	3.4863540983781\\
-0.95556640625	3.48907378702325\\
-0.955078125	3.4918237265504\\
-0.95458984375	3.49460392355495\\
-0.9541015625	3.49741438470616\\
-0.95361328125	3.50025511674725\\
-0.953125	3.50312612649544\\
-0.95263671875	3.50602742084194\\
-0.9521484375	3.50895900675213\\
-0.95166015625	3.5119208912655\\
-0.951171875	3.51491308149582\\
-0.95068359375	3.51793558463112\\
-0.9501953125	3.52098840793382\\
-0.94970703125	3.5240715587407\\
-0.94921875	3.52718504446313\\
-0.94873046875	3.53032887258692\\
-0.9482421875	3.53350305067259\\
-0.94775390625	3.53670758635532\\
-0.947265625	3.53994248734497\\
-0.94677734375	3.54320776142637\\
-0.9462890625	3.54650341645907\\
-0.94580078125	3.54982946037774\\
-0.9453125	3.55318590119196\\
-0.94482421875	3.55657274698648\\
-0.9443359375	3.55999000592125\\
-0.94384765625	3.56343768623136\\
-0.943359375	3.56691579622731\\
-0.94287109375	3.57042434429497\\
-0.9423828125	3.57396333889568\\
-0.94189453125	3.57753278856633\\
-0.94140625	3.58113270191942\\
-0.94091796875	3.58476308764314\\
-0.9404296875	3.5884239545015\\
-0.93994140625	3.59211531133431\\
-0.939453125	3.59583716705731\\
-0.93896484375	3.59958953066232\\
-0.9384765625	3.60337241121715\\
-0.93798828125	3.6071858178659\\
-0.9375	3.61102975982879\\
-0.93701171875	3.61490424640253\\
-0.9365234375	3.61880928696014\\
-0.93603515625	3.62274489095116\\
-0.935546875	3.62671106790171\\
-0.93505859375	3.63070782741467\\
-0.9345703125	3.63473517916959\\
-0.93408203125	3.63879313292288\\
-0.93359375	3.64288169850793\\
-0.93310546875	3.64700088583508\\
-0.9326171875	3.6511507048919\\
-0.93212890625	3.655331165743\\
-0.931640625	3.65954227853042\\
-0.93115234375	3.66378405347351\\
-0.9306640625	3.66805650086915\\
-0.93017578125	3.6723596310917\\
-0.9296875	3.67669345459328\\
-0.92919921875	3.6810579819037\\
-0.9287109375	3.68545322363067\\
-0.92822265625	3.68987919045978\\
-0.927734375	3.69433589315473\\
-0.92724609375	3.6988233425573\\
-0.9267578125	3.70334154958755\\
-0.92626953125	3.70789052524387\\
-0.92578125	3.71247028060307\\
-0.92529296875	3.71708082682053\\
-0.9248046875	3.7217221751302\\
-0.92431640625	3.72639433684485\\
-0.923828125	3.73109732335601\\
-0.92333984375	3.73583114613422\\
-0.9228515625	3.74059581672903\\
-0.92236328125	3.74539134676916\\
-0.921875	3.75021774796258\\
-0.92138671875	3.75507503209665\\
-0.9208984375	3.75996321103815\\
-0.92041015625	3.76488229673348\\
-0.919921875	3.76983230120873\\
-0.91943359375	3.77481323656978\\
-0.9189453125	3.77982511500238\\
-0.91845703125	3.78486794877238\\
-0.91796875	3.78994175022568\\
-0.91748046875	3.79504653178846\\
-0.9169921875	3.80018230596726\\
-0.91650390625	3.80534908534912\\
-0.916015625	3.8105468826016\\
-0.91552734375	3.81577571047299\\
-0.9150390625	3.82103558179243\\
-0.91455078125	3.82632650946993\\
-0.9140625	3.83164850649666\\
-0.91357421875	3.83700158594483\\
-0.9130859375	3.84238576096805\\
-0.91259765625	3.84780104480128\\
-0.912109375	3.85324745076107\\
-0.91162109375	3.85872499224559\\
-0.9111328125	3.86423368273479\\
-0.91064453125	3.86977353579055\\
-0.91015625	3.87534456505676\\
-0.90966796875	3.88094678425944\\
-0.9091796875	3.88658020720699\\
-0.90869140625	3.89224484779007\\
-0.908203125	3.89794071998202\\
-0.90771484375	3.90366783783875\\
-0.9072265625	3.90942621549903\\
-0.90673828125	3.91521586718451\\
-0.90625	3.92103680719992\\
-0.90576171875	3.9268890499332\\
-0.9052734375	3.93277260985556\\
-0.90478515625	3.93868750152168\\
-0.904296875	3.94463373956991\\
-0.90380859375	3.95061133872223\\
-0.9033203125	3.95662031378453\\
-0.90283203125	3.96266067964676\\
-0.90234375	3.96873245128286\\
-0.90185546875	3.97483564375121\\
-0.9013671875	3.98097027219451\\
-0.90087890625	3.9871363518401\\
-0.900390625	3.99333389799993\\
-0.89990234375	3.99956292607086\\
-0.8994140625	4.00582345153475\\
-0.89892578125	4.01211548995855\\
-0.8984375	4.01843905699455\\
-0.89794921875	4.02479416838041\\
-0.8974609375	4.03118083993936\\
-0.89697265625	4.0375990875804\\
-0.896484375	4.04404892729843\\
-0.89599609375	4.05053037517426\\
-0.8955078125	4.05704344737497\\
-0.89501953125	4.06358816015391\\
-0.89453125	4.07016452985094\\
-0.89404296875	4.07677257289256\\
-0.8935546875	4.08341230579201\\
-0.89306640625	4.09008374514952\\
-0.892578125	4.09678690765235\\
-0.89208984375	4.10352181007513\\
-0.8916015625	4.1102884692798\\
-0.89111328125	4.1170869022159\\
-0.890625	4.12391712592074\\
-0.89013671875	4.13077915751947\\
-0.8896484375	4.13767301422537\\
-0.88916015625	4.14459871333983\\
-0.888671875	4.15155627225276\\
-0.88818359375	4.15854570844248\\
-0.8876953125	4.16556703947616\\
-0.88720703125	4.17262028300975\\
-0.88671875	4.1797054567883\\
-0.88623046875	4.18682257864605\\
-0.8857421875	4.19397166650663\\
-0.88525390625	4.20115273838331\\
-0.884765625	4.20836581237897\\
-0.88427734375	4.21561090668646\\
-0.8837890625	4.22288803958874\\
-0.88330078125	4.23019722945893\\
-0.8828125	4.23753849476068\\
-0.88232421875	4.24491185404817\\
-0.8818359375	4.25231732596643\\
-0.88134765625	4.25975492925137\\
-0.880859375	4.26722468273014\\
-0.88037109375	4.27472660532115\\
-0.8798828125	4.28226071603433\\
-0.87939453125	4.2898270339713\\
-0.87890625	4.29742557832557\\
-0.87841796875	4.30505636838267\\
-0.8779296875	4.31271942352043\\
-0.87744140625	4.32041476320906\\
-0.876953125	4.32814240701142\\
-0.87646484375	4.33590237458318\\
-0.8759765625	4.34369468567299\\
-0.87548828125	4.35151936012269\\
-0.875	4.35937641786754\\
-0.87451171875	4.36726587893634\\
-0.8740234375	4.37518776345169\\
-0.87353515625	4.38314209163011\\
-0.873046875	4.39112888378235\\
-0.87255859375	4.39914816031349\\
-0.8720703125	4.40719994172315\\
-0.87158203125	4.41528424860574\\
-0.87109375	4.42340110165063\\
-0.87060546875	4.43155052164234\\
-0.8701171875	4.43973252946081\\
-0.86962890625	4.44794714608145\\
-0.869140625	4.45619439257555\\
-0.86865234375	4.46447429011032\\
-0.8681640625	4.47278685994922\\
-0.86767578125	4.48113212345207\\
-0.8671875	4.48951010207528\\
-0.86669921875	4.49792081737216\\
-0.8662109375	4.50636429099298\\
-0.86572265625	4.51484054468531\\
-0.865234375	4.52334960029417\\
-0.86474609375	4.53189147976223\\
-0.8642578125	4.54046620513013\\
-0.86376953125	4.54907379853654\\
-0.86328125	4.55771428221853\\
-0.86279296875	4.56638767851167\\
-0.8623046875	4.57509400985035\\
-0.86181640625	4.58383329876798\\
-0.861328125	4.59260556789711\\
-0.86083984375	4.60141083996984\\
-0.8603515625	4.61024913781786\\
-0.85986328125	4.61912048437285\\
-0.859375	4.62802490266654\\
-0.85888671875	4.63696241583109\\
-0.8583984375	4.64593304709919\\
-0.85791015625	4.65493681980443\\
-0.857421875	4.6639737573814\\
-0.85693359375	4.67304388336603\\
-0.8564453125	4.68214722139575\\
-0.85595703125	4.6912837952098\\
-0.85546875	4.70045362864938\\
-0.85498046875	4.70965674565802\\
-0.8544921875	4.71889317028163\\
-0.85400390625	4.72816292666893\\
-0.853515625	4.73746603907161\\
-0.85302734375	4.74680253184459\\
-0.8525390625	4.75617242944623\\
-0.85205078125	4.76557575643862\\
-0.8515625	4.77501253748784\\
-0.85107421875	4.78448279736415\\
-0.8505859375	4.79398656094233\\
-0.85009765625	4.8035238532018\\
-0.849609375	4.81309469922707\\
-0.84912109375	4.82269912420781\\
-0.8486328125	4.83233715343913\\
-0.84814453125	4.84200881232206\\
-0.84765625	4.85171412636347\\
-0.84716796875	4.86145312117656\\
-0.8466796875	4.87122582248108\\
-0.84619140625	4.88103225610352\\
-0.845703125	4.89087244797755\\
-0.84521484375	4.90074642414405\\
-0.8447265625	4.91065421075153\\
-0.84423828125	4.92059583405638\\
-0.84375	4.93057132042311\\
-0.84326171875	4.94058069632469\\
-0.8427734375	4.95062398834269\\
-0.84228515625	4.96070122316769\\
-0.841796875	4.97081242759949\\
-0.84130859375	4.98095762854744\\
-0.8408203125	4.99113685303063\\
-0.84033203125	5.00135012817823\\
-0.83984375	5.01159748122978\\
-0.83935546875	5.02187893953548\\
-0.8388671875	5.0321945305564\\
-0.83837890625	5.04254428186487\\
-0.837890625	5.0529282211447\\
-0.83740234375	5.06334637619147\\
-0.8369140625	5.07379877491285\\
-0.83642578125	5.08428544532893\\
-0.8359375	5.09480641557242\\
-0.83544921875	5.10536171388897\\
-0.8349609375	5.11595136863755\\
-0.83447265625	5.12657540829061\\
-0.833984375	5.13723386143455\\
-0.83349609375	5.14792675676989\\
-0.8330078125	5.15865412311157\\
-0.83251953125	5.16941598938936\\
-0.83203125	5.18021238464808\\
-0.83154296875	5.19104333804797\\
-0.8310546875	5.20190887886486\\
-0.83056640625	5.21280903649069\\
-0.830078125	5.22374384043369\\
-0.82958984375	5.23471332031871\\
-0.8291015625	5.24571750588753\\
-0.82861328125	5.25675642699922\\
-0.828125	5.26783011363043\\
-0.82763671875	5.27893859587569\\
-0.8271484375	5.29008190394783\\
-0.82666015625	5.30126006817814\\
-0.826171875	5.31247311901687\\
-0.82568359375	5.3237210870334\\
-0.8251953125	5.33500400291668\\
-0.82470703125	5.34632189747556\\
-0.82421875	5.35767480163902\\
-0.82373046875	5.36906274645661\\
-0.8232421875	5.38048576309874\\
-0.82275390625	5.39194388285707\\
-0.822265625	5.40343713714473\\
-0.82177734375	5.41496555749681\\
-0.8212890625	5.42652917557054\\
-0.82080078125	5.43812802314582\\
-0.8203125	5.44976213212542\\
-0.81982421875	5.46143153453537\\
-0.8193359375	5.47313626252535\\
-0.81884765625	5.48487634836898\\
-0.818359375	5.49665182446426\\
-0.81787109375	5.50846272333379\\
-0.8173828125	5.52030907762527\\
-0.81689453125	5.53219092011178\\
-0.81640625	5.54410828369212\\
-0.81591796875	5.55606120139128\\
-0.8154296875	5.56804970636066\\
-0.81494140625	5.58007383187856\\
-0.814453125	5.59213361135046\\
-0.81396484375	5.60422907830947\\
-0.8134765625	5.61636026641665\\
-0.81298828125	5.62852720946138\\
-0.8125	5.64072994136173\\
-0.81201171875	5.6529684961649\\
-0.8115234375	5.66524290804754\\
-0.81103515625	5.67755321131617\\
-0.810546875	5.6898994404075\\
-0.81005859375	5.7022816298889\\
-0.8095703125	5.7146998144587\\
-0.80908203125	5.72715402894667\\
-0.80859375	5.73964430831436\\
-0.80810546875	5.75217068765546\\
-0.8076171875	5.7647332021963\\
-0.80712890625	5.7773318872961\\
-0.806640625	5.78996677844754\\
-0.80615234375	5.80263791127703\\
-0.8056640625	5.81534532154515\\
-0.80517578125	5.82808904514708\\
-0.8046875	5.84086911811298\\
-0.80419921875	5.8536855766084\\
-0.8037109375	5.86653845693474\\
-0.80322265625	5.87942779552959\\
-0.802734375	5.89235362896721\\
-0.80224609375	5.90531599395889\\
-0.8017578125	5.91831492735342\\
-0.80126953125	5.93135046613754\\
-0.80078125	5.94442264743621\\
-0.80029296875	5.9575315085132\\
-0.7998046875	5.9706770867715\\
-0.79931640625	5.98385941975367\\
-0.798828125	5.99707854514236\\
-0.79833984375	6.01033450076064\\
-0.7978515625	6.02362732457254\\
-0.79736328125	6.03695705468346\\
-0.796875	6.05032372934057\\
-0.79638671875	6.06372738693335\\
-0.7958984375	6.07716806599396\\
-0.79541015625	6.0906458051976\\
-0.794921875	6.10416064336323\\
-0.79443359375	6.11771261945376\\
-0.7939453125	6.13130177257661\\
-0.79345703125	6.14492814198419\\
-0.79296875	6.15859176707435\\
-0.79248046875	6.17229268739076\\
-0.7919921875	6.18603094262355\\
-0.79150390625	6.19980657260958\\
-0.791015625	6.21361961733304\\
-0.79052734375	6.2274701169259\\
-0.7900390625	6.24135811166833\\
-0.78955078125	6.25528364198925\\
-0.7890625	6.26924674846676\\
-0.78857421875	6.28324747182869\\
-0.7880859375	6.29728585295296\\
-0.78759765625	6.31136193286822\\
-0.787109375	6.32547575275417\\
-0.78662109375	6.33962735394224\\
-0.7861328125	6.35381677791595\\
-0.78564453125	6.36804406631143\\
-0.78515625	6.38230926091801\\
-0.78466796875	6.39661240367855\\
-0.7841796875	6.41095353669013\\
-0.78369140625	6.42533270220442\\
-0.783203125	6.43974994262832\\
-0.78271484375	6.45420530052431\\
-0.7822265625	6.46869881861109\\
-0.78173828125	6.48323053976412\\
-0.78125	6.49780050701601\\
-0.78076171875	6.51240876355715\\
-0.7802734375	6.5270553527362\\
-0.77978515625	6.54174031806062\\
-0.779296875	6.55646370319723\\
-0.77880859375	6.5712255519727\\
-0.7783203125	6.58602590837411\\
-0.77783203125	6.60086481654953\\
-0.77734375	6.61574232080849\\
-0.77685546875	6.63065846562254\\
-0.7763671875	6.64561329562583\\
-0.77587890625	6.66060685561572\\
-0.775390625	6.67563919055318\\
-0.77490234375	6.69071034556348\\
-0.7744140625	6.70582036593665\\
-0.77392578125	6.72096929712815\\
-0.7734375	6.73615718475941\\
-0.77294921875	6.75138407461829\\
-0.7724609375	6.7666500126598\\
-0.77197265625	6.78195504500656\\
-0.771484375	6.7972992179495\\
-0.77099609375	6.81268257794828\\
-0.7705078125	6.82810517163206\\
-0.77001953125	6.84356704579989\\
-0.76953125	6.85906824742148\\
-0.76904296875	6.87460882363773\\
-0.7685546875	6.89018882176124\\
-0.76806640625	6.905808289277\\
-0.767578125	6.92146727384301\\
-0.76708984375	6.93716582329083\\
-0.7666015625	6.95290398562624\\
-0.76611328125	6.96868180902977\\
-0.765625	6.98449934185738\\
-0.76513671875	7.00035663264113\\
-0.7646484375	7.01625373008966\\
-0.76416015625	7.03219068308895\\
-0.763671875	7.04816754070287\\
-0.76318359375	7.06418435217384\\
-0.7626953125	7.08024116692347\\
-0.76220703125	7.09633803455317\\
-0.76171875	7.11247500484481\\
-0.76123046875	7.12865212776142\\
-0.7607421875	7.14486945344775\\
-0.76025390625	7.16112703223097\\
-0.759765625	7.17742491462129\\
-0.75927734375	7.19376315131266\\
-0.7587890625	7.21014179318342\\
-0.75830078125	7.22656089129702\\
-0.7578125	7.2430204969025\\
-0.75732421875	7.25952066143544\\
-0.7568359375	7.27606143651843\\
-0.75634765625	7.2926428739618\\
-0.755859375	7.30926502576432\\
-0.75537109375	7.32592794411395\\
-0.7548828125	7.3426316813884\\
-0.75439453125	7.35937629015595\\
-0.75390625	7.37616182317606\\
-0.75341796875	7.39298833340006\\
-0.7529296875	7.40985587397203\\
-0.75244140625	7.42676449822928\\
-0.751953125	7.44371425970318\\
-0.75146484375	7.4607052121199\\
-0.7509765625	7.47773740940105\\
-0.75048828125	7.49481090566451\\
-0.75	7.51192575522509\\
-0.74951171875	7.52908201259523\\
-0.7490234375	7.5462797324858\\
-0.74853515625	7.56351896980688\\
-0.748046875	7.58079977966836\\
-0.74755859375	7.59812221738086\\
-0.7470703125	7.61548633845635\\
-0.74658203125	7.63289219860901\\
-0.74609375	7.65033985375588\\
-0.74560546875	7.66782936001769\\
-0.7451171875	7.68536077371973\\
-0.74462890625	7.70293415139239\\
-0.744140625	7.72054954977216\\
-0.74365234375	7.73820702580224\\
-0.7431640625	7.75590663663351\\
-0.74267578125	7.77364843962514\\
-0.7421875	7.79143249234549\\
-0.74169921875	7.80925885257289\\
-0.7412109375	7.82712757829644\\
-0.74072265625	7.84503872771681\\
-0.740234375	7.86299235924703\\
-0.73974609375	7.8809885315134\\
-0.7392578125	7.89902730335617\\
-0.73876953125	7.91710873383054\\
-0.73828125	7.93523288220735\\
-0.73779296875	7.95339980797389\\
-0.7373046875	7.9716095708349\\
-0.73681640625	7.98986223071332\\
-0.736328125	8.00815784775108\\
-0.73583984375	8.02649648231007\\
-0.7353515625	8.04487819497296\\
-0.73486328125	8.06330304654397\\
-0.734375	8.08177109804988\\
-0.73388671875	8.10028241074084\\
-0.7333984375	8.11883704609119\\
-0.73291015625	8.13743506580047\\
-0.732421875	8.15607653179419\\
-0.73193359375	8.17476150622476\\
-0.7314453125	8.19349005147243\\
-0.73095703125	8.21226223014615\\
-0.73046875	8.23107810508443\\
-0.72998046875	8.24993773935636\\
-0.7294921875	8.26884119626247\\
-0.72900390625	8.28778853933566\\
-0.728515625	8.30677983234204\\
-0.72802734375	8.32581513928205\\
-0.7275390625	8.34489452439121\\
-0.72705078125	8.36401805214124\\
-0.7265625	8.38318578724078\\
-0.72607421875	8.40239779463659\\
-0.7255859375	8.42165413951434\\
-0.72509765625	8.44095488729965\\
-0.724609375	8.46030010365901\\
-0.72412109375	8.4796898545008\\
-0.7236328125	8.49912420597627\\
-0.72314453125	8.51860322448045\\
-0.72265625	8.5381269766533\\
-0.72216796875	8.55769552938051\\
-0.7216796875	8.57730894979465\\
-0.72119140625	8.59696730527611\\
-0.720703125	8.61667066345411\\
-0.72021484375	8.63641909220783\\
-0.7197265625	8.65621265966726\\
-0.71923828125	8.67605143421434\\
-0.71875	8.69593548448399\\
-0.71826171875	8.71586487936514\\
-0.7177734375	8.73583968800174\\
-0.71728515625	8.75585997979391\\
-0.716796875	8.7759258243989\\
-0.71630859375	8.7960372917322\\
-0.7158203125	8.81619445196867\\
-0.71533203125	8.83639737554348\\
-0.71484375	8.85664613315333\\
-0.71435546875	8.87694079575749\\
-0.7138671875	8.89728143457886\\
-0.71337890625	8.91766812110515\\
-0.712890625	8.93810092708994\\
-0.71240234375	8.9585799245538\\
-0.7119140625	8.9791051857854\\
-0.71142578125	8.99967678334271\\
-0.7109375	9.02029479005404\\
-0.71044921875	9.04095927901925\\
-0.7099609375	9.0616703236109\\
-0.70947265625	9.08242799747533\\
-0.708984375	9.10323237453394\\
-0.70849609375	9.12408352898424\\
-0.7080078125	9.1449815353011\\
-0.70751953125	9.16592646823795\\
-0.70703125	9.18691840282782\\
-0.70654296875	9.20795741438483\\
-0.7060546875	9.22904357850506\\
-0.70556640625	9.25017697106797\\
-0.705078125	9.27135766823758\\
-0.70458984375	9.29258574646365\\
-0.7041015625	9.31386128248294\\
-0.70361328125	9.33518435332045\\
-0.703125	9.35655503629067\\
-0.70263671875	9.37797340899883\\
-0.7021484375	9.39943954934214\\
-0.70166015625	9.42095353551106\\
-0.701171875	9.44251544599063\\
-0.70068359375	9.46412535956173\\
-0.7001953125	9.48578335530229\\
-0.69970703125	9.50748951258869\\
-0.69921875	9.52924391109702\\
-0.69873046875	9.55104663080446\\
-0.6982421875	9.57289775199044\\
-0.69775390625	9.59479735523814\\
-0.697265625	9.61674552143574\\
-0.69677734375	9.6387423317778\\
-0.6962890625	9.6607878677666\\
-0.69580078125	9.6828822112135\\
-0.6953125	9.70502544424027\\
-0.69482421875	9.72721764928053\\
-0.6943359375	9.74945890908117\\
-0.69384765625	9.77174930670356\\
-0.693359375	9.79408892552517\\
-0.69287109375	9.81647784924091\\
-0.6923828125	9.83891616186442\\
-0.69189453125	9.86140394772972\\
-0.69140625	9.88394129149241\\
-0.69091796875	9.90652827813134\\
-0.6904296875	9.92916499294989\\
-0.68994140625	9.95185152157754\\
-0.689453125	9.97458794997128\\
-0.68896484375	9.99737436441714\\
-0.6884765625	10.0202108515315\\
-0.68798828125	10.043097498263\\
-0.6875	10.0660343918936\\
-0.68701171875	10.0890216200403\\
-0.6865234375	10.1120592706566\\
-0.68603515625	10.1351474320342\\
-0.685546875	10.1582861928043\\
-0.68505859375	10.1814756419394\\
-0.6845703125	10.2047158687546\\
-0.68408203125	10.2280069629094\\
-0.68359375	10.2513490144091\\
-0.68310546875	10.2747421136065\\
-0.6826171875	10.2981863512037\\
-0.68212890625	10.3216818182532\\
-0.681640625	10.3452286061598\\
-0.68115234375	10.3688268066825\\
-0.6806640625	10.3924765119357\\
-0.68017578125	10.4161778143911\\
-0.6796875	10.4399308068792\\
-0.67919921875	10.4637355825912\\
-0.6787109375	10.4875922350804\\
-0.67822265625	10.5115008582641\\
-0.677734375	10.5354615464253\\
-0.67724609375	10.5594743942144\\
-0.6767578125	10.5835394966506\\
-0.67626953125	10.6076569491241\\
-0.67578125	10.6318268473977\\
-0.67529296875	10.6560492876085\\
-0.6748046875	10.6803243662696\\
-0.67431640625	10.7046521802718\\
-0.673828125	10.729032826886\\
-0.67333984375	10.753466403764\\
-0.6728515625	10.7779530089412\\
-0.67236328125	10.8024927408381\\
-0.671875	10.8270856982618\\
-0.67138671875	10.8517319804085\\
-0.6708984375	10.8764316868649\\
-0.67041015625	10.9011849176099\\
-0.669921875	10.9259917730172\\
-0.66943359375	10.9508523538566\\
-0.6689453125	10.9757667612958\\
-0.66845703125	11.0007350969029\\
-0.66796875	11.025757462648\\
-0.66748046875	11.0508339609048\\
-0.6669921875	11.0759646944532\\
-0.66650390625	11.101149766481\\
-0.666015625	11.1263892805857\\
-0.66552734375	11.1516833407765\\
-0.6650390625	11.1770320514767\\
-0.66455078125	11.2024355175253\\
-0.6640625	11.2278938441793\\
-0.66357421875	11.2534071371154\\
-0.6630859375	11.2789755024324\\
-0.66259765625	11.3045990466533\\
-0.662109375	11.330277876727\\
-0.66162109375	11.3560121000306\\
-0.6611328125	11.3818018243718\\
-0.66064453125	11.4076471579904\\
-0.66015625	11.4335482095611\\
-0.65966796875	11.4595050881951\\
-0.6591796875	11.4855179034426\\
-0.65869140625	11.5115867652949\\
-0.658203125	11.5377117841864\\
-0.65771484375	11.5638930709973\\
-0.6572265625	11.5901307370553\\
-0.65673828125	11.6164248941378\\
-0.65625	11.6427756544749\\
-0.65576171875	11.6691831307506\\
-0.6552734375	11.695647436106\\
-0.65478515625	11.7221686841411\\
-0.654296875	11.7487469889171\\
-0.65380859375	11.775382464959\\
-0.6533203125	11.8020752272578\\
-0.65283203125	11.8288253912728\\
-0.65234375	11.8556330729339\\
-0.65185546875	11.8824983886441\\
-0.6513671875	11.9094214552822\\
-0.65087890625	11.9364023902046\\
-0.650390625	11.9634413112481\\
-0.64990234375	11.9905383367325\\
-0.6494140625	12.0176935854625\\
-0.64892578125	12.0449071767309\\
-0.6484375	12.0721792303205\\
-0.64794921875	12.0995098665071\\
-0.6474609375	12.1268992060616\\
-0.64697265625	12.1543473702528\\
-0.646484375	12.1818544808498\\
-0.64599609375	12.209420660125\\
-0.6455078125	12.237046030856\\
-0.64501953125	12.2647307163289\\
-0.64453125	12.2924748403405\\
-0.64404296875	12.3202785272012\\
-0.6435546875	12.3481419017375\\
-0.64306640625	12.3760650892947\\
-0.642578125	12.4040482157398\\
-0.64208984375	12.4320914074641\\
-0.6416015625	12.4601947913859\\
-0.64111328125	12.4883584949531\\
-0.640625	12.5165826461466\\
-0.64013671875	12.5448673734823\\
-0.6396484375	12.5732128060145\\
-0.63916015625	12.6016190733384\\
-0.638671875	12.6300863055935\\
-0.63818359375	12.6586146334655\\
-0.6376953125	12.6872041881902\\
-0.63720703125	12.7158551015558\\
-0.63671875	12.7445675059062\\
-0.63623046875	12.7733415341436\\
-0.6357421875	12.8021773197318\\
-0.63525390625	12.8310749966991\\
-0.634765625	12.8600346996411\\
-0.63427734375	12.8890565637243\\
-0.6337890625	12.9181407246884\\
-0.63330078125	12.9472873188499\\
-0.6328125	12.9764964831052\\
-0.63232421875	13.0057683549335\\
-0.6318359375	13.0351030724001\\
-0.63134765625	13.0645007741596\\
-0.630859375	13.0939615994587\\
-0.63037109375	13.1234856881402\\
-0.6298828125	13.1530731806454\\
-0.62939453125	13.1827242180179\\
-0.62890625	13.2124389419067\\
-0.62841796875	13.2422174945694\\
-0.6279296875	13.2720600188759\\
-0.62744140625	13.3019666583112\\
-0.626953125	13.3319375569792\\
-0.62646484375	13.361972859606\\
-0.6259765625	13.3920727115434\\
-0.62548828125	13.422237258772\\
-0.625	13.4524666479051\\
-0.62451171875	13.482761026192\\
-0.6240234375	13.5131205415214\\
-0.62353515625	13.5435453424253\\
-0.623046875	13.5740355780822\\
-0.62255859375	13.604591398321\\
-0.6220703125	13.6352129536243\\
-0.62158203125	13.6659003951323\\
-0.62109375	13.6966538746466\\
-0.62060546875	13.7274735446335\\
-0.6201171875	13.7583595582279\\
-0.61962890625	13.7893120692373\\
-0.619140625	13.8203312321452\\
-0.61865234375	13.8514172021153\\
-0.6181640625	13.8825701349949\\
-0.61767578125	13.9137901873192\\
-0.6171875	13.9450775163148\\
-0.61669921875	13.976432279904\\
-0.6162109375	14.0078546367085\\
-0.61572265625	14.0393447460535\\
-0.615234375	14.0709027679714\\
-0.61474609375	14.1025288632065\\
-0.6142578125	14.1342231932184\\
-0.61376953125	14.1659859201865\\
-0.61328125	14.1978172070138\\
-0.61279296875	14.2297172173315\\
-0.6123046875	14.2616861155028\\
-0.61181640625	14.2937240666273\\
-0.611328125	14.3258312365452\\
-0.61083984375	14.3580077918416\\
-0.6103515625	14.3902538998508\\
-0.60986328125	14.4225697286607\\
-0.609375	14.4549554471172\\
-0.60888671875	14.4874112248284\\
-0.6083984375	14.5199372321694\\
-0.60791015625	14.5525336402862\\
-0.607421875	14.5852006211011\\
-0.60693359375	14.6179383473163\\
-0.6064453125	14.6507469924193\\
-0.60595703125	14.6836267306867\\
-0.60546875	14.7165777371896\\
-0.60498046875	14.7496001877979\\
-0.6044921875	14.782694259185\\
-0.60400390625	14.8158601288326\\
-0.603515625	14.8490979750358\\
-0.60302734375	14.8824079769072\\
-0.6025390625	14.9157903143827\\
-0.60205078125	14.9492451682256\\
-0.6015625	14.9827727200319\\
-0.60107421875	15.0163731522353\\
-0.6005859375	15.0500466481123\\
-0.60009765625	15.0837933917867\\
-0.599609375	15.1176135682355\\
-0.59912109375	15.1515073632934\\
-0.5986328125	15.1854749636581\\
-0.59814453125	15.219516556896\\
-0.59765625	15.2536323314466\\
-0.59716796875	15.2878224766285\\
-0.5966796875	15.3220871826445\\
-0.59619140625	15.3564266405866\\
-0.595703125	15.3908410424423\\
-0.59521484375	15.4253305810991\\
-0.5947265625	15.4598954503505\\
-0.59423828125	15.4945358449016\\
-0.59375	15.5292519603744\\
-0.59326171875	15.5640439933136\\
-0.5927734375	15.5989121411923\\
-0.59228515625	15.6338566024176\\
-0.591796875	15.6688775763364\\
-0.59130859375	15.7039752632413\\
-0.5908203125	15.7391498643764\\
-0.59033203125	15.7744015819431\\
-0.58984375	15.8097306191062\\
-0.58935546875	15.8451371799997\\
-0.5888671875	15.8806214697331\\
-0.58837890625	15.9161836943974\\
-0.587890625	15.951824061071\\
-0.58740234375	15.9875427778261\\
-0.5869140625	16.0233400537351\\
-0.58642578125	16.0592160988765\\
-0.5859375	16.0951711243416\\
-0.58544921875	16.1312053422405\\
-0.5849609375	16.1673189657089\\
-0.58447265625	16.2035122089145\\
-0.583984375	16.2397852870634\\
-0.58349609375	16.2761384164067\\
-0.5830078125	16.3125718142474\\
-0.58251953125	16.3490856989468\\
-0.58203125	16.3856802899314\\
-0.58154296875	16.4223558076999\\
-0.5810546875	16.4591124738295\\
-0.58056640625	16.4959505109835\\
-0.580078125	16.5328701429178\\
-0.57958984375	16.5698715944881\\
-0.5791015625	16.6069550916572\\
-0.57861328125	16.6441208615016\\
-0.578125	16.6813691322194\\
-0.57763671875	16.7187001331369\\
-0.5771484375	16.7561140947165\\
-0.57666015625	16.7936112485638\\
-0.576171875	16.831191827435\\
-0.57568359375	16.8688560652446\\
-0.5751953125	16.9066041970728\\
-0.57470703125	16.9444364591733\\
-0.57421875	16.9823530889809\\
-0.57373046875	17.0203543251193\\
-0.5732421875	17.0584404074088\\
-0.57275390625	17.0966115768746\\
-0.572265625	17.1348680757541\\
-0.57177734375	17.1732101475056\\
-0.5712890625	17.2116380368159\\
-0.57080078125	17.2501519896089\\
-0.5703125	17.2887522530534\\
-0.56982421875	17.3274390755716\\
-0.5693359375	17.3662127068476\\
-0.56884765625	17.4050733978357\\
-0.568359375	17.4440214007689\\
-0.56787109375	17.4830569691674\\
-0.5673828125	17.5221803578476\\
-0.56689453125	17.5613918229306\\
-0.56640625	17.6006916218508\\
-0.56591796875	17.6400800133653\\
-0.5654296875	17.6795572575624\\
-0.56494140625	17.7191236158708\\
-0.564453125	17.758779351069\\
-0.56396484375	17.7985247272938\\
-0.5634765625	17.8383600100503\\
-0.56298828125	17.8782854662209\\
-0.5625	17.9183013640746\\
-0.56201171875	17.9584079732772\\
-0.5615234375	17.9986055648998\\
-0.56103515625	18.0388944114294\\
-0.560546875	18.0792747867784\\
-0.56005859375	18.1197469662943\\
-0.5595703125	18.1603112267697\\
-0.55908203125	18.2009678464525\\
-0.55859375	18.2417171050558\\
-0.55810546875	18.2825592837682\\
-0.5576171875	18.3234946652643\\
-0.55712890625	18.3645235337144\\
-0.556640625	18.4056461747958\\
-0.55615234375	18.446862875703\\
-0.5556640625	18.4881739251582\\
-0.55517578125	18.5295796134224\\
-0.5546875	18.5710802323061\\
-0.55419921875	18.6126760751799\\
-0.5537109375	18.6543674369862\\
-0.55322265625	18.69615461425\\
-0.552734375	18.73803790509\\
-0.55224609375	18.7800176092304\\
-0.5517578125	18.8220940280117\\
-0.55126953125	18.8642674644029\\
-0.55078125	18.9065382230127\\
-0.55029296875	18.9489066101015\\
-0.5498046875	18.9913729335933\\
-0.54931640625	19.0339375030872\\
-0.548828125	19.0766006298704\\
-0.54833984375	19.1193626269296\\
-0.5478515625	19.1622238089633\\
-0.54736328125	19.2051844923948\\
-0.546875	19.2482449953843\\
-0.54638671875	19.2914056378418\\
-0.5458984375	19.3346667414394\\
-0.54541015625	19.3780286296248\\
-0.544921875	19.4214916276337\\
-0.54443359375	19.4650560625035\\
-0.5439453125	19.5087222630861\\
-0.54345703125	19.5524905600614\\
-0.54296875	19.5963612859507\\
-0.54248046875	19.6403347751308\\
-0.5419921875	19.6844113638468\\
-0.54150390625	19.728591390227\\
-0.541015625	19.772875194296\\
-0.54052734375	19.8172631179895\\
-0.5400390625	19.8617555051682\\
-0.53955078125	19.9063527016323\\
-0.5390625	19.951055055136\\
-0.53857421875	19.9958629154023\\
-0.5380859375	20.0407766341374\\
-0.53759765625	20.0857965650462\\
-0.537109375	20.1309230638471\\
-0.53662109375	20.176156488287\\
-0.5361328125	20.2214971981572\\
-0.53564453125	20.2669455553083\\
-0.53515625	20.3125019236667\\
-0.53466796875	20.3581666692496\\
-0.5341796875	20.4039401601811\\
-0.53369140625	20.449822766709\\
-0.533203125	20.4958148612203\\
-0.53271484375	20.5419168182579\\
-0.5322265625	20.5881290145374\\
-0.53173828125	20.6344518289636\\
-0.53125	20.6808856426474\\
-0.53076171875	20.727430838923\\
-0.5302734375	20.7740878033654\\
-0.52978515625	20.8208569238072\\
-0.529296875	20.8677385903567\\
-0.52880859375	20.9147331954153\\
-0.5283203125	20.9618411336959\\
-0.52783203125	21.0090628022405\\
-0.52734375	21.0563986004387\\
-0.52685546875	21.1038489300463\\
-0.5263671875	21.1514141952036\\
-0.52587890625	21.1990948024547\\
-0.525390625	21.2468911607661\\
-0.52490234375	21.294803681546\\
-0.5244140625	21.3428327786641\\
-0.52392578125	21.3909788684703\\
-0.5234375	21.4392423698156\\
-0.52294921875	21.4876237040711\\
-0.5224609375	21.5361232951489\\
-0.52197265625	21.5847415695219\\
-0.521484375	21.6334789562449\\
-0.52099609375	21.6823358869751\\
-0.5205078125	21.7313127959932\\
-0.52001953125	21.7804101202246\\
-0.51953125	21.8296282992608\\
-0.51904296875	21.878967775381\\
-0.5185546875	21.9284289935743\\
-0.51806640625	21.9780124015613\\
-0.517578125	22.0277184498168\\
-0.51708984375	22.0775475915922\\
-0.5166015625	22.1275002829383\\
-0.51611328125	22.1775769827283\\
-0.515625	22.227778152681\\
-0.51513671875	22.2781042573845\\
-0.5146484375	22.3285557643195\\
-0.51416015625	22.3791331438837\\
-0.513671875	22.429836869416\\
-0.51318359375	22.4806674172205\\
-0.5126953125	22.531625266592\\
-0.51220703125	22.5827108998401\\
-0.51171875	22.6339248023153\\
-0.51123046875	22.6852674624338\\
-0.5107421875	22.7367393717038\\
-0.51025390625	22.7883410247511\\
-0.509765625	22.840072919346\\
-0.50927734375	22.8919355564291\\
-0.5087890625	22.943929440139\\
-0.50830078125	22.996055077839\\
-0.5078125	23.0483129801444\\
-0.50732421875	23.1007036609508\\
-0.5068359375	23.1532276374616\\
-0.50634765625	23.2058854302168\\
-0.505859375	23.2586775631209\\
-0.50537109375	23.3116045634728\\
-0.5048828125	23.3646669619943\\
-0.50439453125	23.4178652928601\\
-0.50390625	23.4712000937274\\
-0.50341796875	23.5246719057663\\
-0.5029296875	23.5782812736901\\
-0.50244140625	23.6320287457865\\
-0.501953125	23.6859148739484\\
-0.50146484375	23.739940213706\\
-0.5009765625	23.7941053242579\\
-0.50048828125	23.8484107685039\\
-0.5	23.9028571130775\\
-0.49951171875	23.9574449283785\\
-0.4990234375	24.0121747886068\\
-0.49853515625	24.0670472717955\\
-0.498046875	24.1220629598453\\
-0.49755859375	24.1772224385589\\
-0.4970703125	24.2325262976756\\
-0.49658203125	24.2879751309065\\
-0.49609375	24.3435695359701\\
-0.49560546875	24.3993101146283\\
-0.4951171875	24.4551974727225\\
-0.49462890625	24.5112322202108\\
-0.494140625	24.5674149712045\\
-0.49365234375	24.6237463440065\\
-0.4931640625	24.6802269611486\\
-0.49267578125	24.73685744943\\
-0.4921875	24.7936384399568\\
-0.49169921875	24.8505705681803\\
-0.4912109375	24.9076544739377\\
-0.49072265625	24.9648908014915\\
-0.490234375	25.0222801995706\\
-0.48974609375	25.0798233214112\\
-0.4892578125	25.1375208247986\\
-0.48876953125	25.1953733721087\\
-0.48828125	25.2533816303514\\
-0.48779296875	25.3115462712126\\
-0.4873046875	25.3698679710986\\
-0.48681640625	25.4283474111791\\
-0.486328125	25.4869852774329\\
-0.48583984375	25.5457822606918\\
-0.4853515625	25.604739056687\\
-0.48486328125	25.6638563660944\\
-0.484375	25.7231348945818\\
-0.48388671875	25.7825753528556\\
-0.4833984375	25.8421784567089\\
-0.48291015625	25.9019449270692\\
-0.482421875	25.9618754900477\\
-0.48193359375	26.0219708769884\\
-0.4814453125	26.0822318245179\\
-0.48095703125	26.1426590745963\\
-0.48046875	26.2032533745681\\
-0.47998046875	26.2640154772139\\
-0.4794921875	26.3249461408026\\
-0.47900390625	26.3860461291449\\
-0.478515625	26.4473162116463\\
-0.47802734375	26.5087571633612\\
-0.4775390625	26.5703697650485\\
-0.47705078125	26.6321548032264\\
-0.4765625	26.6941130702289\\
-0.47607421875	26.7562453642625\\
-0.4755859375	26.8185524894635\\
-0.47509765625	26.881035255957\\
-0.474609375	26.9436944799147\\
-0.47412109375	27.0065309836154\\
-0.4736328125	27.0695455955048\\
-0.47314453125	27.1327391502568\\
-0.47265625	27.1961124888352\\
-0.47216796875	27.259666458556\\
-0.4716796875	27.3234019131514\\
-0.47119140625	27.3873197128329\\
-0.470703125	27.451420724357\\
-0.47021484375	27.5157058210901\\
-0.4697265625	27.5801758830756\\
-0.46923828125	27.6448317971006\\
-0.46875	27.7096744567644\\
-0.46826171875	27.7747047625467\\
-0.4677734375	27.8399236218783\\
-0.46728515625	27.905331949211\\
-0.466796875	27.9709306660892\\
-0.46630859375	28.0367207012224\\
-0.4658203125	28.1027029905584\\
-0.46533203125	28.1688784773572\\
-0.46484375	28.2352481122665\\
-0.46435546875	28.3018128533972\\
-0.4638671875	28.368573666401\\
-0.46337890625	28.4355315245476\\
-0.462890625	28.502687408804\\
-0.46240234375	28.5700423079139\\
-0.4619140625	28.6375972184795\\
-0.46142578125	28.7053531450422\\
-0.4609375	28.7733111001664\\
-0.46044921875	28.8414721045229\\
-0.4599609375	28.9098371869739\\
-0.45947265625	28.9784073846597\\
-0.458984375	29.047183743085\\
-0.45849609375	29.1161673162075\\
-0.4580078125	29.1853591665275\\
-0.45751953125	29.2547603651779\\
-0.45703125	29.3243719920163\\
-0.45654296875	29.3941951357176\\
-0.4560546875	29.4642308938679\\
-0.45556640625	29.5344803730597\\
-0.455078125	29.6049446889887\\
-0.45458984375	29.6756249665507\\
-0.4541015625	29.7465223399408\\
-0.45361328125	29.817637952754\\
-0.453125	29.8889729580858\\
-0.45263671875	29.9605285186355\\
-0.4521484375	30.0323058068098\\
-0.45166015625	30.1043060048288\\
-0.451171875	30.1765303048318\\
-0.45068359375	30.2489799089864\\
-0.4501953125	30.3216560295972\\
-0.44970703125	30.394559889217\\
-0.44921875	30.4676927207591\\
-0.44873046875	30.5410557676115\\
-0.4482421875	30.6146502837512\\
-0.44775390625	30.6884775338618\\
-0.447265625	30.7625387934514\\
-0.44677734375	30.8368353489723\\
-0.4462890625	30.9113684979428\\
-0.44580078125	30.9861395490695\\
-0.4453125	31.0611498223722\\
-0.44482421875	31.13640064931\\
-0.4443359375	31.2118933729088\\
-0.44384765625	31.2876293478911\\
-0.443359375	31.3636099408069\\
-0.44287109375	31.4398365301664\\
-0.4423828125	31.5163105065751\\
-0.44189453125	31.5930332728691\\
-0.44140625	31.6700062442545\\
-0.44091796875	31.7472308484461\\
-0.4404296875	31.8247085258095\\
-0.43994140625	31.9024407295049\\
-0.439453125	31.9804289256325\\
-0.43896484375	32.0586745933793\\
-0.4384765625	32.1371792251687\\
-0.43798828125	32.215944326812\\
-0.4375	32.2949714176607\\
-0.43701171875	32.3742620307626\\
-0.4365234375	32.4538177130184\\
-0.43603515625	32.5336400253407\\
-0.435546875	32.6137305428158\\
-0.43505859375	32.6940908548671\\
-0.4345703125	32.77472256542\\
-0.43408203125	32.8556272930702\\
-0.43359375	32.936806671253\\
-0.43310546875	33.0182623484157\\
-0.4326171875	33.0999959881917\\
-0.43212890625	33.1820092695769\\
-0.431640625	33.2643038871086\\
-0.43115234375	33.3468815510468\\
-0.4306640625	33.429743987557\\
-0.43017578125	33.5128929388967\\
-0.4296875	33.596330163603\\
-0.42919921875	33.6800574366832\\
-0.4287109375	33.7640765498085\\
-0.42822265625	33.8483893115086\\
-0.427734375	33.9329975473699\\
-0.42724609375	34.0179031002362\\
-0.4267578125	34.1031078304115\\
-0.42626953125	34.1886136158657\\
-0.42578125	34.2744223524426\\
-0.42529296875	34.3605359540707\\
-0.4248046875	34.4469563529765\\
-0.42431640625	34.5336854999009\\
-0.423828125	34.6207253643176\\
-0.42333984375	34.7080779346544\\
-0.4228515625	34.7957452185181\\
-0.42236328125	34.8837292429204\\
-0.421875	34.9720320545086\\
-0.42138671875	35.060655719797\\
-0.4208984375	35.1496023254032\\
-0.42041015625	35.2388739782855\\
-0.419921875	35.3284728059839\\
-0.41943359375	35.4184009568645\\
-0.4189453125	35.5086606003652\\
-0.41845703125	35.5992539272461\\
-0.41796875	35.6901831498414\\
-0.41748046875	35.7814505023151\\
-0.4169921875	35.8730582409185\\
-0.41650390625	35.9650086442528\\
-0.416015625	36.0573040135319\\
-0.41552734375	36.1499466728498\\
-0.4150390625	36.242938969451\\
-0.41455078125	36.3362832740027\\
-0.4140625	36.4299819808711\\
-0.41357421875	36.5240375084002\\
-0.4130859375	36.6184522991934\\
-0.41259765625	36.7132288203972\\
-0.412109375	36.8083695639901\\
-0.41162109375	36.9038770470719\\
-0.4111328125	36.9997538121573\\
-0.41064453125	37.0960024274712\\
-0.41015625	37.1926254872487\\
-0.40966796875	37.2896256120353\\
-0.4091796875	37.3870054489919\\
-0.40869140625	37.4847676722016\\
-0.408203125	37.582914982979\\
-0.40771484375	37.6814501101827\\
-0.4072265625	37.7803758105297\\
-0.40673828125	37.879694868912\\
-0.40625	37.9794100987172\\
-0.40576171875	38.079524342149\\
-0.4052734375	38.1800404705523\\
-0.40478515625	38.2809613847382\\
-0.404296875	38.3822900153131\\
-0.40380859375	38.4840293230076\\
-0.4033203125	38.5861822990097\\
-0.40283203125	38.6887519652968\\
-0.40234375	38.7917413749714\\
-0.40185546875	38.8951536125976\\
-0.4013671875	38.9989917945381\\
-0.40087890625	39.1032590692934\\
-0.400390625	39.2079586178401\\
-0.39990234375	39.3130936539724\\
-0.3994140625	39.4186674246412\\
-0.39892578125	39.5246832102956\\
-0.3984375	39.6311443252232\\
-0.39794921875	39.7380541178903\\
-0.3974609375	39.8454159712818\\
-0.39697265625	39.9532333032397\\
-0.396484375	40.0615095668005\\
-0.39599609375	40.1702482505317\\
-0.3955078125	40.2794528788653\\
-0.39501953125	40.3891270124297\\
-0.39453125	40.4992742483789\\
-0.39404296875	40.6098982207173\\
-0.3935546875	40.7210026006227\\
-0.39306640625	40.8325910967638\\
-0.392578125	40.9446674556137\\
-0.39208984375	41.0572354617573\\
-0.3916015625	41.170298938194\\
-0.39111328125	41.2838617466323\\
-0.390625	41.3979277877791\\
-0.39013671875	41.5125010016197\\
-0.3896484375	41.6275853676895\\
-0.38916015625	41.7431849053362\\
-0.388671875	41.8593036739716\\
-0.38818359375	41.9759457733125\\
-0.3876953125	42.0931153436084\\
-0.38720703125	42.210816565857\\
-0.38671875	42.3290536620045\\
-0.38623046875	42.4478308951307\\
-0.3857421875	42.5671525696176\\
-0.38525390625	42.6870230312988\\
-0.384765625	42.8074466675909\\
-0.38427734375	42.928427907603\\
-0.3837890625	43.0499712222238\\
-0.38330078125	43.1720811241849\\
-0.3828125	43.2947621680983\\
-0.38232421875	43.4180189504652\\
-0.3818359375	43.5418561096565\\
-0.38134765625	43.6662783258606\\
-0.380859375	43.7912903209978\\
-0.38037109375	43.9168968585976\\
-0.3798828125	44.0431027436387\\
-0.37939453125	44.1699128223462\\
-0.37890625	44.2973319819454\\
-0.37841796875	44.4253651503689\\
-0.3779296875	44.5540172959135\\
-0.37744140625	44.6832934268426\\
-0.376953125	44.8131985909336\\
-0.37646484375	44.9437378749639\\
-0.3759765625	45.0749164041318\\
-0.37548828125	45.2067393414113\\
-0.375	45.3392118868308\\
-0.37451171875	45.4723392766778\\
-0.3740234375	45.6061267826186\\
-0.37353515625	45.7405797107328\\
-0.373046875	45.8757034004529\\
-0.37255859375	46.0115032234072\\
-0.3720703125	46.1479845821571\\
-0.37158203125	46.2851529088225\\
-0.37109375	46.4230136635913\\
-0.37060546875	46.5615723331018\\
-0.3701171875	46.7008344286931\\
-0.36962890625	46.8408054845142\\
-0.369140625	46.9814910554843\\
-0.36865234375	47.1228967150941\\
-0.3681640625	47.2650280530389\\
-0.36767578125	47.4078906726741\\
-0.3671875	47.551490188281\\
-0.36669921875	47.6958322221325\\
-0.3662109375	47.8409224013478\\
-0.36572265625	47.9867663545199\\
-0.365234375	48.1333697081072\\
-0.36474609375	48.280738082571\\
-0.3642578125	48.4288770882461\\
-0.36376953125	48.5777923209276\\
-0.36328125	48.7274893571596\\
-0.36279296875	48.8779737492028\\
-0.3623046875	49.0292510196686\\
-0.36181640625	49.1813266557956\\
-0.361328125	49.3342061033493\\
-0.36083984375	49.4878947601218\\
-0.3603515625	49.6423979690108\\
-0.35986328125	49.7977210106492\\
-0.359375	49.9538690955634\\
-0.35888671875	50.1108473558298\\
-0.3583984375	50.2686608362021\\
-0.35791015625	50.4273144846782\\
-0.357421875	50.5868131424756\\
-0.35693359375	50.747161533379\\
-0.3564453125	50.9083642524259\\
-0.35595703125	51.0704257538933\\
-0.35546875	51.2333503385428\\
-0.35498046875	51.3971421400849\\
-0.3544921875	51.5618051108164\\
-0.35400390625	51.7273430063867\\
-0.353515625	51.8937593696412\\
-0.35302734375	52.0610575134932\\
-0.3525390625	52.2292405027683\\
-0.35205078125	52.3983111349672\\
-0.3515625	52.5682719198843\\
-0.35107421875	52.7391250580212\\
-0.3505859375	52.9108724177333\\
-0.35009765625	53.0835155110338\\
-0.349609375	53.257055467993\\
-0.34912109375	53.4314930096495\\
-0.3486328125	53.6068284193622\\
-0.34814453125	53.7830615125211\\
-0.34765625	53.960191604528\\
-0.34716796875	54.1382174769673\\
-0.3466796875	54.3171373418703\\
-0.34619140625	54.4969488039817\\
-0.345703125	54.6776488209334\\
-0.34521484375	54.8592336612203\\
-0.3447265625	55.0416988598833\\
-0.34423828125	55.2250391717874\\
-0.34375	55.409248522393\\
-0.34326171875	55.5943199559087\\
-0.3427734375	55.7802455807148\\
-0.34228515625	55.9670165119482\\
-0.341796875	56.1546228111357\\
-0.34130859375	56.3430534227616\\
-0.3408203125	56.5322961076671\\
-0.34033203125	56.7223373731636\\
-0.33984375	56.9131623997633\\
-0.33935546875	57.1047549644238\\
-0.3388671875	57.2970973602113\\
-0.33837890625	57.4901703123012\\
-0.337890625	57.6839528902355\\
-0.33740234375	57.878422416373\\
-0.3369140625	58.0735543704857\\
-0.33642578125	58.2693222904634\\
-0.3359375	58.4656976691176\\
-0.33544921875	58.6626498470926\\
-0.3349609375	58.8601459019223\\
-0.33447265625	59.0581505332996\\
-0.333984375	59.2566259446662\\
-0.33349609375	59.455531721264\\
-0.3330078125	59.6548247048398\\
-0.33251953125	59.8544588652449\\
-0.33203125	60.0543851692329\\
-0.33154296875	60.2545514468131\\
-0.3310546875	60.4549022555973\\
-0.33056640625	60.6553787436578\\
-0.330078125	60.8559185114873\\
-0.32958984375	61.0564554737611\\
-0.3291015625	61.2569197216882\\
-0.32861328125	61.4572373868663\\
-0.328125	61.6573305076477\\
-0.32763671875	61.8571168991785\\
-0.3271484375	62.056510028374\\
-0.32666015625	62.2554188952599\\
-0.326171875	62.4537479222343\\
-0.32568359375	62.6513968529547\\
-0.3251953125	62.8482606627173\\
-0.32470703125	63.0442294823332\\
-0.32421875	63.2391885376516\\
-0.32373046875	63.4330181070411\\
-0.3232421875	63.6255934992514\\
-0.32275390625	63.8167850542113\\
-0.322265625	64.006458169422\\
-0.32177734375	64.1944733546748\\
-0.3212890625	64.3806863178926\\
-0.32080078125	64.5649480848902\\
-0.3203125	64.7471051558472\\
-0.31982421875	64.9269997012001\\
-0.3193359375	65.1044697995586\\
-0.31884765625	65.2793497200353\\
-0.318359375	65.4514702511713\\
-0.31787109375	65.6206590782827\\
-0.3173828125	65.7867412106785\\
-0.31689453125	65.9495394597194\\
-0.31640625	66.1088749681344\\
-0.31591796875	66.2645677903718\\
-0.3154296875	66.4164375230642\\
-0.31494140625	66.5643039838891\\
-0.314453125	66.7079879362825\\
-0.31396484375	66.8473118565536\\
-0.3134765625	66.9821007390363\\
-0.31298828125	67.1121829339608\\
-0.3125	67.2373910118011\\
-0.31201171875	67.3575626469307\\
-0.3115234375	67.4725415125885\\
-0.31103515625	67.5821781783612\\
-0.310546875	67.6863310007609\\
-0.31005859375	67.7848669969199\\
-0.3095703125	67.8776626911011\\
-0.30908203125	67.9646049235242\\
-0.30859375	68.0455916110497\\
-0.30810546875	68.1205324495177\\
-0.3076171875	68.1893495479911\\
-0.30712890625	68.2519779858664\\
-0.306640625	68.3083662847059\\
-0.30615234375	68.3584767877735\\
-0.3056640625	68.4022859415147\\
-0.30517578125	68.4397844746932\\
-0.3046875	68.4709774723919\\
-0.30419921875	68.4958843437553\\
-0.3037109375	68.5145386839598\\
-0.30322265625	68.5269880325543\\
-0.302734375	68.5332935318892\\
-0.30224609375	68.5335294908267\\
-0.3017578125	68.5277828602785\\
-0.30126953125	68.5161526283009\\
-0.30078125	68.4987491434519\\
-0.30029296875	68.4756933759151\\
-0.2998046875	68.4471161264373\\
-0.29931640625	68.4131571934678\\
-0.298828125	68.3739645090057\\
-0.29833984375	68.3296932535599\\
-0.2978515625	68.2805049603431\\
-0.29736328125	68.2265666183636\\
-0.296875	68.1680497834619\\
-0.29638671875	68.1051297056263\\
-0.2958984375	68.0379844800655\\
-0.29541015625	67.9667942286812\\
-0.294921875	67.891740317592\\
-0.29443359375	67.8130046154683\\
-0.2939453125	67.7307687964728\\
-0.29345703125	67.6452136907006\\
-0.29296875	67.5565186841644\\
-0.29248046875	67.4648611695351\\
-0.2919921875	67.3704160481374\\
-0.29150390625	67.2733552830206\\
-0.291015625	67.1738475023393\\
-0.29052734375	67.072057651783\\
-0.2900390625	66.96814669435\\
-0.28955078125	66.8622713554348\\
-0.2890625	66.7545839108967\\
-0.28857421875	66.6452320155902\\
-0.2880859375	66.5343585696774\\
-0.28759765625	66.4221016199673\\
-0.287109375	66.3085942934698\\
-0.28662109375	66.19396476037\\
-0.2861328125	66.0783362236638\\
-0.28564453125	65.9618269327539\\
-0.28515625	65.8445502184291\\
-0.28466796875	65.7266145467234\\
-0.2841796875	65.6081235893229\\
-0.28369140625	65.4891763082902\\
-0.283203125	65.3698670530571\\
-0.28271484375	65.2502856677473\\
-0.2822265625	65.130517607086\\
-0.28173828125	65.0106440592536\\
-0.28125	64.8907420742344\\
-0.28076171875	64.770884696308\\
-0.2802734375	64.6511410995018\\
-0.27978515625	64.5315767249247\\
-0.279296875	64.4122534190369\\
-0.27880859375	64.2932295720218\\
-0.2783203125	64.1745602555222\\
-0.27783203125	64.0562973591117\\
-0.27734375	63.9384897249546\\
-0.27685546875	63.8211832801823\\
-0.2763671875	63.7044211666086\\
-0.27587890625	63.588243867446\\
-0.275390625	63.472689330774\\
-0.27490234375	63.3577930895303\\
-0.2744140625	63.2435883778833\\
-0.27392578125	63.1301062438528\\
-0.2734375	63.0173756581017\\
-0.27294921875	62.9054236188472\\
-0.2724609375	62.7942752528762\\
-0.27197265625	62.683953912665\\
-0.271484375	62.5744812696268\\
-0.27099609375	62.465877403539\\
-0.2705078125	62.3581608882002\\
-0.27001953125	62.251348873393\\
-0.26953125	62.1454571632405\\
-0.26904296875	62.0405002910345\\
-0.2685546875	61.9364915906469\\
-0.26806640625	61.8334432646182\\
-0.267578125	61.7313664490329\\
-0.26708984375	61.6302712752938\\
-0.2666015625	61.5301669289019\\
-0.26611328125	61.4310617053532\\
-0.265625	61.3329630632729\\
-0.26513671875	61.2358776748857\\
-0.2646484375	61.1398114739386\\
-0.26416015625	61.0447697011892\\
-0.263671875	60.9507569475505\\
-0.26318359375	60.8577771950119\\
-0.2626953125	60.7658338554291\\
-0.26220703125	60.6749298072779\\
-0.26171875	60.585067430477\\
-0.26123046875	60.49624863936\\
-0.2607421875	60.4084749138958\\
-0.26025390625	60.3217473292346\\
-0.259765625	60.2360665836662\\
-0.25927734375	60.1514330250678\\
-0.2587890625	60.0678466759196\\
-0.25830078125	59.9853072569567\\
-0.2578125	59.903814209529\\
-0.25732421875	59.8233667167382\\
-0.2568359375	59.7439637234094\\
-0.25634765625	59.6656039549638\\
-0.255859375	59.5882859352442\\
-0.25537109375	59.5120080033573\\
-0.2548828125	59.4367683295735\\
-0.25439453125	59.3625649303453\\
-0.25390625	59.289395682484\\
-0.25341796875	59.2172583365448\\
-0.2529296875	59.1461505294578\\
-0.25244140625	59.0760697964534\\
-0.251953125	59.007013582313\\
-0.25146484375	58.9389792519859\\
-0.2509765625	58.8719641006085\\
-0.25048828125	58.8059653629528\\
-0.25	58.7409802223433\\
-0.24951171875	58.6770058190662\\
-0.2490234375	58.6140392583035\\
-0.24853515625	58.5520776176125\\
-0.248046875	58.4911179539838\\
-0.24755859375	58.4311573104963\\
-0.2470703125	58.3721927225924\\
-0.24658203125	58.3142212239985\\
-0.24609375	58.2572398523068\\
-0.24560546875	58.2012456542393\\
-0.2451171875	58.1462356906148\\
-0.24462890625	58.09220704103\\
-0.244140625	58.0391568082801\\
-0.24365234375	57.9870821225242\\
-0.2431640625	57.9359801452193\\
-0.24267578125	57.8858480728287\\
-0.2421875	57.836683140327\\
-0.24169921875	57.7884826245033\\
-0.2412109375	57.741243847085\\
-0.24072265625	57.6949641776862\\
-0.240234375	57.6496410365938\\
-0.23974609375	57.6052718974015\\
-0.2392578125	57.5618542895015\\
-0.23876953125	57.5193858004394\\
-0.23828125	57.4778640781466\\
-0.23779296875	57.4372868330534\\
-0.2373046875	57.3976518400941\\
-0.23681640625	57.3589569406058\\
-0.236328125	57.3212000441354\\
-0.23583984375	57.2843791301555\\
-0.2353515625	57.2484922496954\\
-0.23486328125	57.213537526899\\
-0.234375	57.1795131605066\\
-0.23388671875	57.1464174252726\\
-0.2333984375	57.114248673321\\
-0.23291015625	57.0830053354439\\
-0.232421875	57.052685922349\\
-0.23193359375	57.0232890258561\\
-0.2314453125	56.994813320053\\
-0.23095703125	56.9672575624097\\
-0.23046875	56.9406205948554\\
-0.22998046875	56.9149013448247\\
-0.2294921875	56.8900988262722\\
-0.22900390625	56.8662121406614\\
-0.228515625	56.8432404779311\\
-0.22802734375	56.8211831174409\\
-0.2275390625	56.8000394288959\\
-0.22705078125	56.7798088732637\\
-0.2265625	56.7604910036706\\
-0.22607421875	56.7420854662917\\
-0.2255859375	56.7245920012357\\
-0.22509765625	56.7080104434177\\
-0.224609375	56.692340723434\\
-0.22412109375	56.6775828684338\\
-0.2236328125	56.6637370029904\\
-0.22314453125	56.6508033499758\\
-0.22265625	56.63878223144\\
-0.22216796875	56.6276740694958\\
-0.2216796875	56.61747938721\\
-0.22119140625	56.6081988095078\\
-0.220703125	56.599833064084\\
-0.22021484375	56.5923829823279\\
-0.2197265625	56.5858495002647\\
-0.21923828125	56.5802336595091\\
-0.21875	56.5755366082374\\
-0.21826171875	56.5717596021777\\
-0.2177734375	56.5689040056193\\
-0.21728515625	56.566971292444\\
-0.216796875	56.5659630471785\\
-0.21630859375	56.5658809660707\\
-0.2158203125	56.5667268581914\\
-0.21533203125	56.5685026465605\\
-0.21484375	56.5712103693026\\
-0.21435546875	56.5748521808275\\
-0.2138671875	56.5794303530439\\
-0.21337890625	56.5849472766012\\
-0.212890625	56.5914054621634\\
-0.21240234375	56.5988075417172\\
-0.2119140625	56.6071562699131\\
-0.21142578125	56.61645452544\\
-0.2109375	56.6267053124396\\
-0.21044921875	56.6379117619536\\
-0.2099609375	56.6500771334137\\
-0.20947265625	56.6632048161666\\
-0.208984375	56.6772983310415\\
-0.20849609375	56.6923613319598\\
-0.2080078125	56.7083976075844\\
-0.20751953125	56.7254110830139\\
-0.20703125	56.7434058215242\\
-0.20654296875	56.7623860263469\\
-0.2060546875	56.7823560425036\\
-0.20556640625	56.803320358682\\
-0.205078125	56.8252836091599\\
-0.20458984375	56.8482505757814\\
-0.2041015625	56.8722261899793\\
-0.20361328125	56.8972155348513\\
-0.203125	56.923223847286\\
-0.20263671875	56.9502565201426\\
-0.2021484375	56.9783191044813\\
-0.20166015625	57.0074173118513\\
-0.201171875	57.0375570166293\\
-0.20068359375	57.0687442584168\\
-0.2001953125	57.1009852444897\\
-0.19970703125	57.134286352309\\
-0.19921875	57.1686541320813\\
-0.19873046875	57.2040953093834\\
-0.1982421875	57.2406167878398\\
-0.19775390625	57.2782256518598\\
-0.197265625	57.3169291694314\\
-0.19677734375	57.3567347949717\\
-0.1962890625	57.3976501722388\\
-0.19580078125	57.4396831372966\\
-0.1953125	57.482841721537\\
-0.19482421875	57.527134154761\\
-0.1943359375	57.572568868312\\
-0.19384765625	57.6191544982647\\
-0.193359375	57.6668998886662\\
-0.19287109375	57.7158140948299\\
-0.1923828125	57.7659063866759\\
-0.19189453125	57.8171862521229\\
-0.19140625	57.8696634005231\\
-0.19091796875	57.9233477661371\\
-0.1904296875	57.9782495116511\\
-0.18994140625	58.0343790317271\\
-0.189453125	58.0917469565842\\
-0.18896484375	58.1503641556063\\
-0.1884765625	58.2102417409703\\
-0.18798828125	58.2713910712877\\
-0.1875	58.3338237552555\\
-0.18701171875	58.397551655306\\
-0.1865234375	58.4625868912502\\
-0.18603515625	58.528941843898\\
-0.185546875	58.5966291586581\\
-0.18505859375	58.6656617490892\\
-0.1845703125	58.7360528004062\\
-0.18408203125	58.8078157729124\\
-0.18359375	58.8809644053515\\
-0.18310546875	58.9555127181601\\
-0.1826171875	59.0314750165966\\
-0.18212890625	59.1088658937309\\
-0.181640625	59.1877002332697\\
-0.18115234375	59.2679932121876\\
-0.1806640625	59.3497603031421\\
-0.18017578125	59.4330172766342\\
-0.1796875	59.5177802028858\\
-0.17919921875	59.6040654533918\\
-0.1787109375	59.6918897021066\\
-0.17822265625	59.7812699262175\\
-0.177734375	59.8722234064579\\
-0.17724609375	59.9647677269009\\
-0.1767578125	60.058920774172\\
-0.17626953125	60.1547007360182\\
-0.17578125	60.2521260991553\\
-0.17529296875	60.3512156463144\\
-0.1748046875	60.4519884523997\\
-0.17431640625	60.5544638796565\\
-0.173828125	60.6586615717474\\
-0.17333984375	60.764601446619\\
-0.1728515625	60.8723036880235\\
-0.17236328125	60.9817887355617\\
-0.171875	61.0930772730933\\
-0.17138671875	61.2061902153337\\
-0.1708984375	61.3211486924664\\
-0.17041015625	61.4379740325547\\
-0.169921875	61.5566877415333\\
-0.16943359375	61.6773114805345\\
-0.1689453125	61.7998670402783\\
-0.16845703125	61.9243763122341\\
-0.16796875	62.0508612562203\\
-0.16748046875	62.179343864107\\
-0.1669921875	62.3098461192085\\
-0.16650390625	62.4423899509568\\
-0.166015625	62.5769971843862\\
-0.16552734375	62.7136894839114\\
-0.1650390625	62.8524882908474\\
-0.16455078125	62.9934147540529\\
-0.1640625	63.1364896530279\\
-0.16357421875	63.2817333127409\\
-0.1630859375	63.429165509369\\
-0.16259765625	63.5788053660849\\
-0.162109375	63.7306712379399\\
-0.16162109375	63.8847805847891\\
-0.1611328125	64.0411498311234\\
-0.16064453125	64.1997942115811\\
-0.16015625	64.3607276007805\\
-0.15966796875	64.5239623260079\\
-0.1591796875	64.6895089611896\\
-0.15869140625	64.857376100404\\
-0.158203125	65.0275701090936\\
-0.15771484375	65.2000948509532\\
-0.1572265625	65.3749513883479\\
-0.15673828125	65.5521376539218\\
-0.15625	65.7316480909362\\
-0.15576171875	65.9134732596695\\
-0.1552734375	66.0975994070826\\
-0.15478515625	66.2840079967736\\
-0.154296875	66.4726751961073\\
-0.15380859375	66.6635713172717\\
-0.1533203125	66.8566602088962\\
-0.15283203125	67.0518985948051\\
-0.15234375	67.2492353564339\\
-0.15185546875	67.4486107554667\\
-0.1513671875	67.6499555933501\\
-0.15087890625	67.8531903045486\\
-0.150390625	68.0582239806975\\
-0.14990234375	68.2649533232877\\
-0.1494140625	68.47326152315\\
-0.14892578125	68.6830170658606\\
-0.1484375	68.8940724632962\\
-0.14794921875	69.1062629129925\\
-0.1474609375	69.3194048887167\\
-0.14697265625	69.5332946678478\\
-0.146484375	69.7477068038023\\
-0.14599609375	69.962392554895\\
-0.1455078125	70.1770782847647\\
-0.14501953125	70.39146385385\\
-0.14453125	70.6052210263787\\
-0.14404296875	70.8179919230093\\
-0.1435546875	71.0293875555273\\
-0.14306640625	71.2389864868536\\
-0.142578125	71.4463336669303\\
-0.14208984375	71.6509395025801\\
-0.1416015625	71.8522792269875\\
-0.14111328125	72.0497926415781\\
-0.140625	72.2428843093453\\
-0.14013671875	72.4309242834837\\
-0.1396484375	72.6132494577906\\
-0.13916015625	72.7891656249892\\
-0.138671875	72.9579503249127\\
-0.13818359375	73.1188565556438\\
-0.1376953125	73.2711174062789\\
-0.13720703125	73.4139516494901\\
-0.13671875	73.5465703049776\\
-0.13623046875	73.6681841514211\\
-0.1357421875	73.7780121251099\\
-0.13525390625	73.8752904993544\\
-0.134765625	73.9592826918853\\
-0.13427734375	74.0292895003877\\
-0.1337890625	74.0846595221591\\
-0.13330078125	74.1247994763643\\
-0.1328125	74.1491841199601\\
-0.13232421875	74.1573654346381\\
-0.1318359375	74.1489807646955\\
-0.13134765625	74.1237596062414\\
-0.130859375	74.0815287867828\\
-0.13037109375	74.0222158295939\\
-0.1298828125	73.9458503663812\\
-0.12939453125	73.8525635402575\\
-0.12890625	73.7425854235896\\
-0.12841796875	73.6162405560722\\
-0.1279296875	73.4739417819543\\
-0.12744140625	73.3161826267052\\
-0.126953125	73.1435284991436\\
-0.12646484375	72.9566070329797\\
-0.1259765625	72.7560978914659\\
-0.12548828125	72.5427223514909\\
-0.125	72.3172329612346\\
-0.12451171875	72.0804035316592\\
-0.1240234375	71.8330196802242\\
-0.12353515625	71.5758700989157\\
-0.123046875	71.3097386713831\\
-0.12255859375	71.0353975184541\\
-0.1220703125	70.7536010098313\\
-0.12158203125	70.4650807437319\\
-0.12109375	70.1705414665357\\
-0.12060546875	69.8706578813458\\
-0.1201171875	69.5660722775273\\
-0.11962890625	69.2573929022759\\
-0.119140625	68.9451929892657\\
-0.11865234375	68.6300103576325\\
-0.1181640625	68.3123474960616\\
-0.11767578125	67.9926720507756\\
-0.1171875	67.6714176419307\\
-0.11669921875	67.3489849398285\\
-0.1162109375	67.0257429397256\\
-0.11572265625	66.7020303816427\\
-0.115234375	66.3781572689862\\
-0.11474609375	66.0544064468365\\
-0.1142578125	65.7310352073143\\
-0.11376953125	65.4082768953025\\
-0.11328125	65.0863424931067\\
-0.11279296875	64.7654221672286\\
-0.1123046875	64.4456867644052\\
-0.11181640625	64.1272892474589\\
-0.111328125	63.8103660643563\\
-0.11083984375	63.4950384462161\\
-0.1103515625	63.1814136319553\\
-0.10986328125	62.8695860187959\\
-0.109375	62.559638239096\\
-0.10888671875	62.2516421649196\\
-0.1083984375	61.9456598424749\\
-0.10791015625	61.6417443590898\\
-0.107421875	61.339940645758\\
-0.10693359375	61.0402862185237\\
-0.1064453125	60.7428118621273\\
-0.10595703125	60.447542259366\\
-0.10546875	60.1544965696271\\
-0.10498046875	59.8636889599975\\
-0.1044921875	59.5751290922309\\
-0.10400390625	59.2888225687553\\
-0.103515625	59.0047713407547\\
-0.10302734375	58.7229740811893\\
-0.1025390625	58.443426525482\\
-0.10205078125	58.166121782406\\
-0.1015625	57.891050617577\\
-0.10107421875	57.6182017117579\\
-0.1005859375	57.3475618960605\\
-0.10009765625	57.0791163659543\\
-0.099609375	56.8128488758649\\
-0.09912109375	56.5487419159991\\
-0.0986328125	56.286776872913\\
-0.09814453125	56.0269341752145\\
-0.09765625	55.7691934256865\\
-0.09716796875	55.5135335210017\\
-0.0966796875	55.2599327601116\\
-0.09619140625	55.0083689423052\\
-0.095703125	54.7588194558354\\
-0.09521484375	54.511261357952\\
-0.0947265625	54.2656714470983\\
-0.09423828125	54.0220263279647\\
-0.09375	53.7803024700367\\
-0.09326171875	53.5404762602213\\
-0.0927734375	53.3025240500719\\
-0.09228515625	53.0664221981087\\
-0.091796875	52.8321471076672\\
-0.09130859375	52.5996752606838\\
-0.0908203125	52.3689832477821\\
-0.09033203125	52.1400477950019\\
-0.08984375	51.9128457874712\\
-0.08935546875	51.6873542903052\\
-0.0888671875	51.4635505669856\\
-0.08837890625	51.241412095453\\
-0.087890625	51.0209165821241\\
-0.08740234375	50.8020419740282\\
-0.0869140625	50.5847664692376\\
-0.08642578125	50.3690685257546\\
-0.0859375	50.1549268690015\\
-0.08544921875	49.9423204980432\\
-0.0849609375	49.7312286906721\\
-0.08447265625	49.5216310074565\\
-0.083984375	49.313507294864\\
-0.08349609375	49.1068376875417\\
-0.0830078125	48.9016026098464\\
-0.08251953125	48.6977827766949\\
-0.08203125	48.495359193806\\
-0.08154296875	48.2943131573997\\
-0.0810546875	48.0946262534089\\
-0.08056640625	47.8962803562562\\
-0.080078125	47.6992576272452\\
-0.07958984375	47.5035405126087\\
-0.0791015625	47.3091117412536\\
-0.07861328125	47.1159543222399\\
-0.078125	46.9240515420232\\
-0.07763671875	46.7333869614934\\
-0.0771484375	46.5439444128369\\
-0.07666015625	46.3557079962435\\
-0.076171875	46.1686620764839\\
-0.07568359375	45.9827912793748\\
-0.0751953125	45.7980804881538\\
-0.07470703125	45.6145148397755\\
-0.07421875	45.4320797211477\\
-0.07373046875	45.2507607653199\\
-0.0732421875	45.0705438476366\\
-0.07275390625	44.8914150818648\\
-0.072265625	44.7133608163073\\
-0.07177734375	44.5363676299106\\
-0.0712890625	44.3604223283747\\
-0.07080078125	44.1855119402712\\
-0.0703125	44.0116237131779\\
-0.06982421875	43.8387451098343\\
-0.0693359375	43.6668638043227\\
-0.06884765625	43.4959676782815\\
-0.068359375	43.3260448171511\\
-0.06787109375	43.1570835064587\\
-0.0673828125	42.9890722281456\\
-0.06689453125	42.8219996569354\\
-0.06640625	42.6558546567507\\
-0.06591796875	42.4906262771756\\
-0.0654296875	42.3263037499699\\
-0.06494140625	42.1628764856326\\
-0.064453125	42.0003340700182\\
-0.06396484375	41.8386662610054\\
-0.0634765625	41.6778629852206\\
-0.06298828125	41.5179143348149\\
-0.0625	41.3588105642952\\
-0.06201171875	41.2005420874119\\
-0.0615234375	41.0430994740996\\
-0.06103515625	40.8864734474739\\
-0.060546875	40.7306548808834\\
-0.06005859375	40.5756347950148\\
-0.0595703125	40.4214043550541\\
-0.05908203125	40.2679548679006\\
-0.05859375	40.1152777794342\\
-0.05810546875	39.9633646718372\\
-0.0576171875	39.8122072609667\\
-0.05712890625	39.6617973937805\\
-0.056640625	39.5121270458132\\
-0.05615234375	39.3631883187021\\
-0.0556640625	39.214973437766\\
-0.05517578125	39.0674747496293\\
-0.0546875	38.9206847198976\\
-0.05419921875	38.7745959308781\\
-0.0537109375	38.62920107935\\
-0.05322265625	38.484492974378\\
-0.052734375	38.3404645351732\\
-0.05224609375	38.1971087889973\\
-0.0517578125	38.0544188691107\\
-0.05126953125	37.9123880127637\\
-0.05078125	37.7710095592294\\
-0.05029296875	37.6302769478784\\
-0.0498046875	37.4901837162933\\
-0.04931640625	37.3507234984227\\
-0.048828125	37.2118900227755\\
-0.04833984375	37.0736771106513\\
-0.0478515625	36.9360786744093\\
-0.04736328125	36.7990887157737\\
-0.046875	36.6627013241743\\
-0.04638671875	36.5269106751223\\
-0.0458984375	36.3917110286211\\
-0.04541015625	36.2570967276095\\
-0.044921875	36.1230621964384\\
-0.04443359375	35.9896019393797\\
-0.0439453125	35.8567105391663\\
-0.04345703125	35.7243826555625\\
-0.04296875	35.5926130239661\\
-0.04248046875	35.4613964540372\\
-0.0419921875	35.3307278283588\\
-0.04150390625	35.2006021011229\\
-0.041015625	35.0710142968459\\
-0.04052734375	34.9419595091098\\
-0.0400390625	34.8134328993305\\
-0.03955078125	34.685429695551\\
-0.0390625	34.5579451912606\\
-0.03857421875	34.430974744238\\
-0.0380859375	34.3045137754182\\
-0.03759765625	34.1785577677841\\
-0.037109375	34.0531022652798\\
-0.03662109375	33.9281428717465\\
-0.0361328125	33.8036752498815\\
-0.03564453125	33.6796951202165\\
-0.03515625	33.556198260119\\
-0.03466796875	33.4331805028131\\
-0.0341796875	33.31063773642\\
-0.03369140625	33.1885659030191\\
-0.033203125	33.066960997727\\
-0.03271484375	32.9458190677966\\
-0.0322265625	32.8251362117327\\
-0.03173828125	32.7049085784273\\
-0.03125	32.585132366311\\
-0.03076171875	32.4658038225223\\
-0.0302734375	32.3469192420926\\
-0.02978515625	32.2284749671486\\
-0.029296875	32.1104673861302\\
-0.02880859375	31.9928929330238\\
-0.0283203125	31.8757480866106\\
-0.02783203125	31.7590293697304\\
-0.02734375	31.6427333485599\\
-0.02685546875	31.5268566319047\\
-0.0263671875	31.4113958705057\\
-0.02587890625	31.2963477563588\\
-0.025390625	31.1817090220478\\
-0.02490234375	31.067476440091\\
-0.0244140625	30.9536468222998\\
-0.02392578125	30.84021701915\\
-0.0234375	30.7271839191649\\
-0.02294921875	30.6145444483114\\
-0.0224609375	30.5022955694059\\
-0.02197265625	30.3904342815337\\
-0.021484375	30.2789576194775\\
-0.02099609375	30.1678626531583\\
-0.0205078125	30.0571464870863\\
-0.02001953125	29.9468062598223\\
-0.01953125	29.8368391434487\\
-0.01904296875	29.727242343052\\
-0.0185546875	29.6180130962132\\
-0.01806640625	29.5091486725093\\
-0.017578125	29.4006463730228\\
-0.01708984375	29.2925035298617\\
-0.0166015625	29.1847175056867\\
-0.01611328125	29.0772856932495\\
-0.015625	28.9702055149369\\
-0.01513671875	28.8634744223259\\
-0.0146484375	28.7570898957448\\
-0.01416015625	28.6510494438439\\
-0.013671875	28.5453506031735\\
-0.01318359375	28.4399909377687\\
-0.0126953125	28.3349680387438\\
-0.01220703125	28.2302795238918\\
-0.01171875	28.1259230372925\\
-0.01123046875	28.0218962489276\\
-0.0107421875	27.9181968543017\\
-0.01025390625	27.8148225740715\\
-0.009765625	27.7117711536802\\
-0.00927734375	27.6090403629998\\
-0.0087890625	27.5066279959785\\
-0.00830078125	27.4045318702949\\
-0.0078125	27.3027498270187\\
-0.00732421875	27.2012797302764\\
-0.0068359375	27.1001194669237\\
-0.00634765625	26.999266946223\\
-0.005859375	26.8987200995275\\
-0.00537109375	26.798476879969\\
-0.0048828125	26.6985352621533\\
-0.00439453125	26.5988932418592\\
-0.00390625	26.4995488357434\\
-0.00341796875	26.40050008105\\
-0.0029296875	26.301745035326\\
-0.00244140625	26.2032817761404\\
-0.001953125	26.1051084008085\\
-0.00146484375	26.0072230261214\\
-0.0009765625	25.9096237880794\\
-0.00048828125	25.8123088416305\\
0	25.7152763604126\\
0.00048828125	25.8123088416305\\
0.0009765625	25.9096237880794\\
0.00146484375	26.0072230261214\\
0.001953125	26.1051084008085\\
0.00244140625	26.2032817761404\\
0.0029296875	26.301745035326\\
0.00341796875	26.40050008105\\
0.00390625	26.4995488357434\\
0.00439453125	26.5988932418592\\
0.0048828125	26.6985352621533\\
0.00537109375	26.798476879969\\
0.005859375	26.8987200995275\\
0.00634765625	26.999266946223\\
0.0068359375	27.1001194669237\\
0.00732421875	27.2012797302764\\
0.0078125	27.3027498270187\\
0.00830078125	27.4045318702949\\
0.0087890625	27.5066279959785\\
0.00927734375	27.6090403629998\\
0.009765625	27.7117711536802\\
0.01025390625	27.8148225740715\\
0.0107421875	27.9181968543017\\
0.01123046875	28.0218962489276\\
0.01171875	28.1259230372925\\
0.01220703125	28.2302795238918\\
0.0126953125	28.3349680387438\\
0.01318359375	28.4399909377687\\
0.013671875	28.5453506031735\\
0.01416015625	28.6510494438439\\
0.0146484375	28.7570898957448\\
0.01513671875	28.8634744223259\\
0.015625	28.9702055149369\\
0.01611328125	29.0772856932495\\
0.0166015625	29.1847175056867\\
0.01708984375	29.2925035298617\\
0.017578125	29.4006463730228\\
0.01806640625	29.5091486725093\\
0.0185546875	29.6180130962132\\
0.01904296875	29.727242343052\\
0.01953125	29.8368391434487\\
0.02001953125	29.9468062598223\\
0.0205078125	30.0571464870863\\
0.02099609375	30.1678626531583\\
0.021484375	30.2789576194775\\
0.02197265625	30.3904342815337\\
0.0224609375	30.5022955694059\\
0.02294921875	30.6145444483114\\
0.0234375	30.7271839191649\\
0.02392578125	30.84021701915\\
0.0244140625	30.9536468222998\\
0.02490234375	31.067476440091\\
0.025390625	31.1817090220478\\
0.02587890625	31.2963477563588\\
0.0263671875	31.4113958705057\\
0.02685546875	31.5268566319047\\
0.02734375	31.6427333485599\\
0.02783203125	31.7590293697304\\
0.0283203125	31.8757480866106\\
0.02880859375	31.9928929330238\\
0.029296875	32.1104673861302\\
0.02978515625	32.2284749671486\\
0.0302734375	32.3469192420926\\
0.03076171875	32.4658038225223\\
0.03125	32.585132366311\\
0.03173828125	32.7049085784273\\
0.0322265625	32.8251362117327\\
0.03271484375	32.9458190677966\\
0.033203125	33.066960997727\\
0.03369140625	33.1885659030191\\
0.0341796875	33.31063773642\\
0.03466796875	33.4331805028131\\
0.03515625	33.556198260119\\
0.03564453125	33.6796951202165\\
0.0361328125	33.8036752498815\\
0.03662109375	33.9281428717465\\
0.037109375	34.0531022652798\\
0.03759765625	34.1785577677841\\
0.0380859375	34.3045137754182\\
0.03857421875	34.430974744238\\
0.0390625	34.5579451912606\\
0.03955078125	34.685429695551\\
0.0400390625	34.8134328993305\\
0.04052734375	34.9419595091098\\
0.041015625	35.0710142968459\\
0.04150390625	35.2006021011229\\
0.0419921875	35.3307278283588\\
0.04248046875	35.4613964540372\\
0.04296875	35.5926130239661\\
0.04345703125	35.7243826555625\\
0.0439453125	35.8567105391663\\
0.04443359375	35.9896019393797\\
0.044921875	36.1230621964384\\
0.04541015625	36.2570967276095\\
0.0458984375	36.3917110286211\\
0.04638671875	36.5269106751223\\
0.046875	36.6627013241743\\
0.04736328125	36.7990887157737\\
0.0478515625	36.9360786744093\\
0.04833984375	37.0736771106513\\
0.048828125	37.2118900227755\\
0.04931640625	37.3507234984227\\
0.0498046875	37.4901837162933\\
0.05029296875	37.6302769478784\\
0.05078125	37.7710095592294\\
0.05126953125	37.9123880127637\\
0.0517578125	38.0544188691107\\
0.05224609375	38.1971087889973\\
0.052734375	38.3404645351732\\
0.05322265625	38.484492974378\\
0.0537109375	38.62920107935\\
0.05419921875	38.7745959308781\\
0.0546875	38.9206847198976\\
0.05517578125	39.0674747496293\\
0.0556640625	39.214973437766\\
0.05615234375	39.3631883187021\\
0.056640625	39.5121270458132\\
0.05712890625	39.6617973937805\\
0.0576171875	39.8122072609667\\
0.05810546875	39.9633646718372\\
0.05859375	40.1152777794342\\
0.05908203125	40.2679548679006\\
0.0595703125	40.4214043550541\\
0.06005859375	40.5756347950148\\
0.060546875	40.7306548808834\\
0.06103515625	40.8864734474739\\
0.0615234375	41.0430994740996\\
0.06201171875	41.2005420874119\\
0.0625	41.3588105642952\\
0.06298828125	41.5179143348149\\
0.0634765625	41.6778629852206\\
0.06396484375	41.8386662610054\\
0.064453125	42.0003340700182\\
0.06494140625	42.1628764856326\\
0.0654296875	42.3263037499699\\
0.06591796875	42.4906262771756\\
0.06640625	42.6558546567507\\
0.06689453125	42.8219996569354\\
0.0673828125	42.9890722281456\\
0.06787109375	43.1570835064587\\
0.068359375	43.3260448171511\\
0.06884765625	43.4959676782815\\
0.0693359375	43.6668638043227\\
0.06982421875	43.8387451098343\\
0.0703125	44.0116237131779\\
0.07080078125	44.1855119402712\\
0.0712890625	44.3604223283747\\
0.07177734375	44.5363676299106\\
0.072265625	44.7133608163073\\
0.07275390625	44.8914150818648\\
0.0732421875	45.0705438476366\\
0.07373046875	45.2507607653199\\
0.07421875	45.4320797211477\\
0.07470703125	45.6145148397755\\
0.0751953125	45.7980804881538\\
0.07568359375	45.9827912793748\\
0.076171875	46.1686620764839\\
0.07666015625	46.3557079962435\\
0.0771484375	46.5439444128369\\
0.07763671875	46.7333869614934\\
0.078125	46.9240515420232\\
0.07861328125	47.1159543222399\\
0.0791015625	47.3091117412536\\
0.07958984375	47.5035405126087\\
0.080078125	47.6992576272452\\
0.08056640625	47.8962803562562\\
0.0810546875	48.0946262534089\\
0.08154296875	48.2943131573997\\
0.08203125	48.495359193806\\
0.08251953125	48.6977827766949\\
0.0830078125	48.9016026098464\\
0.08349609375	49.1068376875417\\
0.083984375	49.313507294864\\
0.08447265625	49.5216310074565\\
0.0849609375	49.7312286906721\\
0.08544921875	49.9423204980432\\
0.0859375	50.1549268690015\\
0.08642578125	50.3690685257546\\
0.0869140625	50.5847664692376\\
0.08740234375	50.8020419740282\\
0.087890625	51.0209165821241\\
0.08837890625	51.241412095453\\
0.0888671875	51.4635505669856\\
0.08935546875	51.6873542903052\\
0.08984375	51.9128457874712\\
0.09033203125	52.1400477950019\\
0.0908203125	52.3689832477821\\
0.09130859375	52.5996752606838\\
0.091796875	52.8321471076672\\
0.09228515625	53.0664221981087\\
0.0927734375	53.3025240500719\\
0.09326171875	53.5404762602213\\
0.09375	53.7803024700367\\
0.09423828125	54.0220263279647\\
0.0947265625	54.2656714470983\\
0.09521484375	54.511261357952\\
0.095703125	54.7588194558354\\
0.09619140625	55.0083689423052\\
0.0966796875	55.2599327601116\\
0.09716796875	55.5135335210017\\
0.09765625	55.7691934256865\\
0.09814453125	56.0269341752145\\
0.0986328125	56.286776872913\\
0.09912109375	56.5487419159991\\
0.099609375	56.8128488758649\\
0.10009765625	57.0791163659543\\
0.1005859375	57.3475618960605\\
0.10107421875	57.6182017117579\\
0.1015625	57.891050617577\\
0.10205078125	58.166121782406\\
0.1025390625	58.443426525482\\
0.10302734375	58.7229740811893\\
0.103515625	59.0047713407547\\
0.10400390625	59.2888225687553\\
0.1044921875	59.5751290922309\\
0.10498046875	59.8636889599975\\
0.10546875	60.1544965696271\\
0.10595703125	60.447542259366\\
0.1064453125	60.7428118621273\\
0.10693359375	61.0402862185237\\
0.107421875	61.339940645758\\
0.10791015625	61.6417443590898\\
0.1083984375	61.9456598424749\\
0.10888671875	62.2516421649196\\
0.109375	62.559638239096\\
0.10986328125	62.8695860187959\\
0.1103515625	63.1814136319553\\
0.11083984375	63.4950384462161\\
0.111328125	63.8103660643563\\
0.11181640625	64.1272892474589\\
0.1123046875	64.4456867644052\\
0.11279296875	64.7654221672286\\
0.11328125	65.0863424931067\\
0.11376953125	65.4082768953025\\
0.1142578125	65.7310352073143\\
0.11474609375	66.0544064468365\\
0.115234375	66.3781572689862\\
0.11572265625	66.7020303816427\\
0.1162109375	67.0257429397256\\
0.11669921875	67.3489849398285\\
0.1171875	67.6714176419307\\
0.11767578125	67.9926720507756\\
0.1181640625	68.3123474960616\\
0.11865234375	68.6300103576325\\
0.119140625	68.9451929892657\\
0.11962890625	69.2573929022759\\
0.1201171875	69.5660722775273\\
0.12060546875	69.8706578813458\\
0.12109375	70.1705414665357\\
0.12158203125	70.4650807437319\\
0.1220703125	70.7536010098313\\
0.12255859375	71.0353975184541\\
0.123046875	71.3097386713831\\
0.12353515625	71.5758700989157\\
0.1240234375	71.8330196802242\\
0.12451171875	72.0804035316592\\
0.125	72.3172329612346\\
0.12548828125	72.5427223514909\\
0.1259765625	72.7560978914659\\
0.12646484375	72.9566070329797\\
0.126953125	73.1435284991436\\
0.12744140625	73.3161826267052\\
0.1279296875	73.4739417819543\\
0.12841796875	73.6162405560722\\
0.12890625	73.7425854235896\\
0.12939453125	73.8525635402575\\
0.1298828125	73.9458503663812\\
0.13037109375	74.0222158295939\\
0.130859375	74.0815287867828\\
0.13134765625	74.1237596062414\\
0.1318359375	74.1489807646955\\
0.13232421875	74.1573654346381\\
0.1328125	74.1491841199601\\
0.13330078125	74.1247994763643\\
0.1337890625	74.0846595221591\\
0.13427734375	74.0292895003877\\
0.134765625	73.9592826918853\\
0.13525390625	73.8752904993544\\
0.1357421875	73.7780121251099\\
0.13623046875	73.6681841514211\\
0.13671875	73.5465703049776\\
0.13720703125	73.4139516494901\\
0.1376953125	73.2711174062789\\
0.13818359375	73.1188565556438\\
0.138671875	72.9579503249127\\
0.13916015625	72.7891656249892\\
0.1396484375	72.6132494577906\\
0.14013671875	72.4309242834837\\
0.140625	72.2428843093453\\
0.14111328125	72.0497926415781\\
0.1416015625	71.8522792269875\\
0.14208984375	71.6509395025801\\
0.142578125	71.4463336669303\\
0.14306640625	71.2389864868536\\
0.1435546875	71.0293875555273\\
0.14404296875	70.8179919230093\\
0.14453125	70.6052210263787\\
0.14501953125	70.39146385385\\
0.1455078125	70.1770782847647\\
0.14599609375	69.962392554895\\
0.146484375	69.7477068038023\\
0.14697265625	69.5332946678478\\
0.1474609375	69.3194048887167\\
0.14794921875	69.1062629129925\\
0.1484375	68.8940724632962\\
0.14892578125	68.6830170658606\\
0.1494140625	68.47326152315\\
0.14990234375	68.2649533232877\\
0.150390625	68.0582239806975\\
0.15087890625	67.8531903045486\\
0.1513671875	67.6499555933501\\
0.15185546875	67.4486107554667\\
0.15234375	67.2492353564339\\
0.15283203125	67.0518985948051\\
0.1533203125	66.8566602088962\\
0.15380859375	66.6635713172717\\
0.154296875	66.4726751961073\\
0.15478515625	66.2840079967736\\
0.1552734375	66.0975994070826\\
0.15576171875	65.9134732596695\\
0.15625	65.7316480909362\\
0.15673828125	65.5521376539218\\
0.1572265625	65.3749513883479\\
0.15771484375	65.2000948509532\\
0.158203125	65.0275701090936\\
0.15869140625	64.857376100404\\
0.1591796875	64.6895089611896\\
0.15966796875	64.5239623260079\\
0.16015625	64.3607276007805\\
0.16064453125	64.1997942115811\\
0.1611328125	64.0411498311234\\
0.16162109375	63.8847805847891\\
0.162109375	63.7306712379399\\
0.16259765625	63.5788053660849\\
0.1630859375	63.429165509369\\
0.16357421875	63.2817333127409\\
0.1640625	63.1364896530279\\
0.16455078125	62.9934147540529\\
0.1650390625	62.8524882908474\\
0.16552734375	62.7136894839114\\
0.166015625	62.5769971843862\\
0.16650390625	62.4423899509568\\
0.1669921875	62.3098461192085\\
0.16748046875	62.179343864107\\
0.16796875	62.0508612562203\\
0.16845703125	61.9243763122341\\
0.1689453125	61.7998670402783\\
0.16943359375	61.6773114805345\\
0.169921875	61.5566877415333\\
0.17041015625	61.4379740325547\\
0.1708984375	61.3211486924664\\
0.17138671875	61.2061902153337\\
0.171875	61.0930772730933\\
0.17236328125	60.9817887355617\\
0.1728515625	60.8723036880235\\
0.17333984375	60.764601446619\\
0.173828125	60.6586615717474\\
0.17431640625	60.5544638796565\\
0.1748046875	60.4519884523997\\
0.17529296875	60.3512156463144\\
0.17578125	60.2521260991553\\
0.17626953125	60.1547007360182\\
0.1767578125	60.058920774172\\
0.17724609375	59.9647677269009\\
0.177734375	59.8722234064579\\
0.17822265625	59.7812699262175\\
0.1787109375	59.6918897021066\\
0.17919921875	59.6040654533918\\
0.1796875	59.5177802028858\\
0.18017578125	59.4330172766342\\
0.1806640625	59.3497603031421\\
0.18115234375	59.2679932121876\\
0.181640625	59.1877002332697\\
0.18212890625	59.1088658937309\\
0.1826171875	59.0314750165966\\
0.18310546875	58.9555127181601\\
0.18359375	58.8809644053515\\
0.18408203125	58.8078157729124\\
0.1845703125	58.7360528004062\\
0.18505859375	58.6656617490892\\
0.185546875	58.5966291586581\\
0.18603515625	58.528941843898\\
0.1865234375	58.4625868912502\\
0.18701171875	58.397551655306\\
0.1875	58.3338237552555\\
0.18798828125	58.2713910712877\\
0.1884765625	58.2102417409703\\
0.18896484375	58.1503641556063\\
0.189453125	58.0917469565842\\
0.18994140625	58.0343790317271\\
0.1904296875	57.9782495116511\\
0.19091796875	57.9233477661371\\
0.19140625	57.8696634005231\\
0.19189453125	57.8171862521229\\
0.1923828125	57.7659063866759\\
0.19287109375	57.7158140948299\\
0.193359375	57.6668998886662\\
0.19384765625	57.6191544982647\\
0.1943359375	57.572568868312\\
0.19482421875	57.527134154761\\
0.1953125	57.482841721537\\
0.19580078125	57.4396831372966\\
0.1962890625	57.3976501722388\\
0.19677734375	57.3567347949717\\
0.197265625	57.3169291694314\\
0.19775390625	57.2782256518598\\
0.1982421875	57.2406167878398\\
0.19873046875	57.2040953093834\\
0.19921875	57.1686541320813\\
0.19970703125	57.134286352309\\
0.2001953125	57.1009852444897\\
0.20068359375	57.0687442584168\\
0.201171875	57.0375570166293\\
0.20166015625	57.0074173118513\\
0.2021484375	56.9783191044813\\
0.20263671875	56.9502565201426\\
0.203125	56.923223847286\\
0.20361328125	56.8972155348513\\
0.2041015625	56.8722261899793\\
0.20458984375	56.8482505757814\\
0.205078125	56.8252836091599\\
0.20556640625	56.803320358682\\
0.2060546875	56.7823560425036\\
0.20654296875	56.7623860263469\\
0.20703125	56.7434058215242\\
0.20751953125	56.7254110830139\\
0.2080078125	56.7083976075844\\
0.20849609375	56.6923613319598\\
0.208984375	56.6772983310415\\
0.20947265625	56.6632048161666\\
0.2099609375	56.6500771334137\\
0.21044921875	56.6379117619536\\
0.2109375	56.6267053124396\\
0.21142578125	56.61645452544\\
0.2119140625	56.6071562699131\\
0.21240234375	56.5988075417172\\
0.212890625	56.5914054621634\\
0.21337890625	56.5849472766012\\
0.2138671875	56.5794303530439\\
0.21435546875	56.5748521808275\\
0.21484375	56.5712103693026\\
0.21533203125	56.5685026465605\\
0.2158203125	56.5667268581914\\
0.21630859375	56.5658809660707\\
0.216796875	56.5659630471785\\
0.21728515625	56.566971292444\\
0.2177734375	56.5689040056193\\
0.21826171875	56.5717596021777\\
0.21875	56.5755366082374\\
0.21923828125	56.5802336595091\\
0.2197265625	56.5858495002647\\
0.22021484375	56.5923829823279\\
0.220703125	56.599833064084\\
0.22119140625	56.6081988095078\\
0.2216796875	56.61747938721\\
0.22216796875	56.6276740694958\\
0.22265625	56.63878223144\\
0.22314453125	56.6508033499758\\
0.2236328125	56.6637370029904\\
0.22412109375	56.6775828684338\\
0.224609375	56.692340723434\\
0.22509765625	56.7080104434177\\
0.2255859375	56.7245920012357\\
0.22607421875	56.7420854662917\\
0.2265625	56.7604910036706\\
0.22705078125	56.7798088732637\\
0.2275390625	56.8000394288959\\
0.22802734375	56.8211831174409\\
0.228515625	56.8432404779311\\
0.22900390625	56.8662121406614\\
0.2294921875	56.8900988262722\\
0.22998046875	56.9149013448247\\
0.23046875	56.9406205948554\\
0.23095703125	56.9672575624097\\
0.2314453125	56.994813320053\\
0.23193359375	57.0232890258561\\
0.232421875	57.052685922349\\
0.23291015625	57.0830053354439\\
0.2333984375	57.114248673321\\
0.23388671875	57.1464174252726\\
0.234375	57.1795131605066\\
0.23486328125	57.213537526899\\
0.2353515625	57.2484922496954\\
0.23583984375	57.2843791301555\\
0.236328125	57.3212000441354\\
0.23681640625	57.3589569406058\\
0.2373046875	57.3976518400941\\
0.23779296875	57.4372868330534\\
0.23828125	57.4778640781466\\
0.23876953125	57.5193858004394\\
0.2392578125	57.5618542895015\\
0.23974609375	57.6052718974015\\
0.240234375	57.6496410365938\\
0.24072265625	57.6949641776862\\
0.2412109375	57.741243847085\\
0.24169921875	57.7884826245033\\
0.2421875	57.836683140327\\
0.24267578125	57.8858480728287\\
0.2431640625	57.9359801452193\\
0.24365234375	57.9870821225242\\
0.244140625	58.0391568082801\\
0.24462890625	58.09220704103\\
0.2451171875	58.1462356906148\\
0.24560546875	58.2012456542393\\
0.24609375	58.2572398523068\\
0.24658203125	58.3142212239985\\
0.2470703125	58.3721927225924\\
0.24755859375	58.4311573104963\\
0.248046875	58.4911179539838\\
0.24853515625	58.5520776176125\\
0.2490234375	58.6140392583035\\
0.24951171875	58.6770058190662\\
0.25	58.7409802223433\\
0.25048828125	58.8059653629528\\
0.2509765625	58.8719641006085\\
0.25146484375	58.9389792519859\\
0.251953125	59.007013582313\\
0.25244140625	59.0760697964534\\
0.2529296875	59.1461505294578\\
0.25341796875	59.2172583365448\\
0.25390625	59.289395682484\\
0.25439453125	59.3625649303453\\
0.2548828125	59.4367683295735\\
0.25537109375	59.5120080033573\\
0.255859375	59.5882859352442\\
0.25634765625	59.6656039549638\\
0.2568359375	59.7439637234094\\
0.25732421875	59.8233667167382\\
0.2578125	59.903814209529\\
0.25830078125	59.9853072569567\\
0.2587890625	60.0678466759196\\
0.25927734375	60.1514330250678\\
0.259765625	60.2360665836662\\
0.26025390625	60.3217473292346\\
0.2607421875	60.4084749138958\\
0.26123046875	60.49624863936\\
0.26171875	60.585067430477\\
0.26220703125	60.6749298072779\\
0.2626953125	60.7658338554291\\
0.26318359375	60.8577771950119\\
0.263671875	60.9507569475505\\
0.26416015625	61.0447697011892\\
0.2646484375	61.1398114739386\\
0.26513671875	61.2358776748857\\
0.265625	61.3329630632729\\
0.26611328125	61.4310617053532\\
0.2666015625	61.5301669289019\\
0.26708984375	61.6302712752938\\
0.267578125	61.7313664490329\\
0.26806640625	61.8334432646182\\
0.2685546875	61.9364915906469\\
0.26904296875	62.0405002910345\\
0.26953125	62.1454571632405\\
0.27001953125	62.251348873393\\
0.2705078125	62.3581608882002\\
0.27099609375	62.465877403539\\
0.271484375	62.5744812696268\\
0.27197265625	62.683953912665\\
0.2724609375	62.7942752528762\\
0.27294921875	62.9054236188472\\
0.2734375	63.0173756581017\\
0.27392578125	63.1301062438528\\
0.2744140625	63.2435883778833\\
0.27490234375	63.3577930895303\\
0.275390625	63.472689330774\\
0.27587890625	63.588243867446\\
0.2763671875	63.7044211666086\\
0.27685546875	63.8211832801823\\
0.27734375	63.9384897249546\\
0.27783203125	64.0562973591117\\
0.2783203125	64.1745602555222\\
0.27880859375	64.2932295720218\\
0.279296875	64.4122534190369\\
0.27978515625	64.5315767249247\\
0.2802734375	64.6511410995018\\
0.28076171875	64.770884696308\\
0.28125	64.8907420742344\\
0.28173828125	65.0106440592536\\
0.2822265625	65.130517607086\\
0.28271484375	65.2502856677473\\
0.283203125	65.3698670530571\\
0.28369140625	65.4891763082902\\
0.2841796875	65.6081235893229\\
0.28466796875	65.7266145467234\\
0.28515625	65.8445502184291\\
0.28564453125	65.9618269327539\\
0.2861328125	66.0783362236638\\
0.28662109375	66.19396476037\\
0.287109375	66.3085942934698\\
0.28759765625	66.4221016199673\\
0.2880859375	66.5343585696774\\
0.28857421875	66.6452320155902\\
0.2890625	66.7545839108967\\
0.28955078125	66.8622713554348\\
0.2900390625	66.96814669435\\
0.29052734375	67.072057651783\\
0.291015625	67.1738475023393\\
0.29150390625	67.2733552830206\\
0.2919921875	67.3704160481374\\
0.29248046875	67.4648611695351\\
0.29296875	67.5565186841644\\
0.29345703125	67.6452136907006\\
0.2939453125	67.7307687964728\\
0.29443359375	67.8130046154683\\
0.294921875	67.891740317592\\
0.29541015625	67.9667942286812\\
0.2958984375	68.0379844800655\\
0.29638671875	68.1051297056263\\
0.296875	68.1680497834619\\
0.29736328125	68.2265666183636\\
0.2978515625	68.2805049603431\\
0.29833984375	68.3296932535599\\
0.298828125	68.3739645090057\\
0.29931640625	68.4131571934678\\
0.2998046875	68.4471161264373\\
0.30029296875	68.4756933759151\\
0.30078125	68.4987491434519\\
0.30126953125	68.5161526283009\\
0.3017578125	68.5277828602785\\
0.30224609375	68.5335294908267\\
0.302734375	68.5332935318892\\
0.30322265625	68.5269880325543\\
0.3037109375	68.5145386839598\\
0.30419921875	68.4958843437553\\
0.3046875	68.4709774723919\\
0.30517578125	68.4397844746932\\
0.3056640625	68.4022859415147\\
0.30615234375	68.3584767877735\\
0.306640625	68.3083662847059\\
0.30712890625	68.2519779858664\\
0.3076171875	68.1893495479911\\
0.30810546875	68.1205324495177\\
0.30859375	68.0455916110497\\
0.30908203125	67.9646049235242\\
0.3095703125	67.8776626911011\\
0.31005859375	67.7848669969199\\
0.310546875	67.6863310007609\\
0.31103515625	67.5821781783612\\
0.3115234375	67.4725415125885\\
0.31201171875	67.3575626469307\\
0.3125	67.2373910118011\\
0.31298828125	67.1121829339608\\
0.3134765625	66.9821007390363\\
0.31396484375	66.8473118565536\\
0.314453125	66.7079879362825\\
0.31494140625	66.5643039838891\\
0.3154296875	66.4164375230642\\
0.31591796875	66.2645677903718\\
0.31640625	66.1088749681344\\
0.31689453125	65.9495394597194\\
0.3173828125	65.7867412106785\\
0.31787109375	65.6206590782827\\
0.318359375	65.4514702511713\\
0.31884765625	65.2793497200353\\
0.3193359375	65.1044697995586\\
0.31982421875	64.9269997012001\\
0.3203125	64.7471051558472\\
0.32080078125	64.5649480848902\\
0.3212890625	64.3806863178926\\
0.32177734375	64.1944733546748\\
0.322265625	64.006458169422\\
0.32275390625	63.8167850542113\\
0.3232421875	63.6255934992514\\
0.32373046875	63.4330181070411\\
0.32421875	63.2391885376516\\
0.32470703125	63.0442294823332\\
0.3251953125	62.8482606627173\\
0.32568359375	62.6513968529547\\
0.326171875	62.4537479222343\\
0.32666015625	62.2554188952599\\
0.3271484375	62.056510028374\\
0.32763671875	61.8571168991785\\
0.328125	61.6573305076477\\
0.32861328125	61.4572373868663\\
0.3291015625	61.2569197216882\\
0.32958984375	61.0564554737611\\
0.330078125	60.8559185114873\\
0.33056640625	60.6553787436578\\
0.3310546875	60.4549022555973\\
0.33154296875	60.2545514468131\\
0.33203125	60.0543851692329\\
0.33251953125	59.8544588652449\\
0.3330078125	59.6548247048398\\
0.33349609375	59.455531721264\\
0.333984375	59.2566259446662\\
0.33447265625	59.0581505332996\\
0.3349609375	58.8601459019223\\
0.33544921875	58.6626498470926\\
0.3359375	58.4656976691176\\
0.33642578125	58.2693222904634\\
0.3369140625	58.0735543704857\\
0.33740234375	57.878422416373\\
0.337890625	57.6839528902355\\
0.33837890625	57.4901703123012\\
0.3388671875	57.2970973602113\\
0.33935546875	57.1047549644238\\
0.33984375	56.9131623997633\\
0.34033203125	56.7223373731636\\
0.3408203125	56.5322961076671\\
0.34130859375	56.3430534227616\\
0.341796875	56.1546228111357\\
0.34228515625	55.9670165119482\\
0.3427734375	55.7802455807148\\
0.34326171875	55.5943199559087\\
0.34375	55.409248522393\\
0.34423828125	55.2250391717874\\
0.3447265625	55.0416988598833\\
0.34521484375	54.8592336612203\\
0.345703125	54.6776488209334\\
0.34619140625	54.4969488039817\\
0.3466796875	54.3171373418703\\
0.34716796875	54.1382174769673\\
0.34765625	53.960191604528\\
0.34814453125	53.7830615125211\\
0.3486328125	53.6068284193622\\
0.34912109375	53.4314930096495\\
0.349609375	53.257055467993\\
0.35009765625	53.0835155110338\\
0.3505859375	52.9108724177333\\
0.35107421875	52.7391250580212\\
0.3515625	52.5682719198843\\
0.35205078125	52.3983111349672\\
0.3525390625	52.2292405027683\\
0.35302734375	52.0610575134932\\
0.353515625	51.8937593696412\\
0.35400390625	51.7273430063867\\
0.3544921875	51.5618051108164\\
0.35498046875	51.3971421400849\\
0.35546875	51.2333503385428\\
0.35595703125	51.0704257538933\\
0.3564453125	50.9083642524259\\
0.35693359375	50.747161533379\\
0.357421875	50.5868131424756\\
0.35791015625	50.4273144846782\\
0.3583984375	50.2686608362021\\
0.35888671875	50.1108473558298\\
0.359375	49.9538690955634\\
0.35986328125	49.7977210106492\\
0.3603515625	49.6423979690108\\
0.36083984375	49.4878947601218\\
0.361328125	49.3342061033493\\
0.36181640625	49.1813266557956\\
0.3623046875	49.0292510196686\\
0.36279296875	48.8779737492028\\
0.36328125	48.7274893571596\\
0.36376953125	48.5777923209276\\
0.3642578125	48.4288770882461\\
0.36474609375	48.280738082571\\
0.365234375	48.1333697081072\\
0.36572265625	47.9867663545199\\
0.3662109375	47.8409224013478\\
0.36669921875	47.6958322221325\\
0.3671875	47.551490188281\\
0.36767578125	47.4078906726741\\
0.3681640625	47.2650280530389\\
0.36865234375	47.1228967150941\\
0.369140625	46.9814910554843\\
0.36962890625	46.8408054845142\\
0.3701171875	46.7008344286931\\
0.37060546875	46.5615723331018\\
0.37109375	46.4230136635913\\
0.37158203125	46.2851529088225\\
0.3720703125	46.1479845821571\\
0.37255859375	46.0115032234072\\
0.373046875	45.8757034004529\\
0.37353515625	45.7405797107328\\
0.3740234375	45.6061267826186\\
0.37451171875	45.4723392766778\\
0.375	45.3392118868308\\
0.37548828125	45.2067393414113\\
0.3759765625	45.0749164041318\\
0.37646484375	44.9437378749639\\
0.376953125	44.8131985909336\\
0.37744140625	44.6832934268426\\
0.3779296875	44.5540172959135\\
0.37841796875	44.4253651503689\\
0.37890625	44.2973319819454\\
0.37939453125	44.1699128223462\\
0.3798828125	44.0431027436387\\
0.38037109375	43.9168968585976\\
0.380859375	43.7912903209978\\
0.38134765625	43.6662783258606\\
0.3818359375	43.5418561096565\\
0.38232421875	43.4180189504652\\
0.3828125	43.2947621680983\\
0.38330078125	43.1720811241849\\
0.3837890625	43.0499712222238\\
0.38427734375	42.928427907603\\
0.384765625	42.8074466675909\\
0.38525390625	42.6870230312988\\
0.3857421875	42.5671525696176\\
0.38623046875	42.4478308951307\\
0.38671875	42.3290536620045\\
0.38720703125	42.210816565857\\
0.3876953125	42.0931153436084\\
0.38818359375	41.9759457733125\\
0.388671875	41.8593036739716\\
0.38916015625	41.7431849053362\\
0.3896484375	41.6275853676895\\
0.39013671875	41.5125010016197\\
0.390625	41.3979277877791\\
0.39111328125	41.2838617466323\\
0.3916015625	41.170298938194\\
0.39208984375	41.0572354617573\\
0.392578125	40.9446674556137\\
0.39306640625	40.8325910967638\\
0.3935546875	40.7210026006227\\
0.39404296875	40.6098982207173\\
0.39453125	40.4992742483789\\
0.39501953125	40.3891270124297\\
0.3955078125	40.2794528788653\\
0.39599609375	40.1702482505317\\
0.396484375	40.0615095668005\\
0.39697265625	39.9532333032397\\
0.3974609375	39.8454159712818\\
0.39794921875	39.7380541178903\\
0.3984375	39.6311443252232\\
0.39892578125	39.5246832102956\\
0.3994140625	39.4186674246412\\
0.39990234375	39.3130936539724\\
0.400390625	39.2079586178401\\
0.40087890625	39.1032590692934\\
0.4013671875	38.9989917945381\\
0.40185546875	38.8951536125976\\
0.40234375	38.7917413749714\\
0.40283203125	38.6887519652968\\
0.4033203125	38.5861822990097\\
0.40380859375	38.4840293230076\\
0.404296875	38.3822900153131\\
0.40478515625	38.2809613847382\\
0.4052734375	38.1800404705523\\
0.40576171875	38.079524342149\\
0.40625	37.9794100987172\\
0.40673828125	37.879694868912\\
0.4072265625	37.7803758105297\\
0.40771484375	37.6814501101827\\
0.408203125	37.582914982979\\
0.40869140625	37.4847676722016\\
0.4091796875	37.3870054489919\\
0.40966796875	37.2896256120353\\
0.41015625	37.1926254872487\\
0.41064453125	37.0960024274712\\
0.4111328125	36.9997538121573\\
0.41162109375	36.9038770470719\\
0.412109375	36.8083695639901\\
0.41259765625	36.7132288203972\\
0.4130859375	36.6184522991934\\
0.41357421875	36.5240375084002\\
0.4140625	36.4299819808711\\
0.41455078125	36.3362832740027\\
0.4150390625	36.242938969451\\
0.41552734375	36.1499466728498\\
0.416015625	36.0573040135319\\
0.41650390625	35.9650086442528\\
0.4169921875	35.8730582409185\\
0.41748046875	35.7814505023151\\
0.41796875	35.6901831498414\\
0.41845703125	35.5992539272461\\
0.4189453125	35.5086606003652\\
0.41943359375	35.4184009568645\\
0.419921875	35.3284728059839\\
0.42041015625	35.2388739782855\\
0.4208984375	35.1496023254032\\
0.42138671875	35.060655719797\\
0.421875	34.9720320545086\\
0.42236328125	34.8837292429204\\
0.4228515625	34.7957452185181\\
0.42333984375	34.7080779346544\\
0.423828125	34.6207253643176\\
0.42431640625	34.5336854999009\\
0.4248046875	34.4469563529765\\
0.42529296875	34.3605359540707\\
0.42578125	34.2744223524426\\
0.42626953125	34.1886136158657\\
0.4267578125	34.1031078304115\\
0.42724609375	34.0179031002362\\
0.427734375	33.9329975473699\\
0.42822265625	33.8483893115086\\
0.4287109375	33.7640765498085\\
0.42919921875	33.6800574366832\\
0.4296875	33.596330163603\\
0.43017578125	33.5128929388967\\
0.4306640625	33.429743987557\\
0.43115234375	33.3468815510468\\
0.431640625	33.2643038871086\\
0.43212890625	33.1820092695769\\
0.4326171875	33.0999959881917\\
0.43310546875	33.0182623484157\\
0.43359375	32.936806671253\\
0.43408203125	32.8556272930702\\
0.4345703125	32.77472256542\\
0.43505859375	32.6940908548671\\
0.435546875	32.6137305428158\\
0.43603515625	32.5336400253407\\
0.4365234375	32.4538177130184\\
0.43701171875	32.3742620307626\\
0.4375	32.2949714176607\\
0.43798828125	32.215944326812\\
0.4384765625	32.1371792251687\\
0.43896484375	32.0586745933793\\
0.439453125	31.9804289256325\\
0.43994140625	31.9024407295049\\
0.4404296875	31.8247085258095\\
0.44091796875	31.7472308484461\\
0.44140625	31.6700062442545\\
0.44189453125	31.5930332728691\\
0.4423828125	31.5163105065751\\
0.44287109375	31.4398365301664\\
0.443359375	31.3636099408069\\
0.44384765625	31.2876293478911\\
0.4443359375	31.2118933729088\\
0.44482421875	31.13640064931\\
0.4453125	31.0611498223722\\
0.44580078125	30.9861395490695\\
0.4462890625	30.9113684979428\\
0.44677734375	30.8368353489723\\
0.447265625	30.7625387934514\\
0.44775390625	30.6884775338618\\
0.4482421875	30.6146502837512\\
0.44873046875	30.5410557676115\\
0.44921875	30.4676927207591\\
0.44970703125	30.394559889217\\
0.4501953125	30.3216560295972\\
0.45068359375	30.2489799089864\\
0.451171875	30.1765303048318\\
0.45166015625	30.1043060048288\\
0.4521484375	30.0323058068098\\
0.45263671875	29.9605285186355\\
0.453125	29.8889729580858\\
0.45361328125	29.817637952754\\
0.4541015625	29.7465223399408\\
0.45458984375	29.6756249665507\\
0.455078125	29.6049446889887\\
0.45556640625	29.5344803730597\\
0.4560546875	29.4642308938679\\
0.45654296875	29.3941951357176\\
0.45703125	29.3243719920163\\
0.45751953125	29.2547603651779\\
0.4580078125	29.1853591665275\\
0.45849609375	29.1161673162075\\
0.458984375	29.047183743085\\
0.45947265625	28.9784073846597\\
0.4599609375	28.9098371869739\\
0.46044921875	28.8414721045229\\
0.4609375	28.7733111001664\\
0.46142578125	28.7053531450422\\
0.4619140625	28.6375972184795\\
0.46240234375	28.5700423079139\\
0.462890625	28.502687408804\\
0.46337890625	28.4355315245476\\
0.4638671875	28.368573666401\\
0.46435546875	28.3018128533972\\
0.46484375	28.2352481122665\\
0.46533203125	28.1688784773572\\
0.4658203125	28.1027029905584\\
0.46630859375	28.0367207012224\\
0.466796875	27.9709306660892\\
0.46728515625	27.905331949211\\
0.4677734375	27.8399236218783\\
0.46826171875	27.7747047625467\\
0.46875	27.7096744567644\\
0.46923828125	27.6448317971006\\
0.4697265625	27.5801758830756\\
0.47021484375	27.5157058210901\\
0.470703125	27.451420724357\\
0.47119140625	27.3873197128329\\
0.4716796875	27.3234019131514\\
0.47216796875	27.259666458556\\
0.47265625	27.1961124888352\\
0.47314453125	27.1327391502568\\
0.4736328125	27.0695455955048\\
0.47412109375	27.0065309836154\\
0.474609375	26.9436944799147\\
0.47509765625	26.881035255957\\
0.4755859375	26.8185524894635\\
0.47607421875	26.7562453642625\\
0.4765625	26.6941130702289\\
0.47705078125	26.6321548032264\\
0.4775390625	26.5703697650485\\
0.47802734375	26.5087571633612\\
0.478515625	26.4473162116463\\
0.47900390625	26.3860461291449\\
0.4794921875	26.3249461408026\\
0.47998046875	26.2640154772139\\
0.48046875	26.2032533745681\\
0.48095703125	26.1426590745963\\
0.4814453125	26.0822318245179\\
0.48193359375	26.0219708769884\\
0.482421875	25.9618754900477\\
0.48291015625	25.9019449270692\\
0.4833984375	25.8421784567089\\
0.48388671875	25.7825753528556\\
0.484375	25.7231348945818\\
0.48486328125	25.6638563660944\\
0.4853515625	25.604739056687\\
0.48583984375	25.5457822606918\\
0.486328125	25.4869852774329\\
0.48681640625	25.4283474111791\\
0.4873046875	25.3698679710986\\
0.48779296875	25.3115462712126\\
0.48828125	25.2533816303514\\
0.48876953125	25.1953733721087\\
0.4892578125	25.1375208247986\\
0.48974609375	25.0798233214112\\
0.490234375	25.0222801995706\\
0.49072265625	24.9648908014915\\
0.4912109375	24.9076544739377\\
0.49169921875	24.8505705681803\\
0.4921875	24.7936384399568\\
0.49267578125	24.73685744943\\
0.4931640625	24.6802269611486\\
0.49365234375	24.6237463440065\\
0.494140625	24.5674149712045\\
0.49462890625	24.5112322202108\\
0.4951171875	24.4551974727225\\
0.49560546875	24.3993101146283\\
0.49609375	24.3435695359701\\
0.49658203125	24.2879751309065\\
0.4970703125	24.2325262976756\\
0.49755859375	24.1772224385589\\
0.498046875	24.1220629598453\\
0.49853515625	24.0670472717955\\
0.4990234375	24.0121747886068\\
0.49951171875	23.9574449283785\\
0.5	23.9028571130775\\
0.50048828125	23.8484107685039\\
0.5009765625	23.7941053242579\\
0.50146484375	23.739940213706\\
0.501953125	23.6859148739484\\
0.50244140625	23.6320287457865\\
0.5029296875	23.5782812736901\\
0.50341796875	23.5246719057663\\
0.50390625	23.4712000937274\\
0.50439453125	23.4178652928601\\
0.5048828125	23.3646669619943\\
0.50537109375	23.3116045634728\\
0.505859375	23.2586775631209\\
0.50634765625	23.2058854302168\\
0.5068359375	23.1532276374616\\
0.50732421875	23.1007036609508\\
0.5078125	23.0483129801444\\
0.50830078125	22.996055077839\\
0.5087890625	22.943929440139\\
0.50927734375	22.8919355564291\\
0.509765625	22.840072919346\\
0.51025390625	22.7883410247511\\
0.5107421875	22.7367393717038\\
0.51123046875	22.6852674624338\\
0.51171875	22.6339248023153\\
0.51220703125	22.5827108998401\\
0.5126953125	22.531625266592\\
0.51318359375	22.4806674172205\\
0.513671875	22.429836869416\\
0.51416015625	22.3791331438837\\
0.5146484375	22.3285557643195\\
0.51513671875	22.2781042573845\\
0.515625	22.227778152681\\
0.51611328125	22.1775769827283\\
0.5166015625	22.1275002829383\\
0.51708984375	22.0775475915922\\
0.517578125	22.0277184498168\\
0.51806640625	21.9780124015613\\
0.5185546875	21.9284289935743\\
0.51904296875	21.878967775381\\
0.51953125	21.8296282992608\\
0.52001953125	21.7804101202246\\
0.5205078125	21.7313127959932\\
0.52099609375	21.6823358869751\\
0.521484375	21.6334789562449\\
0.52197265625	21.5847415695219\\
0.5224609375	21.5361232951489\\
0.52294921875	21.4876237040711\\
0.5234375	21.4392423698156\\
0.52392578125	21.3909788684703\\
0.5244140625	21.3428327786641\\
0.52490234375	21.294803681546\\
0.525390625	21.2468911607661\\
0.52587890625	21.1990948024547\\
0.5263671875	21.1514141952036\\
0.52685546875	21.1038489300463\\
0.52734375	21.0563986004387\\
0.52783203125	21.0090628022405\\
0.5283203125	20.9618411336959\\
0.52880859375	20.9147331954153\\
0.529296875	20.8677385903567\\
0.52978515625	20.8208569238072\\
0.5302734375	20.7740878033654\\
0.53076171875	20.727430838923\\
0.53125	20.6808856426474\\
0.53173828125	20.6344518289636\\
0.5322265625	20.5881290145374\\
0.53271484375	20.5419168182579\\
0.533203125	20.4958148612203\\
0.53369140625	20.449822766709\\
0.5341796875	20.4039401601811\\
0.53466796875	20.3581666692496\\
0.53515625	20.3125019236667\\
0.53564453125	20.2669455553083\\
0.5361328125	20.2214971981572\\
0.53662109375	20.176156488287\\
0.537109375	20.1309230638471\\
0.53759765625	20.0857965650462\\
0.5380859375	20.0407766341374\\
0.53857421875	19.9958629154023\\
0.5390625	19.951055055136\\
0.53955078125	19.9063527016323\\
0.5400390625	19.8617555051682\\
0.54052734375	19.8172631179895\\
0.541015625	19.772875194296\\
0.54150390625	19.728591390227\\
0.5419921875	19.6844113638468\\
0.54248046875	19.6403347751308\\
0.54296875	19.5963612859507\\
0.54345703125	19.5524905600614\\
0.5439453125	19.5087222630861\\
0.54443359375	19.4650560625035\\
0.544921875	19.4214916276337\\
0.54541015625	19.3780286296248\\
0.5458984375	19.3346667414394\\
0.54638671875	19.2914056378418\\
0.546875	19.2482449953843\\
0.54736328125	19.2051844923948\\
0.5478515625	19.1622238089633\\
0.54833984375	19.1193626269296\\
0.548828125	19.0766006298704\\
0.54931640625	19.0339375030872\\
0.5498046875	18.9913729335933\\
0.55029296875	18.9489066101015\\
0.55078125	18.9065382230127\\
0.55126953125	18.8642674644029\\
0.5517578125	18.8220940280117\\
0.55224609375	18.7800176092304\\
0.552734375	18.73803790509\\
0.55322265625	18.69615461425\\
0.5537109375	18.6543674369862\\
0.55419921875	18.6126760751799\\
0.5546875	18.5710802323061\\
0.55517578125	18.5295796134224\\
0.5556640625	18.4881739251582\\
0.55615234375	18.446862875703\\
0.556640625	18.4056461747958\\
0.55712890625	18.3645235337144\\
0.5576171875	18.3234946652643\\
0.55810546875	18.2825592837682\\
0.55859375	18.2417171050558\\
0.55908203125	18.2009678464525\\
0.5595703125	18.1603112267697\\
0.56005859375	18.1197469662943\\
0.560546875	18.0792747867784\\
0.56103515625	18.0388944114294\\
0.5615234375	17.9986055648998\\
0.56201171875	17.9584079732772\\
0.5625	17.9183013640746\\
0.56298828125	17.8782854662209\\
0.5634765625	17.8383600100503\\
0.56396484375	17.7985247272938\\
0.564453125	17.758779351069\\
0.56494140625	17.7191236158708\\
0.5654296875	17.6795572575624\\
0.56591796875	17.6400800133653\\
0.56640625	17.6006916218508\\
0.56689453125	17.5613918229306\\
0.5673828125	17.5221803578476\\
0.56787109375	17.4830569691674\\
0.568359375	17.4440214007689\\
0.56884765625	17.4050733978357\\
0.5693359375	17.3662127068476\\
0.56982421875	17.3274390755716\\
0.5703125	17.2887522530534\\
0.57080078125	17.2501519896089\\
0.5712890625	17.2116380368159\\
0.57177734375	17.1732101475056\\
0.572265625	17.1348680757541\\
0.57275390625	17.0966115768746\\
0.5732421875	17.0584404074088\\
0.57373046875	17.0203543251193\\
0.57421875	16.9823530889809\\
0.57470703125	16.9444364591733\\
0.5751953125	16.9066041970728\\
0.57568359375	16.8688560652446\\
0.576171875	16.831191827435\\
0.57666015625	16.7936112485638\\
0.5771484375	16.7561140947165\\
0.57763671875	16.7187001331369\\
0.578125	16.6813691322194\\
0.57861328125	16.6441208615016\\
0.5791015625	16.6069550916572\\
0.57958984375	16.5698715944881\\
0.580078125	16.5328701429178\\
0.58056640625	16.4959505109835\\
0.5810546875	16.4591124738295\\
0.58154296875	16.4223558076999\\
0.58203125	16.3856802899314\\
0.58251953125	16.3490856989468\\
0.5830078125	16.3125718142474\\
0.58349609375	16.2761384164067\\
0.583984375	16.2397852870634\\
0.58447265625	16.2035122089145\\
0.5849609375	16.1673189657089\\
0.58544921875	16.1312053422405\\
0.5859375	16.0951711243416\\
0.58642578125	16.0592160988765\\
0.5869140625	16.0233400537351\\
0.58740234375	15.9875427778261\\
0.587890625	15.951824061071\\
0.58837890625	15.9161836943974\\
0.5888671875	15.8806214697331\\
0.58935546875	15.8451371799997\\
0.58984375	15.8097306191062\\
0.59033203125	15.7744015819431\\
0.5908203125	15.7391498643764\\
0.59130859375	15.7039752632413\\
0.591796875	15.6688775763364\\
0.59228515625	15.6338566024176\\
0.5927734375	15.5989121411923\\
0.59326171875	15.5640439933136\\
0.59375	15.5292519603744\\
0.59423828125	15.4945358449016\\
0.5947265625	15.4598954503505\\
0.59521484375	15.4253305810991\\
0.595703125	15.3908410424423\\
0.59619140625	15.3564266405866\\
0.5966796875	15.3220871826445\\
0.59716796875	15.2878224766285\\
0.59765625	15.2536323314466\\
0.59814453125	15.219516556896\\
0.5986328125	15.1854749636581\\
0.59912109375	15.1515073632934\\
0.599609375	15.1176135682355\\
0.60009765625	15.0837933917867\\
0.6005859375	15.0500466481123\\
0.60107421875	15.0163731522353\\
0.6015625	14.9827727200319\\
0.60205078125	14.9492451682256\\
0.6025390625	14.9157903143827\\
0.60302734375	14.8824079769072\\
0.603515625	14.8490979750358\\
0.60400390625	14.8158601288326\\
0.6044921875	14.782694259185\\
0.60498046875	14.7496001877979\\
0.60546875	14.7165777371896\\
0.60595703125	14.6836267306867\\
0.6064453125	14.6507469924193\\
0.60693359375	14.6179383473163\\
0.607421875	14.5852006211011\\
0.60791015625	14.5525336402862\\
0.6083984375	14.5199372321694\\
0.60888671875	14.4874112248284\\
0.609375	14.4549554471172\\
0.60986328125	14.4225697286607\\
0.6103515625	14.3902538998508\\
0.61083984375	14.3580077918416\\
0.611328125	14.3258312365452\\
0.61181640625	14.2937240666273\\
0.6123046875	14.2616861155028\\
0.61279296875	14.2297172173315\\
0.61328125	14.1978172070138\\
0.61376953125	14.1659859201865\\
0.6142578125	14.1342231932184\\
0.61474609375	14.1025288632065\\
0.615234375	14.0709027679714\\
0.61572265625	14.0393447460535\\
0.6162109375	14.0078546367085\\
0.61669921875	13.976432279904\\
0.6171875	13.9450775163148\\
0.61767578125	13.9137901873192\\
0.6181640625	13.8825701349949\\
0.61865234375	13.8514172021153\\
0.619140625	13.8203312321452\\
0.61962890625	13.7893120692373\\
0.6201171875	13.7583595582279\\
0.62060546875	13.7274735446335\\
0.62109375	13.6966538746466\\
0.62158203125	13.6659003951323\\
0.6220703125	13.6352129536243\\
0.62255859375	13.604591398321\\
0.623046875	13.5740355780822\\
0.62353515625	13.5435453424253\\
0.6240234375	13.5131205415214\\
0.62451171875	13.482761026192\\
0.625	13.4524666479051\\
0.62548828125	13.422237258772\\
0.6259765625	13.3920727115434\\
0.62646484375	13.361972859606\\
0.626953125	13.3319375569792\\
0.62744140625	13.3019666583112\\
0.6279296875	13.2720600188759\\
0.62841796875	13.2422174945694\\
0.62890625	13.2124389419067\\
0.62939453125	13.1827242180179\\
0.6298828125	13.1530731806454\\
0.63037109375	13.1234856881402\\
0.630859375	13.0939615994587\\
0.63134765625	13.0645007741596\\
0.6318359375	13.0351030724001\\
0.63232421875	13.0057683549335\\
0.6328125	12.9764964831052\\
0.63330078125	12.9472873188499\\
0.6337890625	12.9181407246884\\
0.63427734375	12.8890565637243\\
0.634765625	12.8600346996411\\
0.63525390625	12.8310749966991\\
0.6357421875	12.8021773197318\\
0.63623046875	12.7733415341436\\
0.63671875	12.7445675059062\\
0.63720703125	12.7158551015558\\
0.6376953125	12.6872041881902\\
0.63818359375	12.6586146334655\\
0.638671875	12.6300863055935\\
0.63916015625	12.6016190733384\\
0.6396484375	12.5732128060145\\
0.64013671875	12.5448673734823\\
0.640625	12.5165826461466\\
0.64111328125	12.4883584949531\\
0.6416015625	12.4601947913859\\
0.64208984375	12.4320914074641\\
0.642578125	12.4040482157398\\
0.64306640625	12.3760650892947\\
0.6435546875	12.3481419017375\\
0.64404296875	12.3202785272012\\
0.64453125	12.2924748403405\\
0.64501953125	12.2647307163289\\
0.6455078125	12.237046030856\\
0.64599609375	12.209420660125\\
0.646484375	12.1818544808498\\
0.64697265625	12.1543473702528\\
0.6474609375	12.1268992060616\\
0.64794921875	12.0995098665071\\
0.6484375	12.0721792303205\\
0.64892578125	12.0449071767309\\
0.6494140625	12.0176935854625\\
0.64990234375	11.9905383367325\\
0.650390625	11.9634413112481\\
0.65087890625	11.9364023902046\\
0.6513671875	11.9094214552822\\
0.65185546875	11.8824983886441\\
0.65234375	11.8556330729339\\
0.65283203125	11.8288253912728\\
0.6533203125	11.8020752272578\\
0.65380859375	11.775382464959\\
0.654296875	11.7487469889171\\
0.65478515625	11.7221686841411\\
0.6552734375	11.695647436106\\
0.65576171875	11.6691831307506\\
0.65625	11.6427756544749\\
0.65673828125	11.6164248941378\\
0.6572265625	11.5901307370553\\
0.65771484375	11.5638930709973\\
0.658203125	11.5377117841864\\
0.65869140625	11.5115867652949\\
0.6591796875	11.4855179034426\\
0.65966796875	11.4595050881951\\
0.66015625	11.4335482095611\\
0.66064453125	11.4076471579904\\
0.6611328125	11.3818018243718\\
0.66162109375	11.3560121000306\\
0.662109375	11.330277876727\\
0.66259765625	11.3045990466533\\
0.6630859375	11.2789755024324\\
0.66357421875	11.2534071371154\\
0.6640625	11.2278938441793\\
0.66455078125	11.2024355175253\\
0.6650390625	11.1770320514767\\
0.66552734375	11.1516833407765\\
0.666015625	11.1263892805857\\
0.66650390625	11.101149766481\\
0.6669921875	11.0759646944532\\
0.66748046875	11.0508339609048\\
0.66796875	11.025757462648\\
0.66845703125	11.0007350969029\\
0.6689453125	10.9757667612958\\
0.66943359375	10.9508523538566\\
0.669921875	10.9259917730172\\
0.67041015625	10.9011849176099\\
0.6708984375	10.8764316868649\\
0.67138671875	10.8517319804085\\
0.671875	10.8270856982618\\
0.67236328125	10.8024927408381\\
0.6728515625	10.7779530089412\\
0.67333984375	10.753466403764\\
0.673828125	10.729032826886\\
0.67431640625	10.7046521802718\\
0.6748046875	10.6803243662696\\
0.67529296875	10.6560492876085\\
0.67578125	10.6318268473977\\
0.67626953125	10.6076569491241\\
0.6767578125	10.5835394966506\\
0.67724609375	10.5594743942144\\
0.677734375	10.5354615464253\\
0.67822265625	10.5115008582641\\
0.6787109375	10.4875922350804\\
0.67919921875	10.4637355825912\\
0.6796875	10.4399308068792\\
0.68017578125	10.4161778143911\\
0.6806640625	10.3924765119357\\
0.68115234375	10.3688268066825\\
0.681640625	10.3452286061598\\
0.68212890625	10.3216818182532\\
0.6826171875	10.2981863512037\\
0.68310546875	10.2747421136065\\
0.68359375	10.2513490144091\\
0.68408203125	10.2280069629094\\
0.6845703125	10.2047158687546\\
0.68505859375	10.1814756419394\\
0.685546875	10.1582861928043\\
0.68603515625	10.1351474320342\\
0.6865234375	10.1120592706566\\
0.68701171875	10.0890216200403\\
0.6875	10.0660343918936\\
0.68798828125	10.043097498263\\
0.6884765625	10.0202108515315\\
0.68896484375	9.99737436441714\\
0.689453125	9.97458794997128\\
0.68994140625	9.95185152157754\\
0.6904296875	9.92916499294989\\
0.69091796875	9.90652827813134\\
0.69140625	9.88394129149241\\
0.69189453125	9.86140394772972\\
0.6923828125	9.83891616186442\\
0.69287109375	9.81647784924091\\
0.693359375	9.79408892552517\\
0.69384765625	9.77174930670356\\
0.6943359375	9.74945890908117\\
0.69482421875	9.72721764928053\\
0.6953125	9.70502544424027\\
0.69580078125	9.6828822112135\\
0.6962890625	9.6607878677666\\
0.69677734375	9.6387423317778\\
0.697265625	9.61674552143574\\
0.69775390625	9.59479735523814\\
0.6982421875	9.57289775199044\\
0.69873046875	9.55104663080446\\
0.69921875	9.52924391109702\\
0.69970703125	9.50748951258869\\
0.7001953125	9.48578335530229\\
0.70068359375	9.46412535956173\\
0.701171875	9.44251544599063\\
0.70166015625	9.42095353551106\\
0.7021484375	9.39943954934214\\
0.70263671875	9.37797340899883\\
0.703125	9.35655503629067\\
0.70361328125	9.33518435332045\\
0.7041015625	9.31386128248294\\
0.70458984375	9.29258574646365\\
0.705078125	9.27135766823758\\
0.70556640625	9.25017697106797\\
0.7060546875	9.22904357850506\\
0.70654296875	9.20795741438483\\
0.70703125	9.18691840282782\\
0.70751953125	9.16592646823795\\
0.7080078125	9.1449815353011\\
0.70849609375	9.12408352898424\\
0.708984375	9.10323237453394\\
0.70947265625	9.08242799747533\\
0.7099609375	9.0616703236109\\
0.71044921875	9.04095927901925\\
0.7109375	9.02029479005404\\
0.71142578125	8.99967678334271\\
0.7119140625	8.9791051857854\\
0.71240234375	8.9585799245538\\
0.712890625	8.93810092708994\\
0.71337890625	8.91766812110515\\
0.7138671875	8.89728143457886\\
0.71435546875	8.87694079575749\\
0.71484375	8.85664613315333\\
0.71533203125	8.83639737554348\\
0.7158203125	8.81619445196867\\
0.71630859375	8.7960372917322\\
0.716796875	8.7759258243989\\
0.71728515625	8.75585997979391\\
0.7177734375	8.73583968800174\\
0.71826171875	8.71586487936514\\
0.71875	8.69593548448399\\
0.71923828125	8.67605143421434\\
0.7197265625	8.65621265966726\\
0.72021484375	8.63641909220783\\
0.720703125	8.61667066345411\\
0.72119140625	8.59696730527611\\
0.7216796875	8.57730894979465\\
0.72216796875	8.55769552938051\\
0.72265625	8.5381269766533\\
0.72314453125	8.51860322448045\\
0.7236328125	8.49912420597627\\
0.72412109375	8.4796898545008\\
0.724609375	8.46030010365901\\
0.72509765625	8.44095488729965\\
0.7255859375	8.42165413951434\\
0.72607421875	8.40239779463659\\
0.7265625	8.38318578724078\\
0.72705078125	8.36401805214124\\
0.7275390625	8.34489452439121\\
0.72802734375	8.32581513928205\\
0.728515625	8.30677983234204\\
0.72900390625	8.28778853933566\\
0.7294921875	8.26884119626247\\
0.72998046875	8.24993773935636\\
0.73046875	8.23107810508443\\
0.73095703125	8.21226223014615\\
0.7314453125	8.19349005147243\\
0.73193359375	8.17476150622476\\
0.732421875	8.15607653179419\\
0.73291015625	8.13743506580047\\
0.7333984375	8.11883704609119\\
0.73388671875	8.10028241074084\\
0.734375	8.08177109804988\\
0.73486328125	8.06330304654397\\
0.7353515625	8.04487819497296\\
0.73583984375	8.02649648231007\\
0.736328125	8.00815784775108\\
0.73681640625	7.98986223071332\\
0.7373046875	7.9716095708349\\
0.73779296875	7.95339980797389\\
0.73828125	7.93523288220735\\
0.73876953125	7.91710873383054\\
0.7392578125	7.89902730335617\\
0.73974609375	7.8809885315134\\
0.740234375	7.86299235924703\\
0.74072265625	7.84503872771681\\
0.7412109375	7.82712757829644\\
0.74169921875	7.80925885257289\\
0.7421875	7.79143249234549\\
0.74267578125	7.77364843962514\\
0.7431640625	7.75590663663351\\
0.74365234375	7.73820702580224\\
0.744140625	7.72054954977216\\
0.74462890625	7.70293415139239\\
0.7451171875	7.68536077371973\\
0.74560546875	7.66782936001769\\
0.74609375	7.65033985375588\\
0.74658203125	7.63289219860901\\
0.7470703125	7.61548633845635\\
0.74755859375	7.59812221738086\\
0.748046875	7.58079977966836\\
0.74853515625	7.56351896980688\\
0.7490234375	7.5462797324858\\
0.74951171875	7.52908201259523\\
0.75	7.51192575522509\\
0.75048828125	7.49481090566451\\
0.7509765625	7.47773740940105\\
0.75146484375	7.4607052121199\\
0.751953125	7.44371425970318\\
0.75244140625	7.42676449822928\\
0.7529296875	7.40985587397203\\
0.75341796875	7.39298833340006\\
0.75390625	7.37616182317606\\
0.75439453125	7.35937629015595\\
0.7548828125	7.3426316813884\\
0.75537109375	7.32592794411395\\
0.755859375	7.30926502576432\\
0.75634765625	7.2926428739618\\
0.7568359375	7.27606143651843\\
0.75732421875	7.25952066143544\\
0.7578125	7.2430204969025\\
0.75830078125	7.22656089129702\\
0.7587890625	7.21014179318342\\
0.75927734375	7.19376315131266\\
0.759765625	7.17742491462129\\
0.76025390625	7.16112703223097\\
0.7607421875	7.14486945344775\\
0.76123046875	7.12865212776142\\
0.76171875	7.11247500484481\\
0.76220703125	7.09633803455317\\
0.7626953125	7.08024116692347\\
0.76318359375	7.06418435217384\\
0.763671875	7.04816754070287\\
0.76416015625	7.03219068308895\\
0.7646484375	7.01625373008966\\
0.76513671875	7.00035663264113\\
0.765625	6.98449934185738\\
0.76611328125	6.96868180902977\\
0.7666015625	6.95290398562624\\
0.76708984375	6.93716582329083\\
0.767578125	6.92146727384301\\
0.76806640625	6.905808289277\\
0.7685546875	6.89018882176124\\
0.76904296875	6.87460882363773\\
0.76953125	6.85906824742148\\
0.77001953125	6.84356704579989\\
0.7705078125	6.82810517163206\\
0.77099609375	6.81268257794828\\
0.771484375	6.7972992179495\\
0.77197265625	6.78195504500656\\
0.7724609375	6.7666500126598\\
0.77294921875	6.75138407461829\\
0.7734375	6.73615718475941\\
0.77392578125	6.72096929712815\\
0.7744140625	6.70582036593665\\
0.77490234375	6.69071034556348\\
0.775390625	6.67563919055318\\
0.77587890625	6.66060685561572\\
0.7763671875	6.64561329562583\\
0.77685546875	6.63065846562254\\
0.77734375	6.61574232080849\\
0.77783203125	6.60086481654953\\
0.7783203125	6.58602590837411\\
0.77880859375	6.5712255519727\\
0.779296875	6.55646370319723\\
0.77978515625	6.54174031806062\\
0.7802734375	6.5270553527362\\
0.78076171875	6.51240876355715\\
0.78125	6.49780050701601\\
0.78173828125	6.48323053976412\\
0.7822265625	6.46869881861109\\
0.78271484375	6.45420530052431\\
0.783203125	6.43974994262832\\
0.78369140625	6.42533270220442\\
0.7841796875	6.41095353669013\\
0.78466796875	6.39661240367855\\
0.78515625	6.38230926091801\\
0.78564453125	6.36804406631143\\
0.7861328125	6.35381677791595\\
0.78662109375	6.33962735394224\\
0.787109375	6.32547575275417\\
0.78759765625	6.31136193286822\\
0.7880859375	6.29728585295296\\
0.78857421875	6.28324747182869\\
0.7890625	6.26924674846676\\
0.78955078125	6.25528364198925\\
0.7900390625	6.24135811166833\\
0.79052734375	6.2274701169259\\
0.791015625	6.21361961733304\\
0.79150390625	6.19980657260958\\
0.7919921875	6.18603094262355\\
0.79248046875	6.17229268739076\\
0.79296875	6.15859176707435\\
0.79345703125	6.14492814198419\\
0.7939453125	6.13130177257661\\
0.79443359375	6.11771261945376\\
0.794921875	6.10416064336323\\
0.79541015625	6.0906458051976\\
0.7958984375	6.07716806599396\\
0.79638671875	6.06372738693335\\
0.796875	6.05032372934057\\
0.79736328125	6.03695705468346\\
0.7978515625	6.02362732457254\\
0.79833984375	6.01033450076064\\
0.798828125	5.99707854514236\\
0.79931640625	5.98385941975367\\
0.7998046875	5.9706770867715\\
0.80029296875	5.9575315085132\\
0.80078125	5.94442264743621\\
0.80126953125	5.93135046613754\\
0.8017578125	5.91831492735342\\
0.80224609375	5.90531599395889\\
0.802734375	5.89235362896721\\
0.80322265625	5.87942779552959\\
0.8037109375	5.86653845693474\\
0.80419921875	5.8536855766084\\
0.8046875	5.84086911811298\\
0.80517578125	5.82808904514708\\
0.8056640625	5.81534532154515\\
0.80615234375	5.80263791127703\\
0.806640625	5.78996677844754\\
0.80712890625	5.7773318872961\\
0.8076171875	5.7647332021963\\
0.80810546875	5.75217068765546\\
0.80859375	5.73964430831436\\
0.80908203125	5.72715402894667\\
0.8095703125	5.7146998144587\\
0.81005859375	5.7022816298889\\
0.810546875	5.6898994404075\\
0.81103515625	5.67755321131617\\
0.8115234375	5.66524290804754\\
0.81201171875	5.6529684961649\\
0.8125	5.64072994136173\\
0.81298828125	5.62852720946138\\
0.8134765625	5.61636026641665\\
0.81396484375	5.60422907830947\\
0.814453125	5.59213361135046\\
0.81494140625	5.58007383187856\\
0.8154296875	5.56804970636066\\
0.81591796875	5.55606120139128\\
0.81640625	5.54410828369212\\
0.81689453125	5.53219092011178\\
0.8173828125	5.52030907762527\\
0.81787109375	5.50846272333379\\
0.818359375	5.49665182446426\\
0.81884765625	5.48487634836898\\
0.8193359375	5.47313626252535\\
0.81982421875	5.46143153453537\\
0.8203125	5.44976213212542\\
0.82080078125	5.43812802314582\\
0.8212890625	5.42652917557054\\
0.82177734375	5.41496555749681\\
0.822265625	5.40343713714473\\
0.82275390625	5.39194388285707\\
0.8232421875	5.38048576309874\\
0.82373046875	5.36906274645661\\
0.82421875	5.35767480163902\\
0.82470703125	5.34632189747556\\
0.8251953125	5.33500400291668\\
0.82568359375	5.3237210870334\\
0.826171875	5.31247311901687\\
0.82666015625	5.30126006817814\\
0.8271484375	5.29008190394783\\
0.82763671875	5.27893859587569\\
0.828125	5.26783011363043\\
0.82861328125	5.25675642699922\\
0.8291015625	5.24571750588753\\
0.82958984375	5.23471332031871\\
0.830078125	5.22374384043369\\
0.83056640625	5.21280903649069\\
0.8310546875	5.20190887886486\\
0.83154296875	5.19104333804797\\
0.83203125	5.18021238464808\\
0.83251953125	5.16941598938936\\
0.8330078125	5.15865412311157\\
0.83349609375	5.14792675676989\\
0.833984375	5.13723386143455\\
0.83447265625	5.12657540829061\\
0.8349609375	5.11595136863755\\
0.83544921875	5.10536171388897\\
0.8359375	5.09480641557242\\
0.83642578125	5.08428544532893\\
0.8369140625	5.07379877491285\\
0.83740234375	5.06334637619147\\
0.837890625	5.0529282211447\\
0.83837890625	5.04254428186487\\
0.8388671875	5.0321945305564\\
0.83935546875	5.02187893953548\\
0.83984375	5.01159748122978\\
0.84033203125	5.00135012817823\\
0.8408203125	4.99113685303063\\
0.84130859375	4.98095762854744\\
0.841796875	4.97081242759949\\
0.84228515625	4.96070122316769\\
0.8427734375	4.95062398834269\\
0.84326171875	4.94058069632469\\
0.84375	4.93057132042311\\
0.84423828125	4.92059583405638\\
0.8447265625	4.91065421075153\\
0.84521484375	4.90074642414405\\
0.845703125	4.89087244797755\\
0.84619140625	4.88103225610352\\
0.8466796875	4.87122582248108\\
0.84716796875	4.86145312117656\\
0.84765625	4.85171412636347\\
0.84814453125	4.84200881232206\\
0.8486328125	4.83233715343913\\
0.84912109375	4.82269912420781\\
0.849609375	4.81309469922707\\
0.85009765625	4.8035238532018\\
0.8505859375	4.79398656094233\\
0.85107421875	4.78448279736415\\
0.8515625	4.77501253748784\\
0.85205078125	4.76557575643862\\
0.8525390625	4.75617242944623\\
0.85302734375	4.74680253184459\\
0.853515625	4.73746603907161\\
0.85400390625	4.72816292666893\\
0.8544921875	4.71889317028163\\
0.85498046875	4.70965674565802\\
0.85546875	4.70045362864938\\
0.85595703125	4.6912837952098\\
0.8564453125	4.68214722139575\\
0.85693359375	4.67304388336603\\
0.857421875	4.6639737573814\\
0.85791015625	4.65493681980443\\
0.8583984375	4.64593304709919\\
0.85888671875	4.63696241583109\\
0.859375	4.62802490266654\\
0.85986328125	4.61912048437285\\
0.8603515625	4.61024913781786\\
0.86083984375	4.60141083996984\\
0.861328125	4.59260556789711\\
0.86181640625	4.58383329876798\\
0.8623046875	4.57509400985035\\
0.86279296875	4.56638767851167\\
0.86328125	4.55771428221853\\
0.86376953125	4.54907379853654\\
0.8642578125	4.54046620513013\\
0.86474609375	4.53189147976223\\
0.865234375	4.52334960029417\\
0.86572265625	4.51484054468531\\
0.8662109375	4.50636429099298\\
0.86669921875	4.49792081737216\\
0.8671875	4.48951010207528\\
0.86767578125	4.48113212345207\\
0.8681640625	4.47278685994922\\
0.86865234375	4.46447429011032\\
0.869140625	4.45619439257555\\
0.86962890625	4.44794714608145\\
0.8701171875	4.43973252946081\\
0.87060546875	4.43155052164234\\
0.87109375	4.42340110165063\\
0.87158203125	4.41528424860574\\
0.8720703125	4.40719994172315\\
0.87255859375	4.39914816031349\\
0.873046875	4.39112888378235\\
0.87353515625	4.38314209163011\\
0.8740234375	4.37518776345169\\
0.87451171875	4.36726587893634\\
0.875	4.35937641786754\\
0.87548828125	4.35151936012269\\
0.8759765625	4.34369468567299\\
0.87646484375	4.33590237458318\\
0.876953125	4.32814240701142\\
0.87744140625	4.32041476320906\\
0.8779296875	4.31271942352043\\
0.87841796875	4.30505636838267\\
0.87890625	4.29742557832557\\
0.87939453125	4.2898270339713\\
0.8798828125	4.28226071603433\\
0.88037109375	4.27472660532115\\
0.880859375	4.26722468273014\\
0.88134765625	4.25975492925137\\
0.8818359375	4.25231732596643\\
0.88232421875	4.24491185404817\\
0.8828125	4.23753849476068\\
0.88330078125	4.23019722945893\\
0.8837890625	4.22288803958874\\
0.88427734375	4.21561090668646\\
0.884765625	4.20836581237897\\
0.88525390625	4.20115273838331\\
0.8857421875	4.19397166650663\\
0.88623046875	4.18682257864605\\
0.88671875	4.1797054567883\\
0.88720703125	4.17262028300975\\
0.8876953125	4.16556703947616\\
0.88818359375	4.15854570844248\\
0.888671875	4.15155627225276\\
0.88916015625	4.14459871333983\\
0.8896484375	4.13767301422537\\
0.89013671875	4.13077915751947\\
0.890625	4.12391712592074\\
0.89111328125	4.1170869022159\\
0.8916015625	4.1102884692798\\
0.89208984375	4.10352181007513\\
0.892578125	4.09678690765235\\
0.89306640625	4.09008374514952\\
0.8935546875	4.08341230579201\\
0.89404296875	4.07677257289256\\
0.89453125	4.07016452985094\\
0.89501953125	4.06358816015391\\
0.8955078125	4.05704344737497\\
0.89599609375	4.05053037517426\\
0.896484375	4.04404892729843\\
0.89697265625	4.0375990875804\\
0.8974609375	4.03118083993936\\
0.89794921875	4.02479416838041\\
0.8984375	4.01843905699455\\
0.89892578125	4.01211548995855\\
0.8994140625	4.00582345153475\\
0.89990234375	3.99956292607086\\
0.900390625	3.99333389799993\\
0.90087890625	3.9871363518401\\
0.9013671875	3.98097027219451\\
0.90185546875	3.97483564375121\\
0.90234375	3.96873245128286\\
0.90283203125	3.96266067964676\\
0.9033203125	3.95662031378453\\
0.90380859375	3.95061133872223\\
0.904296875	3.94463373956991\\
0.90478515625	3.93868750152168\\
0.9052734375	3.93277260985556\\
0.90576171875	3.9268890499332\\
0.90625	3.92103680719992\\
0.90673828125	3.91521586718451\\
0.9072265625	3.90942621549903\\
0.90771484375	3.90366783783875\\
0.908203125	3.89794071998202\\
0.90869140625	3.89224484779007\\
0.9091796875	3.88658020720699\\
0.90966796875	3.88094678425944\\
0.91015625	3.87534456505676\\
0.91064453125	3.86977353579055\\
0.9111328125	3.86423368273479\\
0.91162109375	3.85872499224559\\
0.912109375	3.85324745076107\\
0.91259765625	3.84780104480128\\
0.9130859375	3.84238576096805\\
0.91357421875	3.83700158594483\\
0.9140625	3.83164850649666\\
0.91455078125	3.82632650946993\\
0.9150390625	3.82103558179243\\
0.91552734375	3.81577571047299\\
0.916015625	3.8105468826016\\
0.91650390625	3.80534908534912\\
0.9169921875	3.80018230596726\\
0.91748046875	3.79504653178846\\
0.91796875	3.78994175022568\\
0.91845703125	3.78486794877238\\
0.9189453125	3.77982511500238\\
0.91943359375	3.77481323656978\\
0.919921875	3.76983230120873\\
0.92041015625	3.76488229673348\\
0.9208984375	3.75996321103815\\
0.92138671875	3.75507503209665\\
0.921875	3.75021774796258\\
0.92236328125	3.74539134676916\\
0.9228515625	3.74059581672903\\
0.92333984375	3.73583114613422\\
0.923828125	3.73109732335601\\
0.92431640625	3.72639433684485\\
0.9248046875	3.7217221751302\\
0.92529296875	3.71708082682053\\
0.92578125	3.71247028060307\\
0.92626953125	3.70789052524387\\
0.9267578125	3.70334154958755\\
0.92724609375	3.6988233425573\\
0.927734375	3.69433589315473\\
0.92822265625	3.68987919045978\\
0.9287109375	3.68545322363067\\
0.92919921875	3.6810579819037\\
0.9296875	3.67669345459328\\
0.93017578125	3.6723596310917\\
0.9306640625	3.66805650086915\\
0.93115234375	3.66378405347351\\
0.931640625	3.65954227853042\\
0.93212890625	3.655331165743\\
0.9326171875	3.6511507048919\\
0.93310546875	3.64700088583508\\
0.93359375	3.64288169850793\\
0.93408203125	3.63879313292288\\
0.9345703125	3.63473517916959\\
0.93505859375	3.63070782741467\\
0.935546875	3.62671106790171\\
0.93603515625	3.62274489095116\\
0.9365234375	3.61880928696014\\
0.93701171875	3.61490424640253\\
0.9375	3.61102975982879\\
0.93798828125	3.6071858178659\\
0.9384765625	3.60337241121715\\
0.93896484375	3.59958953066232\\
0.939453125	3.59583716705731\\
0.93994140625	3.59211531133431\\
0.9404296875	3.5884239545015\\
0.94091796875	3.58476308764314\\
0.94140625	3.58113270191942\\
0.94189453125	3.57753278856633\\
0.9423828125	3.57396333889568\\
0.94287109375	3.57042434429497\\
0.943359375	3.56691579622731\\
0.94384765625	3.56343768623136\\
0.9443359375	3.55999000592125\\
0.94482421875	3.55657274698648\\
0.9453125	3.55318590119196\\
0.94580078125	3.54982946037774\\
0.9462890625	3.54650341645907\\
0.94677734375	3.54320776142637\\
0.947265625	3.53994248734497\\
0.94775390625	3.53670758635532\\
0.9482421875	3.53350305067259\\
0.94873046875	3.53032887258692\\
0.94921875	3.52718504446313\\
0.94970703125	3.5240715587407\\
0.9501953125	3.52098840793382\\
0.95068359375	3.51793558463112\\
0.951171875	3.51491308149582\\
0.95166015625	3.5119208912655\\
0.9521484375	3.50895900675213\\
0.95263671875	3.50602742084194\\
0.953125	3.50312612649544\\
};
\addplot [color=mycolor2,solid]
  table[row sep=crcr]{0.953125	3.50312612649544\\
0.95361328125	3.50025511674725\\
0.9541015625	3.49741438470616\\
0.95458984375	3.49460392355495\\
0.955078125	3.4918237265504\\
0.95556640625	3.48907378702325\\
0.9560546875	3.4863540983781\\
0.95654296875	3.4836646540933\\
0.95703125	3.48100544772103\\
0.95751953125	3.47837647288713\\
0.9580078125	3.47577772329106\\
0.95849609375	3.47320919270592\\
0.958984375	3.47067087497826\\
0.95947265625	3.46816276402818\\
0.9599609375	3.46568485384914\\
0.96044921875	3.46323713850802\\
0.9609375	3.46081961214495\\
0.96142578125	3.45843226897345\\
0.9619140625	3.45607510328006\\
0.96240234375	3.45374810942467\\
0.962890625	3.45145128184013\\
0.96337890625	3.44918461503252\\
0.9638671875	3.44694810358076\\
0.96435546875	3.44474174213686\\
0.96484375	3.4425655254257\\
0.96533203125	3.44041944824506\\
0.9658203125	3.43830350546551\\
0.96630859375	3.43621769203046\\
0.966796875	3.43416200295601\\
0.96728515625	3.43213643333097\\
0.9677734375	3.43014097831683\\
0.96826171875	3.42817563314765\\
0.96875	3.42624039313008\\
0.96923828125	3.42433525364332\\
0.9697265625	3.422460210139\\
0.97021484375	3.42061525814125\\
0.970703125	3.41880039324657\\
0.97119140625	3.41701561112384\\
0.9716796875	3.4152609075143\\
0.97216796875	3.41353627823147\\
0.97265625	3.41184171916108\\
0.97314453125	3.41017722626113\\
0.9736328125	3.40854279556182\\
0.97412109375	3.40693842316546\\
0.974609375	3.40536410524649\\
0.97509765625	3.40381983805146\\
0.9755859375	3.40230561789894\\
0.97607421875	3.40082144117951\\
0.9765625	3.39936730435576\\
0.97705078125	3.39794320396227\\
0.9775390625	3.39654913660544\\
0.97802734375	3.39518509896369\\
0.978515625	3.39385108778723\\
0.97900390625	3.39254709989812\\
0.9794921875	3.39127313219027\\
0.97998046875	3.3900291816293\\
0.98046875	3.38881524525266\\
0.98095703125	3.3876313201695\\
0.9814453125	3.38647740356071\\
0.98193359375	3.38535349267878\\
0.982421875	3.38425958484792\\
0.98291015625	3.38319567746399\\
0.9833984375	3.38216176799442\\
0.98388671875	3.38115785397824\\
0.984375	3.38018393302606\\
0.98486328125	3.37924000282003\\
0.9853515625	3.37832606111384\\
0.98583984375	3.37744210573266\\
0.986328125	3.37658813457317\\
0.98681640625	3.3757641456035\\
0.9873046875	3.37497013686328\\
0.98779296875	3.3742061064635\\
0.98828125	3.37347205258664\\
0.98876953125	3.37276797348654\\
0.9892578125	3.37209386748846\\
0.98974609375	3.37144973298901\\
0.990234375	3.37083556845615\\
0.99072265625	3.37025137242921\\
0.9912109375	3.36969714351885\\
0.99169921875	3.36917288040705\\
0.9921875	3.36867858184707\\
0.99267578125	3.36821424666352\\
0.9931640625	3.36777987375226\\
0.99365234375	3.36737546208045\\
0.994140625	3.36700101068648\\
0.99462890625	3.36665651868008\\
0.9951171875	3.36634198524216\\
0.99560546875	3.36605740962491\\
0.99609375	3.36580279115175\\
0.99658203125	3.36557812921734\\
0.9970703125	3.36538342328758\\
0.99755859375	3.36521867289957\\
0.998046875	3.36508387766164\\
0.99853515625	3.36497903725335\\
0.9990234375	3.36490415142544\\
0.99951171875	3.36485921999991\\
};
\addlegendentry{AR(6) Model};

\addplot [color=mycolor3,solid,forget plot]
  table[row sep=crcr]{-1	3.53736373969883\\
-0.99951171875	3.53737880807819\\
-0.9990234375	3.53742401325063\\
-0.99853515625	3.53749935531936\\
-0.998046875	3.53760483445634\\
-0.99755859375	3.53774045090232\\
-0.9970703125	3.53790620496682\\
-0.99658203125	3.53810209702825\\
-0.99609375	3.53832812753368\\
-0.99560546875	3.53858429699905\\
-0.9951171875	3.53887060600918\\
-0.99462890625	3.53918705521754\\
-0.994140625	3.53953364534655\\
-0.99365234375	3.53991037718739\\
-0.9931640625	3.54031725160011\\
-0.99267578125	3.54075426951355\\
-0.9921875	3.54122143192543\\
-0.99169921875	3.54171873990233\\
-0.9912109375	3.54224619457967\\
-0.99072265625	3.54280379716178\\
-0.990234375	3.54339154892185\\
-0.98974609375	3.54400945120201\\
-0.9892578125	3.54465750541319\\
-0.98876953125	3.54533571303539\\
-0.98828125	3.54604407561745\\
-0.98779296875	3.54678259477718\\
-0.9873046875	3.5475512722014\\
-0.98681640625	3.54835010964583\\
-0.986328125	3.54917910893527\\
-0.98583984375	3.55003827196347\\
-0.9853515625	3.55092760069325\\
-0.98486328125	3.55184709715644\\
-0.984375	3.55279676345399\\
-0.98388671875	3.55377660175591\\
-0.9833984375	3.55478661430126\\
-0.98291015625	3.55582680339833\\
-0.982421875	3.55689717142446\\
-0.98193359375	3.55799772082622\\
-0.9814453125	3.55912845411937\\
-0.98095703125	3.56028937388882\\
-0.98046875	3.56148048278879\\
-0.97998046875	3.56270178354274\\
-0.9794921875	3.56395327894338\\
-0.97900390625	3.56523497185276\\
-0.978515625	3.56654686520226\\
-0.97802734375	3.56788896199262\\
-0.9775390625	3.56926126529401\\
-0.97705078125	3.57066377824596\\
-0.9765625	3.57209650405746\\
-0.97607421875	3.57355944600699\\
-0.9755859375	3.57505260744254\\
-0.97509765625	3.57657599178164\\
-0.974609375	3.57812960251135\\
-0.97412109375	3.57971344318837\\
-0.9736328125	3.581327517439\\
-0.97314453125	3.58297182895923\\
-0.97265625	3.58464638151472\\
-0.97216796875	3.58635117894088\\
-0.9716796875	3.58808622514291\\
-0.97119140625	3.58985152409576\\
-0.970703125	3.59164707984425\\
-0.97021484375	3.59347289650307\\
-0.9697265625	3.59532897825682\\
-0.96923828125	3.59721532936006\\
-0.96875	3.59913195413732\\
-0.96826171875	3.6010788569832\\
-0.9677734375	3.60305604236229\\
-0.96728515625	3.60506351480938\\
-0.966796875	3.60710127892934\\
-0.96630859375	3.60916933939728\\
-0.9658203125	3.61126770095852\\
-0.96533203125	3.61339636842865\\
-0.96484375	3.61555534669364\\
-0.96435546875	3.61774464070975\\
-0.9638671875	3.61996425550368\\
-0.96337890625	3.62221419617263\\
-0.962890625	3.62449446788423\\
-0.96240234375	3.62680507587674\\
-0.9619140625	3.62914602545895\\
-0.96142578125	3.63151732201033\\
-0.9609375	3.63391897098104\\
-0.96044921875	3.63635097789203\\
-0.9599609375	3.63881334833497\\
-0.95947265625	3.64130608797242\\
-0.958984375	3.64382920253785\\
-0.95849609375	3.64638269783568\\
-0.9580078125	3.6489665797413\\
-0.95751953125	3.65158085420123\\
-0.95703125	3.654225527233\\
-0.95654296875	3.65690060492542\\
-0.9560546875	3.65960609343844\\
-0.95556640625	3.66234199900332\\
-0.955078125	3.6651083279227\\
-0.95458984375	3.66790508657057\\
-0.9541015625	3.67073228139235\\
-0.95361328125	3.67358991890504\\
-0.953125	3.67647800569714\\
-0.95263671875	3.67939654842888\\
-0.9521484375	3.68234555383209\\
-0.95166015625	3.68532502871042\\
-0.951171875	3.68833497993929\\
-0.95068359375	3.69137541446606\\
-0.9501953125	3.69444633930998\\
-0.94970703125	3.69754776156233\\
-0.94921875	3.70067968838647\\
-0.94873046875	3.70384212701791\\
-0.9482421875	3.70703508476438\\
-0.94775390625	3.71025856900585\\
-0.947265625	3.71351258719462\\
-0.94677734375	3.7167971468555\\
-0.9462890625	3.72011225558563\\
-0.94580078125	3.72345792105486\\
-0.9453125	3.72683415100557\\
-0.94482421875	3.73024095325281\\
-0.9443359375	3.73367833568451\\
-0.94384765625	3.73714630626131\\
-0.943359375	3.74064487301684\\
-0.94287109375	3.74417404405769\\
-0.9423828125	3.74773382756352\\
-0.94189453125	3.75132423178712\\
-0.94140625	3.75494526505453\\
-0.94091796875	3.75859693576498\\
-0.9404296875	3.76227925239121\\
-0.93994140625	3.76599222347927\\
-0.939453125	3.76973585764885\\
-0.93896484375	3.77351016359315\\
-0.9384765625	3.77731515007911\\
-0.93798828125	3.78115082594748\\
-0.9375	3.78501720011271\\
-0.93701171875	3.78891428156335\\
-0.9365234375	3.79284207936188\\
-0.93603515625	3.79680060264489\\
-0.935546875	3.80078986062312\\
-0.93505859375	3.80480986258168\\
-0.9345703125	3.80886061787995\\
-0.93408203125	3.81294213595175\\
-0.93359375	3.8170544263055\\
-0.93310546875	3.82119749852417\\
-0.9326171875	3.8253713622655\\
-0.93212890625	3.82957602726196\\
-0.931640625	3.83381150332099\\
-0.93115234375	3.83807780032492\\
-0.9306640625	3.84237492823125\\
-0.93017578125	3.84670289707255\\
-0.9296875	3.85106171695674\\
-0.92919921875	3.85545139806704\\
-0.9287109375	3.85987195066213\\
-0.92822265625	3.86432338507625\\
-0.927734375	3.86880571171925\\
-0.92724609375	3.87331894107675\\
-0.9267578125	3.8778630837102\\
-0.92626953125	3.88243815025699\\
-0.92578125	3.88704415143053\\
-0.92529296875	3.8916810980204\\
-0.9248046875	3.8963490008924\\
-0.92431640625	3.90104787098864\\
-0.923828125	3.90577771932773\\
-0.92333984375	3.91053855700477\\
-0.9228515625	3.91533039519156\\
-0.92236328125	3.92015324513664\\
-0.921875	3.92500711816538\\
-0.92138671875	3.92989202568018\\
-0.9208984375	3.93480797916046\\
-0.92041015625	3.93975499016285\\
-0.919921875	3.94473307032128\\
-0.91943359375	3.94974223134704\\
-0.9189453125	3.95478248502899\\
-0.91845703125	3.95985384323359\\
-0.91796875	3.96495631790502\\
-0.91748046875	3.97008992106534\\
-0.9169921875	3.97525466481454\\
-0.91650390625	3.98045056133075\\
-0.916015625	3.9856776228702\\
-0.91552734375	3.9909358617675\\
-0.9150390625	3.99622529043566\\
-0.91455078125	4.00154592136624\\
-0.9140625	4.00689776712947\\
-0.91357421875	4.01228084037436\\
-0.9130859375	4.0176951538288\\
-0.91259765625	4.0231407202997\\
-0.912109375	4.02861755267317\\
-0.91162109375	4.03412566391454\\
-0.9111328125	4.03966506706852\\
-0.91064453125	4.0452357752594\\
-0.91015625	4.05083780169103\\
-0.90966796875	4.05647115964707\\
-0.9091796875	4.06213586249106\\
-0.90869140625	4.06783192366657\\
-0.908203125	4.07355935669733\\
-0.90771484375	4.0793181751873\\
-0.9072265625	4.0851083928209\\
-0.90673828125	4.09093002336308\\
-0.90625	4.09678308065944\\
-0.90576171875	4.10266757863641\\
-0.9052734375	4.10858353130131\\
-0.90478515625	4.11453095274258\\
-0.904296875	4.12050985712987\\
-0.90380859375	4.12652025871412\\
-0.9033203125	4.13256217182782\\
-0.90283203125	4.13863561088501\\
-0.90234375	4.14474059038152\\
-0.90185546875	4.1508771248951\\
-0.9013671875	4.15704522908549\\
-0.90087890625	4.16324491769461\\
-0.900390625	4.16947620554681\\
-0.89990234375	4.17573910754868\\
-0.8994140625	4.18203363868965\\
-0.89892578125	4.18835981404175\\
-0.8984375	4.19471764875998\\
-0.89794921875	4.20110715808238\\
-0.8974609375	4.20752835733014\\
-0.89697265625	4.21398126190783\\
-0.896484375	4.2204658873035\\
-0.89599609375	4.22698224908881\\
-0.8955078125	4.23353036291929\\
-0.89501953125	4.24011024453433\\
-0.89453125	4.24672190975747\\
-0.89404296875	4.25336537449646\\
-0.8935546875	4.26004065474351\\
-0.89306640625	4.26674776657534\\
-0.892578125	4.27348672615339\\
-0.89208984375	4.28025754972403\\
-0.8916015625	4.2870602536186\\
-0.89111328125	4.29389485425366\\
-0.890625	4.3007613681312\\
-0.89013671875	4.30765981183857\\
-0.8896484375	4.3145902020489\\
-0.88916015625	4.32155255552117\\
-0.888671875	4.32854688910034\\
-0.88818359375	4.33557321971752\\
-0.8876953125	4.34263156439023\\
-0.88720703125	4.34972194022239\\
-0.88671875	4.35684436440469\\
-0.88623046875	4.36399885421461\\
-0.8857421875	4.37118542701664\\
-0.88525390625	4.37840410026254\\
-0.884765625	4.38565489149127\\
-0.88427734375	4.39293781832945\\
-0.8837890625	4.40025289849139\\
-0.88330078125	4.40760014977919\\
-0.8828125	4.41497959008308\\
-0.88232421875	4.42239123738155\\
-0.8818359375	4.42983510974138\\
-0.88134765625	4.43731122531808\\
-0.880859375	4.44481960235583\\
-0.88037109375	4.45236025918783\\
-0.8798828125	4.45993321423632\\
-0.87939453125	4.46753848601297\\
-0.87890625	4.47517609311885\\
-0.87841796875	4.4828460542448\\
-0.8779296875	4.49054838817146\\
-0.87744140625	4.49828311376956\\
-0.876953125	4.50605025000008\\
-0.87646484375	4.51384981591447\\
-0.8759765625	4.52168183065472\\
-0.87548828125	4.52954631345372\\
-0.875	4.53744328363533\\
-0.87451171875	4.54537276061465\\
-0.8740234375	4.55333476389816\\
-0.87353515625	4.56132931308391\\
-0.873046875	4.56935642786178\\
-0.87255859375	4.57741612801364\\
-0.8720703125	4.58550843341353\\
-0.87158203125	4.5936333640279\\
-0.87109375	4.60179093991578\\
-0.87060546875	4.60998118122899\\
-0.8701171875	4.6182041082124\\
-0.86962890625	4.62645974120397\\
-0.869140625	4.6347481006352\\
-0.86865234375	4.64306920703109\\
-0.8681640625	4.65142308101056\\
-0.86767578125	4.65980974328651\\
-0.8671875	4.66822921466607\\
-0.86669921875	4.67668151605086\\
-0.8662109375	4.68516666843714\\
-0.86572265625	4.69368469291608\\
-0.865234375	4.70223561067394\\
-0.86474609375	4.71081944299223\\
-0.8642578125	4.71943621124809\\
-0.86376953125	4.72808593691431\\
-0.86328125	4.73676864155972\\
-0.86279296875	4.74548434684928\\
-0.8623046875	4.75423307454444\\
-0.86181640625	4.76301484650317\\
-0.861328125	4.77182968468036\\
-0.86083984375	4.78067761112801\\
-0.8603515625	4.78955864799536\\
-0.85986328125	4.79847281752923\\
-0.859375	4.80742014207419\\
-0.85888671875	4.81640064407284\\
-0.8583984375	4.82541434606593\\
-0.85791015625	4.83446127069275\\
-0.857421875	4.84354144069122\\
-0.85693359375	4.85265487889824\\
-0.8564453125	4.86180160824983\\
-0.85595703125	4.87098165178145\\
-0.85546875	4.88019503262821\\
-0.85498046875	4.88944177402504\\
-0.8544921875	4.89872189930703\\
-0.85400390625	4.90803543190966\\
-0.853515625	4.91738239536901\\
-0.85302734375	4.92676281332195\\
-0.8525390625	4.93617670950657\\
-0.85205078125	4.9456241077622\\
-0.8515625	4.95510503202981\\
-0.85107421875	4.96461950635225\\
-0.8505859375	4.97416755487444\\
-0.85009765625	4.98374920184363\\
-0.849609375	4.99336447160971\\
-0.84912109375	5.00301338862545\\
-0.8486328125	5.01269597744671\\
-0.84814453125	5.02241226273274\\
-0.84765625	5.03216226924645\\
-0.84716796875	5.04194602185458\\
-0.8466796875	5.05176354552812\\
-0.84619140625	5.06161486534242\\
-0.845703125	5.07150000647759\\
-0.84521484375	5.08141899421866\\
-0.8447265625	5.09137185395583\\
-0.84423828125	5.10135861118489\\
-0.84375	5.11137929150735\\
-0.84326171875	5.12143392063079\\
-0.8427734375	5.13152252436904\\
-0.84228515625	5.14164512864255\\
-0.841796875	5.15180175947863\\
-0.84130859375	5.16199244301179\\
-0.8408203125	5.17221720548387\\
-0.84033203125	5.18247607324446\\
-0.83984375	5.19276907275108\\
-0.83935546875	5.2030962305696\\
-0.8388671875	5.21345757337439\\
-0.83837890625	5.22385312794864\\
-0.837890625	5.23428292118473\\
-0.83740234375	5.24474698008436\\
-0.8369140625	5.25524533175903\\
-0.83642578125	5.26577800343018\\
-0.8359375	5.27634502242957\\
-0.83544921875	5.28694641619953\\
-0.8349609375	5.29758221229329\\
-0.83447265625	5.30825243837525\\
-0.833984375	5.31895712222129\\
-0.83349609375	5.32969629171907\\
-0.8330078125	5.34046997486834\\
-0.83251953125	5.35127819978128\\
-0.83203125	5.36212099468272\\
-0.83154296875	5.3729983879105\\
-0.8310546875	5.38391040791578\\
-0.83056640625	5.39485708326337\\
-0.830078125	5.40583844263199\\
-0.82958984375	5.41685451481459\\
-0.8291015625	5.42790532871876\\
-0.82861328125	5.43899091336689\\
-0.828125	5.45011129789663\\
-0.82763671875	5.46126651156111\\
-0.8271484375	5.4724565837294\\
-0.82666015625	5.48368154388658\\
-0.826171875	5.4949414216344\\
-0.82568359375	5.50623624669127\\
-0.8251953125	5.51756604889286\\
-0.82470703125	5.5289308581923\\
-0.82421875	5.54033070466048\\
-0.82373046875	5.55176561848653\\
-0.8232421875	5.56323562997798\\
-0.82275390625	5.57474076956117\\
-0.822265625	5.58628106778169\\
-0.82177734375	5.59785655530457\\
-0.8212890625	5.60946726291462\\
-0.82080078125	5.62111322151693\\
-0.8203125	5.63279446213714\\
-0.81982421875	5.64451101592165\\
-0.8193359375	5.65626291413814\\
-0.81884765625	5.66805018817593\\
-0.818359375	5.6798728695462\\
-0.81787109375	5.69173098988244\\
-0.8173828125	5.70362458094074\\
-0.81689453125	5.71555367460029\\
-0.81640625	5.72751830286352\\
-0.81591796875	5.73951849785666\\
-0.8154296875	5.75155429183001\\
-0.81494140625	5.7636257171583\\
-0.814453125	5.77573280634112\\
-0.81396484375	5.78787559200321\\
-0.8134765625	5.80005410689487\\
-0.81298828125	5.81226838389235\\
-0.8125	5.82451845599819\\
-0.81201171875	5.83680435634161\\
-0.8115234375	5.84912611817898\\
-0.81103515625	5.86148377489394\\
-0.810546875	5.87387735999812\\
-0.81005859375	5.88630690713127\\
-0.8095703125	5.89877245006175\\
-0.80908203125	5.91127402268694\\
-0.80859375	5.92381165903354\\
-0.80810546875	5.93638539325808\\
-0.8076171875	5.94899525964719\\
-0.80712890625	5.96164129261813\\
-0.806640625	5.97432352671904\\
-0.80615234375	5.98704199662947\\
-0.8056640625	5.99979673716068\\
-0.80517578125	6.01258778325616\\
-0.8046875	6.02541516999192\\
-0.80419921875	6.038278932577\\
-0.8037109375	6.05117910635375\\
-0.80322265625	6.06411572679836\\
-0.802734375	6.0770888295213\\
-0.80224609375	6.09009845026758\\
-0.8017578125	6.1031446249173\\
-0.80126953125	6.11622738948609\\
-0.80078125	6.12934678012535\\
-0.80029296875	6.14250283312291\\
-0.7998046875	6.15569558490336\\
-0.79931640625	6.16892507202835\\
-0.798828125	6.18219133119729\\
-0.79833984375	6.19549439924755\\
-0.7978515625	6.20883431315498\\
-0.79736328125	6.2222111100344\\
-0.796875	6.23562482713995\\
-0.79638671875	6.24907550186563\\
-0.7958984375	6.26256317174566\\
-0.79541015625	6.27608787445488\\
-0.794921875	6.28964964780942\\
-0.79443359375	6.30324852976692\\
-0.7939453125	6.31688455842714\\
-0.79345703125	6.33055777203225\\
-0.79296875	6.3442682089675\\
-0.79248046875	6.35801590776152\\
-0.7919921875	6.37180090708681\\
-0.79150390625	6.38562324576034\\
-0.791015625	6.39948296274378\\
-0.79052734375	6.41338009714415\\
-0.7900390625	6.42731468821433\\
-0.78955078125	6.44128677535333\\
-0.7890625	6.45529639810694\\
-0.78857421875	6.46934359616816\\
-0.7880859375	6.48342840937773\\
-0.78759765625	6.49755087772451\\
-0.787109375	6.51171104134598\\
-0.78662109375	6.52590894052895\\
-0.7861328125	6.54014461570969\\
-0.78564453125	6.55441810747472\\
-0.78515625	6.56872945656122\\
-0.78466796875	6.58307870385745\\
-0.7841796875	6.59746589040339\\
-0.78369140625	6.61189105739109\\
-0.783203125	6.62635424616535\\
-0.78271484375	6.64085549822412\\
-0.7822265625	6.65539485521898\\
-0.78173828125	6.6699723589558\\
-0.78125	6.68458805139513\\
-0.78076171875	6.69924197465276\\
-0.7802734375	6.71393417100027\\
-0.77978515625	6.72866468286555\\
-0.779296875	6.74343355283327\\
-0.77880859375	6.75824082364553\\
-0.7783203125	6.77308653820225\\
-0.77783203125	6.78797073956183\\
-0.77734375	6.80289347094164\\
-0.77685546875	6.81785477571853\\
-0.7763671875	6.83285469742949\\
-0.77587890625	6.84789327977207\\
-0.775390625	6.86297056660495\\
-0.77490234375	6.87808660194863\\
-0.7744140625	6.8932414299858\\
-0.77392578125	6.90843509506202\\
-0.7734375	6.9236676416863\\
-0.77294921875	6.93893911453153\\
-0.7724609375	6.9542495584352\\
-0.77197265625	6.96959901839988\\
-0.771484375	6.98498753959389\\
-0.77099609375	7.00041516735173\\
-0.7705078125	7.01588194717476\\
-0.77001953125	7.03138792473186\\
-0.76953125	7.04693314585982\\
-0.76904296875	7.06251765656409\\
-0.7685546875	7.07814150301935\\
-0.76806640625	7.09380473156999\\
-0.767578125	7.10950738873091\\
-0.76708984375	7.1252495211879\\
-0.7666015625	7.14103117579849\\
-0.76611328125	7.15685239959231\\
-0.765625	7.17271323977186\\
-0.76513671875	7.1886137437131\\
-0.7646484375	7.20455395896607\\
-0.76416015625	7.22053393325543\\
-0.763671875	7.23655371448121\\
-0.76318359375	7.25261335071938\\
-0.7626953125	7.26871289022248\\
-0.76220703125	7.28485238142019\\
-0.76171875	7.30103187292016\\
-0.76123046875	7.31725141350841\\
-0.7607421875	7.33351105215019\\
-0.76025390625	7.34981083799048\\
-0.759765625	7.3661508203547\\
-0.75927734375	7.38253104874937\\
-0.7587890625	7.39895157286273\\
-0.75830078125	7.41541244256549\\
-0.7578125	7.4319137079114\\
-0.75732421875	7.44845541913797\\
-0.7568359375	7.46503762666711\\
-0.75634765625	7.48166038110586\\
-0.755859375	7.498323733247\\
-0.75537109375	7.51502773406983\\
-0.7548828125	7.53177243474075\\
-0.75439453125	7.54855788661398\\
-0.75390625	7.56538414123237\\
-0.75341796875	7.58225125032788\\
-0.7529296875	7.59915926582254\\
-0.75244140625	7.6161082398289\\
-0.751953125	7.63309822465093\\
-0.75146484375	7.65012927278463\\
-0.7509765625	7.66720143691882\\
-0.75048828125	7.68431476993579\\
-0.75	7.7014693249121\\
-0.74951171875	7.71866515511923\\
-0.7490234375	7.73590231402432\\
-0.74853515625	7.75318085529105\\
-0.748046875	7.77050083278012\\
-0.74755859375	7.78786230055025\\
-0.7470703125	7.80526531285877\\
-0.74658203125	7.82270992416246\\
-0.74609375	7.84019618911826\\
-0.74560546875	7.85772416258397\\
-0.7451171875	7.87529389961921\\
-0.74462890625	7.89290545548595\\
-0.744140625	7.91055888564948\\
-0.74365234375	7.92825424577902\\
-0.7431640625	7.94599159174869\\
-0.74267578125	7.96377097963811\\
-0.7421875	7.98159246573326\\
-0.74169921875	7.99945610652733\\
-0.7412109375	8.01736195872155\\
-0.74072265625	8.0353100792257\\
-0.740234375	8.05330052515936\\
-0.73974609375	8.07133335385234\\
-0.7392578125	8.08940862284573\\
-0.73876953125	8.10752638989262\\
-0.73828125	8.12568671295896\\
-0.73779296875	8.14388965022432\\
-0.7373046875	8.16213526008283\\
-0.73681640625	8.18042360114399\\
-0.736328125	8.19875473223341\\
-0.73583984375	8.21712871239379\\
-0.7353515625	8.23554560088572\\
-0.73486328125	8.2540054571885\\
-0.734375	8.27250834100112\\
-0.73388671875	8.29105431224292\\
-0.7333984375	8.3096434310547\\
-0.73291015625	8.32827575779946\\
-0.732421875	8.34695135306319\\
-0.73193359375	8.365670277656\\
-0.7314453125	8.38443259261284\\
-0.73095703125	8.40323835919438\\
-0.73046875	8.42208763888803\\
-0.72998046875	8.44098049340873\\
-0.7294921875	8.45991698469995\\
-0.72900390625	8.47889717493453\\
-0.728515625	8.4979211265156\\
-0.72802734375	8.51698890207762\\
-0.7275390625	8.53610056448715\\
-0.72705078125	8.55525617684393\\
-0.7265625	8.57445580248169\\
-0.72607421875	8.5936995049692\\
-0.7255859375	8.61298734811118\\
-0.72509765625	8.63231939594923\\
-0.724609375	8.65169571276287\\
-0.72412109375	8.6711163630704\\
-0.7236328125	8.69058141162992\\
-0.72314453125	8.71009092344037\\
-0.72265625	8.7296449637425\\
-0.72216796875	8.74924359801965\\
-0.7216796875	8.76888689199913\\
-0.72119140625	8.78857491165281\\
-0.720703125	8.80830772319842\\
-0.72021484375	8.82808539310049\\
-0.7197265625	8.84790798807125\\
-0.71923828125	8.86777557507181\\
-0.71875	8.88768822131307\\
-0.71826171875	8.90764599425685\\
-0.7177734375	8.92764896161685\\
-0.71728515625	8.94769719135971\\
-0.716796875	8.96779075170615\\
-0.71630859375	8.98792971113187\\
-0.7158203125	9.00811413836879\\
-0.71533203125	9.02834410240599\\
-0.71484375	9.04861967249084\\
-0.71435546875	9.06894091813003\\
-0.7138671875	9.08930790909075\\
-0.71337890625	9.10972071540173\\
-0.712890625	9.13017940735431\\
-0.71240234375	9.15068405550363\\
-0.7119140625	9.17123473066962\\
-0.71142578125	9.19183150393825\\
-0.7109375	9.21247444666261\\
-0.71044921875	9.23316363046401\\
-0.7099609375	9.25389912723312\\
-0.70947265625	9.27468100913115\\
-0.708984375	9.29550934859104\\
-0.70849609375	9.31638421831851\\
-0.7080078125	9.33730569129324\\
-0.70751953125	9.35827384077022\\
-0.70703125	9.37928874028067\\
-0.70654296875	9.4003504636334\\
-0.7060546875	9.42145908491595\\
-0.70556640625	9.44261467849579\\
-0.705078125	9.46381731902148\\
-0.70458984375	9.48506708142405\\
-0.7041015625	9.50636404091797\\
-0.70361328125	9.52770827300254\\
-0.703125	9.5490998534632\\
-0.70263671875	9.57053885837249\\
-0.7021484375	9.59202536409161\\
-0.70166015625	9.61355944727145\\
-0.701171875	9.63514118485401\\
-0.70068359375	9.65677065407359\\
-0.7001953125	9.67844793245802\\
-0.69970703125	9.70017309783011\\
-0.69921875	9.72194622830879\\
-0.69873046875	9.74376740231042\\
-0.6982421875	9.76563669855019\\
-0.69775390625	9.78755419604338\\
-0.697265625	9.80951997410672\\
-0.69677734375	9.83153411235963\\
-0.6962890625	9.85359669072567\\
-0.69580078125	9.87570778943381\\
-0.6953125	9.89786748901987\\
-0.69482421875	9.92007587032773\\
-0.6943359375	9.94233301451091\\
-0.69384765625	9.9646390030338\\
-0.693359375	9.98699391767301\\
-0.69287109375	10.009397840519\\
-0.6923828125	10.0318508539772\\
-0.69189453125	10.0543530407697\\
-0.69140625	10.0769044839362\\
-0.69091796875	10.0995052668363\\
-0.6904296875	10.1221554731498\\
-0.68994140625	10.1448551868792\\
-0.689453125	10.1676044923504\\
-0.68896484375	10.1904034742148\\
-0.6884765625	10.2132522174502\\
-0.68798828125	10.2361508073625\\
-0.6875	10.2590993295873\\
-0.68701171875	10.2820978700914\\
-0.6865234375	10.3051465151741\\
-0.68603515625	10.3282453514688\\
-0.685546875	10.3513944659447\\
-0.68505859375	10.3745939459082\\
-0.6845703125	10.3978438790043\\
-0.68408203125	10.4211443532185\\
-0.68359375	10.444495456878\\
-0.68310546875	10.4678972786535\\
-0.6826171875	10.4913499075608\\
-0.68212890625	10.5148534329623\\
-0.681640625	10.5384079445686\\
-0.68115234375	10.5620135324401\\
-0.6806640625	10.5856702869886\\
-0.68017578125	10.6093782989793\\
-0.6796875	10.6331376595317\\
-0.67919921875	10.656948460122\\
-0.6787109375	10.6808107925844\\
-0.67822265625	10.7047247491128\\
-0.677734375	10.7286904222623\\
-0.67724609375	10.7527079049515\\
-0.6767578125	10.7767772904636\\
-0.67626953125	10.8008986724481\\
-0.67578125	10.8250721449231\\
-0.67529296875	10.8492978022766\\
-0.6748046875	10.8735757392681\\
-0.67431640625	10.8979060510308\\
-0.673828125	10.9222888330732\\
-0.67333984375	10.9467241812806\\
-0.6728515625	10.9712121919173\\
-0.67236328125	10.9957529616282\\
-0.671875	11.0203465874406\\
-0.67138671875	11.0449931667661\\
-0.6708984375	11.0696927974022\\
-0.67041015625	11.0944455775347\\
-0.669921875	11.119251605739\\
-0.66943359375	11.1441109809821\\
-0.6689453125	11.1690238026247\\
-0.66845703125	11.1939901704228\\
-0.66796875	11.2190101845299\\
-0.66748046875	11.2440839454988\\
-0.6669921875	11.2692115542833\\
-0.66650390625	11.2943931122405\\
-0.666015625	11.3196287211325\\
-0.66552734375	11.3449184831286\\
-0.6650390625	11.3702625008071\\
-0.66455078125	11.3956608771572\\
-0.6640625	11.4211137155813\\
-0.66357421875	11.4466211198967\\
-0.6630859375	11.4721831943381\\
-0.66259765625	11.4978000435589\\
-0.662109375	11.5234717726341\\
-0.66162109375	11.5491984870617\\
-0.6611328125	11.5749802927651\\
-0.66064453125	11.6008172960951\\
-0.66015625	11.6267096038321\\
-0.65966796875	11.6526573231882\\
-0.6591796875	11.6786605618093\\
-0.65869140625	11.7047194277772\\
-0.658203125	11.7308340296118\\
-0.65771484375	11.7570044762733\\
-0.6572265625	11.7832308771646\\
-0.65673828125	11.809513342133\\
-0.65625	11.8358519814728\\
-0.65576171875	11.8622469059276\\
-0.6552734375	11.8886982266923\\
-0.65478515625	11.9152060554155\\
-0.654296875	11.9417705042017\\
-0.65380859375	11.9683916856135\\
-0.6533203125	11.9950697126745\\
-0.65283203125	12.0218046988707\\
-0.65234375	12.0485967581537\\
-0.65185546875	12.0754460049422\\
-0.6513671875	12.1023525541254\\
-0.65087890625	12.1293165210645\\
-0.650390625	12.1563380215953\\
-0.64990234375	12.1834171720311\\
-0.6494140625	12.2105540891646\\
-0.64892578125	12.2377488902706\\
-0.6484375	12.2650016931081\\
-0.64794921875	12.2923126159236\\
-0.6474609375	12.3196817774526\\
-0.64697265625	12.3471092969229\\
-0.646484375	12.3745952940566\\
-0.64599609375	12.4021398890732\\
-0.6455078125	12.4297432026914\\
-0.64501953125	12.4574053561325\\
-0.64453125	12.4851264711225\\
-0.64404296875	12.5129066698947\\
-0.6435546875	12.5407460751927\\
-0.64306640625	12.5686448102726\\
-0.642578125	12.5966029989062\\
-0.64208984375	12.6246207653831\\
-0.6416015625	12.652698234514\\
-0.64111328125	12.6808355316327\\
-0.640625	12.7090327825997\\
-0.64013671875	12.7372901138043\\
-0.6396484375	12.7656076521677\\
-0.63916015625	12.7939855251455\\
-0.638671875	12.8224238607311\\
-0.63818359375	12.8509227874579\\
-0.6376953125	12.8794824344025\\
-0.63720703125	12.9081029311875\\
-0.63671875	12.9367844079844\\
-0.63623046875	12.9655269955165\\
-0.6357421875	12.9943308250619\\
-0.63525390625	13.0231960284564\\
-0.634765625	13.0521227380965\\
-0.63427734375	13.0811110869424\\
-0.6337890625	13.1101612085209\\
-0.63330078125	13.1392732369286\\
-0.6328125	13.168447306835\\
-0.63232421875	13.1976835534854\\
-0.6318359375	13.2269821127041\\
-0.63134765625	13.2563431208975\\
-0.630859375	13.2857667150572\\
-0.63037109375	13.3152530327634\\
-0.6298828125	13.3448022121878\\
-0.62939453125	13.3744143920969\\
-0.62890625	13.4040897118554\\
-0.62841796875	13.4338283114292\\
-0.6279296875	13.4636303313888\\
-0.62744140625	13.4934959129127\\
-0.626953125	13.5234251977905\\
-0.62646484375	13.5534183284266\\
-0.6259765625	13.5834754478431\\
-0.62548828125	13.6135966996836\\
-0.625	13.6437822282165\\
-0.62451171875	13.6740321783382\\
-0.6240234375	13.7043466955771\\
-0.62353515625	13.7347259260966\\
-0.623046875	13.7651700166989\\
-0.62255859375	13.7956791148282\\
-0.6220703125	13.8262533685749\\
-0.62158203125	13.8568929266786\\
-0.62109375	13.8875979385318\\
-0.62060546875	13.918368554184\\
-0.6201171875	13.9492049243447\\
-0.61962890625	13.9801072003879\\
-0.619140625	14.0110755343549\\
-0.61865234375	14.0421100789588\\
-0.6181640625	14.0732109875881\\
-0.61767578125	14.1043784143101\\
-0.6171875	14.1356125138752\\
-0.61669921875	14.1669134417208\\
-0.6162109375	14.1982813539748\\
-0.61572265625	14.2297164074598\\
-0.615234375	14.2612187596968\\
-0.61474609375	14.2927885689097\\
-0.6142578125	14.3244259940287\\
-0.61376953125	14.3561311946944\\
-0.61328125	14.3879043312626\\
-0.61279296875	14.4197455648073\\
-0.6123046875	14.4516550571257\\
-0.61181640625	14.4836329707419\\
-0.611328125	14.5156794689111\\
-0.61083984375	14.5477947156241\\
-0.6103515625	14.5799788756114\\
-0.60986328125	14.6122321143472\\
-0.609375	14.6445545980542\\
-0.60888671875	14.6769464937074\\
-0.6083984375	14.7094079690391\\
-0.60791015625	14.7419391925428\\
-0.607421875	14.7745403334777\\
-0.60693359375	14.8072115618736\\
-0.6064453125	14.8399530485348\\
-0.60595703125	14.872764965045\\
-0.60546875	14.9056474837719\\
-0.60498046875	14.9386007778715\\
-0.6044921875	14.9716250212932\\
-0.60400390625	15.004720388784\\
-0.603515625	15.0378870558937\\
-0.60302734375	15.071125198979\\
-0.6025390625	15.1044349952089\\
-0.60205078125	15.1378166225695\\
-0.6015625	15.1712702598681\\
-0.60107421875	15.2047960867392\\
-0.6005859375	15.2383942836484\\
-0.60009765625	15.2720650318981\\
-0.599609375	15.3058085136323\\
-0.59912109375	15.3396249118412\\
-0.5986328125	15.3735144103672\\
-0.59814453125	15.407477193909\\
-0.59765625	15.4415134480274\\
-0.59716796875	15.4756233591506\\
-0.5966796875	15.5098071145788\\
-0.59619140625	15.54406490249\\
-0.595703125	15.5783969119454\\
-0.59521484375	15.6128033328942\\
-0.5947265625	15.6472843561795\\
-0.59423828125	15.6818401735437\\
-0.59375	15.7164709776338\\
-0.59326171875	15.751176962007\\
-0.5927734375	15.7859583211362\\
-0.59228515625	15.8208152504159\\
-0.591796875	15.8557479461675\\
-0.59130859375	15.8907566056453\\
-0.5908203125	15.9258414270421\\
-0.59033203125	15.9610026094949\\
-0.58984375	15.9962403530909\\
-0.58935546875	16.0315548588736\\
-0.5888671875	16.066946328848\\
-0.58837890625	16.1024149659874\\
-0.587890625	16.1379609742391\\
-0.58740234375	16.1735845585303\\
-0.5869140625	16.2092859247744\\
-0.58642578125	16.2450652798774\\
-0.5859375	16.2809228317434\\
-0.58544921875	16.3168587892819\\
-0.5849609375	16.3528733624131\\
-0.58447265625	16.388966762075\\
-0.583984375	16.4251392002295\\
-0.58349609375	16.461390889869\\
-0.5830078125	16.4977220450228\\
-0.58251953125	16.5341328807636\\
-0.58203125	16.5706236132145\\
-0.58154296875	16.6071944595555\\
-0.5810546875	16.6438456380298\\
-0.58056640625	16.6805773679514\\
-0.580078125	16.7173898697115\\
-0.57958984375	16.7542833647852\\
-0.5791015625	16.7912580757388\\
-0.57861328125	16.8283142262369\\
-0.578125	16.8654520410491\\
-0.57763671875	16.9026717460574\\
-0.5771484375	16.9399735682632\\
-0.57666015625	16.9773577357951\\
-0.576171875	17.0148244779151\\
-0.57568359375	17.0523740250272\\
-0.5751953125	17.0900066086841\\
-0.57470703125	17.1277224615948\\
-0.57421875	17.1655218176323\\
-0.57373046875	17.2034049118412\\
-0.5732421875	17.2413719804452\\
-0.57275390625	17.2794232608551\\
-0.572265625	17.3175589916763\\
-0.57177734375	17.3557794127171\\
-0.5712890625	17.394084764996\\
-0.57080078125	17.4324752907506\\
-0.5703125	17.4709512334448\\
-0.56982421875	17.5095128377773\\
-0.5693359375	17.54816034969\\
-0.56884765625	17.5868940163761\\
-0.568359375	17.6257140862882\\
-0.56787109375	17.6646208091473\\
-0.5673828125	17.7036144359505\\
-0.56689453125	17.7426952189805\\
-0.56640625	17.7818634118134\\
-0.56591796875	17.821119269328\\
-0.5654296875	17.860463047714\\
-0.56494140625	17.8998950044815\\
-0.564453125	17.9394153984696\\
-0.56396484375	17.9790244898552\\
-0.5634765625	18.0187225401626\\
-0.56298828125	18.0585098122721\\
-0.5625	18.0983865704298\\
-0.56201171875	18.1383530802565\\
-0.5615234375	18.1784096087574\\
-0.56103515625	18.2185564243311\\
-0.560546875	18.2587937967798\\
-0.56005859375	18.2991219973187\\
-0.5595703125	18.3395412985855\\
-0.55908203125	18.3800519746506\\
-0.55859375	18.4206543010264\\
-0.55810546875	18.4613485546783\\
-0.5576171875	18.5021350140339\\
-0.55712890625	18.5430139589932\\
-0.556640625	18.5839856709395\\
-0.55615234375	18.6250504327492\\
-0.5556640625	18.6662085288024\\
-0.55517578125	18.7074602449933\\
-0.5546875	18.7488058687412\\
-0.55419921875	18.7902456890007\\
-0.5537109375	18.831779996273\\
-0.55322265625	18.8734090826164\\
-0.552734375	18.9151332416575\\
-0.55224609375	18.9569527686025\\
-0.5517578125	18.9988679602479\\
-0.55126953125	19.0408791149922\\
-0.55078125	19.0829865328474\\
-0.55029296875	19.1251905154501\\
-0.5498046875	19.1674913660732\\
-0.54931640625	19.209889389638\\
-0.548828125	19.2523848927257\\
-0.54833984375	19.2949781835893\\
-0.5478515625	19.3376695721656\\
-0.54736328125	19.3804593700878\\
-0.546875	19.423347890697\\
-0.54638671875	19.4663354490554\\
-0.5458984375	19.509422361958\\
-0.54541015625	19.5526089479459\\
-0.544921875	19.5958955273183\\
-0.54443359375	19.639282422146\\
-0.5439453125	19.6827699562839\\
-0.54345703125	19.7263584553841\\
-0.54296875	19.7700482469094\\
-0.54248046875	19.8138396601463\\
-0.5419921875	19.8577330262184\\
-0.54150390625	19.9017286781002\\
-0.541015625	19.9458269506306\\
-0.54052734375	19.9900281805267\\
-0.5400390625	20.034332706398\\
-0.53955078125	20.0787408687598\\
-0.5390625	20.1232530100484\\
-0.53857421875	20.1678694746344\\
-0.5380859375	20.2125906088378\\
-0.53759765625	20.2574167609427\\
-0.537109375	20.3023482812114\\
-0.53662109375	20.3473855218997\\
-0.5361328125	20.392528837272\\
-0.53564453125	20.4377785836163\\
-0.53515625	20.4831351192595\\
-0.53466796875	20.5285988045827\\
-0.5341796875	20.5741700020373\\
-0.53369140625	20.61984907616\\
-0.533203125	20.6656363935895\\
-0.53271484375	20.711532323082\\
-0.5322265625	20.7575372355272\\
-0.53173828125	20.8036515039656\\
-0.53125	20.8498755036041\\
-0.53076171875	20.8962096118328\\
-0.5302734375	20.9426542082423\\
-0.52978515625	20.9892096746404\\
-0.529296875	21.0358763950692\\
-0.52880859375	21.0826547558224\\
-0.5283203125	21.1295451454629\\
-0.52783203125	21.1765479548408\\
-0.52734375	21.2236635771104\\
-0.52685546875	21.2708924077491\\
-0.5263671875	21.3182348445748\\
-0.52587890625	21.3656912877652\\
-0.525390625	21.4132621398753\\
-0.52490234375	21.4609478058569\\
-0.5244140625	21.5087486930773\\
-0.52392578125	21.5566652113381\\
-0.5234375	21.604697772895\\
-0.52294921875	21.6528467924769\\
-0.5224609375	21.7011126873059\\
-0.52197265625	21.7494958771167\\
-0.521484375	21.7979967841772\\
-0.52099609375	21.8466158333083\\
-0.5205078125	21.8953534519047\\
-0.52001953125	21.9442100699557\\
-0.51953125	21.9931861200654\\
-0.51904296875	22.0422820374747\\
-0.5185546875	22.0914982600819\\
-0.51806640625	22.1408352284647\\
-0.517578125	22.1902933859015\\
-0.51708984375	22.2398731783939\\
-0.5166015625	22.2895750546884\\
-0.51611328125	22.339399466299\\
-0.515625	22.3893468675299\\
-0.51513671875	22.4394177154983\\
-0.5146484375	22.4896124701577\\
-0.51416015625	22.5399315943207\\
-0.513671875	22.5903755536835\\
-0.51318359375	22.6409448168486\\
-0.5126953125	22.6916398553499\\
-0.51220703125	22.7424611436763\\
-0.51171875	22.7934091592968\\
-0.51123046875	22.8444843826848\\
-0.5107421875	22.8956872973438\\
-0.51025390625	22.9470183898319\\
-0.509765625	22.9984781497886\\
-0.50927734375	23.0500670699595\\
-0.5087890625	23.1017856462231\\
-0.50830078125	23.1536343776171\\
-0.5078125	23.2056137663649\\
-0.50732421875	23.2577243179031\\
-0.5068359375	23.3099665409081\\
-0.50634765625	23.3623409473242\\
-0.505859375	23.4148480523909\\
-0.50537109375	23.4674883746718\\
-0.5048828125	23.5202624360823\\
-0.50439453125	23.5731707619188\\
-0.50390625	23.6262138808873\\
-0.50341796875	23.6793923251335\\
-0.5029296875	23.7327066302716\\
-0.50244140625	23.7861573354151\\
-0.501953125	23.8397449832063\\
-0.50146484375	23.8934701198479\\
-0.5009765625	23.9473332951328\\
-0.50048828125	24.0013350624765\\
-0.5	24.0554759789482\\
-0.49951171875	24.1097566053029\\
-0.4990234375	24.1641775060133\\
-0.49853515625	24.2187392493035\\
-0.498046875	24.273442407181\\
-0.49755859375	24.3282875554707\\
-0.4970703125	24.3832752738484\\
-0.49658203125	24.438406145875\\
-0.49609375	24.4936807590312\\
-0.49560546875	24.5490997047521\\
-0.4951171875	24.6046635784626\\
-0.49462890625	24.6603729796131\\
-0.494140625	24.7162285117157\\
-0.49365234375	24.77223078238\\
-0.4931640625	24.8283804033511\\
-0.49267578125	24.8846779905456\\
-0.4921875	24.9411241640904\\
-0.49169921875	24.99771954836\\
-0.4912109375	25.0544647720156\\
-0.49072265625	25.1113604680436\\
-0.490234375	25.1684072737953\\
-0.48974609375	25.2256058310267\\
-0.4892578125	25.2829567859388\\
-0.48876953125	25.3404607892182\\
-0.48828125	25.3981184960787\\
-0.48779296875	25.4559305663023\\
-0.4873046875	25.513897664282\\
-0.48681640625	25.5720204590643\\
-0.486328125	25.6302996243919\\
-0.48583984375	25.6887358387479\\
-0.4853515625	25.7473297853993\\
-0.48486328125	25.8060821524423\\
-0.484375	25.8649936328467\\
-0.48388671875	25.9240649245021\\
-0.4833984375	25.9832967302637\\
-0.48291015625	26.0426897579993\\
-0.482421875	26.1022447206364\\
-0.48193359375	26.1619623362097\\
-0.4814453125	26.2218433279099\\
-0.48095703125	26.2818884241324\\
-0.48046875	26.3420983585267\\
-0.47998046875	26.4024738700466\\
-0.4794921875	26.4630157030006\\
-0.47900390625	26.5237246071034\\
-0.478515625	26.5846013375275\\
-0.47802734375	26.6456466549557\\
-0.4775390625	26.706861325634\\
-0.47705078125	26.7682461214257\\
-0.4765625	26.8298018198653\\
-0.47607421875	26.8915292042134\\
-0.4755859375	26.9534290635126\\
-0.47509765625	27.0155021926436\\
-0.474609375	27.0777493923819\\
-0.47412109375	27.1401714694559\\
-0.4736328125	27.2027692366047\\
-0.47314453125	27.2655435126373\\
-0.47265625	27.3284951224924\\
-0.47216796875	27.3916248972983\\
-0.4716796875	27.4549336744347\\
-0.47119140625	27.5184222975941\\
-0.470703125	27.5820916168444\\
-0.47021484375	27.6459424886926\\
-0.4697265625	27.7099757761484\\
-0.46923828125	27.7741923487897\\
-0.46875	27.8385930828278\\
-0.46826171875	27.903178861174\\
-0.4677734375	27.967950573507\\
-0.46728515625	28.0329091163408\\
-0.466796875	28.0980553930938\\
-0.46630859375	28.1633903141588\\
-0.4658203125	28.228914796973\\
-0.46533203125	28.2946297660898\\
-0.46484375	28.3605361532516\\
-0.46435546875	28.4266348974621\\
-0.4638671875	28.4929269450616\\
-0.46337890625	28.5594132498011\\
-0.462890625	28.6260947729188\\
-0.46240234375	28.6929724832168\\
-0.4619140625	28.7600473571393\\
-0.46142578125	28.8273203788511\\
-0.4609375	28.8947925403175\\
-0.46044921875	28.9624648413853\\
-0.4599609375	29.0303382898645\\
-0.45947265625	29.098413901611\\
-0.458984375	29.1666927006108\\
-0.45849609375	29.2351757190647\\
-0.4580078125	29.3038639974742\\
-0.45751953125	29.3727585847286\\
-0.45703125	29.4418605381936\\
-0.45654296875	29.5111709237998\\
-0.4560546875	29.5806908161336\\
-0.45556640625	29.6504212985286\\
-0.455078125	29.7203634631581\\
-0.45458984375	29.7905184111291\\
-0.4541015625	29.8608872525775\\
-0.45361328125	29.9314711067638\\
-0.453125	30.0022711021709\\
-0.45263671875	30.0732883766027\\
-0.4521484375	30.1445240772841\\
-0.45166015625	30.2159793609621\\
-0.451171875	30.287655394008\\
-0.45068359375	30.3595533525217\\
-0.4501953125	30.4316744224366\\
-0.44970703125	30.5040197996255\\
-0.44921875	30.5765906900094\\
-0.44873046875	30.6493883096658\\
-0.4482421875	30.7224138849396\\
-0.44775390625	30.7956686525549\\
-0.447265625	30.8691538597287\\
-0.44677734375	30.9428707642855\\
-0.4462890625	31.0168206347735\\
-0.44580078125	31.091004750583\\
-0.4453125	31.165424402065\\
-0.44482421875	31.240080890652\\
-0.4443359375	31.3149755289814\\
-0.44384765625	31.3901096410184\\
-0.443359375	31.4654845621819\\
-0.44287109375	31.5411016394718\\
-0.4423828125	31.6169622315978\\
-0.44189453125	31.6930677091095\\
-0.44140625	31.7694194545291\\
-0.44091796875	31.8460188624847\\
-0.4404296875	31.9228673398458\\
-0.43994140625	31.9999663058614\\
-0.439453125	32.0773171922978\\
-0.43896484375	32.1549214435809\\
-0.4384765625	32.2327805169371\\
-0.43798828125	32.3108958825396\\
-0.4375	32.3892690236531\\
-0.43701171875	32.4679014367833\\
-0.4365234375	32.5467946318262\\
-0.43603515625	32.6259501322204\\
-0.435546875	32.7053694751009\\
-0.43505859375	32.7850542114553\\
-0.4345703125	32.8650059062817\\
-0.43408203125	32.9452261387485\\
-0.43359375	33.0257165023564\\
-0.43310546875	33.1064786051026\\
-0.4326171875	33.1875140696466\\
-0.43212890625	33.2688245334792\\
-0.431640625	33.3504116490919\\
-0.43115234375	33.4322770841503\\
-0.4306640625	33.5144225216684\\
-0.43017578125	33.5968496601855\\
-0.4296875	33.6795602139459\\
-0.42919921875	33.7625559130793\\
-0.4287109375	33.8458385037857\\
-0.42822265625	33.9294097485206\\
-0.427734375	34.0132714261836\\
-0.42724609375	34.0974253323089\\
-0.4267578125	34.1818732792587\\
-0.42626953125	34.2666170964185\\
-0.42578125	34.3516586303942\\
-0.42529296875	34.4369997452139\\
-0.4248046875	34.522642322529\\
-0.42431640625	34.6085882618206\\
-0.423828125	34.6948394806067\\
-0.42333984375	34.7813979146519\\
-0.4228515625	34.8682655181817\\
-0.42236328125	34.9554442640958\\
-0.421875	35.0429361441879\\
-0.42138671875	35.1307431693643\\
-0.4208984375	35.2188673698686\\
-0.42041015625	35.3073107955062\\
-0.419921875	35.396075515873\\
-0.41943359375	35.485163620587\\
-0.4189453125	35.5745772195204\\
-0.41845703125	35.6643184430373\\
-0.41796875	35.7543894422319\\
-0.41748046875	35.8447923891703\\
-0.4169921875	35.9355294771346\\
-0.41650390625	36.0266029208703\\
-0.416015625	36.1180149568356\\
-0.41552734375	36.2097678434539\\
-0.4150390625	36.301863861369\\
-0.41455078125	36.3943053137026\\
-0.4140625	36.4870945263151\\
-0.41357421875	36.5802338480683\\
-0.4130859375	36.6737256510916\\
-0.41259765625	36.76757233105\\
-0.412109375	36.8617763074155\\
-0.41162109375	36.9563400237408\\
-0.4111328125	37.051265947935\\
-0.41064453125	37.1465565725433\\
-0.41015625	37.2422144150279\\
-0.40966796875	37.3382420180514\\
-0.4091796875	37.4346419497641\\
-0.40869140625	37.5314168040917\\
-0.408203125	37.6285692010274\\
-0.40771484375	37.7261017869245\\
-0.4072265625	37.8240172347926\\
-0.40673828125	37.9223182445946\\
-0.40625	38.0210075435479\\
-0.40576171875	38.1200878864253\\
-0.4052734375	38.2195620558595\\
-0.40478515625	38.3194328626488\\
-0.404296875	38.4197031460647\\
-0.40380859375	38.5203757741607\\
-0.4033203125	38.621453644084\\
-0.40283203125	38.7229396823865\\
-0.40234375	38.824836845339\\
-0.40185546875	38.9271481192457\\
-0.4013671875	39.0298765207597\\
-0.40087890625	39.1330250971989\\
-0.400390625	39.2365969268638\\
-0.39990234375	39.3405951193542\\
-0.3994140625	39.4450228158874\\
-0.39892578125	39.5498831896153\\
-0.3984375	39.6551794459419\\
-0.39794921875	39.7609148228412\\
-0.3974609375	39.8670925911719\\
-0.39697265625	39.9737160549933\\
-0.396484375	40.0807885518792\\
-0.39599609375	40.18831345323\\
-0.3955078125	40.2962941645824\\
-0.39501953125	40.4047341259177\\
-0.39453125	40.5136368119663\\
-0.39404296875	40.6230057325098\\
-0.3935546875	40.7328444326786\\
-0.39306640625	40.8431564932467\\
-0.392578125	40.9539455309202\\
-0.39208984375	41.0652151986224\\
-0.3916015625	41.1769691857714\\
-0.39111328125	41.2892112185532\\
-0.390625	41.4019450601863\\
-0.39013671875	41.5151745111798\\
-0.3896484375	41.628903409582\\
-0.38916015625	41.743135631222\\
-0.388671875	41.8578750899378\\
-0.38818359375	41.973125737798\\
-0.3876953125	42.0888915653081\\
-0.38720703125	42.2051766016064\\
-0.38671875	42.3219849146454\\
-0.38623046875	42.439320611359\\
-0.3857421875	42.5571878378125\\
-0.38525390625	42.6755907793375\\
-0.384765625	42.7945336606471\\
-0.38427734375	42.9140207459313\\
-0.3837890625	43.0340563389325\\
-0.38330078125	43.154644782997\\
-0.3828125	43.2757904611034\\
-0.38232421875	43.397497795863\\
-0.3818359375	43.5197712494956\\
-0.38134765625	43.6426153237733\\
-0.380859375	43.7660345599326\\
-0.38037109375	43.8900335385539\\
-0.3798828125	44.0146168794048\\
-0.37939453125	44.1397892412436\\
-0.37890625	44.2655553215835\\
-0.37841796875	44.391919856412\\
-0.3779296875	44.5188876198643\\
-0.37744140625	44.6464634238481\\
-0.376953125	44.7746521176136\\
-0.37646484375	44.9034585872699\\
-0.3759765625	45.0328877552402\\
-0.37548828125	45.1629445796558\\
-0.375	45.2936340536804\\
-0.37451171875	45.4249612047648\\
-0.3740234375	45.5569310938263\\
-0.37353515625	45.6895488143454\\
-0.373046875	45.8228194913799\\
-0.37255859375	45.9567482804864\\
-0.3720703125	46.0913403665471\\
-0.37158203125	46.2266009624929\\
-0.37109375	46.3625353079212\\
-0.37060546875	46.4991486675967\\
-0.3701171875	46.6364463298307\\
-0.36962890625	46.7744336047344\\
-0.369140625	46.9131158223312\\
-0.36865234375	47.0524983305291\\
-0.3681640625	47.1925864929367\\
-0.36767578125	47.3333856865173\\
-0.3671875	47.4749012990723\\
-0.36669921875	47.6171387265413\\
-0.3662109375	47.7601033701099\\
-0.36572265625	47.9038006331126\\
-0.365234375	48.048235917721\\
-0.36474609375	48.1934146214005\\
-0.3642578125	48.3393421331274\\
-0.36376953125	48.4860238293477\\
-0.36328125	48.6334650696657\\
-0.36279296875	48.7816711922442\\
-0.3623046875	48.9306475089022\\
-0.36181640625	49.0803992998894\\
-0.361328125	49.2309318083229\\
-0.36083984375	49.3822502342613\\
-0.3603515625	49.5343597284015\\
-0.35986328125	49.6872653853705\\
-0.359375	49.8409722365937\\
-0.35888671875	49.9954852427122\\
-0.3583984375	50.1508092855254\\
-0.35791015625	50.3069491594309\\
-0.357421875	50.4639095623334\\
-0.35693359375	50.6216950859925\\
-0.3564453125	50.7803102057773\\
-0.35595703125	50.9397592697954\\
-0.35546875	51.1000464873601\\
-0.35498046875	51.2611759167593\\
-0.3544921875	51.4231514522865\\
-0.35400390625	51.5859768104951\\
-0.353515625	51.7496555156297\\
-0.35302734375	51.9141908841916\\
-0.3525390625	52.0795860085899\\
-0.35205078125	52.2458437398294\\
-0.3515625	52.4129666691821\\
-0.35107421875	52.5809571087875\\
-0.3505859375	52.7498170711248\\
-0.35009765625	52.9195482472967\\
-0.349609375	53.0901519840642\\
-0.34912109375	53.2616292595618\\
-0.3486328125	53.43398065763\\
-0.34814453125	53.6072063406923\\
-0.34765625	53.7813060211006\\
-0.34716796875	53.9562789308762\\
-0.3466796875	54.1321237897599\\
-0.34619140625	54.3088387714988\\
-0.345703125	54.4864214682725\\
-0.34521484375	54.6648688531782\\
-0.3447265625	54.8441772406851\\
-0.34423828125	55.0243422449555\\
-0.34375	55.2053587359498\\
-0.34326171875	55.3872207932072\\
-0.3427734375	55.569921657211\\
-0.34228515625	55.753453678232\\
-0.341796875	55.937808262555\\
-0.34130859375	56.1229758159813\\
-0.3408203125	56.3089456845129\\
-0.34033203125	56.4957060921114\\
-0.33984375	56.6832440754406\\
-0.33935546875	56.8715454154959\\
-0.3388671875	57.0605945660268\\
-0.33837890625	57.2503745786749\\
-0.337890625	57.4408670247434\\
-0.33740234375	57.6320519135297\\
-0.3369140625	57.8239076071678\\
-0.33642578125	58.0164107319283\\
-0.3359375	58.209536085948\\
-0.33544921875	58.4032565433763\\
-0.3349609375	58.5975429549467\\
-0.33447265625	58.7923640450029\\
-0.333984375	58.9876863050402\\
-0.33349609375	59.1834738838513\\
-0.3330078125	59.3796884744045\\
-0.33251953125	59.5762891976191\\
-0.33203125	59.7732324832485\\
-0.33154296875	59.9704719481342\\
-0.3310546875	60.1679582721489\\
-0.33056640625	60.3656390722094\\
-0.330078125	60.5634587748049\\
-0.32958984375	60.7613584875709\\
-0.3291015625	60.9592758705042\\
-0.32861328125	61.1571450075298\\
-0.328125	61.3548962791927\\
-0.32763671875	61.5524562373818\\
-0.3271484375	61.7497474830814\\
-0.32666015625	61.9466885482759\\
-0.326171875	62.1431937832412\\
-0.32568359375	62.3391732506006\\
-0.3251953125	62.5345326276301\\
-0.32470703125	62.7291731184659\\
-0.32421875	62.9229913779535\\
-0.32373046875	63.1158794490663\\
-0.3232421875	63.3077247158989\\
-0.32275390625	63.4984098744038\\
-0.322265625	63.6878129231234\\
-0.32177734375	63.875807176295\\
-0.3212890625	64.0622613017631\\
-0.32080078125	64.2470393862126\\
-0.3203125	64.4300010302566\\
-0.31982421875	64.6110014759142\\
-0.3193359375	64.789891768983\\
-0.31884765625	64.9665189587063\\
-0.318359375	65.1407263370277\\
-0.31787109375	65.312353719519\\
-0.3173828125	65.481237769824\\
-0.31689453125	65.6472123691512\\
-0.31640625	65.8101090319622\\
-0.31591796875	65.9697573685557\\
-0.3154296875	66.1259855947206\\
-0.31494140625	66.2786210880494\\
-0.314453125	66.4274909898391\\
-0.31396484375	66.5724228507933\\
-0.3134765625	66.7132453179739\\
-0.31298828125	66.8497888596413\\
-0.3125	66.9818865237956\\
-0.31201171875	67.1093747253675\\
-0.3115234375	67.2320940562187\\
-0.31103515625	67.3498901112505\\
-0.310546875	67.4626143232366\\
-0.31005859375	67.5701247982779\\
-0.3095703125	67.6722871432413\\
-0.30908203125	67.7689752760915\\
-0.30859375	67.8600722097326\\
-0.30810546875	67.9454707998611\\
-0.3076171875	68.0250744473749\\
-0.30712890625	68.0987977461306\\
-0.306640625	68.1665670672919\\
-0.30615234375	68.2283210721512\\
-0.3056640625	68.284011146118\\
-0.30517578125	68.3336017476016\\
-0.3046875	68.3770706666301\\
-0.30419921875	68.4144091893734\\
-0.3037109375	68.4456221660871\\
-0.30322265625	68.4707279814636\\
-0.302734375	68.4897584278324\\
-0.30224609375	68.5027584831232\\
-0.3017578125	68.5097859969033\\
-0.30126953125	68.5109112891508\\
-0.30078125	68.5062166676041\\
-0.30029296875	68.4957958706187\\
-0.2998046875	68.4797534433501\\
-0.29931640625	68.4582040557908\\
-0.298828125	68.4312717717236\\
-0.29833984375	68.3990892779601\\
-0.2978515625	68.3617970833847\\
-0.29736328125	68.319542697233\\
-0.296875	68.2724797958507\\
-0.29638671875	68.2207673867516\\
-0.2958984375	68.1645689782952\\
-0.29541015625	68.1040517626751\\
-0.294921875	68.0393858191844\\
-0.29443359375	67.9707433439422\\
-0.2939453125	67.898297911442\\
-0.29345703125	67.8222237724538\\
-0.29296875	67.7426951919555\\
-0.29248046875	67.6598858299633\\
-0.2919921875	67.5739681673364\\
-0.29150390625	67.4851129779052\\
-0.291015625	67.3934888475847\\
-0.29052734375	67.2992617405183\\
-0.2900390625	67.2025946117455\\
-0.28955078125	67.103647065423\\
-0.2890625	67.0025750572004\\
-0.28857421875	66.8995306390261\\
-0.2880859375	66.7946617443798\\
-0.28759765625	66.6881120117244\\
-0.287109375	66.5800206437986\\
-0.28662109375	66.4705223002928\\
-0.2861328125	66.3597470213736\\
-0.28564453125	66.247820179516\\
-0.28515625	66.1348624571328\\
-0.28466796875	66.0209898475115\\
-0.2841796875	65.9063136766764\\
-0.28369140625	65.7909406438532\\
-0.283203125	65.6749728783599\\
-0.28271484375	65.5585080108215\\
-0.2822265625	65.4416392567813\\
-0.28173828125	65.3244555108699\\
-0.28125	65.2070414498618\\
-0.28076171875	65.0894776430622\\
-0.2802734375	64.9718406686062\\
-0.27978515625	64.8542032343787\\
-0.279296875	64.7366343023943\\
-0.27880859375	64.6191992155818\\
-0.2783203125	64.5019598260413\\
-0.27783203125	64.3849746239462\\
-0.27734375	64.2682988663523\\
-0.27685546875	64.1519847052777\\
-0.2763671875	64.0360813145015\\
-0.27587890625	63.920635014596\\
-0.275390625	63.8056893957941\\
-0.27490234375	63.6912854383436\\
-0.2744140625	63.5774616300708\\
-0.27392578125	63.4642540809204\\
-0.2734375	63.3516966342862\\
-0.27294921875	63.2398209749964\\
-0.2724609375	63.1286567338464\\
-0.27197265625	63.0182315886079\\
-0.271484375	62.9085713614778\\
-0.27099609375	62.7997001129383\\
-0.2705078125	62.6916402320526\\
-0.27001953125	62.5844125231972\\
-0.26953125	62.4780362892932\\
-0.26904296875	62.3725294115734\\
-0.2685546875	62.2679084259621\\
-0.26806640625	62.1641885961341\\
-0.267578125	62.0613839833395\\
-0.26708984375	61.9595075130859\\
-0.2666015625	61.8585710387595\\
-0.26611328125	61.7585854022948\\
-0.265625	61.6595604919845\\
-0.26513671875	61.5615052975332\\
-0.2646484375	61.4644279624508\\
-0.26416015625	61.368335833895\\
-0.263671875	61.2732355100558\\
-0.26318359375	61.1791328851886\\
-0.2626953125	61.0860331923862\\
-0.26220703125	60.993941044193\\
-0.26171875	60.9028604711516\\
-0.26123046875	60.812794958374\\
-0.2607421875	60.72374748023\\
-0.26025390625	60.6357205332314\\
-0.259765625	60.5487161672092\\
-0.25927734375	60.4627360148473\\
-0.2587890625	60.3777813196696\\
-0.25830078125	60.29385296254\\
-0.2578125	60.2109514867573\\
-0.25732421875	60.1290771218113\\
-0.2568359375	60.0482298058655\\
-0.25634765625	59.9684092070348\\
-0.255859375	59.889614743514\\
-0.25537109375	59.8118456026152\\
-0.2548828125	59.7351007587798\\
-0.25439453125	59.6593789906059\\
-0.25390625	59.5846788969472\\
-0.25341796875	59.5109989121354\\
-0.2529296875	59.4383373203652\\
-0.25244140625	59.3666922692929\\
-0.251953125	59.2960617828848\\
-0.25146484375	59.2264437735601\\
-0.2509765625	59.1578360536599\\
-0.25048828125	59.0902363462866\\
-0.25	59.0236422955391\\
-0.24951171875	58.9580514761852\\
-0.2490234375	58.8934614027931\\
-0.24853515625	58.8298695383581\\
-0.248046875	58.7672733024497\\
-0.24755859375	58.7056700789039\\
-0.2470703125	58.6450572230898\\
-0.24658203125	58.58543206877\\
-0.24609375	58.5267919345791\\
-0.24560546875	58.4691341301425\\
-0.2451171875	58.4124559618555\\
-0.24462890625	58.3567547383372\\
-0.244140625	58.3020277755879\\
-0.24365234375	58.2482724018533\\
-0.2431640625	58.1954859622244\\
-0.24267578125	58.1436658229817\\
-0.2421875	58.0928093756996\\
-0.24169921875	58.0429140411246\\
-0.2412109375	57.9939772728419\\
-0.24072265625	57.945996560743\\
-0.240234375	57.8989694343028\\
-0.23974609375	57.8528934656806\\
-0.2392578125	57.8077662726549\\
-0.23876953125	57.7635855214035\\
-0.23828125	57.7203489291317\\
-0.23779296875	57.6780542665689\\
-0.2373046875	57.6366993603307\\
-0.23681640625	57.5962820951637\\
-0.236328125	57.5568004160728\\
-0.23583984375	57.5182523303475\\
-0.2353515625	57.4806359094823\\
-0.23486328125	57.4439492910126\\
-0.234375	57.4081906802544\\
-0.23388671875	57.3733583519724\\
-0.2333984375	57.3394506519669\\
-0.23291015625	57.3064659985947\\
-0.232421875	57.2744028842222\\
-0.23193359375	57.2432598766221\\
-0.2314453125	57.2130356203133\\
-0.23095703125	57.1837288378483\\
-0.23046875	57.1553383310561\\
-0.22998046875	57.1278629822429\\
-0.2294921875	57.10130175535\\
-0.22900390625	57.0756536970807\\
-0.228515625	57.0509179379905\\
-0.22802734375	57.0270936935541\\
-0.2275390625	57.0041802652003\\
-0.22705078125	56.9821770413292\\
-0.2265625	56.9610834983061\\
-0.22607421875	56.9408992014398\\
-0.2255859375	56.9216238059454\\
-0.22509765625	56.9032570578956\\
-0.224609375	56.8857987951608\\
-0.22412109375	56.8692489483427\\
-0.2236328125	56.8536075417017\\
-0.22314453125	56.8388746940817\\
-0.22265625	56.8250506198318\\
-0.22216796875	56.8121356297314\\
-0.2216796875	56.8001301319151\\
-0.22119140625	56.7890346328021\\
-0.220703125	56.7788497380331\\
-0.22021484375	56.769576153414\\
-0.2197265625	56.7612146858669\\
-0.21923828125	56.753766244396\\
-0.21875	56.7472318410614\\
-0.21826171875	56.7416125919695\\
-0.2177734375	56.7369097182782\\
-0.21728515625	56.733124547217\\
-0.216796875	56.7302585131272\\
-0.21630859375	56.72831315852\\
-0.2158203125	56.7272901351564\\
-0.21533203125	56.7271912051468\\
-0.21484375	56.728018242077\\
-0.21435546875	56.7297732321537\\
-0.2138671875	56.7324582753818\\
-0.21337890625	56.7360755867602\\
-0.212890625	56.7406274975148\\
-0.21240234375	56.7461164563529\\
-0.2119140625	56.7525450307493\\
-0.21142578125	56.7599159082673\\
-0.2109375	56.768231897907\\
-0.21044921875	56.7774959314887\\
-0.2099609375	56.7877110650713\\
-0.20947265625	56.798880480406\\
-0.208984375	56.8110074864239\\
-0.20849609375	56.8240955207636\\
-0.2080078125	56.8381481513365\\
-0.20751953125	56.8531690779298\\
-0.20703125	56.8691621338503\\
-0.20654296875	56.8861312876129\\
-0.2060546875	56.9040806446634\\
-0.20556640625	56.9230144491514\\
-0.205078125	56.942937085744\\
-0.20458984375	56.9638530814841\\
-0.2041015625	56.9857671076932\\
-0.20361328125	57.0086839819242\\
-0.203125	57.0326086699559\\
-0.20263671875	57.0575462878385\\
-0.2021484375	57.0835021039883\\
-0.20166015625	57.1104815413285\\
-0.201171875	57.1384901794809\\
-0.20068359375	57.1675337570087\\
-0.2001953125	57.1976181737095\\
-0.19970703125	57.228749492959\\
-0.19921875	57.2609339441059\\
-0.19873046875	57.2941779249198\\
-0.1982421875	57.3284880040902\\
-0.19775390625	57.3638709237781\\
-0.197265625	57.4003336022177\\
-0.19677734375	57.437883136375\\
-0.1962890625	57.4765268046525\\
-0.19580078125	57.5162720696498\\
-0.1953125	57.5571265809705\\
-0.19482421875	57.5990981780864\\
-0.1943359375	57.642194893242\\
-0.19384765625	57.6864249544149\\
-0.193359375	57.7317967883199\\
-0.19287109375	57.7783190234584\\
-0.1923828125	57.8260004932133\\
-0.19189453125	57.8748502389808\\
-0.19140625	57.9248775133498\\
-0.19091796875	57.9760917833088\\
-0.1904296875	58.0285027334895\\
-0.18994140625	58.0821202694433\\
-0.189453125	58.1369545209361\\
-0.18896484375	58.1930158452701\\
-0.1884765625	58.2503148306194\\
-0.18798828125	58.3088622993736\\
-0.1875	58.3686693114874\\
-0.18701171875	58.4297471678247\\
-0.1865234375	58.492107413494\\
-0.18603515625	58.5557618411575\\
-0.185546875	58.6207224943156\\
-0.18505859375	58.6870016705474\\
-0.1845703125	58.7546119246959\\
-0.18408203125	58.8235660719898\\
-0.18359375	58.8938771910818\\
-0.18310546875	58.9655586269878\\
-0.1826171875	59.0386239939131\\
-0.18212890625	59.1130871779411\\
-0.181640625	59.1889623395633\\
-0.18115234375	59.2662639160317\\
-0.1806640625	59.3450066234999\\
-0.18017578125	59.4252054589276\\
-0.1796875	59.5068757017225\\
-0.17919921875	59.5900329150688\\
-0.1787109375	59.6746929469237\\
-0.17822265625	59.7608719306236\\
-0.177734375	59.8485862850655\\
-0.17724609375	59.9378527144015\\
-0.1767578125	60.0286882072056\\
-0.17626953125	60.1211100350363\\
-0.17578125	60.2151357503388\\
-0.17529296875	60.3107831836117\\
-0.1748046875	60.4080704397549\\
-0.17431640625	60.5070158935127\\
-0.173828125	60.6076381839165\\
-0.17333984375	60.7099562076235\\
-0.1728515625	60.8139891110296\\
-0.17236328125	60.9197562810362\\
-0.171875	61.0272773343322\\
-0.17138671875	61.1365721050319\\
-0.1708984375	61.2476606305134\\
-0.17041015625	61.3605631352692\\
-0.169921875	61.4753000125686\\
-0.16943359375	61.5918918037161\\
-0.1689453125	61.7103591746706\\
-0.16845703125	61.8307228897556\\
-0.16796875	61.9530037821811\\
-0.16748046875	62.0772227210591\\
-0.1669921875	62.2034005745749\\
-0.16650390625	62.3315581689355\\
-0.166015625	62.4617162426839\\
-0.16552734375	62.5938953959357\\
-0.1650390625	62.7281160340415\\
-0.16455078125	62.8643983051434\\
-0.1640625	63.0027620310352\\
-0.16357421875	63.1432266306932\\
-0.1630859375	63.2858110357702\\
-0.16259765625	63.4305335972969\\
-0.162109375	63.5774119827593\\
-0.16162109375	63.7264630626407\\
-0.1611328125	63.8777027854463\\
-0.16064453125	64.0311460401381\\
-0.16015625	64.1868065048076\\
-0.15966796875	64.3446964803284\\
-0.1591796875	64.5048267076057\\
-0.15869140625	64.667206166946\\
-0.158203125	64.8318418579296\\
-0.15771484375	64.9987385580634\\
-0.1572265625	65.1678985583398\\
-0.15673828125	65.3393213737003\\
-0.15625	65.5130034262622\\
-0.15576171875	65.6889376990166\\
-0.1552734375	65.867113357561\\
-0.15478515625	66.0475153373026\\
-0.154296875	66.2301238934116\\
-0.15380859375	66.414914110693\\
-0.1533203125	66.6018553704373\\
-0.15283203125	66.7909107712148\\
-0.15234375	66.9820365005324\\
-0.15185546875	67.1751811542667\\
-0.1513671875	67.3702850008107\\
-0.15087890625	67.5672791870118\\
-0.150390625	67.7660848831695\\
-0.14990234375	67.9666123646566\\
-0.1494140625	68.1687600282214\\
-0.14892578125	68.3724133415829\\
-0.1484375	68.5774437257725\\
-0.14794921875	68.7837073707111\\
-0.1474609375	68.9910439857956\\
-0.14697265625	69.1992754889177\\
-0.146484375	69.4082046393224\\
-0.14599609375	69.6176136221002\\
-0.1455078125	69.8272625949661\\
-0.14501953125	70.0368882113541\\
-0.14453125	70.2462021377221\\
-0.14404296875	70.4548895874262\\
-0.1435546875	70.6626078985711\\
-0.14306640625	70.8689851887448\\
-0.142578125	71.073619125666\\
-0.14208984375	71.2760758591411\\
-0.1416015625	71.4758891663861\\
-0.14111328125	71.6725598693913\\
-0.140625	71.8655555892563\\
-0.14013671875	72.0543109079723\\
-0.1396484375	72.2382280123518\\
-0.13916015625	72.4166778972693\\
-0.138671875	72.5890022052092\\
-0.13818359375	72.7545157757752\\
-0.1376953125	72.9125099714082\\
-0.13720703125	73.0622568335104\\
-0.13671875	73.2030141058099\\
-0.13623046875	73.3340311388971\\
-0.1357421875	73.4545556612674\\
-0.13525390625	73.5638413684446\\
-0.134765625	73.6611562437022\\
-0.13427734375	73.7457914830375\\
-0.1337890625	73.8170708554878\\
-0.13330078125	73.8743602902582\\
-0.1328125	73.917077447309\\
-0.13232421875	73.9447010012956\\
-0.1318359375	73.9567793529796\\
-0.13134765625	73.9529384799204\\
-0.130859375	73.9328886512046\\
-0.13037109375	73.8964297596202\\
-0.1298828125	73.8434550684277\\
-0.12939453125	73.7739532265594\\
-0.12890625	73.6880084724114\\
-0.12841796875	73.5857990181201\\
-0.1279296875	73.4675936784358\\
-0.12744140625	73.3337468762969\\
-0.126953125	73.1846922163075\\
-0.12646484375	73.0209348642165\\
-0.1259765625	72.8430430026216\\
-0.12548828125	72.6516386495387\\
-0.125	72.4473881274965\\
-0.12451171875	72.2309924578158\\
-0.1240234375	72.0031779302089\\
-0.12353515625	71.7646870645223\\
-0.123046875	71.5162701426399\\
-0.12255859375	71.2586774471774\\
-0.1220703125	70.9926523023959\\
-0.12158203125	70.7189249739161\\
-0.12109375	70.4382074490054\\
-0.12060546875	70.1511890893943\\
-0.1201171875	69.8585331244189\\
-0.11962890625	69.5608739336723\\
-0.119140625	69.2588150552028\\
-0.11865234375	68.9529278468559\\
-0.1181640625	68.6437507242175\\
-0.11767578125	68.3317888977879\\
-0.1171875	68.017514533941\\
-0.11669921875	67.7013672681394\\
-0.1162109375	67.3837550041424\\
-0.11572265625	67.0650549390971\\
-0.115234375	66.7456147609797\\
-0.11474609375	66.4257539714961\\
-0.1142578125	66.1057652940392\\
-0.11376953125	65.7859161324601\\
-0.11328125	65.4664500520578\\
-0.11279296875	65.1475882593749\\
-0.1123046875	64.8295310619277\\
-0.11181640625	64.5124592930508\\
-0.111328125	64.1965356904819\\
-0.11083984375	63.8819062203057\\
-0.1103515625	63.568701340342\\
-0.10986328125	63.2570371991621\\
-0.109375	62.9470167686044\\
-0.10888671875	62.6387309090471\\
-0.1083984375	62.3322593677811\\
-0.10791015625	62.027671711683\\
-0.107421875	61.7250281960424\\
-0.10693359375	61.4243805718709\\
-0.1064453125	61.1257728343684\\
-0.10595703125	60.8292419154428\\
-0.10546875	60.534818323311\\
-0.10498046875	60.2425267322752\\
-0.1044921875	59.9523865257586\\
-0.10400390625	59.6644122956473\\
-0.103515625	59.3786143009015\\
-0.10302734375	59.0949988882983\\
-0.1025390625	58.8135688780453\\
-0.10205078125	58.5343239168693\\
-0.1015625	58.2572608010469\\
-0.10107421875	57.9823737716933\\
-0.1005859375	57.7096547844931\\
-0.10009765625	57.4390937559041\\
-0.099609375	57.1706787877261\\
-0.09912109375	56.9043963718076\\
-0.0986328125	56.6402315765145\\
-0.09814453125	56.3781682164795\\
-0.09765625	56.1181890070289\\
-0.09716796875	55.8602757045763\\
-0.0966796875	55.6044092341703\\
-0.09619140625	55.3505698052922\\
-0.095703125	55.0987370169074\\
-0.09521484375	54.8488899526931\\
-0.0947265625	54.6010072672935\\
-0.09423828125	54.3550672643722\\
-0.09375	54.1110479671856\\
-0.09326171875	53.8689271823172\\
-0.0927734375	53.6286825571802\\
-0.09228515625	53.3902916318307\\
-0.091796875	53.1537318855925\\
-0.09130859375	52.9189807789515\\
-0.0908203125	52.6860157911394\\
-0.09033203125	52.4548144537874\\
-0.08984375	52.2253543809988\\
-0.08935546875	51.9976132961657\\
-0.0888671875	51.771569055815\\
-0.08837890625	51.5471996707532\\
-0.087890625	51.324483324756\\
-0.08740234375	51.1033983910211\\
-0.0869140625	50.8839234465917\\
-0.08642578125	50.6660372849319\\
-0.0859375	50.4497189268289\\
-0.08544921875	50.2349476297713\\
-0.0849609375	50.0217028959496\\
-0.08447265625	49.8099644790044\\
-0.083984375	49.599712389642\\
-0.08349609375	49.3909269002256\\
-0.0830078125	49.1835885484373\\
-0.08251953125	48.9776781401052\\
-0.08203125	48.7731767512714\\
-0.08154296875	48.5700657295805\\
-0.0810546875	48.3683266950559\\
-0.08056640625	48.1679415403225\\
-0.080078125	47.9688924303371\\
-0.07958984375	47.7711618016748\\
-0.0791015625	47.5747323614201\\
-0.07861328125	47.3795870857051\\
-0.078125	47.1857092179337\\
-0.07763671875	46.9930822667278\\
-0.0771484375	46.8016900036287\\
-0.07666015625	46.6115164605816\\
-0.076171875	46.4225459272296\\
-0.07568359375	46.2347629480455\\
-0.0751953125	46.0481523193194\\
-0.07470703125	45.8626990860235\\
-0.07421875	45.6783885385733\\
-0.07373046875	45.4952062095023\\
-0.0732421875	45.3131378700622\\
-0.07275390625	45.1321695267663\\
-0.072265625	44.9522874178843\\
-0.07177734375	44.7734780099046\\
-0.0712890625	44.5957279939688\\
-0.07080078125	44.4190242822917\\
-0.0703125	44.2433540045726\\
-0.06982421875	44.0687045044066\\
-0.0693359375	43.8950633357015\\
-0.06884765625	43.7224182591068\\
-0.068359375	43.5507572384599\\
-0.06787109375	43.3800684372546\\
-0.0673828125	43.2103402151351\\
-0.06689453125	43.041561124421\\
-0.06640625	42.8737199066636\\
-0.06591796875	42.7068054892396\\
-0.0654296875	42.5408069819836\\
-0.06494140625	42.3757136738599\\
-0.064453125	42.2115150296774\\
-0.06396484375	42.0482006868501\\
-0.0634765625	41.8857604522007\\
-0.06298828125	41.7241842988124\\
-0.0625	41.5634623629288\\
-0.06201171875	41.4035849409002\\
-0.0615234375	41.2445424861809\\
-0.06103515625	41.0863256063747\\
-0.060546875	40.9289250603302\\
-0.06005859375	40.7723317552864\\
-0.0595703125	40.6165367440675\\
-0.05908203125	40.4615312223284\\
-0.05859375	40.3073065258489\\
-0.05810546875	40.1538541278775\\
-0.0576171875	40.0011656365265\\
-0.05712890625	39.8492327922119\\
-0.056640625	39.6980474651463\\
-0.05615234375	39.5476016528758\\
-0.0556640625	39.3978874778672\\
-0.05517578125	39.248897185141\\
-0.0546875	39.1006231399505\\
-0.05419921875	38.9530578255074\\
-0.0537109375	38.8061938407522\\
-0.05322265625	38.6600238981693\\
-0.052734375	38.5145408216449\\
-0.05224609375	38.3697375443703\\
-0.0517578125	38.225607106785\\
-0.05126953125	38.0821426545636\\
-0.05078125	37.9393374366425\\
-0.05029296875	37.7971848032879\\
-0.0498046875	37.6556782042024\\
-0.04931640625	37.5148111866713\\
-0.048828125	37.3745773937465\\
-0.04833984375	37.2349705624687\\
-0.0478515625	37.0959845221253\\
-0.04736328125	36.9576131925447\\
-0.046875	36.8198505824265\\
-0.04638671875	36.6826907877057\\
-0.0458984375	36.5461279899504\\
-0.04541015625	36.4101564547945\\
-0.044921875	36.2747705304005\\
-0.04443359375	36.1399646459565\\
-0.0439453125	36.0057333102028\\
-0.04345703125	35.8720711099898\\
-0.04296875	35.7389727088661\\
-0.04248046875	35.6064328456951\\
-0.0419921875	35.4744463333007\\
-0.04150390625	35.3430080571413\\
-0.041015625	35.2121129740109\\
-0.04052734375	35.0817561107673\\
-0.0400390625	34.9519325630863\\
-0.03955078125	34.822637494242\\
-0.0390625	34.6938661339119\\
-0.03857421875	34.5656137770067\\
-0.0380859375	34.4378757825245\\
-0.03759765625	34.3106475724281\\
-0.037109375	34.1839246305453\\
-0.03662109375	34.0577025014924\\
-0.0361328125	33.9319767896185\\
-0.03564453125	33.806743157973\\
-0.03515625	33.6819973272919\\
-0.03466796875	33.5577350750074\\
-0.0341796875	33.4339522342752\\
-0.03369140625	33.3106446930228\\
-0.033203125	33.1878083930167\\
-0.03271484375	33.0654393289487\\
-0.0322265625	32.9435335475403\\
-0.03173828125	32.8220871466647\\
-0.03125	32.7010962744872\\
-0.03076171875	32.5805571286224\\
-0.0302734375	32.4604659553086\\
-0.02978515625	32.3408190485975\\
-0.029296875	32.2216127495616\\
-0.02880859375	32.102843445516\\
-0.0283203125	31.9845075692563\\
-0.02783203125	31.8666015983115\\
-0.02734375	31.7491220542109\\
-0.02685546875	31.6320655017665\\
-0.0263671875	31.5154285483684\\
-0.02587890625	31.399207843295\\
-0.025390625	31.2834000770354\\
-0.02490234375	31.168001980626\\
-0.0244140625	31.0530103249997\\
-0.02392578125	30.9384219203471\\
-0.0234375	30.8242336154911\\
-0.02294921875	30.7104422972727\\
-0.0224609375	30.5970448899482\\
-0.02197265625	30.4840383545996\\
-0.021484375	30.3714196885542\\
-0.02099609375	30.2591859248168\\
-0.0205078125	30.1473341315117\\
-0.02001953125	30.0358614113361\\
-0.01953125	29.9247649010229\\
-0.01904296875	29.814041770814\\
-0.0185546875	29.7036892239443\\
-0.01806640625	29.5937044961336\\
-0.017578125	29.4840848550896\\
-0.01708984375	29.3748276000195\\
-0.0166015625	29.2659300611502\\
-0.01611328125	29.1573895992587\\
-0.015625	29.0492036052093\\
-0.01513671875	28.9413694995014\\
-0.0146484375	28.8338847318234\\
-0.01416015625	28.7267467806173\\
-0.013671875	28.6199531526483\\
-0.01318359375	28.513501382585\\
-0.0126953125	28.4073890325852\\
-0.01220703125	28.3016136918904\\
-0.01171875	28.1961729764272\\
-0.01123046875	28.0910645284154\\
-0.0107421875	27.9862860159839\\
-0.01025390625	27.8818351327934\\
-0.009765625	27.777709597665\\
-0.00927734375	27.6739071542163\\
-0.0087890625	27.5704255705037\\
-0.00830078125	27.4672626386705\\
-0.0078125	27.3644161746022\\
-0.00732421875	27.2618840175869\\
-0.0068359375	27.1596640299821\\
-0.00634765625	27.0577540968875\\
-0.005859375	26.9561521258228\\
-0.00537109375	26.8548560464121\\
-0.0048828125	26.7538638100727\\
-0.00439453125	26.6531733897101\\
-0.00390625	26.5527827794179\\
-0.00341796875	26.4526899941828\\
-0.0029296875	26.3528930695947\\
-0.00244140625	26.253390061562\\
-0.001953125	26.1541790460311\\
-0.00146484375	26.0552581187117\\
-0.0009765625	25.9566253948055\\
-0.00048828125	25.8582790087405\\
0	25.7602171139093\\
0.00048828125	25.8582790087405\\
0.0009765625	25.9566253948055\\
0.00146484375	26.0552581187117\\
0.001953125	26.1541790460311\\
0.00244140625	26.253390061562\\
0.0029296875	26.3528930695947\\
0.00341796875	26.4526899941828\\
0.00390625	26.5527827794179\\
0.00439453125	26.6531733897101\\
0.0048828125	26.7538638100727\\
0.00537109375	26.8548560464121\\
0.005859375	26.9561521258228\\
0.00634765625	27.0577540968875\\
0.0068359375	27.1596640299821\\
0.00732421875	27.2618840175869\\
0.0078125	27.3644161746022\\
0.00830078125	27.4672626386705\\
0.0087890625	27.5704255705037\\
0.00927734375	27.6739071542163\\
0.009765625	27.777709597665\\
0.01025390625	27.8818351327934\\
0.0107421875	27.9862860159839\\
0.01123046875	28.0910645284154\\
0.01171875	28.1961729764272\\
0.01220703125	28.3016136918904\\
0.0126953125	28.4073890325852\\
0.01318359375	28.513501382585\\
0.013671875	28.6199531526483\\
0.01416015625	28.7267467806173\\
0.0146484375	28.8338847318234\\
0.01513671875	28.9413694995014\\
0.015625	29.0492036052093\\
0.01611328125	29.1573895992587\\
0.0166015625	29.2659300611502\\
0.01708984375	29.3748276000195\\
0.017578125	29.4840848550896\\
0.01806640625	29.5937044961336\\
0.0185546875	29.7036892239443\\
0.01904296875	29.814041770814\\
0.01953125	29.9247649010229\\
0.02001953125	30.0358614113361\\
0.0205078125	30.1473341315117\\
0.02099609375	30.2591859248168\\
0.021484375	30.3714196885542\\
0.02197265625	30.4840383545996\\
0.0224609375	30.5970448899482\\
0.02294921875	30.7104422972727\\
0.0234375	30.8242336154911\\
0.02392578125	30.9384219203471\\
0.0244140625	31.0530103249997\\
0.02490234375	31.168001980626\\
0.025390625	31.2834000770354\\
0.02587890625	31.399207843295\\
0.0263671875	31.5154285483684\\
0.02685546875	31.6320655017665\\
0.02734375	31.7491220542109\\
0.02783203125	31.8666015983115\\
0.0283203125	31.9845075692563\\
0.02880859375	32.102843445516\\
0.029296875	32.2216127495616\\
0.02978515625	32.3408190485975\\
0.0302734375	32.4604659553086\\
0.03076171875	32.5805571286224\\
0.03125	32.7010962744872\\
0.03173828125	32.8220871466647\\
0.0322265625	32.9435335475403\\
0.03271484375	33.0654393289487\\
0.033203125	33.1878083930167\\
0.03369140625	33.3106446930228\\
0.0341796875	33.4339522342752\\
0.03466796875	33.5577350750074\\
0.03515625	33.6819973272919\\
0.03564453125	33.806743157973\\
0.0361328125	33.9319767896185\\
0.03662109375	34.0577025014924\\
0.037109375	34.1839246305453\\
0.03759765625	34.3106475724281\\
0.0380859375	34.4378757825245\\
0.03857421875	34.5656137770067\\
0.0390625	34.6938661339119\\
0.03955078125	34.822637494242\\
0.0400390625	34.9519325630863\\
0.04052734375	35.0817561107673\\
0.041015625	35.2121129740109\\
0.04150390625	35.3430080571413\\
0.0419921875	35.4744463333007\\
0.04248046875	35.6064328456951\\
0.04296875	35.7389727088661\\
0.04345703125	35.8720711099898\\
0.0439453125	36.0057333102028\\
0.04443359375	36.1399646459565\\
0.044921875	36.2747705304005\\
0.04541015625	36.4101564547945\\
0.0458984375	36.5461279899504\\
0.04638671875	36.6826907877057\\
0.046875	36.8198505824265\\
0.04736328125	36.9576131925447\\
0.0478515625	37.0959845221253\\
0.04833984375	37.2349705624687\\
0.048828125	37.3745773937465\\
0.04931640625	37.5148111866713\\
0.0498046875	37.6556782042024\\
0.05029296875	37.7971848032879\\
0.05078125	37.9393374366425\\
0.05126953125	38.0821426545636\\
0.0517578125	38.225607106785\\
0.05224609375	38.3697375443703\\
0.052734375	38.5145408216449\\
0.05322265625	38.6600238981693\\
0.0537109375	38.8061938407522\\
0.05419921875	38.9530578255074\\
0.0546875	39.1006231399505\\
0.05517578125	39.248897185141\\
0.0556640625	39.3978874778672\\
0.05615234375	39.5476016528758\\
0.056640625	39.6980474651463\\
0.05712890625	39.8492327922119\\
0.0576171875	40.0011656365265\\
0.05810546875	40.1538541278775\\
0.05859375	40.3073065258489\\
0.05908203125	40.4615312223284\\
0.0595703125	40.6165367440675\\
0.06005859375	40.7723317552864\\
0.060546875	40.9289250603302\\
0.06103515625	41.0863256063747\\
0.0615234375	41.2445424861809\\
0.06201171875	41.4035849409002\\
0.0625	41.5634623629288\\
0.06298828125	41.7241842988124\\
0.0634765625	41.8857604522007\\
0.06396484375	42.0482006868501\\
0.064453125	42.2115150296774\\
0.06494140625	42.3757136738599\\
0.0654296875	42.5408069819836\\
0.06591796875	42.7068054892396\\
0.06640625	42.8737199066636\\
0.06689453125	43.041561124421\\
0.0673828125	43.2103402151351\\
0.06787109375	43.3800684372546\\
0.068359375	43.5507572384599\\
0.06884765625	43.7224182591068\\
0.0693359375	43.8950633357015\\
0.06982421875	44.0687045044066\\
0.0703125	44.2433540045726\\
0.07080078125	44.4190242822917\\
0.0712890625	44.5957279939688\\
0.07177734375	44.7734780099046\\
0.072265625	44.9522874178843\\
0.07275390625	45.1321695267663\\
0.0732421875	45.3131378700622\\
0.07373046875	45.4952062095023\\
0.07421875	45.6783885385733\\
0.07470703125	45.8626990860235\\
0.0751953125	46.0481523193194\\
0.07568359375	46.2347629480455\\
0.076171875	46.4225459272296\\
0.07666015625	46.6115164605816\\
0.0771484375	46.8016900036287\\
0.07763671875	46.9930822667278\\
0.078125	47.1857092179337\\
0.07861328125	47.3795870857051\\
0.0791015625	47.5747323614201\\
0.07958984375	47.7711618016748\\
0.080078125	47.9688924303371\\
0.08056640625	48.1679415403225\\
0.0810546875	48.3683266950559\\
0.08154296875	48.5700657295805\\
0.08203125	48.7731767512714\\
0.08251953125	48.9776781401052\\
0.0830078125	49.1835885484373\\
0.08349609375	49.3909269002256\\
0.083984375	49.599712389642\\
0.08447265625	49.8099644790044\\
0.0849609375	50.0217028959496\\
0.08544921875	50.2349476297713\\
0.0859375	50.4497189268289\\
0.08642578125	50.6660372849319\\
0.0869140625	50.8839234465917\\
0.08740234375	51.1033983910211\\
0.087890625	51.324483324756\\
0.08837890625	51.5471996707532\\
0.0888671875	51.771569055815\\
0.08935546875	51.9976132961657\\
0.08984375	52.2253543809988\\
0.09033203125	52.4548144537874\\
0.0908203125	52.6860157911394\\
0.09130859375	52.9189807789515\\
0.091796875	53.1537318855925\\
0.09228515625	53.3902916318307\\
0.0927734375	53.6286825571802\\
0.09326171875	53.8689271823172\\
0.09375	54.1110479671856\\
0.09423828125	54.3550672643722\\
0.0947265625	54.6010072672935\\
0.09521484375	54.8488899526931\\
0.095703125	55.0987370169074\\
0.09619140625	55.3505698052922\\
0.0966796875	55.6044092341703\\
0.09716796875	55.8602757045763\\
0.09765625	56.1181890070289\\
0.09814453125	56.3781682164795\\
0.0986328125	56.6402315765145\\
0.09912109375	56.9043963718076\\
0.099609375	57.1706787877261\\
0.10009765625	57.4390937559041\\
0.1005859375	57.7096547844931\\
0.10107421875	57.9823737716933\\
0.1015625	58.2572608010469\\
0.10205078125	58.5343239168693\\
0.1025390625	58.8135688780453\\
0.10302734375	59.0949988882983\\
0.103515625	59.3786143009015\\
0.10400390625	59.6644122956473\\
0.1044921875	59.9523865257586\\
0.10498046875	60.2425267322752\\
0.10546875	60.534818323311\\
0.10595703125	60.8292419154428\\
0.1064453125	61.1257728343684\\
0.10693359375	61.4243805718709\\
0.107421875	61.7250281960424\\
0.10791015625	62.027671711683\\
0.1083984375	62.3322593677811\\
0.10888671875	62.6387309090471\\
0.109375	62.9470167686044\\
0.10986328125	63.2570371991621\\
0.1103515625	63.568701340342\\
0.11083984375	63.8819062203057\\
0.111328125	64.1965356904819\\
0.11181640625	64.5124592930508\\
0.1123046875	64.8295310619277\\
0.11279296875	65.1475882593749\\
0.11328125	65.4664500520578\\
0.11376953125	65.7859161324601\\
0.1142578125	66.1057652940392\\
0.11474609375	66.4257539714961\\
0.115234375	66.7456147609797\\
0.11572265625	67.0650549390971\\
0.1162109375	67.3837550041424\\
0.11669921875	67.7013672681394\\
0.1171875	68.017514533941\\
0.11767578125	68.3317888977879\\
0.1181640625	68.6437507242175\\
0.11865234375	68.9529278468559\\
0.119140625	69.2588150552028\\
0.11962890625	69.5608739336723\\
0.1201171875	69.8585331244189\\
0.12060546875	70.1511890893943\\
0.12109375	70.4382074490054\\
0.12158203125	70.7189249739161\\
0.1220703125	70.9926523023959\\
0.12255859375	71.2586774471774\\
0.123046875	71.5162701426399\\
0.12353515625	71.7646870645223\\
0.1240234375	72.0031779302089\\
0.12451171875	72.2309924578158\\
0.125	72.4473881274965\\
0.12548828125	72.6516386495387\\
0.1259765625	72.8430430026216\\
0.12646484375	73.0209348642165\\
0.126953125	73.1846922163075\\
0.12744140625	73.3337468762969\\
0.1279296875	73.4675936784358\\
0.12841796875	73.5857990181201\\
0.12890625	73.6880084724114\\
0.12939453125	73.7739532265594\\
0.1298828125	73.8434550684277\\
0.13037109375	73.8964297596202\\
0.130859375	73.9328886512046\\
0.13134765625	73.9529384799204\\
0.1318359375	73.9567793529796\\
0.13232421875	73.9447010012956\\
0.1328125	73.917077447309\\
0.13330078125	73.8743602902582\\
0.1337890625	73.8170708554878\\
0.13427734375	73.7457914830375\\
0.134765625	73.6611562437022\\
0.13525390625	73.5638413684446\\
0.1357421875	73.4545556612674\\
0.13623046875	73.3340311388971\\
0.13671875	73.2030141058099\\
0.13720703125	73.0622568335104\\
0.1376953125	72.9125099714082\\
0.13818359375	72.7545157757752\\
0.138671875	72.5890022052092\\
0.13916015625	72.4166778972693\\
0.1396484375	72.2382280123518\\
0.14013671875	72.0543109079723\\
0.140625	71.8655555892563\\
0.14111328125	71.6725598693913\\
0.1416015625	71.4758891663861\\
0.14208984375	71.2760758591411\\
0.142578125	71.073619125666\\
0.14306640625	70.8689851887448\\
0.1435546875	70.6626078985711\\
0.14404296875	70.4548895874262\\
0.14453125	70.2462021377221\\
0.14501953125	70.0368882113541\\
0.1455078125	69.8272625949661\\
0.14599609375	69.6176136221002\\
0.146484375	69.4082046393224\\
0.14697265625	69.1992754889177\\
0.1474609375	68.9910439857956\\
0.14794921875	68.7837073707111\\
0.1484375	68.5774437257725\\
0.14892578125	68.3724133415829\\
0.1494140625	68.1687600282214\\
0.14990234375	67.9666123646566\\
0.150390625	67.7660848831695\\
0.15087890625	67.5672791870118\\
0.1513671875	67.3702850008107\\
0.15185546875	67.1751811542667\\
0.15234375	66.9820365005324\\
0.15283203125	66.7909107712148\\
0.1533203125	66.6018553704373\\
0.15380859375	66.414914110693\\
0.154296875	66.2301238934116\\
0.15478515625	66.0475153373026\\
0.1552734375	65.867113357561\\
0.15576171875	65.6889376990166\\
0.15625	65.5130034262622\\
0.15673828125	65.3393213737003\\
0.1572265625	65.1678985583398\\
0.15771484375	64.9987385580634\\
0.158203125	64.8318418579296\\
0.15869140625	64.667206166946\\
0.1591796875	64.5048267076057\\
0.15966796875	64.3446964803284\\
0.16015625	64.1868065048076\\
0.16064453125	64.0311460401381\\
0.1611328125	63.8777027854463\\
0.16162109375	63.7264630626407\\
0.162109375	63.5774119827593\\
0.16259765625	63.4305335972969\\
0.1630859375	63.2858110357702\\
0.16357421875	63.1432266306932\\
0.1640625	63.0027620310352\\
0.16455078125	62.8643983051434\\
0.1650390625	62.7281160340415\\
0.16552734375	62.5938953959357\\
0.166015625	62.4617162426839\\
0.16650390625	62.3315581689355\\
0.1669921875	62.2034005745749\\
0.16748046875	62.0772227210591\\
0.16796875	61.9530037821811\\
0.16845703125	61.8307228897556\\
0.1689453125	61.7103591746706\\
0.16943359375	61.5918918037161\\
0.169921875	61.4753000125686\\
0.17041015625	61.3605631352692\\
0.1708984375	61.2476606305134\\
0.17138671875	61.1365721050319\\
0.171875	61.0272773343322\\
0.17236328125	60.9197562810362\\
0.1728515625	60.8139891110296\\
0.17333984375	60.7099562076235\\
0.173828125	60.6076381839165\\
0.17431640625	60.5070158935127\\
0.1748046875	60.4080704397549\\
0.17529296875	60.3107831836117\\
0.17578125	60.2151357503388\\
0.17626953125	60.1211100350363\\
0.1767578125	60.0286882072056\\
0.17724609375	59.9378527144015\\
0.177734375	59.8485862850655\\
0.17822265625	59.7608719306236\\
0.1787109375	59.6746929469237\\
0.17919921875	59.5900329150688\\
0.1796875	59.5068757017225\\
0.18017578125	59.4252054589276\\
0.1806640625	59.3450066234999\\
0.18115234375	59.2662639160317\\
0.181640625	59.1889623395633\\
0.18212890625	59.1130871779411\\
0.1826171875	59.0386239939131\\
0.18310546875	58.9655586269878\\
0.18359375	58.8938771910818\\
0.18408203125	58.8235660719898\\
0.1845703125	58.7546119246959\\
0.18505859375	58.6870016705474\\
0.185546875	58.6207224943156\\
0.18603515625	58.5557618411575\\
0.1865234375	58.492107413494\\
0.18701171875	58.4297471678247\\
0.1875	58.3686693114874\\
0.18798828125	58.3088622993736\\
0.1884765625	58.2503148306194\\
0.18896484375	58.1930158452701\\
0.189453125	58.1369545209361\\
0.18994140625	58.0821202694433\\
0.1904296875	58.0285027334895\\
0.19091796875	57.9760917833088\\
0.19140625	57.9248775133498\\
0.19189453125	57.8748502389808\\
0.1923828125	57.8260004932133\\
0.19287109375	57.7783190234584\\
0.193359375	57.7317967883199\\
0.19384765625	57.6864249544149\\
0.1943359375	57.642194893242\\
0.19482421875	57.5990981780864\\
0.1953125	57.5571265809705\\
0.19580078125	57.5162720696498\\
0.1962890625	57.4765268046525\\
0.19677734375	57.437883136375\\
0.197265625	57.4003336022177\\
0.19775390625	57.3638709237781\\
0.1982421875	57.3284880040902\\
0.19873046875	57.2941779249198\\
0.19921875	57.2609339441059\\
0.19970703125	57.228749492959\\
0.2001953125	57.1976181737095\\
0.20068359375	57.1675337570087\\
0.201171875	57.1384901794809\\
0.20166015625	57.1104815413285\\
0.2021484375	57.0835021039883\\
0.20263671875	57.0575462878385\\
0.203125	57.0326086699559\\
0.20361328125	57.0086839819242\\
0.2041015625	56.9857671076932\\
0.20458984375	56.9638530814841\\
0.205078125	56.942937085744\\
0.20556640625	56.9230144491514\\
0.2060546875	56.9040806446634\\
0.20654296875	56.8861312876129\\
0.20703125	56.8691621338503\\
0.20751953125	56.8531690779298\\
0.2080078125	56.8381481513365\\
0.20849609375	56.8240955207636\\
0.208984375	56.8110074864239\\
0.20947265625	56.798880480406\\
0.2099609375	56.7877110650713\\
0.21044921875	56.7774959314887\\
0.2109375	56.768231897907\\
0.21142578125	56.7599159082673\\
0.2119140625	56.7525450307493\\
0.21240234375	56.7461164563529\\
0.212890625	56.7406274975148\\
0.21337890625	56.7360755867602\\
0.2138671875	56.7324582753818\\
0.21435546875	56.7297732321537\\
0.21484375	56.728018242077\\
0.21533203125	56.7271912051468\\
0.2158203125	56.7272901351564\\
0.21630859375	56.72831315852\\
0.216796875	56.7302585131272\\
0.21728515625	56.733124547217\\
0.2177734375	56.7369097182782\\
0.21826171875	56.7416125919695\\
0.21875	56.7472318410614\\
0.21923828125	56.753766244396\\
0.2197265625	56.7612146858669\\
0.22021484375	56.769576153414\\
0.220703125	56.7788497380331\\
0.22119140625	56.7890346328021\\
0.2216796875	56.8001301319151\\
0.22216796875	56.8121356297314\\
0.22265625	56.8250506198318\\
0.22314453125	56.8388746940817\\
0.2236328125	56.8536075417017\\
0.22412109375	56.8692489483427\\
0.224609375	56.8857987951608\\
0.22509765625	56.9032570578956\\
0.2255859375	56.9216238059454\\
0.22607421875	56.9408992014398\\
0.2265625	56.9610834983061\\
0.22705078125	56.9821770413292\\
0.2275390625	57.0041802652003\\
0.22802734375	57.0270936935541\\
0.228515625	57.0509179379905\\
0.22900390625	57.0756536970807\\
0.2294921875	57.10130175535\\
0.22998046875	57.1278629822429\\
0.23046875	57.1553383310561\\
0.23095703125	57.1837288378483\\
0.2314453125	57.2130356203133\\
0.23193359375	57.2432598766221\\
0.232421875	57.2744028842222\\
0.23291015625	57.3064659985947\\
0.2333984375	57.3394506519669\\
0.23388671875	57.3733583519724\\
0.234375	57.4081906802544\\
0.23486328125	57.4439492910126\\
0.2353515625	57.4806359094823\\
0.23583984375	57.5182523303475\\
0.236328125	57.5568004160728\\
0.23681640625	57.5962820951637\\
0.2373046875	57.6366993603307\\
0.23779296875	57.6780542665689\\
0.23828125	57.7203489291317\\
0.23876953125	57.7635855214035\\
0.2392578125	57.8077662726549\\
0.23974609375	57.8528934656806\\
0.240234375	57.8989694343028\\
0.24072265625	57.945996560743\\
0.2412109375	57.9939772728419\\
0.24169921875	58.0429140411246\\
0.2421875	58.0928093756996\\
0.24267578125	58.1436658229817\\
0.2431640625	58.1954859622244\\
0.24365234375	58.2482724018533\\
0.244140625	58.3020277755879\\
0.24462890625	58.3567547383372\\
0.2451171875	58.4124559618555\\
0.24560546875	58.4691341301425\\
0.24609375	58.5267919345791\\
0.24658203125	58.58543206877\\
0.2470703125	58.6450572230898\\
0.24755859375	58.7056700789039\\
0.248046875	58.7672733024497\\
0.24853515625	58.8298695383581\\
0.2490234375	58.8934614027931\\
0.24951171875	58.9580514761852\\
0.25	59.0236422955391\\
0.25048828125	59.0902363462866\\
0.2509765625	59.1578360536599\\
0.25146484375	59.2264437735601\\
0.251953125	59.2960617828848\\
0.25244140625	59.3666922692929\\
0.2529296875	59.4383373203652\\
0.25341796875	59.5109989121354\\
0.25390625	59.5846788969472\\
0.25439453125	59.6593789906059\\
0.2548828125	59.7351007587798\\
0.25537109375	59.8118456026152\\
0.255859375	59.889614743514\\
0.25634765625	59.9684092070348\\
0.2568359375	60.0482298058655\\
0.25732421875	60.1290771218113\\
0.2578125	60.2109514867573\\
0.25830078125	60.29385296254\\
0.2587890625	60.3777813196696\\
0.25927734375	60.4627360148473\\
0.259765625	60.5487161672092\\
0.26025390625	60.6357205332314\\
0.2607421875	60.72374748023\\
0.26123046875	60.812794958374\\
0.26171875	60.9028604711516\\
0.26220703125	60.993941044193\\
0.2626953125	61.0860331923862\\
0.26318359375	61.1791328851886\\
0.263671875	61.2732355100558\\
0.26416015625	61.368335833895\\
0.2646484375	61.4644279624508\\
0.26513671875	61.5615052975332\\
0.265625	61.6595604919845\\
0.26611328125	61.7585854022948\\
0.2666015625	61.8585710387595\\
0.26708984375	61.9595075130859\\
0.267578125	62.0613839833395\\
0.26806640625	62.1641885961341\\
0.2685546875	62.2679084259621\\
0.26904296875	62.3725294115734\\
0.26953125	62.4780362892932\\
0.27001953125	62.5844125231972\\
0.2705078125	62.6916402320526\\
0.27099609375	62.7997001129383\\
0.271484375	62.9085713614778\\
0.27197265625	63.0182315886079\\
0.2724609375	63.1286567338464\\
0.27294921875	63.2398209749964\\
0.2734375	63.3516966342862\\
0.27392578125	63.4642540809204\\
0.2744140625	63.5774616300708\\
0.27490234375	63.6912854383436\\
0.275390625	63.8056893957941\\
0.27587890625	63.920635014596\\
0.2763671875	64.0360813145015\\
0.27685546875	64.1519847052777\\
0.27734375	64.2682988663523\\
0.27783203125	64.3849746239462\\
0.2783203125	64.5019598260413\\
0.27880859375	64.6191992155818\\
0.279296875	64.7366343023943\\
0.27978515625	64.8542032343787\\
0.2802734375	64.9718406686062\\
0.28076171875	65.0894776430622\\
0.28125	65.2070414498618\\
0.28173828125	65.3244555108699\\
0.2822265625	65.4416392567813\\
0.28271484375	65.5585080108215\\
0.283203125	65.6749728783599\\
0.28369140625	65.7909406438532\\
0.2841796875	65.9063136766764\\
0.28466796875	66.0209898475115\\
0.28515625	66.1348624571328\\
0.28564453125	66.247820179516\\
0.2861328125	66.3597470213736\\
0.28662109375	66.4705223002928\\
0.287109375	66.5800206437986\\
0.28759765625	66.6881120117244\\
0.2880859375	66.7946617443798\\
0.28857421875	66.8995306390261\\
0.2890625	67.0025750572004\\
0.28955078125	67.103647065423\\
0.2900390625	67.2025946117455\\
0.29052734375	67.2992617405183\\
0.291015625	67.3934888475847\\
0.29150390625	67.4851129779052\\
0.2919921875	67.5739681673364\\
0.29248046875	67.6598858299633\\
0.29296875	67.7426951919555\\
0.29345703125	67.8222237724538\\
0.2939453125	67.898297911442\\
0.29443359375	67.9707433439422\\
0.294921875	68.0393858191844\\
0.29541015625	68.1040517626751\\
0.2958984375	68.1645689782952\\
0.29638671875	68.2207673867516\\
0.296875	68.2724797958507\\
0.29736328125	68.319542697233\\
0.2978515625	68.3617970833847\\
0.29833984375	68.3990892779601\\
0.298828125	68.4312717717236\\
0.29931640625	68.4582040557908\\
0.2998046875	68.4797534433501\\
0.30029296875	68.4957958706187\\
0.30078125	68.5062166676041\\
0.30126953125	68.5109112891508\\
0.3017578125	68.5097859969033\\
0.30224609375	68.5027584831232\\
0.302734375	68.4897584278324\\
0.30322265625	68.4707279814636\\
0.3037109375	68.4456221660871\\
0.30419921875	68.4144091893734\\
0.3046875	68.3770706666301\\
0.30517578125	68.3336017476016\\
0.3056640625	68.284011146118\\
0.30615234375	68.2283210721512\\
0.306640625	68.1665670672919\\
0.30712890625	68.0987977461306\\
0.3076171875	68.0250744473749\\
0.30810546875	67.9454707998611\\
0.30859375	67.8600722097326\\
0.30908203125	67.7689752760915\\
0.3095703125	67.6722871432413\\
0.31005859375	67.5701247982779\\
0.310546875	67.4626143232366\\
0.31103515625	67.3498901112505\\
0.3115234375	67.2320940562187\\
0.31201171875	67.1093747253675\\
0.3125	66.9818865237956\\
0.31298828125	66.8497888596413\\
0.3134765625	66.7132453179739\\
0.31396484375	66.5724228507933\\
0.314453125	66.4274909898391\\
0.31494140625	66.2786210880494\\
0.3154296875	66.1259855947206\\
0.31591796875	65.9697573685557\\
0.31640625	65.8101090319622\\
0.31689453125	65.6472123691512\\
0.3173828125	65.481237769824\\
0.31787109375	65.312353719519\\
0.318359375	65.1407263370277\\
0.31884765625	64.9665189587063\\
0.3193359375	64.789891768983\\
0.31982421875	64.6110014759142\\
0.3203125	64.4300010302566\\
0.32080078125	64.2470393862126\\
0.3212890625	64.0622613017631\\
0.32177734375	63.875807176295\\
0.322265625	63.6878129231234\\
0.32275390625	63.4984098744038\\
0.3232421875	63.3077247158989\\
0.32373046875	63.1158794490663\\
0.32421875	62.9229913779535\\
0.32470703125	62.7291731184659\\
0.3251953125	62.5345326276301\\
0.32568359375	62.3391732506006\\
0.326171875	62.1431937832412\\
0.32666015625	61.9466885482759\\
0.3271484375	61.7497474830814\\
0.32763671875	61.5524562373818\\
0.328125	61.3548962791927\\
0.32861328125	61.1571450075298\\
0.3291015625	60.9592758705042\\
0.32958984375	60.7613584875709\\
0.330078125	60.5634587748049\\
0.33056640625	60.3656390722094\\
0.3310546875	60.1679582721489\\
0.33154296875	59.9704719481342\\
0.33203125	59.7732324832485\\
0.33251953125	59.5762891976191\\
0.3330078125	59.3796884744045\\
0.33349609375	59.1834738838513\\
0.333984375	58.9876863050402\\
0.33447265625	58.7923640450029\\
0.3349609375	58.5975429549467\\
0.33544921875	58.4032565433763\\
0.3359375	58.209536085948\\
0.33642578125	58.0164107319283\\
0.3369140625	57.8239076071678\\
0.33740234375	57.6320519135297\\
0.337890625	57.4408670247434\\
0.33837890625	57.2503745786749\\
0.3388671875	57.0605945660268\\
0.33935546875	56.8715454154959\\
0.33984375	56.6832440754406\\
0.34033203125	56.4957060921114\\
0.3408203125	56.3089456845129\\
0.34130859375	56.1229758159813\\
0.341796875	55.937808262555\\
0.34228515625	55.753453678232\\
0.3427734375	55.569921657211\\
0.34326171875	55.3872207932072\\
0.34375	55.2053587359498\\
0.34423828125	55.0243422449555\\
0.3447265625	54.8441772406851\\
0.34521484375	54.6648688531782\\
0.345703125	54.4864214682725\\
0.34619140625	54.3088387714988\\
0.3466796875	54.1321237897599\\
0.34716796875	53.9562789308762\\
0.34765625	53.7813060211006\\
0.34814453125	53.6072063406923\\
0.3486328125	53.43398065763\\
0.34912109375	53.2616292595618\\
0.349609375	53.0901519840642\\
0.35009765625	52.9195482472967\\
0.3505859375	52.7498170711248\\
0.35107421875	52.5809571087875\\
0.3515625	52.4129666691821\\
0.35205078125	52.2458437398294\\
0.3525390625	52.0795860085899\\
0.35302734375	51.9141908841916\\
0.353515625	51.7496555156297\\
0.35400390625	51.5859768104951\\
0.3544921875	51.4231514522865\\
0.35498046875	51.2611759167593\\
0.35546875	51.1000464873601\\
0.35595703125	50.9397592697954\\
0.3564453125	50.7803102057773\\
0.35693359375	50.6216950859925\\
0.357421875	50.4639095623334\\
0.35791015625	50.3069491594309\\
0.3583984375	50.1508092855254\\
0.35888671875	49.9954852427122\\
0.359375	49.8409722365937\\
0.35986328125	49.6872653853705\\
0.3603515625	49.5343597284015\\
0.36083984375	49.3822502342613\\
0.361328125	49.2309318083229\\
0.36181640625	49.0803992998894\\
0.3623046875	48.9306475089022\\
0.36279296875	48.7816711922442\\
0.36328125	48.6334650696657\\
0.36376953125	48.4860238293477\\
0.3642578125	48.3393421331274\\
0.36474609375	48.1934146214005\\
0.365234375	48.048235917721\\
0.36572265625	47.9038006331126\\
0.3662109375	47.7601033701099\\
0.36669921875	47.6171387265413\\
0.3671875	47.4749012990723\\
0.36767578125	47.3333856865173\\
0.3681640625	47.1925864929367\\
0.36865234375	47.0524983305291\\
0.369140625	46.9131158223312\\
0.36962890625	46.7744336047344\\
0.3701171875	46.6364463298307\\
0.37060546875	46.4991486675967\\
0.37109375	46.3625353079212\\
0.37158203125	46.2266009624929\\
0.3720703125	46.0913403665471\\
0.37255859375	45.9567482804864\\
0.373046875	45.8228194913799\\
0.37353515625	45.6895488143454\\
0.3740234375	45.5569310938263\\
0.37451171875	45.4249612047648\\
0.375	45.2936340536804\\
0.37548828125	45.1629445796558\\
0.3759765625	45.0328877552402\\
0.37646484375	44.9034585872699\\
0.376953125	44.7746521176136\\
0.37744140625	44.6464634238481\\
0.3779296875	44.5188876198643\\
0.37841796875	44.391919856412\\
0.37890625	44.2655553215835\\
0.37939453125	44.1397892412436\\
0.3798828125	44.0146168794048\\
0.38037109375	43.8900335385539\\
0.380859375	43.7660345599326\\
0.38134765625	43.6426153237733\\
0.3818359375	43.5197712494956\\
0.38232421875	43.397497795863\\
0.3828125	43.2757904611034\\
0.38330078125	43.154644782997\\
0.3837890625	43.0340563389325\\
0.38427734375	42.9140207459313\\
0.384765625	42.7945336606471\\
0.38525390625	42.6755907793375\\
0.3857421875	42.5571878378125\\
0.38623046875	42.439320611359\\
0.38671875	42.3219849146454\\
0.38720703125	42.2051766016064\\
0.3876953125	42.0888915653081\\
0.38818359375	41.973125737798\\
0.388671875	41.8578750899378\\
0.38916015625	41.743135631222\\
0.3896484375	41.628903409582\\
0.39013671875	41.5151745111798\\
0.390625	41.4019450601863\\
0.39111328125	41.2892112185532\\
0.3916015625	41.1769691857714\\
0.39208984375	41.0652151986224\\
0.392578125	40.9539455309202\\
0.39306640625	40.8431564932467\\
0.3935546875	40.7328444326786\\
0.39404296875	40.6230057325098\\
0.39453125	40.5136368119663\\
0.39501953125	40.4047341259177\\
0.3955078125	40.2962941645824\\
0.39599609375	40.18831345323\\
0.396484375	40.0807885518792\\
0.39697265625	39.9737160549933\\
0.3974609375	39.8670925911719\\
0.39794921875	39.7609148228412\\
0.3984375	39.6551794459419\\
0.39892578125	39.5498831896153\\
0.3994140625	39.4450228158874\\
0.39990234375	39.3405951193542\\
0.400390625	39.2365969268638\\
0.40087890625	39.1330250971989\\
0.4013671875	39.0298765207597\\
0.40185546875	38.9271481192457\\
0.40234375	38.824836845339\\
0.40283203125	38.7229396823865\\
0.4033203125	38.621453644084\\
0.40380859375	38.5203757741607\\
0.404296875	38.4197031460647\\
0.40478515625	38.3194328626488\\
0.4052734375	38.2195620558595\\
0.40576171875	38.1200878864253\\
0.40625	38.0210075435479\\
0.40673828125	37.9223182445946\\
0.4072265625	37.8240172347926\\
0.40771484375	37.7261017869245\\
0.408203125	37.6285692010274\\
0.40869140625	37.5314168040917\\
0.4091796875	37.4346419497641\\
0.40966796875	37.3382420180514\\
0.41015625	37.2422144150279\\
0.41064453125	37.1465565725433\\
0.4111328125	37.051265947935\\
0.41162109375	36.9563400237408\\
0.412109375	36.8617763074155\\
0.41259765625	36.76757233105\\
0.4130859375	36.6737256510916\\
0.41357421875	36.5802338480683\\
0.4140625	36.4870945263151\\
0.41455078125	36.3943053137026\\
0.4150390625	36.301863861369\\
0.41552734375	36.2097678434539\\
0.416015625	36.1180149568356\\
0.41650390625	36.0266029208703\\
0.4169921875	35.9355294771346\\
0.41748046875	35.8447923891703\\
0.41796875	35.7543894422319\\
0.41845703125	35.6643184430373\\
0.4189453125	35.5745772195204\\
0.41943359375	35.485163620587\\
0.419921875	35.396075515873\\
0.42041015625	35.3073107955062\\
0.4208984375	35.2188673698686\\
0.42138671875	35.1307431693643\\
0.421875	35.0429361441879\\
0.42236328125	34.9554442640958\\
0.4228515625	34.8682655181817\\
0.42333984375	34.7813979146519\\
0.423828125	34.6948394806067\\
0.42431640625	34.6085882618206\\
0.4248046875	34.522642322529\\
0.42529296875	34.4369997452139\\
0.42578125	34.3516586303942\\
0.42626953125	34.2666170964185\\
0.4267578125	34.1818732792587\\
0.42724609375	34.0974253323089\\
0.427734375	34.0132714261836\\
0.42822265625	33.9294097485206\\
0.4287109375	33.8458385037857\\
0.42919921875	33.7625559130793\\
0.4296875	33.6795602139459\\
0.43017578125	33.5968496601855\\
0.4306640625	33.5144225216684\\
0.43115234375	33.4322770841503\\
0.431640625	33.3504116490919\\
0.43212890625	33.2688245334792\\
0.4326171875	33.1875140696466\\
0.43310546875	33.1064786051026\\
0.43359375	33.0257165023564\\
0.43408203125	32.9452261387485\\
0.4345703125	32.8650059062817\\
0.43505859375	32.7850542114553\\
0.435546875	32.7053694751009\\
0.43603515625	32.6259501322204\\
0.4365234375	32.5467946318262\\
0.43701171875	32.4679014367833\\
0.4375	32.3892690236531\\
0.43798828125	32.3108958825396\\
0.4384765625	32.2327805169371\\
0.43896484375	32.1549214435809\\
0.439453125	32.0773171922978\\
0.43994140625	31.9999663058614\\
0.4404296875	31.9228673398458\\
0.44091796875	31.8460188624847\\
0.44140625	31.7694194545291\\
0.44189453125	31.6930677091095\\
0.4423828125	31.6169622315978\\
0.44287109375	31.5411016394718\\
0.443359375	31.4654845621819\\
0.44384765625	31.3901096410184\\
0.4443359375	31.3149755289814\\
0.44482421875	31.240080890652\\
0.4453125	31.165424402065\\
0.44580078125	31.091004750583\\
0.4462890625	31.0168206347735\\
0.44677734375	30.9428707642855\\
0.447265625	30.8691538597287\\
0.44775390625	30.7956686525549\\
0.4482421875	30.7224138849396\\
0.44873046875	30.6493883096658\\
0.44921875	30.5765906900094\\
0.44970703125	30.5040197996255\\
0.4501953125	30.4316744224366\\
0.45068359375	30.3595533525217\\
0.451171875	30.287655394008\\
0.45166015625	30.2159793609621\\
0.4521484375	30.1445240772841\\
0.45263671875	30.0732883766027\\
0.453125	30.0022711021709\\
0.45361328125	29.9314711067638\\
0.4541015625	29.8608872525775\\
0.45458984375	29.7905184111291\\
0.455078125	29.7203634631581\\
0.45556640625	29.6504212985286\\
0.4560546875	29.5806908161336\\
0.45654296875	29.5111709237998\\
0.45703125	29.4418605381936\\
0.45751953125	29.3727585847286\\
0.4580078125	29.3038639974742\\
0.45849609375	29.2351757190647\\
0.458984375	29.1666927006108\\
0.45947265625	29.098413901611\\
0.4599609375	29.0303382898645\\
0.46044921875	28.9624648413853\\
0.4609375	28.8947925403175\\
0.46142578125	28.8273203788511\\
0.4619140625	28.7600473571393\\
0.46240234375	28.6929724832168\\
0.462890625	28.6260947729188\\
0.46337890625	28.5594132498011\\
0.4638671875	28.4929269450616\\
0.46435546875	28.4266348974621\\
0.46484375	28.3605361532516\\
0.46533203125	28.2946297660898\\
0.4658203125	28.228914796973\\
0.46630859375	28.1633903141588\\
0.466796875	28.0980553930938\\
0.46728515625	28.0329091163408\\
0.4677734375	27.967950573507\\
0.46826171875	27.903178861174\\
0.46875	27.8385930828278\\
0.46923828125	27.7741923487897\\
0.4697265625	27.7099757761484\\
0.47021484375	27.6459424886926\\
0.470703125	27.5820916168444\\
0.47119140625	27.5184222975941\\
0.4716796875	27.4549336744347\\
0.47216796875	27.3916248972983\\
0.47265625	27.3284951224924\\
0.47314453125	27.2655435126373\\
0.4736328125	27.2027692366047\\
0.47412109375	27.1401714694559\\
0.474609375	27.0777493923819\\
0.47509765625	27.0155021926436\\
0.4755859375	26.9534290635126\\
0.47607421875	26.8915292042134\\
0.4765625	26.8298018198653\\
0.47705078125	26.7682461214257\\
0.4775390625	26.706861325634\\
0.47802734375	26.6456466549557\\
0.478515625	26.5846013375275\\
0.47900390625	26.5237246071034\\
0.4794921875	26.4630157030006\\
0.47998046875	26.4024738700466\\
0.48046875	26.3420983585267\\
0.48095703125	26.2818884241324\\
0.4814453125	26.2218433279099\\
0.48193359375	26.1619623362097\\
0.482421875	26.1022447206364\\
0.48291015625	26.0426897579993\\
0.4833984375	25.9832967302637\\
0.48388671875	25.9240649245021\\
0.484375	25.8649936328467\\
0.48486328125	25.8060821524423\\
0.4853515625	25.7473297853993\\
0.48583984375	25.6887358387479\\
0.486328125	25.6302996243919\\
0.48681640625	25.5720204590643\\
0.4873046875	25.513897664282\\
0.48779296875	25.4559305663023\\
0.48828125	25.3981184960787\\
0.48876953125	25.3404607892182\\
0.4892578125	25.2829567859388\\
0.48974609375	25.2256058310267\\
0.490234375	25.1684072737953\\
0.49072265625	25.1113604680436\\
0.4912109375	25.0544647720156\\
0.49169921875	24.99771954836\\
0.4921875	24.9411241640904\\
0.49267578125	24.8846779905456\\
0.4931640625	24.8283804033511\\
0.49365234375	24.77223078238\\
0.494140625	24.7162285117157\\
0.49462890625	24.6603729796131\\
0.4951171875	24.6046635784626\\
0.49560546875	24.5490997047521\\
0.49609375	24.4936807590312\\
0.49658203125	24.438406145875\\
0.4970703125	24.3832752738484\\
0.49755859375	24.3282875554707\\
0.498046875	24.273442407181\\
0.49853515625	24.2187392493035\\
0.4990234375	24.1641775060133\\
0.49951171875	24.1097566053029\\
0.5	24.0554759789482\\
0.50048828125	24.0013350624765\\
0.5009765625	23.9473332951328\\
0.50146484375	23.8934701198479\\
0.501953125	23.8397449832063\\
0.50244140625	23.7861573354151\\
0.5029296875	23.7327066302716\\
0.50341796875	23.6793923251335\\
0.50390625	23.6262138808873\\
0.50439453125	23.5731707619188\\
0.5048828125	23.5202624360823\\
0.50537109375	23.4674883746718\\
0.505859375	23.4148480523909\\
0.50634765625	23.3623409473242\\
0.5068359375	23.3099665409081\\
0.50732421875	23.2577243179031\\
0.5078125	23.2056137663649\\
0.50830078125	23.1536343776171\\
0.5087890625	23.1017856462231\\
0.50927734375	23.0500670699595\\
0.509765625	22.9984781497886\\
0.51025390625	22.9470183898319\\
0.5107421875	22.8956872973438\\
0.51123046875	22.8444843826848\\
0.51171875	22.7934091592968\\
0.51220703125	22.7424611436763\\
0.5126953125	22.6916398553499\\
0.51318359375	22.6409448168486\\
0.513671875	22.5903755536835\\
0.51416015625	22.5399315943207\\
0.5146484375	22.4896124701577\\
0.51513671875	22.4394177154983\\
0.515625	22.3893468675299\\
0.51611328125	22.339399466299\\
0.5166015625	22.2895750546884\\
0.51708984375	22.2398731783939\\
0.517578125	22.1902933859015\\
0.51806640625	22.1408352284647\\
0.5185546875	22.0914982600819\\
0.51904296875	22.0422820374747\\
0.51953125	21.9931861200654\\
0.52001953125	21.9442100699557\\
0.5205078125	21.8953534519047\\
0.52099609375	21.8466158333083\\
0.521484375	21.7979967841772\\
0.52197265625	21.7494958771167\\
0.5224609375	21.7011126873059\\
0.52294921875	21.6528467924769\\
0.5234375	21.604697772895\\
0.52392578125	21.5566652113381\\
0.5244140625	21.5087486930773\\
0.52490234375	21.4609478058569\\
0.525390625	21.4132621398753\\
0.52587890625	21.3656912877652\\
0.5263671875	21.3182348445748\\
0.52685546875	21.2708924077491\\
0.52734375	21.2236635771104\\
0.52783203125	21.1765479548408\\
0.5283203125	21.1295451454629\\
0.52880859375	21.0826547558224\\
0.529296875	21.0358763950692\\
0.52978515625	20.9892096746404\\
0.5302734375	20.9426542082423\\
0.53076171875	20.8962096118328\\
0.53125	20.8498755036041\\
0.53173828125	20.8036515039656\\
0.5322265625	20.7575372355272\\
0.53271484375	20.711532323082\\
0.533203125	20.6656363935895\\
0.53369140625	20.61984907616\\
0.5341796875	20.5741700020373\\
0.53466796875	20.5285988045827\\
0.53515625	20.4831351192595\\
0.53564453125	20.4377785836163\\
0.5361328125	20.392528837272\\
0.53662109375	20.3473855218997\\
0.537109375	20.3023482812114\\
0.53759765625	20.2574167609427\\
0.5380859375	20.2125906088378\\
0.53857421875	20.1678694746344\\
0.5390625	20.1232530100484\\
0.53955078125	20.0787408687598\\
0.5400390625	20.034332706398\\
0.54052734375	19.9900281805267\\
0.541015625	19.9458269506306\\
0.54150390625	19.9017286781002\\
0.5419921875	19.8577330262184\\
0.54248046875	19.8138396601463\\
0.54296875	19.7700482469094\\
0.54345703125	19.7263584553841\\
0.5439453125	19.6827699562839\\
0.54443359375	19.639282422146\\
0.544921875	19.5958955273183\\
0.54541015625	19.5526089479459\\
0.5458984375	19.509422361958\\
0.54638671875	19.4663354490554\\
0.546875	19.423347890697\\
0.54736328125	19.3804593700878\\
0.5478515625	19.3376695721656\\
0.54833984375	19.2949781835893\\
0.548828125	19.2523848927257\\
0.54931640625	19.209889389638\\
0.5498046875	19.1674913660732\\
0.55029296875	19.1251905154501\\
0.55078125	19.0829865328474\\
0.55126953125	19.0408791149922\\
0.5517578125	18.9988679602479\\
0.55224609375	18.9569527686025\\
0.552734375	18.9151332416575\\
0.55322265625	18.8734090826164\\
0.5537109375	18.831779996273\\
0.55419921875	18.7902456890007\\
0.5546875	18.7488058687412\\
0.55517578125	18.7074602449933\\
0.5556640625	18.6662085288024\\
0.55615234375	18.6250504327492\\
0.556640625	18.5839856709395\\
0.55712890625	18.5430139589932\\
0.5576171875	18.5021350140339\\
0.55810546875	18.4613485546783\\
0.55859375	18.4206543010264\\
0.55908203125	18.3800519746506\\
0.5595703125	18.3395412985855\\
0.56005859375	18.2991219973187\\
0.560546875	18.2587937967798\\
0.56103515625	18.2185564243311\\
0.5615234375	18.1784096087574\\
0.56201171875	18.1383530802565\\
0.5625	18.0983865704298\\
0.56298828125	18.0585098122721\\
0.5634765625	18.0187225401626\\
0.56396484375	17.9790244898552\\
0.564453125	17.9394153984696\\
0.56494140625	17.8998950044815\\
0.5654296875	17.860463047714\\
0.56591796875	17.821119269328\\
0.56640625	17.7818634118134\\
0.56689453125	17.7426952189805\\
0.5673828125	17.7036144359505\\
0.56787109375	17.6646208091473\\
0.568359375	17.6257140862882\\
0.56884765625	17.5868940163761\\
0.5693359375	17.54816034969\\
0.56982421875	17.5095128377773\\
0.5703125	17.4709512334448\\
0.57080078125	17.4324752907506\\
0.5712890625	17.394084764996\\
0.57177734375	17.3557794127171\\
0.572265625	17.3175589916763\\
0.57275390625	17.2794232608551\\
0.5732421875	17.2413719804452\\
0.57373046875	17.2034049118412\\
0.57421875	17.1655218176323\\
0.57470703125	17.1277224615948\\
0.5751953125	17.0900066086841\\
0.57568359375	17.0523740250272\\
0.576171875	17.0148244779151\\
0.57666015625	16.9773577357951\\
0.5771484375	16.9399735682632\\
0.57763671875	16.9026717460574\\
0.578125	16.8654520410491\\
0.57861328125	16.8283142262369\\
0.5791015625	16.7912580757388\\
0.57958984375	16.7542833647852\\
0.580078125	16.7173898697115\\
0.58056640625	16.6805773679514\\
0.5810546875	16.6438456380298\\
0.58154296875	16.6071944595555\\
0.58203125	16.5706236132145\\
0.58251953125	16.5341328807636\\
0.5830078125	16.4977220450228\\
0.58349609375	16.461390889869\\
0.583984375	16.4251392002295\\
0.58447265625	16.388966762075\\
0.5849609375	16.3528733624131\\
0.58544921875	16.3168587892819\\
0.5859375	16.2809228317434\\
0.58642578125	16.2450652798774\\
0.5869140625	16.2092859247744\\
0.58740234375	16.1735845585303\\
0.587890625	16.1379609742391\\
0.58837890625	16.1024149659874\\
0.5888671875	16.066946328848\\
0.58935546875	16.0315548588736\\
0.58984375	15.9962403530909\\
0.59033203125	15.9610026094949\\
0.5908203125	15.9258414270421\\
0.59130859375	15.8907566056453\\
0.591796875	15.8557479461675\\
0.59228515625	15.8208152504159\\
0.5927734375	15.7859583211362\\
0.59326171875	15.751176962007\\
0.59375	15.7164709776338\\
0.59423828125	15.6818401735437\\
0.5947265625	15.6472843561795\\
0.59521484375	15.6128033328942\\
0.595703125	15.5783969119454\\
0.59619140625	15.54406490249\\
0.5966796875	15.5098071145788\\
0.59716796875	15.4756233591506\\
0.59765625	15.4415134480274\\
0.59814453125	15.407477193909\\
0.5986328125	15.3735144103672\\
0.59912109375	15.3396249118412\\
0.599609375	15.3058085136323\\
0.60009765625	15.2720650318981\\
0.6005859375	15.2383942836484\\
0.60107421875	15.2047960867392\\
0.6015625	15.1712702598681\\
0.60205078125	15.1378166225695\\
0.6025390625	15.1044349952089\\
0.60302734375	15.071125198979\\
0.603515625	15.0378870558937\\
0.60400390625	15.004720388784\\
0.6044921875	14.9716250212932\\
0.60498046875	14.9386007778715\\
0.60546875	14.9056474837719\\
0.60595703125	14.872764965045\\
0.6064453125	14.8399530485348\\
0.60693359375	14.8072115618736\\
0.607421875	14.7745403334777\\
0.60791015625	14.7419391925428\\
0.6083984375	14.7094079690391\\
0.60888671875	14.6769464937074\\
0.609375	14.6445545980542\\
0.60986328125	14.6122321143472\\
0.6103515625	14.5799788756114\\
0.61083984375	14.5477947156241\\
0.611328125	14.5156794689111\\
0.61181640625	14.4836329707419\\
0.6123046875	14.4516550571257\\
0.61279296875	14.4197455648073\\
0.61328125	14.3879043312626\\
0.61376953125	14.3561311946944\\
0.6142578125	14.3244259940287\\
0.61474609375	14.2927885689097\\
0.615234375	14.2612187596968\\
0.61572265625	14.2297164074598\\
0.6162109375	14.1982813539748\\
0.61669921875	14.1669134417208\\
0.6171875	14.1356125138752\\
0.61767578125	14.1043784143101\\
0.6181640625	14.0732109875881\\
0.61865234375	14.0421100789588\\
0.619140625	14.0110755343549\\
0.61962890625	13.9801072003879\\
0.6201171875	13.9492049243447\\
0.62060546875	13.918368554184\\
0.62109375	13.8875979385318\\
0.62158203125	13.8568929266786\\
0.6220703125	13.8262533685749\\
0.62255859375	13.7956791148282\\
0.623046875	13.7651700166989\\
0.62353515625	13.7347259260966\\
0.6240234375	13.7043466955771\\
0.62451171875	13.6740321783382\\
0.625	13.6437822282165\\
0.62548828125	13.6135966996836\\
0.6259765625	13.5834754478431\\
0.62646484375	13.5534183284266\\
0.626953125	13.5234251977905\\
0.62744140625	13.4934959129127\\
0.6279296875	13.4636303313888\\
0.62841796875	13.4338283114292\\
0.62890625	13.4040897118554\\
0.62939453125	13.3744143920969\\
0.6298828125	13.3448022121878\\
0.63037109375	13.3152530327634\\
0.630859375	13.2857667150572\\
0.63134765625	13.2563431208975\\
0.6318359375	13.2269821127041\\
0.63232421875	13.1976835534854\\
0.6328125	13.168447306835\\
0.63330078125	13.1392732369286\\
0.6337890625	13.1101612085209\\
0.63427734375	13.0811110869424\\
0.634765625	13.0521227380965\\
0.63525390625	13.0231960284564\\
0.6357421875	12.9943308250619\\
0.63623046875	12.9655269955165\\
0.63671875	12.9367844079844\\
0.63720703125	12.9081029311875\\
0.6376953125	12.8794824344025\\
0.63818359375	12.8509227874579\\
0.638671875	12.8224238607311\\
0.63916015625	12.7939855251455\\
0.6396484375	12.7656076521677\\
0.64013671875	12.7372901138043\\
0.640625	12.7090327825997\\
0.64111328125	12.6808355316327\\
0.6416015625	12.652698234514\\
0.64208984375	12.6246207653831\\
0.642578125	12.5966029989062\\
0.64306640625	12.5686448102726\\
0.6435546875	12.5407460751927\\
0.64404296875	12.5129066698947\\
0.64453125	12.4851264711225\\
0.64501953125	12.4574053561325\\
0.6455078125	12.4297432026914\\
0.64599609375	12.4021398890732\\
0.646484375	12.3745952940566\\
0.64697265625	12.3471092969229\\
0.6474609375	12.3196817774526\\
0.64794921875	12.2923126159236\\
0.6484375	12.2650016931081\\
0.64892578125	12.2377488902706\\
0.6494140625	12.2105540891646\\
0.64990234375	12.1834171720311\\
0.650390625	12.1563380215953\\
0.65087890625	12.1293165210645\\
0.6513671875	12.1023525541254\\
0.65185546875	12.0754460049422\\
0.65234375	12.0485967581537\\
0.65283203125	12.0218046988707\\
0.6533203125	11.9950697126745\\
0.65380859375	11.9683916856135\\
0.654296875	11.9417705042017\\
0.65478515625	11.9152060554155\\
0.6552734375	11.8886982266923\\
0.65576171875	11.8622469059276\\
0.65625	11.8358519814728\\
0.65673828125	11.809513342133\\
0.6572265625	11.7832308771646\\
0.65771484375	11.7570044762733\\
0.658203125	11.7308340296118\\
0.65869140625	11.7047194277772\\
0.6591796875	11.6786605618093\\
0.65966796875	11.6526573231882\\
0.66015625	11.6267096038321\\
0.66064453125	11.6008172960951\\
0.6611328125	11.5749802927651\\
0.66162109375	11.5491984870617\\
0.662109375	11.5234717726341\\
0.66259765625	11.4978000435589\\
0.6630859375	11.4721831943381\\
0.66357421875	11.4466211198967\\
0.6640625	11.4211137155813\\
0.66455078125	11.3956608771572\\
0.6650390625	11.3702625008071\\
0.66552734375	11.3449184831286\\
0.666015625	11.3196287211325\\
0.66650390625	11.2943931122405\\
0.6669921875	11.2692115542833\\
0.66748046875	11.2440839454988\\
0.66796875	11.2190101845299\\
0.66845703125	11.1939901704228\\
0.6689453125	11.1690238026247\\
0.66943359375	11.1441109809821\\
0.669921875	11.119251605739\\
0.67041015625	11.0944455775347\\
0.6708984375	11.0696927974022\\
0.67138671875	11.0449931667661\\
0.671875	11.0203465874406\\
0.67236328125	10.9957529616282\\
0.6728515625	10.9712121919173\\
0.67333984375	10.9467241812806\\
0.673828125	10.9222888330732\\
0.67431640625	10.8979060510308\\
0.6748046875	10.8735757392681\\
0.67529296875	10.8492978022766\\
0.67578125	10.8250721449231\\
0.67626953125	10.8008986724481\\
0.6767578125	10.7767772904636\\
0.67724609375	10.7527079049515\\
0.677734375	10.7286904222623\\
0.67822265625	10.7047247491128\\
0.6787109375	10.6808107925844\\
0.67919921875	10.656948460122\\
0.6796875	10.6331376595317\\
0.68017578125	10.6093782989793\\
0.6806640625	10.5856702869886\\
0.68115234375	10.5620135324401\\
0.681640625	10.5384079445686\\
0.68212890625	10.5148534329623\\
0.6826171875	10.4913499075608\\
0.68310546875	10.4678972786535\\
0.68359375	10.444495456878\\
0.68408203125	10.4211443532185\\
0.6845703125	10.3978438790043\\
0.68505859375	10.3745939459082\\
0.685546875	10.3513944659447\\
0.68603515625	10.3282453514688\\
0.6865234375	10.3051465151741\\
0.68701171875	10.2820978700914\\
0.6875	10.2590993295873\\
0.68798828125	10.2361508073625\\
0.6884765625	10.2132522174502\\
0.68896484375	10.1904034742148\\
0.689453125	10.1676044923504\\
0.68994140625	10.1448551868792\\
0.6904296875	10.1221554731498\\
0.69091796875	10.0995052668363\\
0.69140625	10.0769044839362\\
0.69189453125	10.0543530407697\\
0.6923828125	10.0318508539772\\
0.69287109375	10.009397840519\\
0.693359375	9.98699391767301\\
0.69384765625	9.9646390030338\\
0.6943359375	9.94233301451091\\
0.69482421875	9.92007587032773\\
0.6953125	9.89786748901987\\
0.69580078125	9.87570778943381\\
0.6962890625	9.85359669072567\\
0.69677734375	9.83153411235963\\
0.697265625	9.80951997410672\\
0.69775390625	9.78755419604338\\
0.6982421875	9.76563669855019\\
0.69873046875	9.74376740231042\\
0.69921875	9.72194622830879\\
0.69970703125	9.70017309783011\\
0.7001953125	9.67844793245802\\
0.70068359375	9.65677065407359\\
0.701171875	9.63514118485401\\
0.70166015625	9.61355944727145\\
0.7021484375	9.59202536409161\\
0.70263671875	9.57053885837249\\
0.703125	9.5490998534632\\
0.70361328125	9.52770827300254\\
0.7041015625	9.50636404091797\\
0.70458984375	9.48506708142405\\
0.705078125	9.46381731902148\\
0.70556640625	9.44261467849579\\
0.7060546875	9.42145908491595\\
0.70654296875	9.4003504636334\\
0.70703125	9.37928874028067\\
0.70751953125	9.35827384077022\\
0.7080078125	9.33730569129324\\
0.70849609375	9.31638421831851\\
0.708984375	9.29550934859104\\
0.70947265625	9.27468100913115\\
0.7099609375	9.25389912723312\\
0.71044921875	9.23316363046401\\
0.7109375	9.21247444666261\\
0.71142578125	9.19183150393825\\
0.7119140625	9.17123473066962\\
0.71240234375	9.15068405550363\\
0.712890625	9.13017940735431\\
0.71337890625	9.10972071540173\\
0.7138671875	9.08930790909075\\
0.71435546875	9.06894091813003\\
0.71484375	9.04861967249084\\
0.71533203125	9.02834410240599\\
0.7158203125	9.00811413836879\\
0.71630859375	8.98792971113187\\
0.716796875	8.96779075170615\\
0.71728515625	8.94769719135971\\
0.7177734375	8.92764896161685\\
0.71826171875	8.90764599425685\\
0.71875	8.88768822131307\\
0.71923828125	8.86777557507181\\
0.7197265625	8.84790798807125\\
0.72021484375	8.82808539310049\\
0.720703125	8.80830772319842\\
0.72119140625	8.78857491165281\\
0.7216796875	8.76888689199913\\
0.72216796875	8.74924359801965\\
0.72265625	8.7296449637425\\
0.72314453125	8.71009092344037\\
0.7236328125	8.69058141162992\\
0.72412109375	8.6711163630704\\
0.724609375	8.65169571276287\\
0.72509765625	8.63231939594923\\
0.7255859375	8.61298734811118\\
0.72607421875	8.5936995049692\\
0.7265625	8.57445580248169\\
0.72705078125	8.55525617684393\\
0.7275390625	8.53610056448715\\
0.72802734375	8.51698890207762\\
0.728515625	8.4979211265156\\
0.72900390625	8.47889717493453\\
0.7294921875	8.45991698469995\\
0.72998046875	8.44098049340873\\
0.73046875	8.42208763888803\\
0.73095703125	8.40323835919438\\
0.7314453125	8.38443259261284\\
0.73193359375	8.365670277656\\
0.732421875	8.34695135306319\\
0.73291015625	8.32827575779946\\
0.7333984375	8.3096434310547\\
0.73388671875	8.29105431224292\\
0.734375	8.27250834100112\\
0.73486328125	8.2540054571885\\
0.7353515625	8.23554560088572\\
0.73583984375	8.21712871239379\\
0.736328125	8.19875473223341\\
0.73681640625	8.18042360114399\\
0.7373046875	8.16213526008283\\
0.73779296875	8.14388965022432\\
0.73828125	8.12568671295896\\
0.73876953125	8.10752638989262\\
0.7392578125	8.08940862284573\\
0.73974609375	8.07133335385234\\
0.740234375	8.05330052515936\\
0.74072265625	8.0353100792257\\
0.7412109375	8.01736195872155\\
0.74169921875	7.99945610652733\\
0.7421875	7.98159246573326\\
0.74267578125	7.96377097963811\\
0.7431640625	7.94599159174869\\
0.74365234375	7.92825424577902\\
0.744140625	7.91055888564948\\
0.74462890625	7.89290545548595\\
0.7451171875	7.87529389961921\\
0.74560546875	7.85772416258397\\
0.74609375	7.84019618911826\\
0.74658203125	7.82270992416246\\
0.7470703125	7.80526531285877\\
0.74755859375	7.78786230055025\\
0.748046875	7.77050083278012\\
0.74853515625	7.75318085529105\\
0.7490234375	7.73590231402432\\
0.74951171875	7.71866515511923\\
0.75	7.7014693249121\\
0.75048828125	7.68431476993579\\
0.7509765625	7.66720143691882\\
0.75146484375	7.65012927278463\\
0.751953125	7.63309822465093\\
0.75244140625	7.6161082398289\\
0.7529296875	7.59915926582254\\
0.75341796875	7.58225125032788\\
0.75390625	7.56538414123237\\
0.75439453125	7.54855788661398\\
0.7548828125	7.53177243474075\\
0.75537109375	7.51502773406983\\
0.755859375	7.498323733247\\
0.75634765625	7.48166038110586\\
0.7568359375	7.46503762666711\\
0.75732421875	7.44845541913797\\
0.7578125	7.4319137079114\\
0.75830078125	7.41541244256549\\
0.7587890625	7.39895157286273\\
0.75927734375	7.38253104874937\\
0.759765625	7.3661508203547\\
0.76025390625	7.34981083799048\\
0.7607421875	7.33351105215019\\
0.76123046875	7.31725141350841\\
0.76171875	7.30103187292016\\
0.76220703125	7.28485238142019\\
0.7626953125	7.26871289022248\\
0.76318359375	7.25261335071938\\
0.763671875	7.23655371448121\\
0.76416015625	7.22053393325543\\
0.7646484375	7.20455395896607\\
0.76513671875	7.1886137437131\\
0.765625	7.17271323977186\\
0.76611328125	7.15685239959231\\
0.7666015625	7.14103117579849\\
0.76708984375	7.1252495211879\\
0.767578125	7.10950738873091\\
0.76806640625	7.09380473156999\\
0.7685546875	7.07814150301935\\
0.76904296875	7.06251765656409\\
0.76953125	7.04693314585982\\
0.77001953125	7.03138792473186\\
0.7705078125	7.01588194717476\\
0.77099609375	7.00041516735173\\
0.771484375	6.98498753959389\\
0.77197265625	6.96959901839988\\
0.7724609375	6.9542495584352\\
0.77294921875	6.93893911453153\\
0.7734375	6.9236676416863\\
0.77392578125	6.90843509506202\\
0.7744140625	6.8932414299858\\
0.77490234375	6.87808660194863\\
0.775390625	6.86297056660495\\
0.77587890625	6.84789327977207\\
0.7763671875	6.83285469742949\\
0.77685546875	6.81785477571853\\
0.77734375	6.80289347094164\\
0.77783203125	6.78797073956183\\
0.7783203125	6.77308653820225\\
0.77880859375	6.75824082364553\\
0.779296875	6.74343355283327\\
0.77978515625	6.72866468286555\\
0.7802734375	6.71393417100027\\
0.78076171875	6.69924197465276\\
0.78125	6.68458805139513\\
0.78173828125	6.6699723589558\\
0.7822265625	6.65539485521898\\
0.78271484375	6.64085549822412\\
0.783203125	6.62635424616535\\
0.78369140625	6.61189105739109\\
0.7841796875	6.59746589040339\\
0.78466796875	6.58307870385745\\
0.78515625	6.56872945656122\\
0.78564453125	6.55441810747472\\
0.7861328125	6.54014461570969\\
0.78662109375	6.52590894052895\\
0.787109375	6.51171104134598\\
0.78759765625	6.49755087772451\\
0.7880859375	6.48342840937773\\
0.78857421875	6.46934359616816\\
0.7890625	6.45529639810694\\
0.78955078125	6.44128677535333\\
0.7900390625	6.42731468821433\\
0.79052734375	6.41338009714415\\
0.791015625	6.39948296274378\\
0.79150390625	6.38562324576034\\
0.7919921875	6.37180090708681\\
0.79248046875	6.35801590776152\\
0.79296875	6.3442682089675\\
0.79345703125	6.33055777203225\\
0.7939453125	6.31688455842714\\
0.79443359375	6.30324852976692\\
0.794921875	6.28964964780942\\
0.79541015625	6.27608787445488\\
0.7958984375	6.26256317174566\\
0.79638671875	6.24907550186563\\
0.796875	6.23562482713995\\
0.79736328125	6.2222111100344\\
0.7978515625	6.20883431315498\\
0.79833984375	6.19549439924755\\
0.798828125	6.18219133119729\\
0.79931640625	6.16892507202835\\
0.7998046875	6.15569558490336\\
0.80029296875	6.14250283312291\\
0.80078125	6.12934678012535\\
0.80126953125	6.11622738948609\\
0.8017578125	6.1031446249173\\
0.80224609375	6.09009845026758\\
0.802734375	6.0770888295213\\
0.80322265625	6.06411572679836\\
0.8037109375	6.05117910635375\\
0.80419921875	6.038278932577\\
0.8046875	6.02541516999192\\
0.80517578125	6.01258778325616\\
0.8056640625	5.99979673716068\\
0.80615234375	5.98704199662947\\
0.806640625	5.97432352671904\\
0.80712890625	5.96164129261813\\
0.8076171875	5.94899525964719\\
0.80810546875	5.93638539325808\\
0.80859375	5.92381165903354\\
0.80908203125	5.91127402268694\\
0.8095703125	5.89877245006175\\
0.81005859375	5.88630690713127\\
0.810546875	5.87387735999812\\
0.81103515625	5.86148377489394\\
0.8115234375	5.84912611817898\\
0.81201171875	5.83680435634161\\
0.8125	5.82451845599819\\
0.81298828125	5.81226838389235\\
0.8134765625	5.80005410689487\\
0.81396484375	5.78787559200321\\
0.814453125	5.77573280634112\\
0.81494140625	5.7636257171583\\
0.8154296875	5.75155429183001\\
0.81591796875	5.73951849785666\\
0.81640625	5.72751830286352\\
0.81689453125	5.71555367460029\\
0.8173828125	5.70362458094074\\
0.81787109375	5.69173098988244\\
0.818359375	5.6798728695462\\
0.81884765625	5.66805018817593\\
0.8193359375	5.65626291413814\\
0.81982421875	5.64451101592165\\
0.8203125	5.63279446213714\\
0.82080078125	5.62111322151693\\
0.8212890625	5.60946726291462\\
0.82177734375	5.59785655530457\\
0.822265625	5.58628106778169\\
0.82275390625	5.57474076956117\\
0.8232421875	5.56323562997798\\
0.82373046875	5.55176561848653\\
0.82421875	5.54033070466048\\
0.82470703125	5.5289308581923\\
0.8251953125	5.51756604889286\\
0.82568359375	5.50623624669127\\
0.826171875	5.4949414216344\\
0.82666015625	5.48368154388658\\
0.8271484375	5.4724565837294\\
0.82763671875	5.46126651156111\\
0.828125	5.45011129789663\\
0.82861328125	5.43899091336689\\
0.8291015625	5.42790532871876\\
0.82958984375	5.41685451481459\\
0.830078125	5.40583844263199\\
0.83056640625	5.39485708326337\\
0.8310546875	5.38391040791578\\
0.83154296875	5.3729983879105\\
0.83203125	5.36212099468272\\
0.83251953125	5.35127819978128\\
0.8330078125	5.34046997486834\\
0.83349609375	5.32969629171907\\
0.833984375	5.31895712222129\\
0.83447265625	5.30825243837525\\
0.8349609375	5.29758221229329\\
0.83544921875	5.28694641619953\\
0.8359375	5.27634502242957\\
0.83642578125	5.26577800343018\\
0.8369140625	5.25524533175903\\
0.83740234375	5.24474698008436\\
0.837890625	5.23428292118473\\
0.83837890625	5.22385312794864\\
0.8388671875	5.21345757337439\\
0.83935546875	5.2030962305696\\
0.83984375	5.19276907275108\\
0.84033203125	5.18247607324446\\
0.8408203125	5.17221720548387\\
0.84130859375	5.16199244301179\\
0.841796875	5.15180175947863\\
0.84228515625	5.14164512864255\\
0.8427734375	5.13152252436904\\
0.84326171875	5.12143392063079\\
0.84375	5.11137929150735\\
0.84423828125	5.10135861118489\\
0.8447265625	5.09137185395583\\
0.84521484375	5.08141899421866\\
0.845703125	5.07150000647759\\
0.84619140625	5.06161486534242\\
0.8466796875	5.05176354552812\\
0.84716796875	5.04194602185458\\
0.84765625	5.03216226924645\\
0.84814453125	5.02241226273274\\
0.8486328125	5.01269597744671\\
0.84912109375	5.00301338862545\\
0.849609375	4.99336447160971\\
0.85009765625	4.98374920184363\\
0.8505859375	4.97416755487444\\
0.85107421875	4.96461950635225\\
0.8515625	4.95510503202981\\
0.85205078125	4.9456241077622\\
0.8525390625	4.93617670950657\\
0.85302734375	4.92676281332195\\
0.853515625	4.91738239536901\\
0.85400390625	4.90803543190966\\
0.8544921875	4.89872189930703\\
0.85498046875	4.88944177402504\\
0.85546875	4.88019503262821\\
0.85595703125	4.87098165178145\\
0.8564453125	4.86180160824983\\
0.85693359375	4.85265487889824\\
0.857421875	4.84354144069122\\
0.85791015625	4.83446127069275\\
0.8583984375	4.82541434606593\\
0.85888671875	4.81640064407284\\
0.859375	4.80742014207419\\
0.85986328125	4.79847281752923\\
0.8603515625	4.78955864799536\\
0.86083984375	4.78067761112801\\
0.861328125	4.77182968468036\\
0.86181640625	4.76301484650317\\
0.8623046875	4.75423307454444\\
0.86279296875	4.74548434684928\\
0.86328125	4.73676864155972\\
0.86376953125	4.72808593691431\\
0.8642578125	4.71943621124809\\
0.86474609375	4.71081944299223\\
0.865234375	4.70223561067394\\
0.86572265625	4.69368469291608\\
0.8662109375	4.68516666843714\\
0.86669921875	4.67668151605086\\
0.8671875	4.66822921466607\\
0.86767578125	4.65980974328651\\
0.8681640625	4.65142308101056\\
0.86865234375	4.64306920703109\\
0.869140625	4.6347481006352\\
0.86962890625	4.62645974120397\\
0.8701171875	4.6182041082124\\
0.87060546875	4.60998118122899\\
0.87109375	4.60179093991578\\
0.87158203125	4.5936333640279\\
0.8720703125	4.58550843341353\\
0.87255859375	4.57741612801364\\
0.873046875	4.56935642786178\\
0.87353515625	4.56132931308391\\
0.8740234375	4.55333476389816\\
0.87451171875	4.54537276061465\\
0.875	4.53744328363533\\
0.87548828125	4.52954631345372\\
0.8759765625	4.52168183065472\\
0.87646484375	4.51384981591447\\
0.876953125	4.50605025000008\\
0.87744140625	4.49828311376956\\
0.8779296875	4.49054838817146\\
0.87841796875	4.4828460542448\\
0.87890625	4.47517609311885\\
0.87939453125	4.46753848601297\\
0.8798828125	4.45993321423632\\
0.88037109375	4.45236025918783\\
0.880859375	4.44481960235583\\
0.88134765625	4.43731122531808\\
0.8818359375	4.42983510974138\\
0.88232421875	4.42239123738155\\
0.8828125	4.41497959008308\\
0.88330078125	4.40760014977919\\
0.8837890625	4.40025289849139\\
0.88427734375	4.39293781832945\\
0.884765625	4.38565489149127\\
0.88525390625	4.37840410026254\\
0.8857421875	4.37118542701664\\
0.88623046875	4.36399885421461\\
0.88671875	4.35684436440469\\
0.88720703125	4.34972194022239\\
0.8876953125	4.34263156439023\\
0.88818359375	4.33557321971752\\
0.888671875	4.32854688910034\\
0.88916015625	4.32155255552117\\
0.8896484375	4.3145902020489\\
0.89013671875	4.30765981183857\\
0.890625	4.3007613681312\\
0.89111328125	4.29389485425366\\
0.8916015625	4.2870602536186\\
0.89208984375	4.28025754972403\\
0.892578125	4.27348672615339\\
0.89306640625	4.26674776657534\\
0.8935546875	4.26004065474351\\
0.89404296875	4.25336537449646\\
0.89453125	4.24672190975747\\
0.89501953125	4.24011024453433\\
0.8955078125	4.23353036291929\\
0.89599609375	4.22698224908881\\
0.896484375	4.2204658873035\\
0.89697265625	4.21398126190783\\
0.8974609375	4.20752835733014\\
0.89794921875	4.20110715808238\\
0.8984375	4.19471764875998\\
0.89892578125	4.18835981404175\\
0.8994140625	4.18203363868965\\
0.89990234375	4.17573910754868\\
0.900390625	4.16947620554681\\
0.90087890625	4.16324491769461\\
0.9013671875	4.15704522908549\\
0.90185546875	4.1508771248951\\
0.90234375	4.14474059038152\\
0.90283203125	4.13863561088501\\
0.9033203125	4.13256217182782\\
0.90380859375	4.12652025871412\\
0.904296875	4.12050985712987\\
0.90478515625	4.11453095274258\\
0.9052734375	4.10858353130131\\
0.90576171875	4.10266757863641\\
0.90625	4.09678308065944\\
0.90673828125	4.09093002336308\\
0.9072265625	4.0851083928209\\
0.90771484375	4.0793181751873\\
0.908203125	4.07355935669733\\
0.90869140625	4.06783192366657\\
0.9091796875	4.06213586249106\\
0.90966796875	4.05647115964707\\
0.91015625	4.05083780169103\\
0.91064453125	4.0452357752594\\
0.9111328125	4.03966506706852\\
0.91162109375	4.03412566391454\\
0.912109375	4.02861755267317\\
0.91259765625	4.0231407202997\\
0.9130859375	4.0176951538288\\
0.91357421875	4.01228084037436\\
0.9140625	4.00689776712947\\
0.91455078125	4.00154592136624\\
0.9150390625	3.99622529043566\\
0.91552734375	3.9909358617675\\
0.916015625	3.9856776228702\\
0.91650390625	3.98045056133075\\
0.9169921875	3.97525466481454\\
0.91748046875	3.97008992106534\\
0.91796875	3.96495631790502\\
0.91845703125	3.95985384323359\\
0.9189453125	3.95478248502899\\
0.91943359375	3.94974223134704\\
0.919921875	3.94473307032128\\
0.92041015625	3.93975499016285\\
0.9208984375	3.93480797916046\\
0.92138671875	3.92989202568018\\
0.921875	3.92500711816538\\
0.92236328125	3.92015324513664\\
0.9228515625	3.91533039519156\\
0.92333984375	3.91053855700477\\
0.923828125	3.90577771932773\\
0.92431640625	3.90104787098864\\
0.9248046875	3.8963490008924\\
0.92529296875	3.8916810980204\\
0.92578125	3.88704415143053\\
0.92626953125	3.88243815025699\\
0.9267578125	3.8778630837102\\
0.92724609375	3.87331894107675\\
0.927734375	3.86880571171925\\
0.92822265625	3.86432338507625\\
0.9287109375	3.85987195066213\\
0.92919921875	3.85545139806704\\
0.9296875	3.85106171695674\\
0.93017578125	3.84670289707255\\
0.9306640625	3.84237492823125\\
0.93115234375	3.83807780032492\\
0.931640625	3.83381150332099\\
0.93212890625	3.82957602726196\\
0.9326171875	3.8253713622655\\
0.93310546875	3.82119749852417\\
0.93359375	3.8170544263055\\
0.93408203125	3.81294213595175\\
0.9345703125	3.80886061787995\\
0.93505859375	3.80480986258168\\
0.935546875	3.80078986062312\\
0.93603515625	3.79680060264489\\
0.9365234375	3.79284207936188\\
0.93701171875	3.78891428156335\\
0.9375	3.78501720011271\\
0.93798828125	3.78115082594748\\
0.9384765625	3.77731515007911\\
0.93896484375	3.77351016359315\\
0.939453125	3.76973585764885\\
0.93994140625	3.76599222347927\\
0.9404296875	3.76227925239121\\
0.94091796875	3.75859693576498\\
0.94140625	3.75494526505453\\
0.94189453125	3.75132423178712\\
0.9423828125	3.74773382756352\\
0.94287109375	3.74417404405769\\
0.943359375	3.74064487301684\\
0.94384765625	3.73714630626131\\
0.9443359375	3.73367833568451\\
0.94482421875	3.73024095325281\\
0.9453125	3.72683415100557\\
0.94580078125	3.72345792105486\\
0.9462890625	3.72011225558563\\
0.94677734375	3.7167971468555\\
0.947265625	3.71351258719462\\
0.94775390625	3.71025856900585\\
0.9482421875	3.70703508476438\\
0.94873046875	3.70384212701791\\
0.94921875	3.70067968838647\\
0.94970703125	3.69754776156233\\
0.9501953125	3.69444633930998\\
0.95068359375	3.69137541446606\\
0.951171875	3.68833497993929\\
0.95166015625	3.68532502871042\\
0.9521484375	3.68234555383209\\
0.95263671875	3.67939654842888\\
0.953125	3.67647800569714\\
};
\addplot [color=mycolor3,solid]
  table[row sep=crcr]{0.953125	3.67647800569714\\
0.95361328125	3.67358991890504\\
0.9541015625	3.67073228139235\\
0.95458984375	3.66790508657057\\
0.955078125	3.6651083279227\\
0.95556640625	3.66234199900332\\
0.9560546875	3.65960609343844\\
0.95654296875	3.65690060492542\\
0.95703125	3.654225527233\\
0.95751953125	3.65158085420123\\
0.9580078125	3.6489665797413\\
0.95849609375	3.64638269783568\\
0.958984375	3.64382920253785\\
0.95947265625	3.64130608797242\\
0.9599609375	3.63881334833497\\
0.96044921875	3.63635097789203\\
0.9609375	3.63391897098104\\
0.96142578125	3.63151732201033\\
0.9619140625	3.62914602545895\\
0.96240234375	3.62680507587674\\
0.962890625	3.62449446788423\\
0.96337890625	3.62221419617263\\
0.9638671875	3.61996425550368\\
0.96435546875	3.61774464070975\\
0.96484375	3.61555534669364\\
0.96533203125	3.61339636842865\\
0.9658203125	3.61126770095852\\
0.96630859375	3.60916933939728\\
0.966796875	3.60710127892934\\
0.96728515625	3.60506351480938\\
0.9677734375	3.60305604236229\\
0.96826171875	3.6010788569832\\
0.96875	3.59913195413732\\
0.96923828125	3.59721532936006\\
0.9697265625	3.59532897825682\\
0.97021484375	3.59347289650307\\
0.970703125	3.59164707984425\\
0.97119140625	3.58985152409576\\
0.9716796875	3.58808622514291\\
0.97216796875	3.58635117894088\\
0.97265625	3.58464638151472\\
0.97314453125	3.58297182895923\\
0.9736328125	3.581327517439\\
0.97412109375	3.57971344318837\\
0.974609375	3.57812960251135\\
0.97509765625	3.57657599178164\\
0.9755859375	3.57505260744254\\
0.97607421875	3.57355944600699\\
0.9765625	3.57209650405746\\
0.97705078125	3.57066377824596\\
0.9775390625	3.56926126529401\\
0.97802734375	3.56788896199262\\
0.978515625	3.56654686520226\\
0.97900390625	3.56523497185276\\
0.9794921875	3.56395327894338\\
0.97998046875	3.56270178354274\\
0.98046875	3.56148048278879\\
0.98095703125	3.56028937388882\\
0.9814453125	3.55912845411937\\
0.98193359375	3.55799772082622\\
0.982421875	3.55689717142446\\
0.98291015625	3.55582680339833\\
0.9833984375	3.55478661430126\\
0.98388671875	3.55377660175591\\
0.984375	3.55279676345399\\
0.98486328125	3.55184709715644\\
0.9853515625	3.55092760069325\\
0.98583984375	3.55003827196347\\
0.986328125	3.54917910893527\\
0.98681640625	3.54835010964583\\
0.9873046875	3.5475512722014\\
0.98779296875	3.54678259477718\\
0.98828125	3.54604407561745\\
0.98876953125	3.54533571303539\\
0.9892578125	3.54465750541319\\
0.98974609375	3.54400945120201\\
0.990234375	3.54339154892185\\
0.99072265625	3.54280379716178\\
0.9912109375	3.54224619457967\\
0.99169921875	3.54171873990233\\
0.9921875	3.54122143192543\\
0.99267578125	3.54075426951355\\
0.9931640625	3.54031725160011\\
0.99365234375	3.53991037718739\\
0.994140625	3.53953364534655\\
0.99462890625	3.53918705521754\\
0.9951171875	3.53887060600918\\
0.99560546875	3.53858429699905\\
0.99609375	3.53832812753368\\
0.99658203125	3.53810209702825\\
0.9970703125	3.53790620496682\\
0.99755859375	3.53774045090232\\
0.998046875	3.53760483445634\\
0.99853515625	3.53749935531936\\
0.9990234375	3.53742401325063\\
0.99951171875	3.53737880807819\\
};
\addlegendentry{AR(8) Model};

\addplot [color=mycolor4,solid,forget plot]
  table[row sep=crcr]{-1	3.55733584138882\\
-0.99951171875	3.55735095273702\\
-0.9990234375	3.55739628681237\\
-0.99853515625	3.55747184370711\\
-0.998046875	3.557577623575\\
-0.99755859375	3.55771362663128\\
-0.9970703125	3.5578798531527\\
-0.99658203125	3.55807630347754\\
-0.99609375	3.55830297800558\\
-0.99560546875	3.5585598771981\\
-0.9951171875	3.55884700157797\\
-0.99462890625	3.55916435172949\\
-0.994140625	3.55951192829858\\
-0.99365234375	3.55988973199274\\
-0.9931640625	3.56029776358094\\
-0.99267578125	3.56073602389376\\
-0.9921875	3.56120451382341\\
-0.99169921875	3.56170323432367\\
-0.9912109375	3.56223218640985\\
-0.99072265625	3.56279137115905\\
-0.990234375	3.56338078970985\\
-0.98974609375	3.56400044326261\\
-0.9892578125	3.56465033307927\\
-0.98876953125	3.56533046048353\\
-0.98828125	3.56604082686078\\
-0.98779296875	3.56678143365812\\
-0.9873046875	3.56755228238446\\
-0.98681640625	3.56835337461039\\
-0.986328125	3.56918471196841\\
-0.98583984375	3.57004629615276\\
-0.9853515625	3.57093812891956\\
-0.98486328125	3.57186021208675\\
-0.984375	3.57281254753425\\
-0.98388671875	3.57379513720384\\
-0.9833984375	3.57480798309926\\
-0.98291015625	3.57585108728626\\
-0.982421875	3.57692445189255\\
-0.98193359375	3.57802807910791\\
-0.9814453125	3.5791619711842\\
-0.98095703125	3.58032613043535\\
-0.98046875	3.58152055923747\\
-0.97998046875	3.5827452600288\\
-0.9794921875	3.58400023530982\\
-0.97900390625	3.58528548764319\\
-0.978515625	3.58660101965391\\
-0.97802734375	3.5879468340293\\
-0.9775390625	3.58932293351899\\
-0.97705078125	3.59072932093502\\
-0.9765625	3.59216599915185\\
-0.97607421875	3.59363297110645\\
-0.9755859375	3.59513023979827\\
-0.97509765625	3.59665780828939\\
-0.974609375	3.59821567970436\\
-0.97412109375	3.59980385723048\\
-0.9736328125	3.60142234411776\\
-0.97314453125	3.60307114367886\\
-0.97265625	3.6047502592893\\
-0.97216796875	3.60645969438741\\
-0.9716796875	3.60819945247439\\
-0.97119140625	3.60996953711442\\
-0.970703125	3.61176995193462\\
-0.97021484375	3.61360070062517\\
-0.9697265625	3.61546178693931\\
-0.96923828125	3.6173532146935\\
-0.96875	3.6192749877673\\
-0.96826171875	3.6212271101036\\
-0.9677734375	3.62320958570852\\
-0.96728515625	3.62522241865164\\
-0.966796875	3.62726561306591\\
-0.96630859375	3.62933917314777\\
-0.9658203125	3.63144310315719\\
-0.96533203125	3.6335774074178\\
-0.96484375	3.63574209031685\\
-0.96435546875	3.63793715630527\\
-0.9638671875	3.6401626098979\\
-0.96337890625	3.64241845567334\\
-0.962890625	3.64470469827416\\
-0.96240234375	3.64702134240691\\
-0.9619140625	3.64936839284217\\
-0.96142578125	3.65174585441467\\
-0.9609375	3.65415373202329\\
-0.96044921875	3.65659203063123\\
-0.9599609375	3.65906075526598\\
-0.95947265625	3.66155991101944\\
-0.958984375	3.66408950304797\\
-0.95849609375	3.66664953657254\\
-0.9580078125	3.66924001687867\\
-0.95751953125	3.67186094931661\\
-0.95703125	3.67451233930135\\
-0.95654296875	3.67719419231282\\
-0.9560546875	3.67990651389576\\
-0.95556640625	3.68264930965999\\
-0.955078125	3.68542258528042\\
-0.95458984375	3.68822634649707\\
-0.9541015625	3.69106059911526\\
-0.95361328125	3.69392534900561\\
-0.953125	3.69682060210417\\
-0.95263671875	3.6997463644125\\
-0.9521484375	3.70270264199771\\
-0.95166015625	3.70568944099259\\
-0.951171875	3.70870676759568\\
-0.95068359375	3.71175462807141\\
-0.9501953125	3.71483302875006\\
-0.94970703125	3.71794197602795\\
-0.94921875	3.72108147636757\\
-0.94873046875	3.72425153629758\\
-0.9482421875	3.7274521624129\\
-0.94775390625	3.73068336137489\\
-0.947265625	3.73394513991131\\
-0.94677734375	3.73723750481665\\
-0.9462890625	3.74056046295186\\
-0.94580078125	3.74391402124485\\
-0.9453125	3.74729818669033\\
-0.94482421875	3.75071296634988\\
-0.9443359375	3.75415836735231\\
-0.94384765625	3.75763439689348\\
-0.943359375	3.76114106223655\\
-0.94287109375	3.76467837071205\\
-0.9423828125	3.76824632971799\\
-0.94189453125	3.77184494671992\\
-0.94140625	3.77547422925111\\
-0.94091796875	3.77913418491259\\
-0.9404296875	3.78282482137332\\
-0.93994140625	3.78654614637021\\
-0.939453125	3.79029816770831\\
-0.93896484375	3.79408089326086\\
-0.9384765625	3.79789433096947\\
-0.93798828125	3.80173848884423\\
-0.9375	3.80561337496361\\
-0.93701171875	3.80951899747494\\
-0.9365234375	3.81345536459422\\
-0.93603515625	3.81742248460641\\
-0.935546875	3.8214203658654\\
-0.93505859375	3.82544901679429\\
-0.9345703125	3.82950844588537\\
-0.93408203125	3.83359866170032\\
-0.93359375	3.83771967287033\\
-0.93310546875	3.84187148809611\\
-0.9326171875	3.84605411614819\\
-0.93212890625	3.8502675658669\\
-0.931640625	3.85451184616256\\
-0.93115234375	3.85878696601556\\
-0.9306640625	3.86309293447657\\
-0.93017578125	3.86742976066651\\
-0.9296875	3.87179745377686\\
-0.92919921875	3.87619602306972\\
-0.9287109375	3.88062547787782\\
-0.92822265625	3.88508582760481\\
-0.927734375	3.88957708172537\\
-0.92724609375	3.89409924978525\\
-0.9267578125	3.89865234140145\\
-0.92626953125	3.90323636626241\\
-0.92578125	3.9078513341281\\
-0.92529296875	3.91249725483007\\
-0.9248046875	3.91717413827174\\
-0.92431640625	3.92188199442846\\
-0.923828125	3.92662083334763\\
-0.92333984375	3.93139066514882\\
-0.9228515625	3.93619150002407\\
-0.92236328125	3.94102334823782\\
-0.921875	3.94588622012717\\
-0.92138671875	3.950780126102\\
-0.9208984375	3.95570507664513\\
-0.92041015625	3.96066108231241\\
-0.919921875	3.96564815373294\\
-0.91943359375	3.97066630160916\\
-0.9189453125	3.97571553671703\\
-0.91845703125	3.98079586990618\\
-0.91796875	3.98590731209999\\
-0.91748046875	3.99104987429586\\
-0.9169921875	3.99622356756529\\
-0.91650390625	4.00142840305403\\
-0.916015625	4.00666439198223\\
-0.91552734375	4.01193154564462\\
-0.9150390625	4.01722987541068\\
-0.91455078125	4.02255939272473\\
-0.9140625	4.0279201091062\\
-0.91357421875	4.03331203614963\\
-0.9130859375	4.03873518552497\\
-0.91259765625	4.04418956897766\\
-0.912109375	4.04967519832887\\
-0.91162109375	4.05519208547556\\
-0.9111328125	4.0607402423907\\
-0.91064453125	4.06631968112345\\
-0.91015625	4.07193041379931\\
-0.90966796875	4.07757245262021\\
-0.9091796875	4.08324580986486\\
-0.90869140625	4.08895049788869\\
-0.908203125	4.09468652912424\\
-0.90771484375	4.10045391608112\\
-0.9072265625	4.10625267134639\\
-0.90673828125	4.11208280758455\\
-0.90625	4.11794433753781\\
-0.90576171875	4.12383727402628\\
-0.9052734375	4.12976162994804\\
-0.90478515625	4.13571741827944\\
-0.904296875	4.14170465207521\\
-0.90380859375	4.14772334446864\\
-0.9033203125	4.15377350867178\\
-0.90283203125	4.15985515797563\\
-0.90234375	4.1659683057502\\
-0.90185546875	4.17211296544492\\
-0.9013671875	4.17828915058865\\
-0.90087890625	4.18449687478989\\
-0.900390625	4.19073615173699\\
-0.89990234375	4.19700699519829\\
-0.8994140625	4.20330941902245\\
-0.89892578125	4.20964343713844\\
-0.8984375	4.21600906355587\\
-0.89794921875	4.22240631236511\\
-0.8974609375	4.2288351977375\\
-0.89697265625	4.2352957339256\\
-0.896484375	4.24178793526327\\
-0.89599609375	4.24831181616591\\
-0.8955078125	4.25486739113077\\
-0.89501953125	4.26145467473691\\
-0.89453125	4.26807368164562\\
-0.89404296875	4.27472442660051\\
-0.8935546875	4.28140692442773\\
-0.89306640625	4.28812119003615\\
-0.892578125	4.29486723841758\\
-0.89208984375	4.30164508464703\\
-0.8916015625	4.30845474388278\\
-0.89111328125	4.31529623136668\\
-0.890625	4.32216956242439\\
-0.89013671875	4.32907475246545\\
-0.8896484375	4.3360118169836\\
-0.88916015625	4.34298077155699\\
-0.888671875	4.34998163184834\\
-0.88818359375	4.35701441360509\\
-0.8876953125	4.36407913265986\\
-0.88720703125	4.37117580493032\\
-0.88671875	4.37830444641966\\
-0.88623046875	4.38546507321671\\
-0.8857421875	4.39265770149614\\
-0.88525390625	4.39988234751875\\
-0.884765625	4.40713902763161\\
-0.88427734375	4.41442775826828\\
-0.8837890625	4.42174855594915\\
-0.88330078125	4.42910143728147\\
-0.8828125	4.43648641895971\\
-0.88232421875	4.44390351776581\\
-0.8818359375	4.45135275056922\\
-0.88134765625	4.45883413432735\\
-0.880859375	4.46634768608566\\
-0.88037109375	4.47389342297786\\
-0.8798828125	4.48147136222631\\
-0.87939453125	4.48908152114206\\
-0.87890625	4.49672391712518\\
-0.87841796875	4.50439856766499\\
-0.8779296875	4.51210549034022\\
-0.87744140625	4.51984470281935\\
-0.876953125	4.52761622286081\\
-0.87646484375	4.53542006831315\\
-0.8759765625	4.54325625711534\\
-0.87548828125	4.551124807297\\
-0.875	4.55902573697866\\
-0.87451171875	4.56695906437194\\
-0.8740234375	4.57492480777989\\
-0.87353515625	4.58292298559706\\
-0.873046875	4.59095361630995\\
-0.87255859375	4.59901671849717\\
-0.8720703125	4.60711231082958\\
-0.87158203125	4.61524041207074\\
-0.87109375	4.62340104107703\\
-0.87060546875	4.63159421679784\\
-0.8701171875	4.63981995827604\\
-0.86962890625	4.64807828464796\\
-0.869140625	4.65636921514389\\
-0.86865234375	4.66469276908818\\
-0.8681640625	4.6730489658995\\
-0.86767578125	4.68143782509122\\
-0.8671875	4.68985936627147\\
-0.86669921875	4.69831360914367\\
-0.8662109375	4.70680057350647\\
-0.86572265625	4.71532027925432\\
-0.865234375	4.72387274637745\\
-0.86474609375	4.73245799496236\\
-0.8642578125	4.74107604519199\\
-0.86376953125	4.74972691734597\\
-0.86328125	4.75841063180088\\
-0.86279296875	4.7671272090306\\
-0.8623046875	4.7758766696065\\
-0.86181640625	4.78465903419776\\
-0.861328125	4.79347432357156\\
-0.86083984375	4.80232255859351\\
-0.8603515625	4.81120376022772\\
-0.85986328125	4.82011794953725\\
-0.859375	4.82906514768431\\
-0.85888671875	4.83804537593053\\
-0.8583984375	4.84705865563726\\
-0.85791015625	4.85610500826585\\
-0.857421875	4.86518445537792\\
-0.85693359375	4.87429701863568\\
-0.8564453125	4.88344271980215\\
-0.85595703125	4.89262158074149\\
-0.85546875	4.90183362341929\\
-0.85498046875	4.91107886990283\\
-0.8544921875	4.92035734236138\\
-0.85400390625	4.92966906306653\\
-0.853515625	4.93901405439239\\
-0.85302734375	4.94839233881598\\
-0.8525390625	4.95780393891746\\
-0.85205078125	4.96724887738047\\
-0.8515625	4.97672717699239\\
-0.85107421875	4.98623886064463\\
-0.8505859375	4.99578395133301\\
-0.85009765625	5.00536247215793\\
-0.849609375	5.01497444632479\\
-0.84912109375	5.02461989714424\\
-0.8486328125	5.0342988480325\\
-0.84814453125	5.04401132251163\\
-0.84765625	5.05375734420989\\
-0.84716796875	5.06353693686198\\
-0.8466796875	5.07335012430944\\
-0.84619140625	5.08319693050087\\
-0.845703125	5.09307737949233\\
-0.84521484375	5.10299149544759\\
-0.8447265625	5.11293930263841\\
-0.84423828125	5.12292082544499\\
-0.84375	5.13293608835616\\
-0.84326171875	5.1429851159698\\
-0.8427734375	5.153067932993\\
-0.84228515625	5.16318456424256\\
-0.841796875	5.17333503464522\\
-0.84130859375	5.18351936923802\\
-0.8408203125	5.19373759316865\\
-0.84033203125	5.20398973169561\\
-0.83984375	5.2142758101888\\
-0.83935546875	5.22459585412959\\
-0.8388671875	5.23494988911143\\
-0.83837890625	5.24533794083987\\
-0.837890625	5.25576003513318\\
-0.83740234375	5.26621619792247\\
-0.8369140625	5.27670645525214\\
-0.83642578125	5.28723083328026\\
-0.8359375	5.2977893582787\\
-0.83544921875	5.30838205663376\\
-0.8349609375	5.31900895484629\\
-0.83447265625	5.32967007953207\\
-0.833984375	5.34036545742236\\
-0.83349609375	5.35109511536391\\
-0.8330078125	5.36185908031951\\
-0.83251953125	5.3726573793684\\
-0.83203125	5.38349003970645\\
-0.83154296875	5.39435708864665\\
-0.8310546875	5.40525855361934\\
-0.83056640625	5.41619446217269\\
-0.830078125	5.42716484197299\\
-0.82958984375	5.43816972080502\\
-0.8291015625	5.44920912657238\\
-0.82861328125	5.46028308729793\\
-0.828125	5.47139163112409\\
-0.82763671875	5.48253478631318\\
-0.8271484375	5.4937125812479\\
-0.82666015625	5.50492504443153\\
-0.826171875	5.51617220448851\\
-0.82568359375	5.52745409016459\\
-0.8251953125	5.53877073032732\\
-0.82470703125	5.5501221539665\\
-0.82421875	5.56150839019435\\
-0.82373046875	5.57292946824608\\
-0.8232421875	5.58438541748017\\
-0.82275390625	5.59587626737874\\
-0.822265625	5.60740204754802\\
-0.82177734375	5.61896278771864\\
-0.8212890625	5.63055851774606\\
-0.82080078125	5.64218926761094\\
-0.8203125	5.65385506741957\\
-0.81982421875	5.66555594740418\\
-0.8193359375	5.67729193792342\\
-0.81884765625	5.68906306946269\\
-0.818359375	5.70086937263456\\
-0.81787109375	5.71271087817918\\
-0.8173828125	5.72458761696461\\
-0.81689453125	5.73649961998735\\
-0.81640625	5.74844691837263\\
-0.81591796875	5.76042954337476\\
-0.8154296875	5.7724475263778\\
-0.81494140625	5.78450089889561\\
-0.814453125	5.79658969257255\\
-0.81396484375	5.80871393918372\\
-0.8134765625	5.82087367063546\\
-0.81298828125	5.83306891896573\\
-0.8125	5.84529971634447\\
-0.81201171875	5.85756609507415\\
-0.8115234375	5.8698680875901\\
-0.81103515625	5.88220572646088\\
-0.810546875	5.89457904438883\\
-0.81005859375	5.90698807421043\\
-0.8095703125	5.91943284889667\\
-0.80908203125	5.93191340155358\\
-0.80859375	5.9444297654226\\
-0.80810546875	5.95698197388103\\
-0.8076171875	5.96957006044245\\
-0.80712890625	5.98219405875717\\
-0.806640625	5.99485400261263\\
-0.80615234375	6.00754992593392\\
-0.8056640625	6.02028186278411\\
-0.80517578125	6.03304984736481\\
-0.8046875	6.04585391401648\\
-0.80419921875	6.05869409721906\\
-0.8037109375	6.07157043159217\\
-0.80322265625	6.08448295189583\\
-0.802734375	6.09743169303071\\
-0.80224609375	6.11041669003869\\
-0.8017578125	6.1234379781032\\
-0.80126953125	6.13649559254989\\
-0.80078125	6.14958956884688\\
-0.80029296875	6.16271994260527\\
-0.7998046875	6.17588674957971\\
-0.79931640625	6.18909002566873\\
-0.798828125	6.20232980691534\\
-0.79833984375	6.21560612950732\\
-0.7978515625	6.22891902977786\\
-0.79736328125	6.24226854420603\\
-0.796875	6.25565470941709\\
-0.79638671875	6.2690775621831\\
-0.7958984375	6.28253713942348\\
-0.79541015625	6.29603347820522\\
-0.794921875	6.30956661574368\\
-0.79443359375	6.32313658940283\\
-0.7939453125	6.33674343669593\\
-0.79345703125	6.3503871952858\\
-0.79296875	6.36406790298556\\
-0.79248046875	6.37778559775892\\
-0.7919921875	6.3915403177208\\
-0.79150390625	6.40533210113783\\
-0.791015625	6.4191609864287\\
-0.79052734375	6.43302701216488\\
-0.7900390625	6.44693021707096\\
-0.78955078125	6.46087064002524\\
-0.7890625	6.4748483200602\\
-0.78857421875	6.48886329636305\\
-0.7880859375	6.50291560827624\\
-0.78759765625	6.5170052952979\\
-0.787109375	6.53113239708247\\
-0.78662109375	6.54529695344117\\
-0.7861328125	6.55949900434245\\
-0.78564453125	6.57373858991271\\
-0.78515625	6.58801575043663\\
-0.78466796875	6.60233052635779\\
-0.7841796875	6.61668295827921\\
-0.78369140625	6.63107308696383\\
-0.783203125	6.64550095333511\\
-0.78271484375	6.65996659847755\\
-0.7822265625	6.67447006363721\\
-0.78173828125	6.68901139022228\\
-0.78125	6.70359061980363\\
-0.78076171875	6.71820779411533\\
-0.7802734375	6.73286295505519\\
-0.77978515625	6.74755614468541\\
-0.779296875	6.76228740523305\\
-0.77880859375	6.77705677909055\\
-0.7783203125	6.79186430881645\\
-0.77783203125	6.8067100371358\\
-0.77734375	6.8215940069408\\
-0.77685546875	6.83651626129133\\
-0.7763671875	6.85147684341558\\
-0.77587890625	6.86647579671057\\
-0.775390625	6.88151316474277\\
-0.77490234375	6.89658899124862\\
-0.7744140625	6.9117033201352\\
-0.77392578125	6.92685619548072\\
-0.7734375	6.94204766153523\\
-0.77294921875	6.95727776272099\\
-0.7724609375	6.97254654363337\\
-0.77197265625	6.98785404904118\\
-0.771484375	7.00320032388744\\
-0.77099609375	7.01858541328985\\
-0.7705078125	7.0340093625415\\
-0.77001953125	7.04947221711143\\
-0.76953125	7.06497402264525\\
-0.76904296875	7.08051482496571\\
-0.7685546875	7.09609467007342\\
-0.76806640625	7.11171360414733\\
-0.767578125	7.1273716735455\\
-0.76708984375	7.14306892480554\\
-0.7666015625	7.15880540464549\\
-0.76611328125	7.17458115996414\\
-0.765625	7.19039623784195\\
-0.76513671875	7.2062506855415\\
-0.7646484375	7.22214455050825\\
-0.76416015625	7.23807788037101\\
-0.763671875	7.25405072294282\\
-0.76318359375	7.27006312622142\\
-0.7626953125	7.28611513838993\\
-0.76220703125	7.30220680781751\\
-0.76171875	7.31833818306018\\
-0.76123046875	7.33450931286115\\
-0.7607421875	7.35072024615178\\
-0.76025390625	7.36697103205207\\
-0.759765625	7.3832617198715\\
-0.75927734375	7.39959235910942\\
-0.7587890625	7.41596299945606\\
-0.75830078125	7.43237369079297\\
-0.7578125	7.4488244831938\\
-0.75732421875	7.46531542692494\\
-0.7568359375	7.48184657244629\\
-0.75634765625	7.49841797041188\\
-0.755859375	7.51502967167051\\
-0.75537109375	7.53168172726661\\
-0.7548828125	7.54837418844084\\
-0.75439453125	7.56510710663073\\
-0.75390625	7.58188053347154\\
-0.75341796875	7.59869452079687\\
-0.7529296875	7.61554912063944\\
-0.75244140625	7.6324443852317\\
-0.751953125	7.64938036700663\\
-0.75146484375	7.66635711859856\\
-0.7509765625	7.68337469284366\\
-0.75048828125	7.70043314278089\\
-0.75	7.71753252165267\\
-0.74951171875	7.73467288290555\\
-0.7490234375	7.75185428019104\\
-0.74853515625	7.76907676736634\\
-0.748046875	7.78634039849502\\
-0.74755859375	7.80364522784795\\
-0.7470703125	7.82099130990378\\
-0.74658203125	7.83837869934999\\
-0.74609375	7.85580745108351\\
-0.74560546875	7.87327762021141\\
-0.7451171875	7.89078926205194\\
-0.74462890625	7.90834243213495\\
-0.744140625	7.92593718620303\\
-0.74365234375	7.94357358021193\\
-0.7431640625	7.96125167033177\\
-0.74267578125	7.97897151294742\\
-0.7421875	7.99673316465959\\
-0.74169921875	8.01453668228544\\
-0.7412109375	8.03238212285954\\
-0.74072265625	8.05026954363458\\
-0.740234375	8.06819900208213\\
-0.73974609375	8.08617055589369\\
-0.7392578125	8.10418426298119\\
-0.73876953125	8.12224018147812\\
-0.73828125	8.1403383697401\\
-0.73779296875	8.15847888634594\\
-0.7373046875	8.17666179009825\\
-0.73681640625	8.1948871400245\\
-0.736328125	8.21315499537773\\
-0.73583984375	8.23146541563743\\
-0.7353515625	8.24981846051043\\
-0.73486328125	8.26821418993166\\
-0.734375	8.28665266406517\\
-0.73388671875	8.30513394330486\\
-0.7333984375	8.32365808827543\\
-0.73291015625	8.34222515983321\\
-0.732421875	8.36083521906708\\
-0.73193359375	8.37948832729928\\
-0.7314453125	8.39818454608644\\
-0.73095703125	8.41692393722036\\
-0.73046875	8.43570656272891\\
-0.72998046875	8.454532484877\\
-0.7294921875	8.47340176616747\\
-0.72900390625	8.49231446934195\\
-0.728515625	8.51127065738181\\
-0.72802734375	8.53027039350915\\
-0.7275390625	8.54931374118756\\
-0.72705078125	8.56840076412328\\
-0.7265625	8.58753152626598\\
-0.72607421875	8.60670609180963\\
-0.7255859375	8.62592452519377\\
-0.72509765625	8.64518689110406\\
-0.724609375	8.66449325447351\\
-0.72412109375	8.68384368048337\\
-0.7236328125	8.70323823456402\\
-0.72314453125	8.72267698239613\\
-0.72265625	8.74215998991138\\
-0.72216796875	8.76168732329366\\
-0.7216796875	8.78125904898003\\
-0.72119140625	8.80087523366153\\
-0.720703125	8.82053594428435\\
-0.72021484375	8.84024124805092\\
-0.7197265625	8.85999121242067\\
-0.71923828125	8.87978590511119\\
-0.71875	8.89962539409925\\
-0.71826171875	8.91950974762181\\
-0.7177734375	8.93943903417709\\
-0.71728515625	8.95941332252544\\
-0.716796875	8.97943268169065\\
-0.71630859375	8.99949718096072\\
-0.7158203125	9.01960688988923\\
-0.71533203125	9.03976187829605\\
-0.71484375	9.05996221626864\\
-0.71435546875	9.08020797416307\\
-0.7138671875	9.10049922260509\\
-0.71337890625	9.12083603249115\\
-0.712890625	9.14121847498963\\
-0.71240234375	9.16164662154181\\
-0.7119140625	9.18212054386297\\
-0.71142578125	9.2026403139436\\
-0.7109375	9.22320600405041\\
-0.71044921875	9.24381768672754\\
-0.7099609375	9.26447543479758\\
-0.70947265625	9.28517932136278\\
-0.708984375	9.30592941980618\\
-0.70849609375	9.32672580379275\\
-0.7080078125	9.34756854727045\\
-0.70751953125	9.36845772447158\\
-0.70703125	9.3893934099137\\
-0.70654296875	9.41037567840111\\
-0.7060546875	9.43140460502562\\
-0.70556640625	9.45248026516818\\
-0.705078125	9.47360273449967\\
-0.70458984375	9.49477208898239\\
-0.7041015625	9.5159884048711\\
-0.70361328125	9.53725175871423\\
-0.703125	9.55856222735533\\
-0.70263671875	9.57991988793383\\
-0.7021484375	9.6013248178868\\
-0.70166015625	9.62277709494975\\
-0.701171875	9.64427679715817\\
-0.70068359375	9.66582400284865\\
-0.7001953125	9.68741879066013\\
-0.69970703125	9.7090612395352\\
-0.69921875	9.73075142872143\\
-0.69873046875	9.75248943777247\\
-0.6982421875	9.77427534654957\\
-0.69775390625	9.79610923522268\\
-0.697265625	9.81799118427189\\
-0.69677734375	9.83992127448863\\
-0.6962890625	9.8618995869771\\
-0.69580078125	9.88392620315552\\
-0.6953125	9.90600120475739\\
-0.69482421875	9.92812467383304\\
-0.6943359375	9.9502966927508\\
-0.69384765625	9.97251734419843\\
-0.693359375	9.99478671118443\\
-0.69287109375	10.0171048770395\\
-0.6923828125	10.0394719254178\\
-0.69189453125	10.0618879402984\\
-0.69140625	10.0843530059869\\
-0.69091796875	10.1068672071163\\
-0.6904296875	10.1294306286489\\
-0.68994140625	10.1520433558776\\
-0.689453125	10.1747054744271\\
-0.68896484375	10.1974170702557\\
-0.6884765625	10.2201782296563\\
-0.68798828125	10.2429890392585\\
-0.6875	10.2658495860295\\
-0.68701171875	10.2887599572755\\
-0.6865234375	10.3117202406439\\
-0.68603515625	10.334730524124\\
-0.685546875	10.3577908960488\\
-0.68505859375	10.3809014450967\\
-0.6845703125	10.4040622602929\\
-0.68408203125	10.4272734310105\\
-0.68359375	10.4505350469729\\
-0.68310546875	10.4738471982543\\
-0.6826171875	10.4972099752824\\
-0.68212890625	10.5206234688389\\
-0.681640625	10.5440877700618\\
-0.68115234375	10.5676029704464\\
-0.6806640625	10.5911691618476\\
-0.68017578125	10.6147864364809\\
-0.6796875	10.6384548869241\\
-0.67919921875	10.6621746061194\\
-0.6787109375	10.6859456873742\\
-0.67822265625	10.7097682243637\\
-0.677734375	10.7336423111316\\
-0.67724609375	10.7575680420925\\
-0.6767578125	10.7815455120333\\
-0.67626953125	10.8055748161146\\
-0.67578125	10.829656049873\\
-0.67529296875	10.8537893092222\\
-0.6748046875	10.8779746904551\\
-0.67431640625	10.9022122902454\\
-0.673828125	10.9265022056491\\
-0.67333984375	10.9508445341069\\
-0.6728515625	10.9752393734451\\
-0.67236328125	10.999686821878\\
-0.671875	11.0241869780095\\
-0.67138671875	11.0487399408346\\
-0.6708984375	11.0733458097417\\
-0.67041015625	11.0980046845142\\
-0.669921875	11.122716665332\\
-0.66943359375	11.1474818527738\\
-0.6689453125	11.1723003478187\\
-0.66845703125	11.1971722518483\\
-0.66796875	11.222097666648\\
-0.66748046875	11.2470766944095\\
-0.6669921875	11.2721094377326\\
-0.66650390625	11.2971959996265\\
-0.666015625	11.3223364835125\\
-0.66552734375	11.3475309932256\\
-0.6650390625	11.3727796330163\\
-0.66455078125	11.3980825075526\\
-0.6640625	11.4234397219223\\
-0.66357421875	11.4488513816345\\
-0.6630859375	11.474317592622\\
-0.66259765625	11.4998384612428\\
-0.662109375	11.5254140942827\\
-0.66162109375	11.551044598957\\
-0.6611328125	11.5767300829125\\
-0.66064453125	11.6024706542298\\
-0.66015625	11.628266421425\\
-0.65966796875	11.6541174934521\\
-0.6591796875	11.680023979705\\
-0.65869140625	11.7059859900195\\
-0.658203125	11.7320036346758\\
-0.65771484375	11.7580770243998\\
-0.6572265625	11.7842062703663\\
-0.65673828125	11.8103914842005\\
-0.65625	11.8366327779801\\
-0.65576171875	11.8629302642381\\
-0.6552734375	11.8892840559644\\
-0.65478515625	11.9156942666081\\
-0.654296875	11.9421610100803\\
-0.65380859375	11.9686844007555\\
-0.6533203125	11.9952645534745\\
-0.65283203125	12.0219015835465\\
-0.65234375	12.0485956067511\\
-0.65185546875	12.0753467393412\\
-0.6513671875	12.1021550980449\\
-0.65087890625	12.1290208000677\\
-0.650390625	12.1559439630952\\
-0.64990234375	12.1829247052956\\
-0.6494140625	12.2099631453215\\
-0.64892578125	12.2370594023128\\
-0.6484375	12.2642135958988\\
-0.64794921875	12.291425846201\\
-0.6474609375	12.3186962738352\\
-0.64697265625	12.3460249999143\\
-0.646484375	12.3734121460502\\
-0.64599609375	12.400857834357\\
-0.6455078125	12.4283621874531\\
-0.64501953125	12.4559253284638\\
-0.64453125	12.483547381024\\
-0.64404296875	12.5112284692806\\
-0.6435546875	12.538968717895\\
-0.64306640625	12.566768252046\\
-0.642578125	12.5946271974322\\
-0.64208984375	12.6225456802747\\
-0.6416015625	12.6505238273197\\
-0.64111328125	12.6785617658413\\
-0.640625	12.7066596236441\\
-0.64013671875	12.7348175290659\\
-0.6396484375	12.7630356109805\\
-0.63916015625	12.7913139988002\\
-0.638671875	12.8196528224791\\
-0.63818359375	12.8480522125153\\
-0.6376953125	12.8765122999541\\
-0.63720703125	12.9050332163905\\
-0.63671875	12.9336150939724\\
-0.63623046875	12.9622580654032\\
-0.6357421875	12.9909622639449\\
-0.63525390625	13.0197278234207\\
-0.634765625	13.0485548782181\\
-0.63427734375	13.0774435632919\\
-0.6337890625	13.1063940141672\\
-0.63330078125	13.1354063669421\\
-0.6328125	13.1644807582911\\
-0.63232421875	13.1936173254676\\
-0.6318359375	13.2228162063076\\
-0.63134765625	13.2520775392322\\
-0.630859375	13.281401463251\\
-0.63037109375	13.3107881179652\\
-0.6298828125	13.3402376435706\\
-0.62939453125	13.3697501808608\\
-0.62890625	13.3993258712305\\
-0.62841796875	13.4289648566784\\
-0.6279296875	13.4586672798109\\
-0.62744140625	13.4884332838448\\
-0.626953125	13.5182630126112\\
-0.62646484375	13.5481566105579\\
-0.6259765625	13.5781142227538\\
-0.62548828125	13.6081359948913\\
-0.625	13.6382220732903\\
-0.62451171875	13.6683726049011\\
-0.6240234375	13.6985877373084\\
-0.62353515625	13.728867618734\\
-0.623046875	13.7592123980409\\
-0.62255859375	13.7896222247363\\
-0.6220703125	13.8200972489758\\
-0.62158203125	13.8506376215659\\
-0.62109375	13.8812434939686\\
-0.62060546875	13.9119150183042\\
-0.6201171875	13.9426523473556\\
-0.61962890625	13.9734556345712\\
-0.619140625	14.0043250340693\\
-0.61865234375	14.0352607006411\\
-0.6181640625	14.066262789755\\
-0.61767578125	14.0973314575602\\
-0.6171875	14.12846686089\\
-0.61669921875	14.1596691572664\\
-0.6162109375	14.1909385049032\\
-0.61572265625	14.2222750627105\\
-0.615234375	14.2536789902978\\
-0.61474609375	14.2851504479789\\
-0.6142578125	14.3166895967749\\
-0.61376953125	14.3482965984189\\
-0.61328125	14.3799716153594\\
-0.61279296875	14.4117148107649\\
-0.6123046875	14.4435263485275\\
-0.61181640625	14.4754063932673\\
-0.611328125	14.5073551103363\\
-0.61083984375	14.5393726658227\\
-0.6103515625	14.5714592265552\\
-0.60986328125	14.6036149601068\\
-0.609375	14.6358400347995\\
-0.60888671875	14.6681346197082\\
-0.6083984375	14.7004988846653\\
-0.60791015625	14.7329330002649\\
-0.607421875	14.7654371378674\\
-0.60693359375	14.7980114696034\\
-0.6064453125	14.8306561683788\\
-0.60595703125	14.8633714078787\\
-0.60546875	14.8961573625722\\
-0.60498046875	14.9290142077169\\
-0.6044921875	14.9619421193636\\
-0.60400390625	14.9949412743607\\
-0.603515625	15.0280118503589\\
-0.60302734375	15.0611540258158\\
-0.6025390625	15.0943679800011\\
-0.60205078125	15.1276538930007\\
-0.6015625	15.161011945722\\
-0.60107421875	15.1944423198982\\
-0.6005859375	15.2279451980937\\
-0.60009765625	15.261520763709\\
-0.599609375	15.295169200985\\
-0.59912109375	15.3288906950087\\
-0.5986328125	15.3626854317178\\
-0.59814453125	15.396553597906\\
-0.59765625	15.4304953812279\\
-0.59716796875	15.4645109702042\\
-0.5966796875	15.4986005542271\\
-0.59619140625	15.5327643235649\\
-0.595703125	15.567002469368\\
-0.59521484375	15.6013151836739\\
-0.5947265625	15.6357026594124\\
-0.59423828125	15.6701650904109\\
-0.59375	15.7047026714003\\
-0.59326171875	15.7393155980201\\
-0.5927734375	15.7740040668239\\
-0.59228515625	15.8087682752852\\
-0.591796875	15.8436084218026\\
-0.59130859375	15.8785247057059\\
-0.5908203125	15.9135173272615\\
-0.59033203125	15.9485864876782\\
-0.58984375	15.9837323891129\\
-0.58935546875	16.0189552346766\\
-0.5888671875	16.05425522844\\
-0.58837890625	16.0896325754398\\
-0.587890625	16.1250874816842\\
-0.58740234375	16.1606201541594\\
-0.5869140625	16.1962308008352\\
-0.58642578125	16.2319196306715\\
-0.5859375	16.2676868536242\\
-0.58544921875	16.3035326806513\\
-0.5849609375	16.3394573237196\\
-0.58447265625	16.3754609958106\\
-0.583984375	16.4115439109271\\
-0.58349609375	16.4477062840994\\
-0.5830078125	16.483948331392\\
-0.58251953125	16.5202702699099\\
-0.58203125	16.5566723178052\\
-0.58154296875	16.5931546942838\\
-0.5810546875	16.6297176196122\\
-0.58056640625	16.6663613151239\\
-0.580078125	16.7030860032265\\
-0.57958984375	16.7398919074083\\
-0.5791015625	16.7767792522452\\
-0.57861328125	16.8137482634081\\
-0.578125	16.8507991676692\\
-0.57763671875	16.8879321929096\\
-0.5771484375	16.9251475681262\\
-0.57666015625	16.9624455234391\\
-0.576171875	16.9998262900983\\
-0.57568359375	17.037290100492\\
-0.5751953125	17.0748371881528\\
-0.57470703125	17.1124677877659\\
-0.57421875	17.1501821351765\\
-0.57373046875	17.1879804673971\\
-0.5732421875	17.2258630226151\\
-0.57275390625	17.2638300402011\\
-0.572265625	17.3018817607155\\
-0.57177734375	17.3400184259176\\
-0.5712890625	17.3782402787725\\
-0.57080078125	17.4165475634596\\
-0.5703125	17.4549405253801\\
-0.56982421875	17.4934194111659\\
-0.5693359375	17.5319844686866\\
-0.56884765625	17.5706359470588\\
-0.568359375	17.6093740966538\\
-0.56787109375	17.648199169106\\
-0.5673828125	17.6871114173216\\
-0.56689453125	17.7261110954868\\
-0.56640625	17.7651984590763\\
-0.56591796875	17.8043737648624\\
-0.5654296875	17.8436372709234\\
-0.56494140625	17.8829892366523\\
-0.564453125	17.9224299227658\\
-0.56396484375	17.9619595913135\\
-0.5634765625	18.0015785056864\\
-0.56298828125	18.0412869306265\\
-0.5625	18.0810851322355\\
-0.56201171875	18.1209733779846\\
-0.5615234375	18.1609519367234\\
-0.56103515625	18.2010210786895\\
-0.560546875	18.2411810755181\\
-0.56005859375	18.2814322002513\\
-0.5595703125	18.3217747273482\\
-0.55908203125	18.362208932694\\
-0.55859375	18.4027350936108\\
-0.55810546875	18.4433534888665\\
-0.5576171875	18.4840643986856\\
-0.55712890625	18.5248681047588\\
-0.556640625	18.5657648902536\\
-0.55615234375	18.6067550398243\\
-0.5556640625	18.6478388396224\\
-0.55517578125	18.6890165773073\\
-0.5546875	18.7302885420564\\
-0.55419921875	18.7716550245764\\
-0.5537109375	18.8131163171135\\
-0.55322265625	18.8546727134644\\
-0.552734375	18.8963245089876\\
-0.55224609375	18.9380720006138\\
-0.5517578125	18.9799154868575\\
-0.55126953125	19.0218552678282\\
-0.55078125	19.0638916452417\\
-0.55029296875	19.1060249224314\\
-0.5498046875	19.14825540436\\
-0.54931640625	19.1905833976311\\
-0.548828125	19.2330092105012\\
-0.54833984375	19.2755331528912\\
-0.5478515625	19.3181555363985\\
-0.54736328125	19.3608766743091\\
-0.546875	19.4036968816101\\
-0.54638671875	19.4466164750014\\
-0.5458984375	19.4896357729087\\
-0.54541015625	19.5327550954955\\
-0.544921875	19.5759747646765\\
-0.54443359375	19.6192951041294\\
-0.5439453125	19.6627164393088\\
-0.54345703125	19.7062390974583\\
-0.54296875	19.7498634076241\\
-0.54248046875	19.7935897006685\\
-0.5419921875	19.8374183092826\\
-0.54150390625	19.8813495680004\\
-0.541015625	19.9253838132119\\
-0.54052734375	19.9695213831773\\
-0.5400390625	20.0137626180408\\
-0.53955078125	20.0581078598443\\
-0.5390625	20.102557452542\\
-0.53857421875	20.1471117420141\\
-0.5380859375	20.1917710760818\\
-0.53759765625	20.2365358045215\\
-0.537109375	20.2814062790795\\
-0.53662109375	20.3263828534872\\
-0.5361328125	20.3714658834752\\
-0.53564453125	20.4166557267894\\
-0.53515625	20.4619527432058\\
-0.53466796875	20.5073572945456\\
-0.5341796875	20.5528697446913\\
-0.53369140625	20.598490459602\\
-0.533203125	20.6442198073294\\
-0.53271484375	20.690058158034\\
-0.5322265625	20.7360058840006\\
-0.53173828125	20.7820633596553\\
-0.53125	20.8282309615817\\
-0.53076171875	20.8745090685374\\
-0.5302734375	20.9208980614708\\
-0.52978515625	20.9673983235385\\
-0.529296875	21.0140102401217\\
-0.52880859375	21.0607341988439\\
-0.5283203125	21.1075705895883\\
-0.52783203125	21.1545198045155\\
-0.52734375	21.201582238081\\
-0.52685546875	21.2487582870533\\
-0.5263671875	21.2960483505323\\
-0.52587890625	21.3434528299674\\
-0.525390625	21.3909721291757\\
-0.52490234375	21.4386066543614\\
-0.5244140625	21.4863568141339\\
-0.52392578125	21.5342230195274\\
-0.5234375	21.5822056840199\\
-0.52294921875	21.6303052235527\\
-0.5224609375	21.6785220565502\\
-0.52197265625	21.7268566039394\\
-0.521484375	21.7753092891699\\
-0.52099609375	21.8238805382347\\
-0.5205078125	21.87257077969\\
-0.52001953125	21.9213804446762\\
-0.51953125	21.9703099669384\\
-0.51904296875	22.0193597828481\\
-0.5185546875	22.0685303314236\\
-0.51806640625	22.1178220543521\\
-0.517578125	22.1672353960116\\
-0.51708984375	22.2167708034918\\
-0.5166015625	22.2664287266174\\
-0.51611328125	22.3162096179697\\
-0.515625	22.3661139329099\\
-0.51513671875	22.4161421296008\\
-0.5146484375	22.4662946690315\\
-0.51416015625	22.516572015039\\
-0.513671875	22.566974634333\\
-0.51318359375	22.6175029965191\\
-0.5126953125	22.6681575741229\\
-0.51220703125	22.7189388426144\\
-0.51171875	22.7698472804324\\
-0.51123046875	22.8208833690099\\
-0.5107421875	22.8720475927977\\
-0.51025390625	22.9233404392915\\
-0.509765625	22.9747623990562\\
-0.50927734375	23.0263139657521\\
-0.5087890625	23.0779956361614\\
-0.50830078125	23.129807910214\\
-0.5078125	23.1817512910147\\
-0.50732421875	23.2338262848694\\
-0.5068359375	23.2860334013134\\
-0.50634765625	23.338373153138\\
-0.505859375	23.3908460564189\\
-0.50537109375	23.4434526305438\\
-0.5048828125	23.4961933982413\\
-0.50439453125	23.5490688856093\\
-0.50390625	23.6020796221441\\
-0.50341796875	23.6552261407699\\
-0.5029296875	23.7085089778681\\
-0.50244140625	23.7619286733076\\
-0.501953125	23.8154857704749\\
-0.50146484375	23.8691808163049\\
-0.5009765625	23.9230143613113\\
-0.50048828125	23.9769869596188\\
-0.5	24.031099168994\\
-0.49951171875	24.0853515508773\\
-0.4990234375	24.1397446704161\\
-0.49853515625	24.1942790964961\\
-0.498046875	24.2489554017757\\
-0.49755859375	24.3037741627184\\
-0.4970703125	24.3587359596269\\
-0.49658203125	24.4138413766774\\
-0.49609375	24.4690910019536\\
-0.49560546875	24.5244854274823\\
-0.4951171875	24.5800252492678\\
-0.49462890625	24.6357110673283\\
-0.494140625	24.6915434857313\\
-0.49365234375	24.7475231126307\\
-0.4931640625	24.8036505603032\\
-0.49267578125	24.8599264451857\\
-0.4921875	24.9163513879127\\
-0.49169921875	24.9729260133551\\
-0.4912109375	25.0296509506579\\
-0.49072265625	25.0865268332797\\
-0.490234375	25.1435542990319\\
-0.48974609375	25.2007339901187\\
-0.4892578125	25.2580665531768\\
-0.48876953125	25.3155526393171\\
-0.48828125	25.3731929041652\\
-0.48779296875	25.4309880079032\\
-0.4873046875	25.4889386153121\\
-0.48681640625	25.5470453958142\\
-0.486328125	25.605309023516\\
-0.48583984375	25.6637301772524\\
-0.4853515625	25.7223095406303\\
-0.48486328125	25.7810478020733\\
-0.484375	25.8399456548672\\
-0.48388671875	25.8990037972049\\
-0.4833984375	25.9582229322334\\
-0.48291015625	26.0176037680999\\
-0.482421875	26.0771470179993\\
-0.48193359375	26.1368534002219\\
-0.4814453125	26.1967236382018\\
-0.48095703125	26.2567584605659\\
-0.48046875	26.316958601183\\
-0.47998046875	26.3773247992144\\
-0.4794921875	26.4378577991641\\
-0.47900390625	26.49855835093\\
-0.478515625	26.5594272098564\\
-0.47802734375	26.6204651367856\\
-0.4775390625	26.6816728981112\\
-0.47705078125	26.7430512658322\\
-0.4765625	26.8046010176069\\
-0.47607421875	26.8663229368081\\
-0.4755859375	26.9282178125781\\
-0.47509765625	26.9902864398861\\
-0.474609375	27.0525296195842\\
-0.47412109375	27.1149481584652\\
-0.4736328125	27.1775428693212\\
-0.47314453125	27.2403145710024\\
-0.47265625	27.3032640884769\\
-0.47216796875	27.3663922528909\\
-0.4716796875	27.4296999016302\\
-0.47119140625	27.4931878783817\\
-0.470703125	27.5568570331964\\
-0.47021484375	27.6207082225526\\
-0.4697265625	27.6847423094201\\
-0.46923828125	27.7489601633249\\
-0.46875	27.8133626604155\\
-0.46826171875	27.8779506835287\\
-0.4677734375	27.9427251222576\\
-0.46728515625	28.007686873019\\
-0.466796875	28.0728368391233\\
-0.46630859375	28.1381759308434\\
-0.4658203125	28.2037050654859\\
-0.46533203125	28.2694251674627\\
-0.46484375	28.3353371683631\\
-0.46435546875	28.4014420070272\\
-0.4638671875	28.4677406296204\\
-0.46337890625	28.534233989708\\
-0.462890625	28.600923048332\\
-0.46240234375	28.6678087740871\\
-0.4619140625	28.7348921431996\\
-0.46142578125	28.8021741396058\\
-0.4609375	28.8696557550318\\
-0.46044921875	28.9373379890751\\
-0.4599609375	29.0052218492854\\
-0.45947265625	29.0733083512485\\
-0.458984375	29.1415985186697\\
-0.45849609375	29.2100933834591\\
-0.4580078125	29.2787939858172\\
-0.45751953125	29.3477013743227\\
-0.45703125	29.4168166060203\\
-0.45654296875	29.4861407465102\\
-0.4560546875	29.5556748700382\\
-0.45556640625	29.625420059588\\
-0.455078125	29.6953774069734\\
-0.45458984375	29.7655480129325\\
-0.4541015625	29.8359329872226\\
-0.45361328125	29.9065334487167\\
-0.453125	29.9773505255011\\
-0.45263671875	30.0483853549741\\
-0.4521484375	30.1196390839457\\
-0.45166015625	30.1911128687396\\
-0.451171875	30.2628078752951\\
-0.45068359375	30.3347252792716\\
-0.4501953125	30.4068662661532\\
-0.44970703125	30.4792320313558\\
-0.44921875	30.5518237803347\\
-0.44873046875	30.6246427286942\\
-0.4482421875	30.6976901022976\\
-0.44775390625	30.7709671373802\\
-0.447265625	30.8444750806626\\
-0.44677734375	30.9182151894651\\
-0.4462890625	30.9921887318249\\
-0.44580078125	31.0663969866136\\
-0.4453125	31.1408412436572\\
-0.44482421875	31.2155228038564\\
-0.4443359375	31.2904429793097\\
-0.44384765625	31.3656030934372\\
-0.443359375	31.441004481107\\
-0.44287109375	31.5166484887615\\
-0.4423828125	31.5925364745474\\
-0.44189453125	31.6686698084456\\
-0.44140625	31.745049872404\\
-0.44091796875	31.8216780604713\\
-0.4404296875	31.8985557789327\\
-0.43994140625	31.9756844464478\\
-0.439453125	32.0530654941895\\
-0.43896484375	32.130700365985\\
-0.4384765625	32.2085905184588\\
-0.43798828125	32.2867374211777\\
-0.4375	32.365142556797\\
-0.43701171875	32.443807421209\\
-0.4365234375	32.5227335236934\\
-0.43603515625	32.6019223870699\\
-0.435546875	32.6813755478519\\
-0.43505859375	32.7610945564029\\
-0.4345703125	32.8410809770951\\
-0.43408203125	32.9213363884691\\
-0.43359375	33.0018623833967\\
-0.43310546875	33.0826605692448\\
-0.4326171875	33.1637325680422\\
-0.43212890625	33.2450800166479\\
-0.431640625	33.3267045669223\\
-0.43115234375	33.4086078858994\\
-0.4306640625	33.4907916559626\\
-0.43017578125	33.5732575750212\\
-0.4296875	33.6560073566913\\
-0.42919921875	33.7390427304756\\
-0.4287109375	33.8223654419492\\
-0.42822265625	33.9059772529456\\
-0.427734375	33.9898799417448\\
-0.42724609375	34.0740753032649\\
-0.4267578125	34.1585651492558\\
-0.42626953125	34.2433513084944\\
-0.42578125	34.3284356269834\\
-0.42529296875	34.4138199681519\\
-0.4248046875	34.4995062130588\\
-0.42431640625	34.5854962605977\\
-0.423828125	34.6717920277064\\
-0.42333984375	34.7583954495761\\
-0.4228515625	34.8453084798665\\
-0.42236328125	34.9325330909197\\
-0.421875	35.02007127398\\
-0.42138671875	35.1079250394144\\
-0.4208984375	35.1960964169362\\
-0.42041015625	35.2845874558314\\
-0.419921875	35.3734002251877\\
-0.41943359375	35.4625368141257\\
-0.4189453125	35.5519993320337\\
-0.41845703125	35.6417899088038\\
-0.41796875	35.7319106950728\\
-0.41748046875	35.8223638624631\\
-0.4169921875	35.9131516038283\\
-0.41650390625	36.0042761335016\\
-0.416015625	36.0957396875452\\
-0.41552734375	36.1875445240041\\
-0.4150390625	36.2796929231613\\
-0.41455078125	36.3721871877976\\
-0.4140625	36.4650296434514\\
-0.41357421875	36.5582226386839\\
-0.4130859375	36.6517685453453\\
-0.41259765625	36.7456697588437\\
-0.412109375	36.8399286984181\\
-0.41162109375	36.934547807412\\
-0.4111328125	37.0295295535516\\
-0.41064453125	37.1248764292248\\
-0.41015625	37.2205909517644\\
-0.40966796875	37.3166756637323\\
-0.4091796875	37.4131331332074\\
-0.40869140625	37.5099659540754\\
-0.408203125	37.6071767463209\\
-0.40771484375	37.7047681563222\\
-0.4072265625	37.802742857148\\
-0.40673828125	37.9011035488567\\
-0.40625	37.9998529587975\\
-0.40576171875	38.0989938419138\\
-0.4052734375	38.1985289810485\\
-0.40478515625	38.2984611872517\\
-0.404296875	38.3987933000888\\
-0.40380859375	38.4995281879518\\
-0.4033203125	38.600668748372\\
-0.40283203125	38.7022179083328\\
-0.40234375	38.8041786245855\\
-0.40185546875	38.9065538839655\\
-0.4013671875	39.0093467037093\\
-0.40087890625	39.1125601317731\\
-0.400390625	39.2161972471509\\
-0.39990234375	39.3202611601947\\
-0.3994140625	39.4247550129336\\
-0.39892578125	39.5296819793932\\
-0.3984375	39.6350452659158\\
-0.39794921875	39.7408481114791\\
-0.3974609375	39.8470937880144\\
-0.39697265625	39.9537856007242\\
-0.396484375	40.0609268883984\\
-0.39599609375	40.168521023729\\
-0.3955078125	40.2765714136221\\
-0.39501953125	40.385081499509\\
-0.39453125	40.4940547576532\\
-0.39404296875	40.6034946994547\\
-0.3935546875	40.7134048717518\\
-0.39306640625	40.823788857117\\
-0.392578125	40.93465027415\\
-0.39208984375	41.0459927777651\\
-0.3916015625	41.1578200594717\\
-0.39111328125	41.2701358476515\\
-0.390625	41.3829439078257\\
-0.39013671875	41.4962480429171\\
-0.3896484375	41.610052093502\\
-0.38916015625	41.7243599380551\\
-0.388671875	41.8391754931823\\
-0.38818359375	41.9545027138457\\
-0.3876953125	42.0703455935739\\
-0.38720703125	42.1867081646625\\
-0.38671875	42.3035944983592\\
-0.38623046875	42.4210087050358\\
-0.3857421875	42.538954934343\\
-0.38525390625	42.6574373753494\\
-0.384765625	42.7764602566619\\
-0.38427734375	42.8960278465267\\
-0.3837890625	43.0161444529087\\
-0.38330078125	43.1368144235494\\
-0.3828125	43.2580421459995\\
-0.38232421875	43.3798320476278\\
-0.3818359375	43.5021885955995\\
-0.38134765625	43.6251162968279\\
-0.380859375	43.7486196978927\\
-0.38037109375	43.8727033849254\\
-0.3798828125	43.9973719834594\\
-0.37939453125	44.122630158241\\
-0.37890625	44.2484826129994\\
-0.37841796875	44.3749340901748\\
-0.3779296875	44.5019893705987\\
-0.37744140625	44.6296532731259\\
-0.376953125	44.7579306542138\\
-0.37646484375	44.8868264074462\\
-0.3759765625	45.0163454629975\\
-0.37548828125	45.1464927870348\\
-0.375	45.2772733810524\\
-0.37451171875	45.4086922811352\\
-0.3740234375	45.5407545571476\\
-0.37353515625	45.673465311841\\
-0.373046875	45.8068296798775\\
-0.37255859375	45.9408528267622\\
-0.3720703125	46.0755399476811\\
-0.37158203125	46.2108962662359\\
-0.37109375	46.3469270330724\\
-0.37060546875	46.4836375243942\\
-0.3701171875	46.6210330403547\\
-0.36962890625	46.7591189033211\\
-0.369140625	46.8979004560034\\
-0.36865234375	47.0373830594371\\
-0.3681640625	47.1775720908141\\
-0.36767578125	47.3184729411525\\
-0.3671875	47.4600910127931\\
-0.36669921875	47.6024317167155\\
-0.3662109375	47.7455004696614\\
-0.36572265625	47.8893026910542\\
-0.365234375	48.033843799702\\
-0.36474609375	48.179129210274\\
-0.3642578125	48.3251643295321\\
-0.36376953125	48.4719545523107\\
-0.36328125	48.619505257221\\
-0.36279296875	48.7678218020718\\
-0.3623046875	48.9169095189855\\
-0.36181640625	49.0667737091939\\
-0.361328125	49.2174196374936\\
-0.36083984375	49.3688525263415\\
-0.3603515625	49.5210775495729\\
-0.35986328125	49.6740998257153\\
-0.359375	49.8279244108782\\
-0.35888671875	49.9825562911932\\
-0.3583984375	50.1380003747794\\
-0.35791015625	50.2942614832074\\
-0.357421875	50.4513443424307\\
-0.35693359375	50.6092535731582\\
-0.3564453125	50.7679936806336\\
-0.35595703125	50.9275690437872\\
-0.35546875	51.0879839037302\\
-0.35498046875	51.2492423515471\\
-0.3544921875	51.4113483153515\\
-0.35400390625	51.5743055465665\\
-0.353515625	51.73811760538\\
-0.35302734375	51.902787845335\\
-0.3525390625	52.0683193970059\\
-0.35205078125	52.2347151507108\\
-0.3515625	52.4019777382049\\
-0.35107421875	52.5701095133057\\
-0.3505859375	52.7391125313864\\
-0.35009765625	52.9089885276818\\
-0.349609375	53.0797388943417\\
-0.34912109375	53.2513646561661\\
-0.3486328125	53.4238664449548\\
-0.34814453125	53.5972444723984\\
-0.34765625	53.7714985014381\\
-0.34716796875	53.9466278160155\\
-0.3466796875	54.1226311891328\\
-0.34619140625	54.299506849141\\
-0.345703125	54.477252444169\\
-0.34521484375	54.6558650046058\\
-0.3447265625	54.8353409035437\\
-0.34423828125	55.0156758150904\\
-0.34375	55.196864670452\\
-0.34326171875	55.3789016116897\\
-0.3427734375	55.5617799430525\\
-0.34228515625	55.7454920797806\\
-0.341796875	55.9300294942809\\
-0.34130859375	56.1153826595749\\
-0.3408203125	56.3015409899083\\
-0.34033203125	56.4884927784337\\
-0.33984375	56.6762251318575\\
-0.33935546875	56.864723901961\\
-0.3388671875	57.0539736139021\\
-0.33837890625	57.2439573912131\\
-0.337890625	57.4346568774122\\
-0.33740234375	57.62605215416\\
-0.3369140625	57.8181216559043\\
-0.33642578125	58.0108420809569\\
-0.3359375	58.2041882989795\\
-0.33544921875	58.3981332548573\\
-0.3349609375	58.5926478689684\\
-0.33447265625	58.7877009338795\\
-0.333984375	58.9832590075219\\
-0.33349609375	59.1792863029395\\
-0.3330078125	59.3757445747296\\
-0.33251953125	59.5725930023425\\
-0.33203125	59.7697880704489\\
-0.33154296875	59.9672834466338\\
-0.3310546875	60.1650298567332\\
-0.33056640625	60.3629749581929\\
-0.330078125	60.561063211897\\
-0.32958984375	60.759235752986\\
-0.3291015625	60.95743026127\\
-0.32861328125	61.1555808319345\\
-0.328125	61.3536178473249\\
-0.32763671875	61.5514678507048\\
-0.3271484375	61.7490534229921\\
-0.32666015625	61.9462930635953\\
-0.326171875	62.1431010765927\\
-0.32568359375	62.3393874636229\\
-0.3251953125	62.5350578249932\\
-0.32470703125	62.7300132706456\\
-0.32421875	62.9241503427443\\
-0.32373046875	63.117360951804\\
-0.3232421875	63.309532328392\\
-0.32275390625	63.5005469925674\\
-0.322265625	63.6902827433362\\
-0.32177734375	63.8786126705062\\
-0.3212890625	64.0654051914008\\
-0.32080078125	64.2505241149583\\
-0.3203125	64.4338287357758\\
-0.31982421875	64.6151739606567\\
-0.3193359375	64.7944104701872\\
-0.31884765625	64.9713849177725\\
-0.318359375	65.1459401684539\\
-0.31787109375	65.3179155796146\\
-0.3173828125	65.4871473254584\\
-0.31689453125	65.6534687668069\\
-0.31640625	65.8167108674067\\
-0.31591796875	65.9767026574616\\
-0.3154296875	66.1332717445949\\
-0.31494140625	66.2862448718508\\
-0.314453125	66.4354485216783\\
-0.31396484375	66.580709564131\\
-0.3134765625	66.7218559467271\\
-0.31298828125	66.8587174226168\\
-0.3125	66.9911263128545\\
-0.31201171875	67.1189182977163\\
-0.3115234375	67.2419332311829\\
-0.31103515625	67.3600159718705\\
-0.310546875	67.4730172229609\\
-0.31005859375	67.5807943729936\\
-0.3095703125	67.6832123288022\\
-0.30908203125	67.7801443314464\\
-0.30859375	67.8714727456758\\
-0.30810546875	67.9570898133397\\
-0.3076171875	68.0368983612048\\
-0.30712890625	68.1108124538873\\
-0.306640625	68.1787579830524\\
-0.30615234375	68.2406731846793\\
-0.3056640625	68.2965090770073\\
-0.30517578125	68.3462298128157\\
-0.3046875	68.3898129408252\\
-0.30419921875	68.4272495723253\\
-0.3037109375	68.4585444505216\\
-0.30322265625	68.4837159215524\\
-0.302734375	68.5027958076184\\
-0.30224609375	68.5158291841349\\
-0.3017578125	68.5228740642496\\
-0.30126953125	68.5240009954097\\
-0.30078125	68.5192925738771\\
-0.30029296875	68.5088428841771\\
-0.2998046875	68.492756871359\\
-0.29931640625	68.4711496546905\\
-0.298828125	68.4441457919044\\
-0.29833984375	68.4118785034741\\
-0.2978515625	68.3744888664957\\
-0.29736328125	68.3321249877169\\
-0.296875	68.2849411650105\\
-0.29638671875	68.2330970462031\\
-0.2958984375	68.1767567936399\\
-0.29541015625	68.1160882622431\\
-0.294921875	68.0512621980729\\
-0.29443359375	67.9824514636294\\
-0.2939453125	67.9098302952895\\
-0.29345703125	67.8335735974305\\
-0.29296875	67.7538562769459\\
-0.29248046875	67.6708526210228\\
-0.2919921875	67.5847357202693\\
-0.29150390625	67.4956769385359\\
-0.291015625	67.4038454300775\\
-0.29052734375	67.3094077040989\\
-0.2900390625	67.2125272361586\\
-0.28955078125	67.1133641254286\\
-0.2890625	67.0120747964072\\
-0.28857421875	66.9088117433258\\
-0.2880859375	66.8037233152225\\
-0.28759765625	66.6969535394521\\
-0.287109375	66.5886419812313\\
-0.28662109375	66.4789236367293\\
-0.2861328125	66.3679288571563\\
-0.28564453125	66.2557833012831\\
-0.28515625	66.1426079138498\\
-0.28466796875	66.028518927375\\
-0.2841796875	65.9136278849461\\
-0.28369140625	65.7980416816671\\
-0.283203125	65.6818626225554\\
-0.28271484375	65.5651884947969\\
-0.2822265625	65.4481126523927\\
-0.28173828125	65.3307241113729\\
-0.28125	65.2131076538837\\
-0.28076171875	65.0953439395874\\
-0.2802734375	64.9775096229562\\
-0.27978515625	64.8596774751534\\
-0.279296875	64.7419165093465\\
-0.27880859375	64.6242921083905\\
-0.2783203125	64.5068661539448\\
-0.27783203125	64.3896971561948\\
-0.27734375	64.2728403834462\\
-0.27685546875	64.1563479909424\\
-0.2763671875	64.0402691483672\\
-0.27587890625	63.9246501655361\\
-0.275390625	63.8095346158857\\
-0.27490234375	63.6949634574096\\
-0.2744140625	63.580975150769\\
-0.27392578125	63.4676057743375\\
-0.2734375	63.3548891360111\\
-0.27294921875	63.2428568816268\\
-0.2724609375	63.13153859991\\
-0.27197265625	63.0209619238572\\
-0.271484375	62.911152628534\\
-0.27099609375	62.8021347252597\\
-0.2705078125	62.6939305521936\\
-0.27001953125	62.5865608613434\\
-0.26953125	62.4800449020427\\
-0.26904296875	62.3744005009456\\
-0.2685546875	62.2696441386159\\
-0.26806640625	62.1657910227712\\
-0.267578125	62.0628551582836\\
-0.26708984375	61.9608494140116\\
-0.2666015625	61.8597855865679\\
-0.26611328125	61.7596744611132\\
-0.265625	61.6605258692777\\
-0.26513671875	61.5623487443193\\
-0.2646484375	61.4651511736075\\
-0.26416015625	61.3689404485475\\
-0.263671875	61.2737231120353\\
-0.26318359375	61.179505003554\\
-0.2626953125	61.0862913020031\\
-0.26220703125	60.9940865663583\\
-0.26171875	60.9028947742628\\
-0.26123046875	60.8127193586328\\
-0.2607421875	60.7235632423755\\
-0.26025390625	60.635428871301\\
-0.259765625	60.5483182453181\\
-0.25927734375	60.4622329479869\\
-0.2587890625	60.3771741745174\\
-0.25830078125	60.293142758281\\
-0.2578125	60.2101391959111\\
-0.25732421875	60.1281636710624\\
-0.2568359375	60.0472160768986\\
-0.25634765625	59.9672960373664\\
-0.255859375	59.8884029273203\\
-0.25537109375	59.8105358915618\\
-0.2548828125	59.7336938628392\\
-0.25439453125	59.6578755788709\\
-0.25390625	59.583079598436\\
-0.25341796875	59.5093043165853\\
-0.2529296875	59.4365479790195\\
-0.25244140625	59.3648086956728\\
-0.251953125	59.2940844535522\\
-0.25146484375	59.2243731288663\\
-0.2509765625	59.1556724984825\\
-0.25048828125	59.0879802507524\\
-0.25	59.0212939957344\\
-0.24951171875	58.9556112748499\\
-0.2490234375	58.8909295700038\\
-0.24853515625	58.8272463121949\\
-0.248046875	58.7645588896486\\
-0.24755859375	58.7028646554957\\
-0.2470703125	58.6421609350248\\
-0.24658203125	58.582445032526\\
-0.24609375	58.5237142377595\\
-0.24560546875	58.465965832055\\
-0.2451171875	58.4091970940807\\
-0.24462890625	58.3534053052807\\
-0.244140625	58.2985877550204\\
-0.24365234375	58.2447417454347\\
-0.2431640625	58.1918645960181\\
-0.24267578125	58.1399536479512\\
-0.2421875	58.089006268196\\
-0.24169921875	58.0390198533567\\
-0.2412109375	57.9899918333398\\
-0.24072265625	57.9419196748033\\
-0.240234375	57.8948008844246\\
-0.23974609375	57.848633011989\\
-0.2392578125	57.8034136533142\\
-0.23876953125	57.7591404530155\\
-0.23828125	57.715811107126\\
-0.23779296875	57.6734233655807\\
-0.2373046875	57.6319750345679\\
-0.23681640625	57.5914639787638\\
-0.236328125	57.5518881234504\\
-0.23583984375	57.5132454565289\\
-0.2353515625	57.4755340304346\\
-0.23486328125	57.4387519639588\\
-0.234375	57.4028974439823\\
-0.23388671875	57.3679687271331\\
-0.2333984375	57.3339641413642\\
-0.23291015625	57.3008820874664\\
-0.232421875	57.2687210405139\\
-0.23193359375	57.2374795512505\\
-0.2314453125	57.2071562474222\\
-0.23095703125	57.1777498350589\\
-0.23046875	57.1492590997069\\
-0.22998046875	57.1216829076231\\
-0.2294921875	57.0950202069271\\
-0.22900390625	57.0692700287186\\
-0.228515625	57.0444314881633\\
-0.22802734375	57.0205037855509\\
-0.2275390625	56.9974862073259\\
-0.22705078125	56.9753781270963\\
-0.2265625	56.9541790066225\\
-0.22607421875	56.9338883967901\\
-0.2255859375	56.9145059385663\\
-0.22509765625	56.8960313639453\\
-0.224609375	56.8784644968839\\
-0.22412109375	56.8618052542286\\
-0.2236328125	56.8460536466402\\
-0.22314453125	56.8312097795106\\
-0.22265625	56.8172738538825\\
-0.22216796875	56.804246167367\\
-0.2216796875	56.7921271150671\\
-0.22119140625	56.7809171905012\\
-0.220703125	56.7706169865355\\
-0.22021484375	56.7612271963239\\
-0.2197265625	56.7527486142564\\
-0.21923828125	56.7451821369195\\
-0.21875	56.7385287640679\\
-0.21826171875	56.7327895996103\\
-0.2177734375	56.7279658526116\\
-0.21728515625	56.7240588383115\\
-0.216796875	56.7210699791592\\
-0.21630859375	56.7190008058706\\
-0.2158203125	56.7178529585037\\
-0.21533203125	56.71762818756\\
-0.21484375	56.7183283551008\\
-0.21435546875	56.7199554358984\\
-0.2138671875	56.7225115186052\\
-0.21337890625	56.7259988069536\\
-0.212890625	56.7304196209808\\
-0.21240234375	56.735776398287\\
-0.2119140625	56.7420716953182\\
-0.21142578125	56.7493081886858\\
-0.2109375	56.757488676514\\
-0.21044921875	56.7666160798243\\
-0.2099609375	56.7766934439511\\
-0.20947265625	56.7877239399965\\
-0.208984375	56.7997108663181\\
-0.20849609375	56.8126576500552\\
-0.2080078125	56.8265678486972\\
-0.20751953125	56.8414451516844\\
-0.20703125	56.8572933820555\\
-0.20654296875	56.8741164981353\\
-0.2060546875	56.8919185952608\\
-0.20556640625	56.9107039075548\\
-0.205078125	56.9304768097425\\
-0.20458984375	56.9512418190118\\
-0.2041015625	56.9730035969211\\
-0.20361328125	56.9957669513521\\
-0.203125	57.0195368385104\\
-0.20263671875	57.0443183649752\\
-0.2021484375	57.0701167897982\\
-0.20166015625	57.0969375266496\\
-0.201171875	57.124786146015\\
-0.20068359375	57.1536683774466\\
-0.2001953125	57.1835901118592\\
-0.19970703125	57.2145574038853\\
-0.19921875	57.2465764742759\\
-0.19873046875	57.2796537123573\\
-0.1982421875	57.3137956785406\\
-0.19775390625	57.3490091068829\\
-0.197265625	57.385300907701\\
-0.19677734375	57.4226781702404\\
-0.1962890625	57.4611481653953\\
-0.19580078125	57.5007183484807\\
-0.1953125	57.5413963620574\\
-0.19482421875	57.5831900388079\\
-0.1943359375	57.6261074044642\\
-0.19384765625	57.6701566807829\\
-0.193359375	57.7153462885706\\
-0.19287109375	57.761684850755\\
-0.1923828125	57.8091811955019\\
-0.19189453125	57.8578443593753\\
-0.19140625	57.9076835905393\\
-0.19091796875	57.958708351996\\
-0.1904296875	58.0109283248606\\
-0.18994140625	58.0643534116672\\
-0.189453125	58.1189937397036\\
-0.18896484375	58.1748596643674\\
-0.1884765625	58.2319617725436\\
-0.18798828125	58.2903108859908\\
-0.1875	58.3499180647378\\
-0.18701171875	58.4107946104782\\
-0.1865234375	58.4729520699562\\
-0.18603515625	58.5364022383358\\
-0.185546875	58.6011571625457\\
-0.18505859375	58.6672291445828\\
-0.1845703125	58.7346307447709\\
-0.18408203125	58.8033747849561\\
-0.18359375	58.8734743516243\\
-0.18310546875	58.9449427989275\\
-0.1826171875	59.0177937516038\\
-0.18212890625	59.092041107761\\
-0.181640625	59.1676990415166\\
-0.18115234375	59.2447820054611\\
-0.1806640625	59.3233047329213\\
-0.18017578125	59.4032822399962\\
-0.1796875	59.4847298273362\\
-0.17919921875	59.5676630816238\\
-0.1787109375	59.6520978767306\\
-0.17822265625	59.7380503744929\\
-0.177734375	59.8255370250818\\
-0.17724609375	59.9145745668937\\
-0.1767578125	60.0051800259308\\
-0.17626953125	60.09737071459\\
-0.17578125	60.1911642298136\\
-0.17529296875	60.2865784505156\\
-0.1748046875	60.383631534214\\
-0.17431640625	60.4823419127766\\
-0.173828125	60.5827282871903\\
-0.17333984375	60.6848096212481\\
-0.1728515625	60.7886051340381\\
-0.17236328125	60.8941342911168\\
-0.171875	61.0014167942212\\
-0.17138671875	61.1104725693742\\
-0.1708984375	61.2213217532291\\
-0.17041015625	61.3339846774506\\
-0.169921875	61.448481850964\\
-0.16943359375	61.5648339398341\\
-0.1689453125	61.6830617445539\\
-0.16845703125	61.8031861744755\\
-0.16796875	61.9252282191062\\
-0.16748046875	62.0492089159593\\
-0.1669921875	62.1751493146191\\
-0.16650390625	62.3030704366541\\
-0.166015625	62.4329932309668\\
-0.16552734375	62.564938524143\\
-0.1650390625	62.6989269653144\\
-0.16455078125	62.8349789649996\\
-0.1640625	62.973114627349\\
-0.16357421875	63.1133536751599\\
-0.1630859375	63.2557153669663\\
-0.16259765625	63.4002184054572\\
-0.162109375	63.5468808363865\\
-0.16162109375	63.6957199370888\\
-0.1611328125	63.8467520936192\\
-0.16064453125	63.9999926654479\\
-0.16015625	64.1554558365588\\
-0.15966796875	64.3131544516855\\
-0.1591796875	64.4730998363339\\
-0.15869140625	64.6353015990927\\
-0.158203125	64.7997674146556\\
-0.15771484375	64.9665027858111\\
-0.1572265625	65.1355107825462\\
-0.15673828125	65.3067917562661\\
-0.15625	65.4803430269885\\
-0.15576171875	65.6561585412204\\
-0.1552734375	65.8342284980787\\
-0.15478515625	66.014538941084\\
-0.154296875	66.1970713128892\\
-0.15380859375	66.3818019701098\\
-0.1533203125	66.568701655262\\
-0.15283203125	66.7577349227685\\
-0.15234375	66.9488595158882\\
-0.15185546875	67.1420256914253\\
-0.1513671875	67.3371754890906\\
-0.15087890625	67.5342419424937\\
-0.150390625	67.7331482289192\\
-0.14990234375	67.933806755328\\
-0.1494140625	68.136118178458\\
-0.14892578125	68.3399703574553\\
-0.1484375	68.5452372382478\\
-0.14794921875	68.7517776698809\\
-0.1474609375	68.9594341542757\\
-0.14697265625	69.1680315324715\\
-0.146484375	69.3773756123332\\
-0.14599609375	69.5872517450667\\
-0.1455078125	69.7974233606739\\
-0.14501953125	70.0076304757999\\
-0.14453125	70.2175881912539\\
-0.14404296875	70.4269852009213\\
-0.1435546875	70.6354823387872\\
-0.14306640625	70.8427111963171\\
-0.142578125	71.0482728485693\\
-0.14208984375	71.2517367338479\\
-0.1416015625	71.4526397384932\\
-0.14111328125	71.6504855451667\\
-0.140625	71.8447443094877\\
-0.14013671875	72.0348527357361\\
-0.1396484375	72.2202146269543\\
-0.13916015625	72.4002019877052\\
-0.138671875	72.5741567581536\\
-0.13818359375	72.7413932554212\\
-0.1376953125	72.9012013914191\\
-0.13720703125	73.0528507250429\\
-0.13671875	73.1955953898912\\
-0.13623046875	73.328679916338\\
-0.1357421875	73.4513459386407\\
-0.13525390625	73.5628397441769\\
-0.134765625	73.6624205837011\\
-0.13427734375	73.7493696201965\\
-0.1337890625	73.8229993513813\\
-0.13330078125	73.882663299975\\
-0.1328125	73.9277657292276\\
-0.13232421875	73.9577711124055\\
-0.1318359375	73.9722130667966\\
-0.13134765625	73.9707024582023\\
-0.130859375	73.9529343925971\\
-0.13037109375	73.9186938384844\\
-0.1298828125	73.8678596659413\\
-0.12939453125	73.8004069444514\\
-0.12890625	73.7164074081735\\
-0.12841796875	73.6160280700234\\
-0.1279296875	73.499528039915\\
-0.12744140625	73.3672536727569\\
-0.126953125	73.2196322335971\\
-0.12646484375	73.0571643169712\\
-0.1259765625	72.8804152923053\\
-0.12548828125	72.6900060660543\\
-0.125	72.4866034541106\\
-0.12451171875	72.2709104464439\\
-0.1240234375	72.0436566220469\\
-0.12353515625	71.8055889392126\\
-0.123046875	71.5574630869815\\
-0.12255859375	71.3000355414952\\
-0.1220703125	71.0340564288361\\
-0.12158203125	70.7602632558878\\
-0.12109375	70.4793755347443\\
-0.12060546875	70.1920902952593\\
-0.1201171875	69.8990784550181\\
-0.11962890625	69.600981996559\\
-0.119140625	69.2984118876881\\
-0.11865234375	68.9919466717795\\
-0.1181640625	68.6821316502948\\
-0.11767578125	68.3694785787368\\
-0.1171875	68.0544657989452\\
-0.11669921875	67.7375387345403\\
-0.1162109375	67.4191106816141\\
-0.11572265625	67.0995638329654\\
-0.115234375	66.7792504809114\\
-0.11474609375	66.4584943504658\\
-0.1142578125	66.1375920213229\\
-0.11376953125	65.8168144034079\\
-0.11328125	65.4964082365368\\
-0.11279296875	65.1765975900709\\
-0.1123046875	64.8575853430991\\
-0.11181640625	64.539554629867\\
-0.111328125	64.2226702387086\\
-0.11083984375	63.9070799558049\\
-0.1103515625	63.5929158476588\\
-0.10986328125	63.2802954783071\\
-0.109375	62.9693230590517\\
-0.10888671875	62.6600905298974\\
-0.1083984375	62.352678573021\\
-0.10791015625	62.0471575594643\\
-0.107421875	61.7435884309267\\
-0.10693359375	61.4420235190123\\
-0.1064453125	61.1425073046503\\
-0.10595703125	60.8450771206388\\
-0.10546875	60.5497638003836\\
-0.10498046875	60.2565922759931\\
-0.1044921875	59.9655821288604\\
-0.10400390625	59.6767480958429\\
-0.103515625	59.3901005340556\\
-0.10302734375	59.1056458471864\\
-0.1025390625	58.8233868761374\\
-0.10205078125	58.5433232566292\\
-0.1015625	58.2654517462853\\
-0.10107421875	57.989766523563\\
-0.1005859375	57.7162594607371\\
-0.10009765625	57.4449203730155\\
-0.099609375	57.1757372457085\\
-0.09912109375	56.9086964412524\\
-0.0986328125	56.6437828877396\\
-0.09814453125	56.3809802505034\\
-0.09765625	56.1202710881717\\
-0.09716796875	55.861636994502\\
-0.0966796875	55.6050587272047\\
-0.09619140625	55.3505163248669\\
-0.095703125	55.0979892129948\\
-0.09521484375	54.8474563001135\\
-0.0947265625	54.5988960647859\\
-0.09423828125	54.3522866343347\\
-0.09375	54.1076058559947\\
-0.09326171875	53.8648313611569\\
-0.0927734375	53.6239406233054\\
-0.09228515625	53.3849110102084\\
-0.091796875	53.1477198308651\\
-0.09130859375	52.9123443776724\\
-0.0908203125	52.6787619642435\\
-0.09033203125	52.4469499592501\\
-0.08984375	52.2168858166585\\
-0.08935546875	51.9885471026741\\
-0.0888671875	51.7619115196901\\
-0.08837890625	51.5369569275151\\
-0.087890625	51.3136613621192\\
-0.08740234375	51.0920030521313\\
-0.0869140625	50.8719604332868\\
-0.08642578125	50.6535121610165\\
-0.0859375	50.4366371213481\\
-0.08544921875	50.2213144402747\\
-0.0849609375	50.0075234917355\\
-0.08447265625	49.7952439043379\\
-0.083984375	49.5844555669389\\
-0.08349609375	49.3751386331966\\
-0.0830078125	49.1672735251886\\
-0.08251953125	48.9608409361898\\
-0.08203125	48.7558218326897\\
-0.08154296875	48.5521974557273\\
-0.0810546875	48.349949321608\\
-0.08056640625	48.149059222072\\
-0.080078125	47.9495092239614\\
-0.07958984375	47.7512816684485\\
-0.0791015625	47.5543591698646\\
-0.07861328125	47.3587246141778\\
-0.078125	47.1643611571541\\
-0.07763671875	46.9712522222446\\
-0.0771484375	46.7793814982235\\
-0.07666015625	46.5887329366122\\
-0.076171875	46.3992907489158\\
-0.07568359375	46.2110394036927\\
-0.0751953125	46.0239636234855\\
-0.07470703125	45.8380483816275\\
-0.07421875	45.6532788989484\\
-0.07373046875	45.4696406403925\\
-0.0732421875	45.2871193115651\\
-0.07275390625	45.1057008552221\\
-0.072265625	44.9253714477146\\
-0.07177734375	44.7461174953976\\
-0.0712890625	44.5679256310167\\
-0.07080078125	44.3907827100788\\
-0.0703125	44.2146758072168\\
-0.06982421875	44.0395922125536\\
-0.0693359375	43.8655194280757\\
-0.06884765625	43.6924451640172\\
-0.068359375	43.5203573352636\\
-0.06787109375	43.3492440577797\\
-0.0673828125	43.1790936450627\\
-0.06689453125	43.0098946046277\\
-0.06640625	42.841635634527\\
-0.06591796875	42.6743056199062\\
-0.0654296875	42.5078936296008\\
-0.06494140625	42.342388912773\\
-0.064453125	42.1777808955938\\
-0.06396484375	42.0140591779688\\
-0.0634765625	41.851213530312\\
-0.06298828125	41.6892338903651\\
-0.0625	41.5281103600676\\
-0.06201171875	41.3678332024753\\
-0.0615234375	41.2083928387277\\
-0.06103515625	41.0497798450665\\
-0.060546875	40.8919849499052\\
-0.06005859375	40.7349990309478\\
-0.0595703125	40.5788131123594\\
-0.05908203125	40.4234183619869\\
-0.05859375	40.2688060886297\\
-0.05810546875	40.1149677393618\\
-0.0576171875	39.9618948969026\\
-0.05712890625	39.8095792770374\\
-0.056640625	39.6580127260886\\
-0.05615234375	39.5071872184324\\
-0.0556640625	39.357094854067\\
-0.05517578125	39.2077278562256\\
-0.0546875	39.0590785690384\\
-0.05419921875	38.911139455239\\
-0.0537109375	38.7639030939189\\
-0.05322265625	38.617362178324\\
-0.052734375	38.4715095136988\\
-0.05224609375	38.326338015171\\
-0.0517578125	38.1818407056815\\
-0.05126953125	38.0380107139554\\
-0.05078125	37.8948412725145\\
-0.05029296875	37.7523257157313\\
-0.0498046875	37.6104574779218\\
-0.04931640625	37.4692300914791\\
-0.048828125	37.3286371850445\\
-0.04833984375	37.1886724817166\\
-0.0478515625	37.0493297972975\\
-0.04736328125	36.9106030385761\\
-0.046875	36.7724862016459\\
-0.04638671875	36.6349733702586\\
-0.0458984375	36.4980587142116\\
-0.04541015625	36.3617364877691\\
-0.044921875	36.2260010281162\\
-0.04443359375	36.090846753845\\
-0.0439453125	35.9562681634726\\
-0.04345703125	35.8222598339889\\
-0.04296875	35.6888164194359\\
-0.04248046875	35.5559326495165\\
-0.0419921875	35.42360332823\\
-0.04150390625	35.2918233325397\\
-0.041015625	35.1605876110647\\
-0.04052734375	35.0298911828008\\
-0.0400390625	34.8997291358671\\
-0.03955078125	34.7700966262788\\
-0.0390625	34.6409888767452\\
-0.03857421875	34.5124011754932\\
-0.0380859375	34.3843288751133\\
-0.03759765625	34.2567673914315\\
-0.037109375	34.1297122024031\\
-0.03662109375	34.0031588470292\\
-0.0361328125	33.8771029242961\\
-0.03564453125	33.7515400921355\\
-0.03515625	33.6264660664065\\
-0.03466796875	33.5018766198987\\
-0.0341796875	33.3777675813542\\
-0.03369140625	33.2541348345109\\
-0.033203125	33.1309743171645\\
-0.03271484375	33.0082820202494\\
-0.0322265625	32.8860539869382\\
-0.03173828125	32.76428631176\\
-0.03125	32.6429751397349\\
-0.03076171875	32.5221166655271\\
-0.0302734375	32.4017071326154\\
-0.02978515625	32.2817428324779\\
-0.029296875	32.1622201037955\\
-0.02880859375	32.0431353316698\\
-0.0283203125	31.924484946857\\
-0.02783203125	31.8062654250165\\
-0.02734375	31.6884732859748\\
-0.02685546875	31.5711050930035\\
-0.0263671875	31.4541574521115\\
-0.02587890625	31.3376270113516\\
-0.025390625	31.2215104601396\\
-0.02490234375	31.1058045285876\\
-0.0244140625	30.9905059868494\\
-0.02392578125	30.8756116444799\\
-0.0234375	30.7611183498047\\
-0.02294921875	30.6470229893045\\
-0.0224609375	30.5333224870087\\
-0.02197265625	30.4200138039027\\
-0.021484375	30.3070939373454\\
-0.02099609375	30.1945599204982\\
-0.0205078125	30.082408821764\\
-0.02001953125	29.9706377442387\\
-0.01953125	29.8592438251713\\
-0.01904296875	29.7482242354343\\
-0.0185546875	29.6375761790054\\
-0.01806640625	29.5272968924575\\
-0.017578125	29.4173836444588\\
-0.01708984375	29.3078337352825\\
-0.0166015625	29.1986444963244\\
-0.01611328125	29.0898132896315\\
-0.015625	28.9813375074373\\
-0.01513671875	28.8732145717066\\
-0.0146484375	28.7654419336886\\
-0.01416015625	28.6580170734781\\
-0.013671875	28.5509374995847\\
-0.01318359375	28.44420074851\\
-0.0126953125	28.337804384332\\
-0.01220703125	28.2317459982977\\
-0.01171875	28.1260232084224\\
-0.01123046875	28.0206336590969\\
-0.0107421875	27.9155750207009\\
-0.01025390625	27.8108449892244\\
-0.009765625	27.7064412858944\\
-0.00927734375	27.6023616568098\\
-0.0087890625	27.4986038725818\\
-0.00830078125	27.3951657279808\\
-0.0078125	27.2920450415899\\
-0.00732421875	27.1892396554638\\
-0.0068359375	27.0867474347944\\
-0.00634765625	26.9845662675822\\
-0.005859375	26.8826940643125\\
-0.00537109375	26.7811287576385\\
-0.0048828125	26.679868302069\\
-0.00439453125	26.5789106736619\\
-0.00390625	26.4782538697227\\
-0.00341796875	26.3778959085085\\
-0.0029296875	26.2778348289369\\
-0.00244140625	26.1780686902999\\
-0.001953125	26.0785955719823\\
-0.00146484375	25.9794135731857\\
-0.0009765625	25.8805208126564\\
-0.00048828125	25.7819154284186\\
0	25.6835955775109\\
0.00048828125	25.7819154284186\\
0.0009765625	25.8805208126564\\
0.00146484375	25.9794135731857\\
0.001953125	26.0785955719823\\
0.00244140625	26.1780686902999\\
0.0029296875	26.2778348289369\\
0.00341796875	26.3778959085085\\
0.00390625	26.4782538697227\\
0.00439453125	26.5789106736619\\
0.0048828125	26.679868302069\\
0.00537109375	26.7811287576385\\
0.005859375	26.8826940643125\\
0.00634765625	26.9845662675822\\
0.0068359375	27.0867474347944\\
0.00732421875	27.1892396554638\\
0.0078125	27.2920450415899\\
0.00830078125	27.3951657279808\\
0.0087890625	27.4986038725818\\
0.00927734375	27.6023616568098\\
0.009765625	27.7064412858944\\
0.01025390625	27.8108449892244\\
0.0107421875	27.9155750207009\\
0.01123046875	28.0206336590969\\
0.01171875	28.1260232084224\\
0.01220703125	28.2317459982977\\
0.0126953125	28.337804384332\\
0.01318359375	28.44420074851\\
0.013671875	28.5509374995847\\
0.01416015625	28.6580170734781\\
0.0146484375	28.7654419336886\\
0.01513671875	28.8732145717066\\
0.015625	28.9813375074373\\
0.01611328125	29.0898132896315\\
0.0166015625	29.1986444963244\\
0.01708984375	29.3078337352825\\
0.017578125	29.4173836444588\\
0.01806640625	29.5272968924575\\
0.0185546875	29.6375761790054\\
0.01904296875	29.7482242354343\\
0.01953125	29.8592438251713\\
0.02001953125	29.9706377442387\\
0.0205078125	30.082408821764\\
0.02099609375	30.1945599204982\\
0.021484375	30.3070939373454\\
0.02197265625	30.4200138039027\\
0.0224609375	30.5333224870087\\
0.02294921875	30.6470229893045\\
0.0234375	30.7611183498047\\
0.02392578125	30.8756116444799\\
0.0244140625	30.9905059868494\\
0.02490234375	31.1058045285876\\
0.025390625	31.2215104601396\\
0.02587890625	31.3376270113516\\
0.0263671875	31.4541574521115\\
0.02685546875	31.5711050930035\\
0.02734375	31.6884732859748\\
0.02783203125	31.8062654250165\\
0.0283203125	31.924484946857\\
0.02880859375	32.0431353316698\\
0.029296875	32.1622201037955\\
0.02978515625	32.2817428324779\\
0.0302734375	32.4017071326154\\
0.03076171875	32.5221166655271\\
0.03125	32.6429751397349\\
0.03173828125	32.76428631176\\
0.0322265625	32.8860539869382\\
0.03271484375	33.0082820202494\\
0.033203125	33.1309743171645\\
0.03369140625	33.2541348345109\\
0.0341796875	33.3777675813542\\
0.03466796875	33.5018766198987\\
0.03515625	33.6264660664065\\
0.03564453125	33.7515400921355\\
0.0361328125	33.8771029242961\\
0.03662109375	34.0031588470292\\
0.037109375	34.1297122024031\\
0.03759765625	34.2567673914315\\
0.0380859375	34.3843288751133\\
0.03857421875	34.5124011754932\\
0.0390625	34.6409888767452\\
0.03955078125	34.7700966262788\\
0.0400390625	34.8997291358671\\
0.04052734375	35.0298911828008\\
0.041015625	35.1605876110647\\
0.04150390625	35.2918233325397\\
0.0419921875	35.42360332823\\
0.04248046875	35.5559326495165\\
0.04296875	35.6888164194359\\
0.04345703125	35.8222598339889\\
0.0439453125	35.9562681634726\\
0.04443359375	36.090846753845\\
0.044921875	36.2260010281162\\
0.04541015625	36.3617364877691\\
0.0458984375	36.4980587142116\\
0.04638671875	36.6349733702586\\
0.046875	36.7724862016459\\
0.04736328125	36.9106030385761\\
0.0478515625	37.0493297972975\\
0.04833984375	37.1886724817166\\
0.048828125	37.3286371850445\\
0.04931640625	37.4692300914791\\
0.0498046875	37.6104574779218\\
0.05029296875	37.7523257157313\\
0.05078125	37.8948412725145\\
0.05126953125	38.0380107139554\\
0.0517578125	38.1818407056815\\
0.05224609375	38.326338015171\\
0.052734375	38.4715095136988\\
0.05322265625	38.617362178324\\
0.0537109375	38.7639030939189\\
0.05419921875	38.911139455239\\
0.0546875	39.0590785690384\\
0.05517578125	39.2077278562256\\
0.0556640625	39.357094854067\\
0.05615234375	39.5071872184324\\
0.056640625	39.6580127260886\\
0.05712890625	39.8095792770374\\
0.0576171875	39.9618948969026\\
0.05810546875	40.1149677393618\\
0.05859375	40.2688060886297\\
0.05908203125	40.4234183619869\\
0.0595703125	40.5788131123594\\
0.06005859375	40.7349990309478\\
0.060546875	40.8919849499052\\
0.06103515625	41.0497798450665\\
0.0615234375	41.2083928387277\\
0.06201171875	41.3678332024753\\
0.0625	41.5281103600676\\
0.06298828125	41.6892338903651\\
0.0634765625	41.851213530312\\
0.06396484375	42.0140591779688\\
0.064453125	42.1777808955938\\
0.06494140625	42.342388912773\\
0.0654296875	42.5078936296008\\
0.06591796875	42.6743056199062\\
0.06640625	42.841635634527\\
0.06689453125	43.0098946046277\\
0.0673828125	43.1790936450627\\
0.06787109375	43.3492440577797\\
0.068359375	43.5203573352636\\
0.06884765625	43.6924451640172\\
0.0693359375	43.8655194280757\\
0.06982421875	44.0395922125536\\
0.0703125	44.2146758072168\\
0.07080078125	44.3907827100788\\
0.0712890625	44.5679256310167\\
0.07177734375	44.7461174953976\\
0.072265625	44.9253714477146\\
0.07275390625	45.1057008552221\\
0.0732421875	45.2871193115651\\
0.07373046875	45.4696406403925\\
0.07421875	45.6532788989484\\
0.07470703125	45.8380483816275\\
0.0751953125	46.0239636234855\\
0.07568359375	46.2110394036927\\
0.076171875	46.3992907489158\\
0.07666015625	46.5887329366122\\
0.0771484375	46.7793814982235\\
0.07763671875	46.9712522222446\\
0.078125	47.1643611571541\\
0.07861328125	47.3587246141778\\
0.0791015625	47.5543591698646\\
0.07958984375	47.7512816684485\\
0.080078125	47.9495092239614\\
0.08056640625	48.149059222072\\
0.0810546875	48.349949321608\\
0.08154296875	48.5521974557273\\
0.08203125	48.7558218326897\\
0.08251953125	48.9608409361898\\
0.0830078125	49.1672735251886\\
0.08349609375	49.3751386331966\\
0.083984375	49.5844555669389\\
0.08447265625	49.7952439043379\\
0.0849609375	50.0075234917355\\
0.08544921875	50.2213144402747\\
0.0859375	50.4366371213481\\
0.08642578125	50.6535121610165\\
0.0869140625	50.8719604332868\\
0.08740234375	51.0920030521313\\
0.087890625	51.3136613621192\\
0.08837890625	51.5369569275151\\
0.0888671875	51.7619115196901\\
0.08935546875	51.9885471026741\\
0.08984375	52.2168858166585\\
0.09033203125	52.4469499592501\\
0.0908203125	52.6787619642435\\
0.09130859375	52.9123443776724\\
0.091796875	53.1477198308651\\
0.09228515625	53.3849110102084\\
0.0927734375	53.6239406233054\\
0.09326171875	53.8648313611569\\
0.09375	54.1076058559947\\
0.09423828125	54.3522866343347\\
0.0947265625	54.5988960647859\\
0.09521484375	54.8474563001135\\
0.095703125	55.0979892129948\\
0.09619140625	55.3505163248669\\
0.0966796875	55.6050587272047\\
0.09716796875	55.861636994502\\
0.09765625	56.1202710881717\\
0.09814453125	56.3809802505034\\
0.0986328125	56.6437828877396\\
0.09912109375	56.9086964412524\\
0.099609375	57.1757372457085\\
0.10009765625	57.4449203730155\\
0.1005859375	57.7162594607371\\
0.10107421875	57.989766523563\\
0.1015625	58.2654517462853\\
0.10205078125	58.5433232566292\\
0.1025390625	58.8233868761374\\
0.10302734375	59.1056458471864\\
0.103515625	59.3901005340556\\
0.10400390625	59.6767480958429\\
0.1044921875	59.9655821288604\\
0.10498046875	60.2565922759931\\
0.10546875	60.5497638003836\\
0.10595703125	60.8450771206388\\
0.1064453125	61.1425073046503\\
0.10693359375	61.4420235190123\\
0.107421875	61.7435884309267\\
0.10791015625	62.0471575594643\\
0.1083984375	62.352678573021\\
0.10888671875	62.6600905298974\\
0.109375	62.9693230590517\\
0.10986328125	63.2802954783071\\
0.1103515625	63.5929158476588\\
0.11083984375	63.9070799558049\\
0.111328125	64.2226702387086\\
0.11181640625	64.539554629867\\
0.1123046875	64.8575853430991\\
0.11279296875	65.1765975900709\\
0.11328125	65.4964082365368\\
0.11376953125	65.8168144034079\\
0.1142578125	66.1375920213229\\
0.11474609375	66.4584943504658\\
0.115234375	66.7792504809114\\
0.11572265625	67.0995638329654\\
0.1162109375	67.4191106816141\\
0.11669921875	67.7375387345403\\
0.1171875	68.0544657989452\\
0.11767578125	68.3694785787368\\
0.1181640625	68.6821316502948\\
0.11865234375	68.9919466717795\\
0.119140625	69.2984118876881\\
0.11962890625	69.600981996559\\
0.1201171875	69.8990784550181\\
0.12060546875	70.1920902952593\\
0.12109375	70.4793755347443\\
0.12158203125	70.7602632558878\\
0.1220703125	71.0340564288361\\
0.12255859375	71.3000355414952\\
0.123046875	71.5574630869815\\
0.12353515625	71.8055889392126\\
0.1240234375	72.0436566220469\\
0.12451171875	72.2709104464439\\
0.125	72.4866034541106\\
0.12548828125	72.6900060660543\\
0.1259765625	72.8804152923053\\
0.12646484375	73.0571643169712\\
0.126953125	73.2196322335971\\
0.12744140625	73.3672536727569\\
0.1279296875	73.499528039915\\
0.12841796875	73.6160280700234\\
0.12890625	73.7164074081735\\
0.12939453125	73.8004069444514\\
0.1298828125	73.8678596659413\\
0.13037109375	73.9186938384844\\
0.130859375	73.9529343925971\\
0.13134765625	73.9707024582023\\
0.1318359375	73.9722130667966\\
0.13232421875	73.9577711124055\\
0.1328125	73.9277657292276\\
0.13330078125	73.882663299975\\
0.1337890625	73.8229993513813\\
0.13427734375	73.7493696201965\\
0.134765625	73.6624205837011\\
0.13525390625	73.5628397441769\\
0.1357421875	73.4513459386407\\
0.13623046875	73.328679916338\\
0.13671875	73.1955953898912\\
0.13720703125	73.0528507250429\\
0.1376953125	72.9012013914191\\
0.13818359375	72.7413932554212\\
0.138671875	72.5741567581536\\
0.13916015625	72.4002019877052\\
0.1396484375	72.2202146269543\\
0.14013671875	72.0348527357361\\
0.140625	71.8447443094877\\
0.14111328125	71.6504855451667\\
0.1416015625	71.4526397384932\\
0.14208984375	71.2517367338479\\
0.142578125	71.0482728485693\\
0.14306640625	70.8427111963171\\
0.1435546875	70.6354823387872\\
0.14404296875	70.4269852009213\\
0.14453125	70.2175881912539\\
0.14501953125	70.0076304757999\\
0.1455078125	69.7974233606739\\
0.14599609375	69.5872517450667\\
0.146484375	69.3773756123332\\
0.14697265625	69.1680315324715\\
0.1474609375	68.9594341542757\\
0.14794921875	68.7517776698809\\
0.1484375	68.5452372382478\\
0.14892578125	68.3399703574553\\
0.1494140625	68.136118178458\\
0.14990234375	67.933806755328\\
0.150390625	67.7331482289192\\
0.15087890625	67.5342419424937\\
0.1513671875	67.3371754890906\\
0.15185546875	67.1420256914253\\
0.15234375	66.9488595158882\\
0.15283203125	66.7577349227685\\
0.1533203125	66.568701655262\\
0.15380859375	66.3818019701098\\
0.154296875	66.1970713128892\\
0.15478515625	66.014538941084\\
0.1552734375	65.8342284980787\\
0.15576171875	65.6561585412204\\
0.15625	65.4803430269885\\
0.15673828125	65.3067917562661\\
0.1572265625	65.1355107825462\\
0.15771484375	64.9665027858111\\
0.158203125	64.7997674146556\\
0.15869140625	64.6353015990927\\
0.1591796875	64.4730998363339\\
0.15966796875	64.3131544516855\\
0.16015625	64.1554558365588\\
0.16064453125	63.9999926654479\\
0.1611328125	63.8467520936192\\
0.16162109375	63.6957199370888\\
0.162109375	63.5468808363865\\
0.16259765625	63.4002184054572\\
0.1630859375	63.2557153669663\\
0.16357421875	63.1133536751599\\
0.1640625	62.973114627349\\
0.16455078125	62.8349789649996\\
0.1650390625	62.6989269653144\\
0.16552734375	62.564938524143\\
0.166015625	62.4329932309668\\
0.16650390625	62.3030704366541\\
0.1669921875	62.1751493146191\\
0.16748046875	62.0492089159593\\
0.16796875	61.9252282191062\\
0.16845703125	61.8031861744755\\
0.1689453125	61.6830617445539\\
0.16943359375	61.5648339398341\\
0.169921875	61.448481850964\\
0.17041015625	61.3339846774506\\
0.1708984375	61.2213217532291\\
0.17138671875	61.1104725693742\\
0.171875	61.0014167942212\\
0.17236328125	60.8941342911168\\
0.1728515625	60.7886051340381\\
0.17333984375	60.6848096212481\\
0.173828125	60.5827282871903\\
0.17431640625	60.4823419127766\\
0.1748046875	60.383631534214\\
0.17529296875	60.2865784505156\\
0.17578125	60.1911642298136\\
0.17626953125	60.09737071459\\
0.1767578125	60.0051800259308\\
0.17724609375	59.9145745668937\\
0.177734375	59.8255370250818\\
0.17822265625	59.7380503744929\\
0.1787109375	59.6520978767306\\
0.17919921875	59.5676630816238\\
0.1796875	59.4847298273362\\
0.18017578125	59.4032822399962\\
0.1806640625	59.3233047329213\\
0.18115234375	59.2447820054611\\
0.181640625	59.1676990415166\\
0.18212890625	59.092041107761\\
0.1826171875	59.0177937516038\\
0.18310546875	58.9449427989275\\
0.18359375	58.8734743516243\\
0.18408203125	58.8033747849561\\
0.1845703125	58.7346307447709\\
0.18505859375	58.6672291445828\\
0.185546875	58.6011571625457\\
0.18603515625	58.5364022383358\\
0.1865234375	58.4729520699562\\
0.18701171875	58.4107946104782\\
0.1875	58.3499180647378\\
0.18798828125	58.2903108859908\\
0.1884765625	58.2319617725436\\
0.18896484375	58.1748596643674\\
0.189453125	58.1189937397036\\
0.18994140625	58.0643534116672\\
0.1904296875	58.0109283248606\\
0.19091796875	57.958708351996\\
0.19140625	57.9076835905393\\
0.19189453125	57.8578443593753\\
0.1923828125	57.8091811955019\\
0.19287109375	57.761684850755\\
0.193359375	57.7153462885706\\
0.19384765625	57.6701566807829\\
0.1943359375	57.6261074044642\\
0.19482421875	57.5831900388079\\
0.1953125	57.5413963620574\\
0.19580078125	57.5007183484807\\
0.1962890625	57.4611481653953\\
0.19677734375	57.4226781702404\\
0.197265625	57.385300907701\\
0.19775390625	57.3490091068829\\
0.1982421875	57.3137956785406\\
0.19873046875	57.2796537123573\\
0.19921875	57.2465764742759\\
0.19970703125	57.2145574038853\\
0.2001953125	57.1835901118592\\
0.20068359375	57.1536683774466\\
0.201171875	57.124786146015\\
0.20166015625	57.0969375266496\\
0.2021484375	57.0701167897982\\
0.20263671875	57.0443183649752\\
0.203125	57.0195368385104\\
0.20361328125	56.9957669513521\\
0.2041015625	56.9730035969211\\
0.20458984375	56.9512418190118\\
0.205078125	56.9304768097425\\
0.20556640625	56.9107039075548\\
0.2060546875	56.8919185952608\\
0.20654296875	56.8741164981353\\
0.20703125	56.8572933820555\\
0.20751953125	56.8414451516844\\
0.2080078125	56.8265678486972\\
0.20849609375	56.8126576500552\\
0.208984375	56.7997108663181\\
0.20947265625	56.7877239399965\\
0.2099609375	56.7766934439511\\
0.21044921875	56.7666160798243\\
0.2109375	56.757488676514\\
0.21142578125	56.7493081886858\\
0.2119140625	56.7420716953182\\
0.21240234375	56.735776398287\\
0.212890625	56.7304196209808\\
0.21337890625	56.7259988069536\\
0.2138671875	56.7225115186052\\
0.21435546875	56.7199554358984\\
0.21484375	56.7183283551008\\
0.21533203125	56.71762818756\\
0.2158203125	56.7178529585037\\
0.21630859375	56.7190008058706\\
0.216796875	56.7210699791592\\
0.21728515625	56.7240588383115\\
0.2177734375	56.7279658526116\\
0.21826171875	56.7327895996103\\
0.21875	56.7385287640679\\
0.21923828125	56.7451821369195\\
0.2197265625	56.7527486142564\\
0.22021484375	56.7612271963239\\
0.220703125	56.7706169865355\\
0.22119140625	56.7809171905012\\
0.2216796875	56.7921271150671\\
0.22216796875	56.804246167367\\
0.22265625	56.8172738538825\\
0.22314453125	56.8312097795106\\
0.2236328125	56.8460536466402\\
0.22412109375	56.8618052542286\\
0.224609375	56.8784644968839\\
0.22509765625	56.8960313639453\\
0.2255859375	56.9145059385663\\
0.22607421875	56.9338883967901\\
0.2265625	56.9541790066225\\
0.22705078125	56.9753781270963\\
0.2275390625	56.9974862073259\\
0.22802734375	57.0205037855509\\
0.228515625	57.0444314881633\\
0.22900390625	57.0692700287186\\
0.2294921875	57.0950202069271\\
0.22998046875	57.1216829076231\\
0.23046875	57.1492590997069\\
0.23095703125	57.1777498350589\\
0.2314453125	57.2071562474222\\
0.23193359375	57.2374795512505\\
0.232421875	57.2687210405139\\
0.23291015625	57.3008820874664\\
0.2333984375	57.3339641413642\\
0.23388671875	57.3679687271331\\
0.234375	57.4028974439823\\
0.23486328125	57.4387519639588\\
0.2353515625	57.4755340304346\\
0.23583984375	57.5132454565289\\
0.236328125	57.5518881234504\\
0.23681640625	57.5914639787638\\
0.2373046875	57.6319750345679\\
0.23779296875	57.6734233655807\\
0.23828125	57.715811107126\\
0.23876953125	57.7591404530155\\
0.2392578125	57.8034136533142\\
0.23974609375	57.848633011989\\
0.240234375	57.8948008844246\\
0.24072265625	57.9419196748033\\
0.2412109375	57.9899918333398\\
0.24169921875	58.0390198533567\\
0.2421875	58.089006268196\\
0.24267578125	58.1399536479512\\
0.2431640625	58.1918645960181\\
0.24365234375	58.2447417454347\\
0.244140625	58.2985877550204\\
0.24462890625	58.3534053052807\\
0.2451171875	58.4091970940807\\
0.24560546875	58.465965832055\\
0.24609375	58.5237142377595\\
0.24658203125	58.582445032526\\
0.2470703125	58.6421609350248\\
0.24755859375	58.7028646554957\\
0.248046875	58.7645588896486\\
0.24853515625	58.8272463121949\\
0.2490234375	58.8909295700038\\
0.24951171875	58.9556112748499\\
0.25	59.0212939957344\\
0.25048828125	59.0879802507524\\
0.2509765625	59.1556724984825\\
0.25146484375	59.2243731288663\\
0.251953125	59.2940844535522\\
0.25244140625	59.3648086956728\\
0.2529296875	59.4365479790195\\
0.25341796875	59.5093043165853\\
0.25390625	59.583079598436\\
0.25439453125	59.6578755788709\\
0.2548828125	59.7336938628392\\
0.25537109375	59.8105358915618\\
0.255859375	59.8884029273203\\
0.25634765625	59.9672960373664\\
0.2568359375	60.0472160768986\\
0.25732421875	60.1281636710624\\
0.2578125	60.2101391959111\\
0.25830078125	60.293142758281\\
0.2587890625	60.3771741745174\\
0.25927734375	60.4622329479869\\
0.259765625	60.5483182453181\\
0.26025390625	60.635428871301\\
0.2607421875	60.7235632423755\\
0.26123046875	60.8127193586328\\
0.26171875	60.9028947742628\\
0.26220703125	60.9940865663583\\
0.2626953125	61.0862913020031\\
0.26318359375	61.179505003554\\
0.263671875	61.2737231120353\\
0.26416015625	61.3689404485475\\
0.2646484375	61.4651511736075\\
0.26513671875	61.5623487443193\\
0.265625	61.6605258692777\\
0.26611328125	61.7596744611132\\
0.2666015625	61.8597855865679\\
0.26708984375	61.9608494140116\\
0.267578125	62.0628551582836\\
0.26806640625	62.1657910227712\\
0.2685546875	62.2696441386159\\
0.26904296875	62.3744005009456\\
0.26953125	62.4800449020427\\
0.27001953125	62.5865608613434\\
0.2705078125	62.6939305521936\\
0.27099609375	62.8021347252597\\
0.271484375	62.911152628534\\
0.27197265625	63.0209619238572\\
0.2724609375	63.13153859991\\
0.27294921875	63.2428568816268\\
0.2734375	63.3548891360111\\
0.27392578125	63.4676057743375\\
0.2744140625	63.580975150769\\
0.27490234375	63.6949634574096\\
0.275390625	63.8095346158857\\
0.27587890625	63.9246501655361\\
0.2763671875	64.0402691483672\\
0.27685546875	64.1563479909424\\
0.27734375	64.2728403834462\\
0.27783203125	64.3896971561948\\
0.2783203125	64.5068661539448\\
0.27880859375	64.6242921083905\\
0.279296875	64.7419165093465\\
0.27978515625	64.8596774751534\\
0.2802734375	64.9775096229562\\
0.28076171875	65.0953439395874\\
0.28125	65.2131076538837\\
0.28173828125	65.3307241113729\\
0.2822265625	65.4481126523927\\
0.28271484375	65.5651884947969\\
0.283203125	65.6818626225554\\
0.28369140625	65.7980416816671\\
0.2841796875	65.9136278849461\\
0.28466796875	66.028518927375\\
0.28515625	66.1426079138498\\
0.28564453125	66.2557833012831\\
0.2861328125	66.3679288571563\\
0.28662109375	66.4789236367293\\
0.287109375	66.5886419812313\\
0.28759765625	66.6969535394521\\
0.2880859375	66.8037233152225\\
0.28857421875	66.9088117433258\\
0.2890625	67.0120747964072\\
0.28955078125	67.1133641254286\\
0.2900390625	67.2125272361586\\
0.29052734375	67.3094077040989\\
0.291015625	67.4038454300775\\
0.29150390625	67.4956769385359\\
0.2919921875	67.5847357202693\\
0.29248046875	67.6708526210228\\
0.29296875	67.7538562769459\\
0.29345703125	67.8335735974305\\
0.2939453125	67.9098302952895\\
0.29443359375	67.9824514636294\\
0.294921875	68.0512621980729\\
0.29541015625	68.1160882622431\\
0.2958984375	68.1767567936399\\
0.29638671875	68.2330970462031\\
0.296875	68.2849411650105\\
0.29736328125	68.3321249877169\\
0.2978515625	68.3744888664957\\
0.29833984375	68.4118785034741\\
0.298828125	68.4441457919044\\
0.29931640625	68.4711496546905\\
0.2998046875	68.492756871359\\
0.30029296875	68.5088428841771\\
0.30078125	68.5192925738771\\
0.30126953125	68.5240009954097\\
0.3017578125	68.5228740642496\\
0.30224609375	68.5158291841349\\
0.302734375	68.5027958076184\\
0.30322265625	68.4837159215524\\
0.3037109375	68.4585444505216\\
0.30419921875	68.4272495723253\\
0.3046875	68.3898129408252\\
0.30517578125	68.3462298128157\\
0.3056640625	68.2965090770073\\
0.30615234375	68.2406731846793\\
0.306640625	68.1787579830524\\
0.30712890625	68.1108124538873\\
0.3076171875	68.0368983612048\\
0.30810546875	67.9570898133397\\
0.30859375	67.8714727456758\\
0.30908203125	67.7801443314464\\
0.3095703125	67.6832123288022\\
0.31005859375	67.5807943729936\\
0.310546875	67.4730172229609\\
0.31103515625	67.3600159718705\\
0.3115234375	67.2419332311829\\
0.31201171875	67.1189182977163\\
0.3125	66.9911263128545\\
0.31298828125	66.8587174226168\\
0.3134765625	66.7218559467271\\
0.31396484375	66.580709564131\\
0.314453125	66.4354485216783\\
0.31494140625	66.2862448718508\\
0.3154296875	66.1332717445949\\
0.31591796875	65.9767026574616\\
0.31640625	65.8167108674067\\
0.31689453125	65.6534687668069\\
0.3173828125	65.4871473254584\\
0.31787109375	65.3179155796146\\
0.318359375	65.1459401684539\\
0.31884765625	64.9713849177725\\
0.3193359375	64.7944104701872\\
0.31982421875	64.6151739606567\\
0.3203125	64.4338287357758\\
0.32080078125	64.2505241149583\\
0.3212890625	64.0654051914008\\
0.32177734375	63.8786126705062\\
0.322265625	63.6902827433362\\
0.32275390625	63.5005469925674\\
0.3232421875	63.309532328392\\
0.32373046875	63.117360951804\\
0.32421875	62.9241503427443\\
0.32470703125	62.7300132706456\\
0.3251953125	62.5350578249932\\
0.32568359375	62.3393874636229\\
0.326171875	62.1431010765927\\
0.32666015625	61.9462930635953\\
0.3271484375	61.7490534229921\\
0.32763671875	61.5514678507048\\
0.328125	61.3536178473249\\
0.32861328125	61.1555808319345\\
0.3291015625	60.95743026127\\
0.32958984375	60.759235752986\\
0.330078125	60.561063211897\\
0.33056640625	60.3629749581929\\
0.3310546875	60.1650298567332\\
0.33154296875	59.9672834466338\\
0.33203125	59.7697880704489\\
0.33251953125	59.5725930023425\\
0.3330078125	59.3757445747296\\
0.33349609375	59.1792863029395\\
0.333984375	58.9832590075219\\
0.33447265625	58.7877009338795\\
0.3349609375	58.5926478689684\\
0.33544921875	58.3981332548573\\
0.3359375	58.2041882989795\\
0.33642578125	58.0108420809569\\
0.3369140625	57.8181216559043\\
0.33740234375	57.62605215416\\
0.337890625	57.4346568774122\\
0.33837890625	57.2439573912131\\
0.3388671875	57.0539736139021\\
0.33935546875	56.864723901961\\
0.33984375	56.6762251318575\\
0.34033203125	56.4884927784337\\
0.3408203125	56.3015409899083\\
0.34130859375	56.1153826595749\\
0.341796875	55.9300294942809\\
0.34228515625	55.7454920797806\\
0.3427734375	55.5617799430525\\
0.34326171875	55.3789016116897\\
0.34375	55.196864670452\\
0.34423828125	55.0156758150904\\
0.3447265625	54.8353409035437\\
0.34521484375	54.6558650046058\\
0.345703125	54.477252444169\\
0.34619140625	54.299506849141\\
0.3466796875	54.1226311891328\\
0.34716796875	53.9466278160155\\
0.34765625	53.7714985014381\\
0.34814453125	53.5972444723984\\
0.3486328125	53.4238664449548\\
0.34912109375	53.2513646561661\\
0.349609375	53.0797388943417\\
0.35009765625	52.9089885276818\\
0.3505859375	52.7391125313864\\
0.35107421875	52.5701095133057\\
0.3515625	52.4019777382049\\
0.35205078125	52.2347151507108\\
0.3525390625	52.0683193970059\\
0.35302734375	51.902787845335\\
0.353515625	51.73811760538\\
0.35400390625	51.5743055465665\\
0.3544921875	51.4113483153515\\
0.35498046875	51.2492423515471\\
0.35546875	51.0879839037302\\
0.35595703125	50.9275690437872\\
0.3564453125	50.7679936806336\\
0.35693359375	50.6092535731582\\
0.357421875	50.4513443424307\\
0.35791015625	50.2942614832074\\
0.3583984375	50.1380003747794\\
0.35888671875	49.9825562911932\\
0.359375	49.8279244108782\\
0.35986328125	49.6740998257153\\
0.3603515625	49.5210775495729\\
0.36083984375	49.3688525263415\\
0.361328125	49.2174196374936\\
0.36181640625	49.0667737091939\\
0.3623046875	48.9169095189855\\
0.36279296875	48.7678218020718\\
0.36328125	48.619505257221\\
0.36376953125	48.4719545523107\\
0.3642578125	48.3251643295321\\
0.36474609375	48.179129210274\\
0.365234375	48.033843799702\\
0.36572265625	47.8893026910542\\
0.3662109375	47.7455004696614\\
0.36669921875	47.6024317167155\\
0.3671875	47.4600910127931\\
0.36767578125	47.3184729411525\\
0.3681640625	47.1775720908141\\
0.36865234375	47.0373830594371\\
0.369140625	46.8979004560034\\
0.36962890625	46.7591189033211\\
0.3701171875	46.6210330403547\\
0.37060546875	46.4836375243942\\
0.37109375	46.3469270330724\\
0.37158203125	46.2108962662359\\
0.3720703125	46.0755399476811\\
0.37255859375	45.9408528267622\\
0.373046875	45.8068296798775\\
0.37353515625	45.673465311841\\
0.3740234375	45.5407545571476\\
0.37451171875	45.4086922811352\\
0.375	45.2772733810524\\
0.37548828125	45.1464927870348\\
0.3759765625	45.0163454629975\\
0.37646484375	44.8868264074462\\
0.376953125	44.7579306542138\\
0.37744140625	44.6296532731259\\
0.3779296875	44.5019893705987\\
0.37841796875	44.3749340901748\\
0.37890625	44.2484826129994\\
0.37939453125	44.122630158241\\
0.3798828125	43.9973719834594\\
0.38037109375	43.8727033849254\\
0.380859375	43.7486196978927\\
0.38134765625	43.6251162968279\\
0.3818359375	43.5021885955995\\
0.38232421875	43.3798320476278\\
0.3828125	43.2580421459995\\
0.38330078125	43.1368144235494\\
0.3837890625	43.0161444529087\\
0.38427734375	42.8960278465267\\
0.384765625	42.7764602566619\\
0.38525390625	42.6574373753494\\
0.3857421875	42.538954934343\\
0.38623046875	42.4210087050358\\
0.38671875	42.3035944983592\\
0.38720703125	42.1867081646625\\
0.3876953125	42.0703455935739\\
0.38818359375	41.9545027138457\\
0.388671875	41.8391754931823\\
0.38916015625	41.7243599380551\\
0.3896484375	41.610052093502\\
0.39013671875	41.4962480429171\\
0.390625	41.3829439078257\\
0.39111328125	41.2701358476515\\
0.3916015625	41.1578200594717\\
0.39208984375	41.0459927777651\\
0.392578125	40.93465027415\\
0.39306640625	40.823788857117\\
0.3935546875	40.7134048717518\\
0.39404296875	40.6034946994547\\
0.39453125	40.4940547576532\\
0.39501953125	40.385081499509\\
0.3955078125	40.2765714136221\\
0.39599609375	40.168521023729\\
0.396484375	40.0609268883984\\
0.39697265625	39.9537856007242\\
0.3974609375	39.8470937880144\\
0.39794921875	39.7408481114791\\
0.3984375	39.6350452659158\\
0.39892578125	39.5296819793932\\
0.3994140625	39.4247550129336\\
0.39990234375	39.3202611601947\\
0.400390625	39.2161972471509\\
0.40087890625	39.1125601317731\\
0.4013671875	39.0093467037093\\
0.40185546875	38.9065538839655\\
0.40234375	38.8041786245855\\
0.40283203125	38.7022179083328\\
0.4033203125	38.600668748372\\
0.40380859375	38.4995281879518\\
0.404296875	38.3987933000888\\
0.40478515625	38.2984611872517\\
0.4052734375	38.1985289810485\\
0.40576171875	38.0989938419138\\
0.40625	37.9998529587975\\
0.40673828125	37.9011035488567\\
0.4072265625	37.802742857148\\
0.40771484375	37.7047681563222\\
0.408203125	37.6071767463209\\
0.40869140625	37.5099659540754\\
0.4091796875	37.4131331332074\\
0.40966796875	37.3166756637323\\
0.41015625	37.2205909517644\\
0.41064453125	37.1248764292248\\
0.4111328125	37.0295295535516\\
0.41162109375	36.934547807412\\
0.412109375	36.8399286984181\\
0.41259765625	36.7456697588437\\
0.4130859375	36.6517685453453\\
0.41357421875	36.5582226386839\\
0.4140625	36.4650296434514\\
0.41455078125	36.3721871877976\\
0.4150390625	36.2796929231613\\
0.41552734375	36.1875445240041\\
0.416015625	36.0957396875452\\
0.41650390625	36.0042761335016\\
0.4169921875	35.9131516038283\\
0.41748046875	35.8223638624631\\
0.41796875	35.7319106950728\\
0.41845703125	35.6417899088038\\
0.4189453125	35.5519993320337\\
0.41943359375	35.4625368141257\\
0.419921875	35.3734002251877\\
0.42041015625	35.2845874558314\\
0.4208984375	35.1960964169362\\
0.42138671875	35.1079250394144\\
0.421875	35.02007127398\\
0.42236328125	34.9325330909197\\
0.4228515625	34.8453084798665\\
0.42333984375	34.7583954495761\\
0.423828125	34.6717920277064\\
0.42431640625	34.5854962605977\\
0.4248046875	34.4995062130588\\
0.42529296875	34.4138199681519\\
0.42578125	34.3284356269834\\
0.42626953125	34.2433513084944\\
0.4267578125	34.1585651492558\\
0.42724609375	34.0740753032649\\
0.427734375	33.9898799417448\\
0.42822265625	33.9059772529456\\
0.4287109375	33.8223654419492\\
0.42919921875	33.7390427304756\\
0.4296875	33.6560073566913\\
0.43017578125	33.5732575750212\\
0.4306640625	33.4907916559626\\
0.43115234375	33.4086078858994\\
0.431640625	33.3267045669223\\
0.43212890625	33.2450800166479\\
0.4326171875	33.1637325680422\\
0.43310546875	33.0826605692448\\
0.43359375	33.0018623833967\\
0.43408203125	32.9213363884691\\
0.4345703125	32.8410809770951\\
0.43505859375	32.7610945564029\\
0.435546875	32.6813755478519\\
0.43603515625	32.6019223870699\\
0.4365234375	32.5227335236934\\
0.43701171875	32.443807421209\\
0.4375	32.365142556797\\
0.43798828125	32.2867374211777\\
0.4384765625	32.2085905184588\\
0.43896484375	32.130700365985\\
0.439453125	32.0530654941895\\
0.43994140625	31.9756844464478\\
0.4404296875	31.8985557789327\\
0.44091796875	31.8216780604713\\
0.44140625	31.745049872404\\
0.44189453125	31.6686698084456\\
0.4423828125	31.5925364745474\\
0.44287109375	31.5166484887615\\
0.443359375	31.441004481107\\
0.44384765625	31.3656030934372\\
0.4443359375	31.2904429793097\\
0.44482421875	31.2155228038564\\
0.4453125	31.1408412436572\\
0.44580078125	31.0663969866136\\
0.4462890625	30.9921887318249\\
0.44677734375	30.9182151894651\\
0.447265625	30.8444750806626\\
0.44775390625	30.7709671373802\\
0.4482421875	30.6976901022976\\
0.44873046875	30.6246427286942\\
0.44921875	30.5518237803347\\
0.44970703125	30.4792320313558\\
0.4501953125	30.4068662661532\\
0.45068359375	30.3347252792716\\
0.451171875	30.2628078752951\\
0.45166015625	30.1911128687396\\
0.4521484375	30.1196390839457\\
0.45263671875	30.0483853549741\\
0.453125	29.9773505255011\\
0.45361328125	29.9065334487167\\
0.4541015625	29.8359329872226\\
0.45458984375	29.7655480129325\\
0.455078125	29.6953774069734\\
0.45556640625	29.625420059588\\
0.4560546875	29.5556748700382\\
0.45654296875	29.4861407465102\\
0.45703125	29.4168166060203\\
0.45751953125	29.3477013743227\\
0.4580078125	29.2787939858172\\
0.45849609375	29.2100933834591\\
0.458984375	29.1415985186697\\
0.45947265625	29.0733083512485\\
0.4599609375	29.0052218492854\\
0.46044921875	28.9373379890751\\
0.4609375	28.8696557550318\\
0.46142578125	28.8021741396058\\
0.4619140625	28.7348921431996\\
0.46240234375	28.6678087740871\\
0.462890625	28.600923048332\\
0.46337890625	28.534233989708\\
0.4638671875	28.4677406296204\\
0.46435546875	28.4014420070272\\
0.46484375	28.3353371683631\\
0.46533203125	28.2694251674627\\
0.4658203125	28.2037050654859\\
0.46630859375	28.1381759308434\\
0.466796875	28.0728368391233\\
0.46728515625	28.007686873019\\
0.4677734375	27.9427251222576\\
0.46826171875	27.8779506835287\\
0.46875	27.8133626604155\\
0.46923828125	27.7489601633249\\
0.4697265625	27.6847423094201\\
0.47021484375	27.6207082225526\\
0.470703125	27.5568570331964\\
0.47119140625	27.4931878783817\\
0.4716796875	27.4296999016302\\
0.47216796875	27.3663922528909\\
0.47265625	27.3032640884769\\
0.47314453125	27.2403145710024\\
0.4736328125	27.1775428693212\\
0.47412109375	27.1149481584652\\
0.474609375	27.0525296195842\\
0.47509765625	26.9902864398861\\
0.4755859375	26.9282178125781\\
0.47607421875	26.8663229368081\\
0.4765625	26.8046010176069\\
0.47705078125	26.7430512658322\\
0.4775390625	26.6816728981112\\
0.47802734375	26.6204651367856\\
0.478515625	26.5594272098564\\
0.47900390625	26.49855835093\\
0.4794921875	26.4378577991641\\
0.47998046875	26.3773247992144\\
0.48046875	26.316958601183\\
0.48095703125	26.2567584605659\\
0.4814453125	26.1967236382018\\
0.48193359375	26.1368534002219\\
0.482421875	26.0771470179993\\
0.48291015625	26.0176037680999\\
0.4833984375	25.9582229322334\\
0.48388671875	25.8990037972049\\
0.484375	25.8399456548672\\
0.48486328125	25.7810478020733\\
0.4853515625	25.7223095406303\\
0.48583984375	25.6637301772524\\
0.486328125	25.605309023516\\
0.48681640625	25.5470453958142\\
0.4873046875	25.4889386153121\\
0.48779296875	25.4309880079032\\
0.48828125	25.3731929041652\\
0.48876953125	25.3155526393171\\
0.4892578125	25.2580665531768\\
0.48974609375	25.2007339901187\\
0.490234375	25.1435542990319\\
0.49072265625	25.0865268332797\\
0.4912109375	25.0296509506579\\
0.49169921875	24.9729260133551\\
0.4921875	24.9163513879127\\
0.49267578125	24.8599264451857\\
0.4931640625	24.8036505603032\\
0.49365234375	24.7475231126307\\
0.494140625	24.6915434857313\\
0.49462890625	24.6357110673283\\
0.4951171875	24.5800252492678\\
0.49560546875	24.5244854274823\\
0.49609375	24.4690910019536\\
0.49658203125	24.4138413766774\\
0.4970703125	24.3587359596269\\
0.49755859375	24.3037741627184\\
0.498046875	24.2489554017757\\
0.49853515625	24.1942790964961\\
0.4990234375	24.1397446704161\\
0.49951171875	24.0853515508773\\
0.5	24.031099168994\\
0.50048828125	23.9769869596188\\
0.5009765625	23.9230143613113\\
0.50146484375	23.8691808163049\\
0.501953125	23.8154857704749\\
0.50244140625	23.7619286733076\\
0.5029296875	23.7085089778681\\
0.50341796875	23.6552261407699\\
0.50390625	23.6020796221441\\
0.50439453125	23.5490688856093\\
0.5048828125	23.4961933982413\\
0.50537109375	23.4434526305438\\
0.505859375	23.3908460564189\\
0.50634765625	23.338373153138\\
0.5068359375	23.2860334013134\\
0.50732421875	23.2338262848694\\
0.5078125	23.1817512910147\\
0.50830078125	23.129807910214\\
0.5087890625	23.0779956361614\\
0.50927734375	23.0263139657521\\
0.509765625	22.9747623990562\\
0.51025390625	22.9233404392915\\
0.5107421875	22.8720475927977\\
0.51123046875	22.8208833690099\\
0.51171875	22.7698472804324\\
0.51220703125	22.7189388426144\\
0.5126953125	22.6681575741229\\
0.51318359375	22.6175029965191\\
0.513671875	22.566974634333\\
0.51416015625	22.516572015039\\
0.5146484375	22.4662946690315\\
0.51513671875	22.4161421296008\\
0.515625	22.3661139329099\\
0.51611328125	22.3162096179697\\
0.5166015625	22.2664287266174\\
0.51708984375	22.2167708034918\\
0.517578125	22.1672353960116\\
0.51806640625	22.1178220543521\\
0.5185546875	22.0685303314236\\
0.51904296875	22.0193597828481\\
0.51953125	21.9703099669384\\
0.52001953125	21.9213804446762\\
0.5205078125	21.87257077969\\
0.52099609375	21.8238805382347\\
0.521484375	21.7753092891699\\
0.52197265625	21.7268566039394\\
0.5224609375	21.6785220565502\\
0.52294921875	21.6303052235527\\
0.5234375	21.5822056840199\\
0.52392578125	21.5342230195274\\
0.5244140625	21.4863568141339\\
0.52490234375	21.4386066543614\\
0.525390625	21.3909721291757\\
0.52587890625	21.3434528299674\\
0.5263671875	21.2960483505323\\
0.52685546875	21.2487582870533\\
0.52734375	21.201582238081\\
0.52783203125	21.1545198045155\\
0.5283203125	21.1075705895883\\
0.52880859375	21.0607341988439\\
0.529296875	21.0140102401217\\
0.52978515625	20.9673983235385\\
0.5302734375	20.9208980614708\\
0.53076171875	20.8745090685374\\
0.53125	20.8282309615817\\
0.53173828125	20.7820633596553\\
0.5322265625	20.7360058840006\\
0.53271484375	20.690058158034\\
0.533203125	20.6442198073294\\
0.53369140625	20.598490459602\\
0.5341796875	20.5528697446913\\
0.53466796875	20.5073572945456\\
0.53515625	20.4619527432058\\
0.53564453125	20.4166557267894\\
0.5361328125	20.3714658834752\\
0.53662109375	20.3263828534872\\
0.537109375	20.2814062790795\\
0.53759765625	20.2365358045215\\
0.5380859375	20.1917710760818\\
0.53857421875	20.1471117420141\\
0.5390625	20.102557452542\\
0.53955078125	20.0581078598443\\
0.5400390625	20.0137626180408\\
0.54052734375	19.9695213831773\\
0.541015625	19.9253838132119\\
0.54150390625	19.8813495680004\\
0.5419921875	19.8374183092826\\
0.54248046875	19.7935897006685\\
0.54296875	19.7498634076241\\
0.54345703125	19.7062390974583\\
0.5439453125	19.6627164393088\\
0.54443359375	19.6192951041294\\
0.544921875	19.5759747646765\\
0.54541015625	19.5327550954955\\
0.5458984375	19.4896357729087\\
0.54638671875	19.4466164750014\\
0.546875	19.4036968816101\\
0.54736328125	19.3608766743091\\
0.5478515625	19.3181555363985\\
0.54833984375	19.2755331528912\\
0.548828125	19.2330092105012\\
0.54931640625	19.1905833976311\\
0.5498046875	19.14825540436\\
0.55029296875	19.1060249224314\\
0.55078125	19.0638916452417\\
0.55126953125	19.0218552678282\\
0.5517578125	18.9799154868575\\
0.55224609375	18.9380720006138\\
0.552734375	18.8963245089876\\
0.55322265625	18.8546727134644\\
0.5537109375	18.8131163171135\\
0.55419921875	18.7716550245764\\
0.5546875	18.7302885420564\\
0.55517578125	18.6890165773073\\
0.5556640625	18.6478388396224\\
0.55615234375	18.6067550398243\\
0.556640625	18.5657648902536\\
0.55712890625	18.5248681047588\\
0.5576171875	18.4840643986856\\
0.55810546875	18.4433534888665\\
0.55859375	18.4027350936108\\
0.55908203125	18.362208932694\\
0.5595703125	18.3217747273482\\
0.56005859375	18.2814322002513\\
0.560546875	18.2411810755181\\
0.56103515625	18.2010210786895\\
0.5615234375	18.1609519367234\\
0.56201171875	18.1209733779846\\
0.5625	18.0810851322355\\
0.56298828125	18.0412869306265\\
0.5634765625	18.0015785056864\\
0.56396484375	17.9619595913135\\
0.564453125	17.9224299227658\\
0.56494140625	17.8829892366523\\
0.5654296875	17.8436372709234\\
0.56591796875	17.8043737648624\\
0.56640625	17.7651984590763\\
0.56689453125	17.7261110954868\\
0.5673828125	17.6871114173216\\
0.56787109375	17.648199169106\\
0.568359375	17.6093740966538\\
0.56884765625	17.5706359470588\\
0.5693359375	17.5319844686866\\
0.56982421875	17.4934194111659\\
0.5703125	17.4549405253801\\
0.57080078125	17.4165475634596\\
0.5712890625	17.3782402787725\\
0.57177734375	17.3400184259176\\
0.572265625	17.3018817607155\\
0.57275390625	17.2638300402011\\
0.5732421875	17.2258630226151\\
0.57373046875	17.1879804673971\\
0.57421875	17.1501821351765\\
0.57470703125	17.1124677877659\\
0.5751953125	17.0748371881528\\
0.57568359375	17.037290100492\\
0.576171875	16.9998262900983\\
0.57666015625	16.9624455234391\\
0.5771484375	16.9251475681262\\
0.57763671875	16.8879321929096\\
0.578125	16.8507991676692\\
0.57861328125	16.8137482634081\\
0.5791015625	16.7767792522452\\
0.57958984375	16.7398919074083\\
0.580078125	16.7030860032265\\
0.58056640625	16.6663613151239\\
0.5810546875	16.6297176196122\\
0.58154296875	16.5931546942838\\
0.58203125	16.5566723178052\\
0.58251953125	16.5202702699099\\
0.5830078125	16.483948331392\\
0.58349609375	16.4477062840994\\
0.583984375	16.4115439109271\\
0.58447265625	16.3754609958106\\
0.5849609375	16.3394573237196\\
0.58544921875	16.3035326806513\\
0.5859375	16.2676868536242\\
0.58642578125	16.2319196306715\\
0.5869140625	16.1962308008352\\
0.58740234375	16.1606201541594\\
0.587890625	16.1250874816842\\
0.58837890625	16.0896325754398\\
0.5888671875	16.05425522844\\
0.58935546875	16.0189552346766\\
0.58984375	15.9837323891129\\
0.59033203125	15.9485864876782\\
0.5908203125	15.9135173272615\\
0.59130859375	15.8785247057059\\
0.591796875	15.8436084218026\\
0.59228515625	15.8087682752852\\
0.5927734375	15.7740040668239\\
0.59326171875	15.7393155980201\\
0.59375	15.7047026714003\\
0.59423828125	15.6701650904109\\
0.5947265625	15.6357026594124\\
0.59521484375	15.6013151836739\\
0.595703125	15.567002469368\\
0.59619140625	15.5327643235649\\
0.5966796875	15.4986005542271\\
0.59716796875	15.4645109702042\\
0.59765625	15.4304953812279\\
0.59814453125	15.396553597906\\
0.5986328125	15.3626854317178\\
0.59912109375	15.3288906950087\\
0.599609375	15.295169200985\\
0.60009765625	15.261520763709\\
0.6005859375	15.2279451980937\\
0.60107421875	15.1944423198982\\
0.6015625	15.161011945722\\
0.60205078125	15.1276538930007\\
0.6025390625	15.0943679800011\\
0.60302734375	15.0611540258158\\
0.603515625	15.0280118503589\\
0.60400390625	14.9949412743607\\
0.6044921875	14.9619421193636\\
0.60498046875	14.9290142077169\\
0.60546875	14.8961573625722\\
0.60595703125	14.8633714078787\\
0.6064453125	14.8306561683788\\
0.60693359375	14.7980114696034\\
0.607421875	14.7654371378674\\
0.60791015625	14.7329330002649\\
0.6083984375	14.7004988846653\\
0.60888671875	14.6681346197082\\
0.609375	14.6358400347995\\
0.60986328125	14.6036149601068\\
0.6103515625	14.5714592265552\\
0.61083984375	14.5393726658227\\
0.611328125	14.5073551103363\\
0.61181640625	14.4754063932673\\
0.6123046875	14.4435263485275\\
0.61279296875	14.4117148107649\\
0.61328125	14.3799716153594\\
0.61376953125	14.3482965984189\\
0.6142578125	14.3166895967749\\
0.61474609375	14.2851504479789\\
0.615234375	14.2536789902978\\
0.61572265625	14.2222750627105\\
0.6162109375	14.1909385049032\\
0.61669921875	14.1596691572664\\
0.6171875	14.12846686089\\
0.61767578125	14.0973314575602\\
0.6181640625	14.066262789755\\
0.61865234375	14.0352607006411\\
0.619140625	14.0043250340693\\
0.61962890625	13.9734556345712\\
0.6201171875	13.9426523473556\\
0.62060546875	13.9119150183042\\
0.62109375	13.8812434939686\\
0.62158203125	13.8506376215659\\
0.6220703125	13.8200972489758\\
0.62255859375	13.7896222247363\\
0.623046875	13.7592123980409\\
0.62353515625	13.728867618734\\
0.6240234375	13.6985877373084\\
0.62451171875	13.6683726049011\\
0.625	13.6382220732903\\
0.62548828125	13.6081359948913\\
0.6259765625	13.5781142227538\\
0.62646484375	13.5481566105579\\
0.626953125	13.5182630126112\\
0.62744140625	13.4884332838448\\
0.6279296875	13.4586672798109\\
0.62841796875	13.4289648566784\\
0.62890625	13.3993258712305\\
0.62939453125	13.3697501808608\\
0.6298828125	13.3402376435706\\
0.63037109375	13.3107881179652\\
0.630859375	13.281401463251\\
0.63134765625	13.2520775392322\\
0.6318359375	13.2228162063076\\
0.63232421875	13.1936173254676\\
0.6328125	13.1644807582911\\
0.63330078125	13.1354063669421\\
0.6337890625	13.1063940141672\\
0.63427734375	13.0774435632919\\
0.634765625	13.0485548782181\\
0.63525390625	13.0197278234207\\
0.6357421875	12.9909622639449\\
0.63623046875	12.9622580654032\\
0.63671875	12.9336150939724\\
0.63720703125	12.9050332163905\\
0.6376953125	12.8765122999541\\
0.63818359375	12.8480522125153\\
0.638671875	12.8196528224791\\
0.63916015625	12.7913139988002\\
0.6396484375	12.7630356109805\\
0.64013671875	12.7348175290659\\
0.640625	12.7066596236441\\
0.64111328125	12.6785617658413\\
0.6416015625	12.6505238273197\\
0.64208984375	12.6225456802747\\
0.642578125	12.5946271974322\\
0.64306640625	12.566768252046\\
0.6435546875	12.538968717895\\
0.64404296875	12.5112284692806\\
0.64453125	12.483547381024\\
0.64501953125	12.4559253284638\\
0.6455078125	12.4283621874531\\
0.64599609375	12.400857834357\\
0.646484375	12.3734121460502\\
0.64697265625	12.3460249999143\\
0.6474609375	12.3186962738352\\
0.64794921875	12.291425846201\\
0.6484375	12.2642135958988\\
0.64892578125	12.2370594023128\\
0.6494140625	12.2099631453215\\
0.64990234375	12.1829247052956\\
0.650390625	12.1559439630952\\
0.65087890625	12.1290208000677\\
0.6513671875	12.1021550980449\\
0.65185546875	12.0753467393412\\
0.65234375	12.0485956067511\\
0.65283203125	12.0219015835465\\
0.6533203125	11.9952645534745\\
0.65380859375	11.9686844007555\\
0.654296875	11.9421610100803\\
0.65478515625	11.9156942666081\\
0.6552734375	11.8892840559644\\
0.65576171875	11.8629302642381\\
0.65625	11.8366327779801\\
0.65673828125	11.8103914842005\\
0.6572265625	11.7842062703663\\
0.65771484375	11.7580770243998\\
0.658203125	11.7320036346758\\
0.65869140625	11.7059859900195\\
0.6591796875	11.680023979705\\
0.65966796875	11.6541174934521\\
0.66015625	11.628266421425\\
0.66064453125	11.6024706542298\\
0.6611328125	11.5767300829125\\
0.66162109375	11.551044598957\\
0.662109375	11.5254140942827\\
0.66259765625	11.4998384612428\\
0.6630859375	11.474317592622\\
0.66357421875	11.4488513816345\\
0.6640625	11.4234397219223\\
0.66455078125	11.3980825075526\\
0.6650390625	11.3727796330163\\
0.66552734375	11.3475309932256\\
0.666015625	11.3223364835125\\
0.66650390625	11.2971959996265\\
0.6669921875	11.2721094377326\\
0.66748046875	11.2470766944095\\
0.66796875	11.222097666648\\
0.66845703125	11.1971722518483\\
0.6689453125	11.1723003478187\\
0.66943359375	11.1474818527738\\
0.669921875	11.122716665332\\
0.67041015625	11.0980046845142\\
0.6708984375	11.0733458097417\\
0.67138671875	11.0487399408346\\
0.671875	11.0241869780095\\
0.67236328125	10.999686821878\\
0.6728515625	10.9752393734451\\
0.67333984375	10.9508445341069\\
0.673828125	10.9265022056491\\
0.67431640625	10.9022122902454\\
0.6748046875	10.8779746904551\\
0.67529296875	10.8537893092222\\
0.67578125	10.829656049873\\
0.67626953125	10.8055748161146\\
0.6767578125	10.7815455120333\\
0.67724609375	10.7575680420925\\
0.677734375	10.7336423111316\\
0.67822265625	10.7097682243637\\
0.6787109375	10.6859456873742\\
0.67919921875	10.6621746061194\\
0.6796875	10.6384548869241\\
0.68017578125	10.6147864364809\\
0.6806640625	10.5911691618476\\
0.68115234375	10.5676029704464\\
0.681640625	10.5440877700618\\
0.68212890625	10.5206234688389\\
0.6826171875	10.4972099752824\\
0.68310546875	10.4738471982543\\
0.68359375	10.4505350469729\\
0.68408203125	10.4272734310105\\
0.6845703125	10.4040622602929\\
0.68505859375	10.3809014450967\\
0.685546875	10.3577908960488\\
0.68603515625	10.334730524124\\
0.6865234375	10.3117202406439\\
0.68701171875	10.2887599572755\\
0.6875	10.2658495860295\\
0.68798828125	10.2429890392585\\
0.6884765625	10.2201782296563\\
0.68896484375	10.1974170702557\\
0.689453125	10.1747054744271\\
0.68994140625	10.1520433558776\\
0.6904296875	10.1294306286489\\
0.69091796875	10.1068672071163\\
0.69140625	10.0843530059869\\
0.69189453125	10.0618879402984\\
0.6923828125	10.0394719254178\\
0.69287109375	10.0171048770395\\
0.693359375	9.99478671118443\\
0.69384765625	9.97251734419843\\
0.6943359375	9.9502966927508\\
0.69482421875	9.92812467383304\\
0.6953125	9.90600120475739\\
0.69580078125	9.88392620315552\\
0.6962890625	9.8618995869771\\
0.69677734375	9.83992127448863\\
0.697265625	9.81799118427189\\
0.69775390625	9.79610923522268\\
0.6982421875	9.77427534654957\\
0.69873046875	9.75248943777247\\
0.69921875	9.73075142872143\\
0.69970703125	9.7090612395352\\
0.7001953125	9.68741879066013\\
0.70068359375	9.66582400284865\\
0.701171875	9.64427679715817\\
0.70166015625	9.62277709494975\\
0.7021484375	9.6013248178868\\
0.70263671875	9.57991988793383\\
0.703125	9.55856222735533\\
0.70361328125	9.53725175871423\\
0.7041015625	9.5159884048711\\
0.70458984375	9.49477208898239\\
0.705078125	9.47360273449967\\
0.70556640625	9.45248026516818\\
0.7060546875	9.43140460502562\\
0.70654296875	9.41037567840111\\
0.70703125	9.3893934099137\\
0.70751953125	9.36845772447158\\
0.7080078125	9.34756854727045\\
0.70849609375	9.32672580379275\\
0.708984375	9.30592941980618\\
0.70947265625	9.28517932136278\\
0.7099609375	9.26447543479758\\
0.71044921875	9.24381768672754\\
0.7109375	9.22320600405041\\
0.71142578125	9.2026403139436\\
0.7119140625	9.18212054386297\\
0.71240234375	9.16164662154181\\
0.712890625	9.14121847498963\\
0.71337890625	9.12083603249115\\
0.7138671875	9.10049922260509\\
0.71435546875	9.08020797416307\\
0.71484375	9.05996221626864\\
0.71533203125	9.03976187829605\\
0.7158203125	9.01960688988923\\
0.71630859375	8.99949718096072\\
0.716796875	8.97943268169065\\
0.71728515625	8.95941332252544\\
0.7177734375	8.93943903417709\\
0.71826171875	8.91950974762181\\
0.71875	8.89962539409925\\
0.71923828125	8.87978590511119\\
0.7197265625	8.85999121242067\\
0.72021484375	8.84024124805092\\
0.720703125	8.82053594428435\\
0.72119140625	8.80087523366153\\
0.7216796875	8.78125904898003\\
0.72216796875	8.76168732329366\\
0.72265625	8.74215998991138\\
0.72314453125	8.72267698239613\\
0.7236328125	8.70323823456402\\
0.72412109375	8.68384368048337\\
0.724609375	8.66449325447351\\
0.72509765625	8.64518689110406\\
0.7255859375	8.62592452519377\\
0.72607421875	8.60670609180963\\
0.7265625	8.58753152626598\\
0.72705078125	8.56840076412328\\
0.7275390625	8.54931374118756\\
0.72802734375	8.53027039350915\\
0.728515625	8.51127065738181\\
0.72900390625	8.49231446934195\\
0.7294921875	8.47340176616747\\
0.72998046875	8.454532484877\\
0.73046875	8.43570656272891\\
0.73095703125	8.41692393722036\\
0.7314453125	8.39818454608644\\
0.73193359375	8.37948832729928\\
0.732421875	8.36083521906708\\
0.73291015625	8.34222515983321\\
0.7333984375	8.32365808827543\\
0.73388671875	8.30513394330486\\
0.734375	8.28665266406517\\
0.73486328125	8.26821418993166\\
0.7353515625	8.24981846051043\\
0.73583984375	8.23146541563743\\
0.736328125	8.21315499537773\\
0.73681640625	8.1948871400245\\
0.7373046875	8.17666179009825\\
0.73779296875	8.15847888634594\\
0.73828125	8.1403383697401\\
0.73876953125	8.12224018147812\\
0.7392578125	8.10418426298119\\
0.73974609375	8.08617055589369\\
0.740234375	8.06819900208213\\
0.74072265625	8.05026954363458\\
0.7412109375	8.03238212285954\\
0.74169921875	8.01453668228544\\
0.7421875	7.99673316465959\\
0.74267578125	7.97897151294742\\
0.7431640625	7.96125167033177\\
0.74365234375	7.94357358021193\\
0.744140625	7.92593718620303\\
0.74462890625	7.90834243213495\\
0.7451171875	7.89078926205194\\
0.74560546875	7.87327762021141\\
0.74609375	7.85580745108351\\
0.74658203125	7.83837869934999\\
0.7470703125	7.82099130990378\\
0.74755859375	7.80364522784795\\
0.748046875	7.78634039849502\\
0.74853515625	7.76907676736634\\
0.7490234375	7.75185428019104\\
0.74951171875	7.73467288290555\\
0.75	7.71753252165267\\
0.75048828125	7.70043314278089\\
0.7509765625	7.68337469284366\\
0.75146484375	7.66635711859856\\
0.751953125	7.64938036700663\\
0.75244140625	7.6324443852317\\
0.7529296875	7.61554912063944\\
0.75341796875	7.59869452079687\\
0.75390625	7.58188053347154\\
0.75439453125	7.56510710663073\\
0.7548828125	7.54837418844084\\
0.75537109375	7.53168172726661\\
0.755859375	7.51502967167051\\
0.75634765625	7.49841797041188\\
0.7568359375	7.48184657244629\\
0.75732421875	7.46531542692494\\
0.7578125	7.4488244831938\\
0.75830078125	7.43237369079297\\
0.7587890625	7.41596299945606\\
0.75927734375	7.39959235910942\\
0.759765625	7.3832617198715\\
0.76025390625	7.36697103205207\\
0.7607421875	7.35072024615178\\
0.76123046875	7.33450931286115\\
0.76171875	7.31833818306018\\
0.76220703125	7.30220680781751\\
0.7626953125	7.28611513838993\\
0.76318359375	7.27006312622142\\
0.763671875	7.25405072294282\\
0.76416015625	7.23807788037101\\
0.7646484375	7.22214455050825\\
0.76513671875	7.2062506855415\\
0.765625	7.19039623784195\\
0.76611328125	7.17458115996414\\
0.7666015625	7.15880540464549\\
0.76708984375	7.14306892480554\\
0.767578125	7.1273716735455\\
0.76806640625	7.11171360414733\\
0.7685546875	7.09609467007342\\
0.76904296875	7.08051482496571\\
0.76953125	7.06497402264525\\
0.77001953125	7.04947221711143\\
0.7705078125	7.0340093625415\\
0.77099609375	7.01858541328985\\
0.771484375	7.00320032388744\\
0.77197265625	6.98785404904118\\
0.7724609375	6.97254654363337\\
0.77294921875	6.95727776272099\\
0.7734375	6.94204766153523\\
0.77392578125	6.92685619548072\\
0.7744140625	6.9117033201352\\
0.77490234375	6.89658899124862\\
0.775390625	6.88151316474277\\
0.77587890625	6.86647579671057\\
0.7763671875	6.85147684341558\\
0.77685546875	6.83651626129133\\
0.77734375	6.8215940069408\\
0.77783203125	6.8067100371358\\
0.7783203125	6.79186430881645\\
0.77880859375	6.77705677909055\\
0.779296875	6.76228740523305\\
0.77978515625	6.74755614468541\\
0.7802734375	6.73286295505519\\
0.78076171875	6.71820779411533\\
0.78125	6.70359061980363\\
0.78173828125	6.68901139022228\\
0.7822265625	6.67447006363721\\
0.78271484375	6.65996659847755\\
0.783203125	6.64550095333511\\
0.78369140625	6.63107308696383\\
0.7841796875	6.61668295827921\\
0.78466796875	6.60233052635779\\
0.78515625	6.58801575043663\\
0.78564453125	6.57373858991271\\
0.7861328125	6.55949900434245\\
0.78662109375	6.54529695344117\\
0.787109375	6.53113239708247\\
0.78759765625	6.5170052952979\\
0.7880859375	6.50291560827624\\
0.78857421875	6.48886329636305\\
0.7890625	6.4748483200602\\
0.78955078125	6.46087064002524\\
0.7900390625	6.44693021707096\\
0.79052734375	6.43302701216488\\
0.791015625	6.4191609864287\\
0.79150390625	6.40533210113783\\
0.7919921875	6.3915403177208\\
0.79248046875	6.37778559775892\\
0.79296875	6.36406790298556\\
0.79345703125	6.3503871952858\\
0.7939453125	6.33674343669593\\
0.79443359375	6.32313658940283\\
0.794921875	6.30956661574368\\
0.79541015625	6.29603347820522\\
0.7958984375	6.28253713942348\\
0.79638671875	6.2690775621831\\
0.796875	6.25565470941709\\
0.79736328125	6.24226854420603\\
0.7978515625	6.22891902977786\\
0.79833984375	6.21560612950732\\
0.798828125	6.20232980691534\\
0.79931640625	6.18909002566873\\
0.7998046875	6.17588674957971\\
0.80029296875	6.16271994260527\\
0.80078125	6.14958956884688\\
0.80126953125	6.13649559254989\\
0.8017578125	6.1234379781032\\
0.80224609375	6.11041669003869\\
0.802734375	6.09743169303071\\
0.80322265625	6.08448295189583\\
0.8037109375	6.07157043159217\\
0.80419921875	6.05869409721906\\
0.8046875	6.04585391401648\\
0.80517578125	6.03304984736481\\
0.8056640625	6.02028186278411\\
0.80615234375	6.00754992593392\\
0.806640625	5.99485400261263\\
0.80712890625	5.98219405875717\\
0.8076171875	5.96957006044245\\
0.80810546875	5.95698197388103\\
0.80859375	5.9444297654226\\
0.80908203125	5.93191340155358\\
0.8095703125	5.91943284889667\\
0.81005859375	5.90698807421043\\
0.810546875	5.89457904438883\\
0.81103515625	5.88220572646088\\
0.8115234375	5.8698680875901\\
0.81201171875	5.85756609507415\\
0.8125	5.84529971634447\\
0.81298828125	5.83306891896573\\
0.8134765625	5.82087367063546\\
0.81396484375	5.80871393918372\\
0.814453125	5.79658969257255\\
0.81494140625	5.78450089889561\\
0.8154296875	5.7724475263778\\
0.81591796875	5.76042954337476\\
0.81640625	5.74844691837263\\
0.81689453125	5.73649961998735\\
0.8173828125	5.72458761696461\\
0.81787109375	5.71271087817918\\
0.818359375	5.70086937263456\\
0.81884765625	5.68906306946269\\
0.8193359375	5.67729193792342\\
0.81982421875	5.66555594740418\\
0.8203125	5.65385506741957\\
0.82080078125	5.64218926761094\\
0.8212890625	5.63055851774606\\
0.82177734375	5.61896278771864\\
0.822265625	5.60740204754802\\
0.82275390625	5.59587626737874\\
0.8232421875	5.58438541748017\\
0.82373046875	5.57292946824608\\
0.82421875	5.56150839019435\\
0.82470703125	5.5501221539665\\
0.8251953125	5.53877073032732\\
0.82568359375	5.52745409016459\\
0.826171875	5.51617220448851\\
0.82666015625	5.50492504443153\\
0.8271484375	5.4937125812479\\
0.82763671875	5.48253478631318\\
0.828125	5.47139163112409\\
0.82861328125	5.46028308729793\\
0.8291015625	5.44920912657238\\
0.82958984375	5.43816972080502\\
0.830078125	5.42716484197299\\
0.83056640625	5.41619446217269\\
0.8310546875	5.40525855361934\\
0.83154296875	5.39435708864665\\
0.83203125	5.38349003970645\\
0.83251953125	5.3726573793684\\
0.8330078125	5.36185908031951\\
0.83349609375	5.35109511536391\\
0.833984375	5.34036545742236\\
0.83447265625	5.32967007953207\\
0.8349609375	5.31900895484629\\
0.83544921875	5.30838205663376\\
0.8359375	5.2977893582787\\
0.83642578125	5.28723083328026\\
0.8369140625	5.27670645525214\\
0.83740234375	5.26621619792247\\
0.837890625	5.25576003513318\\
0.83837890625	5.24533794083987\\
0.8388671875	5.23494988911143\\
0.83935546875	5.22459585412959\\
0.83984375	5.2142758101888\\
0.84033203125	5.20398973169561\\
0.8408203125	5.19373759316865\\
0.84130859375	5.18351936923802\\
0.841796875	5.17333503464522\\
0.84228515625	5.16318456424256\\
0.8427734375	5.153067932993\\
0.84326171875	5.1429851159698\\
0.84375	5.13293608835616\\
0.84423828125	5.12292082544499\\
0.8447265625	5.11293930263841\\
0.84521484375	5.10299149544759\\
0.845703125	5.09307737949233\\
0.84619140625	5.08319693050087\\
0.8466796875	5.07335012430944\\
0.84716796875	5.06353693686198\\
0.84765625	5.05375734420989\\
0.84814453125	5.04401132251163\\
0.8486328125	5.0342988480325\\
0.84912109375	5.02461989714424\\
0.849609375	5.01497444632479\\
0.85009765625	5.00536247215793\\
0.8505859375	4.99578395133301\\
0.85107421875	4.98623886064463\\
0.8515625	4.97672717699239\\
0.85205078125	4.96724887738047\\
0.8525390625	4.95780393891746\\
0.85302734375	4.94839233881598\\
0.853515625	4.93901405439239\\
0.85400390625	4.92966906306653\\
0.8544921875	4.92035734236138\\
0.85498046875	4.91107886990283\\
0.85546875	4.90183362341929\\
0.85595703125	4.89262158074149\\
0.8564453125	4.88344271980215\\
0.85693359375	4.87429701863568\\
0.857421875	4.86518445537792\\
0.85791015625	4.85610500826585\\
0.8583984375	4.84705865563726\\
0.85888671875	4.83804537593053\\
0.859375	4.82906514768431\\
0.85986328125	4.82011794953725\\
0.8603515625	4.81120376022772\\
0.86083984375	4.80232255859351\\
0.861328125	4.79347432357156\\
0.86181640625	4.78465903419776\\
0.8623046875	4.7758766696065\\
0.86279296875	4.7671272090306\\
0.86328125	4.75841063180088\\
0.86376953125	4.74972691734597\\
0.8642578125	4.74107604519199\\
0.86474609375	4.73245799496236\\
0.865234375	4.72387274637745\\
0.86572265625	4.71532027925432\\
0.8662109375	4.70680057350647\\
0.86669921875	4.69831360914367\\
0.8671875	4.68985936627147\\
0.86767578125	4.68143782509122\\
0.8681640625	4.6730489658995\\
0.86865234375	4.66469276908818\\
0.869140625	4.65636921514389\\
0.86962890625	4.64807828464796\\
0.8701171875	4.63981995827604\\
0.87060546875	4.63159421679784\\
0.87109375	4.62340104107703\\
0.87158203125	4.61524041207074\\
0.8720703125	4.60711231082958\\
0.87255859375	4.59901671849717\\
0.873046875	4.59095361630995\\
0.87353515625	4.58292298559706\\
0.8740234375	4.57492480777989\\
0.87451171875	4.56695906437194\\
0.875	4.55902573697866\\
0.87548828125	4.551124807297\\
0.8759765625	4.54325625711534\\
0.87646484375	4.53542006831315\\
0.876953125	4.52761622286081\\
0.87744140625	4.51984470281935\\
0.8779296875	4.51210549034022\\
0.87841796875	4.50439856766499\\
0.87890625	4.49672391712518\\
0.87939453125	4.48908152114206\\
0.8798828125	4.48147136222631\\
0.88037109375	4.47389342297786\\
0.880859375	4.46634768608566\\
0.88134765625	4.45883413432735\\
0.8818359375	4.45135275056922\\
0.88232421875	4.44390351776581\\
0.8828125	4.43648641895971\\
0.88330078125	4.42910143728147\\
0.8837890625	4.42174855594915\\
0.88427734375	4.41442775826828\\
0.884765625	4.40713902763161\\
0.88525390625	4.39988234751875\\
0.8857421875	4.39265770149614\\
0.88623046875	4.38546507321671\\
0.88671875	4.37830444641966\\
0.88720703125	4.37117580493032\\
0.8876953125	4.36407913265986\\
0.88818359375	4.35701441360509\\
0.888671875	4.34998163184834\\
0.88916015625	4.34298077155699\\
0.8896484375	4.3360118169836\\
0.89013671875	4.32907475246545\\
0.890625	4.32216956242439\\
0.89111328125	4.31529623136668\\
0.8916015625	4.30845474388278\\
0.89208984375	4.30164508464703\\
0.892578125	4.29486723841758\\
0.89306640625	4.28812119003615\\
0.8935546875	4.28140692442773\\
0.89404296875	4.27472442660051\\
0.89453125	4.26807368164562\\
0.89501953125	4.26145467473691\\
0.8955078125	4.25486739113077\\
0.89599609375	4.24831181616591\\
0.896484375	4.24178793526327\\
0.89697265625	4.2352957339256\\
0.8974609375	4.2288351977375\\
0.89794921875	4.22240631236511\\
0.8984375	4.21600906355587\\
0.89892578125	4.20964343713844\\
0.8994140625	4.20330941902245\\
0.89990234375	4.19700699519829\\
0.900390625	4.19073615173699\\
0.90087890625	4.18449687478989\\
0.9013671875	4.17828915058865\\
0.90185546875	4.17211296544492\\
0.90234375	4.1659683057502\\
0.90283203125	4.15985515797563\\
0.9033203125	4.15377350867178\\
0.90380859375	4.14772334446864\\
0.904296875	4.14170465207521\\
0.90478515625	4.13571741827944\\
0.9052734375	4.12976162994804\\
0.90576171875	4.12383727402628\\
0.90625	4.11794433753781\\
0.90673828125	4.11208280758455\\
0.9072265625	4.10625267134639\\
0.90771484375	4.10045391608112\\
0.908203125	4.09468652912424\\
0.90869140625	4.08895049788869\\
0.9091796875	4.08324580986486\\
0.90966796875	4.07757245262021\\
0.91015625	4.07193041379931\\
0.91064453125	4.06631968112345\\
0.9111328125	4.0607402423907\\
0.91162109375	4.05519208547556\\
0.912109375	4.04967519832887\\
0.91259765625	4.04418956897766\\
0.9130859375	4.03873518552497\\
0.91357421875	4.03331203614963\\
0.9140625	4.0279201091062\\
0.91455078125	4.02255939272473\\
0.9150390625	4.01722987541068\\
0.91552734375	4.01193154564462\\
0.916015625	4.00666439198223\\
0.91650390625	4.00142840305403\\
0.9169921875	3.99622356756529\\
0.91748046875	3.99104987429586\\
0.91796875	3.98590731209999\\
0.91845703125	3.98079586990618\\
0.9189453125	3.97571553671703\\
0.91943359375	3.97066630160916\\
0.919921875	3.96564815373294\\
0.92041015625	3.96066108231241\\
0.9208984375	3.95570507664513\\
0.92138671875	3.950780126102\\
0.921875	3.94588622012717\\
0.92236328125	3.94102334823782\\
0.9228515625	3.93619150002407\\
0.92333984375	3.93139066514882\\
0.923828125	3.92662083334763\\
0.92431640625	3.92188199442846\\
0.9248046875	3.91717413827174\\
0.92529296875	3.91249725483007\\
0.92578125	3.9078513341281\\
0.92626953125	3.90323636626241\\
0.9267578125	3.89865234140145\\
0.92724609375	3.89409924978525\\
0.927734375	3.88957708172537\\
0.92822265625	3.88508582760481\\
0.9287109375	3.88062547787782\\
0.92919921875	3.87619602306972\\
0.9296875	3.87179745377686\\
0.93017578125	3.86742976066651\\
0.9306640625	3.86309293447657\\
0.93115234375	3.85878696601556\\
0.931640625	3.85451184616256\\
0.93212890625	3.8502675658669\\
0.9326171875	3.84605411614819\\
0.93310546875	3.84187148809611\\
0.93359375	3.83771967287033\\
0.93408203125	3.83359866170032\\
0.9345703125	3.82950844588537\\
0.93505859375	3.82544901679429\\
0.935546875	3.8214203658654\\
0.93603515625	3.81742248460641\\
0.9365234375	3.81345536459422\\
0.93701171875	3.80951899747494\\
0.9375	3.80561337496361\\
0.93798828125	3.80173848884423\\
0.9384765625	3.79789433096947\\
0.93896484375	3.79408089326086\\
0.939453125	3.79029816770831\\
0.93994140625	3.78654614637021\\
0.9404296875	3.78282482137332\\
0.94091796875	3.77913418491259\\
0.94140625	3.77547422925111\\
0.94189453125	3.77184494671992\\
0.9423828125	3.76824632971799\\
0.94287109375	3.76467837071205\\
0.943359375	3.76114106223655\\
0.94384765625	3.75763439689348\\
0.9443359375	3.75415836735231\\
0.94482421875	3.75071296634988\\
0.9453125	3.74729818669033\\
0.94580078125	3.74391402124485\\
0.9462890625	3.74056046295186\\
0.94677734375	3.73723750481665\\
0.947265625	3.73394513991131\\
0.94775390625	3.73068336137489\\
0.9482421875	3.7274521624129\\
0.94873046875	3.72425153629758\\
0.94921875	3.72108147636757\\
0.94970703125	3.71794197602795\\
0.9501953125	3.71483302875006\\
0.95068359375	3.71175462807141\\
0.951171875	3.70870676759568\\
0.95166015625	3.70568944099259\\
0.9521484375	3.70270264199771\\
0.95263671875	3.6997463644125\\
0.953125	3.69682060210417\\
};
\addplot [color=mycolor4,solid]
  table[row sep=crcr]{0.953125	3.69682060210417\\
0.95361328125	3.69392534900561\\
0.9541015625	3.69106059911526\\
0.95458984375	3.68822634649707\\
0.955078125	3.68542258528042\\
0.95556640625	3.68264930965999\\
0.9560546875	3.67990651389576\\
0.95654296875	3.67719419231282\\
0.95703125	3.67451233930135\\
0.95751953125	3.67186094931661\\
0.9580078125	3.66924001687867\\
0.95849609375	3.66664953657254\\
0.958984375	3.66408950304797\\
0.95947265625	3.66155991101944\\
0.9599609375	3.65906075526598\\
0.96044921875	3.65659203063123\\
0.9609375	3.65415373202329\\
0.96142578125	3.65174585441467\\
0.9619140625	3.64936839284217\\
0.96240234375	3.64702134240691\\
0.962890625	3.64470469827416\\
0.96337890625	3.64241845567334\\
0.9638671875	3.6401626098979\\
0.96435546875	3.63793715630527\\
0.96484375	3.63574209031685\\
0.96533203125	3.6335774074178\\
0.9658203125	3.63144310315719\\
0.96630859375	3.62933917314777\\
0.966796875	3.62726561306591\\
0.96728515625	3.62522241865164\\
0.9677734375	3.62320958570852\\
0.96826171875	3.6212271101036\\
0.96875	3.6192749877673\\
0.96923828125	3.6173532146935\\
0.9697265625	3.61546178693931\\
0.97021484375	3.61360070062517\\
0.970703125	3.61176995193462\\
0.97119140625	3.60996953711442\\
0.9716796875	3.60819945247439\\
0.97216796875	3.60645969438741\\
0.97265625	3.6047502592893\\
0.97314453125	3.60307114367886\\
0.9736328125	3.60142234411776\\
0.97412109375	3.59980385723048\\
0.974609375	3.59821567970436\\
0.97509765625	3.59665780828939\\
0.9755859375	3.59513023979827\\
0.97607421875	3.59363297110645\\
0.9765625	3.59216599915185\\
0.97705078125	3.59072932093502\\
0.9775390625	3.58932293351899\\
0.97802734375	3.5879468340293\\
0.978515625	3.58660101965391\\
0.97900390625	3.58528548764319\\
0.9794921875	3.58400023530982\\
0.97998046875	3.5827452600288\\
0.98046875	3.58152055923747\\
0.98095703125	3.58032613043535\\
0.9814453125	3.5791619711842\\
0.98193359375	3.57802807910791\\
0.982421875	3.57692445189255\\
0.98291015625	3.57585108728626\\
0.9833984375	3.57480798309926\\
0.98388671875	3.57379513720384\\
0.984375	3.57281254753425\\
0.98486328125	3.57186021208675\\
0.9853515625	3.57093812891956\\
0.98583984375	3.57004629615276\\
0.986328125	3.56918471196841\\
0.98681640625	3.56835337461039\\
0.9873046875	3.56755228238446\\
0.98779296875	3.56678143365812\\
0.98828125	3.56604082686078\\
0.98876953125	3.56533046048353\\
0.9892578125	3.56465033307927\\
0.98974609375	3.56400044326261\\
0.990234375	3.56338078970985\\
0.99072265625	3.56279137115905\\
0.9912109375	3.56223218640985\\
0.99169921875	3.56170323432367\\
0.9921875	3.56120451382341\\
0.99267578125	3.56073602389376\\
0.9931640625	3.56029776358094\\
0.99365234375	3.55988973199274\\
0.994140625	3.55951192829858\\
0.99462890625	3.55916435172949\\
0.9951171875	3.55884700157797\\
0.99560546875	3.5585598771981\\
0.99609375	3.55830297800558\\
0.99658203125	3.55807630347754\\
0.9970703125	3.5578798531527\\
0.99755859375	3.55771362663128\\
0.998046875	3.557577623575\\
0.99853515625	3.55747184370711\\
0.9990234375	3.55739628681237\\
0.99951171875	3.55735095273702\\
};
\addlegendentry{AR(14) Model};

\end{axis}
\end{tikzpicture}%}
	\caption{\textit{PSD of Signal, and Estimated Models for varying orders - with 10,000 samples}}
	\label{fig:2_2_c}
\end{figure}




\end{document}

