\documentclass[./main.tex]{subfiles} 
\begin{document}

\subsection{Periodogram-based Methods Applied to Real-World Data}

\subsubsection{The Sunspot Time Series}

Presented with the Sunspot Time Series data, there are a number of preprocessing techniques we can apply to it before plotting the Periodogram. These are plotted in Figure \ref{fig:1_4_a}. We observe that for the original data there is a high power at 0 frequency. This is the influence of the mean signal, so for other signals we mean centre the data. We can see that the detrended version (the black line) appears to suffer from less noise or jitter in the frequency domain. Taking the logarithm of data and mean centering it (the pink like) showed very interesting results. The line at points appears to jitter even less, for example at 0.35, the other lines suffer from a signficant drop, but it moves little by comparison.

\begin{figure}[h]
	\centering
	\resizebox{\textwidth}{!}{% This file was created by matlab2tikz v0.4.7 (commit cce90cce73bbb51499829a32f6a3d8f221bfad3c) running on MATLAB 8.3.
% Copyright (c) 2008--2014, Nico Schlömer <nico.schloemer@gmail.com>
% All rights reserved.
% Minimal pgfplots version: 1.3
% 
% The latest updates can be retrieved from
%   http://www.mathworks.com/matlabcentral/fileexchange/22022-matlab2tikz
% where you can also make suggestions and rate matlab2tikz.
% 
%
% defining custom colors
\definecolor{mycolor1}{rgb}{0.00000,0.00000,0.17241}%
\definecolor{mycolor2}{rgb}{1.00000,0.10345,0.72414}%
%
\begin{tikzpicture}

\begin{axis}[%
width=7in,
height=2.8in,
scale only axis,
xmin=0,
xmax=1,
xlabel={Normalised Frequency},
xmajorgrids,
ymin=-10,
ymax=170,
ylabel={Power (dB)},
ymajorgrids,
title={Sunspot Periodogram, with various preprocessing methods applied},
axis x line*=bottom,
axis y line*=left,
legend style={draw=black,fill=white,legend cell align=left}
]
\addplot [color=blue,solid]
  table[row sep=crcr]{-1	54.6195193956365\\
-0.998263888888889	51.144526621686\\
-0.996527777777778	59.8297479466518\\
-0.994791666666667	72.8820400576052\\
-0.993055555555556	80.4487008793162\\
-0.991319444444444	84.0651881628394\\
-0.989583333333333	84.6393923866235\\
-0.987847222222222	83.2348856622312\\
-0.986111111111111	82.6038602661371\\
-0.984375	84.7781282285205\\
-0.982638888888889	86.9893445962065\\
-0.980902777777778	86.876303964656\\
-0.979166666666667	84.1721990616269\\
-0.977430555555556	81.0916769126164\\
-0.975694444444444	81.3625790897174\\
-0.973958333333333	82.2869989963824\\
-0.972222222222222	79.9795086362025\\
-0.970486111111111	73.8983598925008\\
-0.96875	71.4966795294092\\
-0.967013888888889	75.037776785967\\
-0.965277777777778	74.1749220189929\\
-0.963541666666667	69.3958935440523\\
-0.961805555555556	76.6917713765674\\
-0.960069444444444	85.2348320656686\\
-0.958333333333333	88.6185544132282\\
-0.956597222222222	87.4121259058447\\
-0.954861111111111	81.3032518702256\\
-0.953125	68.4966509966492\\
-0.951388888888889	63.3641660616117\\
-0.949652777777778	73.8374029244498\\
-0.947916666666667	80.1023495616297\\
-0.946180555555556	84.4864238116157\\
-0.944444444444444	87.1723965731343\\
-0.942708333333333	88.1105416228263\\
-0.940972222222222	88.2758264687342\\
-0.939236111111111	88.9697798695176\\
-0.9375	89.7960973664361\\
-0.935763888888889	89.2257017648301\\
-0.934027777777778	86.3088738778435\\
-0.932291666666667	80.7344396644502\\
-0.930555555555556	71.9069837416938\\
-0.928819444444444	52.4786826204944\\
-0.927083333333333	59.6820326345394\\
-0.925347222222222	77.018464000994\\
-0.923611111111111	84.032527781582\\
-0.921875	85.9895709874381\\
-0.920138888888889	83.9207439235528\\
-0.918402777777778	78.1126976343461\\
-0.916666666666667	69.6591336000207\\
-0.914930555555556	64.5046876164492\\
-0.913194444444444	69.3772821029138\\
-0.911458333333333	77.0538328437846\\
-0.909722222222222	81.5624357645856\\
-0.907986111111111	81.3004444809841\\
-0.90625	74.1857743759757\\
-0.904513888888889	58.395410967342\\
-0.902777777777778	74.3192016708202\\
-0.901041666666667	81.5259807541695\\
-0.899305555555556	80.8710214935648\\
-0.897569444444444	75.3896335553656\\
-0.895833333333333	79.6172558473633\\
-0.894097222222222	87.6821231362653\\
-0.892361111111111	90.7240826691747\\
-0.890625	88.5910075093948\\
-0.888888888888889	79.2854605020709\\
-0.887152777777778	43.3718052938499\\
-0.885416666666667	76.3274996402945\\
-0.883680555555556	84.6920920097817\\
-0.881944444444444	85.4678717473144\\
-0.880208333333333	81.0111131430399\\
-0.878472222222222	72.5951671235057\\
-0.876736111111111	72.477500891049\\
-0.875	78.4417794500417\\
-0.873263888888889	80.5218781456106\\
-0.871527777777778	79.1776546271778\\
-0.869791666666667	77.6463408917373\\
-0.868055555555556	81.3628378376814\\
-0.866319444444444	87.1896896314133\\
-0.864583333333333	90.959430655473\\
-0.862847222222222	91.6871331617218\\
-0.861111111111111	88.6459153612585\\
-0.859375	80.3828921935322\\
-0.857638888888889	74.3587405363664\\
-0.855902777777778	84.1053302889223\\
-0.854166666666667	88.9413657765602\\
-0.852430555555556	87.9734566466834\\
-0.850694444444444	81.238292412675\\
-0.848958333333333	73.9771524029758\\
-0.847222222222222	79.8103701028832\\
-0.845486111111111	81.6663434386633\\
-0.84375	73.5031530922949\\
-0.842013888888889	57.0190913290982\\
-0.840277777777778	81.649780361361\\
-0.838541666666667	88.7419572404219\\
-0.836805555555556	88.3479085289368\\
-0.835069444444444	79.8680484697962\\
-0.833333333333333	25.3220511449291\\
-0.831597222222222	77.2221935520859\\
-0.829861111111111	85.3454528588257\\
-0.828125	85.3413253457908\\
-0.826388888888889	79.3936054475115\\
-0.824652777777778	65.8595642466184\\
-0.822916666666667	28.8796167529766\\
-0.821180555555556	43.0698614495742\\
-0.819444444444444	49.6467709074348\\
-0.817708333333333	51.2779264658017\\
-0.815972222222222	38.8774692090198\\
-0.814236111111111	66.7725167865278\\
-0.8125	80.7748756363307\\
-0.810763888888889	87.2052065414333\\
-0.809027777777778	88.6149707766699\\
-0.807291666666667	85.8745521579945\\
-0.805555555555556	82.2725191439689\\
-0.803819444444444	85.082551332735\\
-0.802083333333333	88.9741755204538\\
-0.800347222222222	89.2852869866607\\
-0.798611111111111	85.6589110460017\\
-0.796875	78.6703594398817\\
-0.795138888888889	70.4107179562757\\
-0.793402777777778	59.652153630672\\
-0.791666666666667	58.786272147152\\
-0.789930555555556	75.4249497483181\\
-0.788194444444444	82.3602724069002\\
-0.786458333333333	83.0686698242257\\
-0.784722222222222	78.3901906508395\\
-0.782986111111111	72.5700497751857\\
-0.78125	77.1537843369589\\
-0.779513888888889	80.9433538465549\\
-0.777777777777778	79.444756952014\\
-0.776041666666667	73.0845523710305\\
-0.774305555555556	68.1980767570481\\
-0.772569444444444	71.0360442666042\\
-0.770833333333333	71.4893778803248\\
-0.769097222222222	69.0649461423797\\
-0.767361111111111	69.9958269927106\\
-0.765625	73.278822954909\\
-0.763888888888889	74.0048918574933\\
-0.762152777777778	71.566622630781\\
-0.760416666666667	66.4371130637972\\
-0.758680555555556	60.6110219727775\\
-0.756944444444444	65.1632211954121\\
-0.755208333333333	73.7210263764705\\
-0.753472222222222	78.269863565838\\
-0.751736111111111	78.8685906229852\\
-0.75	76.9907608624821\\
-0.748263888888889	76.2768065069699\\
-0.746527777777778	77.4296877034478\\
-0.744791666666667	75.8800197392796\\
-0.743055555555556	66.9157086596394\\
-0.741319444444444	50.272590424658\\
-0.739583333333333	71.8224115456894\\
-0.737847222222222	79.7318321117764\\
-0.736111111111111	82.1031197626554\\
-0.734375	81.4243628512774\\
-0.732638888888889	77.8440433601119\\
-0.730902777777778	68.3380790102177\\
-0.729166666666667	59.8873431828867\\
-0.727430555555556	76.6529780265972\\
-0.725694444444444	82.9633761631573\\
-0.723958333333333	81.4007223867741\\
-0.722222222222222	71.1473284809027\\
-0.720486111111111	80.9847896489561\\
-0.71875	92.9916270294461\\
-0.717013888888889	98.0080890919222\\
-0.715277777777778	98.1621828251794\\
-0.713541666666667	93.5590422050874\\
-0.711805555555556	83.0396107046742\\
-0.710069444444444	76.8261726227873\\
-0.708333333333333	83.9660951884025\\
-0.706597222222222	84.5931436606274\\
-0.704861111111111	77.2558656866618\\
-0.703125	64.437345323224\\
-0.701388888888889	76.2746686342221\\
-0.699652777777778	81.6136223610145\\
-0.697916666666667	79.5233378093388\\
-0.696180555555556	70.0342537095206\\
-0.694444444444444	70.1389462227247\\
-0.692708333333333	82.3066246982492\\
-0.690972222222222	88.6341828535189\\
-0.689236111111111	91.025256643183\\
-0.6875	90.1501487019417\\
-0.685763888888889	85.5707266152413\\
-0.684027777777778	76.0095952685352\\
-0.682291666666667	69.6600217985477\\
-0.680555555555556	77.9302463447317\\
-0.678819444444444	81.8258709509328\\
-0.677083333333333	80.8638842138629\\
-0.675347222222222	75.4848892748275\\
-0.673611111111111	63.9088075123886\\
-0.671875	28.8575638804298\\
-0.670138888888889	56.205408353405\\
-0.668402777777778	63.5758449783295\\
-0.666666666666667	61.7172947530873\\
-0.664930555555556	56.63093365545\\
-0.663194444444444	66.474176540921\\
-0.661458333333333	73.9964887106229\\
-0.659722222222222	75.0904120424339\\
-0.657986111111111	68.8958622801982\\
-0.65625	42.1970240677726\\
-0.654513888888889	61.6020945352215\\
-0.652777777777778	74.3454684983274\\
-0.651041666666667	81.7934560173046\\
-0.649305555555556	88.4072678139561\\
-0.647569444444444	93.0867199813768\\
-0.645833333333333	94.6013674837996\\
-0.644097222222222	92.0494179358055\\
-0.642361111111111	83.9529508520603\\
-0.640625	76.980318903129\\
-0.638888888888889	86.3109211515586\\
-0.637152777777778	90.4203826928251\\
-0.635416666666667	87.3661570014495\\
-0.633680555555556	71.0507489120065\\
-0.631944444444444	72.6391319067472\\
-0.630208333333333	90.0428931888965\\
-0.628472222222222	95.2714675430681\\
-0.626736111111111	94.8961397767871\\
-0.625	89.3612819015866\\
-0.623263888888889	75.7928869409664\\
-0.621527777777778	61.1478813226059\\
-0.619791666666667	76.6303455874833\\
-0.618055555555556	80.7602125107911\\
-0.616319444444444	82.1682722849374\\
-0.614583333333333	84.5181027361009\\
-0.612847222222222	85.7621511569838\\
-0.611111111111111	82.7012777464209\\
-0.609375	70.0965563639926\\
-0.607638888888889	60.7181517281374\\
-0.605902777777778	81.0156027016814\\
-0.604166666666667	86.2458577444683\\
-0.602430555555556	84.1107250847411\\
-0.600694444444444	71.7073369399955\\
-0.598958333333333	56.0756936242229\\
-0.597222222222222	80.0736070096643\\
-0.595486111111111	85.2800212656777\\
-0.59375	83.1843843892033\\
-0.592013888888889	72.7487396208621\\
-0.590277777777778	60.3732603643948\\
-0.588541666666667	78.3393642049967\\
-0.586805555555556	86.3199075378826\\
-0.585069444444444	90.5204227727799\\
-0.583333333333333	92.8251629700207\\
-0.581597222222222	93.1851485013559\\
-0.579861111111111	90.7877982202677\\
-0.578125	83.9958513724674\\
-0.576388888888889	66.2271069622743\\
-0.574652777777778	62.1297606043843\\
-0.572916666666667	79.6850492358072\\
-0.571180555555556	84.7437489370974\\
-0.569444444444444	85.1678468317865\\
-0.567708333333333	82.3824165766594\\
-0.565972222222222	76.4647810313981\\
-0.564236111111111	67.0663974096717\\
-0.5625	62.3680107550452\\
-0.560763888888889	69.1452321232044\\
-0.559027777777778	72.3125187767182\\
-0.557291666666667	69.4728349802509\\
-0.555555555555556	69.1269420805085\\
-0.553819444444444	82.792920463223\\
-0.552083333333333	92.4536011444222\\
-0.550347222222222	97.5972298869029\\
-0.548611111111111	99.1715774533344\\
-0.546875	97.5353387537371\\
-0.545138888888889	93.0369491107423\\
-0.543402777777778	87.2652886261158\\
-0.541666666666667	82.3521017768617\\
-0.539930555555556	72.8382971643286\\
-0.538194444444444	58.6849623400191\\
-0.536458333333333	85.4424568591732\\
-0.534722222222222	94.3603817299315\\
-0.532986111111111	96.6529576780961\\
-0.53125	93.5667471445441\\
-0.529513888888889	83.8370527909283\\
-0.527777777777778	72.1358224776339\\
-0.526041666666667	79.3741224780858\\
-0.524305555555556	77.1854130661819\\
-0.522569444444444	54.25493217849\\
-0.520833333333333	84.2470854793621\\
-0.519097222222222	95.1518822897195\\
-0.517361111111111	98.7582697044466\\
-0.515625	97.4107068430287\\
-0.513888888888889	90.5278990017208\\
-0.512152777777778	70.417112880658\\
-0.510416666666667	74.3598148374428\\
-0.508680555555556	90.2541654935757\\
-0.506944444444444	96.1101948633418\\
-0.505208333333333	97.7809807799684\\
-0.503472222222222	96.051507087067\\
-0.501736111111111	90.5467730064451\\
-0.5	80.6389089062811\\
-0.498263888888889	69.8530116664942\\
-0.496527777777778	59.8371391042878\\
-0.494791666666667	60.597633874652\\
-0.493055555555556	84.2977909750284\\
-0.491319444444444	94.1657571676553\\
-0.489583333333333	98.3224016891213\\
-0.487847222222222	98.5408704732461\\
-0.486111111111111	95.3818676044848\\
-0.484375	88.7603228080435\\
-0.482638888888889	78.2793696550577\\
-0.480902777777778	78.0138012255435\\
-0.479166666666667	88.8037640460727\\
-0.477430555555556	94.8380652504796\\
-0.475694444444444	96.4857585405377\\
-0.473958333333333	94.4078799448891\\
-0.472222222222222	90.6751329761304\\
-0.470486111111111	90.7517884531823\\
-0.46875	92.7834263697159\\
-0.467013888888889	92.2402201964761\\
-0.465277777777778	91.8321231319534\\
-0.463541666666667	96.6226468600294\\
-0.461805555555556	101.292767013213\\
-0.460069444444444	102.321255636124\\
-0.458333333333333	98.8515783729325\\
-0.456597222222222	89.5076519121094\\
-0.454861111111111	80.7665805654189\\
-0.453125	89.2338868894895\\
-0.451388888888889	92.9241722587592\\
-0.449652777777778	91.9884732434817\\
-0.447916666666667	89.5166033230581\\
-0.446180555555556	87.1483868360922\\
-0.444444444444444	80.734599907371\\
-0.442708333333333	61.3992697662226\\
-0.440972222222222	85.3547908620072\\
-0.439236111111111	96.3553070781673\\
-0.4375	100.420225525965\\
-0.435763888888889	99.8939438065896\\
-0.434027777777778	95.1076753187338\\
-0.432291666666667	87.6994035592087\\
-0.430555555555556	88.9743728064136\\
-0.428819444444444	94.4063862545348\\
-0.427083333333333	96.0578661203076\\
-0.425347222222222	93.4184545800262\\
-0.423611111111111	85.2510153847794\\
-0.421875	75.386861504833\\
-0.420138888888889	89.0967724811288\\
-0.418402777777778	98.6571143815199\\
-0.416666666666667	103.978597735007\\
-0.414930555555556	106.6422955468\\
-0.413194444444444	107.102195215363\\
-0.411458333333333	105.216168153856\\
-0.409722222222222	100.098342457065\\
-0.407986111111111	88.292341183783\\
-0.40625	54.4829042694759\\
-0.404513888888889	92.2734892390083\\
-0.402777777777778	101.624607007434\\
-0.401041666666667	105.044787378864\\
-0.399305555555556	104.263895054894\\
-0.397569444444444	98.2363842313519\\
-0.395833333333333	80.4980847011452\\
-0.394097222222222	90.0121498930992\\
-0.392361111111111	102.709484485761\\
-0.390625	106.728225111965\\
-0.388888888888889	105.038057633433\\
-0.387152777777778	95.9817053429302\\
-0.385416666666667	83.2231460796612\\
-0.383680555555556	100.733646279531\\
-0.381944444444444	107.97808478597\\
-0.380208333333333	109.064504628748\\
-0.378472222222222	105.210096285839\\
-0.376736111111111	96.2925969168961\\
-0.375	88.9493323476262\\
-0.373263888888889	90.0238441533013\\
-0.371527777777778	84.2734604638482\\
-0.369791666666667	93.4292121816113\\
-0.368055555555556	107.404308478756\\
-0.366319444444444	114.34519591754\\
-0.364583333333333	116.541791889564\\
-0.362847222222222	114.594364408852\\
-0.361111111111111	107.831257617117\\
-0.359375	92.8782309545543\\
-0.357638888888889	37.8022088588558\\
-0.355902777777778	72.2242821626779\\
-0.354166666666667	76.7038611328867\\
-0.352430555555556	100.658842700999\\
-0.350694444444444	110.94928474756\\
-0.348958333333333	115.294424728468\\
-0.347222222222222	115.525059355661\\
-0.345486111111111	112.341841486357\\
-0.34375	106.958483442039\\
-0.342013888888889	102.954087809727\\
-0.340277777777778	101.19823078807\\
-0.338541666666667	97.5222888359219\\
-0.336805555555556	94.3117524914624\\
-0.335069444444444	99.3952730581164\\
-0.333333333333333	103.947533978059\\
-0.331597222222222	104.341603895901\\
-0.329861111111111	100.642165194677\\
-0.328125	92.8960821816983\\
-0.326388888888889	81.0726192480121\\
-0.324652777777778	69.5819298580414\\
-0.322916666666667	81.8002783177442\\
-0.321180555555556	91.9191216294575\\
-0.319444444444444	96.3277794695746\\
-0.317708333333333	96.3688390560169\\
-0.315972222222222	93.5740453928295\\
-0.314236111111111	91.6310679963797\\
-0.3125	91.8058074544992\\
-0.310763888888889	89.5665530286911\\
-0.309027777777778	81.1547546923029\\
-0.307291666666667	68.9000729612865\\
-0.305555555555556	76.6723092450581\\
-0.303819444444444	79.2159720806553\\
-0.302083333333333	82.3321559993889\\
-0.300347222222222	91.0552172370125\\
-0.298611111111111	96.8904500694297\\
-0.296875	99.0509470577395\\
-0.295138888888889	99.3917052086975\\
-0.293402777777778	100.021089331975\\
-0.291666666666667	100.409033981041\\
-0.289930555555556	97.9532869577505\\
-0.288194444444444	89.4380152368322\\
-0.286458333333333	74.5733249666231\\
-0.284722222222222	91.3249110686645\\
-0.282986111111111	101.075553217167\\
-0.28125	106.657202493134\\
-0.279513888888889	109.830642047972\\
-0.277777777777778	110.046007433063\\
-0.276041666666667	105.331416155358\\
-0.274305555555556	87.0744266075841\\
-0.272569444444444	94.8403050853752\\
-0.270833333333333	110.44757670424\\
-0.269097222222222	115.516276300265\\
-0.267361111111111	114.947023674955\\
-0.265625	108.486273337691\\
-0.263888888888889	91.4659559680069\\
-0.262152777777778	96.9465839729238\\
-0.260416666666667	107.959401089071\\
-0.258680555555556	111.359413520431\\
-0.256944444444444	111.352982267791\\
-0.255208333333333	109.678648401691\\
-0.253472222222222	107.600761618908\\
-0.251736111111111	108.630686714102\\
-0.25	114.031835400912\\
-0.248263888888889	118.797577483639\\
-0.246527777777778	120.583566944193\\
-0.244791666666667	118.950099142421\\
-0.243055555555556	114.77977938203\\
-0.241319444444444	114.627658791451\\
-0.239583333333333	120.092298320967\\
-0.237847222222222	123.83438947662\\
-0.236111111111111	124.345229772143\\
-0.234375	121.754329047145\\
-0.232638888888889	116.526393957831\\
-0.230902777777778	111.35544531332\\
-0.229166666666667	110.990113595303\\
-0.227430555555556	112.35359111871\\
-0.225694444444444	111.653713178768\\
-0.223958333333333	107.958427689307\\
-0.222222222222222	99.776960630358\\
-0.220486111111111	78.415816785375\\
-0.21875	92.8753884777097\\
-0.217013888888889	107.727372915877\\
-0.215277777777778	115.901659053053\\
-0.213541666666667	121.24437123012\\
-0.211805555555556	124.302369740734\\
-0.210069444444444	124.424679506654\\
-0.208333333333333	119.444146434492\\
-0.206597222222222	98.3599635702274\\
-0.204861111111111	117.696556892006\\
-0.203125	132.380085195638\\
-0.201388888888889	138.851903061687\\
-0.199652777777778	140.594447392379\\
-0.197916666666667	137.657983299685\\
-0.196180555555556	126.873302103856\\
-0.194444444444444	94.2660405449815\\
-0.192708333333333	130.372807850111\\
-0.190972222222222	137.62061774722\\
-0.189236111111111	137.445976657087\\
-0.1875	131.217122860578\\
-0.185763888888889	128.65567924737\\
-0.184027777777778	138.470775488954\\
-0.182291666666667	144.148485835282\\
-0.180555555555556	145.228156078968\\
-0.178819444444444	142.262762609205\\
-0.177083333333333	134.841983488946\\
-0.175347222222222	121.252626976122\\
-0.173611111111111	98.7015096116029\\
-0.171875	89.6047238121848\\
-0.170138888888889	115.553819203374\\
-0.168402777777778	128.577876341936\\
-0.166666666666667	134.293377978783\\
-0.164930555555556	134.705591203672\\
-0.163194444444444	129.440040673686\\
-0.161458333333333	113.254951470659\\
-0.159722222222222	109.46034365623\\
-0.157986111111111	124.742040548181\\
-0.15625	127.736905749857\\
-0.154513888888889	124.360477270452\\
-0.152777777777778	113.79165391325\\
-0.151041666666667	85.4901870457164\\
-0.149305555555556	106.014237869905\\
-0.147569444444444	114.214264822596\\
-0.145833333333333	116.345847722821\\
-0.144097222222222	114.594123518472\\
-0.142361111111111	107.463735060949\\
-0.140625	97.5257950939507\\
-0.138888888888889	109.394442105985\\
-0.137152777777778	116.532447875748\\
-0.135416666666667	117.786115772848\\
-0.133680555555556	114.698066294135\\
-0.131944444444444	108.09709600794\\
-0.130208333333333	97.7498872206536\\
-0.128472222222222	76.1286389143384\\
-0.126736111111111	100.309377283113\\
-0.125	111.935195830405\\
-0.123263888888889	115.885093036261\\
-0.121527777777778	114.235059081799\\
-0.119791666666667	107.41338966213\\
-0.118055555555556	103.471971698015\\
-0.116319444444444	107.772583197145\\
-0.114583333333333	106.945451678681\\
-0.112847222222222	99.3695933034125\\
-0.111111111111111	104.002598387513\\
-0.109375	112.349599685375\\
-0.107638888888889	114.145910549942\\
-0.105902777777778	111.004016878441\\
-0.104166666666667	107.337161977519\\
-0.102430555555556	107.092949286723\\
-0.100694444444444	101.82340284085\\
-0.0989583333333334	85.036201520947\\
-0.0972222222222222	110.969440628234\\
-0.0954861111111112	120.783636138651\\
-0.09375	123.473322327991\\
-0.0920138888888888	120.776663572797\\
-0.0902777777777778	112.790835609745\\
-0.0885416666666666	107.89561407939\\
-0.0868055555555556	113.245963647972\\
-0.0850694444444444	114.633831542126\\
-0.0833333333333334	111.719317627712\\
-0.0815972222222222	108.25160700812\\
-0.0798611111111112	104.755795371584\\
-0.078125	89.7293301256767\\
-0.0763888888888888	101.132547014176\\
-0.0746527777777778	117.253291440546\\
-0.0729166666666666	123.394010540889\\
-0.0711805555555556	124.053220296104\\
-0.0694444444444444	120.052258884849\\
-0.0677083333333334	112.466875417792\\
-0.0659722222222222	109.350736133258\\
-0.0642361111111112	110.73719904529\\
-0.0625	107.312407288061\\
-0.0607638888888888	101.041551701961\\
-0.0590277777777778	107.324432228473\\
-0.0572916666666666	112.927993472753\\
-0.0555555555555556	113.372471613667\\
-0.0538194444444444	110.477244579864\\
-0.0520833333333334	108.206429788068\\
-0.0503472222222222	108.018417982546\\
-0.0486111111111112	104.950269043468\\
-0.046875	97.2728890656268\\
-0.0451388888888888	106.970850831594\\
-0.0434027777777778	117.546649979454\\
-0.0416666666666666	122.034476193578\\
-0.0399305555555556	121.40994174093\\
-0.0381944444444444	116.93361915962\\
-0.0364583333333334	120.0424187116\\
-0.0347222222222222	128.45070451607\\
-0.0329861111111112	132.449989140821\\
-0.03125	131.264471337737\\
-0.0295138888888888	123.022703427311\\
-0.0277777777777778	93.4044296434755\\
-0.0260416666666666	117.891681052889\\
-0.0243055555555556	127.091131878628\\
-0.0225694444444444	131.538908304864\\
-0.0208333333333334	136.577610051584\\
-0.0190972222222222	140.40763709936\\
-0.0173611111111112	140.186619708778\\
-0.015625	132.863905796937\\
-0.0138888888888888	117.924810499009\\
-0.0121527777777778	136.868061449831\\
-0.0104166666666666	143.896628220488\\
-0.00868055555555558	141.516127819124\\
-0.00694444444444442	117.938773507857\\
-0.00520833333333337	143.282920331124\\
-0.00347222222222221	157.594669127593\\
-0.00173611111111116	163.897166738469\\
0	165.781722970912\\
0.00173611111111116	163.897166738469\\
0.00347222222222232	157.594669127593\\
0.00520833333333326	143.282920331124\\
0.00694444444444442	117.938773507857\\
0.00868055555555558	141.516127819124\\
0.0104166666666667	143.896628220488\\
0.0121527777777777	136.868061449831\\
0.0138888888888888	117.924810499009\\
0.015625	132.863905796937\\
0.0173611111111112	140.186619708778\\
0.0190972222222223	140.40763709936\\
0.0208333333333333	136.577610051584\\
0.0225694444444444	131.538908304864\\
0.0243055555555556	127.091131878628\\
0.0260416666666667	117.891681052889\\
0.0277777777777777	93.4044296434755\\
0.0295138888888888	123.022703427311\\
0.03125	131.264471337737\\
0.0329861111111112	132.449989140821\\
0.0347222222222223	128.45070451607\\
0.0364583333333333	120.0424187116\\
0.0381944444444444	116.93361915962\\
0.0399305555555556	121.40994174093\\
0.0416666666666667	122.034476193578\\
0.0434027777777777	117.546649979454\\
0.0451388888888888	106.970850831594\\
0.046875	97.2728890656268\\
0.0486111111111112	104.950269043468\\
0.0503472222222223	108.018417982546\\
0.0520833333333333	108.206429788068\\
0.0538194444444444	110.477244579864\\
0.0555555555555556	113.372471613667\\
0.0572916666666667	112.927993472753\\
0.0590277777777777	107.324432228473\\
0.0607638888888888	101.041551701961\\
0.0625	107.312407288061\\
0.0642361111111112	110.73719904529\\
0.0659722222222223	109.350736133258\\
0.0677083333333333	112.466875417792\\
0.0694444444444444	120.052258884849\\
0.0711805555555556	124.053220296104\\
0.0729166666666667	123.394010540889\\
0.0746527777777777	117.253291440546\\
0.0763888888888888	101.132547014176\\
0.078125	89.7293301256767\\
0.0798611111111112	104.755795371584\\
0.0815972222222223	108.25160700812\\
0.0833333333333333	111.719317627712\\
0.0850694444444444	114.633831542126\\
0.0868055555555556	113.245963647972\\
0.0885416666666667	107.89561407939\\
0.0902777777777777	112.790835609745\\
0.0920138888888888	120.776663572797\\
0.09375	123.473322327991\\
0.0954861111111112	120.783636138651\\
0.0972222222222223	110.969440628234\\
0.0989583333333333	85.036201520947\\
0.100694444444444	101.82340284085\\
0.102430555555556	107.092949286723\\
0.104166666666667	107.337161977519\\
0.105902777777778	111.004016878441\\
0.107638888888889	114.145910549942\\
0.109375	112.349599685375\\
0.111111111111111	104.002598387513\\
0.112847222222222	99.3695933034125\\
0.114583333333333	106.945451678681\\
0.116319444444444	107.772583197145\\
0.118055555555556	103.471971698015\\
0.119791666666667	107.41338966213\\
0.121527777777778	114.235059081799\\
0.123263888888889	115.885093036261\\
0.125	111.935195830405\\
0.126736111111111	100.309377283113\\
0.128472222222222	76.1286389143384\\
0.130208333333333	97.7498872206536\\
0.131944444444444	108.09709600794\\
0.133680555555556	114.698066294135\\
0.135416666666667	117.786115772848\\
0.137152777777778	116.532447875748\\
0.138888888888889	109.394442105985\\
0.140625	97.5257950939507\\
0.142361111111111	107.463735060949\\
0.144097222222222	114.594123518472\\
0.145833333333333	116.345847722821\\
0.147569444444444	114.214264822596\\
0.149305555555556	106.014237869905\\
0.151041666666667	85.4901870457164\\
0.152777777777778	113.79165391325\\
0.154513888888889	124.360477270452\\
0.15625	127.736905749857\\
0.157986111111111	124.742040548181\\
0.159722222222222	109.46034365623\\
0.161458333333333	113.254951470659\\
0.163194444444444	129.440040673686\\
0.164930555555556	134.705591203672\\
0.166666666666667	134.293377978783\\
0.168402777777778	128.577876341936\\
0.170138888888889	115.553819203374\\
0.171875	89.6047238121848\\
0.173611111111111	98.7015096116029\\
0.175347222222222	121.252626976122\\
0.177083333333333	134.841983488946\\
0.178819444444444	142.262762609205\\
0.180555555555556	145.228156078968\\
0.182291666666667	144.148485835282\\
0.184027777777778	138.470775488954\\
0.185763888888889	128.65567924737\\
0.1875	131.217122860578\\
0.189236111111111	137.445976657087\\
0.190972222222222	137.62061774722\\
0.192708333333333	130.372807850111\\
0.194444444444444	94.2660405449815\\
0.196180555555556	126.873302103856\\
0.197916666666667	137.657983299685\\
0.199652777777778	140.594447392379\\
0.201388888888889	138.851903061687\\
0.203125	132.380085195638\\
0.204861111111111	117.696556892006\\
0.206597222222222	98.3599635702274\\
0.208333333333333	119.444146434492\\
0.210069444444444	124.424679506654\\
0.211805555555556	124.302369740734\\
0.213541666666667	121.24437123012\\
0.215277777777778	115.901659053053\\
0.217013888888889	107.727372915877\\
0.21875	92.8753884777097\\
0.220486111111111	78.415816785375\\
0.222222222222222	99.776960630358\\
0.223958333333333	107.958427689307\\
0.225694444444444	111.653713178768\\
0.227430555555556	112.35359111871\\
0.229166666666667	110.990113595303\\
0.230902777777778	111.35544531332\\
0.232638888888889	116.526393957831\\
0.234375	121.754329047145\\
0.236111111111111	124.345229772143\\
0.237847222222222	123.83438947662\\
0.239583333333333	120.092298320967\\
0.241319444444444	114.627658791451\\
0.243055555555556	114.77977938203\\
0.244791666666667	118.950099142421\\
0.246527777777778	120.583566944193\\
0.248263888888889	118.797577483639\\
0.25	114.031835400912\\
0.251736111111111	108.630686714102\\
0.253472222222222	107.600761618908\\
0.255208333333333	109.678648401691\\
0.256944444444444	111.352982267791\\
0.258680555555556	111.359413520431\\
0.260416666666667	107.959401089071\\
0.262152777777778	96.9465839729238\\
0.263888888888889	91.4659559680069\\
0.265625	108.486273337691\\
0.267361111111111	114.947023674955\\
0.269097222222222	115.516276300265\\
0.270833333333333	110.44757670424\\
0.272569444444444	94.8403050853752\\
0.274305555555556	87.0744266075841\\
0.276041666666667	105.331416155358\\
0.277777777777778	110.046007433063\\
0.279513888888889	109.830642047972\\
0.28125	106.657202493134\\
0.282986111111111	101.075553217167\\
0.284722222222222	91.3249110686645\\
0.286458333333333	74.5733249666231\\
0.288194444444444	89.4380152368322\\
0.289930555555556	97.9532869577505\\
0.291666666666667	100.409033981041\\
0.293402777777778	100.021089331975\\
0.295138888888889	99.3917052086975\\
0.296875	99.0509470577395\\
0.298611111111111	96.8904500694297\\
0.300347222222222	91.0552172370125\\
0.302083333333333	82.3321559993889\\
0.303819444444444	79.2159720806553\\
0.305555555555556	76.6723092450581\\
0.307291666666667	68.9000729612865\\
0.309027777777778	81.1547546923029\\
0.310763888888889	89.5665530286911\\
0.3125	91.8058074544992\\
0.314236111111111	91.6310679963797\\
0.315972222222222	93.5740453928295\\
0.317708333333333	96.3688390560169\\
0.319444444444444	96.3277794695746\\
0.321180555555556	91.9191216294575\\
0.322916666666667	81.8002783177442\\
0.324652777777778	69.5819298580414\\
0.326388888888889	81.0726192480121\\
0.328125	92.8960821816983\\
0.329861111111111	100.642165194677\\
0.331597222222222	104.341603895901\\
0.333333333333333	103.947533978059\\
0.335069444444444	99.3952730581164\\
0.336805555555556	94.3117524914624\\
0.338541666666667	97.5222888359219\\
0.340277777777778	101.19823078807\\
0.342013888888889	102.954087809727\\
0.34375	106.958483442039\\
0.345486111111111	112.341841486357\\
0.347222222222222	115.525059355661\\
0.348958333333333	115.294424728468\\
0.350694444444444	110.94928474756\\
0.352430555555556	100.658842700999\\
0.354166666666667	76.7038611328867\\
0.355902777777778	72.2242821626779\\
0.357638888888889	37.8022088588558\\
0.359375	92.8782309545543\\
0.361111111111111	107.831257617117\\
0.362847222222222	114.594364408852\\
0.364583333333333	116.541791889564\\
0.366319444444444	114.34519591754\\
0.368055555555556	107.404308478756\\
0.369791666666667	93.4292121816113\\
0.371527777777778	84.2734604638482\\
0.373263888888889	90.0238441533013\\
0.375	88.9493323476262\\
0.376736111111111	96.2925969168961\\
0.378472222222222	105.210096285839\\
0.380208333333333	109.064504628748\\
0.381944444444444	107.97808478597\\
0.383680555555556	100.733646279531\\
0.385416666666667	83.2231460796612\\
0.387152777777778	95.9817053429302\\
0.388888888888889	105.038057633433\\
0.390625	106.728225111965\\
0.392361111111111	102.709484485761\\
0.394097222222222	90.0121498930992\\
0.395833333333333	80.4980847011452\\
0.397569444444444	98.2363842313519\\
0.399305555555556	104.263895054894\\
0.401041666666667	105.044787378864\\
0.402777777777778	101.624607007434\\
0.404513888888889	92.2734892390083\\
0.40625	54.4829042694759\\
0.407986111111111	88.292341183783\\
0.409722222222222	100.098342457065\\
0.411458333333333	105.216168153856\\
0.413194444444444	107.102195215363\\
0.414930555555556	106.6422955468\\
0.416666666666667	103.978597735007\\
0.418402777777778	98.6571143815199\\
0.420138888888889	89.0967724811288\\
0.421875	75.386861504833\\
0.423611111111111	85.2510153847794\\
0.425347222222222	93.4184545800262\\
0.427083333333333	96.0578661203076\\
0.428819444444444	94.4063862545348\\
0.430555555555556	88.9743728064136\\
0.432291666666667	87.6994035592087\\
0.434027777777778	95.1076753187338\\
0.435763888888889	99.8939438065896\\
0.4375	100.420225525965\\
0.439236111111111	96.3553070781673\\
0.440972222222222	85.3547908620072\\
0.442708333333333	61.3992697662226\\
0.444444444444444	80.734599907371\\
0.446180555555556	87.1483868360922\\
0.447916666666667	89.5166033230581\\
0.449652777777778	91.9884732434817\\
0.451388888888889	92.9241722587592\\
0.453125	89.2338868894895\\
0.454861111111111	80.7665805654189\\
0.456597222222222	89.5076519121094\\
0.458333333333333	98.8515783729325\\
0.460069444444444	102.321255636124\\
0.461805555555556	101.292767013213\\
0.463541666666667	96.6226468600294\\
0.465277777777778	91.8321231319534\\
0.467013888888889	92.2402201964761\\
0.46875	92.7834263697159\\
0.470486111111111	90.7517884531823\\
0.472222222222222	90.6751329761304\\
0.473958333333333	94.4078799448891\\
0.475694444444444	96.4857585405377\\
0.477430555555556	94.8380652504796\\
0.479166666666667	88.8037640460727\\
0.480902777777778	78.0138012255435\\
0.482638888888889	78.2793696550577\\
0.484375	88.7603228080435\\
0.486111111111111	95.3818676044848\\
0.487847222222222	98.5408704732461\\
0.489583333333333	98.3224016891213\\
0.491319444444444	94.1657571676553\\
0.493055555555556	84.2977909750284\\
0.494791666666667	60.597633874652\\
0.496527777777778	59.8371391042878\\
0.498263888888889	69.8530116664942\\
0.5	80.6389089062811\\
0.501736111111111	90.5467730064451\\
0.503472222222222	96.051507087067\\
0.505208333333333	97.7809807799684\\
0.506944444444444	96.1101948633418\\
0.508680555555556	90.2541654935757\\
0.510416666666667	74.3598148374428\\
0.512152777777778	70.417112880658\\
0.513888888888889	90.5278990017208\\
0.515625	97.4107068430287\\
0.517361111111111	98.7582697044466\\
0.519097222222222	95.1518822897195\\
0.520833333333333	84.2470854793621\\
0.522569444444444	54.25493217849\\
0.524305555555556	77.1854130661819\\
0.526041666666667	79.3741224780858\\
0.527777777777778	72.1358224776339\\
0.529513888888889	83.8370527909283\\
0.53125	93.5667471445441\\
0.532986111111111	96.6529576780961\\
0.534722222222222	94.3603817299315\\
0.536458333333333	85.4424568591732\\
0.538194444444444	58.6849623400191\\
0.539930555555556	72.8382971643286\\
0.541666666666667	82.3521017768617\\
0.543402777777778	87.2652886261158\\
0.545138888888889	93.0369491107423\\
0.546875	97.5353387537371\\
0.548611111111111	99.1715774533344\\
0.550347222222222	97.5972298869029\\
0.552083333333333	92.4536011444222\\
0.553819444444444	82.792920463223\\
0.555555555555556	69.1269420805085\\
0.557291666666667	69.4728349802509\\
0.559027777777778	72.3125187767182\\
0.560763888888889	69.1452321232044\\
0.5625	62.3680107550452\\
0.564236111111111	67.0663974096717\\
0.565972222222222	76.4647810313981\\
0.567708333333333	82.3824165766594\\
0.569444444444444	85.1678468317865\\
0.571180555555556	84.7437489370974\\
0.572916666666667	79.6850492358072\\
0.574652777777778	62.1297606043843\\
0.576388888888889	66.2271069622743\\
0.578125	83.9958513724674\\
0.579861111111111	90.7877982202677\\
0.581597222222222	93.1851485013559\\
0.583333333333333	92.8251629700207\\
0.585069444444444	90.5204227727799\\
0.586805555555556	86.3199075378826\\
0.588541666666667	78.3393642049967\\
0.590277777777778	60.3732603643948\\
0.592013888888889	72.7487396208621\\
0.59375	83.1843843892033\\
0.595486111111111	85.2800212656777\\
0.597222222222222	80.0736070096643\\
0.598958333333333	56.0756936242229\\
0.600694444444444	71.7073369399955\\
0.602430555555556	84.1107250847411\\
0.604166666666667	86.2458577444683\\
0.605902777777778	81.0156027016814\\
0.607638888888889	60.7181517281374\\
0.609375	70.0965563639926\\
0.611111111111111	82.7012777464209\\
0.612847222222222	85.7621511569838\\
0.614583333333333	84.5181027361009\\
0.616319444444444	82.1682722849374\\
0.618055555555556	80.7602125107911\\
0.619791666666667	76.6303455874833\\
0.621527777777778	61.1478813226059\\
0.623263888888889	75.7928869409664\\
0.625	89.3612819015866\\
0.626736111111111	94.8961397767871\\
0.628472222222222	95.2714675430681\\
0.630208333333333	90.0428931888965\\
0.631944444444444	72.6391319067472\\
0.633680555555556	71.0507489120065\\
0.635416666666667	87.3661570014495\\
0.637152777777778	90.4203826928251\\
0.638888888888889	86.3109211515586\\
0.640625	76.980318903129\\
0.642361111111111	83.9529508520603\\
0.644097222222222	92.0494179358055\\
0.645833333333333	94.6013674837996\\
0.647569444444444	93.0867199813768\\
0.649305555555556	88.4072678139561\\
0.651041666666667	81.7934560173046\\
0.652777777777778	74.3454684983274\\
0.654513888888889	61.6020945352215\\
0.65625	42.1970240677726\\
0.657986111111111	68.8958622801982\\
0.659722222222222	75.0904120424339\\
0.661458333333333	73.9964887106229\\
0.663194444444444	66.474176540921\\
0.664930555555556	56.63093365545\\
0.666666666666667	61.7172947530873\\
0.668402777777778	63.5758449783295\\
0.670138888888889	56.205408353405\\
0.671875	28.8575638804298\\
0.673611111111111	63.9088075123886\\
0.675347222222222	75.4848892748275\\
0.677083333333333	80.8638842138629\\
0.678819444444444	81.8258709509328\\
0.680555555555556	77.9302463447317\\
0.682291666666667	69.6600217985477\\
0.684027777777778	76.0095952685352\\
0.685763888888889	85.5707266152413\\
0.6875	90.1501487019417\\
0.689236111111111	91.025256643183\\
0.690972222222222	88.6341828535189\\
0.692708333333333	82.3066246982492\\
0.694444444444444	70.1389462227247\\
0.696180555555556	70.0342537095206\\
0.697916666666667	79.5233378093388\\
0.699652777777778	81.6136223610145\\
0.701388888888889	76.2746686342221\\
0.703125	64.437345323224\\
0.704861111111111	77.2558656866618\\
0.706597222222222	84.5931436606274\\
0.708333333333333	83.9660951884025\\
0.710069444444444	76.8261726227873\\
0.711805555555556	83.0396107046742\\
0.713541666666667	93.5590422050874\\
0.715277777777778	98.1621828251794\\
0.717013888888889	98.0080890919222\\
0.71875	92.9916270294461\\
0.720486111111111	80.9847896489561\\
0.722222222222222	71.1473284809027\\
0.723958333333333	81.4007223867741\\
0.725694444444444	82.9633761631573\\
0.727430555555556	76.6529780265972\\
0.729166666666667	59.8873431828867\\
0.730902777777778	68.3380790102177\\
0.732638888888889	77.8440433601119\\
0.734375	81.4243628512774\\
0.736111111111111	82.1031197626554\\
0.737847222222222	79.7318321117764\\
0.739583333333333	71.8224115456894\\
0.741319444444444	50.272590424658\\
0.743055555555556	66.9157086596394\\
0.744791666666667	75.8800197392796\\
0.746527777777778	77.4296877034478\\
0.748263888888889	76.2768065069699\\
0.75	76.9907608624821\\
0.751736111111111	78.8685906229852\\
0.753472222222222	78.269863565838\\
0.755208333333333	73.7210263764705\\
0.756944444444444	65.1632211954121\\
0.758680555555556	60.6110219727775\\
0.760416666666667	66.4371130637972\\
0.762152777777778	71.566622630781\\
0.763888888888889	74.0048918574933\\
0.765625	73.278822954909\\
0.767361111111111	69.9958269927106\\
0.769097222222222	69.0649461423797\\
0.770833333333333	71.4893778803248\\
0.772569444444444	71.0360442666042\\
0.774305555555556	68.1980767570481\\
0.776041666666667	73.0845523710305\\
0.777777777777778	79.444756952014\\
0.779513888888889	80.9433538465549\\
0.78125	77.1537843369589\\
0.782986111111111	72.5700497751857\\
0.784722222222222	78.3901906508395\\
0.786458333333333	83.0686698242257\\
0.788194444444444	82.3602724069002\\
0.789930555555556	75.4249497483181\\
0.791666666666667	58.786272147152\\
0.793402777777778	59.652153630672\\
0.795138888888889	70.4107179562757\\
0.796875	78.6703594398817\\
0.798611111111111	85.6589110460017\\
0.800347222222222	89.2852869866607\\
0.802083333333333	88.9741755204538\\
0.803819444444444	85.082551332735\\
0.805555555555556	82.2725191439689\\
0.807291666666667	85.8745521579945\\
0.809027777777778	88.6149707766699\\
0.810763888888889	87.2052065414333\\
0.8125	80.7748756363307\\
0.814236111111111	66.7725167865278\\
0.815972222222222	38.8774692090198\\
0.817708333333333	51.2779264658017\\
0.819444444444444	49.6467709074348\\
0.821180555555556	43.0698614495742\\
0.822916666666667	28.8796167529766\\
0.824652777777778	65.8595642466184\\
0.826388888888889	79.3936054475115\\
0.828125	85.3413253457908\\
0.829861111111111	85.3454528588257\\
0.831597222222222	77.2221935520859\\
0.833333333333333	25.3220511449291\\
0.835069444444444	79.8680484697962\\
0.836805555555556	88.3479085289368\\
0.838541666666667	88.7419572404219\\
0.840277777777778	81.649780361361\\
0.842013888888889	57.0190913290982\\
0.84375	73.5031530922949\\
0.845486111111111	81.6663434386633\\
0.847222222222222	79.8103701028832\\
0.848958333333333	73.9771524029758\\
0.850694444444444	81.238292412675\\
0.852430555555556	87.9734566466834\\
0.854166666666667	88.9413657765602\\
0.855902777777778	84.1053302889223\\
0.857638888888889	74.3587405363664\\
0.859375	80.3828921935322\\
0.861111111111111	88.6459153612585\\
0.862847222222222	91.6871331617218\\
0.864583333333333	90.959430655473\\
0.866319444444444	87.1896896314133\\
0.868055555555556	81.3628378376814\\
0.869791666666667	77.6463408917373\\
0.871527777777778	79.1776546271778\\
0.873263888888889	80.5218781456106\\
0.875	78.4417794500417\\
0.876736111111111	72.477500891049\\
0.878472222222222	72.5951671235057\\
0.880208333333333	81.0111131430399\\
0.881944444444444	85.4678717473144\\
0.883680555555556	84.6920920097817\\
0.885416666666667	76.3274996402945\\
0.887152777777778	43.3718052938499\\
0.888888888888889	79.2854605020709\\
0.890625	88.5910075093948\\
0.892361111111111	90.7240826691747\\
0.894097222222222	87.6821231362653\\
0.895833333333333	79.6172558473633\\
0.897569444444444	75.3896335553656\\
0.899305555555556	80.8710214935648\\
0.901041666666667	81.5259807541695\\
0.902777777777778	74.3192016708202\\
0.904513888888889	58.395410967342\\
0.90625	74.1857743759757\\
0.907986111111111	81.3004444809841\\
0.909722222222222	81.5624357645856\\
0.911458333333333	77.0538328437846\\
0.913194444444444	69.3772821029138\\
0.914930555555556	64.5046876164492\\
0.916666666666667	69.6591336000207\\
0.918402777777778	78.1126976343461\\
0.920138888888889	83.9207439235528\\
0.921875	85.9895709874381\\
0.923611111111111	84.032527781582\\
0.925347222222222	77.018464000994\\
0.927083333333333	59.6820326345394\\
0.928819444444444	52.4786826204944\\
0.930555555555556	71.9069837416938\\
0.932291666666667	80.7344396644502\\
0.934027777777778	86.3088738778435\\
0.935763888888889	89.2257017648301\\
0.9375	89.7960973664361\\
0.939236111111111	88.9697798695176\\
0.940972222222222	88.2758264687342\\
0.942708333333333	88.1105416228263\\
0.944444444444444	87.1723965731343\\
0.946180555555556	84.4864238116157\\
0.947916666666667	80.1023495616297\\
0.949652777777778	73.8374029244498\\
0.951388888888889	63.3641660616117\\
0.953125	68.4966509966492\\
0.954861111111111	81.3032518702256\\
0.956597222222222	87.4121259058447\\
0.958333333333333	88.6185544132282\\
0.960069444444444	85.2348320656686\\
0.961805555555556	76.6917713765674\\
0.963541666666667	69.3958935440523\\
0.965277777777778	74.1749220189929\\
0.967013888888889	75.037776785967\\
0.96875	71.4966795294092\\
0.970486111111111	73.8983598925008\\
0.972222222222222	79.9795086362025\\
0.973958333333333	82.2869989963824\\
0.975694444444444	81.3625790897174\\
0.977430555555556	81.0916769126164\\
0.979166666666667	84.1721990616269\\
0.980902777777778	86.876303964656\\
0.982638888888889	86.9893445962065\\
0.984375	84.7781282285205\\
0.986111111111111	82.6038602661371\\
0.987847222222222	83.2348856622312\\
0.989583333333333	84.6393923866235\\
0.991319444444444	84.0651881628394\\
0.993055555555556	80.4487008793162\\
0.994791666666667	72.8820400576052\\
0.996527777777778	59.8297479466518\\
0.998263888888889	51.144526621686\\
};
\addlegendentry{Original Data};

\addplot [color=red,solid]
  table[row sep=crcr]{-1	54.6195193956359\\
-0.998263888888889	56.4634162425436\\
-0.996527777777778	53.2579482943771\\
-0.994791666666667	66.4835800412787\\
-0.993055555555556	80.4487008793162\\
-0.991319444444444	86.9915197298748\\
-0.989583333333333	88.0665016843963\\
-0.987847222222222	84.6693232266961\\
-0.986111111111111	82.6038602661371\\
-0.984375	87.9914910930404\\
-0.982638888888889	91.6125844605194\\
-0.980902777777778	90.4678226033272\\
-0.979166666666667	84.1721990616269\\
-0.977430555555556	77.4417044680171\\
-0.975694444444444	81.5592625027476\\
-0.973958333333333	83.5650556525401\\
-0.972222222222222	79.9795086362025\\
-0.970486111111111	76.5119687448267\\
-0.96875	80.4037216211075\\
-0.967013888888889	81.3197493197072\\
-0.965277777777778	74.1749220189928\\
-0.963541666666667	68.3994696678691\\
-0.961805555555556	82.6826582576182\\
-0.960069444444444	88.7006581519333\\
-0.958333333333333	88.6185544132283\\
-0.956597222222222	83.0620434941229\\
-0.954861111111111	71.2296152184941\\
-0.953125	62.5257322426344\\
-0.951388888888889	63.3641660616117\\
-0.949652777777778	63.4078422920839\\
-0.947916666666667	76.1687159331623\\
-0.946180555555556	84.4184957785129\\
-0.944444444444444	87.1723965731343\\
-0.942708333333333	85.7093612442623\\
-0.940972222222222	82.7600855305042\\
-0.939236111111111	84.9890454319102\\
-0.9375	89.7960973664361\\
-0.935763888888889	92.0793006763513\\
-0.934027777777778	90.8939389243572\\
-0.932291666666667	85.4344922245655\\
-0.930555555555556	71.9069837416938\\
-0.928819444444444	49.1937093238558\\
-0.927083333333333	76.0667746477134\\
-0.925347222222222	82.7538887852624\\
-0.923611111111111	84.032527781582\\
-0.921875	82.0809001921081\\
-0.920138888888889	77.7328392587231\\
-0.918402777777778	72.077260676746\\
-0.916666666666667	69.6591336000207\\
-0.914930555555556	74.1275404422395\\
-0.913194444444444	79.2752381851081\\
-0.911458333333333	81.9942124500894\\
-0.909722222222222	81.5624357645856\\
-0.907986111111111	76.7837501088163\\
-0.90625	62.417996191153\\
-0.904513888888889	53.1866589144205\\
-0.902777777777778	74.3192016708202\\
-0.901041666666667	80.472019663996\\
-0.899305555555556	81.1589484335572\\
-0.897569444444444	79.2022636729145\\
-0.895833333333333	79.6172558473633\\
-0.894097222222222	83.4188521673275\\
-0.892361111111111	85.8145972749062\\
-0.890625	84.8307138881177\\
-0.888888888888889	79.2854605020708\\
-0.887152777777778	65.7020481203978\\
-0.885416666666667	66.5366490709038\\
-0.883680555555556	79.9972240668504\\
-0.881944444444444	85.4678717473144\\
-0.880208333333333	85.9509691041779\\
-0.878472222222222	81.5246462975824\\
-0.876736111111111	73.4788950995947\\
-0.875	78.4417794500417\\
-0.873263888888889	85.3833047530452\\
-0.871527777777778	86.3979892738523\\
-0.869791666666667	82.157643509553\\
-0.868055555555556	81.3628378376814\\
-0.866319444444444	89.6113166803636\\
-0.864583333333333	94.4819329985031\\
-0.862847222222222	94.4052546944778\\
-0.861111111111111	88.6459153612585\\
-0.859375	73.9069763845588\\
-0.857638888888889	77.685666371019\\
-0.855902777777778	87.447135289381\\
-0.854166666666667	88.9413657765602\\
-0.852430555555556	84.5043972722795\\
-0.850694444444444	76.0632058555955\\
-0.848958333333333	76.3893373609393\\
-0.847222222222222	79.8103701028832\\
-0.845486111111111	76.6970922035977\\
-0.84375	64.8232052124182\\
-0.842013888888889	71.2235917128984\\
-0.840277777777778	81.649780361361\\
-0.838541666666667	84.8255853950306\\
-0.836805555555556	82.5928204774657\\
-0.835069444444444	73.3558719945592\\
-0.833333333333333	25.3220511449284\\
-0.831597222222222	70.0074954665546\\
-0.829861111111111	79.4510182603316\\
-0.828125	81.5488028225\\
-0.826388888888889	79.3936054475115\\
-0.824652777777778	74.2637559313553\\
-0.822916666666667	68.30170391545\\
-0.821180555555556	62.3193704738494\\
-0.819444444444444	49.6467709074351\\
-0.817708333333333	51.9504892287614\\
-0.815972222222222	67.8788472145802\\
-0.814236111111111	75.6830203025841\\
-0.8125	80.7748756363307\\
-0.810763888888889	84.5673992097145\\
-0.809027777777778	86.5281394124206\\
-0.807291666666667	85.7953950891063\\
-0.805555555555556	82.2725191439689\\
-0.803819444444444	79.7891796080601\\
-0.802083333333333	83.2365559632725\\
-0.800347222222222	86.3773453155245\\
-0.798611111111111	85.6589110460018\\
-0.796875	79.9474340893657\\
-0.795138888888889	65.9638548221257\\
-0.793402777777778	32.1466551384892\\
-0.791666666666667	58.7862721471522\\
-0.789930555555556	64.7375696066335\\
-0.788194444444444	72.5236416782629\\
-0.786458333333333	78.0563456837503\\
-0.784722222222222	78.3901906508395\\
-0.782986111111111	72.1834358758904\\
-0.78125	63.4522652722089\\
-0.779513888888889	73.8565612343044\\
-0.777777777777778	79.444756952014\\
-0.776041666666667	77.9777596182824\\
-0.774305555555556	67.948231705152\\
-0.772569444444444	55.6150058743299\\
-0.770833333333333	71.4893778803248\\
-0.769097222222222	75.8237322045822\\
-0.767361111111111	73.3297022349802\\
-0.765625	69.3248474074702\\
-0.763888888888889	74.0048918574933\\
-0.762152777777778	78.7115325856567\\
-0.760416666666667	78.7975242979664\\
-0.758680555555556	73.6564034792638\\
-0.756944444444444	65.1632211954121\\
-0.755208333333333	71.1514417933051\\
-0.753472222222222	78.3046414081252\\
-0.751736111111111	80.1174324455292\\
-0.75	76.9907608624822\\
-0.748263888888889	68.4228146338827\\
-0.746527777777778	61.1085404998901\\
-0.744791666666667	66.0412976860913\\
-0.743055555555556	66.9157086596395\\
-0.741319444444444	67.4617439354576\\
-0.739583333333333	74.9351128294906\\
-0.737847222222222	80.7051767376876\\
-0.736111111111111	82.1031197626554\\
-0.734375	79.2104384594324\\
-0.732638888888889	72.0032063787215\\
-0.730902777777778	61.7112301547465\\
-0.729166666666667	59.8873431828867\\
-0.727430555555556	69.261536242402\\
-0.725694444444444	76.2142966630147\\
-0.723958333333333	77.5406222113867\\
-0.722222222222222	71.1473284809026\\
-0.720486111111111	73.3628293004128\\
-0.71875	88.1822570794298\\
-0.717013888888889	95.7260856948412\\
-0.715277777777778	98.1621828251794\\
-0.713541666666667	96.2005854798249\\
-0.711805555555556	89.5176165773546\\
-0.710069444444444	80.9684875642588\\
-0.708333333333333	83.9660951884025\\
-0.706597222222222	86.8346534230503\\
-0.704861111111111	83.7511017578218\\
-0.703125	75.9056242423589\\
-0.701388888888889	76.2746686342221\\
-0.699652777777778	80.9180704025599\\
-0.697916666666667	80.155097159993\\
-0.696180555555556	74.3526065345709\\
-0.694444444444444	70.1389462227247\\
-0.692708333333333	75.182865025223\\
-0.690972222222222	81.6057472759345\\
-0.689236111111111	87.0480681966658\\
-0.6875	90.1501487019417\\
-0.685763888888889	89.7932374663137\\
-0.684027777777778	85.1478388462812\\
-0.682291666666667	77.2555686724385\\
-0.680555555555556	77.9302463447318\\
-0.678819444444444	81.9383377004225\\
-0.677083333333333	80.5097600879856\\
-0.675347222222222	72.8081633473505\\
-0.673611111111111	63.9088075123886\\
-0.671875	65.7695658658945\\
-0.670138888888889	60.6167565704468\\
-0.668402777777778	24.9518334117497\\
-0.666666666666667	61.7172947530872\\
-0.664930555555556	65.575283824703\\
-0.663194444444444	50.2398379504649\\
-0.661458333333333	60.241258493634\\
-0.659722222222222	75.0904120424339\\
-0.657986111111111	77.6598772076331\\
-0.65625	71.5108999465172\\
-0.654513888888889	62.9386726191769\\
-0.652777777777778	74.3454684983274\\
-0.651041666666667	79.8977286570208\\
-0.649305555555556	83.6281254399417\\
-0.647569444444444	89.8439723271007\\
-0.645833333333333	94.6013674837996\\
-0.644097222222222	95.1601994639866\\
-0.642361111111111	90.0022938769532\\
-0.640625	78.8526305518046\\
-0.638888888888889	86.3109211515586\\
-0.637152777777778	93.3464546843081\\
-0.635416666666667	92.2656951575264\\
-0.633680555555556	79.2510944535521\\
-0.631944444444444	72.6391319067472\\
-0.630208333333333	93.2213668300783\\
-0.628472222222222	98.5537916903455\\
-0.626736111111111	97.2559395915512\\
-0.625	89.3612819015865\\
-0.623263888888889	72.6884316762753\\
-0.621527777777778	78.4309228709951\\
-0.619791666666667	83.0986763919756\\
-0.618055555555556	80.7602125107911\\
-0.616319444444444	82.7090610373543\\
-0.614583333333333	88.3212636424006\\
-0.612847222222222	89.1601643779253\\
-0.611111111111111	82.7012777464209\\
-0.609375	55.5042956461214\\
-0.607638888888889	76.1754767849571\\
-0.605902777777778	85.9603839175002\\
-0.604166666666667	86.2458577444683\\
-0.602430555555556	78.8276705505832\\
-0.600694444444444	59.9605307385236\\
-0.598958333333333	72.94079572121\\
-0.597222222222222	80.0736070096643\\
-0.595486111111111	79.1255376445775\\
-0.59375	71.3088041602578\\
-0.592013888888889	56.5466554177746\\
-0.590277777777778	60.3732603643948\\
-0.588541666666667	67.590321272645\\
-0.586805555555556	77.3390165485636\\
-0.585069444444444	86.8827617405539\\
-0.583333333333333	92.8251629700207\\
-0.581597222222222	94.9996522565615\\
-0.579861111111111	93.2173480712842\\
-0.578125	86.072026025541\\
-0.576388888888889	66.2271069622743\\
-0.574652777777778	71.2214343392866\\
-0.572916666666667	83.416438376494\\
-0.571180555555556	86.2059676743654\\
-0.569444444444444	85.1678468317865\\
-0.567708333333333	82.5962145001111\\
-0.565972222222222	78.0172711114495\\
-0.564236111111111	65.5657470180412\\
-0.5625	62.3680107550452\\
-0.560763888888889	79.2487043617849\\
-0.559027777777778	83.7352851736058\\
-0.557291666666667	79.8201089935943\\
-0.555555555555556	69.1269420805085\\
-0.553819444444444	84.4979784122465\\
-0.552083333333333	94.5619718184053\\
-0.550347222222222	98.718131871948\\
-0.548611111111111	99.1715774533344\\
-0.546875	97.1311326349748\\
-0.545138888888889	93.842443096287\\
-0.543402777777778	89.7887702772737\\
-0.541666666666667	82.3521017768617\\
-0.539930555555556	60.4190387856098\\
-0.538194444444444	78.3852456890257\\
-0.536458333333333	90.4386547609073\\
-0.534722222222222	94.3603817299315\\
-0.532986111111111	93.2763354583693\\
-0.53125	87.4192309296156\\
-0.529513888888889	76.9079576534063\\
-0.527777777777778	72.1358224776339\\
-0.526041666666667	72.1065921084484\\
-0.524305555555556	56.6852704679003\\
-0.522569444444444	66.6222477904486\\
-0.520833333333333	84.2470854793621\\
-0.519097222222222	91.4549280648706\\
-0.517361111111111	94.3060979516109\\
-0.515625	94.1167831531484\\
-0.513888888888889	90.5278990017208\\
-0.512152777777778	80.2871876385094\\
-0.510416666666667	62.6861223190267\\
-0.508680555555556	87.1511191512281\\
-0.506944444444444	96.1101948633418\\
-0.505208333333333	99.079716059507\\
-0.503472222222222	97.8046591394004\\
-0.501736111111111	92.2175321248057\\
-0.5	80.6389089062811\\
-0.498263888888889	55.7423060408892\\
-0.496527777777778	63.0040579869352\\
-0.494791666666667	75.0293522456125\\
-0.493055555555556	84.2977909750284\\
-0.491319444444444	90.6996427166262\\
-0.489583333333333	94.1778491761088\\
-0.487847222222222	95.5504340740742\\
-0.486111111111111	95.3818676044848\\
-0.484375	93.2844773788822\\
-0.482638888888889	87.9693383010236\\
-0.480902777777778	81.9292590231391\\
-0.479166666666667	88.8037640460727\\
-0.477430555555556	95.9318563076568\\
-0.475694444444444	98.3884322124829\\
-0.473958333333333	96.5498632549698\\
-0.472222222222222	90.6751329761304\\
-0.470486111111111	85.4038101931879\\
-0.46875	88.6994772054879\\
-0.467013888888889	91.5891788202839\\
-0.465277777777778	91.8321231319534\\
-0.463541666666667	93.261055730137\\
-0.461805555555556	97.1126327344473\\
-0.460069444444444	99.6748576177026\\
-0.458333333333333	98.8515783729325\\
-0.456597222222222	93.386626463067\\
-0.454861111111111	82.371967473759\\
-0.453125	84.7815004475717\\
-0.451388888888889	92.9241722587592\\
-0.449652777777778	95.1994017758572\\
-0.447916666666667	93.2185977716416\\
-0.446180555555556	88.315236808342\\
-0.444444444444444	80.734599907371\\
-0.442708333333333	60.0733644028311\\
-0.440972222222222	78.0166964574185\\
-0.439236111111111	93.7577992021176\\
-0.4375	100.420225525965\\
-0.435763888888889	101.7350014368\\
-0.434027777777778	97.7980236930052\\
-0.432291666666667	87.6159357548471\\
-0.430555555555556	88.9743728064136\\
-0.428819444444444	98.1357493419189\\
-0.427083333333333	100.696173545326\\
-0.425347222222222	97.2338867314295\\
-0.423611111111111	85.2510153847794\\
-0.421875	80.2428987335186\\
-0.420138888888889	95.2831642691523\\
-0.418402777777778	101.561697632978\\
-0.416666666666667	103.978597735007\\
-0.414930555555556	104.572763523483\\
-0.413194444444444	104.293943925068\\
-0.411458333333333	103.139138233202\\
-0.409722222222222	100.098342457065\\
-0.407986111111111	92.7050118660602\\
-0.40625	75.0470280643189\\
-0.404513888888889	90.2518897838701\\
-0.402777777777778	101.624607007434\\
-0.401041666666667	105.855860883494\\
-0.399305555555556	105.131216205207\\
-0.397569444444444	98.500249877505\\
-0.395833333333333	80.4980847011451\\
-0.394097222222222	92.9976449147229\\
-0.392361111111111	103.730970375455\\
-0.390625	106.915372142983\\
-0.388888888888889	105.038057633433\\
-0.387152777777778	97.8499507316562\\
-0.385416666666667	92.2240570567414\\
-0.383680555555556	101.816062938611\\
-0.381944444444444	107.97808478597\\
-0.380208333333333	109.361497419941\\
-0.378472222222222	106.519602662132\\
-0.376736111111111	99.1233184213079\\
-0.375	88.9493323476262\\
-0.373263888888889	87.85702708553\\
-0.371527777777778	90.145319603883\\
-0.369791666666667	97.6725438640515\\
-0.368055555555556	107.404308478756\\
-0.366319444444444	113.465516865135\\
-0.364583333333333	115.604415025823\\
-0.362847222222222	113.920890345547\\
-0.361111111111111	107.831257617117\\
-0.359375	95.4026788998661\\
-0.357638888888889	76.685156102872\\
-0.355902777777778	69.386507321964\\
-0.354166666666667	76.7038611328867\\
-0.352430555555556	100.945915888795\\
-0.350694444444444	111.118520261359\\
-0.348958333333333	115.432093830117\\
-0.347222222222222	115.525059355661\\
-0.345486111111111	111.738454558592\\
-0.34375	104.582660194072\\
-0.342013888888889	99.6822564196618\\
-0.340277777777778	101.19823078807\\
-0.338541666666667	101.328454129634\\
-0.336805555555556	100.103035741606\\
-0.335069444444444	101.72536161886\\
-0.333333333333333	103.947533978059\\
-0.331597222222222	102.953731661394\\
-0.329861111111111	97.3312594873937\\
-0.328125	87.0926255915462\\
-0.326388888888889	81.0726192480121\\
-0.324652777777778	76.0220975443474\\
-0.322916666666667	48.6567634500013\\
-0.321180555555556	87.2927131995774\\
-0.319444444444444	96.3277794695746\\
-0.317708333333333	97.2957257742382\\
-0.315972222222222	91.6101658707598\\
-0.314236111111111	84.8538337799585\\
-0.3125	91.8058074544992\\
-0.310763888888889	94.9854396633932\\
-0.309027777777778	90.4221174538562\\
-0.307291666666667	70.9648248604453\\
-0.305555555555556	76.672309245058\\
-0.303819444444444	89.2308735536522\\
-0.302083333333333	92.8611348436894\\
-0.300347222222222	95.1490298779732\\
-0.298611111111111	96.8904500694297\\
-0.296875	95.7657069264155\\
-0.295138888888889	92.5093053656185\\
-0.293402777777778	95.2767733199028\\
-0.291666666666667	100.409033981041\\
-0.289930555555556	101.148119706008\\
-0.288194444444444	96.1688724003939\\
-0.286458333333333	87.2630028462499\\
-0.284722222222222	91.3249110686645\\
-0.282986111111111	97.4871622175674\\
-0.28125	102.057865514316\\
-0.279513888888889	107.117701273167\\
-0.277777777777778	110.046007433063\\
-0.276041666666667	108.128056171947\\
-0.274305555555556	96.7053694471598\\
-0.272569444444444	92.4766329717402\\
-0.270833333333333	110.44757670424\\
-0.269097222222222	116.439686552454\\
-0.267361111111111	116.184857878698\\
-0.265625	109.7207513506\\
-0.263888888888889	91.4659559680069\\
-0.262152777777778	93.4539435371554\\
-0.260416666666667	105.432144756189\\
-0.258680555555556	109.307509118213\\
-0.256944444444444	111.352982267791\\
-0.255208333333333	112.152481764832\\
-0.253472222222222	110.195860689434\\
-0.251736111111111	108.139200957532\\
-0.25	114.031835400912\\
-0.248263888888889	120.333958636406\\
-0.246527777777778	122.594450654554\\
-0.244791666666667	120.53347126049\\
-0.243055555555556	114.77977938203\\
-0.241319444444444	113.951568828315\\
-0.239583333333333	120.473700660018\\
-0.237847222222222	124.287167255985\\
-0.236111111111111	124.345229772143\\
-0.234375	120.893355401653\\
-0.232638888888889	114.168660435951\\
-0.230902777777778	108.370533246484\\
-0.229166666666667	110.990113595303\\
-0.227430555555556	114.051738369587\\
-0.225694444444444	113.677033355004\\
-0.223958333333333	109.450791845398\\
-0.222222222222222	99.776960630358\\
-0.220486111111111	72.6794289368937\\
-0.21875	95.5100687129682\\
-0.217013888888889	108.778794690671\\
-0.215277777777778	115.901659053053\\
-0.213541666666667	120.236538326784\\
-0.211805555555556	122.738732149049\\
-0.210069444444444	123.137001376218\\
-0.208333333333333	119.444146434492\\
-0.206597222222222	103.618777006007\\
-0.204861111111111	114.767675011527\\
-0.203125	131.447194298882\\
-0.201388888888889	138.851903061687\\
-0.199652777777778	141.173995066351\\
-0.197916666666667	138.621717314746\\
-0.196180555555556	128.124829838592\\
-0.194444444444444	94.2660405449815\\
-0.192708333333333	131.392441967603\\
-0.190972222222222	138.536579483092\\
-0.189236111111111	138.041003152297\\
-0.1875	131.217122860578\\
-0.185763888888889	129.42659899085\\
-0.184027777777778	139.441615848315\\
-0.182291666666667	144.664647474725\\
-0.180555555555556	145.228156078968\\
-0.178819444444444	141.687582801754\\
-0.177083333333333	133.627478739391\\
-0.175347222222222	119.564383316968\\
-0.173611111111111	98.7015096116029\\
-0.171875	98.4469898373329\\
-0.170138888888889	119.111601687833\\
-0.168402777777778	129.826790707971\\
-0.166666666666667	134.293377978783\\
-0.164930555555556	133.838901063939\\
-0.163194444444444	127.882069436613\\
-0.161458333333333	111.342961286959\\
-0.159722222222222	109.46034365623\\
-0.157986111111111	123.152371001321\\
-0.15625	126.031831107942\\
-0.154513888888889	123.04519412388\\
-0.152777777777778	113.79165391325\\
-0.151041666666667	98.0505933405695\\
-0.149305555555556	108.928744205746\\
-0.147569444444444	115.73799741387\\
-0.145833333333333	116.345847722821\\
-0.144097222222222	111.986651078665\\
-0.142361111111111	102.329962200651\\
-0.140625	100.587064474485\\
-0.138888888888889	109.394442105985\\
-0.137152777777778	113.732572799186\\
-0.135416666666667	114.578592127336\\
-0.133680555555556	112.810006270045\\
-0.131944444444444	108.09709600794\\
-0.130208333333333	99.6375817727883\\
-0.128472222222222	98.8589788928844\\
-0.126736111111111	107.675004115\\
-0.125	111.935195830405\\
-0.123263888888889	111.540004646314\\
-0.121527777777778	106.967721767307\\
-0.119791666666667	101.271564053798\\
-0.118055555555556	103.471971698015\\
-0.116319444444444	107.31500210908\\
-0.114583333333333	107.610196540452\\
-0.112847222222222	105.039554989124\\
-0.111111111111111	104.002598387513\\
-0.109375	107.66708187464\\
-0.107638888888889	110.941927365887\\
-0.105902777777778	111.121377501776\\
-0.104166666666667	107.337161977518\\
-0.102430555555556	97.468004064255\\
-0.100694444444444	80.3312019404505\\
-0.0989583333333334	99.9398855089219\\
-0.0972222222222222	110.969440628234\\
-0.0954861111111112	117.070573674788\\
-0.09375	119.700030781074\\
-0.0920138888888888	118.681334902732\\
-0.0902777777777778	112.790835609745\\
-0.0885416666666666	99.514007278502\\
-0.0868055555555556	102.30747609001\\
-0.0850694444444444	111.000091069347\\
-0.0833333333333334	111.719317627712\\
-0.0815972222222222	105.254994638838\\
-0.0798611111111112	85.1931779639488\\
-0.078125	86.1879065001206\\
-0.0763888888888888	101.132547014176\\
-0.0746527777777778	110.300996047316\\
-0.0729166666666666	117.400978582715\\
-0.0711805555555556	120.9125889966\\
-0.0694444444444444	120.052258884849\\
-0.0677083333333334	113.308075299577\\
-0.0659722222222222	93.6238271419446\\
-0.0642361111111112	98.3168991561599\\
-0.0625	107.312407288061\\
-0.0607638888888888	102.729350907755\\
-0.0590277777777778	65.2486876376298\\
-0.0572916666666666	104.562286006577\\
-0.0555555555555556	113.372471613667\\
-0.0538194444444444	113.372983104858\\
-0.0520833333333334	105.021435803714\\
-0.0503472222222222	81.3264656775161\\
-0.0486111111111112	104.950269043468\\
-0.046875	114.356693947664\\
-0.0451388888888888	118.718417341967\\
-0.0434027777777778	120.987726767965\\
-0.0416666666666666	122.034476193578\\
-0.0399305555555556	123.080268325442\\
-0.0381944444444444	125.435366493343\\
-0.0364583333333334	127.93132391172\\
-0.0347222222222222	128.45070451607\\
-0.0329861111111112	125.722110000893\\
-0.03125	118.751165823084\\
-0.0295138888888888	106.246630563411\\
-0.0277777777777778	93.4044296434755\\
-0.0260416666666666	110.441039543657\\
-0.0243055555555556	124.538111656654\\
-0.0225694444444444	132.84719948386\\
-0.0208333333333334	136.577610051584\\
-0.0190972222222222	135.953166825629\\
-0.0173611111111112	129.901738230384\\
-0.015625	114.412762812262\\
-0.0138888888888888	117.924810499009\\
-0.0121527777777778	127.498153940198\\
-0.0104166666666666	127.854618168235\\
-0.00868055555555558	121.38734337324\\
-0.00694444444444442	117.938773507857\\
-0.00520833333333337	125.003498582895\\
-0.00347222222222221	126.4745477766\\
-0.00173611111111116	118.985713276342\\
0	-447.902333492743\\
0.00173611111111116	118.985713276342\\
0.00347222222222232	126.4745477766\\
0.00520833333333326	125.003498582895\\
0.00694444444444442	117.938773507857\\
0.00868055555555558	121.38734337324\\
0.0104166666666667	127.854618168235\\
0.0121527777777777	127.498153940198\\
0.0138888888888888	117.924810499009\\
0.015625	114.412762812262\\
0.0173611111111112	129.901738230384\\
0.0190972222222223	135.953166825629\\
0.0208333333333333	136.577610051584\\
0.0225694444444444	132.84719948386\\
0.0243055555555556	124.538111656654\\
0.0260416666666667	110.441039543657\\
0.0277777777777777	93.4044296434755\\
0.0295138888888888	106.246630563411\\
0.03125	118.751165823084\\
0.0329861111111112	125.722110000893\\
0.0347222222222223	128.45070451607\\
0.0364583333333333	127.93132391172\\
0.0381944444444444	125.435366493343\\
0.0399305555555556	123.080268325442\\
0.0416666666666667	122.034476193578\\
0.0434027777777777	120.987726767965\\
0.0451388888888888	118.718417341967\\
0.046875	114.356693947664\\
0.0486111111111112	104.950269043468\\
0.0503472222222223	81.3264656775161\\
0.0520833333333333	105.021435803714\\
0.0538194444444444	113.372983104858\\
0.0555555555555556	113.372471613667\\
0.0572916666666667	104.562286006577\\
0.0590277777777777	65.2486876376298\\
0.0607638888888888	102.729350907755\\
0.0625	107.312407288061\\
0.0642361111111112	98.3168991561599\\
0.0659722222222223	93.6238271419446\\
0.0677083333333333	113.308075299577\\
0.0694444444444444	120.052258884849\\
0.0711805555555556	120.9125889966\\
0.0729166666666667	117.400978582715\\
0.0746527777777777	110.300996047316\\
0.0763888888888888	101.132547014176\\
0.078125	86.1879065001206\\
0.0798611111111112	85.1931779639488\\
0.0815972222222223	105.254994638838\\
0.0833333333333333	111.719317627712\\
0.0850694444444444	111.000091069347\\
0.0868055555555556	102.30747609001\\
0.0885416666666667	99.514007278502\\
0.0902777777777777	112.790835609745\\
0.0920138888888888	118.681334902732\\
0.09375	119.700030781074\\
0.0954861111111112	117.070573674788\\
0.0972222222222223	110.969440628234\\
0.0989583333333333	99.9398855089219\\
0.100694444444444	80.3312019404505\\
0.102430555555556	97.468004064255\\
0.104166666666667	107.337161977518\\
0.105902777777778	111.121377501776\\
0.107638888888889	110.941927365887\\
0.109375	107.66708187464\\
0.111111111111111	104.002598387513\\
0.112847222222222	105.039554989124\\
0.114583333333333	107.610196540452\\
0.116319444444444	107.31500210908\\
0.118055555555556	103.471971698015\\
0.119791666666667	101.271564053798\\
0.121527777777778	106.967721767307\\
0.123263888888889	111.540004646314\\
0.125	111.935195830405\\
0.126736111111111	107.675004115\\
0.128472222222222	98.8589788928844\\
0.130208333333333	99.6375817727883\\
0.131944444444444	108.09709600794\\
0.133680555555556	112.810006270045\\
0.135416666666667	114.578592127336\\
0.137152777777778	113.732572799186\\
0.138888888888889	109.394442105985\\
0.140625	100.587064474485\\
0.142361111111111	102.329962200651\\
0.144097222222222	111.986651078665\\
0.145833333333333	116.345847722821\\
0.147569444444444	115.73799741387\\
0.149305555555556	108.928744205746\\
0.151041666666667	98.0505933405695\\
0.152777777777778	113.79165391325\\
0.154513888888889	123.04519412388\\
0.15625	126.031831107942\\
0.157986111111111	123.152371001321\\
0.159722222222222	109.46034365623\\
0.161458333333333	111.342961286959\\
0.163194444444444	127.882069436613\\
0.164930555555556	133.838901063939\\
0.166666666666667	134.293377978783\\
0.168402777777778	129.826790707971\\
0.170138888888889	119.111601687833\\
0.171875	98.4469898373329\\
0.173611111111111	98.7015096116029\\
0.175347222222222	119.564383316968\\
0.177083333333333	133.627478739391\\
0.178819444444444	141.687582801754\\
0.180555555555556	145.228156078968\\
0.182291666666667	144.664647474725\\
0.184027777777778	139.441615848315\\
0.185763888888889	129.42659899085\\
0.1875	131.217122860578\\
0.189236111111111	138.041003152297\\
0.190972222222222	138.536579483092\\
0.192708333333333	131.392441967603\\
0.194444444444444	94.2660405449815\\
0.196180555555556	128.124829838592\\
0.197916666666667	138.621717314746\\
0.199652777777778	141.173995066351\\
0.201388888888889	138.851903061687\\
0.203125	131.447194298882\\
0.204861111111111	114.767675011527\\
0.206597222222222	103.618777006007\\
0.208333333333333	119.444146434492\\
0.210069444444444	123.137001376218\\
0.211805555555556	122.738732149049\\
0.213541666666667	120.236538326784\\
0.215277777777778	115.901659053053\\
0.217013888888889	108.778794690671\\
0.21875	95.5100687129682\\
0.220486111111111	72.6794289368937\\
0.222222222222222	99.776960630358\\
0.223958333333333	109.450791845398\\
0.225694444444444	113.677033355004\\
0.227430555555556	114.051738369587\\
0.229166666666667	110.990113595303\\
0.230902777777778	108.370533246484\\
0.232638888888889	114.168660435951\\
0.234375	120.893355401653\\
0.236111111111111	124.345229772143\\
0.237847222222222	124.287167255985\\
0.239583333333333	120.473700660018\\
0.241319444444444	113.951568828315\\
0.243055555555556	114.77977938203\\
0.244791666666667	120.53347126049\\
0.246527777777778	122.594450654554\\
0.248263888888889	120.333958636406\\
0.25	114.031835400912\\
0.251736111111111	108.139200957532\\
0.253472222222222	110.195860689434\\
0.255208333333333	112.152481764832\\
0.256944444444444	111.352982267791\\
0.258680555555556	109.307509118213\\
0.260416666666667	105.432144756189\\
0.262152777777778	93.4539435371554\\
0.263888888888889	91.4659559680069\\
0.265625	109.7207513506\\
0.267361111111111	116.184857878698\\
0.269097222222222	116.439686552454\\
0.270833333333333	110.44757670424\\
0.272569444444444	92.4766329717402\\
0.274305555555556	96.7053694471598\\
0.276041666666667	108.128056171947\\
0.277777777777778	110.046007433063\\
0.279513888888889	107.117701273167\\
0.28125	102.057865514316\\
0.282986111111111	97.4871622175674\\
0.284722222222222	91.3249110686645\\
0.286458333333333	87.2630028462499\\
0.288194444444444	96.1688724003939\\
0.289930555555556	101.148119706008\\
0.291666666666667	100.409033981041\\
0.293402777777778	95.2767733199028\\
0.295138888888889	92.5093053656185\\
0.296875	95.7657069264155\\
0.298611111111111	96.8904500694297\\
0.300347222222222	95.1490298779732\\
0.302083333333333	92.8611348436894\\
0.303819444444444	89.2308735536522\\
0.305555555555556	76.672309245058\\
0.307291666666667	70.9648248604453\\
0.309027777777778	90.4221174538562\\
0.310763888888889	94.9854396633932\\
0.3125	91.8058074544992\\
0.314236111111111	84.8538337799585\\
0.315972222222222	91.6101658707598\\
0.317708333333333	97.2957257742382\\
0.319444444444444	96.3277794695746\\
0.321180555555556	87.2927131995774\\
0.322916666666667	48.6567634500013\\
0.324652777777778	76.0220975443474\\
0.326388888888889	81.0726192480121\\
0.328125	87.0926255915462\\
0.329861111111111	97.3312594873937\\
0.331597222222222	102.953731661394\\
0.333333333333333	103.947533978059\\
0.335069444444444	101.72536161886\\
0.336805555555556	100.103035741606\\
0.338541666666667	101.328454129634\\
0.340277777777778	101.19823078807\\
0.342013888888889	99.6822564196618\\
0.34375	104.582660194072\\
0.345486111111111	111.738454558592\\
0.347222222222222	115.525059355661\\
0.348958333333333	115.432093830117\\
0.350694444444444	111.118520261359\\
0.352430555555556	100.945915888795\\
0.354166666666667	76.7038611328867\\
0.355902777777778	69.386507321964\\
0.357638888888889	76.685156102872\\
0.359375	95.4026788998661\\
0.361111111111111	107.831257617117\\
0.362847222222222	113.920890345547\\
0.364583333333333	115.604415025823\\
0.366319444444444	113.465516865135\\
0.368055555555556	107.404308478756\\
0.369791666666667	97.6725438640515\\
0.371527777777778	90.145319603883\\
0.373263888888889	87.85702708553\\
0.375	88.9493323476262\\
0.376736111111111	99.1233184213079\\
0.378472222222222	106.519602662132\\
0.380208333333333	109.361497419941\\
0.381944444444444	107.97808478597\\
0.383680555555556	101.816062938611\\
0.385416666666667	92.2240570567414\\
0.387152777777778	97.8499507316562\\
0.388888888888889	105.038057633433\\
0.390625	106.915372142983\\
0.392361111111111	103.730970375455\\
0.394097222222222	92.9976449147229\\
0.395833333333333	80.4980847011451\\
0.397569444444444	98.500249877505\\
0.399305555555556	105.131216205207\\
0.401041666666667	105.855860883494\\
0.402777777777778	101.624607007434\\
0.404513888888889	90.2518897838701\\
0.40625	75.0470280643189\\
0.407986111111111	92.7050118660602\\
0.409722222222222	100.098342457065\\
0.411458333333333	103.139138233202\\
0.413194444444444	104.293943925068\\
0.414930555555556	104.572763523483\\
0.416666666666667	103.978597735007\\
0.418402777777778	101.561697632978\\
0.420138888888889	95.2831642691523\\
0.421875	80.2428987335186\\
0.423611111111111	85.2510153847794\\
0.425347222222222	97.2338867314295\\
0.427083333333333	100.696173545326\\
0.428819444444444	98.1357493419189\\
0.430555555555556	88.9743728064136\\
0.432291666666667	87.6159357548471\\
0.434027777777778	97.7980236930052\\
0.435763888888889	101.7350014368\\
0.4375	100.420225525965\\
0.439236111111111	93.7577992021176\\
0.440972222222222	78.0166964574185\\
0.442708333333333	60.0733644028311\\
0.444444444444444	80.734599907371\\
0.446180555555556	88.315236808342\\
0.447916666666667	93.2185977716416\\
0.449652777777778	95.1994017758572\\
0.451388888888889	92.9241722587592\\
0.453125	84.7815004475717\\
0.454861111111111	82.371967473759\\
0.456597222222222	93.386626463067\\
0.458333333333333	98.8515783729325\\
0.460069444444444	99.6748576177026\\
0.461805555555556	97.1126327344473\\
0.463541666666667	93.261055730137\\
0.465277777777778	91.8321231319534\\
0.467013888888889	91.5891788202839\\
0.46875	88.6994772054879\\
0.470486111111111	85.4038101931879\\
0.472222222222222	90.6751329761304\\
0.473958333333333	96.5498632549698\\
0.475694444444444	98.3884322124829\\
0.477430555555556	95.9318563076568\\
0.479166666666667	88.8037640460727\\
0.480902777777778	81.9292590231391\\
0.482638888888889	87.9693383010236\\
0.484375	93.2844773788822\\
0.486111111111111	95.3818676044848\\
0.487847222222222	95.5504340740742\\
0.489583333333333	94.1778491761088\\
0.491319444444444	90.6996427166262\\
0.493055555555556	84.2977909750284\\
0.494791666666667	75.0293522456125\\
0.496527777777778	63.0040579869352\\
0.498263888888889	55.7423060408892\\
0.5	80.6389089062811\\
0.501736111111111	92.2175321248057\\
0.503472222222222	97.8046591394004\\
0.505208333333333	99.079716059507\\
0.506944444444444	96.1101948633418\\
0.508680555555556	87.1511191512281\\
0.510416666666667	62.6861223190267\\
0.512152777777778	80.2871876385094\\
0.513888888888889	90.5278990017208\\
0.515625	94.1167831531484\\
0.517361111111111	94.3060979516109\\
0.519097222222222	91.4549280648706\\
0.520833333333333	84.2470854793621\\
0.522569444444444	66.6222477904486\\
0.524305555555556	56.6852704679003\\
0.526041666666667	72.1065921084484\\
0.527777777777778	72.1358224776339\\
0.529513888888889	76.9079576534063\\
0.53125	87.4192309296156\\
0.532986111111111	93.2763354583693\\
0.534722222222222	94.3603817299315\\
0.536458333333333	90.4386547609073\\
0.538194444444444	78.3852456890257\\
0.539930555555556	60.4190387856098\\
0.541666666666667	82.3521017768617\\
0.543402777777778	89.7887702772737\\
0.545138888888889	93.842443096287\\
0.546875	97.1311326349748\\
0.548611111111111	99.1715774533344\\
0.550347222222222	98.718131871948\\
0.552083333333333	94.5619718184053\\
0.553819444444444	84.4979784122465\\
0.555555555555556	69.1269420805085\\
0.557291666666667	79.8201089935943\\
0.559027777777778	83.7352851736058\\
0.560763888888889	79.2487043617849\\
0.5625	62.3680107550452\\
0.564236111111111	65.5657470180412\\
0.565972222222222	78.0172711114495\\
0.567708333333333	82.5962145001111\\
0.569444444444444	85.1678468317865\\
0.571180555555556	86.2059676743654\\
0.572916666666667	83.416438376494\\
0.574652777777778	71.2214343392866\\
0.576388888888889	66.2271069622743\\
0.578125	86.072026025541\\
0.579861111111111	93.2173480712842\\
0.581597222222222	94.9996522565615\\
0.583333333333333	92.8251629700207\\
0.585069444444444	86.8827617405539\\
0.586805555555556	77.3390165485636\\
0.588541666666667	67.590321272645\\
0.590277777777778	60.3732603643948\\
0.592013888888889	56.5466554177746\\
0.59375	71.3088041602578\\
0.595486111111111	79.1255376445775\\
0.597222222222222	80.0736070096643\\
0.598958333333333	72.94079572121\\
0.600694444444444	59.9605307385236\\
0.602430555555556	78.8276705505832\\
0.604166666666667	86.2458577444683\\
0.605902777777778	85.9603839175002\\
0.607638888888889	76.1754767849571\\
0.609375	55.5042956461214\\
0.611111111111111	82.7012777464209\\
0.612847222222222	89.1601643779253\\
0.614583333333333	88.3212636424006\\
0.616319444444444	82.7090610373543\\
0.618055555555556	80.7602125107911\\
0.619791666666667	83.0986763919756\\
0.621527777777778	78.4309228709951\\
0.623263888888889	72.6884316762753\\
0.625	89.3612819015865\\
0.626736111111111	97.2559395915512\\
0.628472222222222	98.5537916903455\\
0.630208333333333	93.2213668300783\\
0.631944444444444	72.6391319067472\\
0.633680555555556	79.2510944535521\\
0.635416666666667	92.2656951575264\\
0.637152777777778	93.3464546843081\\
0.638888888888889	86.3109211515586\\
0.640625	78.8526305518046\\
0.642361111111111	90.0022938769532\\
0.644097222222222	95.1601994639866\\
0.645833333333333	94.6013674837996\\
0.647569444444444	89.8439723271007\\
0.649305555555556	83.6281254399417\\
0.651041666666667	79.8977286570208\\
0.652777777777778	74.3454684983274\\
0.654513888888889	62.9386726191769\\
0.65625	71.5108999465172\\
0.657986111111111	77.6598772076331\\
0.659722222222222	75.0904120424339\\
0.661458333333333	60.241258493634\\
0.663194444444444	50.2398379504649\\
0.664930555555556	65.575283824703\\
0.666666666666667	61.7172947530872\\
0.668402777777778	24.9518334117497\\
0.670138888888889	60.6167565704468\\
0.671875	65.7695658658945\\
0.673611111111111	63.9088075123886\\
0.675347222222222	72.8081633473505\\
0.677083333333333	80.5097600879856\\
0.678819444444444	81.9383377004225\\
0.680555555555556	77.9302463447318\\
0.682291666666667	77.2555686724385\\
0.684027777777778	85.1478388462812\\
0.685763888888889	89.7932374663137\\
0.6875	90.1501487019417\\
0.689236111111111	87.0480681966658\\
0.690972222222222	81.6057472759345\\
0.692708333333333	75.182865025223\\
0.694444444444444	70.1389462227247\\
0.696180555555556	74.3526065345709\\
0.697916666666667	80.155097159993\\
0.699652777777778	80.9180704025599\\
0.701388888888889	76.2746686342221\\
0.703125	75.9056242423589\\
0.704861111111111	83.7511017578218\\
0.706597222222222	86.8346534230503\\
0.708333333333333	83.9660951884025\\
0.710069444444444	80.9684875642588\\
0.711805555555556	89.5176165773546\\
0.713541666666667	96.2005854798249\\
0.715277777777778	98.1621828251794\\
0.717013888888889	95.7260856948412\\
0.71875	88.1822570794298\\
0.720486111111111	73.3628293004128\\
0.722222222222222	71.1473284809026\\
0.723958333333333	77.5406222113867\\
0.725694444444444	76.2142966630147\\
0.727430555555556	69.261536242402\\
0.729166666666667	59.8873431828867\\
0.730902777777778	61.7112301547465\\
0.732638888888889	72.0032063787215\\
0.734375	79.2104384594324\\
0.736111111111111	82.1031197626554\\
0.737847222222222	80.7051767376876\\
0.739583333333333	74.9351128294906\\
0.741319444444444	67.4617439354576\\
0.743055555555556	66.9157086596395\\
0.744791666666667	66.0412976860913\\
0.746527777777778	61.1085404998901\\
0.748263888888889	68.4228146338827\\
0.75	76.9907608624822\\
0.751736111111111	80.1174324455292\\
0.753472222222222	78.3046414081252\\
0.755208333333333	71.1514417933051\\
0.756944444444444	65.1632211954121\\
0.758680555555556	73.6564034792638\\
0.760416666666667	78.7975242979664\\
0.762152777777778	78.7115325856567\\
0.763888888888889	74.0048918574933\\
0.765625	69.3248474074702\\
0.767361111111111	73.3297022349802\\
0.769097222222222	75.8237322045822\\
0.770833333333333	71.4893778803248\\
0.772569444444444	55.6150058743299\\
0.774305555555556	67.948231705152\\
0.776041666666667	77.9777596182824\\
0.777777777777778	79.444756952014\\
0.779513888888889	73.8565612343044\\
0.78125	63.4522652722089\\
0.782986111111111	72.1834358758904\\
0.784722222222222	78.3901906508395\\
0.786458333333333	78.0563456837503\\
0.788194444444444	72.5236416782629\\
0.789930555555556	64.7375696066335\\
0.791666666666667	58.7862721471522\\
0.793402777777778	32.1466551384892\\
0.795138888888889	65.9638548221257\\
0.796875	79.9474340893657\\
0.798611111111111	85.6589110460018\\
0.800347222222222	86.3773453155245\\
0.802083333333333	83.2365559632725\\
0.803819444444444	79.7891796080601\\
0.805555555555556	82.2725191439689\\
0.807291666666667	85.7953950891063\\
0.809027777777778	86.5281394124206\\
0.810763888888889	84.5673992097145\\
0.8125	80.7748756363307\\
0.814236111111111	75.6830203025841\\
0.815972222222222	67.8788472145802\\
0.817708333333333	51.9504892287614\\
0.819444444444444	49.6467709074351\\
0.821180555555556	62.3193704738494\\
0.822916666666667	68.30170391545\\
0.824652777777778	74.2637559313553\\
0.826388888888889	79.3936054475115\\
0.828125	81.5488028225\\
0.829861111111111	79.4510182603316\\
0.831597222222222	70.0074954665546\\
0.833333333333333	25.3220511449284\\
0.835069444444444	73.3558719945592\\
0.836805555555556	82.5928204774657\\
0.838541666666667	84.8255853950306\\
0.840277777777778	81.649780361361\\
0.842013888888889	71.2235917128984\\
0.84375	64.8232052124182\\
0.845486111111111	76.6970922035977\\
0.847222222222222	79.8103701028832\\
0.848958333333333	76.3893373609393\\
0.850694444444444	76.0632058555955\\
0.852430555555556	84.5043972722795\\
0.854166666666667	88.9413657765602\\
0.855902777777778	87.447135289381\\
0.857638888888889	77.685666371019\\
0.859375	73.9069763845588\\
0.861111111111111	88.6459153612585\\
0.862847222222222	94.4052546944778\\
0.864583333333333	94.4819329985031\\
0.866319444444444	89.6113166803636\\
0.868055555555556	81.3628378376814\\
0.869791666666667	82.157643509553\\
0.871527777777778	86.3979892738523\\
0.873263888888889	85.3833047530452\\
0.875	78.4417794500417\\
0.876736111111111	73.4788950995947\\
0.878472222222222	81.5246462975824\\
0.880208333333333	85.9509691041779\\
0.881944444444444	85.4678717473144\\
0.883680555555556	79.9972240668504\\
0.885416666666667	66.5366490709038\\
0.887152777777778	65.7020481203978\\
0.888888888888889	79.2854605020708\\
0.890625	84.8307138881177\\
0.892361111111111	85.8145972749062\\
0.894097222222222	83.4188521673275\\
0.895833333333333	79.6172558473633\\
0.897569444444444	79.2022636729145\\
0.899305555555556	81.1589484335572\\
0.901041666666667	80.472019663996\\
0.902777777777778	74.3192016708202\\
0.904513888888889	53.1866589144205\\
0.90625	62.417996191153\\
0.907986111111111	76.7837501088163\\
0.909722222222222	81.5624357645856\\
0.911458333333333	81.9942124500894\\
0.913194444444444	79.2752381851081\\
0.914930555555556	74.1275404422395\\
0.916666666666667	69.6591336000207\\
0.918402777777778	72.077260676746\\
0.920138888888889	77.7328392587231\\
0.921875	82.0809001921081\\
0.923611111111111	84.032527781582\\
0.925347222222222	82.7538887852624\\
0.927083333333333	76.0667746477134\\
0.928819444444444	49.1937093238558\\
0.930555555555556	71.9069837416938\\
0.932291666666667	85.4344922245655\\
0.934027777777778	90.8939389243572\\
0.935763888888889	92.0793006763513\\
0.9375	89.7960973664361\\
0.939236111111111	84.9890454319102\\
0.940972222222222	82.7600855305042\\
0.942708333333333	85.7093612442623\\
0.944444444444444	87.1723965731343\\
0.946180555555556	84.4184957785129\\
0.947916666666667	76.1687159331623\\
0.949652777777778	63.4078422920839\\
0.951388888888889	63.3641660616117\\
0.953125	62.5257322426344\\
0.954861111111111	71.2296152184941\\
0.956597222222222	83.0620434941229\\
0.958333333333333	88.6185544132283\\
0.960069444444444	88.7006581519333\\
0.961805555555556	82.6826582576182\\
0.963541666666667	68.3994696678691\\
0.965277777777778	74.1749220189928\\
0.967013888888889	81.3197493197072\\
0.96875	80.4037216211075\\
0.970486111111111	76.5119687448267\\
0.972222222222222	79.9795086362025\\
0.973958333333333	83.5650556525401\\
0.975694444444444	81.5592625027476\\
0.977430555555556	77.4417044680171\\
0.979166666666667	84.1721990616269\\
0.980902777777778	90.4678226033272\\
0.982638888888889	91.6125844605194\\
0.984375	87.9914910930404\\
0.986111111111111	82.6038602661371\\
0.987847222222222	84.6693232266961\\
0.989583333333333	88.0665016843963\\
0.991319444444444	86.9915197298748\\
0.993055555555556	80.4487008793162\\
0.994791666666667	66.4835800412787\\
0.996527777777778	53.2579482943771\\
0.998263888888889	56.4634162425436\\
};
\addlegendentry{Mean Centred};

\addplot [color=green,solid]
  table[row sep=crcr]{-1	39.3143869093761\\
-0.998263888888889	53.7330111825301\\
-0.996527777777778	53.25804784872\\
-0.994791666666667	62.7819929487378\\
-0.993055555555556	78.7577321106242\\
-0.991319444444444	86.2440939719655\\
-0.989583333333333	88.0664404730842\\
-0.987847222222222	85.7965706508487\\
-0.986111111111111	84.441566305154\\
-0.984375	88.6002084086833\\
-0.982638888888889	91.612618371193\\
-0.980902777777778	90.4163698795876\\
-0.979166666666667	84.7321120066984\\
-0.977430555555556	79.0998481134285\\
-0.975694444444444	81.559540023773\\
-0.973958333333333	82.2708922608027\\
-0.972222222222222	77.4810015597658\\
-0.970486111111111	74.7126252578609\\
-0.96875	80.4035616138141\\
-0.967013888888889	81.7613716799099\\
-0.965277777777778	75.7730463395469\\
-0.963541666666667	71.3175616131803\\
-0.961805555555556	82.6829077660444\\
-0.960069444444444	88.2677576512679\\
-0.958333333333333	88.1773990694706\\
-0.956597222222222	82.8785564393155\\
-0.954861111111111	71.2298910525738\\
-0.953125	57.2298468139773\\
-0.951388888888889	60.0631653069528\\
-0.949652777777778	65.8413416029305\\
-0.947916666666667	76.1695443320844\\
-0.946180555555556	83.079672581972\\
-0.944444444444444	85.6130552250308\\
-0.942708333333333	84.6590565051107\\
-0.940972222222222	82.7597538326671\\
-0.939236111111111	84.9358617967034\\
-0.9375	89.3363427846342\\
-0.935763888888889	91.7295392067198\\
-0.934027777777778	90.8937579211417\\
-0.932291666666667	85.8308930128101\\
-0.930555555555556	72.718037428497\\
-0.928819444444444	47.5986556733082\\
-0.927083333333333	76.0668127066092\\
-0.925347222222222	82.4495933001562\\
-0.923611111111111	83.2478397383163\\
-0.921875	81.1545205528295\\
-0.920138888888889	77.731943158473\\
-0.918402777777778	74.0568032591551\\
-0.916666666666667	72.1832869427482\\
-0.914930555555556	75.0816922432785\\
-0.913194444444444	79.2756965676402\\
-0.911458333333333	81.2563633827104\\
-0.909722222222222	80.1558541980325\\
-0.907986111111111	75.1497556147862\\
-0.90625	62.4148878394419\\
-0.904513888888889	46.2848422290628\\
-0.902777777777778	70.8012222036852\\
-0.901041666666667	78.7443885788803\\
-0.899305555555556	81.1577543480851\\
-0.897569444444444	80.5024244155214\\
-0.895833333333333	80.2503068505251\\
-0.894097222222222	83.1213647840344\\
-0.892361111111111	85.8141648911918\\
-0.890625	85.4227138109221\\
-0.888888888888889	80.2078116004468\\
-0.887152777777778	65.3353786753741\\
-0.885416666666667	66.5336981044774\\
-0.883680555555556	80.9104955745004\\
-0.881944444444444	86.0010224745062\\
-0.880208333333333	86.0004550596399\\
-0.878472222222222	81.5241288598541\\
-0.876736111111111	75.6628284298569\\
-0.875	80.2817494160497\\
-0.873263888888889	85.7510443147169\\
-0.871527777777778	86.3978859223582\\
-0.869791666666667	83.0564913267616\\
-0.868055555555556	83.4539211498651\\
-0.866319444444444	90.3071106764894\\
-0.864583333333333	94.4823038418113\\
-0.862847222222222	94.1969905763194\\
-0.861111111111111	88.5727744789498\\
-0.859375	75.2078375933967\\
-0.857638888888889	77.6874695167755\\
-0.855902777777778	86.8095138294995\\
-0.854166666666667	88.1859927765113\\
-0.852430555555556	83.6681710013033\\
-0.850694444444444	76.0608305498195\\
-0.848958333333333	78.1581605873364\\
-0.847222222222222	81.4877422858007\\
-0.845486111111111	78.1897554116019\\
-0.84375	64.828238060448\\
-0.842013888888889	70.8875502177986\\
-0.840277777777778	82.2186718513484\\
-0.838541666666667	85.3429902789392\\
-0.836805555555556	82.5937503678553\\
-0.835069444444444	71.9701920880971\\
-0.833333333333333	42.5121814332675\\
-0.831597222222222	72.154370569686\\
-0.829861111111111	79.4527838799703\\
-0.828125	80.3855471828735\\
-0.826388888888889	77.7205179034095\\
-0.824652777777778	73.1931318472517\\
-0.822916666666667	68.3012122639712\\
-0.821180555555556	61.3588795305778\\
-0.819444444444444	41.7149766445959\\
-0.817708333333333	50.5815390735805\\
-0.815972222222222	67.877900155793\\
-0.814236111111111	76.6027712421837\\
-0.8125	82.3117713401813\\
-0.810763888888889	85.7003790528681\\
-0.809027777777778	86.5299185979157\\
-0.807291666666667	84.5058404151433\\
-0.805555555555556	80.4011809399113\\
-0.803819444444444	79.4543290828221\\
-0.802083333333333	83.2375704026441\\
-0.800347222222222	85.4759975657547\\
-0.798611111111111	84.061184206096\\
-0.796875	78.2398473515407\\
-0.795138888888889	65.9576526893173\\
-0.793402777777778	37.2328661295283\\
-0.791666666666667	51.5615569893087\\
-0.789930555555556	63.9505542646417\\
-0.788194444444444	72.5252076900803\\
-0.786458333333333	76.7975807517541\\
-0.784722222222222	76.0713418954348\\
-0.782986111111111	69.2725853350028\\
-0.78125	63.4471533086645\\
-0.779513888888889	73.2874926570751\\
-0.777777777777778	78.1218830955924\\
-0.776041666666667	76.6270181490801\\
-0.774305555555556	67.9426348183687\\
-0.772569444444444	59.3148724539433\\
-0.770833333333333	70.3776444297811\\
-0.769097222222222	74.5624762757839\\
-0.767361111111111	73.3257223457092\\
-0.765625	72.3399373873547\\
-0.763888888888889	76.4197157254105\\
-0.762152777777778	79.5923217481667\\
-0.760416666666667	78.7985817036136\\
-0.758680555555556	73.6588065708677\\
-0.756944444444444	68.8694629847733\\
-0.755208333333333	73.9185026739318\\
-0.753472222222222	78.3084475799163\\
-0.751736111111111	78.3815754129291\\
-0.75	74.0884345166904\\
-0.748263888888889	65.7444964213163\\
-0.746527777777778	61.1087616129032\\
-0.744791666666667	64.2080408874646\\
-0.743055555555556	65.6553134076959\\
-0.741319444444444	68.9523480364982\\
-0.739583333333333	74.9393841272023\\
-0.737847222222222	78.9828737943815\\
-0.736111111111111	79.7263809065993\\
-0.734375	77.2200743904344\\
-0.732638888888889	71.9969264861293\\
-0.730902777777778	66.287759931392\\
-0.729166666666667	66.1554928748349\\
-0.727430555555556	71.8629021520653\\
-0.725694444444444	76.2185857770722\\
-0.723958333333333	75.5423668673403\\
-0.722222222222222	66.1602756494151\\
-0.720486111111111	73.1858921930426\\
-0.71875	88.1831831284828\\
-0.717013888888889	95.4151931385753\\
-0.715277777777778	97.7994265352287\\
-0.713541666666667	95.987019857092\\
-0.711805555555556	89.5175548901514\\
-0.710069444444444	79.7305464729472\\
-0.708333333333333	81.7716503302567\\
-0.706597222222222	85.7842661588102\\
-0.704861111111111	83.7489976465291\\
-0.703125	75.9087265778338\\
-0.701388888888889	73.1176957343236\\
-0.699652777777778	79.0448383832886\\
-0.697916666666667	80.1506769496994\\
-0.696180555555556	76.3150925069696\\
-0.694444444444444	71.7024288329108\\
-0.692708333333333	75.0136216670829\\
-0.690972222222222	81.6054884829079\\
-0.689236111111111	86.9588683189028\\
-0.6875	89.9109573124808\\
-0.685763888888889	89.6124240383872\\
-0.684027777777778	85.1476991295114\\
-0.682291666666667	76.2300613162241\\
-0.680555555555556	74.6557023292504\\
-0.678819444444444	80.1794425105286\\
-0.677083333333333	80.5048356182711\\
-0.675347222222222	75.4792167906643\\
-0.673611111111111	68.3247631485248\\
-0.671875	66.0669990450634\\
-0.670138888888889	60.6126309071814\\
-0.668402777777778	33.7753938656417\\
-0.666666666666667	58.3127011065496\\
-0.664930555555556	63.2408932919946\\
-0.663194444444444	50.2171223067764\\
-0.661458333333333	59.2223165061944\\
-0.659722222222222	73.8828288273285\\
-0.657986111111111	76.9238592186935\\
-0.65625	71.5092830531192\\
-0.654513888888889	58.6649234528496\\
-0.652777777777778	70.0344695887315\\
-0.651041666666667	77.8180918492008\\
-0.649305555555556	83.6241675222379\\
-0.647569444444444	90.3967172643333\\
-0.645833333333333	94.961056007618\\
-0.644097222222222	95.3567022455741\\
-0.642361111111111	90.0034598234011\\
-0.640625	76.8822444234654\\
-0.638888888888889	84.8485300183152\\
-0.637152777777778	92.8763926516385\\
-0.635416666666667	92.2643429249787\\
-0.633680555555556	80.0947608850271\\
-0.631944444444444	70.8576935787819\\
-0.630208333333333	92.8230338144331\\
-0.628472222222222	98.5527692626165\\
-0.626736111111111	97.6439861520073\\
-0.625	90.4601342160714\\
-0.623263888888889	75.5149503610029\\
-0.621527777777778	78.4316925255381\\
-0.619791666666667	83.7688403572418\\
-0.618055555555556	82.9354059782976\\
-0.616319444444444	84.3145057715072\\
-0.614583333333333	88.3244418098028\\
-0.612847222222222	88.422352090244\\
-0.611111111111111	81.2588219094439\\
-0.609375	46.494426378393\\
-0.607638888888889	76.1786977075606\\
-0.605902777777778	85.2982508075033\\
-0.604166666666667	85.1432349000878\\
-0.602430555555556	77.2611702415062\\
-0.600694444444444	59.9376935558448\\
-0.598958333333333	72.8056396254738\\
-0.597222222222222	79.281808827796\\
-0.595486111111111	78.4223334172975\\
-0.59375	71.3059314379268\\
-0.592013888888889	53.4540161723872\\
-0.590277777777778	50.8121301177573\\
-0.588541666666667	66.858399550724\\
-0.586805555555556	77.3424877818344\\
-0.585069444444444	85.9648540346671\\
-0.583333333333333	91.7473208981271\\
-0.581597222222222	94.2862549477712\\
-0.579861111111111	93.2142548467695\\
-0.578125	87.3291801508287\\
-0.576388888888889	71.8104120186904\\
-0.574652777777778	69.314985171905\\
-0.572916666666667	83.4109692940506\\
-0.571180555555556	87.5257721358999\\
-0.569444444444444	87.2682913619894\\
-0.567708333333333	84.3184850383372\\
-0.565972222222222	78.0257620699817\\
-0.564236111111111	60.5901680121263\\
-0.5625	63.5456587849032\\
-0.560763888888889	79.8928864765902\\
-0.559027777777778	83.7368895902058\\
-0.557291666666667	79.8086073846457\\
-0.555555555555556	73.4907437927302\\
-0.553819444444444	85.973290477711\\
-0.552083333333333	94.5653166451017\\
-0.550347222222222	98.052576119394\\
-0.548611111111111	98.1689825130915\\
-0.546875	96.2877600132176\\
-0.545138888888889	93.8383934333513\\
-0.543402777777778	90.7664245859495\\
-0.541666666666667	84.0954746422237\\
-0.539930555555556	65.8849026096883\\
-0.538194444444444	78.3870140597998\\
-0.536458333333333	90.3605077364261\\
-0.534722222222222	94.2904283637427\\
-0.532986111111111	93.2624329863472\\
-0.53125	87.4199639015277\\
-0.529513888888889	75.4880814909726\\
-0.527777777777778	66.193840497023\\
-0.526041666666667	68.8998359834065\\
-0.524305555555556	56.6546562348778\\
-0.522569444444444	64.8175919886811\\
-0.520833333333333	83.4013329778391\\
-0.519097222222222	91.1965631656826\\
-0.517361111111111	94.3057520714442\\
-0.515625	94.0354599134308\\
-0.513888888888889	90.1505488572247\\
-0.512152777777778	79.7823026978996\\
-0.510416666666667	62.7157197688638\\
-0.508680555555556	85.8066276203322\\
-0.506944444444444	94.9836901417199\\
-0.505208333333333	98.3993012208387\\
-0.503472222222222	97.8011063365458\\
-0.501736111111111	93.212164400904\\
-0.5	83.3224976653944\\
-0.498263888888889	63.3760679384775\\
-0.496527777777778	63.0051481471648\\
-0.494791666666667	74.2813745278125\\
-0.493055555555556	82.9179756327449\\
-0.491319444444444	89.9219151023431\\
-0.489583333333333	94.1749197599843\\
-0.487847222222222	95.9577846580952\\
-0.486111111111111	95.7452777978059\\
-0.484375	93.4469480892341\\
-0.482638888888889	87.9710156380638\\
-0.480902777777778	80.2206084454586\\
-0.479166666666667	86.609266349177\\
-0.477430555555556	95.0099765381672\\
-0.475694444444444	98.3845779752275\\
-0.473958333333333	97.2586041611805\\
-0.472222222222222	91.6261538437413\\
-0.470486111111111	84.952722178589\\
-0.46875	88.6922651111896\\
-0.467013888888889	92.8187511984097\\
-0.465277777777778	93.3793883404122\\
-0.463541666666667	93.8675595424979\\
-0.461805555555556	97.1130088383707\\
-0.460069444444444	99.8364088010922\\
-0.458333333333333	99.2802122460265\\
-0.456597222222222	93.9918905340067\\
-0.454861111111111	82.3831780571625\\
-0.453125	83.2851536514538\\
-0.451388888888889	92.0235125181571\\
-0.449652777777778	94.5838276526481\\
-0.447916666666667	93.2134246887283\\
-0.446180555555556	89.7550834609522\\
-0.444444444444444	83.9211787106787\\
-0.442708333333333	65.1933331251901\\
-0.440972222222222	78.0004884113284\\
-0.439236111111111	94.6974518887104\\
-0.4375	101.261648777521\\
-0.435763888888889	102.280519539306\\
-0.434027777777778	97.8025335823473\\
-0.432291666666667	85.8470655202807\\
-0.430555555555556	88.0499355025432\\
-0.428819444444444	98.1361522989074\\
-0.427083333333333	100.696797236145\\
-0.425347222222222	97.0162586538961\\
-0.423611111111111	83.8185874103999\\
-0.421875	78.225759165861\\
-0.420138888888889	95.2807480496021\\
-0.418402777777778	101.801311002953\\
-0.416666666666667	104.318421656918\\
-0.414930555555556	104.856999913713\\
-0.413194444444444	104.296156018197\\
-0.411458333333333	102.73339678962\\
-0.409722222222222	99.3933697579163\\
-0.407986111111111	92.0449163230458\\
-0.40625	75.0539737703488\\
-0.404513888888889	88.6860500009231\\
-0.402777777777778	100.547160884694\\
-0.401041666666667	105.260352722557\\
-0.399305555555556	105.126954915254\\
-0.397569444444444	99.439392844854\\
-0.395833333333333	83.5060654258777\\
-0.394097222222222	91.9654186354563\\
-0.392361111111111	103.726218844285\\
-0.390625	107.511116527849\\
-0.388888888888889	105.979473576891\\
-0.387152777777778	98.7094318744531\\
-0.385416666666667	92.2218504542109\\
-0.383680555555556	102.558206214134\\
-0.381944444444444	108.792420918496\\
-0.380208333333333	109.892210819002\\
-0.378472222222222	106.523761933946\\
-0.376736111111111	98.5150916154645\\
-0.375	89.6123319540869\\
-0.373263888888889	89.6653450212567\\
-0.371527777777778	90.1536806232067\\
-0.369791666666667	97.9713498996658\\
-0.368055555555556	108.024923167942\\
-0.366319444444444	113.842359583074\\
-0.364583333333333	115.607131516352\\
-0.362847222222222	113.517568960059\\
-0.361111111111111	107.020789136189\\
-0.359375	94.3672846194682\\
-0.357638888888889	76.6861957680561\\
-0.355902777777778	63.5662763447049\\
-0.354166666666667	81.257817656033\\
-0.352430555555556	101.862361964192\\
-0.350694444444444	111.122727942265\\
-0.348958333333333	115.016799015749\\
-0.347222222222222	114.935403706306\\
-0.345486111111111	111.23028722602\\
-0.34375	104.577364864482\\
-0.342013888888889	100.021863099844\\
-0.340277777777778	101.081327760572\\
-0.338541666666667	101.263725689987\\
-0.336805555555556	100.105558997395\\
-0.335069444444444	101.050532110714\\
-0.333333333333333	102.869894598332\\
-0.331597222222222	102.104210047945\\
-0.329861111111111	97.3217715138234\\
-0.328125	88.4696825197645\\
-0.326388888888889	80.5531312575741\\
-0.324652777777778	73.2391877647431\\
-0.322916666666667	48.7987624898553\\
-0.321180555555556	85.3056764515798\\
-0.319444444444444	94.5442596124138\\
-0.317708333333333	96.0291561583578\\
-0.315972222222222	91.5938715615236\\
-0.314236111111111	86.4576284147712\\
-0.3125	91.5052947565828\\
-0.310763888888889	94.3951953285023\\
-0.309027777777778	90.412812585286\\
-0.307291666666667	75.0995432167409\\
-0.305555555555556	77.5613048275534\\
-0.303819444444444	89.4012441300853\\
-0.302083333333333	92.8664675073321\\
-0.300347222222222	94.159953646771\\
-0.298611111111111	95.2342378984314\\
-0.296875	94.5347683142076\\
-0.295138888888889	92.5009281418873\\
-0.293402777777778	95.0122038509424\\
-0.291666666666667	99.5592135427064\\
-0.289930555555556	100.454177049175\\
-0.288194444444444	96.1605382657542\\
-0.286458333333333	87.4201272831718\\
-0.284722222222222	89.3499481387288\\
-0.282986111111111	96.522862016512\\
-0.28125	102.053917207751\\
-0.279513888888889	107.106070055899\\
-0.277777777777778	109.804689801781\\
-0.276041666666667	107.857244968099\\
-0.274305555555556	96.7019280945807\\
-0.272569444444444	90.4573152918644\\
-0.270833333333333	109.592351237191\\
-0.269097222222222	116.014238208834\\
-0.267361111111111	116.180077914518\\
-0.265625	110.348339404111\\
-0.263888888888889	93.5842898113147\\
-0.262152777777778	91.6353344178483\\
-0.260416666666667	105.422373249529\\
-0.258680555555556	109.821385391665\\
-0.256944444444444	111.604466380281\\
-0.255208333333333	111.993458473418\\
-0.253472222222222	110.189530584\\
-0.251736111111111	108.967183964019\\
-0.25	114.571524526459\\
-0.248263888888889	120.457922911087\\
-0.246527777777778	122.595074972077\\
-0.244791666666667	120.465472237414\\
-0.243055555555556	114.363702713016\\
-0.241319444444444	113.315729351921\\
-0.239583333333333	120.468357962665\\
-0.237847222222222	124.611738796005\\
-0.236111111111111	124.793963855339\\
-0.234375	121.267437109463\\
-0.232638888888889	114.174412682503\\
-0.230902777777778	108.288832771884\\
-0.229166666666667	111.560048057229\\
-0.227430555555556	114.527579386097\\
-0.225694444444444	113.684249338563\\
-0.223958333333333	108.690759255263\\
-0.222222222222222	97.6121105812024\\
-0.220486111111111	65.1106056324224\\
-0.21875	95.5338971214962\\
-0.217013888888889	107.89198667438\\
-0.215277777777778	115.068923146988\\
-0.213541666666667	119.811104410406\\
-0.211805555555556	122.734397245385\\
-0.210069444444444	123.355602066853\\
-0.208333333333333	119.664387716836\\
-0.206597222222222	103.29166976036\\
-0.204861111111111	114.762411404835\\
-0.203125	131.509235043568\\
-0.201388888888889	138.876507103473\\
-0.199652777777778	141.173546722155\\
-0.197916666666667	138.621458271862\\
-0.196180555555556	128.188885730735\\
-0.194444444444444	94.5588023328814\\
-0.192708333333333	131.310849323299\\
-0.190972222222222	138.535339841118\\
-0.189236111111111	138.155380201775\\
-0.1875	131.590671709959\\
-0.185763888888889	129.685746996362\\
-0.184027777777778	139.442232897053\\
-0.182291666666667	144.672391616491\\
-0.180555555555556	145.268115475889\\
-0.178819444444444	141.744011023249\\
-0.177083333333333	133.629636083847\\
-0.175347222222222	119.167147226044\\
-0.173611111111111	95.9542125048232\\
-0.171875	97.8230401081593\\
-0.170138888888889	119.108569755243\\
-0.168402777777778	129.954330069673\\
-0.166666666666667	134.473939089836\\
-0.164930555555556	134.001221782452\\
-0.163194444444444	127.887261888472\\
-0.161458333333333	110.3495301495\\
-0.159722222222222	109.919083276159\\
-0.157986111111111	123.495564561703\\
-0.15625	126.038631199881\\
-0.154513888888889	122.594063071509\\
-0.152777777777778	112.693013717125\\
-0.151041666666667	98.1390680901422\\
-0.149305555555556	108.951891114736\\
-0.147569444444444	114.944072230435\\
-0.145833333333333	115.284427260449\\
-0.144097222222222	111.020195596097\\
-0.142361111111111	102.288085757744\\
-0.140625	102.440942331738\\
-0.138888888888889	110.679276772324\\
-0.137152777777778	114.53012997817\\
-0.135416666666667	114.599000526693\\
-0.133680555555556	111.66953819281\\
-0.131944444444444	105.748745089231\\
-0.130208333333333	97.1040771767703\\
-0.128472222222222	98.8443988524569\\
-0.126736111111111	107.494680816912\\
-0.125	111.643409629777\\
-0.123263888888889	111.280448647981\\
-0.121527777777778	106.951671621551\\
-0.119791666666667	103.12283040667\\
-0.118055555555556	106.454008057434\\
-0.116319444444444	109.14353816689\\
-0.114583333333333	107.658368812077\\
-0.112847222222222	103.619270759463\\
-0.111111111111111	104.594660537478\\
-0.109375	109.149074753357\\
-0.107638888888889	110.988771821605\\
-0.105902777777778	109.429777332212\\
-0.104166666666667	104.776429214348\\
-0.102430555555556	96.1618689079501\\
-0.100694444444444	80.6347058013019\\
-0.0989583333333334	99.0314216716251\\
-0.0972222222222222	111.589705545742\\
-0.0954861111111112	117.809009609627\\
-0.09375	119.729392654012\\
-0.0920138888888888	117.535222038814\\
-0.0902777777777778	110.071316975322\\
-0.0885416666666666	96.2548622570579\\
-0.0868055555555556	102.382892340042\\
-0.0850694444444444	109.021913046788\\
-0.0833333333333334	108.584689337168\\
-0.0815972222222222	101.800220405479\\
-0.0798611111111112	84.8040782738826\\
-0.078125	83.3909205346635\\
-0.0763888888888888	101.065270955494\\
-0.0746527777777778	111.287867941862\\
-0.0729166666666666	117.447769048441\\
-0.0711805555555556	119.800825521134\\
-0.0694444444444444	118.072921359189\\
-0.0677083333333334	110.923947338704\\
-0.0659722222222222	93.2755457873829\\
-0.0642361111111112	94.9415275386324\\
-0.0625	103.09803002206\\
-0.0607638888888888	98.5539365537765\\
-0.0590277777777778	65.9630060797958\\
-0.0572916666666666	100.220900187234\\
-0.0555555555555556	109.667346308347\\
-0.0538194444444444	110.724006615653\\
-0.0520833333333334	104.777406174927\\
-0.0503472222222222	93.2578598588171\\
-0.0486111111111112	105.927633563884\\
-0.046875	115.106157105967\\
-0.0451388888888888	118.80480977031\\
-0.0434027777777778	119.16056370532\\
-0.0416666666666666	118.523571887236\\
-0.0399305555555556	120.837873611117\\
-0.0381944444444444	125.341299187077\\
-0.0364583333333334	128.378316739118\\
-0.0347222222222222	128.611401414018\\
-0.0329861111111112	125.576970686067\\
-0.03125	118.701344233358\\
-0.0295138888888888	108.393401287113\\
-0.0277777777777778	107.478141608544\\
-0.0260416666666666	116.671655139023\\
-0.0243055555555556	124.944849805677\\
-0.0225694444444444	130.886291603309\\
-0.0208333333333334	133.943092693634\\
-0.0190972222222222	133.732849552086\\
-0.0173611111111112	129.413762926904\\
-0.015625	117.943850194\\
-0.0138888888888888	98.4434950464224\\
-0.0121527777777778	119.565004411865\\
-0.0104166666666666	126.081692834887\\
-0.00868055555555558	126.490316523092\\
-0.00694444444444442	122.614148904159\\
-0.00520833333333337	114.506313809744\\
-0.00347222222222221	100.905940082347\\
-0.00173611111111116	76.575215599895\\
0	-480.341228392235\\
0.00173611111111116	76.575215599895\\
0.00347222222222232	100.905940082347\\
0.00520833333333326	114.506313809744\\
0.00694444444444442	122.614148904159\\
0.00868055555555558	126.490316523092\\
0.0104166666666667	126.081692834887\\
0.0121527777777777	119.565004411865\\
0.0138888888888888	98.4434950464224\\
0.015625	117.943850194\\
0.0173611111111112	129.413762926904\\
0.0190972222222223	133.732849552086\\
0.0208333333333333	133.943092693634\\
0.0225694444444444	130.886291603309\\
0.0243055555555556	124.944849805677\\
0.0260416666666667	116.671655139023\\
0.0277777777777777	107.478141608544\\
0.0295138888888888	108.393401287113\\
0.03125	118.701344233358\\
0.0329861111111112	125.576970686067\\
0.0347222222222223	128.611401414018\\
0.0364583333333333	128.378316739118\\
0.0381944444444444	125.341299187077\\
0.0399305555555556	120.837873611117\\
0.0416666666666667	118.523571887236\\
0.0434027777777777	119.16056370532\\
0.0451388888888888	118.80480977031\\
0.046875	115.106157105967\\
0.0486111111111112	105.927633563884\\
0.0503472222222223	93.2578598588171\\
0.0520833333333333	104.777406174927\\
0.0538194444444444	110.724006615653\\
0.0555555555555556	109.667346308347\\
0.0572916666666667	100.220900187234\\
0.0590277777777777	65.9630060797958\\
0.0607638888888888	98.5539365537765\\
0.0625	103.09803002206\\
0.0642361111111112	94.9415275386324\\
0.0659722222222223	93.2755457873829\\
0.0677083333333333	110.923947338704\\
0.0694444444444444	118.072921359189\\
0.0711805555555556	119.800825521134\\
0.0729166666666667	117.447769048441\\
0.0746527777777777	111.287867941862\\
0.0763888888888888	101.065270955494\\
0.078125	83.3909205346635\\
0.0798611111111112	84.8040782738826\\
0.0815972222222223	101.800220405479\\
0.0833333333333333	108.584689337168\\
0.0850694444444444	109.021913046788\\
0.0868055555555556	102.382892340042\\
0.0885416666666667	96.2548622570579\\
0.0902777777777777	110.071316975322\\
0.0920138888888888	117.535222038814\\
0.09375	119.729392654012\\
0.0954861111111112	117.809009609627\\
0.0972222222222223	111.589705545742\\
0.0989583333333333	99.0314216716251\\
0.100694444444444	80.6347058013019\\
0.102430555555556	96.1618689079501\\
0.104166666666667	104.776429214348\\
0.105902777777778	109.429777332212\\
0.107638888888889	110.988771821605\\
0.109375	109.149074753357\\
0.111111111111111	104.594660537478\\
0.112847222222222	103.619270759463\\
0.114583333333333	107.658368812077\\
0.116319444444444	109.14353816689\\
0.118055555555556	106.454008057434\\
0.119791666666667	103.12283040667\\
0.121527777777778	106.951671621551\\
0.123263888888889	111.280448647981\\
0.125	111.643409629777\\
0.126736111111111	107.494680816912\\
0.128472222222222	98.8443988524569\\
0.130208333333333	97.1040771767703\\
0.131944444444444	105.748745089231\\
0.133680555555556	111.66953819281\\
0.135416666666667	114.599000526693\\
0.137152777777778	114.53012997817\\
0.138888888888889	110.679276772324\\
0.140625	102.440942331738\\
0.142361111111111	102.288085757744\\
0.144097222222222	111.020195596097\\
0.145833333333333	115.284427260449\\
0.147569444444444	114.944072230435\\
0.149305555555556	108.951891114736\\
0.151041666666667	98.1390680901422\\
0.152777777777778	112.693013717125\\
0.154513888888889	122.594063071509\\
0.15625	126.038631199881\\
0.157986111111111	123.495564561703\\
0.159722222222222	109.919083276159\\
0.161458333333333	110.3495301495\\
0.163194444444444	127.887261888472\\
0.164930555555556	134.001221782452\\
0.166666666666667	134.473939089836\\
0.168402777777778	129.954330069673\\
0.170138888888889	119.108569755243\\
0.171875	97.8230401081593\\
0.173611111111111	95.9542125048232\\
0.175347222222222	119.167147226044\\
0.177083333333333	133.629636083847\\
0.178819444444444	141.744011023249\\
0.180555555555556	145.268115475889\\
0.182291666666667	144.672391616491\\
0.184027777777778	139.442232897053\\
0.185763888888889	129.685746996362\\
0.1875	131.590671709959\\
0.189236111111111	138.155380201775\\
0.190972222222222	138.535339841118\\
0.192708333333333	131.310849323299\\
0.194444444444444	94.5588023328814\\
0.196180555555556	128.188885730735\\
0.197916666666667	138.621458271862\\
0.199652777777778	141.173546722155\\
0.201388888888889	138.876507103473\\
0.203125	131.509235043568\\
0.204861111111111	114.762411404835\\
0.206597222222222	103.29166976036\\
0.208333333333333	119.664387716836\\
0.210069444444444	123.355602066853\\
0.211805555555556	122.734397245385\\
0.213541666666667	119.811104410406\\
0.215277777777778	115.068923146988\\
0.217013888888889	107.89198667438\\
0.21875	95.5338971214962\\
0.220486111111111	65.1106056324224\\
0.222222222222222	97.6121105812024\\
0.223958333333333	108.690759255263\\
0.225694444444444	113.684249338563\\
0.227430555555556	114.527579386097\\
0.229166666666667	111.560048057229\\
0.230902777777778	108.288832771884\\
0.232638888888889	114.174412682503\\
0.234375	121.267437109463\\
0.236111111111111	124.793963855339\\
0.237847222222222	124.611738796005\\
0.239583333333333	120.468357962665\\
0.241319444444444	113.315729351921\\
0.243055555555556	114.363702713016\\
0.244791666666667	120.465472237414\\
0.246527777777778	122.595074972077\\
0.248263888888889	120.457922911087\\
0.25	114.571524526459\\
0.251736111111111	108.967183964019\\
0.253472222222222	110.189530584\\
0.255208333333333	111.993458473418\\
0.256944444444444	111.604466380281\\
0.258680555555556	109.821385391665\\
0.260416666666667	105.422373249529\\
0.262152777777778	91.6353344178483\\
0.263888888888889	93.5842898113147\\
0.265625	110.348339404111\\
0.267361111111111	116.180077914518\\
0.269097222222222	116.014238208834\\
0.270833333333333	109.592351237191\\
0.272569444444444	90.4573152918644\\
0.274305555555556	96.7019280945807\\
0.276041666666667	107.857244968099\\
0.277777777777778	109.804689801781\\
0.279513888888889	107.106070055899\\
0.28125	102.053917207751\\
0.282986111111111	96.522862016512\\
0.284722222222222	89.3499481387288\\
0.286458333333333	87.4201272831718\\
0.288194444444444	96.1605382657542\\
0.289930555555556	100.454177049175\\
0.291666666666667	99.5592135427064\\
0.293402777777778	95.0122038509424\\
0.295138888888889	92.5009281418873\\
0.296875	94.5347683142076\\
0.298611111111111	95.2342378984314\\
0.300347222222222	94.159953646771\\
0.302083333333333	92.8664675073321\\
0.303819444444444	89.4012441300853\\
0.305555555555556	77.5613048275534\\
0.307291666666667	75.0995432167409\\
0.309027777777778	90.412812585286\\
0.310763888888889	94.3951953285023\\
0.3125	91.5052947565828\\
0.314236111111111	86.4576284147712\\
0.315972222222222	91.5938715615236\\
0.317708333333333	96.0291561583578\\
0.319444444444444	94.5442596124138\\
0.321180555555556	85.3056764515798\\
0.322916666666667	48.7987624898553\\
0.324652777777778	73.2391877647431\\
0.326388888888889	80.5531312575741\\
0.328125	88.4696825197645\\
0.329861111111111	97.3217715138234\\
0.331597222222222	102.104210047945\\
0.333333333333333	102.869894598332\\
0.335069444444444	101.050532110714\\
0.336805555555556	100.105558997395\\
0.338541666666667	101.263725689987\\
0.340277777777778	101.081327760572\\
0.342013888888889	100.021863099844\\
0.34375	104.577364864482\\
0.345486111111111	111.23028722602\\
0.347222222222222	114.935403706306\\
0.348958333333333	115.016799015749\\
0.350694444444444	111.122727942265\\
0.352430555555556	101.862361964192\\
0.354166666666667	81.257817656033\\
0.355902777777778	63.5662763447049\\
0.357638888888889	76.6861957680561\\
0.359375	94.3672846194682\\
0.361111111111111	107.020789136189\\
0.362847222222222	113.517568960059\\
0.364583333333333	115.607131516352\\
0.366319444444444	113.842359583074\\
0.368055555555556	108.024923167942\\
0.369791666666667	97.9713498996658\\
0.371527777777778	90.1536806232067\\
0.373263888888889	89.6653450212567\\
0.375	89.6123319540869\\
0.376736111111111	98.5150916154645\\
0.378472222222222	106.523761933946\\
0.380208333333333	109.892210819002\\
0.381944444444444	108.792420918496\\
0.383680555555556	102.558206214134\\
0.385416666666667	92.2218504542109\\
0.387152777777778	98.7094318744531\\
0.388888888888889	105.979473576891\\
0.390625	107.511116527849\\
0.392361111111111	103.726218844285\\
0.394097222222222	91.9654186354563\\
0.395833333333333	83.5060654258777\\
0.397569444444444	99.439392844854\\
0.399305555555556	105.126954915254\\
0.401041666666667	105.260352722557\\
0.402777777777778	100.547160884694\\
0.404513888888889	88.6860500009231\\
0.40625	75.0539737703488\\
0.407986111111111	92.0449163230458\\
0.409722222222222	99.3933697579163\\
0.411458333333333	102.73339678962\\
0.413194444444444	104.296156018197\\
0.414930555555556	104.856999913713\\
0.416666666666667	104.318421656918\\
0.418402777777778	101.801311002953\\
0.420138888888889	95.2807480496021\\
0.421875	78.225759165861\\
0.423611111111111	83.8185874103999\\
0.425347222222222	97.0162586538961\\
0.427083333333333	100.696797236145\\
0.428819444444444	98.1361522989074\\
0.430555555555556	88.0499355025432\\
0.432291666666667	85.8470655202807\\
0.434027777777778	97.8025335823473\\
0.435763888888889	102.280519539306\\
0.4375	101.261648777521\\
0.439236111111111	94.6974518887104\\
0.440972222222222	78.0004884113284\\
0.442708333333333	65.1933331251901\\
0.444444444444444	83.9211787106787\\
0.446180555555556	89.7550834609522\\
0.447916666666667	93.2134246887283\\
0.449652777777778	94.5838276526481\\
0.451388888888889	92.0235125181571\\
0.453125	83.2851536514538\\
0.454861111111111	82.3831780571625\\
0.456597222222222	93.9918905340067\\
0.458333333333333	99.2802122460265\\
0.460069444444444	99.8364088010922\\
0.461805555555556	97.1130088383707\\
0.463541666666667	93.8675595424979\\
0.465277777777778	93.3793883404122\\
0.467013888888889	92.8187511984097\\
0.46875	88.6922651111896\\
0.470486111111111	84.952722178589\\
0.472222222222222	91.6261538437413\\
0.473958333333333	97.2586041611805\\
0.475694444444444	98.3845779752275\\
0.477430555555556	95.0099765381672\\
0.479166666666667	86.609266349177\\
0.480902777777778	80.2206084454586\\
0.482638888888889	87.9710156380638\\
0.484375	93.4469480892341\\
0.486111111111111	95.7452777978059\\
0.487847222222222	95.9577846580952\\
0.489583333333333	94.1749197599843\\
0.491319444444444	89.9219151023431\\
0.493055555555556	82.9179756327449\\
0.494791666666667	74.2813745278125\\
0.496527777777778	63.0051481471648\\
0.498263888888889	63.3760679384775\\
0.5	83.3224976653944\\
0.501736111111111	93.212164400904\\
0.503472222222222	97.8011063365458\\
0.505208333333333	98.3993012208387\\
0.506944444444444	94.9836901417199\\
0.508680555555556	85.8066276203322\\
0.510416666666667	62.7157197688638\\
0.512152777777778	79.7823026978996\\
0.513888888888889	90.1505488572247\\
0.515625	94.0354599134308\\
0.517361111111111	94.3057520714442\\
0.519097222222222	91.1965631656826\\
0.520833333333333	83.4013329778391\\
0.522569444444444	64.8175919886811\\
0.524305555555556	56.6546562348778\\
0.526041666666667	68.8998359834065\\
0.527777777777778	66.193840497023\\
0.529513888888889	75.4880814909726\\
0.53125	87.4199639015277\\
0.532986111111111	93.2624329863472\\
0.534722222222222	94.2904283637427\\
0.536458333333333	90.3605077364261\\
0.538194444444444	78.3870140597998\\
0.539930555555556	65.8849026096883\\
0.541666666666667	84.0954746422237\\
0.543402777777778	90.7664245859495\\
0.545138888888889	93.8383934333513\\
0.546875	96.2877600132176\\
0.548611111111111	98.1689825130915\\
0.550347222222222	98.052576119394\\
0.552083333333333	94.5653166451017\\
0.553819444444444	85.973290477711\\
0.555555555555556	73.4907437927302\\
0.557291666666667	79.8086073846457\\
0.559027777777778	83.7368895902058\\
0.560763888888889	79.8928864765902\\
0.5625	63.5456587849032\\
0.564236111111111	60.5901680121263\\
0.565972222222222	78.0257620699817\\
0.567708333333333	84.3184850383372\\
0.569444444444444	87.2682913619894\\
0.571180555555556	87.5257721358999\\
0.572916666666667	83.4109692940506\\
0.574652777777778	69.314985171905\\
0.576388888888889	71.8104120186904\\
0.578125	87.3291801508287\\
0.579861111111111	93.2142548467695\\
0.581597222222222	94.2862549477712\\
0.583333333333333	91.7473208981271\\
0.585069444444444	85.9648540346671\\
0.586805555555556	77.3424877818344\\
0.588541666666667	66.858399550724\\
0.590277777777778	50.8121301177573\\
0.592013888888889	53.4540161723872\\
0.59375	71.3059314379268\\
0.595486111111111	78.4223334172975\\
0.597222222222222	79.281808827796\\
0.598958333333333	72.8056396254738\\
0.600694444444444	59.9376935558448\\
0.602430555555556	77.2611702415062\\
0.604166666666667	85.1432349000878\\
0.605902777777778	85.2982508075033\\
0.607638888888889	76.1786977075606\\
0.609375	46.494426378393\\
0.611111111111111	81.2588219094439\\
0.612847222222222	88.422352090244\\
0.614583333333333	88.3244418098028\\
0.616319444444444	84.3145057715072\\
0.618055555555556	82.9354059782976\\
0.619791666666667	83.7688403572418\\
0.621527777777778	78.4316925255381\\
0.623263888888889	75.5149503610029\\
0.625	90.4601342160714\\
0.626736111111111	97.6439861520073\\
0.628472222222222	98.5527692626165\\
0.630208333333333	92.8230338144331\\
0.631944444444444	70.8576935787819\\
0.633680555555556	80.0947608850271\\
0.635416666666667	92.2643429249787\\
0.637152777777778	92.8763926516385\\
0.638888888888889	84.8485300183152\\
0.640625	76.8822444234654\\
0.642361111111111	90.0034598234011\\
0.644097222222222	95.3567022455741\\
0.645833333333333	94.961056007618\\
0.647569444444444	90.3967172643333\\
0.649305555555556	83.6241675222379\\
0.651041666666667	77.8180918492008\\
0.652777777777778	70.0344695887315\\
0.654513888888889	58.6649234528496\\
0.65625	71.5092830531192\\
0.657986111111111	76.9238592186935\\
0.659722222222222	73.8828288273285\\
0.661458333333333	59.2223165061944\\
0.663194444444444	50.2171223067764\\
0.664930555555556	63.2408932919946\\
0.666666666666667	58.3127011065496\\
0.668402777777778	33.7753938656417\\
0.670138888888889	60.6126309071814\\
0.671875	66.0669990450634\\
0.673611111111111	68.3247631485248\\
0.675347222222222	75.4792167906643\\
0.677083333333333	80.5048356182711\\
0.678819444444444	80.1794425105286\\
0.680555555555556	74.6557023292504\\
0.682291666666667	76.2300613162241\\
0.684027777777778	85.1476991295114\\
0.685763888888889	89.6124240383872\\
0.6875	89.9109573124808\\
0.689236111111111	86.9588683189028\\
0.690972222222222	81.6054884829079\\
0.692708333333333	75.0136216670829\\
0.694444444444444	71.7024288329108\\
0.696180555555556	76.3150925069696\\
0.697916666666667	80.1506769496994\\
0.699652777777778	79.0448383832886\\
0.701388888888889	73.1176957343236\\
0.703125	75.9087265778338\\
0.704861111111111	83.7489976465291\\
0.706597222222222	85.7842661588102\\
0.708333333333333	81.7716503302567\\
0.710069444444444	79.7305464729472\\
0.711805555555556	89.5175548901514\\
0.713541666666667	95.987019857092\\
0.715277777777778	97.7994265352287\\
0.717013888888889	95.4151931385753\\
0.71875	88.1831831284828\\
0.720486111111111	73.1858921930426\\
0.722222222222222	66.1602756494151\\
0.723958333333333	75.5423668673403\\
0.725694444444444	76.2185857770722\\
0.727430555555556	71.8629021520653\\
0.729166666666667	66.1554928748349\\
0.730902777777778	66.287759931392\\
0.732638888888889	71.9969264861293\\
0.734375	77.2200743904344\\
0.736111111111111	79.7263809065993\\
0.737847222222222	78.9828737943815\\
0.739583333333333	74.9393841272023\\
0.741319444444444	68.9523480364982\\
0.743055555555556	65.6553134076959\\
0.744791666666667	64.2080408874646\\
0.746527777777778	61.1087616129032\\
0.748263888888889	65.7444964213163\\
0.75	74.0884345166904\\
0.751736111111111	78.3815754129291\\
0.753472222222222	78.3084475799163\\
0.755208333333333	73.9185026739318\\
0.756944444444444	68.8694629847733\\
0.758680555555556	73.6588065708677\\
0.760416666666667	78.7985817036136\\
0.762152777777778	79.5923217481667\\
0.763888888888889	76.4197157254105\\
0.765625	72.3399373873547\\
0.767361111111111	73.3257223457092\\
0.769097222222222	74.5624762757839\\
0.770833333333333	70.3776444297811\\
0.772569444444444	59.3148724539433\\
0.774305555555556	67.9426348183687\\
0.776041666666667	76.6270181490801\\
0.777777777777778	78.1218830955924\\
0.779513888888889	73.2874926570751\\
0.78125	63.4471533086645\\
0.782986111111111	69.2725853350028\\
0.784722222222222	76.0713418954348\\
0.786458333333333	76.7975807517541\\
0.788194444444444	72.5252076900803\\
0.789930555555556	63.9505542646417\\
0.791666666666667	51.5615569893087\\
0.793402777777778	37.2328661295283\\
0.795138888888889	65.9576526893173\\
0.796875	78.2398473515407\\
0.798611111111111	84.061184206096\\
0.800347222222222	85.4759975657547\\
0.802083333333333	83.2375704026441\\
0.803819444444444	79.4543290828221\\
0.805555555555556	80.4011809399113\\
0.807291666666667	84.5058404151433\\
0.809027777777778	86.5299185979157\\
0.810763888888889	85.7003790528681\\
0.8125	82.3117713401813\\
0.814236111111111	76.6027712421837\\
0.815972222222222	67.877900155793\\
0.817708333333333	50.5815390735805\\
0.819444444444444	41.7149766445959\\
0.821180555555556	61.3588795305778\\
0.822916666666667	68.3012122639712\\
0.824652777777778	73.1931318472517\\
0.826388888888889	77.7205179034095\\
0.828125	80.3855471828735\\
0.829861111111111	79.4527838799703\\
0.831597222222222	72.154370569686\\
0.833333333333333	42.5121814332675\\
0.835069444444444	71.9701920880971\\
0.836805555555556	82.5937503678553\\
0.838541666666667	85.3429902789392\\
0.840277777777778	82.2186718513484\\
0.842013888888889	70.8875502177986\\
0.84375	64.828238060448\\
0.845486111111111	78.1897554116019\\
0.847222222222222	81.4877422858007\\
0.848958333333333	78.1581605873364\\
0.850694444444444	76.0608305498195\\
0.852430555555556	83.6681710013033\\
0.854166666666667	88.1859927765113\\
0.855902777777778	86.8095138294995\\
0.857638888888889	77.6874695167755\\
0.859375	75.2078375933967\\
0.861111111111111	88.5727744789498\\
0.862847222222222	94.1969905763194\\
0.864583333333333	94.4823038418113\\
0.866319444444444	90.3071106764894\\
0.868055555555556	83.4539211498651\\
0.869791666666667	83.0564913267616\\
0.871527777777778	86.3978859223582\\
0.873263888888889	85.7510443147169\\
0.875	80.2817494160497\\
0.876736111111111	75.6628284298569\\
0.878472222222222	81.5241288598541\\
0.880208333333333	86.0004550596399\\
0.881944444444444	86.0010224745062\\
0.883680555555556	80.9104955745004\\
0.885416666666667	66.5336981044774\\
0.887152777777778	65.3353786753741\\
0.888888888888889	80.2078116004468\\
0.890625	85.4227138109221\\
0.892361111111111	85.8141648911918\\
0.894097222222222	83.1213647840344\\
0.895833333333333	80.2503068505251\\
0.897569444444444	80.5024244155214\\
0.899305555555556	81.1577543480851\\
0.901041666666667	78.7443885788803\\
0.902777777777778	70.8012222036852\\
0.904513888888889	46.2848422290628\\
0.90625	62.4148878394419\\
0.907986111111111	75.1497556147862\\
0.909722222222222	80.1558541980325\\
0.911458333333333	81.2563633827104\\
0.913194444444444	79.2756965676402\\
0.914930555555556	75.0816922432785\\
0.916666666666667	72.1832869427482\\
0.918402777777778	74.0568032591551\\
0.920138888888889	77.731943158473\\
0.921875	81.1545205528295\\
0.923611111111111	83.2478397383163\\
0.925347222222222	82.4495933001562\\
0.927083333333333	76.0668127066092\\
0.928819444444444	47.5986556733082\\
0.930555555555556	72.718037428497\\
0.932291666666667	85.8308930128101\\
0.934027777777778	90.8937579211417\\
0.935763888888889	91.7295392067198\\
0.9375	89.3363427846342\\
0.939236111111111	84.9358617967034\\
0.940972222222222	82.7597538326671\\
0.942708333333333	84.6590565051107\\
0.944444444444444	85.6130552250308\\
0.946180555555556	83.079672581972\\
0.947916666666667	76.1695443320844\\
0.949652777777778	65.8413416029305\\
0.951388888888889	60.0631653069528\\
0.953125	57.2298468139773\\
0.954861111111111	71.2298910525738\\
0.956597222222222	82.8785564393155\\
0.958333333333333	88.1773990694706\\
0.960069444444444	88.2677576512679\\
0.961805555555556	82.6829077660444\\
0.963541666666667	71.3175616131803\\
0.965277777777778	75.7730463395469\\
0.967013888888889	81.7613716799099\\
0.96875	80.4035616138141\\
0.970486111111111	74.7126252578609\\
0.972222222222222	77.4810015597658\\
0.973958333333333	82.2708922608027\\
0.975694444444444	81.559540023773\\
0.977430555555556	79.0998481134285\\
0.979166666666667	84.7321120066984\\
0.980902777777778	90.4163698795876\\
0.982638888888889	91.612618371193\\
0.984375	88.6002084086833\\
0.986111111111111	84.441566305154\\
0.987847222222222	85.7965706508487\\
0.989583333333333	88.0664404730842\\
0.991319444444444	86.2440939719655\\
0.993055555555556	78.7577321106242\\
0.994791666666667	62.7819929487378\\
0.996527777777778	53.25804784872\\
0.998263888888889	53.7330111825301\\
};
\addlegendentry{Mean Centred and Detrended};

\addplot [color=mycolor1,solid]
  table[row sep=crcr]{-1	39.3143869093762\\
-0.998263888888889	53.7330111825301\\
-0.996527777777778	53.25804784872\\
-0.994791666666667	62.7819929487378\\
-0.993055555555556	78.7577321106243\\
-0.991319444444444	86.2440939719655\\
-0.989583333333333	88.0664404730842\\
-0.987847222222222	85.7965706508487\\
-0.986111111111111	84.441566305154\\
-0.984375	88.6002084086833\\
-0.982638888888889	91.612618371193\\
-0.980902777777778	90.4163698795875\\
-0.979166666666667	84.7321120066984\\
-0.977430555555556	79.0998481134285\\
-0.975694444444444	81.559540023773\\
-0.973958333333333	82.2708922608027\\
-0.972222222222222	77.4810015597658\\
-0.970486111111111	74.7126252578609\\
-0.96875	80.4035616138141\\
-0.967013888888889	81.7613716799099\\
-0.965277777777778	75.7730463395469\\
-0.963541666666667	71.3175616131804\\
-0.961805555555556	82.6829077660444\\
-0.960069444444444	88.2677576512679\\
-0.958333333333333	88.1773990694706\\
-0.956597222222222	82.8785564393155\\
-0.954861111111111	71.2298910525738\\
-0.953125	57.2298468139774\\
-0.951388888888889	60.0631653069529\\
-0.949652777777778	65.8413416029305\\
-0.947916666666667	76.1695443320844\\
-0.946180555555556	83.0796725819719\\
-0.944444444444444	85.6130552250308\\
-0.942708333333333	84.6590565051107\\
-0.940972222222222	82.7597538326671\\
-0.939236111111111	84.9358617967034\\
-0.9375	89.3363427846342\\
-0.935763888888889	91.7295392067198\\
-0.934027777777778	90.8937579211417\\
-0.932291666666667	85.83089301281\\
-0.930555555555556	72.718037428497\\
-0.928819444444444	47.5986556733081\\
-0.927083333333333	76.0668127066091\\
-0.925347222222222	82.4495933001562\\
-0.923611111111111	83.2478397383163\\
-0.921875	81.1545205528295\\
-0.920138888888889	77.731943158473\\
-0.918402777777778	74.0568032591551\\
-0.916666666666667	72.1832869427483\\
-0.914930555555556	75.0816922432785\\
-0.913194444444444	79.2756965676403\\
-0.911458333333333	81.2563633827103\\
-0.909722222222222	80.1558541980325\\
-0.907986111111111	75.1497556147862\\
-0.90625	62.4148878394419\\
-0.904513888888889	46.2848422290628\\
-0.902777777777778	70.8012222036852\\
-0.901041666666667	78.7443885788802\\
-0.899305555555556	81.1577543480851\\
-0.897569444444444	80.5024244155214\\
-0.895833333333333	80.2503068505251\\
-0.894097222222222	83.1213647840344\\
-0.892361111111111	85.8141648911917\\
-0.890625	85.4227138109221\\
-0.888888888888889	80.2078116004468\\
-0.887152777777778	65.3353786753741\\
-0.885416666666667	66.5336981044773\\
-0.883680555555556	80.9104955745004\\
-0.881944444444444	86.0010224745062\\
-0.880208333333333	86.0004550596399\\
-0.878472222222222	81.5241288598541\\
-0.876736111111111	75.6628284298569\\
-0.875	80.2817494160497\\
-0.873263888888889	85.7510443147169\\
-0.871527777777778	86.3978859223582\\
-0.869791666666667	83.0564913267616\\
-0.868055555555556	83.4539211498651\\
-0.866319444444444	90.3071106764894\\
-0.864583333333333	94.4823038418113\\
-0.862847222222222	94.1969905763194\\
-0.861111111111111	88.5727744789498\\
-0.859375	75.2078375933967\\
-0.857638888888889	77.6874695167755\\
-0.855902777777778	86.8095138294996\\
-0.854166666666667	88.1859927765113\\
-0.852430555555556	83.6681710013033\\
-0.850694444444444	76.0608305498194\\
-0.848958333333333	78.1581605873364\\
-0.847222222222222	81.4877422858007\\
-0.845486111111111	78.189755411602\\
-0.84375	64.828238060448\\
-0.842013888888889	70.8875502177987\\
-0.840277777777778	82.2186718513483\\
-0.838541666666667	85.3429902789392\\
-0.836805555555556	82.5937503678553\\
-0.835069444444444	71.9701920880971\\
-0.833333333333333	42.5121814332678\\
-0.831597222222222	72.1543705696859\\
-0.829861111111111	79.4527838799703\\
-0.828125	80.3855471828735\\
-0.826388888888889	77.7205179034095\\
-0.824652777777778	73.1931318472517\\
-0.822916666666667	68.3012122639712\\
-0.821180555555556	61.3588795305777\\
-0.819444444444444	41.7149766445963\\
-0.817708333333333	50.5815390735804\\
-0.815972222222222	67.877900155793\\
-0.814236111111111	76.6027712421837\\
-0.8125	82.3117713401813\\
-0.810763888888889	85.700379052868\\
-0.809027777777778	86.5299185979157\\
-0.807291666666667	84.5058404151433\\
-0.805555555555556	80.4011809399113\\
-0.803819444444444	79.4543290828221\\
-0.802083333333333	83.2375704026441\\
-0.800347222222222	85.4759975657547\\
-0.798611111111111	84.061184206096\\
-0.796875	78.2398473515408\\
-0.795138888888889	65.9576526893172\\
-0.793402777777778	37.2328661295284\\
-0.791666666666667	51.5615569893088\\
-0.789930555555556	63.9505542646417\\
-0.788194444444444	72.5252076900803\\
-0.786458333333333	76.7975807517541\\
-0.784722222222222	76.0713418954349\\
-0.782986111111111	69.2725853350028\\
-0.78125	63.4471533086647\\
-0.779513888888889	73.2874926570751\\
-0.777777777777778	78.1218830955924\\
-0.776041666666667	76.6270181490801\\
-0.774305555555556	67.9426348183687\\
-0.772569444444444	59.3148724539434\\
-0.770833333333333	70.3776444297812\\
-0.769097222222222	74.562476275784\\
-0.767361111111111	73.3257223457092\\
-0.765625	72.3399373873547\\
-0.763888888888889	76.4197157254105\\
-0.762152777777778	79.5923217481667\\
-0.760416666666667	78.7985817036136\\
-0.758680555555556	73.6588065708677\\
-0.756944444444444	68.8694629847733\\
-0.755208333333333	73.9185026739318\\
-0.753472222222222	78.3084475799163\\
-0.751736111111111	78.3815754129291\\
-0.75	74.0884345166904\\
-0.748263888888889	65.7444964213162\\
-0.746527777777778	61.1087616129033\\
-0.744791666666667	64.2080408874646\\
-0.743055555555556	65.655313407696\\
-0.741319444444444	68.9523480364982\\
-0.739583333333333	74.9393841272023\\
-0.737847222222222	78.9828737943814\\
-0.736111111111111	79.7263809065993\\
-0.734375	77.2200743904345\\
-0.732638888888889	71.9969264861293\\
-0.730902777777778	66.2877599313919\\
-0.729166666666667	66.1554928748349\\
-0.727430555555556	71.8629021520653\\
-0.725694444444444	76.2185857770722\\
-0.723958333333333	75.5423668673403\\
-0.722222222222222	66.1602756494151\\
-0.720486111111111	73.1858921930426\\
-0.71875	88.1831831284828\\
-0.717013888888889	95.4151931385753\\
-0.715277777777778	97.7994265352287\\
-0.713541666666667	95.987019857092\\
-0.711805555555556	89.5175548901514\\
-0.710069444444444	79.7305464729472\\
-0.708333333333333	81.7716503302567\\
-0.706597222222222	85.7842661588102\\
-0.704861111111111	83.7489976465291\\
-0.703125	75.9087265778338\\
-0.701388888888889	73.1176957343237\\
-0.699652777777778	79.0448383832886\\
-0.697916666666667	80.1506769496994\\
-0.696180555555556	76.3150925069696\\
-0.694444444444444	71.7024288329107\\
-0.692708333333333	75.0136216670829\\
-0.690972222222222	81.6054884829078\\
-0.689236111111111	86.9588683189028\\
-0.6875	89.9109573124808\\
-0.685763888888889	89.6124240383872\\
-0.684027777777778	85.1476991295113\\
-0.682291666666667	76.2300613162241\\
-0.680555555555556	74.6557023292504\\
-0.678819444444444	80.1794425105286\\
-0.677083333333333	80.5048356182711\\
-0.675347222222222	75.4792167906643\\
-0.673611111111111	68.3247631485248\\
-0.671875	66.0669990450634\\
-0.670138888888889	60.6126309071815\\
-0.668402777777778	33.7753938656416\\
-0.666666666666667	58.3127011065496\\
-0.664930555555556	63.2408932919947\\
-0.663194444444444	50.2171223067763\\
-0.661458333333333	59.2223165061946\\
-0.659722222222222	73.8828288273286\\
-0.657986111111111	76.9238592186935\\
-0.65625	71.5092830531192\\
-0.654513888888889	58.6649234528496\\
-0.652777777777778	70.0344695887315\\
-0.651041666666667	77.8180918492008\\
-0.649305555555556	83.624167522238\\
-0.647569444444444	90.3967172643333\\
-0.645833333333333	94.961056007618\\
-0.644097222222222	95.3567022455742\\
-0.642361111111111	90.0034598234011\\
-0.640625	76.8822444234654\\
-0.638888888888889	84.8485300183152\\
-0.637152777777778	92.8763926516385\\
-0.635416666666667	92.2643429249787\\
-0.633680555555556	80.0947608850271\\
-0.631944444444444	70.8576935787819\\
-0.630208333333333	92.8230338144331\\
-0.628472222222222	98.5527692626166\\
-0.626736111111111	97.6439861520073\\
-0.625	90.4601342160714\\
-0.623263888888889	75.5149503610029\\
-0.621527777777778	78.4316925255381\\
-0.619791666666667	83.7688403572418\\
-0.618055555555556	82.9354059782976\\
-0.616319444444444	84.3145057715073\\
-0.614583333333333	88.3244418098028\\
-0.612847222222222	88.422352090244\\
-0.611111111111111	81.2588219094439\\
-0.609375	46.4944263783931\\
-0.607638888888889	76.1786977075606\\
-0.605902777777778	85.2982508075033\\
-0.604166666666667	85.1432349000878\\
-0.602430555555556	77.2611702415062\\
-0.600694444444444	59.9376935558447\\
-0.598958333333333	72.8056396254738\\
-0.597222222222222	79.281808827796\\
-0.595486111111111	78.4223334172975\\
-0.59375	71.3059314379268\\
-0.592013888888889	53.4540161723873\\
-0.590277777777778	50.8121301177572\\
-0.588541666666667	66.8583995507239\\
-0.586805555555556	77.3424877818344\\
-0.585069444444444	85.9648540346671\\
-0.583333333333333	91.7473208981271\\
-0.581597222222222	94.2862549477712\\
-0.579861111111111	93.2142548467695\\
-0.578125	87.3291801508287\\
-0.576388888888889	71.8104120186904\\
-0.574652777777778	69.3149851719049\\
-0.572916666666667	83.4109692940506\\
-0.571180555555556	87.5257721358999\\
-0.569444444444444	87.2682913619894\\
-0.567708333333333	84.3184850383372\\
-0.565972222222222	78.0257620699817\\
-0.564236111111111	60.5901680121262\\
-0.5625	63.5456587849033\\
-0.560763888888889	79.8928864765902\\
-0.559027777777778	83.7368895902058\\
-0.557291666666667	79.8086073846457\\
-0.555555555555556	73.4907437927301\\
-0.553819444444444	85.973290477711\\
-0.552083333333333	94.5653166451017\\
-0.550347222222222	98.052576119394\\
-0.548611111111111	98.1689825130915\\
-0.546875	96.2877600132176\\
-0.545138888888889	93.8383934333513\\
-0.543402777777778	90.7664245859495\\
-0.541666666666667	84.0954746422237\\
-0.539930555555556	65.8849026096884\\
-0.538194444444444	78.3870140597998\\
-0.536458333333333	90.3605077364261\\
-0.534722222222222	94.2904283637427\\
-0.532986111111111	93.2624329863472\\
-0.53125	87.4199639015277\\
-0.529513888888889	75.4880814909726\\
-0.527777777777778	66.1938404970231\\
-0.526041666666667	68.8998359834065\\
-0.524305555555556	56.6546562348779\\
-0.522569444444444	64.8175919886811\\
-0.520833333333333	83.4013329778391\\
-0.519097222222222	91.1965631656826\\
-0.517361111111111	94.3057520714442\\
-0.515625	94.0354599134308\\
-0.513888888888889	90.1505488572246\\
-0.512152777777778	79.7823026978996\\
-0.510416666666667	62.7157197688639\\
-0.508680555555556	85.8066276203322\\
-0.506944444444444	94.9836901417199\\
-0.505208333333333	98.3993012208387\\
-0.503472222222222	97.8011063365458\\
-0.501736111111111	93.212164400904\\
-0.5	83.3224976653943\\
-0.498263888888889	63.3760679384775\\
-0.496527777777778	63.0051481471648\\
-0.494791666666667	74.2813745278125\\
-0.493055555555556	82.9179756327449\\
-0.491319444444444	89.9219151023431\\
-0.489583333333333	94.1749197599843\\
-0.487847222222222	95.9577846580952\\
-0.486111111111111	95.7452777978059\\
-0.484375	93.4469480892341\\
-0.482638888888889	87.9710156380638\\
-0.480902777777778	80.2206084454586\\
-0.479166666666667	86.609266349177\\
-0.477430555555556	95.0099765381672\\
-0.475694444444444	98.3845779752275\\
-0.473958333333333	97.2586041611805\\
-0.472222222222222	91.6261538437413\\
-0.470486111111111	84.952722178589\\
-0.46875	88.6922651111896\\
-0.467013888888889	92.8187511984097\\
-0.465277777777778	93.3793883404122\\
-0.463541666666667	93.867559542498\\
-0.461805555555556	97.1130088383707\\
-0.460069444444444	99.8364088010922\\
-0.458333333333333	99.2802122460265\\
-0.456597222222222	93.9918905340067\\
-0.454861111111111	82.3831780571625\\
-0.453125	83.2851536514538\\
-0.451388888888889	92.0235125181571\\
-0.449652777777778	94.5838276526481\\
-0.447916666666667	93.2134246887283\\
-0.446180555555556	89.7550834609522\\
-0.444444444444444	83.9211787106787\\
-0.442708333333333	65.1933331251901\\
-0.440972222222222	78.0004884113284\\
-0.439236111111111	94.6974518887104\\
-0.4375	101.261648777521\\
-0.435763888888889	102.280519539306\\
-0.434027777777778	97.8025335823473\\
-0.432291666666667	85.8470655202807\\
-0.430555555555556	88.0499355025432\\
-0.428819444444444	98.1361522989074\\
-0.427083333333333	100.696797236145\\
-0.425347222222222	97.0162586538961\\
-0.423611111111111	83.8185874104\\
-0.421875	78.225759165861\\
-0.420138888888889	95.2807480496021\\
-0.418402777777778	101.801311002953\\
-0.416666666666667	104.318421656918\\
-0.414930555555556	104.856999913713\\
-0.413194444444444	104.296156018197\\
-0.411458333333333	102.73339678962\\
-0.409722222222222	99.3933697579163\\
-0.407986111111111	92.0449163230459\\
-0.40625	75.0539737703488\\
-0.404513888888889	88.686050000923\\
-0.402777777777778	100.547160884694\\
-0.401041666666667	105.260352722557\\
-0.399305555555556	105.126954915254\\
-0.397569444444444	99.439392844854\\
-0.395833333333333	83.5060654258777\\
-0.394097222222222	91.9654186354563\\
-0.392361111111111	103.726218844285\\
-0.390625	107.511116527849\\
-0.388888888888889	105.979473576891\\
-0.387152777777778	98.7094318744531\\
-0.385416666666667	92.2218504542109\\
-0.383680555555556	102.558206214134\\
-0.381944444444444	108.792420918496\\
-0.380208333333333	109.892210819002\\
-0.378472222222222	106.523761933946\\
-0.376736111111111	98.5150916154645\\
-0.375	89.6123319540869\\
-0.373263888888889	89.6653450212567\\
-0.371527777777778	90.1536806232067\\
-0.369791666666667	97.9713498996658\\
-0.368055555555556	108.024923167942\\
-0.366319444444444	113.842359583074\\
-0.364583333333333	115.607131516352\\
-0.362847222222222	113.517568960059\\
-0.361111111111111	107.020789136189\\
-0.359375	94.3672846194682\\
-0.357638888888889	76.6861957680561\\
-0.355902777777778	63.5662763447049\\
-0.354166666666667	81.257817656033\\
-0.352430555555556	101.862361964192\\
-0.350694444444444	111.122727942265\\
-0.348958333333333	115.016799015749\\
-0.347222222222222	114.935403706306\\
-0.345486111111111	111.23028722602\\
-0.34375	104.577364864482\\
-0.342013888888889	100.021863099844\\
-0.340277777777778	101.081327760572\\
-0.338541666666667	101.263725689987\\
-0.336805555555556	100.105558997395\\
-0.335069444444444	101.050532110714\\
-0.333333333333333	102.869894598332\\
-0.331597222222222	102.104210047945\\
-0.329861111111111	97.3217715138234\\
-0.328125	88.4696825197645\\
-0.326388888888889	80.5531312575741\\
-0.324652777777778	73.239187764743\\
-0.322916666666667	48.7987624898554\\
-0.321180555555556	85.3056764515799\\
-0.319444444444444	94.5442596124138\\
-0.317708333333333	96.0291561583578\\
-0.315972222222222	91.5938715615236\\
-0.314236111111111	86.4576284147712\\
-0.3125	91.5052947565828\\
-0.310763888888889	94.3951953285023\\
-0.309027777777778	90.412812585286\\
-0.307291666666667	75.099543216741\\
-0.305555555555556	77.5613048275534\\
-0.303819444444444	89.4012441300853\\
-0.302083333333333	92.8664675073321\\
-0.300347222222222	94.159953646771\\
-0.298611111111111	95.2342378984314\\
-0.296875	94.5347683142076\\
-0.295138888888889	92.5009281418873\\
-0.293402777777778	95.0122038509424\\
-0.291666666666667	99.5592135427064\\
-0.289930555555556	100.454177049175\\
-0.288194444444444	96.1605382657542\\
-0.286458333333333	87.4201272831718\\
-0.284722222222222	89.3499481387288\\
-0.282986111111111	96.522862016512\\
-0.28125	102.053917207751\\
-0.279513888888889	107.106070055899\\
-0.277777777777778	109.804689801781\\
-0.276041666666667	107.857244968099\\
-0.274305555555556	96.7019280945807\\
-0.272569444444444	90.4573152918643\\
-0.270833333333333	109.592351237191\\
-0.269097222222222	116.014238208834\\
-0.267361111111111	116.180077914518\\
-0.265625	110.348339404111\\
-0.263888888888889	93.5842898113148\\
-0.262152777777778	91.6353344178483\\
-0.260416666666667	105.422373249529\\
-0.258680555555556	109.821385391665\\
-0.256944444444444	111.604466380281\\
-0.255208333333333	111.993458473418\\
-0.253472222222222	110.189530584\\
-0.251736111111111	108.967183964019\\
-0.25	114.571524526459\\
-0.248263888888889	120.457922911087\\
-0.246527777777778	122.595074972077\\
-0.244791666666667	120.465472237414\\
-0.243055555555556	114.363702713016\\
-0.241319444444444	113.315729351921\\
-0.239583333333333	120.468357962665\\
-0.237847222222222	124.611738796005\\
-0.236111111111111	124.793963855339\\
-0.234375	121.267437109463\\
-0.232638888888889	114.174412682503\\
-0.230902777777778	108.288832771884\\
-0.229166666666667	111.560048057229\\
-0.227430555555556	114.527579386097\\
-0.225694444444444	113.684249338563\\
-0.223958333333333	108.690759255263\\
-0.222222222222222	97.6121105812024\\
-0.220486111111111	65.1106056324225\\
-0.21875	95.5338971214962\\
-0.217013888888889	107.89198667438\\
-0.215277777777778	115.068923146988\\
-0.213541666666667	119.811104410406\\
-0.211805555555556	122.734397245385\\
-0.210069444444444	123.355602066853\\
-0.208333333333333	119.664387716836\\
-0.206597222222222	103.29166976036\\
-0.204861111111111	114.762411404835\\
-0.203125	131.509235043568\\
-0.201388888888889	138.876507103473\\
-0.199652777777778	141.173546722155\\
-0.197916666666667	138.621458271862\\
-0.196180555555556	128.188885730735\\
-0.194444444444444	94.5588023328814\\
-0.192708333333333	131.310849323299\\
-0.190972222222222	138.535339841118\\
-0.189236111111111	138.155380201775\\
-0.1875	131.590671709959\\
-0.185763888888889	129.685746996362\\
-0.184027777777778	139.442232897053\\
-0.182291666666667	144.672391616491\\
-0.180555555555556	145.268115475889\\
-0.178819444444444	141.744011023249\\
-0.177083333333333	133.629636083847\\
-0.175347222222222	119.167147226044\\
-0.173611111111111	95.9542125048232\\
-0.171875	97.8230401081593\\
-0.170138888888889	119.108569755243\\
-0.168402777777778	129.954330069673\\
-0.166666666666667	134.473939089836\\
-0.164930555555556	134.001221782452\\
-0.163194444444444	127.887261888472\\
-0.161458333333333	110.3495301495\\
-0.159722222222222	109.919083276159\\
-0.157986111111111	123.495564561703\\
-0.15625	126.038631199881\\
-0.154513888888889	122.594063071509\\
-0.152777777777778	112.693013717125\\
-0.151041666666667	98.1390680901422\\
-0.149305555555556	108.951891114736\\
-0.147569444444444	114.944072230435\\
-0.145833333333333	115.284427260449\\
-0.144097222222222	111.020195596097\\
-0.142361111111111	102.288085757744\\
-0.140625	102.440942331738\\
-0.138888888888889	110.679276772324\\
-0.137152777777778	114.53012997817\\
-0.135416666666667	114.599000526693\\
-0.133680555555556	111.66953819281\\
-0.131944444444444	105.748745089231\\
-0.130208333333333	97.1040771767703\\
-0.128472222222222	98.8443988524569\\
-0.126736111111111	107.494680816912\\
-0.125	111.643409629777\\
-0.123263888888889	111.280448647981\\
-0.121527777777778	106.951671621551\\
-0.119791666666667	103.12283040667\\
-0.118055555555556	106.454008057434\\
-0.116319444444444	109.14353816689\\
-0.114583333333333	107.658368812077\\
-0.112847222222222	103.619270759463\\
-0.111111111111111	104.594660537478\\
-0.109375	109.149074753357\\
-0.107638888888889	110.988771821605\\
-0.105902777777778	109.429777332212\\
-0.104166666666667	104.776429214348\\
-0.102430555555556	96.1618689079501\\
-0.100694444444444	80.6347058013019\\
-0.0989583333333334	99.0314216716251\\
-0.0972222222222222	111.589705545742\\
-0.0954861111111112	117.809009609627\\
-0.09375	119.729392654012\\
-0.0920138888888888	117.535222038814\\
-0.0902777777777778	110.071316975322\\
-0.0885416666666666	96.2548622570579\\
-0.0868055555555556	102.382892340042\\
-0.0850694444444444	109.021913046788\\
-0.0833333333333334	108.584689337168\\
-0.0815972222222222	101.800220405479\\
-0.0798611111111112	84.8040782738826\\
-0.078125	83.3909205346635\\
-0.0763888888888888	101.065270955494\\
-0.0746527777777778	111.287867941862\\
-0.0729166666666666	117.447769048441\\
-0.0711805555555556	119.800825521134\\
-0.0694444444444444	118.072921359189\\
-0.0677083333333334	110.923947338704\\
-0.0659722222222222	93.2755457873829\\
-0.0642361111111112	94.9415275386324\\
-0.0625	103.09803002206\\
-0.0607638888888888	98.5539365537765\\
-0.0590277777777778	65.9630060797958\\
-0.0572916666666666	100.220900187234\\
-0.0555555555555556	109.667346308347\\
-0.0538194444444444	110.724006615653\\
-0.0520833333333334	104.777406174927\\
-0.0503472222222222	93.2578598588171\\
-0.0486111111111112	105.927633563884\\
-0.046875	115.106157105967\\
-0.0451388888888888	118.80480977031\\
-0.0434027777777778	119.16056370532\\
-0.0416666666666666	118.523571887236\\
-0.0399305555555556	120.837873611117\\
-0.0381944444444444	125.341299187077\\
-0.0364583333333334	128.378316739118\\
-0.0347222222222222	128.611401414018\\
-0.0329861111111112	125.576970686067\\
-0.03125	118.701344233358\\
-0.0295138888888888	108.393401287113\\
-0.0277777777777778	107.478141608544\\
-0.0260416666666666	116.671655139023\\
-0.0243055555555556	124.944849805677\\
-0.0225694444444444	130.886291603309\\
-0.0208333333333334	133.943092693634\\
-0.0190972222222222	133.732849552086\\
-0.0173611111111112	129.413762926904\\
-0.015625	117.943850194\\
-0.0138888888888888	98.4434950464224\\
-0.0121527777777778	119.565004411865\\
-0.0104166666666666	126.081692834887\\
-0.00868055555555558	126.490316523092\\
-0.00694444444444442	122.614148904159\\
-0.00520833333333337	114.506313809744\\
-0.00347222222222221	100.905940082347\\
-0.00173611111111116	76.5752155998942\\
0	-450.386753420969\\
0.00173611111111116	76.5752155998942\\
0.00347222222222232	100.905940082347\\
0.00520833333333326	114.506313809744\\
0.00694444444444442	122.614148904159\\
0.00868055555555558	126.490316523092\\
0.0104166666666667	126.081692834887\\
0.0121527777777777	119.565004411865\\
0.0138888888888888	98.4434950464224\\
0.015625	117.943850194\\
0.0173611111111112	129.413762926904\\
0.0190972222222223	133.732849552086\\
0.0208333333333333	133.943092693634\\
0.0225694444444444	130.886291603309\\
0.0243055555555556	124.944849805677\\
0.0260416666666667	116.671655139023\\
0.0277777777777777	107.478141608544\\
0.0295138888888888	108.393401287113\\
0.03125	118.701344233358\\
0.0329861111111112	125.576970686067\\
0.0347222222222223	128.611401414018\\
0.0364583333333333	128.378316739118\\
0.0381944444444444	125.341299187077\\
0.0399305555555556	120.837873611117\\
0.0416666666666667	118.523571887236\\
0.0434027777777777	119.16056370532\\
0.0451388888888888	118.80480977031\\
0.046875	115.106157105967\\
0.0486111111111112	105.927633563884\\
0.0503472222222223	93.2578598588171\\
0.0520833333333333	104.777406174927\\
0.0538194444444444	110.724006615653\\
0.0555555555555556	109.667346308347\\
0.0572916666666667	100.220900187234\\
0.0590277777777777	65.9630060797958\\
0.0607638888888888	98.5539365537765\\
0.0625	103.09803002206\\
0.0642361111111112	94.9415275386324\\
0.0659722222222223	93.2755457873829\\
0.0677083333333333	110.923947338704\\
0.0694444444444444	118.072921359189\\
0.0711805555555556	119.800825521134\\
0.0729166666666667	117.447769048441\\
0.0746527777777777	111.287867941862\\
0.0763888888888888	101.065270955494\\
0.078125	83.3909205346635\\
0.0798611111111112	84.8040782738826\\
0.0815972222222223	101.800220405479\\
0.0833333333333333	108.584689337168\\
0.0850694444444444	109.021913046788\\
0.0868055555555556	102.382892340042\\
0.0885416666666667	96.2548622570579\\
0.0902777777777777	110.071316975322\\
0.0920138888888888	117.535222038814\\
0.09375	119.729392654012\\
0.0954861111111112	117.809009609627\\
0.0972222222222223	111.589705545742\\
0.0989583333333333	99.0314216716251\\
0.100694444444444	80.6347058013019\\
0.102430555555556	96.1618689079501\\
0.104166666666667	104.776429214348\\
0.105902777777778	109.429777332212\\
0.107638888888889	110.988771821605\\
0.109375	109.149074753357\\
0.111111111111111	104.594660537478\\
0.112847222222222	103.619270759463\\
0.114583333333333	107.658368812077\\
0.116319444444444	109.14353816689\\
0.118055555555556	106.454008057434\\
0.119791666666667	103.12283040667\\
0.121527777777778	106.951671621551\\
0.123263888888889	111.280448647981\\
0.125	111.643409629777\\
0.126736111111111	107.494680816912\\
0.128472222222222	98.8443988524569\\
0.130208333333333	97.1040771767703\\
0.131944444444444	105.748745089231\\
0.133680555555556	111.66953819281\\
0.135416666666667	114.599000526693\\
0.137152777777778	114.53012997817\\
0.138888888888889	110.679276772324\\
0.140625	102.440942331738\\
0.142361111111111	102.288085757744\\
0.144097222222222	111.020195596097\\
0.145833333333333	115.284427260449\\
0.147569444444444	114.944072230435\\
0.149305555555556	108.951891114736\\
0.151041666666667	98.1390680901422\\
0.152777777777778	112.693013717125\\
0.154513888888889	122.594063071509\\
0.15625	126.038631199881\\
0.157986111111111	123.495564561703\\
0.159722222222222	109.919083276159\\
0.161458333333333	110.3495301495\\
0.163194444444444	127.887261888472\\
0.164930555555556	134.001221782452\\
0.166666666666667	134.473939089836\\
0.168402777777778	129.954330069673\\
0.170138888888889	119.108569755243\\
0.171875	97.8230401081593\\
0.173611111111111	95.9542125048232\\
0.175347222222222	119.167147226044\\
0.177083333333333	133.629636083847\\
0.178819444444444	141.744011023249\\
0.180555555555556	145.268115475889\\
0.182291666666667	144.672391616491\\
0.184027777777778	139.442232897053\\
0.185763888888889	129.685746996362\\
0.1875	131.590671709959\\
0.189236111111111	138.155380201775\\
0.190972222222222	138.535339841118\\
0.192708333333333	131.310849323299\\
0.194444444444444	94.5588023328814\\
0.196180555555556	128.188885730735\\
0.197916666666667	138.621458271862\\
0.199652777777778	141.173546722155\\
0.201388888888889	138.876507103473\\
0.203125	131.509235043568\\
0.204861111111111	114.762411404835\\
0.206597222222222	103.29166976036\\
0.208333333333333	119.664387716836\\
0.210069444444444	123.355602066853\\
0.211805555555556	122.734397245385\\
0.213541666666667	119.811104410406\\
0.215277777777778	115.068923146988\\
0.217013888888889	107.89198667438\\
0.21875	95.5338971214962\\
0.220486111111111	65.1106056324225\\
0.222222222222222	97.6121105812024\\
0.223958333333333	108.690759255263\\
0.225694444444444	113.684249338563\\
0.227430555555556	114.527579386097\\
0.229166666666667	111.560048057229\\
0.230902777777778	108.288832771884\\
0.232638888888889	114.174412682503\\
0.234375	121.267437109463\\
0.236111111111111	124.793963855339\\
0.237847222222222	124.611738796005\\
0.239583333333333	120.468357962665\\
0.241319444444444	113.315729351921\\
0.243055555555556	114.363702713016\\
0.244791666666667	120.465472237414\\
0.246527777777778	122.595074972077\\
0.248263888888889	120.457922911087\\
0.25	114.571524526459\\
0.251736111111111	108.967183964019\\
0.253472222222222	110.189530584\\
0.255208333333333	111.993458473418\\
0.256944444444444	111.604466380281\\
0.258680555555556	109.821385391665\\
0.260416666666667	105.422373249529\\
0.262152777777778	91.6353344178483\\
0.263888888888889	93.5842898113148\\
0.265625	110.348339404111\\
0.267361111111111	116.180077914518\\
0.269097222222222	116.014238208834\\
0.270833333333333	109.592351237191\\
0.272569444444444	90.4573152918643\\
0.274305555555556	96.7019280945807\\
0.276041666666667	107.857244968099\\
0.277777777777778	109.804689801781\\
0.279513888888889	107.106070055899\\
0.28125	102.053917207751\\
0.282986111111111	96.522862016512\\
0.284722222222222	89.3499481387288\\
0.286458333333333	87.4201272831718\\
0.288194444444444	96.1605382657542\\
0.289930555555556	100.454177049175\\
0.291666666666667	99.5592135427064\\
0.293402777777778	95.0122038509424\\
0.295138888888889	92.5009281418873\\
0.296875	94.5347683142076\\
0.298611111111111	95.2342378984314\\
0.300347222222222	94.159953646771\\
0.302083333333333	92.8664675073321\\
0.303819444444444	89.4012441300853\\
0.305555555555556	77.5613048275534\\
0.307291666666667	75.099543216741\\
0.309027777777778	90.412812585286\\
0.310763888888889	94.3951953285023\\
0.3125	91.5052947565828\\
0.314236111111111	86.4576284147712\\
0.315972222222222	91.5938715615236\\
0.317708333333333	96.0291561583578\\
0.319444444444444	94.5442596124138\\
0.321180555555556	85.3056764515799\\
0.322916666666667	48.7987624898554\\
0.324652777777778	73.239187764743\\
0.326388888888889	80.5531312575741\\
0.328125	88.4696825197645\\
0.329861111111111	97.3217715138234\\
0.331597222222222	102.104210047945\\
0.333333333333333	102.869894598332\\
0.335069444444444	101.050532110714\\
0.336805555555556	100.105558997395\\
0.338541666666667	101.263725689987\\
0.340277777777778	101.081327760572\\
0.342013888888889	100.021863099844\\
0.34375	104.577364864482\\
0.345486111111111	111.23028722602\\
0.347222222222222	114.935403706306\\
0.348958333333333	115.016799015749\\
0.350694444444444	111.122727942265\\
0.352430555555556	101.862361964192\\
0.354166666666667	81.257817656033\\
0.355902777777778	63.5662763447049\\
0.357638888888889	76.6861957680561\\
0.359375	94.3672846194682\\
0.361111111111111	107.020789136189\\
0.362847222222222	113.517568960059\\
0.364583333333333	115.607131516352\\
0.366319444444444	113.842359583074\\
0.368055555555556	108.024923167942\\
0.369791666666667	97.9713498996658\\
0.371527777777778	90.1536806232067\\
0.373263888888889	89.6653450212567\\
0.375	89.6123319540869\\
0.376736111111111	98.5150916154645\\
0.378472222222222	106.523761933946\\
0.380208333333333	109.892210819002\\
0.381944444444444	108.792420918496\\
0.383680555555556	102.558206214134\\
0.385416666666667	92.2218504542109\\
0.387152777777778	98.7094318744531\\
0.388888888888889	105.979473576891\\
0.390625	107.511116527849\\
0.392361111111111	103.726218844285\\
0.394097222222222	91.9654186354563\\
0.395833333333333	83.5060654258777\\
0.397569444444444	99.439392844854\\
0.399305555555556	105.126954915254\\
0.401041666666667	105.260352722557\\
0.402777777777778	100.547160884694\\
0.404513888888889	88.686050000923\\
0.40625	75.0539737703488\\
0.407986111111111	92.0449163230459\\
0.409722222222222	99.3933697579163\\
0.411458333333333	102.73339678962\\
0.413194444444444	104.296156018197\\
0.414930555555556	104.856999913713\\
0.416666666666667	104.318421656918\\
0.418402777777778	101.801311002953\\
0.420138888888889	95.2807480496021\\
0.421875	78.225759165861\\
0.423611111111111	83.8185874104\\
0.425347222222222	97.0162586538961\\
0.427083333333333	100.696797236145\\
0.428819444444444	98.1361522989074\\
0.430555555555556	88.0499355025432\\
0.432291666666667	85.8470655202807\\
0.434027777777778	97.8025335823473\\
0.435763888888889	102.280519539306\\
0.4375	101.261648777521\\
0.439236111111111	94.6974518887104\\
0.440972222222222	78.0004884113284\\
0.442708333333333	65.1933331251901\\
0.444444444444444	83.9211787106787\\
0.446180555555556	89.7550834609522\\
0.447916666666667	93.2134246887283\\
0.449652777777778	94.5838276526481\\
0.451388888888889	92.0235125181571\\
0.453125	83.2851536514538\\
0.454861111111111	82.3831780571625\\
0.456597222222222	93.9918905340067\\
0.458333333333333	99.2802122460265\\
0.460069444444444	99.8364088010922\\
0.461805555555556	97.1130088383707\\
0.463541666666667	93.867559542498\\
0.465277777777778	93.3793883404122\\
0.467013888888889	92.8187511984097\\
0.46875	88.6922651111896\\
0.470486111111111	84.952722178589\\
0.472222222222222	91.6261538437413\\
0.473958333333333	97.2586041611805\\
0.475694444444444	98.3845779752275\\
0.477430555555556	95.0099765381672\\
0.479166666666667	86.609266349177\\
0.480902777777778	80.2206084454586\\
0.482638888888889	87.9710156380638\\
0.484375	93.4469480892341\\
0.486111111111111	95.7452777978059\\
0.487847222222222	95.9577846580952\\
0.489583333333333	94.1749197599843\\
0.491319444444444	89.9219151023431\\
0.493055555555556	82.9179756327449\\
0.494791666666667	74.2813745278125\\
0.496527777777778	63.0051481471648\\
0.498263888888889	63.3760679384775\\
0.5	83.3224976653943\\
0.501736111111111	93.212164400904\\
0.503472222222222	97.8011063365458\\
0.505208333333333	98.3993012208387\\
0.506944444444444	94.9836901417199\\
0.508680555555556	85.8066276203322\\
0.510416666666667	62.7157197688639\\
0.512152777777778	79.7823026978996\\
0.513888888888889	90.1505488572246\\
0.515625	94.0354599134308\\
0.517361111111111	94.3057520714442\\
0.519097222222222	91.1965631656826\\
0.520833333333333	83.4013329778391\\
0.522569444444444	64.8175919886811\\
0.524305555555556	56.6546562348779\\
0.526041666666667	68.8998359834065\\
0.527777777777778	66.1938404970231\\
0.529513888888889	75.4880814909726\\
0.53125	87.4199639015277\\
0.532986111111111	93.2624329863472\\
0.534722222222222	94.2904283637427\\
0.536458333333333	90.3605077364261\\
0.538194444444444	78.3870140597998\\
0.539930555555556	65.8849026096884\\
0.541666666666667	84.0954746422237\\
0.543402777777778	90.7664245859495\\
0.545138888888889	93.8383934333513\\
0.546875	96.2877600132176\\
0.548611111111111	98.1689825130915\\
0.550347222222222	98.052576119394\\
0.552083333333333	94.5653166451017\\
0.553819444444444	85.973290477711\\
0.555555555555556	73.4907437927301\\
0.557291666666667	79.8086073846457\\
0.559027777777778	83.7368895902058\\
0.560763888888889	79.8928864765902\\
0.5625	63.5456587849033\\
0.564236111111111	60.5901680121262\\
0.565972222222222	78.0257620699817\\
0.567708333333333	84.3184850383372\\
0.569444444444444	87.2682913619894\\
0.571180555555556	87.5257721358999\\
0.572916666666667	83.4109692940506\\
0.574652777777778	69.3149851719049\\
0.576388888888889	71.8104120186904\\
0.578125	87.3291801508287\\
0.579861111111111	93.2142548467695\\
0.581597222222222	94.2862549477712\\
0.583333333333333	91.7473208981271\\
0.585069444444444	85.9648540346671\\
0.586805555555556	77.3424877818344\\
0.588541666666667	66.8583995507239\\
0.590277777777778	50.8121301177572\\
0.592013888888889	53.4540161723873\\
0.59375	71.3059314379268\\
0.595486111111111	78.4223334172975\\
0.597222222222222	79.281808827796\\
0.598958333333333	72.8056396254738\\
0.600694444444444	59.9376935558447\\
0.602430555555556	77.2611702415062\\
0.604166666666667	85.1432349000878\\
0.605902777777778	85.2982508075033\\
0.607638888888889	76.1786977075606\\
0.609375	46.4944263783931\\
0.611111111111111	81.2588219094439\\
0.612847222222222	88.422352090244\\
0.614583333333333	88.3244418098028\\
0.616319444444444	84.3145057715073\\
0.618055555555556	82.9354059782976\\
0.619791666666667	83.7688403572418\\
0.621527777777778	78.4316925255381\\
0.623263888888889	75.5149503610029\\
0.625	90.4601342160714\\
0.626736111111111	97.6439861520073\\
0.628472222222222	98.5527692626166\\
0.630208333333333	92.8230338144331\\
0.631944444444444	70.8576935787819\\
0.633680555555556	80.0947608850271\\
0.635416666666667	92.2643429249787\\
0.637152777777778	92.8763926516385\\
0.638888888888889	84.8485300183152\\
0.640625	76.8822444234654\\
0.642361111111111	90.0034598234011\\
0.644097222222222	95.3567022455742\\
0.645833333333333	94.961056007618\\
0.647569444444444	90.3967172643333\\
0.649305555555556	83.624167522238\\
0.651041666666667	77.8180918492008\\
0.652777777777778	70.0344695887315\\
0.654513888888889	58.6649234528496\\
0.65625	71.5092830531192\\
0.657986111111111	76.9238592186935\\
0.659722222222222	73.8828288273286\\
0.661458333333333	59.2223165061946\\
0.663194444444444	50.2171223067763\\
0.664930555555556	63.2408932919947\\
0.666666666666667	58.3127011065496\\
0.668402777777778	33.7753938656416\\
0.670138888888889	60.6126309071815\\
0.671875	66.0669990450634\\
0.673611111111111	68.3247631485248\\
0.675347222222222	75.4792167906643\\
0.677083333333333	80.5048356182711\\
0.678819444444444	80.1794425105286\\
0.680555555555556	74.6557023292504\\
0.682291666666667	76.2300613162241\\
0.684027777777778	85.1476991295113\\
0.685763888888889	89.6124240383872\\
0.6875	89.9109573124808\\
0.689236111111111	86.9588683189028\\
0.690972222222222	81.6054884829078\\
0.692708333333333	75.0136216670829\\
0.694444444444444	71.7024288329107\\
0.696180555555556	76.3150925069696\\
0.697916666666667	80.1506769496994\\
0.699652777777778	79.0448383832886\\
0.701388888888889	73.1176957343237\\
0.703125	75.9087265778338\\
0.704861111111111	83.7489976465291\\
0.706597222222222	85.7842661588102\\
0.708333333333333	81.7716503302567\\
0.710069444444444	79.7305464729472\\
0.711805555555556	89.5175548901514\\
0.713541666666667	95.987019857092\\
0.715277777777778	97.7994265352287\\
0.717013888888889	95.4151931385753\\
0.71875	88.1831831284828\\
0.720486111111111	73.1858921930426\\
0.722222222222222	66.1602756494151\\
0.723958333333333	75.5423668673403\\
0.725694444444444	76.2185857770722\\
0.727430555555556	71.8629021520653\\
0.729166666666667	66.1554928748349\\
0.730902777777778	66.2877599313919\\
0.732638888888889	71.9969264861293\\
0.734375	77.2200743904345\\
0.736111111111111	79.7263809065993\\
0.737847222222222	78.9828737943814\\
0.739583333333333	74.9393841272023\\
0.741319444444444	68.9523480364982\\
0.743055555555556	65.655313407696\\
0.744791666666667	64.2080408874646\\
0.746527777777778	61.1087616129033\\
0.748263888888889	65.7444964213162\\
0.75	74.0884345166904\\
0.751736111111111	78.3815754129291\\
0.753472222222222	78.3084475799163\\
0.755208333333333	73.9185026739318\\
0.756944444444444	68.8694629847733\\
0.758680555555556	73.6588065708677\\
0.760416666666667	78.7985817036136\\
0.762152777777778	79.5923217481667\\
0.763888888888889	76.4197157254105\\
0.765625	72.3399373873547\\
0.767361111111111	73.3257223457092\\
0.769097222222222	74.562476275784\\
0.770833333333333	70.3776444297812\\
0.772569444444444	59.3148724539434\\
0.774305555555556	67.9426348183687\\
0.776041666666667	76.6270181490801\\
0.777777777777778	78.1218830955924\\
0.779513888888889	73.2874926570751\\
0.78125	63.4471533086647\\
0.782986111111111	69.2725853350028\\
0.784722222222222	76.0713418954349\\
0.786458333333333	76.7975807517541\\
0.788194444444444	72.5252076900803\\
0.789930555555556	63.9505542646417\\
0.791666666666667	51.5615569893088\\
0.793402777777778	37.2328661295284\\
0.795138888888889	65.9576526893172\\
0.796875	78.2398473515408\\
0.798611111111111	84.061184206096\\
0.800347222222222	85.4759975657547\\
0.802083333333333	83.2375704026441\\
0.803819444444444	79.4543290828221\\
0.805555555555556	80.4011809399113\\
0.807291666666667	84.5058404151433\\
0.809027777777778	86.5299185979157\\
0.810763888888889	85.700379052868\\
0.8125	82.3117713401813\\
0.814236111111111	76.6027712421837\\
0.815972222222222	67.877900155793\\
0.817708333333333	50.5815390735804\\
0.819444444444444	41.7149766445963\\
0.821180555555556	61.3588795305777\\
0.822916666666667	68.3012122639712\\
0.824652777777778	73.1931318472517\\
0.826388888888889	77.7205179034095\\
0.828125	80.3855471828735\\
0.829861111111111	79.4527838799703\\
0.831597222222222	72.1543705696859\\
0.833333333333333	42.5121814332678\\
0.835069444444444	71.9701920880971\\
0.836805555555556	82.5937503678553\\
0.838541666666667	85.3429902789392\\
0.840277777777778	82.2186718513483\\
0.842013888888889	70.8875502177987\\
0.84375	64.828238060448\\
0.845486111111111	78.189755411602\\
0.847222222222222	81.4877422858007\\
0.848958333333333	78.1581605873364\\
0.850694444444444	76.0608305498194\\
0.852430555555556	83.6681710013033\\
0.854166666666667	88.1859927765113\\
0.855902777777778	86.8095138294996\\
0.857638888888889	77.6874695167755\\
0.859375	75.2078375933967\\
0.861111111111111	88.5727744789498\\
0.862847222222222	94.1969905763194\\
0.864583333333333	94.4823038418113\\
0.866319444444444	90.3071106764894\\
0.868055555555556	83.4539211498651\\
0.869791666666667	83.0564913267616\\
0.871527777777778	86.3978859223582\\
0.873263888888889	85.7510443147169\\
0.875	80.2817494160497\\
0.876736111111111	75.6628284298569\\
0.878472222222222	81.5241288598541\\
0.880208333333333	86.0004550596399\\
0.881944444444444	86.0010224745062\\
0.883680555555556	80.9104955745004\\
0.885416666666667	66.5336981044773\\
0.887152777777778	65.3353786753741\\
0.888888888888889	80.2078116004468\\
0.890625	85.4227138109221\\
0.892361111111111	85.8141648911917\\
0.894097222222222	83.1213647840344\\
0.895833333333333	80.2503068505251\\
0.897569444444444	80.5024244155214\\
0.899305555555556	81.1577543480851\\
0.901041666666667	78.7443885788802\\
0.902777777777778	70.8012222036852\\
0.904513888888889	46.2848422290628\\
0.90625	62.4148878394419\\
0.907986111111111	75.1497556147862\\
0.909722222222222	80.1558541980325\\
0.911458333333333	81.2563633827103\\
0.913194444444444	79.2756965676403\\
0.914930555555556	75.0816922432785\\
0.916666666666667	72.1832869427483\\
0.918402777777778	74.0568032591551\\
0.920138888888889	77.731943158473\\
0.921875	81.1545205528295\\
0.923611111111111	83.2478397383163\\
0.925347222222222	82.4495933001562\\
0.927083333333333	76.0668127066091\\
0.928819444444444	47.5986556733081\\
0.930555555555556	72.718037428497\\
0.932291666666667	85.83089301281\\
0.934027777777778	90.8937579211417\\
0.935763888888889	91.7295392067198\\
0.9375	89.3363427846342\\
0.939236111111111	84.9358617967034\\
0.940972222222222	82.7597538326671\\
0.942708333333333	84.6590565051107\\
0.944444444444444	85.6130552250308\\
0.946180555555556	83.0796725819719\\
0.947916666666667	76.1695443320844\\
0.949652777777778	65.8413416029305\\
0.951388888888889	60.0631653069529\\
0.953125	57.2298468139774\\
0.954861111111111	71.2298910525738\\
0.956597222222222	82.8785564393155\\
0.958333333333333	88.1773990694706\\
0.960069444444444	88.2677576512679\\
0.961805555555556	82.6829077660444\\
0.963541666666667	71.3175616131804\\
0.965277777777778	75.7730463395469\\
0.967013888888889	81.7613716799099\\
0.96875	80.4035616138141\\
0.970486111111111	74.7126252578609\\
0.972222222222222	77.4810015597658\\
0.973958333333333	82.2708922608027\\
0.975694444444444	81.559540023773\\
0.977430555555556	79.0998481134285\\
0.979166666666667	84.7321120066984\\
0.980902777777778	90.4163698795875\\
0.982638888888889	91.612618371193\\
0.984375	88.6002084086833\\
0.986111111111111	84.441566305154\\
0.987847222222222	85.7965706508487\\
0.989583333333333	88.0664404730842\\
0.991319444444444	86.2440939719655\\
0.993055555555556	78.7577321106243\\
0.994791666666667	62.7819929487378\\
0.996527777777778	53.25804784872\\
0.998263888888889	53.7330111825301\\
};
\addlegendentry{Detrended};

\addplot [color=mycolor2,solid]
  table[row sep=crcr]{-1	24.5044762670395\\
-0.998263888888889	21.3779517972072\\
-0.996527777777778	16.6823802415607\\
-0.994791666666667	15.6552287803657\\
-0.993055555555556	22.2938014155795\\
-0.991319444444444	27.8461902751359\\
-0.989583333333333	31.2042632666706\\
-0.987847222222222	33.4187980112127\\
-0.986111111111111	33.7810426376993\\
-0.984375	29.590015975971\\
-0.982638888888889	13.4091454491996\\
-0.980902777777778	24.3695545292672\\
-0.979166666666667	37.8368656298608\\
-0.977430555555556	42.0459672450587\\
-0.975694444444444	39.7497982690858\\
-0.973958333333333	27.6465281521091\\
-0.972222222222222	18.2654041085709\\
-0.970486111111111	37.6109446160749\\
-0.96875	42.8859678278447\\
-0.967013888888889	41.5534372404585\\
-0.965277777777778	33.6174385713286\\
-0.963541666666667	13.9722794452155\\
-0.961805555555556	19.8514200749341\\
-0.960069444444444	27.712627718696\\
-0.958333333333333	29.7311378358931\\
-0.956597222222222	32.9507947596737\\
-0.954861111111111	36.0696874108362\\
-0.953125	35.6918376119954\\
-0.951388888888889	29.7585512956188\\
-0.949652777777778	11.3319603049801\\
-0.947916666666667	8.46752213736928\\
-0.946180555555556	22.6822754846624\\
-0.944444444444444	21.8264326384342\\
-0.942708333333333	3.85731511312707\\
-0.940972222222222	15.4919468948828\\
-0.939236111111111	31.1147962135588\\
-0.9375	36.8890701275635\\
-0.935763888888889	38.0002864144859\\
-0.934027777777778	36.6410118322991\\
-0.932291666666667	36.4641085968031\\
-0.930555555555556	38.6036137832606\\
-0.928819444444444	39.0750795180243\\
-0.927083333333333	34.6159533834001\\
-0.925347222222222	18.8992983266717\\
-0.923611111111111	19.4088478697592\\
-0.921875	34.6567982839451\\
-0.920138888888889	38.9791120566242\\
-0.918402777777778	38.7097068197068\\
-0.916666666666667	36.9615314585576\\
-0.914930555555556	35.562203941365\\
-0.913194444444444	31.7928085204821\\
-0.911458333333333	18.4167845396512\\
-0.909722222222222	19.5854218879846\\
-0.907986111111111	35.5256880272427\\
-0.90625	40.8937066482851\\
-0.904513888888889	40.9636364854981\\
-0.902777777777778	37.5460208018122\\
-0.901041666666667	32.8977516942875\\
-0.899305555555556	28.1382445172326\\
-0.897569444444444	19.2335565381793\\
-0.895833333333333	23.0836527108584\\
-0.894097222222222	35.8959044939046\\
-0.892361111111111	41.5918920413216\\
-0.890625	42.0518088021364\\
-0.888888888888889	37.4578624817154\\
-0.887152777777778	25.4690552999112\\
-0.885416666666667	11.1403607996604\\
-0.883680555555556	27.0879839076077\\
-0.881944444444444	33.8831655552315\\
-0.880208333333333	37.1347317849501\\
-0.878472222222222	38.951481952798\\
-0.876736111111111	40.0486540531982\\
-0.875	40.8997452241778\\
-0.873263888888889	41.2575924054311\\
-0.871527777777778	39.9458463575895\\
-0.869791666666667	35.8471013444349\\
-0.868055555555556	30.8837869870833\\
-0.866319444444444	33.1769766722211\\
-0.864583333333333	37.3622462549738\\
-0.862847222222222	37.6493603441265\\
-0.861111111111111	32.9583773794793\\
-0.859375	20.0772298850662\\
-0.857638888888889	0.418583865893578\\
-0.855902777777778	26.3931503379517\\
-0.854166666666667	35.882381781434\\
-0.852430555555556	40.5731969934036\\
-0.850694444444444	41.5747106692079\\
-0.848958333333333	38.3649133423653\\
-0.847222222222222	29.2616363990821\\
-0.845486111111111	21.0121068269714\\
-0.84375	30.503891780036\\
-0.842013888888889	34.0436001019915\\
-0.840277777777778	30.6832286934012\\
-0.838541666666667	17.3167580746099\\
-0.836805555555556	3.38734108875097\\
-0.835069444444444	25.9347916788639\\
-0.833333333333333	33.7429653268469\\
-0.831597222222222	37.8024499276758\\
-0.829861111111111	39.9033881940781\\
-0.828125	40.8244150061552\\
-0.826388888888889	41.6613563058114\\
-0.824652777777778	42.6769347721991\\
-0.822916666666667	42.3204007964847\\
-0.821180555555556	38.3764196946507\\
-0.819444444444444	27.6768367426495\\
-0.817708333333333	23.018273336393\\
-0.815972222222222	35.3355000814005\\
-0.814236111111111	39.4514092799356\\
-0.8125	37.3649803194283\\
-0.810763888888889	28.5034033432553\\
-0.809027777777778	10.6413958480024\\
-0.807291666666667	16.7091045305009\\
-0.805555555555556	17.1216982769862\\
-0.803819444444444	-5.71417825061052\\
-0.802083333333333	19.1125311801672\\
-0.800347222222222	31.2283655838052\\
-0.798611111111111	35.6843494013051\\
-0.796875	36.6949957600309\\
-0.795138888888889	36.1742010730263\\
-0.793402777777778	34.5777294323561\\
-0.791666666666667	30.6231646897838\\
-0.789930555555556	21.5480197680227\\
-0.788194444444444	-2.95446099309815\\
-0.786458333333333	1.60504328434085\\
-0.784722222222222	6.37290106090552\\
-0.782986111111111	-10.1890461959547\\
-0.78125	12.5486486987558\\
-0.779513888888889	22.8602189444058\\
-0.777777777777778	24.6029123648707\\
-0.776041666666667	21.2194572825798\\
-0.774305555555556	23.5220100200462\\
-0.772569444444444	31.7923521317572\\
-0.770833333333333	36.3862756218435\\
-0.769097222222222	36.81649183\\
-0.767361111111111	33.40930282345\\
-0.765625	26.7541261128339\\
-0.763888888888889	21.2342319560156\\
-0.762152777777778	21.9015144418187\\
-0.760416666666667	24.1838714328503\\
-0.758680555555556	26.8240060601334\\
-0.756944444444444	28.2102927105915\\
-0.755208333333333	25.4784822702558\\
-0.753472222222222	20.8948101875025\\
-0.751736111111111	30.8938888503085\\
-0.75	39.9437338310345\\
-0.748263888888889	43.7986547456719\\
-0.746527777777778	43.175372043949\\
-0.744791666666667	38.2996999083387\\
-0.743055555555556	33.476554612245\\
-0.741319444444444	38.0734955575617\\
-0.739583333333333	42.5219409836173\\
-0.737847222222222	43.0180850670102\\
-0.736111111111111	40.2802561778445\\
-0.734375	36.7438026165637\\
-0.732638888888889	36.0980597031978\\
-0.730902777777778	36.9337325710932\\
-0.729166666666667	36.4130714022096\\
-0.727430555555556	34.051420007944\\
-0.725694444444444	29.904217956032\\
-0.723958333333333	26.0177891650921\\
-0.722222222222222	29.9511941385545\\
-0.720486111111111	36.1768178280161\\
-0.71875	38.7392727937477\\
-0.717013888888889	36.5951337794552\\
-0.715277777777778	27.657334010187\\
-0.713541666666667	-1.90578371657854\\
-0.711805555555556	19.3698703799366\\
-0.710069444444444	25.6671870032853\\
-0.708333333333333	21.263665314594\\
-0.706597222222222	15.0324968791124\\
-0.704861111111111	25.8069702498406\\
-0.703125	31.250363267877\\
-0.701388888888889	30.2003822064968\\
-0.699652777777778	22.7479242929058\\
-0.697916666666667	17.9245337129378\\
-0.696180555555556	28.7122746492557\\
-0.694444444444444	35.8943227315522\\
-0.692708333333333	39.8636189812539\\
-0.690972222222222	41.7735490675264\\
-0.689236111111111	41.6987156587231\\
-0.6875	39.1447395513473\\
-0.685763888888889	33.0230612981062\\
-0.684027777777778	19.8744746838955\\
-0.682291666666667	2.28603738645183\\
-0.680555555555556	24.6737638280196\\
-0.678819444444444	33.6195110511391\\
-0.677083333333333	38.7145951904334\\
-0.675347222222222	41.8787447132209\\
-0.673611111111111	43.4171839843437\\
-0.671875	43.0955050291221\\
-0.670138888888889	40.4220816117727\\
-0.668402777777778	34.7412445781179\\
-0.666666666666667	26.4062585371604\\
-0.664930555555556	22.8139019582652\\
-0.663194444444444	24.2552266516366\\
-0.661458333333333	21.3495002329006\\
-0.659722222222222	17.4346338717805\\
-0.657986111111111	26.3694866693097\\
-0.65625	33.8644706700163\\
-0.654513888888889	37.085940248816\\
-0.652777777777778	37.1932092758839\\
-0.651041666666667	34.8568573121607\\
-0.649305555555556	30.2670265741964\\
-0.647569444444444	25.5115168727896\\
-0.645833333333333	28.7514253016308\\
-0.644097222222222	34.9425875672776\\
-0.642361111111111	37.8901269040001\\
-0.640625	36.9263856404287\\
-0.638888888888889	31.7677610964376\\
-0.637152777777778	23.4487283444542\\
-0.635416666666667	22.6968013524026\\
-0.633680555555556	27.3490507150784\\
-0.631944444444444	29.5743455440457\\
-0.630208333333333	30.1016199715567\\
-0.628472222222222	28.7723695516196\\
-0.626736111111111	24.3373481116073\\
-0.625	24.7241314537698\\
-0.623263888888889	34.5303805666388\\
-0.621527777777778	40.9426195089847\\
-0.619791666666667	42.9033897064986\\
-0.618055555555556	40.6841403079755\\
-0.616319444444444	34.4812144217061\\
-0.614583333333333	30.2388772464469\\
-0.612847222222222	34.6431365836432\\
-0.611111111111111	37.0113111098809\\
-0.609375	34.0383909225794\\
-0.607638888888889	22.80057382526\\
-0.605902777777778	6.81186913264469\\
-0.604166666666667	26.4445692592397\\
-0.602430555555556	31.5681614360497\\
-0.600694444444444	29.0931949684172\\
-0.598958333333333	17.5128977946554\\
-0.597222222222222	15.957089364896\\
-0.595486111111111	29.037986295831\\
-0.59375	32.7883106298409\\
-0.592013888888889	29.8603641971872\\
-0.590277777777778	18.5378681556047\\
-0.588541666666667	7.38507088621085\\
-0.586805555555556	16.7632207959056\\
-0.585069444444444	7.01623687964966\\
-0.583333333333333	14.0525507724429\\
-0.581597222222222	34.2532623772292\\
-0.579861111111111	41.6015952963455\\
-0.578125	42.2232237483777\\
-0.576388888888889	34.797471710024\\
-0.574652777777778	11.0512742586325\\
-0.572916666666667	38.8953891071836\\
-0.571180555555556	48.1169290968844\\
-0.569444444444444	50.1680739250564\\
-0.567708333333333	46.7466895334426\\
-0.565972222222222	38.0662289704679\\
-0.564236111111111	38.3488304849885\\
-0.5625	46.1955097187628\\
-0.560763888888889	48.5708285396447\\
-0.559027777777778	45.2002697330301\\
-0.557291666666667	32.8307506025631\\
-0.555555555555556	13.8774575913107\\
-0.553819444444444	39.2804116266535\\
-0.552083333333333	46.1348794058\\
-0.550347222222222	47.3434614396668\\
-0.548611111111111	44.7425388471948\\
-0.546875	39.4367091797122\\
-0.545138888888889	37.0858439097882\\
-0.543402777777778	41.0703729717654\\
-0.541666666666667	43.9847171812651\\
-0.539930555555556	43.9000204909639\\
-0.538194444444444	42.1778193104484\\
-0.536458333333333	41.4456119837057\\
-0.534722222222222	41.0897539063101\\
-0.532986111111111	36.8932171663574\\
-0.53125	20.6231607249856\\
-0.529513888888889	26.9024426190249\\
-0.527777777777778	41.8518037060212\\
-0.526041666666667	46.6985230533551\\
-0.524305555555556	46.3237445571951\\
-0.522569444444444	42.1442069355108\\
-0.520833333333333	36.6895995586151\\
-0.519097222222222	33.7289546726056\\
-0.517361111111111	30.5378824264356\\
-0.515625	28.2652774383283\\
-0.513888888888889	34.6926369018509\\
-0.512152777777778	39.2759427734406\\
-0.510416666666667	37.9258697983834\\
-0.508680555555556	27.677488957478\\
-0.506944444444444	8.23126440331368\\
-0.505208333333333	30.5539454865244\\
-0.503472222222222	34.6028265913429\\
-0.501736111111111	28.8555873244908\\
-0.5	19.5918799210642\\
-0.498263888888889	33.5124988968742\\
-0.496527777777778	39.879330896835\\
-0.494791666666667	39.4984228162581\\
-0.493055555555556	32.3587928559509\\
-0.491319444444444	9.29909843076401\\
-0.489583333333333	15.953720111865\\
-0.487847222222222	27.3508978679957\\
-0.486111111111111	27.9411259835162\\
-0.484375	22.6566298632986\\
-0.482638888888889	14.6673974365314\\
-0.480902777777778	20.5890767273302\\
-0.479166666666667	29.7975225620311\\
-0.477430555555556	35.5930701953224\\
-0.475694444444444	39.3981905573391\\
-0.473958333333333	42.4693286652398\\
-0.472222222222222	44.8369093378969\\
-0.470486111111111	45.4846186896769\\
-0.46875	43.1997744446466\\
-0.467013888888889	36.3700547910445\\
-0.465277777777778	20.9104717557643\\
-0.463541666666667	12.9556981972628\\
-0.461805555555556	22.0403311668813\\
-0.460069444444444	16.9080612622566\\
-0.458333333333333	10.3001006994216\\
-0.456597222222222	24.7678203586779\\
-0.454861111111111	28.334976868636\\
-0.453125	20.5479879173159\\
-0.451388888888889	9.21102000688122\\
-0.449652777777778	32.8861729654884\\
-0.447916666666667	41.2678990933306\\
-0.446180555555556	43.8150082473711\\
-0.444444444444444	42.4559417492161\\
-0.442708333333333	37.7751110634803\\
-0.440972222222222	29.7168222979761\\
-0.439236111111111	16.4956182828027\\
-0.4375	11.1845989383497\\
-0.435763888888889	27.325764682414\\
-0.434027777777778	36.3154855173659\\
-0.432291666666667	41.1474053170754\\
-0.430555555555556	43.0831397854143\\
-0.428819444444444	42.6148743833793\\
-0.427083333333333	40.0040188667369\\
-0.425347222222222	35.4047001536561\\
-0.423611111111111	28.1178846597087\\
-0.421875	11.9082182345348\\
-0.420138888888889	12.4788752334926\\
-0.418402777777778	31.1467830754025\\
-0.416666666666667	38.2153746045407\\
-0.414930555555556	39.937333530554\\
-0.413194444444444	37.8385006507562\\
-0.411458333333333	34.941932499199\\
-0.409722222222222	36.2881303556672\\
-0.407986111111111	38.3732219564659\\
-0.40625	36.8980630141182\\
-0.404513888888889	30.5299847852139\\
-0.402777777777778	19.4418242180417\\
-0.401041666666667	12.5841926320697\\
-0.399305555555556	8.63630788443037\\
-0.397569444444444	20.0046029659474\\
-0.395833333333333	32.7281916808985\\
-0.394097222222222	39.4902242548065\\
-0.392361111111111	42.9111141940223\\
-0.390625	44.3568490130099\\
-0.388888888888889	43.2784874714784\\
-0.387152777777778	36.1326020437513\\
-0.385416666666667	0.238706599846463\\
-0.383680555555556	42.1920585065471\\
-0.381944444444444	53.6700166150624\\
-0.380208333333333	58.0326832597532\\
-0.378472222222222	57.4798622467229\\
-0.376736111111111	51.3928571793399\\
-0.375	37.1485798087826\\
-0.373263888888889	43.0404152144609\\
-0.371527777777778	52.3910634066377\\
-0.369791666666667	54.4646008803332\\
-0.368055555555556	51.7654663051508\\
-0.366319444444444	47.7337319718119\\
-0.364583333333333	50.1018406320175\\
-0.362847222222222	54.5363625399502\\
-0.361111111111111	55.9603072688733\\
-0.359375	53.8761191118732\\
-0.357638888888889	48.3632421982812\\
-0.355902777777778	40.6916021403797\\
-0.354166666666667	34.6092913308651\\
-0.352430555555556	24.9441988983217\\
-0.350694444444444	25.7639451591033\\
-0.348958333333333	45.1844634229814\\
-0.347222222222222	53.713631969421\\
-0.345486111111111	56.4965084844654\\
-0.34375	54.3229183972381\\
-0.342013888888889	45.450104465206\\
-0.340277777777778	19.996813428253\\
-0.338541666666667	39.282520201416\\
-0.336805555555556	46.8195205745039\\
-0.335069444444444	46.1199126937837\\
-0.333333333333333	39.4872503335907\\
-0.331597222222222	27.3497106486941\\
-0.329861111111111	18.8202151479781\\
-0.328125	29.2551257507832\\
-0.326388888888889	39.9655565018241\\
-0.324652777777778	45.1707712221981\\
-0.322916666666667	44.4006418301437\\
-0.321180555555556	35.5592365698392\\
-0.319444444444444	30.9592699211306\\
-0.317708333333333	44.8902405173543\\
-0.315972222222222	49.9329830278588\\
-0.314236111111111	48.4700055914043\\
-0.3125	39.9271526498026\\
-0.310763888888889	27.1029793391346\\
-0.309027777777778	38.8134157277933\\
-0.307291666666667	44.6935000114505\\
-0.305555555555556	45.3620317581829\\
-0.303819444444444	42.9031978213024\\
-0.302083333333333	36.5802800742809\\
-0.300347222222222	16.8926882160702\\
-0.298611111111111	27.398613815379\\
-0.296875	42.1511727343944\\
-0.295138888888889	46.7437067790995\\
-0.293402777777778	45.8308974081\\
-0.291666666666667	40.9681887814536\\
-0.289930555555556	37.6398335605049\\
-0.288194444444444	39.9881498618083\\
-0.286458333333333	41.0666054857353\\
-0.284722222222222	39.0485438646916\\
-0.282986111111111	35.5369847147092\\
-0.28125	34.4635178014552\\
-0.279513888888889	39.7861825918081\\
-0.277777777777778	46.2821880363452\\
-0.276041666666667	49.4726860764818\\
-0.274305555555556	47.9476358458005\\
-0.272569444444444	40.1284362741472\\
-0.270833333333333	36.0210515183752\\
-0.269097222222222	46.6908627760293\\
-0.267361111111111	51.046821001318\\
-0.265625	49.1186604670006\\
-0.263888888888889	39.7228698728937\\
-0.262152777777778	13.3689432723283\\
-0.260416666666667	27.6280497427817\\
-0.258680555555556	28.9121725448748\\
-0.256944444444444	11.7653871592289\\
-0.255208333333333	29.2670176292091\\
-0.253472222222222	37.6722144546383\\
-0.251736111111111	36.8683373273628\\
-0.25	35.2390351253168\\
-0.248263888888889	41.6274891533232\\
-0.246527777777778	44.9553684113738\\
-0.244791666666667	41.2150352866877\\
-0.243055555555556	32.0586157374927\\
-0.241319444444444	42.7085775938685\\
-0.239583333333333	50.4499753492526\\
-0.237847222222222	51.4152958649616\\
-0.236111111111111	46.1275533834335\\
-0.234375	31.7276279729858\\
-0.232638888888889	25.6128098693316\\
-0.230902777777778	34.2378018089679\\
-0.229166666666667	30.5202529067973\\
-0.227430555555556	6.72180743947297\\
-0.225694444444444	18.1056669567453\\
-0.223958333333333	29.0639231310871\\
-0.222222222222222	33.6588409686418\\
-0.220486111111111	37.7355454975028\\
-0.21875	37.8546423053091\\
-0.217013888888889	28.6869882130817\\
-0.215277777777778	24.9600584265126\\
-0.213541666666667	43.1913190414055\\
-0.211805555555556	49.9399793775552\\
-0.210069444444444	50.464792229714\\
-0.208333333333333	44.3551546916738\\
-0.206597222222222	10.7653013793055\\
-0.204861111111111	46.0084211944769\\
-0.203125	60.8599039016938\\
-0.201388888888889	68.3354454260576\\
-0.199652777777778	71.4207214199165\\
-0.197916666666667	70.2049299451182\\
-0.196180555555556	63.1103898647545\\
-0.194444444444444	52.9206347381438\\
-0.192708333333333	65.4231573083143\\
-0.190972222222222	72.6161004075729\\
-0.189236111111111	73.2368257418213\\
-0.1875	67.0736757060629\\
-0.185763888888889	54.0806662080691\\
-0.184027777777778	67.8281255095975\\
-0.182291666666667	76.9519458020874\\
-0.180555555555556	80.0696409532286\\
-0.178819444444444	79.1965845192551\\
-0.177083333333333	74.9504642823048\\
-0.175347222222222	67.7351541138127\\
-0.173611111111111	59.498138758558\\
-0.171875	55.0459696850377\\
-0.170138888888889	55.2065494870062\\
-0.168402777777778	58.0929266559414\\
-0.166666666666667	60.868177622303\\
-0.164930555555556	61.0592931022421\\
-0.163194444444444	57.5913296806971\\
-0.161458333333333	49.703164958843\\
-0.159722222222222	41.1004802955923\\
-0.157986111111111	45.2968833382513\\
-0.15625	49.719867610399\\
-0.154513888888889	50.6280133998348\\
-0.152777777777778	49.0745201626069\\
-0.151041666666667	45.6040770519079\\
-0.149305555555556	40.3783823638754\\
-0.147569444444444	34.0238669547184\\
-0.145833333333333	27.8904493401761\\
-0.144097222222222	20.7823729651404\\
-0.142361111111111	5.31302858933854\\
-0.140625	5.72493084823126\\
-0.138888888888889	24.5741298611626\\
-0.137152777777778	34.8532361209687\\
-0.135416666666667	41.3967768406619\\
-0.133680555555556	44.504253936682\\
-0.131944444444444	43.5890324807535\\
-0.130208333333333	36.8057953067246\\
-0.128472222222222	25.2790571155511\\
-0.126736111111111	39.1837174275127\\
-0.125	47.6697856370364\\
-0.123263888888889	49.9003279313666\\
-0.121527777777778	46.8951577935959\\
-0.119791666666667	36.7363290709486\\
-0.118055555555556	15.0750464865859\\
-0.116319444444444	32.1951734771344\\
-0.114583333333333	36.1780716921481\\
-0.112847222222222	29.6151199502917\\
-0.111111111111111	10.2352513332948\\
-0.109375	30.2690899887276\\
-0.107638888888889	39.0321492960094\\
-0.105902777777778	43.1753540750047\\
-0.104166666666667	45.7606533959507\\
-0.102430555555556	46.0983587542998\\
-0.100694444444444	42.1279771693443\\
-0.0989583333333334	31.4677851303792\\
-0.0972222222222222	31.1447957025148\\
-0.0954861111111112	39.1610077871036\\
-0.09375	38.3368393771607\\
-0.0920138888888888	30.406428618395\\
-0.0902777777777778	36.8773722492261\\
-0.0885416666666666	44.1395494727921\\
-0.0868055555555556	43.4458884836396\\
-0.0850694444444444	36.0773030508722\\
-0.0833333333333334	39.6407322821122\\
-0.0815972222222222	46.4392381995951\\
-0.0798611111111112	44.5874933931037\\
-0.078125	27.6235813677935\\
-0.0763888888888888	40.7900510537193\\
-0.0746527777777778	52.5502354059437\\
-0.0729166666666666	54.3132667249169\\
-0.0711805555555556	47.7329934344922\\
-0.0694444444444444	28.9509368412552\\
-0.0677083333333334	42.4322610808664\\
-0.0659722222222222	46.6552714118639\\
-0.0642361111111112	35.9115831997318\\
-0.0625	37.2411383128237\\
-0.0607638888888888	53.59954385605\\
-0.0590277777777778	57.5695769899623\\
-0.0572916666666666	53.4587896260693\\
-0.0555555555555556	36.9590091686711\\
-0.0538194444444444	40.000155119053\\
-0.0520833333333334	49.1948906742157\\
-0.0503472222222222	46.0880090016509\\
-0.0486111111111112	40.0059893716729\\
-0.046875	48.5868945575779\\
-0.0451388888888888	49.7120998113074\\
-0.0434027777777778	38.0042316617087\\
-0.0416666666666666	52.7456768933716\\
-0.0399305555555556	64.6110081812053\\
-0.0381944444444444	68.1763371771415\\
-0.0364583333333334	65.7741737348137\\
-0.0347222222222222	59.0809043646687\\
-0.0329861111111112	58.4357743627735\\
-0.03125	60.7358778739066\\
-0.0295138888888888	54.17460442986\\
-0.0277777777777778	25.3245582425924\\
-0.0260416666666666	57.5920429253681\\
-0.0243055555555556	62.6538353452426\\
-0.0225694444444444	55.2113416350826\\
-0.0208333333333334	58.3412139221842\\
-0.0190972222222222	72.3269529684433\\
-0.0173611111111112	76.3254372729932\\
-0.015625	71.1880838062317\\
-0.0138888888888888	39.7441070280803\\
-0.0121527777777778	72.1335573433277\\
-0.0104166666666666	80.647499738312\\
-0.00868055555555558	77.206742682269\\
-0.00694444444444442	28.5940508388006\\
-0.00520833333333337	86.9356222593333\\
-0.00347222222222221	99.3546865164475\\
-0.00173611111111116	104.916239640736\\
0	106.270025274313\\
0.00173611111111116	103.869392126903\\
0.00347222222222232	96.9878122450449\\
0.00520833333333326	81.9593574085695\\
0.00694444444444442	51.4463540334783\\
0.00868055555555558	77.3518282697168\\
0.0104166666666667	77.6575082593111\\
0.0121527777777777	65.4072780135382\\
0.0138888888888888	44.8957044644134\\
0.015625	66.1601528490366\\
0.0173611111111112	64.2190164561235\\
0.0190972222222223	31.1661288539797\\
0.0208333333333333	64.6768059452071\\
0.0225694444444444	74.2914363145473\\
0.0243055555555556	74.7897271650177\\
0.0260416666666667	67.9106865154178\\
0.0277777777777777	49.0918903191747\\
0.0295138888888888	53.0925752595507\\
0.03125	59.4992234579738\\
0.0329861111111112	55.5519926187966\\
0.0347222222222223	50.8840679852838\\
0.0364583333333333	54.4280558372745\\
0.0381944444444444	51.9226706338644\\
0.0399305555555556	42.5940416454611\\
0.0416666666666667	55.3498233184051\\
0.0434027777777777	62.2202757372703\\
0.0451388888888888	60.65088439832\\
0.046875	47.4518004360119\\
0.0486111111111112	37.7678446775946\\
0.0503472222222223	56.249663971724\\
0.0520833333333333	58.3599456567083\\
0.0538194444444444	52.4582146428132\\
0.0555555555555556	49.7887043142252\\
0.0572916666666667	56.2376329729378\\
0.0590277777777777	56.3559498425179\\
0.0607638888888888	46.7307338587001\\
0.0625	38.3472588704785\\
0.0642361111111112	52.085764210135\\
0.0659722222222223	53.489395270521\\
0.0677083333333333	42.7417426116947\\
0.0694444444444444	39.4833773425451\\
0.0711805555555556	55.1726160326041\\
0.0729166666666667	58.9983241191787\\
0.0746527777777777	55.9349188453736\\
0.0763888888888888	46.7029959991591\\
0.078125	37.7505704460538\\
0.0798611111111112	33.7349524772734\\
0.0815972222222223	-2.33547780651535\\
0.0833333333333333	43.3040434784886\\
0.0850694444444444	52.9925714015417\\
0.0868055555555556	53.4366870679633\\
0.0885416666666667	43.8056707206489\\
0.0902777777777777	28.5971192157212\\
0.0920138888888888	50.8258388613289\\
0.09375	56.6296349380428\\
0.0954861111111112	55.5742500985406\\
0.0972222222222223	49.957701935454\\
0.0989583333333333	45.3880021583315\\
0.100694444444444	45.7416606583973\\
0.102430555555556	44.4121451695728\\
0.104166666666667	43.8756243032155\\
0.105902777777778	47.8164116535837\\
0.107638888888889	49.384979624087\\
0.109375	44.9448478539511\\
0.111111111111111	29.0220408357743\\
0.112847222222222	26.193235763382\\
0.114583333333333	39.2211696006854\\
0.116319444444444	39.4363687896691\\
0.118055555555556	30.0138424081673\\
0.119791666666667	-3.17872492687083\\
0.121527777777778	30.3197226628239\\
0.123263888888889	40.7375024152103\\
0.125	46.0674157877589\\
0.126736111111111	47.8090539749482\\
0.128472222222222	45.5994873920212\\
0.130208333333333	38.9991297669795\\
0.131944444444444	32.0375688521418\\
0.133680555555556	35.0879245375884\\
0.135416666666667	40.4913119756271\\
0.137152777777778	45.6583176530179\\
0.138888888888889	48.8412762093753\\
0.140625	47.5147557821872\\
0.142361111111111	37.8084420674366\\
0.144097222222222	30.0273299137158\\
0.145833333333333	47.6167559940001\\
0.147569444444444	53.4871522704896\\
0.149305555555556	52.6626442403147\\
0.151041666666667	45.4757275169601\\
0.152777777777778	33.4972559993749\\
0.154513888888889	37.4585387058445\\
0.15625	41.506790536384\\
0.157986111111111	45.285536662257\\
0.159722222222222	52.5585163793944\\
0.161458333333333	58.0934917503022\\
0.163194444444444	60.5230619606637\\
0.164930555555556	60.4518801897482\\
0.166666666666667	58.5504940352395\\
0.168402777777778	54.859738094086\\
0.170138888888889	48.5607513228859\\
0.171875	39.4759589334377\\
0.173611111111111	32.7057065158126\\
0.175347222222222	36.4075472106275\\
0.177083333333333	51.3491146422514\\
0.178819444444444	62.4416595361096\\
0.180555555555556	68.5452152620868\\
0.182291666666667	70.3270089974361\\
0.184027777777778	67.6217695815312\\
0.185763888888889	59.3429333212799\\
0.1875	52.9502761892786\\
0.189236111111111	60.9523572151984\\
0.190972222222222	63.4806618563598\\
0.192708333333333	57.4727253739759\\
0.194444444444444	15.6098384651919\\
0.196180555555556	57.1653739008962\\
0.197916666666667	67.9357873701204\\
0.199652777777778	70.8920510569517\\
0.201388888888889	68.9861610497494\\
0.203125	62.1529673588774\\
0.204861111111111	47.1951460141896\\
0.206597222222222	21.1504064870906\\
0.208333333333333	42.7256288420453\\
0.210069444444444	44.9653668128747\\
0.211805555555556	42.0545369404302\\
0.213541666666667	42.6253941686373\\
0.215277777777778	47.7202776050753\\
0.217013888888889	50.8733275299687\\
0.21875	49.9342746032426\\
0.220486111111111	42.1332371068729\\
0.222222222222222	13.6439598890532\\
0.223958333333333	45.0749122410296\\
0.225694444444444	55.4688060712486\\
0.227430555555556	59.1571636820911\\
0.229166666666667	58.7678652664657\\
0.230902777777778	54.938735982383\\
0.232638888888889	47.9999989112618\\
0.234375	41.2837016923738\\
0.236111111111111	42.923806173311\\
0.237847222222222	47.7042198512841\\
0.239583333333333	51.4195986266345\\
0.241319444444444	54.0491844074489\\
0.243055555555556	55.5839872810933\\
0.244791666666667	55.6917470883212\\
0.246527777777778	53.8751100445182\\
0.248263888888889	49.509379222343\\
0.25	41.8893401527037\\
0.251736111111111	32.6341123269489\\
0.253472222222222	32.812582447158\\
0.255208333333333	37.2358133673727\\
0.256944444444444	39.4749416481657\\
0.258680555555556	39.7405064492926\\
0.260416666666667	37.8181264739229\\
0.262152777777778	32.9198474318257\\
0.263888888888889	24.41411451286\\
0.265625	13.8657272501982\\
0.267361111111111	11.2191011492488\\
0.269097222222222	20.8405162974048\\
0.270833333333333	29.0871145932915\\
0.272569444444444	32.9111015983444\\
0.274305555555556	33.7964525717767\\
0.276041666666667	35.1459124060278\\
0.277777777777778	37.6734488928816\\
0.279513888888889	37.8117406125461\\
0.28125	33.1136857362809\\
0.282986111111111	28.8382724837018\\
0.284722222222222	36.8252848455659\\
0.286458333333333	41.6280893500288\\
0.288194444444444	40.3182221161844\\
0.289930555555556	33.8051737659963\\
0.291666666666667	35.2062488953513\\
0.293402777777778	42.4433286334256\\
0.295138888888889	44.3156986156416\\
0.296875	40.4861593839746\\
0.298611111111111	32.4777449678425\\
0.300347222222222	33.3228623499832\\
0.302083333333333	37.1430610268595\\
0.303819444444444	34.1356758501395\\
0.305555555555556	19.8302165434201\\
0.307291666666667	23.305357145047\\
0.309027777777778	35.7514279096898\\
0.310763888888889	38.9677527808758\\
0.3125	36.2963031940124\\
0.314236111111111	25.0040614414432\\
0.315972222222222	6.48892654996795\\
0.317708333333333	35.4489329655293\\
0.319444444444444	44.4377081220878\\
0.321180555555556	47.5471208053423\\
0.322916666666667	47.0334101912646\\
0.324652777777778	45.7430181699113\\
0.326388888888889	47.4519177210922\\
0.328125	49.7844304575237\\
0.329861111111111	49.0753956276658\\
0.331597222222222	43.6933467290369\\
0.333333333333333	34.0844446396202\\
0.335069444444444	36.1817623954282\\
0.336805555555556	42.5117459324498\\
0.338541666666667	45.338400342486\\
0.340277777777778	47.8335617133011\\
0.342013888888889	51.2101041193076\\
0.34375	54.0694068117258\\
0.345486111111111	55.7169345678187\\
0.347222222222222	56.1079131499855\\
0.348958333333333	54.6628064227225\\
0.350694444444444	49.7634363579631\\
0.352430555555556	38.3226704211306\\
0.354166666666667	37.0317467128274\\
0.355902777777778	48.343446105772\\
0.357638888888889	51.8247923424645\\
0.359375	49.5258962302258\\
0.361111111111111	43.9433686704098\\
0.362847222222222	44.8107710941938\\
0.364583333333333	48.3742516592462\\
0.366319444444444	46.4868647759521\\
0.368055555555556	37.5700427873818\\
0.369791666666667	40.8410948757651\\
0.371527777777778	50.4134797249538\\
0.373263888888889	52.9417671287571\\
0.375	48.8945108794276\\
0.376736111111111	36.8078924398235\\
0.378472222222222	41.9473841492396\\
0.380208333333333	51.9313480741255\\
0.381944444444444	54.7480845274426\\
0.383680555555556	52.1811768330774\\
0.385416666666667	43.4413297289044\\
0.387152777777778	31.0350570063027\\
0.388888888888889	41.4784085137486\\
0.390625	47.545607355011\\
0.392361111111111	47.7687098985991\\
0.394097222222222	43.0163313355557\\
0.395833333333333	32.7510885870017\\
0.397569444444444	32.2583631486713\\
0.399305555555556	41.919914564974\\
0.401041666666667	46.4888658737267\\
0.402777777777778	46.9738680750246\\
0.404513888888889	43.6819367207709\\
0.40625	35.5460997245903\\
0.407986111111111	22.2792328195642\\
0.409722222222222	27.2747745135796\\
0.411458333333333	33.1190836870962\\
0.413194444444444	34.26074692157\\
0.414930555555556	37.2591066483547\\
0.416666666666667	42.6697119962845\\
0.418402777777778	45.6135683319189\\
0.420138888888889	44.2386664086207\\
0.421875	36.7413671021154\\
0.423611111111111	16.188515360605\\
0.425347222222222	26.2214894993506\\
0.427083333333333	34.7845691774733\\
0.428819444444444	36.9399693072057\\
0.430555555555556	41.5976193606723\\
0.432291666666667	47.9162070731512\\
0.434027777777778	51.7105397869117\\
0.435763888888889	52.4766734813943\\
0.4375	50.850612920876\\
0.439236111111111	47.8712376351846\\
0.440972222222222	44.0256470676118\\
0.442708333333333	37.3590864201488\\
0.444444444444444	19.6919369804335\\
0.446180555555556	18.1225248451317\\
0.447916666666667	34.0401108625715\\
0.449652777777778	36.1825623348511\\
0.451388888888889	29.8224694697537\\
0.453125	16.936242780729\\
0.454861111111111	27.8813758075914\\
0.456597222222222	32.7691563485701\\
0.458333333333333	30.4952904340347\\
0.460069444444444	30.538549826086\\
0.461805555555556	37.5391079842241\\
0.463541666666667	40.8283329139088\\
0.465277777777778	38.0235143131273\\
0.467013888888889	26.3252194904371\\
0.46875	27.8412865678307\\
0.470486111111111	40.4789220555472\\
0.472222222222222	44.811972857137\\
0.473958333333333	43.6654730827894\\
0.475694444444444	36.7663734694488\\
0.477430555555556	20.347540864947\\
0.479166666666667	26.2158675581247\\
0.480902777777778	36.7791656919614\\
0.482638888888889	40.378313358689\\
0.484375	40.361961821251\\
0.486111111111111	37.8415389279535\\
0.487847222222222	33.1184111385208\\
0.489583333333333	26.4594642752714\\
0.491319444444444	21.5612613882208\\
0.493055555555556	24.3448238560535\\
0.494791666666667	28.7245263340136\\
0.496527777777778	31.4052420805628\\
0.498263888888889	32.7252985262515\\
0.5	33.2398197801133\\
0.501736111111111	32.8233250734996\\
0.503472222222222	30.4368912948275\\
0.505208333333333	25.4138341540699\\
0.506944444444444	27.6436995821061\\
0.508680555555556	38.1601470578881\\
0.510416666666667	45.2271300581939\\
0.512152777777778	48.2753738486999\\
0.513888888888889	47.4195208943824\\
0.515625	42.0727655606881\\
0.517361111111111	34.7890900598074\\
0.519097222222222	40.9168404243667\\
0.520833333333333	47.7787623337137\\
0.522569444444444	50.1744328814494\\
0.524305555555556	49.2941966737207\\
0.526041666666667	46.9823506111689\\
0.527777777777778	45.8373604522948\\
0.529513888888889	45.951875901169\\
0.53125	45.2415108933968\\
0.532986111111111	43.0481988818964\\
0.534722222222222	39.7369321311658\\
0.536458333333333	34.6972885364035\\
0.538194444444444	27.5610819644595\\
0.539930555555556	31.3050775656829\\
0.541666666666667	39.6097419469505\\
0.543402777777778	43.3705877650999\\
0.545138888888889	43.8206955088825\\
0.546875	43.7507096898709\\
0.548611111111111	44.9477040246741\\
0.550347222222222	44.9464793517559\\
0.552083333333333	40.2991059397978\\
0.553819444444444	23.6361394920382\\
0.555555555555556	21.8610524591357\\
0.557291666666667	37.9896369669608\\
0.559027777777778	40.1567548737657\\
0.560763888888889	33.1706472258595\\
0.5625	6.66665279950134\\
0.564236111111111	29.7477870933284\\
0.565972222222222	36.465939154638\\
0.567708333333333	32.7087181674674\\
0.569444444444444	18.505587960554\\
0.571180555555556	32.410505200684\\
0.572916666666667	41.8727049215671\\
0.574652777777778	44.710203809771\\
0.576388888888889	44.0191431439988\\
0.578125	42.0358263851782\\
0.579861111111111	40.4199075102341\\
0.581597222222222	39.1842059969015\\
0.583333333333333	39.0909894746588\\
0.585069444444444	41.4541310543482\\
0.586805555555556	44.2108741194178\\
0.588541666666667	45.2609452721672\\
0.590277777777778	44.2354318964555\\
0.592013888888889	41.840732011287\\
0.59375	40.4754170258884\\
0.595486111111111	42.9233606444337\\
0.597222222222222	46.9990611601147\\
0.598958333333333	49.5716862575099\\
0.600694444444444	49.4059478849968\\
0.602430555555556	45.7145172567997\\
0.604166666666667	37.0347044754896\\
0.605902777777778	22.4293775716842\\
0.607638888888889	26.0859093152767\\
0.609375	29.9910882560963\\
0.611111111111111	24.2481992256133\\
0.612847222222222	6.33658466964399\\
0.614583333333333	28.4190272707991\\
0.616319444444444	39.0316202160869\\
0.618055555555556	43.8095388070695\\
0.619791666666667	45.0960164855736\\
0.621527777777778	43.434333105434\\
0.623263888888889	39.1056598995247\\
0.625	33.6609449292908\\
0.626736111111111	30.1676406927147\\
0.628472222222222	27.7638491770783\\
0.630208333333333	29.266891420417\\
0.631944444444444	37.0976502532751\\
0.633680555555556	42.7298825270656\\
0.635416666666667	44.1984690212427\\
0.637152777777778	41.7110523719201\\
0.638888888888889	37.1647068909511\\
0.640625	37.0205124986297\\
0.642361111111111	39.5767857828831\\
0.644097222222222	38.7998961310139\\
0.645833333333333	32.87937624981\\
0.647569444444444	19.55663192166\\
0.649305555555556	-7.96613747029475\\
0.651041666666667	-7.1252087585902\\
0.652777777777778	15.2499728139436\\
0.654513888888889	27.1851217938661\\
0.65625	30.9407428166931\\
0.657986111111111	28.5019220918049\\
0.659722222222222	28.0187668335053\\
0.661458333333333	36.7390614504317\\
0.663194444444444	42.4032216170134\\
0.664930555555556	43.3530658866008\\
0.666666666666667	40.069761340838\\
0.668402777777778	34.3214267882502\\
0.670138888888889	32.9096857102757\\
0.671875	34.3905999311854\\
0.673611111111111	31.758178577169\\
0.675347222222222	22.0203025366881\\
0.677083333333333	14.570423803268\\
0.678819444444444	24.6752683078133\\
0.680555555555556	27.2662170287442\\
0.682291666666667	22.5803085468113\\
0.684027777777778	4.61645131153063\\
0.685763888888889	9.09067452692609\\
0.6875	27.0455755293602\\
0.689236111111111	35.447141060525\\
0.690972222222222	40.164672920139\\
0.692708333333333	42.1992876098492\\
0.694444444444444	41.6507265612136\\
0.696180555555556	38.0917006664294\\
0.697916666666667	30.1191592865648\\
0.699652777777778	15.3975322600232\\
0.701388888888889	19.9596834411\\
0.703125	27.5340629111492\\
0.704861111111111	27.4256886795419\\
0.706597222222222	21.3717931306758\\
0.708333333333333	18.8962267591088\\
0.710069444444444	24.5881893503922\\
0.711805555555556	25.3694282950274\\
0.713541666666667	24.4660356240325\\
0.715277777777778	32.9513249980006\\
0.717013888888889	40.9242108314685\\
0.71875	44.5299890885176\\
0.720486111111111	44.0314902619071\\
0.722222222222222	38.9217021609861\\
0.723958333333333	25.2519099557328\\
0.725694444444444	6.26870106637396\\
0.727430555555556	32.5427536915564\\
0.729166666666667	39.9665888086249\\
0.730902777777778	41.6503372181016\\
0.732638888888889	38.3884347597772\\
0.734375	29.0196826819757\\
0.736111111111111	29.7436439724834\\
0.737847222222222	40.0772613265805\\
0.739583333333333	43.9557662792064\\
0.741319444444444	42.6190945601017\\
0.743055555555556	37.9036677340629\\
0.744791666666667	37.3304320233203\\
0.746527777777778	40.9947995024096\\
0.748263888888889	40.7865122642421\\
0.75	33.2682376584467\\
0.751736111111111	6.83941078574781\\
0.753472222222222	29.3964221266956\\
0.755208333333333	37.1617441454394\\
0.756944444444444	35.694099513896\\
0.758680555555556	25.4950644578419\\
0.760416666666667	22.0044891042462\\
0.762152777777778	33.3096223966344\\
0.763888888888889	36.8370698754291\\
0.765625	36.2378611007225\\
0.767361111111111	35.8361021863884\\
0.769097222222222	37.182430583416\\
0.770833333333333	37.1387647886354\\
0.772569444444444	34.7221646927996\\
0.774305555555556	34.0752748358603\\
0.776041666666667	37.5097057734947\\
0.777777777777778	39.4722033387341\\
0.779513888888889	37.1019452924317\\
0.78125	27.9481423955765\\
0.782986111111111	14.2096099742505\\
0.784722222222222	30.1761137456436\\
0.786458333333333	37.5649753107668\\
0.788194444444444	39.6471195250428\\
0.789930555555556	38.8446426946164\\
0.791666666666667	36.6201710701024\\
0.793402777777778	33.3587017269278\\
0.795138888888889	27.2271925688908\\
0.796875	17.558998322495\\
0.798611111111111	26.8947033498345\\
0.800347222222222	36.4902715851994\\
0.802083333333333	40.4485660156658\\
0.803819444444444	40.3278645107129\\
0.805555555555556	37.4195325873576\\
0.807291666666667	35.1854954113027\\
0.809027777777778	36.5151370326746\\
0.810763888888889	37.4991290169446\\
0.8125	35.0803255898627\\
0.814236111111111	27.6946775708084\\
0.815972222222222	12.5752612661447\\
0.817708333333333	8.9611628037112\\
0.819444444444444	16.6227189422081\\
0.821180555555556	16.69246467281\\
0.822916666666667	11.0550315966849\\
0.824652777777778	-0.952221530683781\\
0.826388888888889	16.258753271932\\
0.828125	29.2720553938337\\
0.829861111111111	35.7736031798617\\
0.831597222222222	37.6804080190675\\
0.833333333333333	35.0591588001206\\
0.835069444444444	25.9784721550263\\
0.836805555555556	-0.0990271206671718\\
0.838541666666667	19.2000545663351\\
0.840277777777778	27.5179058499144\\
0.842013888888889	28.9266226646757\\
0.84375	27.8831828873024\\
0.845486111111111	25.8206552661227\\
0.847222222222222	19.5422749495977\\
0.848958333333333	5.43152551436532\\
0.850694444444444	25.9091982091912\\
0.852430555555556	36.7571896040407\\
0.854166666666667	40.9866457472797\\
0.855902777777778	40.346383865211\\
0.857638888888889	34.3526531126412\\
0.859375	24.599301755842\\
0.861111111111111	32.3251228652463\\
0.862847222222222	39.7039109924153\\
0.864583333333333	41.4396675006762\\
0.866319444444444	38.4135552711769\\
0.868055555555556	30.7742808831571\\
0.869791666666667	26.0906697456738\\
0.871527777777778	31.636047021317\\
0.873263888888889	34.0785312630556\\
0.875	31.4274323473504\\
0.876736111111111	23.3805236683015\\
0.878472222222222	14.6325279987536\\
0.880208333333333	20.6497236111433\\
0.881944444444444	26.781817673594\\
0.883680555555556	31.1956677949753\\
0.885416666666667	34.3282275076896\\
0.887152777777778	34.9152675906799\\
0.888888888888889	31.3276987595295\\
0.890625	19.9398617825981\\
0.892361111111111	5.52818981223019\\
0.894097222222222	23.9686495414671\\
0.895833333333333	28.0409711522282\\
0.897569444444444	22.5069038066088\\
0.899305555555556	-18.8703658458009\\
0.901041666666667	23.6178297443943\\
0.902777777777778	34.6610610166829\\
0.904513888888889	38.2137438573227\\
0.90625	37.4735910562079\\
0.907986111111111	33.6743604551833\\
0.909722222222222	29.4824633416001\\
0.911458333333333	29.3909865367351\\
0.913194444444444	31.376957571714\\
0.914930555555556	31.9258732060924\\
0.916666666666667	30.5817190672318\\
0.918402777777778	28.6638421596451\\
0.920138888888889	28.5533926178056\\
0.921875	29.3266897152079\\
0.923611111111111	27.0825984562973\\
0.925347222222222	16.1642133504065\\
0.927083333333333	3.17629738102331\\
0.928819444444444	27.5230515032979\\
0.930555555555556	34.8998161985844\\
0.932291666666667	35.8277499763303\\
0.934027777777778	31.2336957945432\\
0.935763888888889	18.8277716781235\\
0.9375	-4.49185719280688\\
0.939236111111111	7.2248353960298\\
0.940972222222222	-5.74420218256626\\
0.942708333333333	8.62474226498956\\
0.944444444444444	23.7043883447636\\
0.946180555555556	27.0350802180079\\
0.947916666666667	21.7973235282227\\
0.949652777777778	17.6747788559772\\
0.951388888888889	32.2586393605232\\
0.953125	40.5135003817447\\
0.954861111111111	44.1216433465933\\
0.956597222222222	44.9014204721276\\
0.958333333333333	44.0913798448079\\
0.960069444444444	42.8263467452425\\
0.961805555555556	41.90108552055\\
0.963541666666667	41.3439176973584\\
0.965277777777778	40.4409838179171\\
0.967013888888889	38.1313763044994\\
0.96875	33.383501090157\\
0.970486111111111	25.6659732146321\\
0.972222222222222	17.543619848584\\
0.973958333333333	17.9049197159852\\
0.975694444444444	27.0726550124866\\
0.977430555555556	36.2554925775743\\
0.979166666666667	41.9850502431015\\
0.980902777777778	43.9222413111647\\
0.982638888888889	41.8084059922631\\
0.984375	34.3542816802897\\
0.986111111111111	16.7219709383352\\
0.987847222222222	17.9148386282714\\
0.989583333333333	26.5407823357651\\
0.991319444444444	24.7798791275957\\
0.993055555555556	15.382821310583\\
0.994791666666667	15.54810391338\\
0.996527777777778	23.3907354515927\\
0.998263888888889	25.682935926914\\
};
\addlegendentry{$ \log_{10} $ of Mean Centred };

\end{axis}
\end{tikzpicture}%}
	\caption{\textit{Balanced and Unbalanced Complex Voltages}}
	\label{fig:1_4_a}
\end{figure}

\end{document}

