\documentclass[./main.tex]{subfiles} 
\begin{document}

\section{Adaptive Signal Processing}

\subsection{The Least Mean Squares (LMS) Algorithm}

\subsubsection{Correlation Matrix}
The autocorrelation matrix is been defined as $ R \equiv E \begin{bmatrix} \mathbf{x}[n] \mathbf{x}^T[n] \end{bmatrix} $

\begin{equation}
R = E \begin{bmatrix}
x^2[n-1] & x[n-1] x[n-2] \\
x[n-1] x[n-2] & x^2[n-2]
\end{bmatrix}
\end{equation}
\label{eq:3_1_a_R}

We note that $x^2[n-1] = x^2[n-2] $ since this simply represents a shift in the time domain but will not change the expected value. 

We can square $x[n]$ to give us the expected value of the diagonals:
\begin{subequations}
\begin{align}
E[x[n-1]] &= E[ a_1 x[n - 2] + a_2 x[n-3] + \eta[n-1] ] \\
E[x^2[n-1]] &= E[ a_{1}^2 x^2[n - 2]] + E[ a_{2}^2 x^2[n-3]] + E[ 2 a_{1} a_{2} x[n-1] x[n-2]] + 
\sigma^2 \\
R_{xx}(0) = E[x^2[n-1]] &= a_1^2 R_{xx}(0) + a_2^2 R_{xx}(0) + 2 a_{1} a_{2} R_{xx}(1)
\end{align}
\end{subequations}

We get to the last line using the equality mentioned above, and we can see the make up of $R_{xx}(1)$ in equation \ref{eq:3_1_a_R}. We can conduct a similar process for $R_{xx}(1)$:

\begin{subequations}
\begin{align}
E[x[n-1]] &= E[ a_1 x[n - 2] + a_2 x[n-3] + \eta[n] ] \\
E[x[n-1]x[n-2]] &= E[ a_1 x^2[n - 2] + a_2 x[n-3]x[n-2] + \eta[n-1]x[n-2] ] \\
R_{xx}(1) = E[x[n-1]x[n-2]] &= a_1 R_{xx}(0) + a_2 R_{xx}(1) + 0
\end{align}
\end{subequations}

We then solve these two equations for $ R_{xx}(0) $ and $ R_{xx}(1) $, to determine that the autocorrelation matrix is 
$$ R = \begin{bmatrix}
\frac{25}{27} & \frac{25}{54} \\
 \frac{25}{54} & \frac{25}{27}
\end{bmatrix}
$$

\subsubsection{Implemented LMS Filter}

\subsubsection{Misadjustment}

\subsubsection{Steady State Coefficient Values}

\subsubsection{Leaky LMS Derivation}

\subsubsection{Leaky LMS Results}


\subsection{Adaptive Step Sizes}

\subsubsection{Implemented GASS Algorithms}

\subsubsection{NLMS Algorithm}

\subsubsection{GNGD Algorithm}

Maths and boringness ensue

% \begin{equation}
% S = \frac{1}{N}X' X'^T
% \end{equation}


%  \begin{figure}[h]
%  	\centering 
% 	\resizebox{0.6\textwidth}{!}{\input{q5/q5_cum.tikz}}
%   	\caption{\textit{Cumulative representation of the variance of each eigenvector}}
%   	\label{fig:q5_4}
%  \end{figure}

% \begin{figure}[h]
% 	\centering 
%  	\setlength\figureheight{0.4\textwidth}
% 	\setlength\figurewidth{0.7\textwidth} 
%  	\input{p_1/1.tikz}
%  	\caption{\textit{The four randomly generated subsets}}
%  	\label{fig:q1}
% \end{figure}


%  \begin{figure}[h]
%          \centering
%          \begin{subfigure}[b]{0.45\textwidth}
%             \resizebox{\textwidth}{!}{\input{part_4/q8_num_1.tikz}}
%   			\caption{\textit{1 Tree}}
%          \end{subfigure}
%          ~ %add desired spacing between images, e. g. ~, \quad, \qquad, \hfill etc.
%           %(or a blank line to force the subfigure onto a new line)
%          \begin{subfigure}[b]{0.45\textwidth}
%             \resizebox{\textwidth}{!}{\input{part_4/q8_num_3.tikz}}
%   			\caption{\textit{2 Trees}}
%          \end{subfigure}
 		
%          \begin{subfigure}[b]{0.45\textwidth}
%             \resizebox{\textwidth}{!}{\input{part_4/q8_num_5.tikz}}
%   			\caption{\textit{5 Trees}}
%          \end{subfigure}
%          ~ %add desired spacing between images, e. g. ~, \quad, \qquad, \hfill etc.
%           %(or a blank line to force the subfigure onto a new line)
%          \begin{subfigure}[b]{0.45\textwidth}
%             \resizebox{\textwidth}{!}{\input{part_4/q8_num_10.tikz}}
%   			\caption{\textit{10 Trees}}
%          \end{subfigure}
         
%          \begin{subfigure}[b]{0.45\textwidth}
%             \resizebox{\textwidth}{!}{\input{part_4/q8_num_20.tikz}}
%   			\caption{\textit{20 Trees}}
%          \end{subfigure}
 		
%  		\label{q9i}
% 		\caption{\textit{Varying the Number of Trees in the Forest}}
%  \end{figure}

\end{document}

