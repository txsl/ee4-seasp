\documentclass[./main.tex]{subfiles} 
\begin{document}

\section{Adaptive Signal Processing}

\subsection{The Least Mean Squares (LMS) Algorithm}

\subsubsection{Correlation Matrix}
The autocorrelation matrix is been defined as $ R \equiv E \begin{bmatrix} \mathbf{x}[n] \mathbf{x}^T[n] \end{bmatrix} $

\begin{equation}
R = E \begin{bmatrix}
x^2[n-1] & x[n-1] x[n-2] \\
x[n-1] x[n-2] & x^2[n-2]
\end{bmatrix}
\end{equation}
\label{eq:3_1_a_R}

We note that $x^2[n-1] = x^2[n-2] $ since this simply represents a shift in the time domain but will not change the expected value. 

We can square $x[n]$ to give us the expected value of the diagonals:
\begin{subequations}
\begin{align}
E[x[n-1]] &= E[ a_1 x[n - 2] + a_2 x[n-3] + \eta[n-1] ] \\
E[x^2[n-1]] &= E[ a_{1}^2 x^2[n - 2]] + E[ a_{2}^2 x^2[n-3]] + E[ 2 a_{1} a_{2} x[n-1] x[n-2]] + 
\sigma^2 \\
R_{xx}(0) = E[x^2[n-1]] &= a_1^2 R_{xx}(0) + a_2^2 R_{xx}(0) + 2 a_{1} a_{2} R_{xx}(1)
\end{align}
\end{subequations}

We get to the last line using the equality mentioned above, and we can see the make up of $R_{xx}(1)$ in equation \ref{eq:3_1_a_R}. We can conduct a similar process for $R_{xx}(1)$:

\begin{subequations}
\begin{align}
E[x[n-1]] &= E[ a_1 x[n - 2] + a_2 x[n-3] + \eta[n] ] \\
E[x[n-1]x[n-2]] &= E[ a_1 x^2[n - 2] + a_2 x[n-3]x[n-2] + \eta[n-1]x[n-2] ] \\
R_{xx}(1) = E[x[n-1]x[n-2]] &= a_1 R_{xx}(0) + a_2 R_{xx}(1) + 0
\end{align}
\end{subequations}

We then solve these two equations for $ R_{xx}(0) $ and $ R_{xx}(1) $, to determine that the autocorrelation matrix is 
$$ R = \begin{bmatrix}
\frac{25}{27} & \frac{25}{54} \\[0.3em]
 \frac{25}{54} & \frac{25}{27}
\end{bmatrix}
$$

In order for the filter to converge to the correct parameters, we must satisfy the bounds $ 0 < \mu < \frac{2}{\lambda_{max}} $. In this case, our eigenvalues are $\frac{25}{18}$ and $\frac{25}{54}$. Thus we know that $ 0 < \mu < \frac{108}{25} $ for the LMS to converge in the mean.

\subsubsection{Implemented LMS Filter}
100 iterations of the AR Process $ x[n] = 0.1 x[n-1] + 0.8 x[n-2] + \eta[n] $ have been generated, with 1000 samples per iteration. Figure \ref{fig:q3_1_b_indiv} shows one trial of this filter, whist figure \ref{fig:q3_1_b} shows the mean error taken across 100 iterations.

\begin{figure}[h]
	\centering 
 	\resizebox{\textwidth}{!}{% This file was created by matlab2tikz v0.4.7 (commit 56d6eb80eb584fc4c53cc717e4703ae91cd345df) running on MATLAB 8.4.
% Copyright (c) 2008--2014, Nico Schlömer <nico.schloemer@gmail.com>
% All rights reserved.
% Minimal pgfplots version: 1.3
% 
% The latest updates can be retrieved from
%   http://www.mathworks.com/matlabcentral/fileexchange/22022-matlab2tikz
% where you can also make suggestions and rate matlab2tikz.
% 
%
% defining custom colors
\definecolor{mycolor1}{rgb}{0.00000,0.44700,0.74100}%
\definecolor{mycolor2}{rgb}{0.85000,0.32500,0.09800}%
%
\begin{tikzpicture}

\begin{axis}[%
width=7in,
height=2in,
unbounded coords=jump,
scale only axis,
separate axis lines,
every outer x axis line/.append style={white!15!black},
every x tick label/.append style={font=\color{white!15!black}},
xmin=0,
xmax=1000,
xlabel={Iteration},
xmajorgrids,
every outer y axis line/.append style={white!15!black},
every y tick label/.append style={font=\color{white!15!black}},
ymin=-70,
ymax=10,
ylabel={Squared Prediction Error (dB)},
ymajorgrids,
title style={font=\bfseries},
title={LMS filter with differing step sizes},
legend style={draw=white!15!black,fill=white,legend cell align=left}
]
\addplot [color=mycolor1,solid,forget plot]
  table[row sep=crcr]{1	-inf\\
2	-inf\\
3	0.43025094568348\\
4	0.747768388313661\\
5	0.728481803698056\\
6	1.77388735992989\\
7	-4.77571444232561\\
8	-0.812489820780817\\
9	-10.1685015063846\\
10	3.44843698078326\\
11	-5.07312793898956\\
12	-1.69047749263029\\
13	-14.4580879687276\\
14	-1.04773816726005\\
15	-21.6479395066642\\
16	-8.98268800719441\\
17	-20.0442818006759\\
18	1.49543784373772\\
19	-3.08570967986311\\
20	-4.07373362244785\\
21	-17.8461663715512\\
22	-26.6733682017632\\
23	-7.79228639668468\\
24	-1.8033696208618\\
25	-1.73635625637501\\
26	-48.5544244032526\\
27	-7.08768367399008\\
28	-16.8098695762161\\
29	-0.402308741275068\\
30	-4.97532537140032\\
31	-1.74332589279984\\
32	-17.5557745572386\\
33	-0.599280374028303\\
34	-24.2139037115964\\
35	-23.4934670382709\\
36	-14.196694576188\\
37	-21.9367552979517\\
38	-10.4773398419843\\
39	-8.12298697600824\\
40	-9.95695524191986\\
41	-5.25852949541967\\
42	-20.9013451164015\\
43	-3.67976610653408\\
44	-13.3696050415124\\
45	-9.29179404440364\\
46	-12.7146401970181\\
47	-7.33983312993795\\
48	-6.10184259795077\\
49	-11.6938178321382\\
50	-11.9406199393383\\
51	-14.8794702646616\\
52	-26.6645083682667\\
53	-7.22631471447092\\
54	-16.5326913599465\\
55	-16.9516345429898\\
56	-19.3384505141226\\
57	-36.4302989440911\\
58	-14.0168825406718\\
59	-3.88729127147082\\
60	-0.93752431925949\\
61	-8.59790536526775\\
62	0.472334777063794\\
63	-2.46750435328361\\
64	0.0344982134548861\\
65	-19.7972220096808\\
66	-19.3509599185894\\
67	-6.45421505841203\\
68	-3.53884713676357\\
69	0.553287694399403\\
70	-1.72287139752389\\
71	0.129095627889827\\
72	3.01203175137825\\
73	2.45594180655716\\
74	0.241972031953262\\
75	7.71803718207402\\
76	2.1314079942065\\
77	5.61694439515334\\
78	-5.15667482059079\\
79	6.13529606990213\\
80	-2.87189268992671\\
81	-4.3615437364008\\
82	-0.114142061021746\\
83	-16.6719484236226\\
84	-0.174205436252918\\
85	-21.2975498474651\\
86	-2.6308600289928\\
87	-7.96707277658618\\
88	-4.93261149234844\\
89	-5.67090047446453\\
90	-5.7099567543029\\
91	-25.6991138272069\\
92	-15.9741060827329\\
93	-3.17277129820124\\
94	-21.3183674032492\\
95	-6.5304870138513\\
96	-3.4784074012636\\
97	-1.82192192469511\\
98	-1.01611104288794\\
99	-0.923707402740685\\
100	2.16648840950946\\
101	-11.7133760311976\\
102	-59.346981006539\\
103	1.34381882238831\\
104	-20.8919988679046\\
105	-3.62795372475969\\
106	-5.38672806488254\\
107	-6.73307722091688\\
108	-12.2298833413062\\
109	-0.684051608501252\\
110	-16.4691939223158\\
111	-4.33239597663085\\
112	-22.6600163651394\\
113	-13.9888226914754\\
114	-14.7843217382518\\
115	-18.6814250515769\\
116	-6.15373037027225\\
117	-6.77517604187189\\
118	-1.44245328230471\\
119	-4.41621778997452\\
120	-28.9715702645639\\
121	-19.1701662901075\\
122	-9.62575858673075\\
123	-18.1028837967179\\
124	1.93686242497672\\
125	-1.46735901213168\\
126	-7.43324329699798\\
127	-6.69889136464254\\
128	-8.14275682382474\\
129	-8.70645084742302\\
130	-2.43685153819804\\
131	-6.22070861049959\\
132	-23.1082986453275\\
133	-0.183431834000005\\
134	-19.5003176507559\\
135	-2.92505429100304\\
136	-21.4807871503987\\
137	-7.75806925843452\\
138	-10.6695881784726\\
139	-18.6340080572434\\
140	-20.9038976992871\\
141	-22.32029765498\\
142	-2.60658188460689\\
143	-5.24577606100626\\
144	-6.24144505796311\\
145	-7.6979910733877\\
146	-5.95301304497896\\
147	-5.22117345440113\\
148	-4.68884580900842\\
149	-2.85656111929787\\
150	-9.93223207113662\\
151	-5.7873364445443\\
152	-11.0474537068599\\
153	-16.2465581072583\\
154	-20.8716010238376\\
155	-1.27133900800076\\
156	-1.28633305129198\\
157	-9.77673582659151\\
158	-9.79958761228433\\
159	-13.0659854454967\\
160	-13.0468105057781\\
161	-4.79991658212269\\
162	0.699412996051635\\
163	-19.9998212371662\\
164	-1.42551648050292\\
165	1.1413648083558\\
166	-19.3604316915317\\
167	-0.482681869771264\\
168	-8.09286003628847\\
169	-12.7460970536528\\
170	-10.643886131464\\
171	-9.99029362187015\\
172	-4.15354479394344\\
173	-14.3608286186779\\
174	-4.08504351335276\\
175	3.21087425318557\\
176	-6.89165150673557\\
177	-3.69570326374111\\
178	-11.3409573276249\\
179	-21.9008015587466\\
180	-4.66683744823459\\
181	-0.605711186246677\\
182	-24.3638709929322\\
183	-1.37516135688215\\
184	-9.83300501474917\\
185	-23.0273392387074\\
186	-12.9163720462688\\
187	-5.11112245203798\\
188	-5.37807203937779\\
189	-37.674915092239\\
190	-3.23538723486739\\
191	-12.2488900219575\\
192	-3.07950107746723\\
193	-6.25368072220239\\
194	3.43485351051955\\
195	-44.6515262732306\\
196	-16.6882718586884\\
197	-21.0126816407382\\
198	-8.82091018789905\\
199	-2.19470215644504\\
200	-25.5850424515779\\
201	0.856642360695719\\
202	-20.1130134002607\\
203	-6.68439535395839\\
204	-17.6474547016104\\
205	-10.0735430294569\\
206	-6.90029211400525\\
207	-42.9335029609118\\
208	-5.53822930623422\\
209	-9.4679554402507\\
210	-14.6995525147994\\
211	-6.44514529433799\\
212	-11.8755526325397\\
213	-12.5878517052494\\
214	-0.15052371849438\\
215	-7.06295270804035\\
216	-9.39618951933702\\
217	-12.6139967080537\\
218	-2.13409810902927\\
219	-6.49273050924832\\
220	-5.46661875930236\\
221	-6.7750021898228\\
222	-1.69255628634218\\
223	-6.13730821978894\\
224	-27.642379904636\\
225	-18.053219687239\\
226	-22.2189771486682\\
227	-9.00700434632651\\
228	-9.72447100347559\\
229	0.827697866020099\\
230	-9.14081851558501\\
231	-1.43584469354456\\
232	-14.9499255752706\\
233	-9.81995185443378\\
234	-16.9348181554722\\
235	-0.150939625286832\\
236	-5.31110772776971\\
237	-4.58494842739166\\
238	-2.61494251015093\\
239	-10.7554391108719\\
240	1.99257292632065\\
241	-22.5609749486454\\
242	-9.17977943405751\\
243	-21.3835823253358\\
244	-0.73511703839891\\
245	-5.16361920388377\\
246	-10.1758474585643\\
247	-11.1566886730101\\
248	0.23763069189796\\
249	-14.2695018308924\\
250	-17.2867809540937\\
251	-27.0954331999049\\
252	-11.789786934442\\
253	-8.70934690077886\\
254	-8.82735234004944\\
255	-3.85441742366275\\
256	-8.2300432559291\\
257	-1.09418517389093\\
258	-9.05017510154051\\
259	-0.958996395226468\\
260	-14.9043530190396\\
261	0.554487854341143\\
262	-15.5978993221573\\
263	-21.9435508197223\\
264	-5.04440172256446\\
265	1.07181502949782\\
266	-19.3964660915748\\
267	-6.27551281487791\\
268	-14.4979712024653\\
269	-37.8829220304948\\
270	-7.68828953517906\\
271	-3.30060671329734\\
272	-1.45082600290001\\
273	-12.0144210128425\\
274	-10.8886994198889\\
275	-13.7469812940164\\
276	-7.33595090243837\\
277	-12.6808188550064\\
278	-3.78814005563828\\
279	-24.1942056537046\\
280	-10.3799260958487\\
281	-4.04124104506247\\
282	1.10131127866473\\
283	-22.4059055932361\\
284	-8.21573589135925\\
285	-15.9730637612718\\
286	-9.05416052405207\\
287	-17.8166086033322\\
288	-8.71884873083974\\
289	-6.1215783248566\\
290	-15.5226566697858\\
291	-12.1209501252376\\
292	-12.4299503894821\\
293	-0.0603110186170393\\
294	-5.56325268147259\\
295	1.82619768576189\\
296	0.565709985954454\\
297	-10.7923720578254\\
298	-41.250663380685\\
299	-0.439168446028935\\
300	-14.710897594437\\
301	-5.61429116557109\\
302	-5.82594951370206\\
303	-11.8935592665776\\
304	-9.13258452178781\\
305	-20.3808626967821\\
306	-6.20153787640871\\
307	-11.5064470638619\\
308	-35.0990330473009\\
309	-11.8925356832955\\
310	-9.17318761079333\\
311	-1.61181845403188\\
312	-10.6370930308291\\
313	-14.8845784109279\\
314	-3.07017565228439\\
315	-5.4056030816308\\
316	-7.38020450408895\\
317	-9.50725614013921\\
318	-4.46835820176288\\
319	-6.48619623217505\\
320	-11.9790896756477\\
321	-13.2332534101904\\
322	-7.10759991359656\\
323	-6.90301447433923\\
324	-0.0148765220966608\\
325	-8.99158211550038\\
326	-2.22107053003905\\
327	-10.8973527607266\\
328	-14.8426755038869\\
329	-15.2632855441371\\
330	-7.47532127904178\\
331	-32.7201357298858\\
332	-5.58996861874252\\
333	-1.40660706280571\\
334	-7.52836760779713\\
335	-6.4274222348492\\
336	-2.51625163870962\\
337	-10.3115882667678\\
338	-14.4649255389287\\
339	-20.0860167250173\\
340	-13.6652760730484\\
341	-8.47018364749893\\
342	-11.5996100932538\\
343	-0.74928614414311\\
344	-6.53332743752267\\
345	-32.5404869421466\\
346	-8.58007723658278\\
347	-16.2355758133801\\
348	-2.01332495053569\\
349	-16.888545830489\\
350	-6.31318310584047\\
351	-16.1583282785303\\
352	-15.5851210615584\\
353	-5.28188000315588\\
354	-6.28771230840358\\
355	-15.9449561279755\\
356	-14.8143744479067\\
357	-13.321418659914\\
358	-14.6189252984707\\
359	0.0987068924797096\\
360	-14.6178678616482\\
361	-40.3835433488463\\
362	-9.27644485218577\\
363	-7.63926001859259\\
364	-6.90340395506167\\
365	-14.0164680805349\\
366	-21.2980623885488\\
367	-8.28950746440416\\
368	-4.23140888767518\\
369	-7.06214078606274\\
370	-30.1588440155679\\
371	-6.21093727251847\\
372	-11.0439317699832\\
373	-8.60531399848454\\
374	-24.6133210603592\\
375	-2.84848570998922\\
376	-2.8994599929396\\
377	-22.6847704663401\\
378	-6.70271156494398\\
379	-8.52517387588216\\
380	-8.86154583915572\\
381	-14.8006505724025\\
382	-3.4125300437393\\
383	-2.93571779879158\\
384	-4.88678623092155\\
385	-13.106117372166\\
386	-9.72701045862135\\
387	-39.2660786229949\\
388	-11.017625932731\\
389	-9.00568293701039\\
390	-7.37932595675048\\
391	-1.85505631459912\\
392	-8.62240715640721\\
393	-12.3983334150027\\
394	-22.7767818169357\\
395	-11.7291298134936\\
396	-6.81187105011495\\
397	-4.64500114978239\\
398	-21.7662907148752\\
399	-1.77934824120445\\
400	0.125218661111909\\
401	-11.1717403396609\\
402	-2.78954019914477\\
403	-10.2874590731984\\
404	-7.40698645421417\\
405	-2.90814611781266\\
406	-16.9354331320949\\
407	-11.5420689803989\\
408	-0.0639211466757414\\
409	-6.57859276102424\\
410	-33.2377278603399\\
411	-5.24812253989842\\
412	-2.4964863636335\\
413	-13.1309265645019\\
414	0.489141072879001\\
415	-13.9818293429093\\
416	-7.73531450393352\\
417	-4.87520277233747\\
418	-7.45713381884894\\
419	-3.02282001340882\\
420	-5.62336457664953\\
421	-7.68830489583319\\
422	-5.78793521222928\\
423	-10.8410474383238\\
424	-20.7448031915465\\
425	-26.7191546820951\\
426	-3.75041284394474\\
427	-0.33061110843711\\
428	-8.86338513052298\\
429	2.62351381292135\\
430	-10.4108477219721\\
431	-11.4530526359207\\
432	-16.7485784062428\\
433	-9.86231982623875\\
434	-11.2989517292052\\
435	-10.7198142832018\\
436	-14.804137199541\\
437	-11.6855951250942\\
438	-1.25994215820297\\
439	-5.07216337151014\\
440	-33.8393299269069\\
441	-16.5151568816718\\
442	-5.07501531731218\\
443	-2.12971168559446\\
444	-35.0756753274364\\
445	-5.92335013337092\\
446	-0.473095840296589\\
447	-26.1400869547575\\
448	-3.94091468612942\\
449	-10.0291689099023\\
450	-15.8903220175594\\
451	-32.0052830114191\\
452	5.33444068865327\\
453	-15.3603192554941\\
454	0.496123995734008\\
455	-4.66072022160796\\
456	-29.2126247645028\\
457	-19.1887858641379\\
458	-11.1450754605426\\
459	-3.12382693325077\\
460	-0.322749421099861\\
461	-9.56374211651313\\
462	-25.0431463653979\\
463	-29.5976624206297\\
464	-30.9741838454664\\
465	-14.7689257791092\\
466	-10.4223482498238\\
467	-1.6595326487244\\
468	-10.9210926788278\\
469	-13.4344298520933\\
470	-12.1860967625405\\
471	-5.69095777315691\\
472	-13.864795023162\\
473	-20.5223668109226\\
474	0.697102597996825\\
475	-32.0229157612183\\
476	-7.44505812755193\\
477	-0.509101924591519\\
478	-25.7911202666124\\
479	-7.7172034372522\\
480	-4.29874199620251\\
481	-0.513926102615781\\
482	-3.14408379528459\\
483	-17.2717217805614\\
484	-18.3256365413461\\
485	-41.5486861144305\\
486	-1.76815518705379\\
487	-4.49502105793966\\
488	-7.92177585297917\\
489	-1.38311823364271\\
490	-12.0053808510434\\
491	-9.07090992967265\\
492	-17.108073750939\\
493	-3.13326589431895\\
494	-17.0959935576313\\
495	-11.6607774391748\\
496	-5.03861782314549\\
497	-22.61550887181\\
498	-12.0582511013407\\
499	-3.84636965089376\\
500	-12.0325508282838\\
501	-16.2277774766272\\
502	-23.108720013689\\
503	-3.3086027742397\\
504	-3.21643785178027\\
505	-7.23416680778811\\
506	-15.2137095695021\\
507	-31.2000150233775\\
508	-18.1400049021263\\
509	-15.6213742482584\\
510	-15.7754411887909\\
511	-0.798260663385401\\
512	-8.53429523931161\\
513	-6.67834436912905\\
514	-8.51891119052522\\
515	-5.53007617913101\\
516	-9.5318594993552\\
517	-24.0797118980174\\
518	-9.25642540481672\\
519	-9.60335291562547\\
520	-16.2994743519499\\
521	-15.1211457192869\\
522	-7.69367918358796\\
523	1.20556183677664\\
524	-25.8342377770549\\
525	-31.7427068219919\\
526	-11.4497341190725\\
527	-3.73370626144298\\
528	-13.8295380882914\\
529	0.872524404307534\\
530	-5.82649086578517\\
531	-2.90896818189953\\
532	-2.10303149045969\\
533	2.2465110520343\\
534	-15.4221044214477\\
535	-5.29666738768231\\
536	-21.1247248086288\\
537	-12.1151148511821\\
538	-7.21394346328418\\
539	-3.19359719095\\
540	-11.0487442502111\\
541	-13.2628134255499\\
542	-7.5651009652698\\
543	-15.2206092892901\\
544	-21.2606988953913\\
545	-9.9977454668865\\
546	-45.5654170328089\\
547	-2.84850382647434\\
548	-4.55714155303546\\
549	-9.95843458416454\\
550	-0.698869523900593\\
551	-7.1743037535666\\
552	-20.8953473592461\\
553	-11.9026490832363\\
554	-38.3692474555171\\
555	-42.2481612742308\\
556	-5.57114766422756\\
557	-14.904466331767\\
558	-12.1947363047592\\
559	0.688015154788255\\
560	-8.96321760101413\\
561	-5.85974375826106\\
562	-7.20150049764118\\
563	-6.63830046977555\\
564	-8.15016729212223\\
565	-28.8216901839776\\
566	-5.76663453540817\\
567	-29.9795455866438\\
568	0.265613620060548\\
569	1.10930719556539\\
570	0.115812270807811\\
571	-18.6928300017944\\
572	-15.4693803123112\\
573	-17.8486254480192\\
574	0.489191208592769\\
575	-7.72948994663688\\
576	-4.20616045426194\\
577	-3.77805263701096\\
578	-21.3518682713082\\
579	-11.3837673331655\\
580	-8.60594131325877\\
581	-9.3047180174951\\
582	-7.9540247863066\\
583	0.00870999414495697\\
584	-14.7716137833053\\
585	-17.5019904356332\\
586	-13.4476888743506\\
587	-16.5595339788769\\
588	-6.67669475100715\\
589	-6.00031438391986\\
590	-17.02834150099\\
591	-27.6206659379024\\
592	-42.1400463157761\\
593	-6.95320105831176\\
594	-3.05143502943607\\
595	-10.1998579484666\\
596	-6.19299549579936\\
597	-7.26967880816568\\
598	-5.50280805193553\\
599	-3.46479229503664\\
600	-3.5505316428028\\
601	-6.07464456315418\\
602	-7.10005610709119\\
603	-23.5069402708118\\
604	-1.47396773575405\\
605	-9.50032439186499\\
606	-10.0654513066894\\
607	-4.27788039904159\\
608	-6.40479576434844\\
609	-3.17652324495136\\
610	-19.6206772657859\\
611	-9.9273873950033\\
612	-32.1704573494879\\
613	3.68698053044907\\
614	-23.9653596550328\\
615	1.76588821097145\\
616	-3.20473819389267\\
617	-4.33021115491801\\
618	-13.5640829485746\\
619	-24.2908476354947\\
620	-12.0072381469388\\
621	-14.4011347488313\\
622	-10.8860764659487\\
623	-9.34128186011741\\
624	-5.07955363447223\\
625	-40.2634692825734\\
626	-3.71360142292518\\
627	-28.5541679870504\\
628	-12.3985472599992\\
629	-23.1892290331757\\
630	-35.596754110284\\
631	-25.0110436913346\\
632	0.127524933304013\\
633	3.32877957798488\\
634	0.133878375892051\\
635	-2.83371299443572\\
636	-1.88473624455336\\
637	-28.2358619283542\\
638	1.17296969251143\\
639	-2.13084541192218\\
640	-40.2970028675294\\
641	-11.8998744454147\\
642	-22.9856940497231\\
643	-9.01579017728226\\
644	-15.4364169685861\\
645	-6.20070895270918\\
646	-4.04958909470414\\
647	-13.3941865468538\\
648	-9.86756415844622\\
649	-21.5302935730663\\
650	-5.97118007446096\\
651	-8.38247001740294\\
652	-21.853069924603\\
653	-7.602665903379\\
654	-14.6912870131533\\
655	-24.3053641271935\\
656	-32.4422054308837\\
657	-1.88527657530585\\
658	-20.1251074416768\\
659	-8.31287003792626\\
660	-9.94274772016858\\
661	-10.0827894710283\\
662	1.67337288978527\\
663	-15.0922650209071\\
664	-19.6136125271225\\
665	-13.3305826321431\\
666	-30.4396169599877\\
667	-35.0115892083647\\
668	-7.24251829954811\\
669	-1.31432627985585\\
670	-1.31630871809902\\
671	-4.43521045542003\\
672	-8.48883665582405\\
673	-5.42738233443863\\
674	-2.25061256719011\\
675	-20.0910280705356\\
676	-27.4160479791659\\
677	-12.0737081251085\\
678	-36.0365099868419\\
679	-16.677700211521\\
680	-5.85980378827908\\
681	-23.3551727971431\\
682	-13.8993490720901\\
683	-7.20442047754653\\
684	-3.43467759029338\\
685	-1.55635956172804\\
686	0.770498306392131\\
687	-10.1348987794019\\
688	-6.14888149538873\\
689	-30.0281139554976\\
690	-7.78466214437201\\
691	-15.8413850122248\\
692	-22.1184166883328\\
693	-17.4683349189731\\
694	-1.04140101014104\\
695	-14.5993444381962\\
696	-0.786702893167746\\
697	-1.27735694699546\\
698	-20.3804021130036\\
699	-8.5466692631526\\
700	-14.104006570083\\
701	-4.76245345017327\\
702	2.15127465512754\\
703	-3.72718227287766\\
704	-15.1015832266835\\
705	-7.84686530661002\\
706	-1.63920196115198\\
707	-7.22716666800575\\
708	-35.1152455990923\\
709	-2.92772395497032\\
710	-20.6297712921762\\
711	-42.6417374335661\\
712	0.0154634425120344\\
713	-3.23411349806811\\
714	-8.08518519267919\\
715	-10.9077620433422\\
716	-2.93617784792296\\
717	-3.75440329999919\\
718	-31.1432917568329\\
719	-16.5536012473689\\
720	-14.0217715742595\\
721	-13.6748593413672\\
722	-36.6206746896896\\
723	-6.69307521391782\\
724	-18.3447977412229\\
725	-7.57067217329025\\
726	-14.3970950654001\\
727	-4.46867872667342\\
728	-3.69924010462183\\
729	-20.9337505125014\\
730	-4.55873821472777\\
731	-19.3813990908354\\
732	-10.2209024169865\\
733	-18.8814076518628\\
734	-1.55966240256201\\
735	-9.71751755875059\\
736	-16.5663639614425\\
737	-4.79468707053823\\
738	-4.73063412504759\\
739	-8.82495179717874\\
740	-16.7706829044281\\
741	1.60891568577098\\
742	-2.05681056867433\\
743	-7.33836420916965\\
744	-8.65015563912906\\
745	-5.88455147420146\\
746	-4.95940358060882\\
747	-3.16058877729115\\
748	-41.7990794730517\\
749	-1.01345214199156\\
750	-18.4007104031361\\
751	-49.846290067359\\
752	-8.23508794411972\\
753	-36.9484180011604\\
754	-9.43724272958738\\
755	-16.7431879902433\\
756	-3.87351352040897\\
757	-16.119334684142\\
758	-17.7659657645439\\
759	-15.0552536477019\\
760	-28.9367389540692\\
761	-3.58870730179785\\
762	1.02745077470834\\
763	-4.73028274584358\\
764	-3.24670894336435\\
765	-1.31957864056796\\
766	-10.290799170035\\
767	-7.56479095456034\\
768	-9.51944594941711\\
769	0.840477932815157\\
770	-30.3789457049321\\
771	-10.7200512138749\\
772	-6.58218472263556\\
773	-1.3451873593938\\
774	-7.40233345464585\\
775	2.96447154337405\\
776	-7.17588351343173\\
777	-8.43482943773043\\
778	-17.0988358526246\\
779	-10.4386166821423\\
780	-44.0300309810797\\
781	-12.8031658639016\\
782	-5.71097335134764\\
783	-5.2551142099638\\
784	-15.9890017510059\\
785	-7.18978617982566\\
786	-19.2341111629578\\
787	-23.3114556607909\\
788	-9.47678944987844\\
789	0.505152213322488\\
790	-8.57275952250676\\
791	-16.4013364166092\\
792	3.77834272018523\\
793	-28.32081734927\\
794	-8.11286485406279\\
795	-19.4422690690266\\
796	-15.8366637968864\\
797	3.96500314504433\\
798	-1.3751507753972\\
799	-1.62947788048738\\
800	-8.67430461569993\\
801	-5.26718984007775\\
802	-7.59618630029673\\
803	-4.75891408183013\\
804	-3.01256242156252\\
805	-6.03426969303483\\
806	-6.69692372775619\\
807	-32.9977719770326\\
808	-3.60365253411372\\
809	-5.74410334030796\\
810	-25.900874600035\\
811	-5.02720972653943\\
812	-13.0922812344073\\
813	-2.21490818232857\\
814	-7.1958907573909\\
815	1.15839517020815\\
816	-13.2537765100296\\
817	-5.53016818114055\\
818	-4.29944578756281\\
819	-7.1104663727473\\
820	-17.4763802462491\\
821	1.82040126660967\\
822	-22.0321835069202\\
823	-19.6919530483041\\
824	-8.10076773167172\\
825	0.475542837002909\\
826	-40.2144571935122\\
827	-9.31436562125318\\
828	-15.8300355941659\\
829	-27.4109957466048\\
830	-1.34425108618471\\
831	-8.44653047226327\\
832	-8.28533480224159\\
833	-12.9051118316845\\
834	-9.75135960650038\\
835	-4.16622726545057\\
836	-5.39907668285934\\
837	-4.96126074275313\\
838	-6.56219635914597\\
839	-1.64692667963083\\
840	-6.63015730892722\\
841	-5.27167667702991\\
842	-23.9669389630934\\
843	-3.69163173954876\\
844	-17.6022888087203\\
845	-8.7693426522962\\
846	-6.2238437300108\\
847	-19.442852553487\\
848	-11.9016937955513\\
849	-2.42886293665303\\
850	-12.117108400268\\
851	-8.16951386208511\\
852	-14.0231559792598\\
853	-5.14971727897945\\
854	-4.49138905672075\\
855	-5.35375098177227\\
856	-30.1674623519312\\
857	-11.0509404850787\\
858	-40.721273993602\\
859	-33.4824850447628\\
860	1.48257784725916\\
861	-18.37702894045\\
862	-15.8216095658837\\
863	-4.10081352621567\\
864	-0.317469366501441\\
865	-18.0922410054028\\
866	-39.5057915498132\\
867	-30.3943999937164\\
868	-23.001558734014\\
869	-8.6341308408656\\
870	-5.28352597889362\\
871	-3.42301013978192\\
872	-11.7817758558175\\
873	-1.42070165548929\\
874	-29.3943973215371\\
875	-42.2098204427341\\
876	-6.0815429614938\\
877	-3.00918357326666\\
878	-4.39325513496502\\
879	-3.65425614642029\\
880	-15.9061008361209\\
881	-12.8567027784626\\
882	-6.87600435454132\\
883	-7.3389309000291\\
884	-2.28187206799421\\
885	-3.45887396537691\\
886	-15.6725328529319\\
887	-11.6048833979226\\
888	-3.93301380511644\\
889	-6.44751930837429\\
890	-8.03680307351161\\
891	-7.1166125767525\\
892	-11.5156179237246\\
893	-6.36140851503676\\
894	-13.0308150709951\\
895	-13.0995657784363\\
896	-8.70492354846186\\
897	-13.0483335032339\\
898	-0.43022274553391\\
899	-6.09421914963396\\
900	-1.40792084083004\\
901	-13.1935638154125\\
902	-5.3420066228337\\
903	-9.89935178309493\\
904	-1.88983145929882\\
905	-9.70192564591816\\
906	-20.6640364461068\\
907	-12.6872451050164\\
908	-9.29507684633177\\
909	0.139464233915802\\
910	-20.7029925490139\\
911	-20.8430989136822\\
912	-4.37421542863347\\
913	-2.59459434699724\\
914	-17.2879470799524\\
915	-9.48993464136909\\
916	-10.9982446710527\\
917	-10.9686178973594\\
918	-19.8716014693785\\
919	-7.1350473226593\\
920	-50.1422042746483\\
921	-0.81486099251208\\
922	-26.2222030597389\\
923	-7.09992848606672\\
924	-2.04195381099852\\
925	-3.07514783600764\\
926	-18.4933569132203\\
927	-13.7930632394495\\
928	-8.83464508724192\\
929	-3.99260444658985\\
930	2.43716110108547\\
931	0.278669414403186\\
932	-3.48013168102477\\
933	-27.0311811253683\\
934	-1.51856109671178\\
935	-0.859643631756244\\
936	-7.10966488620363\\
937	-18.0958861642699\\
938	-22.1941346521196\\
939	-3.82573387549001\\
940	-6.6177092015887\\
941	0.304055098029236\\
942	-14.6922681804179\\
943	-7.66073367969004\\
944	-13.8256167651096\\
945	-0.637883928830788\\
946	-2.88765477985662\\
947	-7.82729040001557\\
948	-8.79170310212971\\
949	-12.5786680196291\\
950	-25.1488979073847\\
951	-16.01865656764\\
952	-4.77197710487457\\
953	-12.3643517751098\\
954	-8.20624135176206\\
955	1.92744117036226\\
956	-15.4979656392624\\
957	-8.88993158004709\\
958	-2.26458765620767\\
959	-9.78924886123236\\
960	-11.9112382088684\\
961	-0.370436594949979\\
962	-19.9917864217416\\
963	-10.8675236646707\\
964	-4.33859439641241\\
965	-6.42175007489473\\
966	-14.0283459551597\\
967	-5.62823910404429\\
968	-18.9139767711014\\
969	-30.1191967605326\\
970	0.936029009527683\\
971	0.809044664585411\\
972	-3.7980340489417\\
973	1.30127467164273\\
974	-10.5888270414485\\
975	-2.35180143655874\\
976	-19.5169422446912\\
977	-7.55312688809992\\
978	-8.5455294611957\\
979	-14.7962468912283\\
980	-12.0475448749053\\
981	-12.4571161676131\\
982	-27.9838632201447\\
983	-31.6772516538099\\
984	-4.26681777068129\\
985	-12.7770563254399\\
986	-16.7983159479193\\
987	-8.41217513980787\\
988	-5.7520696224851\\
989	-4.00691579637372\\
990	-25.9416286116529\\
991	-4.06786489762139\\
992	-5.61769437988045\\
993	-7.88163794007604\\
994	-9.30081438627564\\
995	-27.1899879484813\\
996	-6.4501685171406\\
997	-6.48621556150813\\
998	-13.7903099424475\\
999	-7.90266443792498\\
1000	-11.8438339763124\\
};
\addplot [color=mycolor2,solid,forget plot]
  table[row sep=crcr]{1	-inf\\
2	-inf\\
3	0.43025094568348\\
4	0.248715791532654\\
5	-0.578140730819865\\
6	-0.147905767653196\\
7	-13.4597313295181\\
8	-4.33724780446987\\
9	-39.8804898588504\\
10	1.86487098451895\\
11	-15.954275045014\\
12	-9.53084320020564\\
13	-14.8774847790451\\
14	-4.54985670319549\\
15	-8.65592619755544\\
16	-19.765509457253\\
17	-46.0894330991639\\
18	0.410642812731684\\
19	-0.224747047398917\\
20	-8.35763584296983\\
21	-14.3167924688419\\
22	-11.8771133882484\\
23	-7.05048740364373\\
24	-1.38887078515809\\
25	-1.81554101495814\\
26	-16.9215104944932\\
27	-12.7262838840731\\
28	-19.9159788085141\\
29	-2.03792136946509\\
30	-7.89080333704955\\
31	-7.1782954835983\\
32	-14.0044870269206\\
33	-4.45705259606877\\
34	-10.5578608575546\\
35	-7.53231439129816\\
36	-12.8038687959288\\
37	-23.2377514169628\\
38	-11.7999744154638\\
39	-8.35218957436958\\
40	-14.9013676543158\\
41	-8.38056693009075\\
42	-19.5403227519947\\
43	-7.11682942849762\\
44	-24.4278519497733\\
45	-23.1816291073319\\
46	-8.24549253240669\\
47	-10.5748416624793\\
48	-5.88450644164379\\
49	-16.2653949598286\\
50	-8.60955600537443\\
51	-23.172444908705\\
52	-27.8868177079297\\
53	-5.67537654778519\\
54	-17.0001337800702\\
55	-11.2425644253692\\
56	-23.7118685734041\\
57	-33.3437082230842\\
58	-12.4478206189339\\
59	-4.0613946001434\\
60	0.142222260439615\\
61	-12.1416796198861\\
62	-1.42643255769537\\
63	-3.24139048403818\\
64	-3.80684392523489\\
65	-9.10062945753868\\
66	-10.8897942930953\\
67	-6.07523301962611\\
68	-4.86404916516064\\
69	-0.67174836001761\\
70	-4.97747714021386\\
71	-4.66591946619466\\
72	0.0840626891606711\\
73	-3.32310945376423\\
74	-16.942444443709\\
75	3.9701516178174\\
76	-32.109931261415\\
77	-14.5081872832571\\
78	-2.61514874120559\\
79	-1.5417249869632\\
80	-14.6538119956779\\
81	-1.5850957537457\\
82	-3.18175482809625\\
83	-1.36116837738033\\
84	-3.39160880917871\\
85	-18.6681510503365\\
86	-12.175178373289\\
87	-12.7308594957113\\
88	-28.870433638846\\
89	-12.6766875941272\\
90	-37.7669960686693\\
91	-7.12566818108135\\
92	-6.86627168020481\\
93	-3.34408310869509\\
94	-14.1517062598448\\
95	-12.7781077429416\\
96	-2.77455210598549\\
97	-4.25424347319811\\
98	-3.46923190991427\\
99	-5.81580135378116\\
100	-1.8456425862159\\
101	-7.95499577792241\\
102	-5.06288069662945\\
103	-0.409523714045034\\
104	-4.70441589934585\\
105	0.39337177002346\\
106	-7.1948172022951\\
107	-5.67264072740115\\
108	-13.1653443480406\\
109	-0.41721205529289\\
110	-9.20495482382677\\
111	-4.14626886694165\\
112	-16.4131665241762\\
113	-12.4629106224757\\
114	-20.8220791728978\\
115	-19.3188429268406\\
116	-4.59488088291066\\
117	-8.93223191430567\\
118	-2.1283833131855\\
119	-9.79192481481273\\
120	-14.7468761486997\\
121	-20.5348998569625\\
122	-12.7138048865878\\
123	-24.6594042836648\\
124	1.32367057856285\\
125	1.69512262582107\\
126	-10.212707523532\\
127	-7.30395014016232\\
128	-4.08390791755748\\
129	-11.6950185883263\\
130	-2.4073322495768\\
131	-4.00616322198678\\
132	-29.9569589385993\\
133	-0.376178431780676\\
134	-13.7107081966704\\
135	-2.51967482363457\\
136	-14.8737205722407\\
137	-7.11583003491978\\
138	-7.4023948165451\\
139	-14.4533125607986\\
140	-59.0518685583858\\
141	-28.5480861272237\\
142	-3.29613466811644\\
143	-6.58128040761423\\
144	-3.77154426398783\\
145	-6.75182678256487\\
146	-4.00201432506582\\
147	-4.25599133921983\\
148	-4.0674643210586\\
149	-3.78426191419345\\
150	-5.54576364401754\\
151	-3.4417442766336\\
152	-9.85966402301392\\
153	-18.728736054169\\
154	-22.1922652738204\\
155	-1.70147023400789\\
156	-0.754940276446846\\
157	-10.2305306090842\\
158	-8.65765515095844\\
159	-11.4910606326514\\
160	-15.9539001276961\\
161	-4.98846118977759\\
162	0.434486225065143\\
163	-24.5395209925036\\
164	-1.34967053563491\\
165	-0.635408155926055\\
166	-10.4882601826634\\
167	-1.71776192247508\\
168	-18.3046381825013\\
169	-21.4460961762943\\
170	-20.4940206498001\\
171	-13.9753968394376\\
172	-1.94637529927787\\
173	-18.8575220988868\\
174	-3.28911621555517\\
175	2.13671196486066\\
176	-13.5868052603101\\
177	-7.68225753782639\\
178	-4.89575967530296\\
179	-18.4902449838709\\
180	-6.7159482840414\\
181	-1.87602654232011\\
182	-8.74812545673769\\
183	-1.73768613305439\\
184	-29.0917558243021\\
185	-15.2051417627816\\
186	-33.0040961224823\\
187	-4.25827253325717\\
188	-5.50784664053705\\
189	-19.3086174970417\\
190	-3.22844011962583\\
191	-6.68705073574757\\
192	-2.51851892302875\\
193	-11.3513876050994\\
194	2.9401409805314\\
195	-7.7447718130078\\
196	-15.5147632454529\\
197	-11.596209960344\\
198	-14.4081262030227\\
199	-1.56541082281406\\
200	-10.4137514850322\\
201	-0.50639505830003\\
202	-13.8798531515524\\
203	-35.7877512275505\\
204	-22.3367464194124\\
205	-3.42413668817098\\
206	-8.33128142012604\\
207	-20.7703371863912\\
208	-3.24149057888635\\
209	-8.76502223916771\\
210	-15.0192072101514\\
211	-7.80399932620156\\
212	-12.5258770995693\\
213	-21.2751234823683\\
214	0.366452017610111\\
215	-5.80302512748374\\
216	-5.76471081762555\\
217	-10.3104061269797\\
218	-1.11616248153576\\
219	-6.18228311202335\\
220	-7.91518055843342\\
221	-12.2360794587231\\
222	1.32942075319219\\
223	-6.66447667049732\\
224	-16.969388852285\\
225	-64.7561150211056\\
226	-14.2280920524303\\
227	-6.83593576393205\\
228	-11.8242851991257\\
229	1.01107313756505\\
230	-11.3693284957397\\
231	-3.29781134687286\\
232	-11.8946253147355\\
233	-2.95230018480306\\
234	-13.8861064449954\\
235	0.423841467166021\\
236	-3.9987439962069\\
237	-7.92278601908029\\
238	-2.9576230902976\\
239	-5.17884359426004\\
240	0.933955810593795\\
241	-35.4932318704938\\
242	-1.03676219965232\\
243	-25.6726454069663\\
244	-0.891326522597564\\
245	-3.15151518652603\\
246	-16.2907123596176\\
247	-19.8061569964813\\
248	-1.27375381017208\\
249	-17.4194006244046\\
250	-1.72850669726361\\
251	-11.5634673330095\\
252	-8.84765941977895\\
253	-3.92366976048364\\
254	-4.83517257041074\\
255	-6.87992773350498\\
256	-1.87430688729347\\
257	-1.28753585144242\\
258	-4.5039286301793\\
259	-4.24786715580496\\
260	-14.9211754311782\\
261	-4.18759201035649\\
262	-18.9596064875865\\
263	-5.09990042938271\\
264	-5.35234244195808\\
265	2.33514997345925\\
266	-8.05221423727193\\
267	-4.96735873763781\\
268	-8.58397962837653\\
269	-30.8431359468653\\
270	-9.18482483790612\\
271	-2.99035757061642\\
272	-2.0682276739303\\
273	-13.995840247797\\
274	-8.01434621214909\\
275	-12.6496768840246\\
276	-5.42520753927942\\
277	-16.8746322015414\\
278	-4.31296614668495\\
279	-27.1972922327172\\
280	-11.9412816673763\\
281	-4.852520295185\\
282	-0.0619758794104268\\
283	-14.0728006792539\\
284	-11.0292135315728\\
285	-7.84281580331186\\
286	-8.52997078172244\\
287	-27.3647240489765\\
288	-8.56025847133823\\
289	-6.04798035752256\\
290	-22.1605409029835\\
291	-12.5574584193892\\
292	-17.4990586312188\\
293	-0.39527113681324\\
294	-11.5773438375897\\
295	0.868295918859367\\
296	-5.01224897460158\\
297	-14.233147471634\\
298	-6.31443656088843\\
299	1.18061014236667\\
300	-15.6942512144569\\
301	-7.28710744983669\\
302	-4.32016698127808\\
303	-14.0584582310032\\
304	-8.06356947902372\\
305	-46.2715785841593\\
306	-7.03218364211706\\
307	-11.9169654936993\\
308	-23.3960889769103\\
309	-12.4195031199908\\
310	-9.25745613080985\\
311	-2.18533408696715\\
312	-7.6731405756723\\
313	-19.7086479717692\\
314	-2.17342376481615\\
315	-8.31899177777878\\
316	-6.64564192645827\\
317	-15.1087150172531\\
318	-4.45095632325703\\
319	-11.8668295066464\\
320	-9.8454461075979\\
321	-17.8033770534605\\
322	-5.76836794823865\\
323	-7.63117331083943\\
324	0.0393739587086328\\
325	-5.65519158731869\\
326	-0.836158838338418\\
327	-22.1177329173962\\
328	-11.8737085080168\\
329	-9.35788316316339\\
330	-6.29624782029407\\
331	-19.6987744035173\\
332	-5.3931833861016\\
333	-2.45460977530177\\
334	-9.81940516446731\\
335	-11.7778980978109\\
336	-0.75080669269077\\
337	-9.05870304001814\\
338	-14.4686968385398\\
339	-19.1877966336931\\
340	-14.1519386568373\\
341	-8.78985452648246\\
342	-13.383134352709\\
343	-1.07264589916787\\
344	-4.00604568209927\\
345	-36.2516501476471\\
346	-6.72340184067647\\
347	-20.0277215388005\\
348	-2.54386073576003\\
349	-16.3626060280375\\
350	-5.93983850175852\\
351	-18.849158991149\\
352	-14.7489091482656\\
353	-6.00687724296901\\
354	-6.94237434446279\\
355	-21.6163052607284\\
356	-13.337116396682\\
357	-14.4535147609775\\
358	-12.6451505392772\\
359	0.00337659658836496\\
360	-18.8360067217865\\
361	-43.5195127686871\\
362	-10.503144353244\\
363	-7.131532230211\\
364	-6.29918088967983\\
365	-18.7584172576226\\
366	-24.6700132271697\\
367	-7.29754977271099\\
368	-3.4152544840162\\
369	-5.28113579737467\\
370	-19.10190907118\\
371	-5.07778860361532\\
372	-9.55114635066935\\
373	-8.25554995072528\\
374	-22.8773151034125\\
375	-3.11700342418654\\
376	-2.13792090221035\\
377	-37.9015275992529\\
378	-8.08459665024886\\
379	-10.1945396538082\\
380	-12.2596288189823\\
381	-22.6787160860555\\
382	-5.21320313544759\\
383	-5.33147231183452\\
384	-10.5715775169274\\
385	-21.6930798192749\\
386	-4.43391087660453\\
387	-17.8381422929391\\
388	-8.02325541885208\\
389	-8.13609051076778\\
390	-8.03647419024864\\
391	-1.40537753259561\\
392	-9.52511505719185\\
393	-12.1278515050611\\
394	-21.2739780518225\\
395	-11.5991155342881\\
396	-6.23501776603303\\
397	-5.21016954092259\\
398	-19.9277744112141\\
399	-2.02841049603233\\
400	0.878084610957618\\
401	-9.63387188710957\\
402	-3.26177413079191\\
403	-9.61415942996978\\
404	-6.01821888398214\\
405	-3.27879153659565\\
406	-15.3247834703732\\
407	-12.8127116185793\\
408	-0.178744698416836\\
409	-7.33721294431129\\
410	-29.7292714649864\\
411	-5.10234903034316\\
412	-3.05207244792865\\
413	-13.0345677257815\\
414	0.210948872717863\\
415	-12.1608781218706\\
416	-6.0705311823104\\
417	-6.00121193639634\\
418	-8.82498533156486\\
419	-3.19114809194411\\
420	-4.71539109263715\\
421	-9.69186699296466\\
422	-4.51073900642074\\
423	-9.22014954519193\\
424	-13.9665986867379\\
425	-36.1877667301509\\
426	-2.7398822870811\\
427	-0.79430737433541\\
428	-9.74379743618528\\
429	2.53348147809942\\
430	-13.7633420628114\\
431	-16.3881664462044\\
432	-14.7115261558198\\
433	-8.35141943548149\\
434	-13.3144967049388\\
435	-8.77002896806321\\
436	-16.311496143066\\
437	-14.5409783759587\\
438	-1.22270197463312\\
439	-5.43027121967377\\
440	-23.5513139142402\\
441	-15.548917755745\\
442	-4.86107952391944\\
443	-2.65007140657645\\
444	-20.2737767766397\\
445	-5.9576005560585\\
446	-0.719786404823622\\
447	-30.2089624142003\\
448	-2.98092917161937\\
449	-7.51566717536482\\
450	-10.7424060877743\\
451	-22.7217225919392\\
452	5.77305908984779\\
453	-12.3379351503085\\
454	0.558674733374852\\
455	-7.19451832382451\\
456	-6.44183191660754\\
457	-25.3805414625385\\
458	-17.4962940665855\\
459	-5.33327189332858\\
460	2.30754450131783\\
461	-13.4196947503733\\
462	-15.8312332447572\\
463	-18.0954215722373\\
464	-20.7155202361798\\
465	-18.0458618830123\\
466	-12.096052158676\\
467	-2.20509758924676\\
468	-10.2229108343139\\
469	-13.6821876903538\\
470	-13.1330532710225\\
471	-6.00996534296692\\
472	-12.9129972336947\\
473	-17.7716857172869\\
474	1.00004394712256\\
475	-22.842764353214\\
476	-5.762068022711\\
477	-0.707786472421781\\
478	-20.6279119394064\\
479	-8.83775273473901\\
480	-4.4982367099252\\
481	-1.76344199794236\\
482	-6.1768055893765\\
483	-14.5572398912577\\
484	-22.4110075138051\\
485	-13.3514762403034\\
486	-0.757592467768861\\
487	-4.2509769835913\\
488	-5.80032413613023\\
489	-2.03906326034181\\
490	-13.008644622074\\
491	-9.16935955435955\\
492	-15.6186075723327\\
493	-2.69869519191178\\
494	-13.0660320350942\\
495	-9.31131438205468\\
496	-6.35190191572921\\
497	-28.2348407701857\\
498	-10.7723309162092\\
499	-3.22916151742542\\
500	-15.6479431702838\\
501	-18.8475574002893\\
502	-34.1992906286181\\
503	-3.63684665636972\\
504	-3.43600762387974\\
505	-6.52668287762273\\
506	-12.3908076211728\\
507	-20.7725627721256\\
508	-14.6401463005957\\
509	-13.0956891273198\\
510	-19.6419463915671\\
511	-1.23561119034427\\
512	-8.74347204183612\\
513	-6.55130374209166\\
514	-9.84241678295229\\
515	-5.31918320504871\\
516	-10.7341962610682\\
517	-20.1632384836104\\
518	-8.89918414990472\\
519	-11.1758196432044\\
520	-17.6088821655978\\
521	-14.8722177741819\\
522	-7.9792419276715\\
523	1.15663071049821\\
524	-32.1938225069121\\
525	-19.2893659724384\\
526	-9.6921704769288\\
527	-2.66465713877751\\
528	-10.4075776435858\\
529	1.51565950989865\\
530	-6.9599061871537\\
531	-2.48323941572192\\
532	-1.48292406029383\\
533	2.76782533226889\\
534	-16.266728027358\\
535	-4.96596994415257\\
536	-26.276178992204\\
537	-10.4318338367326\\
538	-6.9543749176906\\
539	-3.14596192030055\\
540	-9.53208944636247\\
541	-13.0280007583906\\
542	-6.90520801223442\\
543	-16.2084331325616\\
544	-21.2031073816559\\
545	-10.7854962322213\\
546	-28.1556376731143\\
547	-3.24974838963451\\
548	-6.37768766143832\\
549	-6.08251094027515\\
550	-0.134362429786579\\
551	-6.34254911051956\\
552	-18.7110908453936\\
553	-13.2975418990836\\
554	-34.8796627738423\\
555	-34.6876982532905\\
556	-5.52373656938818\\
557	-16.4049414638815\\
558	-11.4962852004949\\
559	0.574292612450624\\
560	-8.86807928635718\\
561	-4.89401273690989\\
562	-7.64611126493581\\
563	-6.21698016852884\\
564	-6.71186816773646\\
565	-23.3459342086443\\
566	-5.3384901379607\\
567	-24.3002167623127\\
568	0.304712940638948\\
569	0.826164305027739\\
570	-1.73761254123462\\
571	-6.03903561001846\\
572	-60.5205140408559\\
573	-10.1748861337019\\
574	-0.0697668585367028\\
575	-18.8902503434666\\
576	-1.13649356542176\\
577	-2.88688093823731\\
578	-30.3877000716221\\
579	-10.5490101967923\\
580	-7.49988090482513\\
581	-10.1791407327096\\
582	-7.20844323555884\\
583	-0.399107920839908\\
584	-11.7575846475932\\
585	-19.5657384785512\\
586	-11.3836403515027\\
587	-16.2352996145119\\
588	-5.74434914682709\\
589	-6.72677512044062\\
590	-15.8427865774579\\
591	-23.4437053231638\\
592	-35.1770553764435\\
593	-6.53544220730995\\
594	-2.58728299696804\\
595	-8.94719197166395\\
596	-6.87453068717106\\
597	-6.41930088977704\\
598	-6.05353924023181\\
599	-3.76670902482662\\
600	-3.39814798510217\\
601	-5.60333216220638\\
602	-5.98951821682952\\
603	-29.2320873101345\\
604	-0.976205320359706\\
605	-9.17325487828962\\
606	-10.6802600646465\\
607	-3.4948703614922\\
608	-6.04684648975903\\
609	-2.94023289107168\\
610	-21.6733061127525\\
611	-9.44194029170565\\
612	-30.900205334944\\
613	3.62869274721588\\
614	-21.3047412565677\\
615	2.07759147210931\\
616	-2.24636267252461\\
617	-3.2766402115238\\
618	-12.0808925516513\\
619	-25.8387368486522\\
620	-12.1989866916914\\
621	-13.7310165769841\\
622	-10.2069904236855\\
623	-9.07631346262959\\
624	-5.08291178623378\\
625	-25.9780729564206\\
626	-4.20694636978801\\
627	-17.2066484005315\\
628	-10.2048982613143\\
629	-19.8304162637671\\
630	-28.6015443732691\\
631	-28.6607684170775\\
632	0.282605852273272\\
633	3.43292283853526\\
634	-0.310195722467222\\
635	-0.928733599392356\\
636	-1.66817961182659\\
637	-20.1912220071302\\
638	0.941341825250303\\
639	0.739560533426753\\
640	-20.4099318838694\\
641	-10.2868100671939\\
642	-33.5439722332141\\
643	-8.73701172352024\\
644	-18.5446101313581\\
645	-7.13213789107975\\
646	-2.68793733697446\\
647	-13.0251866297053\\
648	-9.65834253493799\\
649	-23.8636602167178\\
650	-6.05296573903466\\
651	-7.20268516947822\\
652	-21.6377382393295\\
653	-7.32207656152244\\
654	-14.498723104889\\
655	-23.3793698833908\\
656	-33.242674051342\\
657	-1.84956467722304\\
658	-21.3755030980291\\
659	-8.79426759388319\\
660	-8.82721196214111\\
661	-10.2482505931434\\
662	1.42205111484306\\
663	-33.4977196133789\\
664	-31.0024948912592\\
665	-9.06396944392911\\
666	-33.9701068991635\\
667	-21.2548599463424\\
668	-6.61478757630601\\
669	-1.34710522691776\\
670	-0.516971031042237\\
671	-3.8922339861324\\
672	-8.88415599420772\\
673	-4.71111304497495\\
674	-2.17787141781532\\
675	-24.4015817724762\\
676	-30.0499741624771\\
677	-13.2049014581634\\
678	-52.3897294325771\\
679	-15.7408352913554\\
680	-6.01039830593462\\
681	-20.8174687124502\\
682	-14.5435320329653\\
683	-7.66873716289889\\
684	-3.41107200396684\\
685	-1.51593318792935\\
686	0.647947005283164\\
687	-8.24672803208708\\
688	-5.94319037312476\\
689	-52.8014389581088\\
690	-8.17645467674611\\
691	-19.2778268988052\\
692	-21.4448925751792\\
693	-15.7704163225094\\
694	-1.00839329562962\\
695	-13.5587756693041\\
696	-1.44465142230788\\
697	-1.51320603539765\\
698	-22.6723426497544\\
699	-6.97649870358639\\
700	-12.9768324374017\\
701	-6.36242681142643\\
702	1.99859777298409\\
703	-4.11764052624053\\
704	-20.9223427810816\\
705	-8.80266075003016\\
706	-0.761140243050454\\
707	-8.82115192088944\\
708	-18.6661551816065\\
709	-2.37575734645702\\
710	-26.0081739393824\\
711	-21.8208268709662\\
712	-0.897270889584049\\
713	-3.41480839224194\\
714	-9.90134115690648\\
715	-13.251396871047\\
716	-2.80439883128123\\
717	-3.36498702616944\\
718	-15.49596193395\\
719	-27.0330897752991\\
720	-10.31938786623\\
721	-16.814621276792\\
722	-18.6916636130482\\
723	-6.22859329472533\\
724	-29.1777534023096\\
725	-6.1216456212101\\
726	-17.1220952720932\\
727	-3.06309841465324\\
728	-3.36707655342376\\
729	-14.2398610234799\\
730	-5.0674542859186\\
731	-45.6936831469255\\
732	-10.1841562040892\\
733	-27.3538427281014\\
734	-1.88065484568317\\
735	-11.2924413443767\\
736	-21.3827949787706\\
737	-4.88468387194725\\
738	-5.13969111193239\\
739	-10.6494145201566\\
740	-14.3158399864824\\
741	1.55340764332225\\
742	-2.39945135186549\\
743	-11.4931025788947\\
744	-7.30868745238179\\
745	-4.98309575280133\\
746	-6.0172667380716\\
747	-1.69258861771902\\
748	-28.8405929751417\\
749	-2.77832754623079\\
750	-17.3595829786286\\
751	-30.7318160452402\\
752	-8.95902108284797\\
753	-62.6747297348401\\
754	-10.4681149761228\\
755	-17.609399541202\\
756	-3.55783824932521\\
757	-16.1263621101221\\
758	-36.6229503161361\\
759	-16.2227136376431\\
760	-19.4877930600755\\
761	-3.75273015904506\\
762	0.722761418865951\\
763	-3.19404085217781\\
764	-1.95481363033993\\
765	-0.748778843535047\\
766	-12.5045017569321\\
767	-6.58331367718755\\
768	-8.07209829898465\\
769	0.156586612767913\\
770	-10.4103440756695\\
771	-9.24392238196459\\
772	-7.46436318547785\\
773	-0.113687590367278\\
774	-11.932949026506\\
775	3.09538031521245\\
776	-4.39810804624016\\
777	-5.35256250722943\\
778	-12.7861115774209\\
779	-13.8206626118744\\
780	-22.5966217988389\\
781	-11.1598651506577\\
782	-5.04608936800662\\
783	-5.42116274936749\\
784	-24.4907614056202\\
785	-7.65675054838539\\
786	-16.1002161072153\\
787	-21.6867732806627\\
788	-8.49670329269138\\
789	0.460952529793122\\
790	-9.64381917937266\\
791	-16.4992000741808\\
792	4.10253645041546\\
793	-14.3197510065135\\
794	-6.00054288585827\\
795	-31.6686155524997\\
796	-23.5595187684226\\
797	4.28290174444384\\
798	-0.747612868841877\\
799	-3.3290616245752\\
800	-8.80386358276304\\
801	-7.13873370491736\\
802	-6.89307755740557\\
803	-5.03547489188127\\
804	-3.87596028282558\\
805	-7.67492797602607\\
806	-7.54510762539574\\
807	-64.3131967952544\\
808	-3.03126770368693\\
809	-6.21447422368911\\
810	-15.2553352553832\\
811	-5.6866677561056\\
812	-9.4304554648834\\
813	-2.86476471880596\\
814	-5.52877679732309\\
815	0.941728934226922\\
816	-8.19666769056533\\
817	-7.67498990718417\\
818	-2.30975802023189\\
819	-6.62920997177487\\
820	-10.4325241297519\\
821	1.53451820847426\\
822	-25.7581662824587\\
823	-14.6287420792681\\
824	-9.68834971145382\\
825	1.17188536234076\\
826	-28.3747824873932\\
827	-15.0255446092789\\
828	-14.6937783621114\\
829	-25.2936877076549\\
830	-1.01611995973147\\
831	-6.99126649228325\\
832	-10.6273753664628\\
833	-10.0187439041064\\
834	-8.91470965417526\\
835	-3.26665724999396\\
836	-5.41970177114207\\
837	-6.7509667088011\\
838	-6.65195877859158\\
839	-1.32584877010951\\
840	-6.82066525677282\\
841	-7.30139717916269\\
842	-25.8253323752012\\
843	-4.11138478131529\\
844	-18.1234700333112\\
845	-7.50726985588705\\
846	-6.30623721393225\\
847	-33.867428392079\\
848	-10.4890050582662\\
849	-2.95496767884212\\
850	-14.4353995523553\\
851	-6.8677783278805\\
852	-14.374743331046\\
853	-3.80990790128838\\
854	-4.32784826125278\\
855	-7.04612515875746\\
856	-19.9209794564177\\
857	-10.4398930063804\\
858	-25.4750412577637\\
859	-24.0443024848993\\
860	1.69617765140432\\
861	-16.8847903463019\\
862	-31.3221056659473\\
863	-4.35454748939609\\
864	0.111237790879042\\
865	-13.7238157086327\\
866	-22.8952940457388\\
867	-21.6598228632501\\
868	-37.5933661897733\\
869	-9.44877351925592\\
870	-5.44078660348344\\
871	-2.96198767471015\\
872	-14.1131213072209\\
873	-0.956115637284362\\
874	-22.0081699678973\\
875	-29.4814548793238\\
876	-4.84207734245469\\
877	-2.46990496464191\\
878	-4.99902941625299\\
879	-3.37832397165656\\
880	-16.8783840321483\\
881	-12.876715965193\\
882	-6.42025867418254\\
883	-8.07732516195951\\
884	-2.40480198600858\\
885	-3.04683136351963\\
886	-12.9796751467909\\
887	-13.5634563005366\\
888	-3.35062327915972\\
889	-5.75114715776686\\
890	-7.18903957551218\\
891	-7.12618698864885\\
892	-12.6868179441424\\
893	-6.59278678032739\\
894	-13.5983844093316\\
895	-13.599421322174\\
896	-8.93336544049956\\
897	-13.3410730170803\\
898	-0.370322272959963\\
899	-5.68699886607695\\
900	-0.874743248259078\\
901	-13.6800351257931\\
902	-5.44078398133316\\
903	-12.3429324522087\\
904	-0.826514451330151\\
905	-9.37006285696232\\
906	-22.2206570081409\\
907	-12.6820577030095\\
908	-9.24467275633541\\
909	0.270388518257898\\
910	-24.0732314437028\\
911	-23.8557887964136\\
912	-4.66081765717513\\
913	-2.41624285322763\\
914	-16.7371116994062\\
915	-10.6402229943921\\
916	-10.9789188569622\\
917	-11.4386800728077\\
918	-21.9029204337801\\
919	-7.14972189924821\\
920	-60.3265885157365\\
921	-0.583265214209386\\
922	-22.9770046767005\\
923	-6.18224340710571\\
924	-2.33253061330063\\
925	-3.57066992583034\\
926	-25.8802873780269\\
927	-19.9972496486788\\
928	-5.84240145023173\\
929	-3.63006323179112\\
930	2.47720741304328\\
931	-0.0550332193384084\\
932	-4.12406887373954\\
933	-19.1154875992663\\
934	-1.59045854237591\\
935	-0.915194782358686\\
936	-5.09108335234636\\
937	-15.2070473108612\\
938	-34.3493966678045\\
939	-4.29898286903224\\
940	-5.50412172151625\\
941	0.397128360461458\\
942	-15.4485437361199\\
943	-7.3463478175422\\
944	-10.6967266232811\\
945	-0.340477447630623\\
946	-5.53658617104567\\
947	-13.8492916315697\\
948	-3.33797612399314\\
949	-11.2029771574883\\
950	-18.359355429668\\
951	-15.3962797972585\\
952	-5.73909582310019\\
953	-9.78357594430317\\
954	-7.78300371392917\\
955	1.97330735158671\\
956	-38.1427337892077\\
957	-9.83729033498274\\
958	-4.56636637956996\\
959	-4.88238807814851\\
960	-15.6542886330967\\
961	-1.75929949337084\\
962	-12.0791300106368\\
963	-21.1478838927062\\
964	-9.52180455235814\\
965	-18.1082077636703\\
966	-13.2359269839113\\
967	-11.805063773381\\
968	-10.1989791064919\\
969	-17.2839669517417\\
970	-0.00524346895963612\\
971	5.43180117894256\\
972	-16.9819290849591\\
973	-2.77458418490346\\
974	-1.80627263469782\\
975	-7.23267124487724\\
976	-20.8883067951749\\
977	-16.1246997743257\\
978	-10.7277946185111\\
979	-25.0569207358551\\
980	-10.7729032312101\\
981	-8.65977680627121\\
982	-25.5256037528861\\
983	-17.3210526589052\\
984	-3.21272941173234\\
985	-24.1560838136328\\
986	-13.4033908367593\\
987	-5.63010452491755\\
988	-4.16924346738355\\
989	-2.19539434411111\\
990	-35.8919502605177\\
991	-4.97547305609451\\
992	-4.66074670813795\\
993	-10.3559490129346\\
994	-9.51842229476979\\
995	-18.8770271003042\\
996	-7.02821472131683\\
997	-6.84811228541333\\
998	-12.8622369089925\\
999	-7.4230724627882\\
1000	-10.0344706505929\\
};
\end{axis}
\end{tikzpicture}%}
   	\caption{\textit{LMS filter error (dB) for the given AR process}}
   	\label{fig:q3_1_b_indiv}
\end{figure}


\begin{figure}[h]
	\centering 
 	\resizebox{\textwidth}{!}{\input{fig/3/3_1_b_leg.tikz}}
   	\caption{\textit{LMS filter error (dB) for the given AR process, averaged over 100 independent trials}}
   	\label{fig:q3_1_b}
\end{figure}


\subsubsection{Misadjustment}

Misadjustment is defined as $ \mathcal{M} = \frac{\mathrm{EMSE}}{\sigma^2_\eta}$, where the total mean squared error $ MSE = \lim_{n \to \infty} \mathbb{E} \{ e^2 (n) \} = \sigma^2_\eta + EMSE $. 

\subsubsection{Steady State Coefficient Values}

\subsubsection{Leaky LMS Derivation}

\subsubsection{Leaky LMS Results}


\subsection{Adaptive Step Sizes}

\subsubsection{Implemented GASS Algorithms}

\subsubsection{NLMS Algorithm}

\subsubsection{GNGD Algorithm}

Maths and boringness ensue

% \begin{equation}
% S = \frac{1}{N}X' X'^T
% \end{equation}


%  \begin{figure}[h]
%  	\centering 
% 	\resizebox{0.6\textwidth}{!}{\input{q5/q5_cum.tikz}}
%   	\caption{\textit{Cumulative representation of the variance of each eigenvector}}
%   	\label{fig:q5_4}
%  \end{figure}

% \begin{figure}[h]
% 	\centering 
%  	\setlength\figureheight{0.4\textwidth}
% 	\setlength\figurewidth{0.7\textwidth} 
%  	\input{p_1/1.tikz}
%  	\caption{\textit{The four randomly generated subsets}}
%  	\label{fig:q1}
% \end{figure}


%  \begin{figure}[h]
%          \centering
%          \begin{subfigure}[b]{0.45\textwidth}
%             \resizebox{\textwidth}{!}{\input{part_4/q8_num_1.tikz}}
%   			\caption{\textit{1 Tree}}
%          \end{subfigure}
%          ~ %add desired spacing between images, e. g. ~, \quad, \qquad, \hfill etc.
%           %(or a blank line to force the subfigure onto a new line)
%          \begin{subfigure}[b]{0.45\textwidth}
%             \resizebox{\textwidth}{!}{\input{part_4/q8_num_3.tikz}}
%   			\caption{\textit{2 Trees}}
%          \end{subfigure}
 		
%          \begin{subfigure}[b]{0.45\textwidth}
%             \resizebox{\textwidth}{!}{\input{part_4/q8_num_5.tikz}}
%   			\caption{\textit{5 Trees}}
%          \end{subfigure}
%          ~ %add desired spacing between images, e. g. ~, \quad, \qquad, \hfill etc.
%           %(or a blank line to force the subfigure onto a new line)
%          \begin{subfigure}[b]{0.45\textwidth}
%             \resizebox{\textwidth}{!}{\input{part_4/q8_num_10.tikz}}
%   			\caption{\textit{10 Trees}}
%          \end{subfigure}
         
%          \begin{subfigure}[b]{0.45\textwidth}
%             \resizebox{\textwidth}{!}{\input{part_4/q8_num_20.tikz}}
%   			\caption{\textit{20 Trees}}
%          \end{subfigure}
 		
%  		\label{q9i}
% 		\caption{\textit{Varying the Number of Trees in the Forest}}
%  \end{figure}

\end{document}

